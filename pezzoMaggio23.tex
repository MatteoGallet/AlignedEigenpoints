\documentclass{amsart}

\usepackage{amsthm}
\usepackage{amsmath}
\usepackage{graphicx}
%%%%%\usepackage{mathrsfs}  %% \mathscr{A}...
\usepackage{amssymb}

\theoremstyle{plain}
\newtheorem{theorem}{Theorem}
\newtheorem{lem}[theorem]{Lemma}
\newtheorem{prop}[theorem]{Proposition}
\newtheorem{cor}[theorem]{Corollary}
\newtheorem{lemma}[theorem]{Lemma}

\theoremstyle{definition}
\newtheorem{definition}[theorem]{Definition}
\newtheorem{example}[theorem]{Example}
\newtheorem{rmk}[theorem]{Remark}



\newcommand{\scl}[2]{\langle #1, #2 \rangle}

\begin{document}

Promemoria
\section{exceptions}
\verb+I conti si trovano nel file file3apr23.sage+

\bigskip
Let $P$ be a point and let $r$ be a line of the plane. If $P$ is
orthogonal to all the points of $r$ (i.e.\ if $\scl{P}{Q} = 0$ for
all $Q \in r$), we say that $P$ is orthogonal to $r$.

\begin{prop}
  The following conditions are equivalent:
  \begin{enumerate}
  \item $P$ is ortogonal to $r$;
  \item $\scl{P}{P} = 0$ and there exists $Q\in r$, $Q\not= P$, such
    that $\scl{P}{Q} = 0$.
  \item $r$ is tangent in $P$ to the isotropic conic. 
  \end{enumerate}
\end{prop}
\begin{proof}
Immediate. (file \verb+contiPezzoMaggio23.sage+)
\end{proof}


\begin{prop}
Let $P_1$ and $P_2$ be two distinct points of the plane, $r=P_1+P_2$,
then the following are equivalent:
\begin{enumerate}
\item $\sigma(P_1, P_2) = 0$;
\item For all $A, B$ distinct points of $r$, it holds: $\sigma(A, B) = 0$;
\item $r$ is tangent to the isotropic conic;
\item If the equation of $r$ is $ax+by+cz = 0$, then $a^2+b^2+c^2 = 0$.
\end{enumerate}
\label{prop2x}
\end{prop}
\begin{proof}
  The equation of the line $r$ is $\scl{P_1\wedge P_2}{(x, y, z)}$, hence
  $a^2+b^2+c^2 = \scl{P_1 \wedge P_2}{P_1 \wedge P_2} = \sigma(P_1, P_2)$,
  this shows that $\sigma(P_1, P_2) = 0$ is equivalent to $a^2+b^2+c^2=0$.
  If $Q = l_1P_1+l_2P_2$ is a point of $r$ and if we substitute its
  coordinates into the equation of the isotropic conic, we get a polynomial
  of degree $2$ in $l_1$ whose discriminant is a non-zero multiple of
  $\sigma(P_1, P_2)$. Hence $r$ is tangent to the isotropic conic if and
  only if $\sigma(P_1, P_2) = 0$. (v. file \texttt{contiPezzoMaggio23.sage}).
\end{proof}

Let $P_1$ and $P_2$ be two distinct points and let 
$r=P_1+P_2$. The notation $\sigma(r) = 0$ will denote the condition
$\sigma(P_1, P_2) = 0$ (the above proposition shows that this notation
is consistent). 

\begin{prop}
  Let $P_1$, $P_2$ be two distinct points of the plane and $r = P_1+P_2$.
  Then the following are equivalent:
  \begin{enumerate}
  \item $\sigma(r) = 0$;
  \item There exists a point $T \in r$ such that $T$ is orthogonal to $r$.
  \end{enumerate}
\end{prop}
\begin{proof}
  If $\sigma(r) = 0$, $r$ is tangent to the isotropic conic in a
  point $T$, hence $\sigma(T, Q) = 0$ for all $Q\in r$. Since
  $\sigma(T, Q) = \scl{T}{T}\scl{Q}{Q}-\scl{T}{Q}^2$, we get
  $\scl{T}{Q} = 0$, hence $r$ is orthogonal to $r$. The converse is
  trivial. 
\end{proof}

Let $P_1, P_2, P_3$ be three collinear points and let
\[
M = [\Phi(P_1), \Phi(P_2), \Phi(P_3)]
\]
In general, the rank of $M$ is six.

\begin{prop}
  Let $P_1$, $P_2$ and $P_3$ be three distinct alligned points.
  Let $r$ be the line passing through
  these points. Let $M=[\Phi(P_1), \Phi(P_2), \Phi(P_3)]$. Then
  \begin{enumerate}
  \item $M$ has rank $\leq 6$;
  \item $M$ has rank $\leq 5$ if and only if one of the points $P_i$ is
    ortogonal to $r$;
  \item $M$ has rank $\leq 5$ if and only if $r$ is tangent to the
    isotropic conic (in one of the points $P_i$);
  \item $M$ has rank $\leq 5$ if and only if has rank $5$.
  \end{enumerate}
  \label{rk3pt}
\end{prop}
\begin{proof}
pare nel file \verb+contoTrePuntiAllineati.sage+
\end{proof}

Let $P_1, P_2, P_3$ be three collinear points of the plane and let $r$
be the line through them. 
\begin{lemma}
  If $Q = w_1P_1+w_2P_2$ is another point of $r$,
  then the rank of the $12\times 10$ matrix
\begin{equation}
M = \left[
\Phi(P_1), \Phi(P_2), \Phi(P_3), \Phi(Q)
\right]
\label{vecchiaMat@@}
\end{equation}
is at most $7$; is less than $7$ if and only if $\sigma(P_1, P_2) = 0$.
\end{lemma}
%%
%
%%the rank of $M$ is less than $6$ if and only if one of the points
%%$P_1, P_2, P_3, Q$ is ortogonal to $r=P_1+P_2$.
\label{condizSigma}
\begin{proof}
  A direct computation shows that all the maximal minors of $M$ are
  zero, so $\mathrm{rank}\,(M) \leq 7$.
  We take the submatrix $M_1$ of $M$ whose rows are:
  \[
\phi_1(P_1), \phi_2(P_1), \phi_1(P_2), \phi_2(P_2),
\phi_1(P_3), \phi_2(P_3), \phi_1(Q)
  \]
  We consider the coordinates of the points as variables and we compute
  the ideal $J_1$, the radical of the ideal of the maximal minors of $M_1$;
  a direct computation shows that $J_1$ is a principal ideal generated by:
  \begin{equation}
    u_1u_2w_1w_2(u_1w_2-u_2w_1)A_1A_2(u_1A_1+u_2A_2)(A_1B_2-A_2B_1)\sigma(P_1,P_2)
    \label{genJ1}
  \end{equation}
  The factors $u_1, u_2, w_1, w_2, u_1w_2-u_2w_1$ are due to the fact that if
  $P_3$ (or $Q$) coincides with $P_1$ (or, respectively, with $P_2$),
  the rank of $M_1$ is less than $7$
  and the same is true if $P_3$ and $Q$ coincide. As a consequence of
  lemma@@@
  we know that if in the matrix $M$, instead of the two rows
  $\phi_1(P_1)$ and $\phi_2(P_1)$ we take, for instance,
  $\phi_2(P_1), \phi_3(P_1)$, the radical of $J_1$ has the factor
  $C_1$ in place of
  the factor $A_1$. If in $M$, in place of the last row
  $\phi_1(Q)$ we take $\phi_2(Q)$ (or $\phi_3(Q)$), the ideal
  $J_1$ has the factor $A_1C_2-A_2C_1$ (respectively, $B_1C_2-B_2C_1$) in
  place of $A_1B_2-A_2B_1$ (these results come from a direct computation
  of the radical of the ideal of the minors).
  Hence, as soon as we know the generator~(\ref{genJ1}), we know all the
  possible generators of the radical of the ideals of all the $7\times 10$
  minors and they are all multiple of $\sigma(P_1, P_2)$. Moreover, if
  all the points are distinct (each given by a triple of not all zero
  coordinates), it is possible to select a minor of order $7$ which
  is zero if and only if $\sigma(P_1, P_2)$ is zero. 
  So far, we have considered only the case in which the matrix
  $M_1$ is given by two vectors of $\Phi(P_1)$, two vectors of
  $\Phi(P_2)$, two vectors of $\Phi(P_3)$ and one vector of $\Phi(Q)$.
  To complete the proof, we have to consider the further case in which
  $M_1$ is given by two vectors of $\Phi(Q)$ and therefore only one
  vector taken from $\Phi(P_1)$ or from $\Phi(P_2)$ or from $\Phi(P_3)$.
  This case is similar to the previous ones, since, by
  proposition~\ref{prop2x}, we can exchange $Q$ with one of the points
  $P_1$ or $P_2$ or $P_3$.
\end{proof}


Let $P_1, P_2, P_3$ be three collinear points of the plane and let $r$
be the line through them. 
\begin{lemma}
  If $Q = w_1P_1+w_2P_2$ is another point of $r$,
  then the rank of the $12\times 10$ matrix
\begin{equation}
M = \left[
\Phi(P_1), \Phi(P_2), \Phi(P_3), \Phi(Q)
\right]
\end{equation}
is $\leq 6$ if and only if is $6$. 
\end{lemma}
\begin{proof}
  Suppose the rank of $M$ is less than $6$. Then, in particular,
  the rank of the matrix
  \[
    [\Phi(P_1), \Phi(P_2), \Phi(P_3)]
  \]
  is less than $6$ so, by proposition~\ref{rk3pt}, has rank $5$ and the
  line $r$ is tangent to the isotropic conic in one of the points
  $P_i$, say in $P_1$. Then consider the matrix
  \[
    [\Phi(P_2), \Phi(P_3), \Phi(Q)]
  \]
  It also has rank $5$ and the line $r$ should be tangent to the isocronic
  conic in one of the points $P_2$ or $P_3$ or $Q$. This is a contradiction.
\end{proof}

Hence we can summarize the two lemmas above with:
\begin{prop}
  Let $P_1, P_2, P_3, Q$ be four collinear points and let
  \[
  M = [\Phi(P_1), \Phi(P_2), \Phi(P_3), \Phi(Q)]
  \]
  Then the following conditions
  are equivalent:
  \begin{enumerate}
  \item $M$ has rank $\leq 7$;
  \item $M$ has rank $\leq 6$ if and only if $r$ is tangent to the
    isotropic conic;
  \item $M$ has rank $\leq 6$ if and only if $\sigma(r) = 0$;
    \item $M$ has rank $\leq 6$ if and only if $M$ has rank $6$. 
  \end{enumerate}
  \label{rango67}
\end{prop}

\begin{prop}
  Assume the matrix $M$ of (\ref{vecchiaMat@@}) has rank $6$. Let $C$ be a
  cubic curve which has three collinear points $P_1$, $P_2$
  and $P_3$ among the eigenpoints. Let $r$ be the line containing the
  points and suppose $\sigma(r) = 0$.
  Then every point of the line $P_1+P_2$ is an eigenpoint of $C$.
  \label{rettaAutop}
\end{prop}
\begin{proof}
(C'\`e una dimostrazione pi\`u ovvia?@@)\\
  Let $Q = w_1P_1+w_2P_2$ be a point of the line $P_1+P_2$. Consider $\phi_i(Q)$.
  Since we assume $\sigma(P_1, P_2)=0$, from the above lemma, we get that
  the rank of $M$ is less than $7$, hence is $6$, then $\phi_1(Q)$
  is a linear combination (with coefficients in the field $K$)
  of six rows of $M$. Assume, for short, that 
  \[
  \phi_1(Q) = \lambda_1\phi_1(P_1)+\lambda_2\phi_2(P_1)+
  \lambda_3\phi_1(P_2)+\lambda_4\phi_2(P_2)+
  \lambda_5\phi_1(P_3)+\lambda_6\phi_2(P_3)
  \]
  If $a_0, \dots, a_9$
  are the coefficients of the cubic $C$, it holds:
  \[
  \scl{\phi_i(P_j)}{(a_0, \dots, a_9)} = 0 \quad \mbox{for $i=1, 2$,
    $j=1, 2, 3$}
  \]
  Hence
  \[
  \scl{\phi_1(Q)}{(a_0, \dots, a_9)} = 0
  \]
  The same is true for $\phi_2(Q)$ and $\phi_3(Q)$, so $Q$ is an
  eigenpoint of $C$.
\end{proof}

\begin{prop}
  Let $r$ be a tangent line to the isotropic conic and let
  $P_1$, $P_2$ and $P_3$ be three points of $r$.
  \begin{enumerate}
    \item If none of the three points is the point of intersection of
      $r$ and the isotropic conic, then a cubic curve which has among
      its eigenpoints $P_1$, $P_2$ and $P_3$, has all the points of
      $r$ as eigenpoints; the family of these cubics is three dimensional.      
    \item If one of the three points is the tangent point of $r$ with
      the isotropic conic, then there is a family of cubic curves
      of dimension $4$ which have $P_1$, $P_2$ and $P_3$ as eigenpoints
      and do not have other eigenpoints on the line $r$.
  \end{enumerate}
\end{prop}
\begin{proof}
  {}From proposition~\ref{rk3pt} we have that the rank of the matrix
  $[\Phi(P_1), \Phi(P_2), \Phi(P_3)]$ is $\leq 6$. If one of the
  three points is the tangent point, then the rank is $5$. In this case, if
  we impose to the generic cubic $c$ of $\mathbb{P}^2$ the condition that
  the three points are eigenpoints, we get $5$ independent conditions on
  the coefficients of $c$ and the family of cubic curves with $P_1, P_2, P_3$
  among the eigenpoints is of dimension $4$. From this, points $(2)$
  follows. If none of the points $P_1, P_2, P_3$ is the tangent point,
  then, from proposition~\ref{rango67} and proposition~\ref{rettaAutop},
  we get that the linearly independent conditions imposed on the
  coefficients of $c$ are $4$ and all the points of $r$ are eigenpoints.
  \end{proof}

\begin{cor}
  Let $P_1$ be a point of the plane. If $P_1$ is out of the isotropic
  conic and if $r$ is
  one of the two tangent lines to the isotropic conic through $P_1$, then
  there is a three dimensional family of cubic curves for which all the
  points of $r$ are eigenpoints. If $P_1$ is on the isotropic conic,
  then we can fix two other points $P_1$ and $P_2$ on $r$ and there is a
  four dimensional family of cubic surfaces with $P_1, P_2, P_3$ as
  eigenpoints. 
\end{cor}

  Let $r$ be a line of the plane and suppose $P_1, P_2, P_3, Q$ are four
  distinct points of $r$. As a consequence of proposition~\ref{rango67}, 
  if $r$ is not tangent to the isotropic conic, the rank of the
  matrix $[\Phi(P_1), \Phi(P_2), \Phi(P_3), \Phi(Q)]$ is $7$. If a cubic
  curve has $P_1, P_2, P_3, Q$ as eigenpoints, then all the points of
  $r$ are eigenpoints (take another point $T$, the rank of the matrix
  obtained from $P_1, P_2, P_3, Q, T$ is still $7$, so $T$ must be an
  eigenpoint). The family of cubic curves with $P_1, P_2, P_3, Q$ as
  eigenpoints (and hence with all the points of $r$ as eigenpoints)
  is a family of dimension $2$. But if we take the cubics given by
  $r^2s$ where $s$ is any line of the plane, we get a family of dimension
  $2$ of cubic curves with all the points of $r$ as eigenpoints, hence
  we have more or less proved:
  \begin{prop}
    Let $r$ be a line of the plane not tangent to the isotropic conic.
    Then all the cubics of the form $r^2s$ have all the points of $r$
    as eigenpoints. Conversely, if we have a cubic which has a line $r$
    of eigenpoints and $r$ is not tangent to the isotropic conic, then
    $c$ is of the form $r^2s$ for a suitable line $s$ of the plane. 
  \end{prop}


  Examples:
  \begin{itemize}
    \item Un file che permette di costrire cubiche con una retta di
      autopunti: \\
      \texttt{lineOfEigenpoints.sage}
    \item Un file che permette di costruire tre punti allineati su una
      retta r che \`e tangente alla conica isotropa nel primo dei tre
      punti: \\
      \texttt{pointOrthogonalToLine.sage}
    \item Matrici di autopunti di rango 8:
      \texttt{examplesMatRank8.sage}
\end{itemize}

\end{document}
