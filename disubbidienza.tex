\documentclass[12pt, a4paper, reqno, captions=tableheading,bibliography=totoc]{scrartcl}

\usepackage[utf8]{inputenc}
\usepackage[T1]{fontenc}
\usepackage{mathtools}
\usepackage{amsthm}
\usepackage{amsmath}
\usepackage{amssymb}
\usepackage{lmodern}
\usepackage[english]{babel}
\usepackage{booktabs}
\usepackage{float}
\usepackage{enumitem}
\usepackage{tikz,pgf}
%\usepackage[utopia]{mathdesign}
\usepackage{palatino}
\usepackage{hyperref}
\hypersetup{
    colorlinks,
    linkcolor={red!50!black},
    citecolor={blue!50!black},
    urlcolor={blue!80!black}
}
\usepackage[nameinlink]{cleveref}

\theoremstyle{plain}
\newtheorem{lemma}{Lemma}[section]
\newtheorem{prop}[lemma]{Proposition}
\newtheorem{theorem}[lemma]{Theorem}
\newtheorem{corollary}[lemma]{Corollary}
\newtheorem{conjecture}[lemma]{Conjecture}
\newtheorem{fact}[lemma]{Fact}
\newtheorem{assumption}[lemma]{Assumption}
\newtheorem*{reduction}{Reduction}
\theoremstyle{definition}
\newtheorem{definition}[lemma]{Definition}
\newtheorem{es}[lemma]{Example}
\newtheorem*{notation}{Notation}
\newtheorem{rmk}[lemma]{Remark}


\newcommand{\N}{\mathbb{N}}
\newcommand{\Z}{\mathbb{Z}}
\newcommand{\Q}{\mathbb{Q}}
\newcommand{\R}{\mathbb{R}}
\newcommand{\C}{\mathbb{C}}
\newcommand{\p}{\mathbb{P}}
\newcommand{\sP}{\mathcal{P}}
\newcommand{\sL}{\mathcal{L}}
\newcommand{\de}{\partial}
\newcommand{\codim}{\mathrm{codim}}

\newcommand{\oo}{\mathcal{O}}
\newcommand{\Bl}{\mathrm{Bl}}

\newcommand{\iso}{\mathcal{Q}}

\newcommand{\imunit}{i}

\newcommand{\SO}{\operatorname{SO}}
\newcommand{\Eig}[1]{\operatorname{Eig}\left( {#1} \right)}
\newcommand{\polq}{{\rm Pol}_Q}
\newcommand{\comment}[1]{}

\newcommand{\scl}[2]{\left\langle {#1}, {#2} \right\rangle}

\newcommand\scalemath[2]{\scalebox{#1}{\mbox{\ensuremath{\displaystyle #2}}}}

\newcommand{\iii}{\textbf{i}}

\title{Eigenpoint collinearities of plane cubics}
\author{}
\date{}

\linespread{1.2}
\setlength{\parindent}{0pt}
\setlength{\parskip}{.25em}

\begin{document}

\maketitle

\section{Introduction}
\label{introduction}

However, there are several examples of cubic polynomials, whose eigenscheme contains one or more triples of aligned points, as for instance the Fermat cubic polynomial.

The goal of this paper is to classify all the situations when we have one or more triples of aligned points inside a zero-dimensional reduced eigenscheme of a cubic plane curve.

\Cref{aligned} clarifies the reasons of interest in the situation of triples of aligned eigenpoints.


\section{Aligned eigenpoints of ternary cubic form}
\label{aligned}

We recall that, given a homogeneous form $f \in \C[x,y,z]_d$ of degree~$d$, the eigenscheme~$\Eig{f}$ of~$f$ is the determinantal scheme defined by the $2 \times 2$ minors of the matrix
%
\begin{equation}
\label{eq:def_matrix}
    \begin{pmatrix}
    x & y & z \\
    \de_x f & \de_y f & \de_z f
    \end{pmatrix}.
\end{equation}
%
Since the eigenschemes of two proportional homogenous forms are the same,
if $C$ is the curve defined by these forms,
we can write $\Eig{C}$ for such eigenscheme and hence talk about the eigenscheme of a plane curve.

The eigenscheme of a general ternary cubic form has no aligned triples of points. This is a consequence of the geometric properties of the classical Geiser map associated with seven points in the plane, and has been proved in \cite[Proposition~4.5]{BGV}.

Moreover, whenever the eigenscheme of a ternary form is zero-dimensional, it never cointains $4$ or more aligned points.
To see this, we recall a property of eigenschemes of ternary forms, namely,
that when zero-dimensional they are somehow ``general'' with respect to conics.

\begin{lemma}
\label{lemma:no_six_conic}
Let $f \in \C[x,y,z]_d$ be a homogeneous form of degree~$d$.
If $E(f)$ is zero-dimensional,
then no degree six subscheme of~$E(f)$ lies on a conic.
\end{lemma}
\begin{proof}
See \cite[Lemma~9.1]{OS1} for the reduced case.
The proof works also in the non-reduced case.
\end{proof}

\begin{corollary}
\label{corollary:general_no_triple}
As a consequence of \Cref{lemma:no_six_conic}, a zero-dimensional eigenscheme~$\Eig(f)$ for $f \in \C[x,y,z]_d$ never contains $4$ or more aligned points.
Moreover, if $\Eig{f}$ contains two triples of aligned points, those must share a point.
\end{corollary}

Therefore, we investigate the situations when we have one or more triples of aligned eigenpoints in a zero-dimensional, reduced eigenscheme of a ternary cubic form.

\section{Invariance under the action of orthogonal matrices}

\begin{definition}
 We define $\mathrm{SO}_3(\mathbb{C})$ to be the complexification of the group of special orthogonal real matrices, namely
 %
 \[
  \mathrm{SO}_3(\mathbb{C}) :=
  \bigl\{
   M \in \mathrm{GL}_3(\C) \, \mid \,
   M M^t = I_3 \  \text{and} \  \det(M) = 1
  \bigr\} \,.
 \]
 %
 The group $\mathrm{SO}_3(\mathbb{C})$ acts on $\C^3$ by matrix multiplication:
 %
 \[
  \begin{array}{ccc}
   \mathrm{SO}_3(\mathbb{C}) \times \C^3 & \rightarrow & \C^3 \\
   (M, v) & \mapsto & Mv
  \end{array}
 \]
 %
 Since all the elements of $\mathrm{SO}_3(\mathbb{C})$ are invertible, the latter action descends to an action on $\p^2(\C)$.

 Moreover, the group~$\mathrm{SO}_3(\mathbb{C})$ acts also on ternary forms via
 \[
  M \cdot f (x,y,z) = f(M^{-1} \cdot \prescript{t} {}( x \ y \ z )  ).
 \]
\end{definition}

\begin{prop}
\label{two_orbits}
 The action of $\mathrm{SO}_3(\mathbb{C})$ on $\p^2(\C)$ has two orbits:
 %
 \begin{align*}
  \mathcal{O}_1 &:=
  \bigl\{
   P \in \p^2(\C) \, | \,
   P = (a:b:c) \  \text{with} \  a^2 + b^2 + c^2 = 0
  \bigr\} \\
  \mathcal{O}_2 &:= \p^2(\C) \setminus \mathcal{O}_1
 \end{align*}
 %
 A representative for $\mathcal{O}_1$ is $(1:\iii:0)$ and a
representative for $\mathcal{O}_2$ is $(1:0:0)$.
\end{prop}
\begin{proof}
 Suppose that $P \in \p^2(\C)$ and $P = (a:b:c)$ with $a^2 + b^2 + c^2 = 0$.
 We produce a matrix $M \in \mathrm{SO}_3(\C)$ such that $M \left(\begin{smallmatrix} 1 \\ \iii \\ 0 \end{smallmatrix}\right)$ and $\left(\begin{smallmatrix} a \\ b \\ c \end{smallmatrix}\right)$ are proportional.
 Up to relabeling the coordinates, we can suppose that $a \neq 0$.
 Hence, by rescaling the coordinates of $P$, we have $P = (1: b: c)$ with $b^2 + c^2 = -1$.
 One can check that the matrix
 %
 \[
  M :=
  \begin{pmatrix}
   -1 & 0 & 0 \\
   0 & \iii b & -\iii c \\
   0 & \iii c & \iii b
  \end{pmatrix}
 \]
 %
 satisfies the requirements.

 Now suppose that $P \in \p^2(\C)$ and $P = (a:b:c)$ with $a^2 + b^2 + c^2 \neq 0$.
 Up to rescaling, we can suppose that $a^2 + b^2 + c^2 = 1$.
 Again, we produce a matrix $M \in \mathrm{SO}_3(\C)$ such that $M \left(\begin{smallmatrix} 1 \\ 0 \\ 0 \end{smallmatrix}\right)$ and $\left(\begin{smallmatrix} a \\ b \\ c \end{smallmatrix}\right)$ are proportional.
 First of all, suppose that $b^2 + c^2 \neq 0$ and let $\omega$ be a root of the polynomial $t^2 - (b^2 + c^2)$ in $\C[t]$.
 Then, the matrix
 %
 \[
   M :=
   \begin{pmatrix}
     a & \omega & 0 \\
     b & -\frac{ab}{\omega} & \frac{c}{\omega} \\
     c & -\frac{ac}{\omega} & -\frac{b}{\omega}
   \end{pmatrix}
 \]
 %
 satisfies the requirements.
 With the same technique, if $a^2 + c^2 \neq 0$, we can produce a matrix $M \in \mathrm{SO}_3(\C)$ that maps $\left(\begin{smallmatrix} 0 \\ 1 \\ 0 \end{smallmatrix}\right)$ to $\left(\begin{smallmatrix} a \\ b \\ c \end{smallmatrix}\right)$; similarly, when $a^2 + b^2 \neq 0$, we can map $\left(\begin{smallmatrix} 0 \\ 0 \\ 1 \end{smallmatrix}\right)$ to $\left(\begin{smallmatrix} a \\ b \\ c \end{smallmatrix}\right)$.
 Since $\left(\begin{smallmatrix} 1 \\ 0 \\ 0 \end{smallmatrix}\right)$, $\left(\begin{smallmatrix} 0 \\ 1 \\ 0 \end{smallmatrix}\right)$, and $\left(\begin{smallmatrix} 0 \\ 0 \\ 1 \end{smallmatrix}\right)$ are all $\mathrm{SO}_3(\C)$-equivalent, the only case to consider is when
 %
 \[
  b^2 + c^2 = a^2 + c^2 = a^2 + b^2 = 0 \,,
 \]
 %
 which, however, can never occur.
\end{proof}

The following result is well known; we recall it for the sake of completeness.

\begin{prop}
 Let $M \in \mathrm{GL}_3(\C)$ and let $f$ be a ternary cubic.
 Let $P = (A: B: C)$ be a point in~$\p^2$.
 Then we have
 \[
  P \in \Eig{f} \iff M \cdot \prescript{t} {}(A \ B \ C) \in \Eig{M \cdot f}.
 \]
\end{prop}

\begin{proof}
In the present proof,
for convenience, we shall consider the transpose of the defining matrix of an eigenscheme.

A point $P = (A: B: C)$ is an eigenpoint for~$f$ if and only if
\begin{equation*}
  \text  {rk}  \begin{pmatrix}
    A & \de_x f(P) \\
    B & \de_y f(P)  \\
    C & \de_z f(P)
    \end{pmatrix}=1,
\end{equation*}
which is equivalent to
\begin{equation}
\label{eq:def_matrix_M}
    \text  {rk} \quad  M  \cdot \begin{pmatrix}
    A & \de_x f(P) \\
    B & \de_y f(P)  \\
    C & \de_z f(P)
    \end{pmatrix}
    =1.
\end{equation}
By setting $\prescript{t} {}(A' \ B' \ C' )= M \cdot \prescript{t} {}(A \ B \ C) $ and $Q=(A':B':C')$, we have that \Cref{eq:def_matrix_M}
is equivalent to
%
\begin{equation}
\label{eq:transformed}
  \text{rk}
  \begin{pmatrix}
    A' &  \\
    B' & M \cdot \nabla f (P) \\
    C' & \\
  \end{pmatrix}=1.
\end{equation}
%
Now we consider the polynomial $M \cdot f$ and we observe that the chain rule gives
%
\begin{gather*}
\partial_x (M\cdot f) = \partial_x \bigl( f(M^{-1}  \ \prescript{t} {} (x \ y \ z)) \bigr) = \prescript{t} {}(M^{-1})^{(1)}(\nabla f) \bigl( M^{-1}\   \prescript{t} {} (x \ y \ z) \bigr) \,, \\
\partial_y (M\cdot f) = \partial_y \bigl( f(M^{-1}  \ \prescript{t} {} (x \ y \ z)) \bigr) = \prescript{t} {}(M^{-1})^{(2)}(\nabla f) \bigl( M^{-1}\   \prescript{t} {} (x \ y \ z) \bigr) \,, \\
\partial_z (M\cdot f) = \partial_z  \bigl( f(M^{-1}  \ \prescript{t} {} (x \ y \ z)) \bigr) = \prescript{t} {}(M^{-1})^{(3)}(\nabla f) \bigl( M^{-1}\   \prescript{t} {} (x \ y \ z) \bigr) \,,
\end{gather*}
%
where $(M^{-1})^{(j)}$ denotes the $j$-th column of the matrix $M^{-1}$. Hence
%
\[
\nabla (M \cdot f) = \prescript{t} {} M^{-1} \cdot (\nabla f) (M^{-1}\   \prescript{t} {} (x \ y \ z)),
\]
%
so we have
%
\[
\nabla (M \cdot f)(Q)=\nabla (M \cdot f)(M \cdot P)=
\prescript{t} {} M^{-1} \cdot (\nabla f) (M^{-1}\   M \cdot P)=\prescript{t} {} M^{-1} \cdot (\nabla f)(P).
\]
%
Finally, if we choose $M \in \mathrm{SO}_3(\mathbb{C})$, we have
$\prescript{t} {} M^{-1}=M$. We deduce that
\Cref{eq:transformed} holds if and only if $Q \in E(M\cdot f)$, so the statement is proved.
\end{proof}

\section{...................}

Imposing that a cubic ternary form has one or more aligned triples of eigenschemes imposes conditions both on the points and on the cubics.
We begin to explore these conditions by introducing, for each point
$P \in \p^2$,
a $3 \times 10$ matrix encoding the condition that $P$ is an eigenpoint of a ternary cubic.

\begin{definition}
\label{definition:matrix_conditions}
 Consider $\p^9 = \p(\C[x,y,z]_3)$, the space of all ternary cubics.
 Throught this paper, we fix the following vector basis for $\C[x,y,z]_3$:
 \[
  (x^3, x^2 y, x y^2, y^3, x^2 z, x y z, y^2 z, x z^2, y z^2, z^3)
 \]
 For $f \in \C[x,y,z]_3$, denote by $[f]$ the corresponding point in~$\p^9$; we denote by $w_f$ the (column) vector of coordinates of~$f$ with respect to the basis above; then $w_f$ is also a vector of projective coordinates of~$[f]$.
 For a point $P \in \p^2$ with coordinates $(A: B: C)$, the condition on elements~$[f]$ of~$\p^9$ that $P$ is an eigenpoint of the ternary cubic form~$f$ can be expressed in the form
 %
 \[
  \Phi(P) \cdot w_f
  = 0 \,,
 \]
 %
 where $\Phi(P)$ is a $3 \times 10$ matrix with entries depending on $A, B, C$.
 The matrix $\Phi(P)$ is called the \emph{matrix of conditions} imposed by~$P$.
We denote by $\phi_1(P)$, $\phi_2(P)$, and~$\phi_3(P)$ the rows of~$\Phi(P)$.
They are
%
\begin{equation}
\label{equation:matrix_conditions_rows}
\begin{gathered}
\scalemath{0.9}{(-3A^2B, A(A^2 - 2B^2), B(2A^2 - B^2), 3AB^2,
 -2ABC, C(A^2 - B^2), 2 ABC,
 -B  C^2, A  C^2, 0)} \,, \\
\scalemath{0.9}{(-3A^2 C,
-2ABC,
-CB^2,
0,
A(A^2-2C^2),
B(A^2 - C^2),
AB^2,
C(2A^2-C^2),
2ABC,
3AC^2)} \,,\\
\scalemath{0.9}{(0,
-A^2C,
-2ABC,
-3CB^2,
A^2 B,
A(B^2 - C^2),
B(B^2-2C^2),
2ABC,
C(2B^2-C^2),
3BC^2)} \,.
\end{gathered}
\end{equation}
%
Moreover, if $P_1, \dotsc, P_n$ are points in the plane, we denote by $\Phi(P_1, \dotsc, P_n)$ the matrix
%
\[
 \left(
 \begin{array}{c}
  \Phi(P_1) \\
  \vdots \\
  \Phi(P_n)
 \end{array}
 \right)
\]
%
namely, the $3n \times 10$ matrix obtained by stacking the matrices of conditions of~$P_1, \dotsc, P_n$ on top of each other.
\end{definition}

\begin{rmk}
From \Cref{equation:matrix_conditions_rows}, it follows that $\Phi(P)$ has never rank strictly lower than~$2$.
Moreover, it holds:
\begin{equation}
  C \, \phi_1(P) - B \, \phi_2(P) + A \, \phi_3(P) = 0,
  \label{eq:base}
\end{equation}
therefore,
among the three vectors, at most two of them are linearly independent.
Hence, the matrix~$\Phi(P)$ has rank~$2$.
\end{rmk}

Hence, if $P_1, \dots, P_n$ are $n$ points of the plane, we have:
\begin{equation}
\label{bound_rank}
\text{rk} \  \Phi(P_1, \dots, P_n) \leq \min \left\{2n, 10 \right\}
\end{equation}

Here we want to study the possible values of the rank of the matrix
$\Phi(P_1, \dots, P_n)$ for several configurations of points $P_1, \dots, P_n$
(and several values of $n$).
We will therefore have to study the ideal~$J_k$ of order $k$ minors of the
involved matrix and to deduce the properties of the rank from the possible
decompositions of the ideal~$J_k$. Most of these computations will be done
with the auxiliary of a computer algebra system. Nevertheless, in many cases,
the result cannot be reached just by brute force, but it is necessary to
make some preprocessing on the ideal~$J_k$. In particular, it turns out that
it is often convenient to first saturate the ideal~$J_k$ with respect to
the conditions that the
points are distinct or that three of them are not aligned (when this is the
case). Another important simplification that we adopt sometimes, makes use
of the action of $\SO_3(\C)$: thanks to it we can assume that one of
the point is either $(1: 0: 0)$ or $(1: \iii: 0)$; see \Cref{two_orbits}.

Finally, several times the following lemma will be extremely useful
to speed up the computations.

\begin{lemma}
\label{lemma:minors}
Let $l_1 < \cdots <l_n$ be $n$ indices (where $3 \leq n \leq 10$) and let $P = (A: B: C)$ be a point of the plane.
Construct three $1 \times n$ matrices $w_1$, $w_2$, $w_3$ by extracting the entries of position $l_1, \dotsc, l_n$ from $\phi_1(P)$, $\phi_2(P)$, and~$\phi_3(P)$, respectively. If $L$ is a $(n-2) \times n$ matrix, set:
  \[
  L_1 := \left(\begin{array}{c}w_1 \\ w_2 \\ L\end{array}  \right), \quad
  L_2 := \left(\begin{array}{c}w_1 \\ w_3 \\ L\end{array}  \right), \quad
  L_3 := \left(\begin{array}{c}w_2 \\ w_3 \\ L\end{array}  \right)
  \]
  Then
  \[
  B \det(L_1) = A \det(L_2), \quad
  C \det(L_1) = A \det(L_3), \quad
  C \det(L_2) = B \det(L_3)
  \]
  hence $(A: B: C) = \bigl( \det(L_1): \det(L_2): \det(L_3) \bigr)$.
\end{lemma}
\begin{proof}
  The thesis easily follows from the equality $C w_1 - B w_2 + A w_3 = 0$, which is a direct consequence of \Cref{eq:base}.
\end{proof}

% \begin{lemma}
%   \label{lemma_for_minors2}
%   Let $P(A: B: C)$, $Q(A': B': C')$ be two points of $\mathbb{P}^2_K$
%   and set $R = \alpha P+\beta Q$ (where $\alpha, \beta \in K$).
%   Suppose $l_1 < \cdots < l_n$ are as above, fix $j \in \{1, 2, 3\}$
%   and   let $w_1$ be the vector obtained
%   from the entries of $\phi_j(P)$ of position $l_1, \dots, l_n$
%   and $w_2$ be the vector analogously obtained from $R$. If
%   $L$ is a square matrix which contains the
%   rows $w_1$ and $w_2$, then $\alpha\beta$ divides $\det(L)$.
% \end{lemma}
% \begin{proof}
%   We consider $\det(L)$ as a polynomial in $\alpha$.
%   If we set $\alpha = 0$,
%   we have that $w_1$ and $w_2$ are proportional, so, if $\alpha = 0$
%   then $\det(L)$ is $0$. Hence $\alpha$ divides $\det(L)$. Same
%   argument for $\beta$.
% \end{proof}


First of all, we evidentiate a property of the lines tanget to the isotropic
conic. Let $P_1, P_2$ be two distinct points in the plane and define
%
\[
  \sigma(P_1, P_2) := \scl{P_1}{P_1} \scl{P_2}{P_2} - \scl{P_1}{P_2}^2 \,.
\]
%
The quantity~$\sigma$ is the discriminant of the intersection between the line~$r$ through~$P_1$ and~$P_2$ and the isotropic conic.


\begin{prop}
  Let $P_1$, $P_2$ be two distinct points in the plane and let $r$ be the line between them.
  Then the following are equivalent:
  \begin{enumerate}
  \item $\sigma(P_1, P_2) = 0$;
  \item the line $r$ is tangent to the isotropic conic at some point;
  \item there exists a point $T \in r$ such that $T$ is orthogonal to $r$.
  \end{enumerate}
\end{prop}
\begin{proof}
  By recalling the characterization of~$\sigma$ as a discriminant, we have that $\sigma(P_1, P_2) = 0$ if and only if $r$ is tangent to the isotropic conic in a point~$T$.
  For such a point~$T$ we then have $\scl{T}{T} = 0$.
  By bilinearity, the condition $\sigma(P_1, P_2) = 0$ implies that $\sigma(T, Q) = 0$ for all $Q \in r$.
  Since  $\sigma(T, Q) = \scl{T}{T}\scl{Q}{Q}-\scl{T}{Q}^2$, we get
  $\scl{T}{Q} = 0$, hence $T$ is orthogonal to $r$.
  The converse is immediate.
\end{proof}

\begin{corollary}
A line $r$ in the plane is tangent to the isotropic conic if and only
if $\sigma(P_1, P_2) = 0$ for all points $P_1, P_2 \in r$ with $P_1 \neq P_2$.
\end{corollary}

\begin{corollary}
A line $r$ in the plane is tangent in a point $P$ to the isotropic conic
if and only if $\scl{P}{Q} = 0$ for all points $Q \in r$.
\end{corollary}

\begin{prop}
\label{pr2}
Let $P_1, P_2, P_3$ be three distinct aligned points of the plane and let
$r$ be the line passing through them. Then:
\begin{itemize}
\item $5 \leq \text{rk}\ \Phi(P_1, P_2, P_3) \leq 6$;
\item
$\text{rk}\ \Phi(P_1, P_2, P_3) = 5$ if and only if $r$ is tangent
to the isotropic conic in one of the three points $P_1, P_2$, or $P_3$.
\end{itemize}
\end{prop}
\begin{proof} We fix the following coordinates for the points:
\[
P_1, P_2, P_3 = (A_1: B_1: C_1), (A_2: B_2: C_2), u_1P_1+u_2P_2.
\]
We have to consider the ideal generated by the $17640$ order six
minors of the matrix $\Phi(P_1, P_2, P_3)$, we can however greately simplify
the computations.
Consider the matrix $M$ constructed with the following six rows:
\[
\phi_{i_1}(P_1), \ \phi_{i_2}(P_1), \ \phi_{j_1}(P_2),\  \phi_{j_2}(P_2),
\phi_{k_1}(P_3), \ \phi_{k_2}(P_3)
\]
where $i_1 \not= i_2 \in \{1, 2, 3\}$; $j_1 \not= j_2 \in \{1, 2, 3\}$;
$k_1 \not= k_2 \in \{1, 2, 3\}$.
First, assume that $i_1=j_1=k_1=1$ and
$i_2=j_2=k_2=2$. Fix six columns
$1\leq l_1 < \cdots l_6 \leq 10$ of $M$ and let $N$ be the order $6$ matrix
obtained from $M$ with these columns. Since two rows of $N$ are obtained from
entries of $\phi_1(P_1)$ and of $\phi_2(P_1)$,
\Cref{lemma:minors} gives that $A_1$ divides $\det(N)$. For a similar
reason, also $A_2$ and $u_1A_1+u_2A_2$ divide $\det(N)$, so
$\det(N) = A_1A_2(u_1A_1+u_2A_2)\cdot D$ for a suitable $D$.
Again by \Cref{lemma:minors}, we also see that if we take different values of
$i_1, i_2, j_1, j_2, k_1, k_2$ in the definition of $M$ above, then
the corresponding $\det(N)$ would be of the form $X_1X_2X_3\cdot D$ where
$X_1$ is a coordinate of $P_1$, $X_2$ is a coordinate of $P_2$ and $X_3$ is
a coordinate of $P_3$ (and $D$ is the same as above). Since each of the three
points have at least one non-zero coordinate, if the order six minor
constructed with the columns $l_1, \dots, l_6$ is zero, then necessarily $D$
must be zero. In this way we see that, in order to have that all the order
six minors of $\Phi(P_1, P_2, P_3)$ are zero, it is enough to compute the
ideal of the order six minors of the matrix $M$ above and divide each
minor by $A_1A_2(u_1A_1+u_2A_2)$. In this way we get an ideal that (after
a saturation w.r.t.\ the condition that $P_1$, $P_2$, and $P_1, P_3$ and
$P_2$, $P_3$ are distinct), has
a very simple primary decomposition which is given by the following three
ideals:
\[
\left(\scl{P_i}{P_1}, \scl{P_i}{P_2},\scl{P_i}{P_3}\right) \quad
\mbox{for $i = 1, 2, 3$}
\]
hence the line $r$ is tangent to the isotropic conic
(in $P_1$ or $P_2$ or $P_3$).
It is easy to see that it is not possible to have that all the order
$4$-minors of $\Phi(P_1, P_2, P_3)$ are zero. @@ siamo sicuri sicuri??@@
\end{proof}

\begin{prop}
\label{manca il riferimento su ancillary    non e': condition_rank_aligned}
%%%  conto si trova su file prop47_5ago.sage
Let $P_1, P_2, P_4$ be three distinct points of the plane. Then:
\begin{itemize}
\item $5 \leq \text{rk}\ \Phi(P_1, P_2, P_4) \leq 6$;
\item if
$\text{rk}\ \Phi(P_1, P_2, P_4) = 5$, then $P_1, P_2, P_4$
 are aligned.
\end{itemize}
In the last case, the line $P_1+P_2+P_4$ is tangent to the isotropic conic
in one of the three points.
\end{prop}
\begin{proof}
We can split the proof into two parts, considering the case in
which $P_4 = (1: 0: 0)$ and the case in which $P_4 = (1: \iii: 0)$.
In both cases, che computation of the ideal of order five minors of the matrix
$\Phi(P_1, P_2, P_4)$ and the saturation of it w.r.t.\ the condition
that the points $P_1, P_2, P_4$ are distinct gives the while ring, so
the matrix cannot have rank smaller then $5$. Meanwhile, the computation
of the ideal of the order six minors of $\Phi(P_1, P_2, P_4)$ and its
saturation w.r.t. the condition that the points are distinct, gives that
$P_1, P_2, P_4$ must be aligned and, according to \Cref{pr2},
thir line is tangent to the isotropic conic in one of the three given points.
\end{proof}


\begin{prop}
\label{prop:condition3+1}
%\label{pr3}
Let $P_1, P_2, P_3, P_4$ be four distinct points of the plane such that
$P_1, P_2, P_3$ are aligned and let $r$
be the line passing thorough them. If
$\text{rk}\ \Phi(P_1, P_2, P_3, P_4) \leq 7$ then $r$ is tangent to the
isotropic conic in one of the three points $P_1, P_2, P_3$.
\end{prop}
\begin{proof}
Also in this case, we distinguish two cases: $P_1 = (1: 0: 0)$ and
$P_1 = (1: \iii: 0)$. In both cases, the ideal of the order $8$
minors of $\Phi(P_1, P_2, P_3, P_4)$ can be computed and saturated
w.r.t.\ the conditions that the points are different and that
$P_1, P_2, P_4$ are not aligned.
The direct inspection of ideal obtained form these procedures gives the thesis.
\end{proof}

\begin{prop}
\label{pr4}
Let $P_1, P_2, P_3, Q$ be four distinct aligned points of the plane and
let $r$ be the line through them. Then:
\begin{itemize}
\item $6 \leq \text{rk} \ \Phi(P_1, P_2, P_3, Q) \leq 7$;
\item $\text{rk} \ \Phi(P_1, P_2, P_3, Q) = 6$ if and only if $r$ is tangent
to the isotropic conic.
\end{itemize}
\end{prop}
\begin{proof}
  @@ va riadattata.@@
  A direct computation shows that all the maximal minors of $M$ are
  zero, so $\mathrm{rank}\,(M) \leq 7$.
  We take the submatrix $M_1$ of $M$ whose rows are:
  \[
    \phi_1(P_1), \phi_2(P_1), \phi_1(P_2), \phi_2(P_2),
    \phi_1(P_3), \phi_2(P_3), \phi_1(Q)
  \]
  We consider the coordinates of the points as variables and we compute
  the ideal $J_1$, the radical of the ideal of the maximal minors of $M_1$;
  a direct computation shows that $J_1$ is a principal ideal generated by:
  %
  \begin{equation}
  \label{eq:genJ1}
  u_1 u_2 w_1 w_2 (u_1w_2-u_2w_1) A_1 A_2 (u_1A_1+u_2A_2) (A_1B_2-A_2B_1)
\sigma(P_1,P_2) \,,
  \end{equation}
  %
  where $P_i = (A_i: B_i: C_i)$ for $i \in \{1,2\}$ and $P_3 = u_1 P_1 + u_2 P_2$.
  The factors $u_1$, $u_2$, $w_1$, $w_2$, and~$u_1w_2-u_2w_1$ appear due to the fact that if
  $P_3$ (or $Q$) coincides with $P_1$ (or, respectively, with $P_2$),
  the rank of $M_1$ is less than $7$
  and the same is true if $P_3$ and $Q$ coincide.
  As a consequence of
  \Cref{lemma:matrix_conditions_rows}, we know that if in the matrix~$M_1$, instead of the two rows
  $\phi_1(P_1)$ and $\phi_2(P_1)$ we take, for instance,
  $\phi_2(P_1), \phi_3(P_1)$, the radical of $J_1$ has the factor
  $C_1$ in place of the factor $A_1$.
  If in $M$, in place of the last row~$\phi_1(Q)$, we take $\phi_2(Q)$ (or $\phi_3(Q)$), the ideal
  $J_1$ has the factor $A_1C_2-A_2C_1$ (respectively, $B_1C_2-B_2C_1$) in
  place of $A_1B_2-A_2B_1$ (these results come from a direct computation
  of the radical of the ideal of the minors).
  Hence, as soon as we know the generator from \Cref{eq:genJ1}, we know all the
  possible generators of the radical of the ideals of all the $7\times 10$
  minors and they are all multiple of $\sigma(P_1, P_2)$. Moreover, if
  all the points are distinct (each given by a triple of not all zero
  coordinates), it is possible to select a minor of order $7$ which
  is zero if and only if $\sigma(P_1, P_2)$ is zero.
  So far, we have considered only the case in which the matrix
  $M_1$ is given by two vectors of $\Phi(P_1)$, two vectors of
  $\Phi(P_2)$, two vectors of $\Phi(P_3)$ and one vector of $\Phi(Q)$.
  To complete the proof, we have to consider the further case in which
  $M_1$ is given by two vectors of $\Phi(Q)$ and therefore only one
  vector taken from $\Phi(P_1)$ or from $\Phi(P_2)$ or from $\Phi(P_3)$.
  This case is similar to the previous ones, since, by
  \Cref{prop:orthogonal_tangent}, we can exchange $Q$ with one of the points
  $P_1$ or $P_2$ or $P_3$.
\end{proof}

We now define three quantities depending on a triple or on a $5$-tuple of points in the plane.
These quantities are crucial to describe what happens when we have aligned eigenpoints.

\begin{definition}
\label{definition:delta1}
 Let $P$, $Q$ and~$R$ be distinct points in the plane.
 We define the quantity
 %
 \[
  \delta_1(P, Q, R) :=
  \scl{P}{P} \scl{Q}{R} - \scl{P}{Q}\scl{P}{R}
  =
  \scl{P\times Q}{P \times R} \,,
 \]
 %
 where $\times$ denotes the cross product. i.e.,
 %
 \[
  P \times Q = (P_2 Q_3 - P_3 Q_2. P_3 Q_1 - P_1 Q_3, P_1 Q_2 - P_2 Q_1) \,.
 \]

\end{definition}

\begin{definition}
\label{definition:delta1b}
 Let $P$, $Q$ and~$R$ be distinct aligned points in the plane.
 We define the quantity
 %
 \[
  \overline{\delta}_1(P, Q, R) :=
  \scl{P}{P} \scl{Q}{R} + \scl{P}{Q}\scl{P}{R} \,.
  \]
 %
\end{definition}

\begin{definition}
\label{Vconf}
Let $P_1, P_2, P_3, P_4, P_5$ be five distinct points of the plane
such that $P_1, P_2, P_3$ and $P_1, P_4, P_5$ are aligned.
We call such a configuration a \emph{$V$-configuration}.
\end{definition}


\begin{definition}
 Let $P_1, \dots, P_5$ be a $V$-configuration.
We define the quantity
 %
 \[
  \delta_2(P_1, P_2, P_3, P_4, P_5) :=
  \scl{P_1}{P_2} \scl{P_1}{P_3} \scl{P_4}{P_5} -
  \scl{P_1}{P_4} \scl{P_1}{P_5} \scl{P_2}{P_3} \,.
 \]
 %
\end{definition}

\begin{lemma}
\label{prop:d1d2}
Let $P_1, \dots, P_5$ be a $V$-configuration. Then
\[
\text{rk}\ \Phi(P_1, \dots, P_5) \leq 9
\quad \mbox{
if and only if} \quad
\delta_1(P_1, P_2, P_4) \cdot \delta_2(P_1, \dots, P_5) = 0.
\]
\end{lemma}
\begin{proof}
We fix the following coordinates of the points:
$P_1, P_2, P_4 = (A_1: B_1: C_1), (A_2: B_2: C_2), (A_4: B_4, C_4)$,
then $P_3 = u_1P_1+u_2P_2$ and $P_5 = v_1P_1+v_2P_4$. We proceed as in
the proof of \Cref{pr2}: we compute the determinant of
the order $10$ matrix $M$ whose rows are
$\phi_i(P_j)$ for $i=1, 2$ and $j=1, \dots, 5$.
Its value is:
5
\begin{gather}
\label{delta1delta2}
A_1A_2A_4(u_1A_1+u_2A_2)(v_1A_1+v_2A_4)u_1^2u_2^2v_1^2v_2^2D^5
\delta_1(P_1,P_2,P_4)\delta_2(P_1,\dots,P_5)
\end{gather}
%
where $D$ is the determinant of the matrix whose rows are $P_1, P_2, P_4$
and is non-zero, as well as are non-zero $u_1, u_2, v_1, v_2$ (since we
assume that $P_1, P_2, P_4$ are not aligned and the points are distinct).
As a consequence of \Cref{lemma:minors}, we have that a different
choice of the rows of $M$ would simply modify in the
determinant from \Cref{delta1delta2} the coordinates
$A_1, A_2, \dots, (v_1A_1+v_2A_4)$ to some other coordinates of the points.
Since every point has at least one non-zero coordinate, we have that
$\text{rk}\ \Phi(P_1, \dots, P_5) \leq 9$ if and only if
\[
\delta_1(P_1, P_2, P_4) \cdot \delta_2(P_1, \dots, P_5) = 0 \,. \qedhere
\]
\end{proof}

\begin{prop}
\label{prop:frecciaFissata}
Let $P_1, \dots, P_5$ be a $V$-configuration of points and assume that
\[
\scl{P_1}{P_2}=0, \quad \scl{P_2}{P_2}=0, \quad \scl{P_1}{P_4}=0,
\quad \scl{P_4}{P_4}=0.
\]
Then the matrix $\Phi(P_1, \dots, P_5)$ has rank $8$.
\end{prop}
\begin{proof}
{From} the above conditions and from \Cref{pr2} we have that
the lines $P_1+P_2$ and the line $P_1+P_4$ are tangent to the isotropic
conic in $P_2$ and $P_4$ respecrtively. The point $P_1$ cannot be on
the isotropic conic, hence, using the
action of $\mathrm{SO}_3(\mathbb{C})$, we can assume $P_1 = (1: 0: 0)$.
Since every element of $\mathrm{SO}_3(\mathbb{C})$ fixes the
isotropic conic and sends a tangent line to it into another
tangent line to it, when we transform the point $P_1$
into $(1: 0: 0)$, we transform the points $P_2$ and $P_4$ into, respectively,
the points $(0: \iii: 1)$ and $(0: -\iii: 1)$ (which are the common points to
the isotropic conic and the tangent lines through $P_1$).
Therefore it is enough to study the
specific configuration of the points given by:
\[
P_1 = (1: 0: 0), \quad P_2=(0: \iii: 1), \quad P_3=u_1P_1+u_2P_2, \quad
P_4 = (0: -\iii: 1), \quad P_5 = v_1P_1+v_2P_4.
\]
In the $15\times 10$ matrix $\Phi(P_1, \dots, P_5)$ we can erase the
rows: $\phi_2(P_1)$ (which is a zero row), the row $\phi_1(P_2)$
(since $\phi_1(P_2)=i\phi_2(P_2)$) and, for similar reason, the
rows: $\phi_1(P_3)$, $\phi_1(P_4)$ and $\phi_1(P_5)$.
Moreover, since the rank of the matrix $\Phi(P_1, P_2, P_3)$ is $5$,
we can also erase the row $\phi_3(P_3)$ and, for the same reason, the
row $\phi_3(P_5)$. The remaining matrix $M$ is a $8\times 10$ matrix whose rank
in general is $8$. It is not possible to have that all the order $8$ minors
of $M$ are zero, since this condition implies that the five points
are not all distinct; for the proof of the latter statement, see
the ancillary file. (@@ (da cancellare) @@\verb+prop:frecciaFissata+)
\end{proof}

\begin{prop}
\label{fantasma}
Probabilmente resta da dimostrare:
se 5 punti in configurazione V sono tali che
$\delta_1(P_1, P_2, P_4) = 0$, \ $\overline{\delta}_1(P_1, P_2, P_3) = 0$,
\ $\overline{\delta}_1(P_1, P_4, P_5) = 0$\,, allora il rango di $\Phi$
non puo' essere $\leq 7$.
\end{prop}


\begin{prop}
\label{pr5}
Let $P_1, \dots, P_5$ be a $V$-configuration of
points. Then we have:
\begin{enumerate}
\item $8 \leq \text{rk}\ \Phi(P_1, \dots, P_5) \leq 10$\,;
\item $\text{rk}\ \Phi(P_1, \dots, P_5) \leq 9$ if and only if
$\delta_1(P_1, P_2, P_4) \cdot \delta_2(P_1, \dots, P_5) =0$\,;
\item $\text{rk}\ \Phi(P_1, \dots, P_5) = 8$ if and only if, one of
the following two conditions is satisfied:
\begin{itemize}
  \item the line $P_1+P_2$ is tangent to the isotropic conic in $P_2$ or $P_3$
and the line $P_1+P_4$ is tangent to the isotropic conic in $P_4$ or $P_5$,
\item $\delta_1(P_1, P_2, P_4) = 0$, \
$\overline{\delta}_1(P_1, P_2, P_3) = 0$,
\ $\overline{\delta}_1(P_1, P_4, P_5) = 0$\,.
\end{itemize}
\end{enumerate}
\end{prop}
\begin{proof}
%%% si basa sui file:
%%% rank_8_2_1_ii_0.sage e
%%% rank_8_1.sage
First, note that if the rank is $\leq 7$, in particular, from
\Cref{prop:condition3+1} applied to $P_1, P_2, P_3$ and $P_4$
the line $P_1+P_2$ is tangent to the isotropic conic (in $P_2$ or $P_3$)
and, analogously, the line $P_1+P_4$ is tangent to the isotropic conic
(in $P_4$ or $P_5$). Hence, from \Cref{prop:frecciaFissata},
the matrix $\Phi(P_1, \dots, P_5)$ cannot have rank less then $8$.
In the rest of the proof, we distinguish two possibilities:
$P_1 = (1:\iii :0)$ and
$P_1 = (1: 0: 0)$. The other points are taken generic,
i.e.\ $P_2, P_4 = (A_2: B_2: C_2), (A_4: B_4: C_4)$ and
$P_3 = u_1P_1+P_2$, $P_5 = v_1P_1+v_2P_4$.
In the first case, the matrix
$M = \Phi(P_1, \dots, P_5)$ can be modified with elementary rows (and
columns) operations in order to transform the study of all the
order $9$ minors of $M$ to the order $7$ minors of the transformed matrix.
The computations require several other strategies, but at the end we are
able to see that in this case (i.e.\ the cas in which $P_1$ is on the
isotropic conic), the matrix $M$ cannot have rank less then $9$.
If $P_1 = (1: 0: 0)$, then the ideal of the order $9$ monors of
$\Phi(P_1, \dots, P_5)$ (after several manipulations and saturations)
gives that there a few possibilities which can be summarized by the
following conditions: either the line $P_1+P_2$ is tangent to the isotropic
conic (in $P_2$ or $P_3$) and the line $P_1+P_4$ is tangent to the isotropic
conic (in $P_4$ or $P_5$) or $\delta_1(P_1, P_2, P_4) = 0$,
\ $\overline{\delta}_1(P_1, P_2, P_3) = 0$,
\ $\overline{\delta}_1(P_1, P_4, P_5) = 0$.
To conclude the proof, we use \Cref{prop:frecciaFissata}, \Cref{fantasma}
and \Cref{prop:d1d2}.

\end{proof}


\section{$\delta_1=0$ or $\delta_2=0$}

Here we want to consider a $V$-configuration of points $P_1, \dots, P_5$
such that $\delta_1(P_1, P_2, P_4) = 0$ or $\delta_2(P_1, \dots, P_5) = 0$.
Moreover, here we assume that the rank of the matrix $M$
of the conditions imposed by
the five points (which, by proposition~\ref{pr5}, is at most $9$)
is $9$, hence there exists precisely one cubic curve having
$P_1, \dots, P_5$ among its eigenpoints. In particular, from
proposition~\ref{pr5}, we have that at least one of the two lines
$P_1+P_2$ and $P_1+P_4$ is not tangent to the isotropic conic (in $P_2$
or $P_3$ or, respectively, in $P_4$ or $P_5$). We assume therefore that
$P_1+P_2$ is not tangent to the isotropic conic and then,
by~\ref{prop:condition3+1}, for each point $P_i$ for $i \in
\{1,\dotsc,4\}$ there exist
two indices $i_1, i_2 \in \{1,2,3\}$ such that the matrix~$\mathcal{H}$
whose rows are $\{ \phi_{i_1}(P_i), \phi_{i_2}(P_i)\}_{i=1}^4$ has rank~$8$.
To simplify the notation, we assume that the two indices are always $1$ and
$2$ for every~$P_i$.
Since the matrix whose rows are the rows of
$\mathcal{H}$ and one the rows of $\Phi(P_5)$ has rank $9$, we
can assume that the rank $9$ is realized by the matrix $\mathcal{H}$ plus
the row $\phi_1(P_5)$.

If $P = (x: y: z)$ is a generic point of the plane,
consider the following three square matrices of order~$10$:
%
\begin{eqnarray}
\label{G1G2G3}
\mathcal{G}_1 =
\left(
\begin{array}{c}
  \mathcal{H} \\
  \phi_1(P_5)\\
  \phi_1(P)
\end{array}
\right),
\quad
\mathcal{G}_2 =
\left(
\begin{array}{c}
  \mathcal{H} \\
  \phi_1(P_5)\\
  \phi_2(P)
\end{array}
\right),
\quad
\mathcal{G}_3 =
\left(
\begin{array}{c}
  \mathcal{H} \\
  \phi_1(P_5)\\
  \phi_3(P)
\end{array}
\right)
\end{eqnarray}
and let $g_1$, $g_2$ and $g_3$ be their determinants.
We know by \Cref{eq:base} that $z \phi_1(P)-y\phi_2(P) +x\phi_3(P) = 0$, hence
%
\[
\det
\left(
\begin{array}{c}
  \mathcal{H} \\
  \phi_1(P_5)\\
  z \phi_1(P)-y\phi_2(P) +x\phi_3(P)
\end{array}
\right) = 0
\]
%
using the multilinearity of the determinant, we get the syzygy~$(z,-y,x)$ among
$g_1, g_2, g_3$.
It holds:
\begin{prop}
 \label{threeG}
  A point $P = (A: B: C)$ of the plane is an eigenpoint of $\mathcal{C}$ if and only if
  $P$ is a common zero of $g_1, g_2, g_3$.
\end{prop}
\begin{proof}
  Suppose that $P$ is an eigenpoint of $\mathcal{C}$. Then the $18\times 10$ matrix
  $\mathcal{M} = \Phi(P_1, \dots, P_5, P)$, which is the matrix
  associated to the linear
  system obtained by imposing that the six points are eigenpoints,
  has rank~$9$, since by assumption the matrix has rank at least~$9$ and we know that the linear system has a solution.
  In particular the matrices
  $\mathcal{G}_1, \mathcal{G}_2$ and $\mathcal{G}_3$, which are
  submatrices of $\mathcal{M}$, have determinant zero, so
  $g_1(P) = g_2(P) = g_3(P) = 0$.

  Conversely, suppose that $P$ is a point such that
  $g_1(P) = g_2(P) = g_3(P) = 0$. Then $\phi_1(P)$ , $\phi_2(P)$ and
  $\phi_3(P)$, the three rows of $\Phi(P)$, are linear combinations of
$\phi_1(P_5)$ and the rows of $\mathcal{H}$, so the $12\times 10$ matrix
  %
  \[
\left(
\begin{array}{c}
  \mathcal{H} \\
  \phi_1(P_5)\\
  \Phi(P)
\end{array}
\right)
\]
%
has rank $9$ and the linear system associated to it has a solution,
which is the cubic $\mathcal{C}$, therefore $P$ is an eigenpoint of $\mathcal{C}$.
\end{proof}

\begin{es} Consider the following five points in a $V$ configuration:
\[
p_1, p_2, p_3, p_4, p_5 = (2: -1: 1), \ (-1: 1: 3), \ (1: 0: 4),\
(3: 6: -1), \ (-1: -7: 2)
\]
Then
\[
\mathcal{H} =
\left(\begin{array}{rrrrrrrrrr}
12 & 4 & -7 & 6 & 4 & 3 & -4 & 1 & 2 & 0 \\
-12 & 4 & -1 & 0 & 4 & -3 & 2 & 7 & -4 & 6 \\
-3 & 1 & 1 & -3 & 6 & 0 & -6 & -9 & -9 & 0 \\
-9 & 6 & -3 & 0 & 17 & -8 & -1 & -21 & -6 & -27 \\
0 & 1 & 0 & 0 & 0 & 4 & 0 & 0 & 16 & 0 \\
-12 & 0 & 0 & 0 & -31 & 0 & 0 & -56 & 0 & 48 \\
-162 & -189 & -108 & 324 & 36 & 27 & -36 & -6 & 3 & 0 \\
27 & 36 & 36 & 0 & 21 & 48 & 108 & -17 & -36 & 9
\end{array}\right)
\]
while
\[
\phi_1(p_5) = (21, 97, 329, -147, -28, -96, 28, 28, -4, 0)
\]
hence the determinant of the matrices $\mathcal{G}_1, \mathcal{G}_2$
and $\mathcal{G}_3$ are:
\begin{eqnarray*}
g_1 & = & 4760x^3 + 8209x^2y - 4331xy^2 - 652y^3 + 4858x^2z + 4752xyz - \\
& & 4858y^2z - 1512xz^2 - 720yz^2\\
g_2 & = & 4948x^3 - 4858x^2y + 2572xy^2 - 8345x^2z + 12544xyz + 652y^2z -\\
& & 1103xz^2 + 4858yz^2 + 720z^3\\
g_3 & = & 4948x^2y - 4858xy^2 + 2572y^3 + 4760x^2z - 136xyz + 8213y^2z + \\
& & 4858xz^2 + 3649yz^2 - 1512z^3
\end{eqnarray*}
and the common zeros of $g_1, g_2, g_3$ (i.e.\ the eigenpoints of the
corresponding cubic curve) are the points $p_1, \dots, p_5$
above and the two other points $p_6 = (1236: 1365: -1396)$ and
$p_7=(672: 1640: -1727)$.
\end{es}

\begin{rmk}
In case the matrix $\Phi(P_1, \dots, P_5)$ has rank $8$, the family of
cubics with $P_1, \dots, P_5$ as eigenpoints is one dimensional, so the
above construction can be modified, by substituting one of the rows of
$\mathcal{H}$ with a random row of elements of the field~$K$.
\end{rmk}

Here we want to consider the following problem:
given five generic points in a $V$-configuration and imposing one
fo the two conditions
$\delta_1(P_1, P_2, P_4) = 0$ or $\delta_2(P_1, \dots, P_5) = 0$, what
can we say about all the eigenpoints of the corresponding cubics?

The matrices $\mathcal{G}_i$ of~(\ref{G1G2G3}) depend on the matrix
$\mathcal{H}$ and on the choice of the row $\phi_1(P_5)$. In particular,
in~(\ref{G1G2G3}) we can take $\phi_2(P_5)$ or $\phi_3(P_5)$ in place
of $\phi_1(P_5)$, obtaining in this way the matrices
$\mathcal{G}_i^{(2)}$ and $\mathcal{G}_i^{(3)}$ ($i=1, 2, 3$) (in order
to keep a coherent notation, we also denote by $\mathcal{G}_i^{(1)}$
the matrices of~(\ref{G1G2G3})). Let $g_i^{(j)}$ ($i=1, 2, 3$, $j = 1, 2, 3$)
be the corresponding determinants. These polynomials depend also on
the choice of the rows of $\Phi(P_1, \dots, P_5)$ used to construct
$\mathcal{H}$, but we have altready observed in similar situations that
\Cref{lemma:minors} implies that the determinants $g_i^{(j)}$
differ only by factors given by different coordinates of the points.
Hence, in the general case, we can construct the following triplets of
polynomials:
\begin{equation}
\left(g_1^{(1)}, g_2^{(1)}, g_3^{(1)}\right),\quad
\left(g_1^{(2)}, g_2^{(2)}, g_3^{(2)}\right),\quad
\left(g_1^{(3)}, g_2^{(3)}, g_3^{(3)}\right)
\label{3tripletsPol}
\end{equation}
each of which has the syzygy $(z, -y, x)$.

The polynomials $g_i^{(j)}$ are degree $3$ polynomials in $x, y, z$ whose
coefficients are polynomials in $A_1, B_1, C_1, A_2, B_2, C_2, A_3, B_3, C_3,
u_1, u_2, v_1, v_2$, so are big polynomials which require
special care in handling.

<<<<<<< HEAD
To construct five points $P_1, \dots, P_5$ which are in a $V$-configuration
and such that $\delta_1(P_1, P_2, P_4)= 0$ is quite easy: $P_1$
and $P_2$ can be taken in an arbitrary way, $P_4$ has to be chosen in such
a way that it satisfies the linear
equation $\scl{P_4}{P_2}\scl{P_1}{P_1}-\scl{P_4}{P_1}\scl{P_1}{P_2} = 0$
and $P_3$ and $P_5$ have to be chosen on the lines $P_1+P_2$ and $P_1+P_4$
respectively. In particular, the corresponding variety of cubic curves
has dimension $7$ ($\mathbb{P}^2\times \mathbb{P}^2 \times
\mathbb{P}^1\times  \mathbb{P}^1$).
If we construct a random example
of five points as above, we get a smooth cubic curve whose $7$ eigenpoints
do not have other collinearities (in addition to those of a
$V$-configuration).

The case in which the $V$-configuration satisfies the condition
$\delta_2(P_1, \dots, P_5) = 0$ is different.

If we define:
\begin{equation}
  \begin{split}
    U_1 & =  \langle P_1, P_2\rangle \left(\langle P_1, P_1\rangle
  \langle P_4,P_5\rangle - \langle P_1, P_4\rangle \langle P_1, P_5\rangle
  \right)\\
  U_2 & =  \langle P_1, P_2\rangle^2\langle P_4, P_5\rangle
  -\langle P_1, P_4\rangle \langle P_1, P_5\rangle \langle P_2, P_2\rangle
  \label{sst2}
  \end{split}
\end{equation}
it holds: $\delta_2(P_1, \dots, P_5) = U_1u_1+U_2u_2$, hence, in order to
have a $V$-configuration satisfying $\delta_2(P_1, \dots, P_5) = 0$ we can
fix $P_1, P_2, P_4$ arbitrarily, $P_5$ on the line $P_1+P_4$ and in this
way $P_3$ is determined by $u_1 = U_2$ and $u_2 = -U_1$. Also in this case
the subvariety of $\mathbb{P}^9$ given by the cubics with 5
eigenpoints satisfying $\delta_2 = 0$ is of dimension $7$
($\mathbb{P}^2\times \mathbb{P}^2 \times \mathbb{P}^2\times \mathbb{P}^1$).

Also in this case, it is easy to construct examples and now it turns out that
the $7$ eigenpoints of the obtained cubics satisfy the further
condition that also $P_6$ and  $P_7$ are aligned with $P_1$. To explain
why, we proceed as follows.

Consider again the
triplets of polynomials of~(\ref{3tripletsPol}), in which we make the
substitution $u_1 = U_2$ and $u_2 = -U_1$ (where $U_1, U_2$ are
given by~\ref{sst2})). Here we
assume that both $U_1$ and $U_2$ are not zero (this case will be considered
later). If at least one of the three triplets obtained is not identically
zero, from it we have three degree three polynomials in $x, y, z$ whose
common zeros are eigenpoints.
After the substitution, the three triplets become:
%
\begin{equation}
\sigma(P_1, P_2) \cdot \Omega^{(i)} \cdot  \left(G_1, G_2, G_3\right),\quad
\hbox{for $i = 1, 2, 3$}
\label{G1G2G3}
\end{equation}
%
where $\sigma(P_1, P_2)$ is defined in~(\ref{sigma12}) and
%
\begin{eqnarray}
\label{3_Omega}
  \begin{split}
  \Omega^{(1)} & = & (s_{14} s_{25} + s_{12} s_{45}) C_{1}
- (s_{14}s_{15}+s_{11}s_{45})C_{2} +(s_{11} s_{25}- s_{12} s_{15})C_4 \\
  \Omega^{(2)} & = & (s_{14} s_{25} + s_{12} s_{45}) B_{1}
- (s_{14} s_{15} + s_{11} s_{45}) B_{2} +(s_{11} s_{25} - s_{12} s_{15}) B_{4} \\
  \Omega^{(3)} & = & (s_{14} s_{25} + s_{12} s_{45}) A_{1}
- (s_{14} s_{15} +s_{11} s_{45})A_{2} +(s_{11} s_{25}-s_{12} s_{15})A_{4}
\end{split}
\end{eqnarray}
%
(here, for the sake of shortness, $s_{ij}$ denotes $\scl{P_i}{P_j}$).

The polynomials $G_1, G_2, G_3$ are three polynomials of degree three
in $x, y, z$ (with coefficients in the variables
$A_1, B_1, C_1, A_2, B_2, C_2, A_4, B_4, C_4, v_1, v_2$) which are independent
of the triplet chosen and, clearly, $z\,G_3-y\,G_2+x\,G_1 = 0$. The common
zeros are the eigenpoints of the cubic curve defined by the matrix
$\Phi(P_1, \dots, P_5)$.

According to [conti Valentina@@], we have that the polynomial
$F := C_1G_1-B_1G_2+A_1G_3$ splits into three factors $r_1$, $r_2$, $r_3$,
linear in $x, y, z$, which
correspond to the line $P_1+P_2$, the line $P_1+P_4$ and the line $P_6+P_7$.
The factorization of $F$ gives that the factor $r_3$ is:
\[
r_3 = 2yv_1A_1B_1^4A_2^2B_2A_4^3-2xv_1B_1^5A_2^2B_2A_4^3+\cdots
-2 yv_1A_1B_1^4A_2^2B_2A_4^3
\]
and is composed by $1965$ monomials. The relevant fact is that when we
evaluate $r_3$ on the coordinates of $P_1$, we get zero. Hence we have:
\begin{prop}
If we have five generic points $P_1, \dots, P_5$ which are in a
$V$-configuration and satisfy the condition $\delta_2(P_1, \dots, P_5) = 0$,
then the unique cubic determined by the condition that $P_1, \dots, P_5$
are eigenpoints, has also the two other eigenpoints $P_6$ and $P_7$
aligned with $P_1$.
\end{prop}

Now we consider the case $\delta_2(P_1, P_2, P_3, P_4, P_5) = 0$.

For some specific values of the coordinates it can happen that either
$U_1=0$ or $U_2=0$ (in which case $P_1$ and $P_3$ or, respectively, $P_2$
and $P_3$ coincide) or $\sigma(P_1, P_2) = 0$ or $\Omega^{(i)} = 0$ for
$i = 1, 2, 3$ (in which case formula~(\ref{G1G2G3}) does not allow to
obtain the polynomials $G_i$). We remark, however, that, similar to the
relation $\delta_2(P1, \dots, P_5) = U_1u_1+U_2u_2$, we also have the
symmetric relation $\delta_2(P_1, \dots, P_5) = -V_1v_1-V_2v_2$, where
$V_1$ and $V_2$ are obtained by the expressions of~(\ref{sst2}), where
we exchange $P_2$ with $P_4$ and $P_3$ with $P_5$. Hence we can repeat
all the above computations obtaining, instead of~(\ref{G1G2G3}) the
three triplets:

\section{Particular cases}
file \verb+contiCasoDegenere2.sage+ and file
\verb+conf_sigma12_sigma14.sage+

\end{document}
