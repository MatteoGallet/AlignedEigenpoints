\documentclass[10pt, a4paper, reqno, captions=tableheading,bibliography=totoc]{scrartcl}

\usepackage[utf8]{inputenc}
\usepackage[T1]{fontenc}
\usepackage{amsthm}
\usepackage{amsmath}
\usepackage{amssymb}
\usepackage{lmodern}
\usepackage[english]{babel}
\usepackage{booktabs}
\usepackage{float}
\usepackage{enumitem}
\usepackage{tikz,pgf}
%\usepackage[utopia]{mathdesign}
\usepackage{palatino}
\usepackage{hyperref}
\hypersetup{
    colorlinks,
    linkcolor={red!50!black},
    citecolor={blue!50!black},
    urlcolor={blue!80!black}
}
\usepackage[nameinlink]{cleveref}

\theoremstyle{plain}
\newtheorem{lemma}{Lemma}[section]
\newtheorem{prop}[lemma]{Proposition}
\newtheorem{theorem}[lemma]{Theorem}
\newtheorem{cor}[lemma]{Corollary}
\newtheorem{conjecture}[lemma]{Conjecture}
\newtheorem{fact}[lemma]{Fact}
\newtheorem{assumption}[lemma]{Assumption}
\newtheorem*{reduction}{Reduction}
\theoremstyle{definition}
\newtheorem{definition}[lemma]{Definition}
\newtheorem{es}[lemma]{Example}
\newtheorem*{notation}{Notation}
\newtheorem{rmk}[lemma]{Remark}


\newcommand{\N}{\mathbb{N}}
\newcommand{\Z}{\mathbb{Z}}
\newcommand{\Q}{\mathbb{Q}}
\newcommand{\R}{\mathbb{R}}
\newcommand{\C}{\mathbb{C}}
\newcommand{\p}{\mathbb{P}}
\newcommand{\sP}{\mathcal{P}}
\newcommand{\sL}{\mathcal{L}}
\newcommand{\de}{\partial}
\newcommand{\codim}{\mathrm{codim}}

\newcommand{\oo}{\mathcal{O}}
\newcommand{\Bl}{\mathrm{Bl}}

\newcommand{\iso}{\mathcal{Q}}

\newcommand{\SO}{\operatorname{SO}}
\newcommand{\Eig}{\operatorname{Eig}}
\newcommand{\polq}{{\rm Pol}_Q}
\newcommand{\comment}[1]{}

\newcommand{\scl}[2]{\left\langle {#1}, {#2} \right\rangle}

\title{Azione di $\mathrm{SO}_3(\mathbb{C})$ su $\p^2(\C)$}
\author{}
\date{}

\linespread{1.2}
\setlength{\parindent}{0pt}
\setlength{\parskip}{.25em}

\begin{document}

\maketitle

\begin{definition}
 We define $\mathrm{SO}_3(\mathbb{C})$ to be the complexification of the group of special orthogonal real matrices, namely
 %
 \[
  \mathrm{SO}_3(\mathbb{C}) :=
  \bigl\{
   M \in \mathrm{GL}_3(\C) \, \mid \,
   M M^t = I_3 \  \text{and} \  \det(M) = 1
  \bigr\} \,.
 \]
 %
 The group $\mathrm{SO}_3(\mathbb{C})$ acts on $\C^3$ by matrix multiplication:
 %
 \[
  \begin{array}{ccc}
   \mathrm{SO}_3(\mathbb{C}) \times \C^3 & \rightarrow & \C^3 \\
   (M, v) & \mapsto & Mv
  \end{array}
 \]
 %
 Since all the elements of $\mathrm{SO}_3(\mathbb{C})$ are invertible, the latter action descends to an action on $\p^2(\C)$.
\end{definition}

\begin{prop}
 The action of $\mathrm{SO}_3(\mathbb{C})$ on $\p^2(\C)$ has two orbits:
 %
 \begin{align*}
  \mathcal{O}_1 &:=
  \bigl\{
   P \in \p^2(\C) \, | \,
   P = (a:b:c) \  \text{with} \  a^2 + b^2 + c^2 = 0
  \bigr\} \\
  \mathcal{O}_2 &:= \p^2(\C) \setminus \mathcal{O}_1
 \end{align*}
 %
 A representative for $\mathcal{O}_1$ is $(1:i:0)$ and a representative for $\mathcal{O}_2$ is $(1:0:0)$.
\end{prop}
\begin{proof}
 Suppose that $P \in \p^2(\C)$ and $P = (a:b:c)$ with $a^2 + b^2 + c^2 = 0$.
 We produce a matrix $M \in \mathrm{SO}_3(\C)$ such that $M \left(\begin{smallmatrix} 1 \\ i \\ 0 \end{smallmatrix}\right)$ and $\left(\begin{smallmatrix} a \\ b \\ c \end{smallmatrix}\right)$ are proportional.
 Up to relabeling the coordinates, we can suppose that $a \neq 0$.
 Hence, by rescaling the coordinates of $P$, we have $P = (1: b: c)$ with $b^2 + c^2 = 0$.
 One can check that the matrix
 %
 \[
  M :=
  \begin{pmatrix}
   -1 & 0 & 0 \\
   0 & ib & -ic \\
   0 & ic & ib
  \end{pmatrix}
 \]
 %
 satisfies the requirements.

 Now suppose that $P \in \p^2(\C)$ and $P = (a:b:c)$ with $a^2 + b^2 + c^2 \neq 0$.
 Up to rescaling, we can suppose that $a^2 + b^2 + c^2 = 1$.
 Again, we produce a matrix $M \in \mathrm{SO}_3(\C)$ such that $M \left(\begin{smallmatrix} 1 \\ 0 \\ 0 \end{smallmatrix}\right)$ and $\left(\begin{smallmatrix} a \\ b \\ c \end{smallmatrix}\right)$ are proportional.
 First of all, suppose that $b^2 + c^2 \neq 0$ and let $\omega$ be a root of the polynomial $t^2 - (b^2 + c^2)$ in $\C[t]$.
 Then, the matrix
 %
 \[
   M :=
   \begin{pmatrix}
     a & \omega & 0 \\
     b & -\frac{ab}{\omega} & \frac{c}{\omega} \\
     c & -\frac{ac}{\omega} & -\frac{b}{\omega}
   \end{pmatrix}
 \]
 %
 satisfies the requirements.
 With the same technique, if $a^2 + c^2 \neq 0$, we can produce a matrix $M \in \mathrm{SO}_3(\C)$ that maps $\left(\begin{smallmatrix} 0 \\ 1 \\ 0 \end{smallmatrix}\right)$ to $\left(\begin{smallmatrix} a \\ b \\ c \end{smallmatrix}\right)$; similarly, when $a^2 + b^2 \neq 0$, we can map $\left(\begin{smallmatrix} 0 \\ 0 \\ 1 \end{smallmatrix}\right)$ to $\left(\begin{smallmatrix} a \\ b \\ c \end{smallmatrix}\right)$.
 Since $\left(\begin{smallmatrix} 1 \\ 0 \\ 0 \end{smallmatrix}\right)$, $\left(\begin{smallmatrix} 0 \\ 1 \\ 0 \end{smallmatrix}\right)$, and $\left(\begin{smallmatrix} 0 \\ 0 \\ 1 \end{smallmatrix}\right)$ are all $\mathrm{SO}_3(\C)$-equivalent, the only case to consider is when
 %
 \[
  b^2 + c^2 = a^2 + c^2 = a^2 + b^2 = 0 \,,
 \]
 %
 which, however, can never occur.
\end{proof}



\end{document}
