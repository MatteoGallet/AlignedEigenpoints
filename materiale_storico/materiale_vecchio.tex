%%%%%%%%%%%%%%%%%%%%%%%%%%%%%%%%%%%%%%%%%%
%%%%%%%%%%%%%%%%%%%%
Then we have a surjective morphism
$$
\alpha : \mathcal{U} \to \mathcal{AL},
$$
assigning to each $[F] \in \mathcal{U}$ the unique triple of eigenpoints. Over the irreducible open subset
$$
\mathcal W := \mathcal {AL}
\setminus \{(P_1,P_2,P_3)\in\mathcal {AL}
\ | \ \sigma(P_1,P_2)=0, s_{11} s_{22} s_{33}=0\}
$$
of aligned triples not lying
on a tangent line to the isotropic conic~$\iso$, with tangency point one of the~$P_i$'s, the fibers of $\alpha$ are projective linear systems of dimension~$3$ by
Proposition \ref{manca il riferimento su ancillary  non e': condition_rank_aligned}. By the Fiber Dimension Theorem this implies that the open subset $\alpha ^{-1} (\mathcal W)$ is irreducible of dimension
$$
\dim \alpha ^{-1} (\mathcal W)
\subset \sU=8 ,
$$
so $\sL$ is a hypersurface. Next we set
$$
\Sigma:=\{(P_1,P_2,P_3)\in\mathcal {AL}
\ | \ \sigma(P_1,P_2)=0, s_{11} s_{22} s_{33}=0\}
$$
and we observe that $\Sigma \subset \overline {\mathcal {AL}}$. Being $\alpha$ a morphism we get
that
$$
\sU \subseteq \overline {\alpha^{-1} (\mathcal W)}.
$$

%that is, any $[F] \in \sU$ such that $\alpha (F)$
%satisfies $\sigma(P_1,P_2)=0$
%is a limit of $[F_t]\in \alpha^{-1} (\mathcal W)$.
%
%Without loss of generality we can assume that the tangent line $l$ to $\iso$ is given by $\iii x-y=0$ and $P_1=(1:\iii:0)$. We consider the pencil of lines $\lambda x + \mu y$ with base point $(0:0:1)$, and
%for any pair $P_2= (A_2: \iii A_2: C_2), P_3
%=(A_3: \iii A_3 :C_3)\in l$, $P_i \neq P_1$, we set
%$$
%P_1^{\lambda, \mu}=(-\mu : \lambda :0), \quad
%P_2^{\lambda, \mu}=(-\mu A_2: \lambda A_2:C_2), \quad
%P_1^{\lambda, \mu}=(-\mu A_3: \lambda A_3:C_3).
%$$
%By adding other three linear conditions, we can construct a one-dimensional family of cubics with general
%member in $\sV$ and the special one satisfying $\sigma (P_1,P_2)=0$



It follows that $\sU$ is irreducible, and therefore $\sL$ is irreducible too.
\end{proof}

\begin{prop}
The class of any ternary form $[F]$ with two or more aligned triples of eigenpoints satisfies
$
[F] \in \sL.
$

\end{prop}

\begin{proof}
Let $\sV \subset \p^9$ denote the locus of classes $[F]$ of cubic forms adimitting a $V$-configuration among its eigenpoints.

 For simplicity we shall only consider the general cubic form
 $[F] \in \sV \setminus \Delta_{3,3}$, that is the ones corresponding to a rank $9$ matrix $\Phi(P_1, \dots, P_5)$; the cases with more alignments can be treated similarly.

 Observe that all the irreducible components of $\sV \setminus \Delta_{3,3}$ are open and nonempty in $\sV$ by
 Remark \ref{rmk_delta_case3}.

If $F$ is general enough, we have
$\rk \,\Phi(P_1, \dots, P_5)=9$ and $\rk \,\Phi(P_1, \dots, P_4)=8$.
We can also assume that $\Eig{F}$ has no other collinearities.
The pencil of cubics with the assigned eigenpoints $P_1, \dots, P_4$
has a general member in $\sU$ since $[F] \not\in \Delta_{3,3}$ and $P_1,P_2,P_3$ are aligned. This proves the statement.
\end{proof}

\begin{prop}
    The locus $\sV$ of cubics that have at least two aligned triples of eigenpoints has codimension~$2$ in~$\p^9$.
\end{prop}
\begin{proof}
In the product $(\p^2)^{5}$, the locus $\sF$ of
unordered $5$-tuples of distinct points containing $4$ collinear points is closed; indeed, a $5$-uple $(P_1,\dots,P_5)$ is in $\sF$ if and only if the points do not impose independent conditions to conics, that is if and only if the $5 \times 6$ matrix
associated with the
conditions of passing through $(P_1,\dots,P_5)$ has rank less or equal to $4$.

We then consider the open subset $\mathcal {Y}:=(\p^2)^{5} \setminus \sF$.
The locus $\mathfrak{V}$ of $V$-configurations in $\sY$ has dimension $8$ and can be characterized as follows:
$$
\mathfrak{V}= \{(P_1,\dots,P_5) \in \sY \ | \ \langle P_1 \times P_2, P_3\rangle=0, \ \langle P_1 \times P_4, P_5\rangle=0\}.
$$
Finally, a $V$-configuration is contained in an eigenscheme of some polynomial if and only if $\delta_1 (P_i,P_j,P_k) \cdot \delta_2 (P_1,\dots, P_5)=0$.

By setting $\Delta_{ijk} :=\{\delta_1 (P_i, P_j,P_k)=0\}$
and $\Delta_2:=\{\delta_2=0\}$, and by considering in $\sV$ the locus
$\mathcal{V}'$ of classes $[F]$ with $\Eig{F}$ containing exactly one $V$-configuration,
we get a morphism
$$
\beta : \mathcal{V}' \to (\bigcup_{i,j,k}\Delta_{i,j,k} \cup \Delta_2) \cap \mathfrak{V} \cap \sY.
$$
By the results of \Cref{rank_8}, the locus with $\rk \, \Phi(P_1, \dots,P_5)\le 8$ is closed, so $\beta$ is generically injective and $\dim \mathcal{V}'=\dim \overline {\mathcal{V}'}=7$.
\end{proof}
%
%%%%%%%%%%%%%%
%VECCHIA PROPOSIZIONE 9.5
\begin{prop}
\textbf{(AGGIUNGERE IPOTESI SU RANGO NON 8)}
Let $P_1, \dotsc, P_7$ be seven points of $\p^2$ in configuration
$(C_5)$. Then they are eigenpoints if and only if
\begin{equation}
s_{26}(s_{45}s_{13}-s_{34}s_{15})+s_{46}(s_{25}s_{13}-s_{23}s_{15}) = 0.
\label{cndC5}
\end{equation}
In this case, the points~$P_3$ and~$P_7$ are then given by the following formulas:
%
\begin{eqnarray}
P_3 & = & (s_{12}(s_{26}s_{45}+s_{46}s_{25})-s_{26}s_{15}s_{24}-s_{46}s_{15}s_{22})P_1-
(s_{11}(s_{26}s_{45}+s_{46}s_{25})-s_{26}s_{15}s_{14}-s_{46}s_{15}s_{12})P_2
\label{p3formula}\\
P_5 & = & (s_{12}s_{13}s_{44}-s_{14}^2 s_{23})P_1
+s_{14}(s_{11} s_{23}-s_{12}s_{13}) P_4 \\
P_6 & = & (-s_{15}s_{24}s_{34}-s_{15}s_{23}s_{44} + s_{13}s_{25}s_{44} +s_{13}s_{24}s_{45})P_2 + (-s_{13}s_{24}s_{25}+2s_{15}s_{22}s_{34}-s_{13}s_{22}s_{45})P_4 ,\\
P_7 & = & (s_{16}(s_{26}s_{45}+s_{24}s_{56})-s_{26}s_{15}s_{46}-s_{24}s_{15}s_{66})P_1-
(s_{11}(s_{26}s_{45}+s_{24}s_{56})-s_{26}s_{15}s_{14}-s_{24}s_{15}s_{16})P_6 ,
\label{p7formula}
\end{eqnarray}
%
\end{prop}

%%%%%%%%%%%%%%%%%%%
%%%%%%%%%%%VECCHI CONTI PER AUTOSCHEMI 1 DIMENSIONALI
\begin{prop}
\label{p2}
Let $C = V(f) \subset \p^2$ be a cubic curve.
Assume that $\dim \Eig{f} = 1$.
%
\begin{enumerate}
  \item If the $1$-dimensional component of $\Eig{f}$ is a line~$L$,
  then the residual subscheme $Z := \mathrm{Res}_l \bigl( \Eig{f} \bigr)$ in~$\Eig{f}$ with respect to~$l$ has degree~$3$.
  \item If the $1$-dimensional component of $\Eig{f}$ is a conic~$q$,
  then the residual subscheme $Z := \mathrm{Res}_q \bigl( \Eig{f} \bigr)$ in~$\Eig{f}$ with respect to~$q$ has degree~$1$.
\end{enumerate}
%
\end{prop}
\begin{proof}
(1)
Let $g_1$, $g_2$, and $g_3$ be the order~$2$ minors determining the eigenscheme of~$f$.
Since $g_1$, $g_2$, and $g_3$ have a common linear component~$l$, by writing
\[
 g_i = l \, h_i, \quad i=1,2,3
\]
we have that $\mathcal{I}_{Z, \p^2} = \mathcal{I}_{\Eig{F}), \p^2} (-1)$
fits in the exact sequence
\[
 0 \to \mathcal{G} \to \oo_{\p^2} (-2) \oplus \oo_{\p^2} (-2) \oplus \oo_{\p^2} (-2) \to \mathcal{I}_{Z,\p^2} \to 0 \,,
\]
where $\mathcal{G}$ is a rank two reflexive sheaf by \cite[Proposition 1]{Hartshorne1980}.
But on a smooth surface reflexive implies locally free (see \cite[Example~1.1.6]{Huybrechts2010}),
so $\mathcal{G}$ is a rank two vector bundle.

Next we observe that the two independent
syzygies between the generators $g_i$ give rise to the syzygies:
\[
z\, h_1 - y\, h_2 + x\, h_3 = 0, \qquad \partial_z f\, h_1 - \partial_y f \, h_2 + \partial_x f \ h_3=0 \,,
\]
which occur in degrees $3$ and $4$. We claim that the two relations are independent, again,
otherwise the $g_i$'s would be identically zero. So we apply \cite[Proposition~12]{Ellia2020}, and we have that $\mathcal{G}$ splits. Observe that as $c_1(\mathcal{I}_{Z,\p^2})=0$, we have $c_1(\mathcal{G})=-6$. Moreover, there is no syzygy in degree~$2$, since
otherwise the $h_i$ would be linearly dependent, so the $h_i$ would belong to a pencil, which is a contradiction \textbf{(giustificare)}.

It follows that the splitting of $\mathcal{G}$ is of the form
%
\[
  \mathcal{G} \cong \oo_{\p^2} (-3) \oplus \oo_{\p^2} (-3) \,.
\]
%
We get the free resolution of the ideal sheaf:
%
\[
  0\to \oo_{\p^2} (-3)\oplus \oo_{\p^2} (-3) \to 3\oo_{\p^2} (-2)\to \mathcal{I}_{Z,\p^2} \to 0 \,.
\]
%
Moreover
%
\[
  c_2 (\mathcal{I}_{Z,\p^2} ) = c_2 \bigl( 3\oo_{\p^2}(-2) \bigr) - c_2 (\mathcal{G}) = 12 - 9 = 3 \,,
\]
%
and $h^0 \bigl( \mathcal{I}_{Z,\p^2}(1) \bigr) = 0$, which proves the first statement.

\medskip
(2) is similar, to be completed.
\end{proof}
