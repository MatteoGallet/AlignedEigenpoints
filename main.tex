\documentclass[10pt, a4paper, reqno, captions=tableheading,bibliography=totoc]{scrartcl}

\usepackage[utf8]{inputenc}
\usepackage[T1]{fontenc}
\usepackage{amsthm}
\usepackage{amsmath}
\usepackage{amssymb}
\usepackage{lmodern}
\usepackage[english]{babel}
\usepackage{booktabs}
\usepackage{float}
\usepackage{enumitem}
\usepackage{tikz,pgf}
%\usepackage[utopia]{mathdesign}
\usepackage{palatino}
\usepackage{hyperref}
\hypersetup{
    colorlinks,
    linkcolor={red!50!black},
    citecolor={blue!50!black},
    urlcolor={blue!80!black}
}
\usepackage[nameinlink]{cleveref}

\theoremstyle{plain}
\newtheorem{lemma}{Lemma}[section]
\newtheorem{prop}[lemma]{Proposition}
\newtheorem{theorem}[lemma]{Theorem}
\newtheorem{cor}[lemma]{Corollary}
\newtheorem{conjecture}[lemma]{Conjecture}
\newtheorem{fact}[lemma]{Fact}
\newtheorem{assumption}[lemma]{Assumption}
\newtheorem*{reduction}{Reduction}
\theoremstyle{definition}
\newtheorem{definition}[lemma]{Definition}
\newtheorem{es}[lemma]{Example}
\newtheorem*{notation}{Notation}
\newtheorem{rmk}[lemma]{Remark}


\newcommand{\N}{\mathbb{N}}
\newcommand{\Z}{\mathbb{Z}}
\newcommand{\Q}{\mathbb{Q}}
\newcommand{\R}{\mathbb{R}}
\newcommand{\C}{\mathbb{C}}
\newcommand{\p}{\mathbb{P}}
\newcommand{\sP}{\mathcal{P}}
\newcommand{\sL}{\mathcal{L}}
\newcommand{\de}{\partial}
\newcommand{\codim}{\mathrm{codim}}

\newcommand{\oo}{\mathcal{O}}
\newcommand{\Bl}{\mathrm{Bl}}

\newcommand{\iso}{\mathcal{Q}}

\newcommand{\SO}{\operatorname{SO}}
\newcommand{\Eig}{\operatorname{Eig}}
\newcommand{\polq}{{\rm Pol}_Q}
\newcommand{\comment}[1]{}

\newcommand{\scl}[2]{\left\langle {#1}, {#2} \right\rangle}

\title{Ternary symmetric tensors with an aligned triple of eigenpoints}
\author{}
\date{}

\linespread{1.2}
\setlength{\parindent}{0pt}
\setlength{\parskip}{.25em}

\begin{document}

\maketitle

\begin{abstract}
Abstract
\end{abstract}

\section{Introduction}

\section{Aligned eigenpoints of ternary cubic forms}

We recall that, given a homogeneous form $f \in \C[x,y,z]_d$ of degree~$d$, the eigenscheme~$E(f)$ of~$f$ is the determinantal scheme defined by the $2 \times 2$ minors of the matrix
%
\begin{equation}
\label{eq:def_matrix}
    \begin{pmatrix}
    x & y & z \\
    \de_x f & \de_y f & \de_z f
    \end{pmatrix}.
\end{equation}
%

The goal of this section is to understand what happens when we have one or more triples of aligned points inside a zero-dimensional reduced eigenscheme.
We start by recalling a property of eigenschemes of ternary forms, namely,
that when zero-dimensional they are somehow ``general'' with respect to conics.

\begin{lemma}
\label{lemma:no_six_conic}
Let $f \in \C[x,y,z]_d$ be a homogeneous form of degree~$d$.
If $E(f)$ is zero-dimensional,
then no degree six subscheme of~$E(f)$ lies on a conic.
\end{lemma}
\begin{proof}
See \cite[Lemma~9.1]{OS1} for the reduced case.
The proof works also in the non-reduced case.
\end{proof}

\begin{rmk}
\label{rmk:general_no_triple}
As a consequence of \Cref{lemma:no_six_conic}, a zero-dimensional eigenscheme~$E(f)$ for $f \in \C[x,y,z]_d$ never contains $4$ or more aligned points.
However, there are several examples of cubic polynomials, whose eigenscheme contains one or more triples of aligned points, as for instance the Fermat cubic polynomial.
\end{rmk}

It turns out, that the general $f \in \C[x,y,z]_d$ has an eigenscheme with no aligned triples. This is a consequence of the geometric properties of the classical Geiser map associated with seven points in the plane, and has been proved in \cite[Proposition 4.5]{BGV}.

\section{First properties of aligned eigenpoints of ternary cubics}

Imposing that a cubic ternary form has one or more aligned triples of eigenschemes imposes conditions both on the points and on the cubics.
We begin to explore these conditions by introducing, for each point $P \in \p^2$,
a $3 \times 10$ matrix encoding the condition that $P$ is an eigenpoint of a ternary cubic.

\begin{definition}
 Consider $\p^9 = \p(\C[x,y,z]_3)$, the space of all ternary cubics and take coordinates $\{ f_{ijk} \}_{i+j+k = 3}$ on it,
 so that $f_{ijk}$ is the coefficient of the monomial $x^i y^j z^k$.
 For a point $P \in \p^2$ with coordinates $(A: B: C)$, the condition on elements of~$\p^9$ that $P$ is an eigenpoint of a ternary cubic can be expressed in the form
 %
 \[
  \Phi(P) \cdot
  \left( \{ f_{ijk} \} \right)
  = 0 \,,
 \]
 %
 where $\Phi(P)$ is a $3 \times 10$ matrix with entries depending on $A, B, C$.
 The matrix $\Phi(P)$ is called the matrix of conditions imposed by~$P$.
 We denote by $\phi_1(P)$, $\phi_2(P)$, and~$\phi_3(P)$ the rows of~$\Phi(P)$.
\end{definition}

\begin{definition}
Let $P \in \p^2$ be a point and let $r$ be a line of the plane.
If $P$ is orthogonal to all the points of $r$ (i.e., if $\scl{P}{Q} = 0$ for
all $Q \in r$), we say that $P$ is \emph{orthogonal to~$r$}.
\end{definition}

\begin{prop}
\label{prop:orthogonal_tangent}
  Let $P \in \p^2$ and let $r \subset \p^2$ be a line through~$P$.
  The following conditions are equivalent:
  \begin{enumerate}
  \item $P$ is orthogonal to~$r$;
  \item $\scl{P}{P} = 0$ and there exists $Q \in r$, $Q \neq P$, such
    that $\scl{P}{Q} = 0$;
  \item $r$ is tangent in~$P$ to the isotropic conic.
  \end{enumerate}
\end{prop}
\begin{proof}
The proof follows from a symbolic computation. See the auxiliary file. (File \verb+contiPezzoMaggio23.sage+)
\end{proof}

\begin{definition}
\label{definition:condition_sigma}
 Let $P_1, P_2 \in \p^2$ be two distinct points.
 We denote
 %
 \[
  \sigma(P_1, P_2) := \scl{P_1}{P_1} \scl{P_2}{P_2} - \scl{P_1}{P_2}^2 \,.
 \]
 %
 In view of \Cref{prop:orthogonal_tangent}, the condition $\sigma(P_1, P_2) = 0$ only depends on the line~$r$ through~$P_1$ and~$P_2$.
 This is why we sometimes simply write $\sigma(r) = 0$.
\end{definition}

\begin{prop}
  Let $P_1$, $P_2$ be two distinct points in the plane and let $r$ be the line between them.
  Then the following are equivalent:
  \begin{enumerate}
  \item $\sigma(r) = 0$;
  \item there exists a point $T \in r$ such that $T$ is orthogonal to $r$.
  \end{enumerate}
\end{prop}
\begin{proof}
  If $\sigma(r) = 0$, then $r$ is tangent to the isotropic conic in a
  point~$T$, hence $\sigma(T, Q) = 0$ for all $Q \in r$.
  Since  $\sigma(T, Q) = \scl{T}{T}\scl{Q}{Q}-\scl{T}{Q}^2$, we get
  $\scl{T}{Q} = 0$, hence $r$ is orthogonal to $r$.
  The converse is trivial.
\end{proof}

\begin{prop}
  %\label{rk3pt}
  \label{prop:rank3points}
  Let $P_1$, $P_2$ and $P_3$ be three distinct aligned points in the plane.
  Let $r$ be the line passing through these points.
  Let $M=[\Phi(P_1), \Phi(P_2), \Phi(P_3)]$, namely,
  the $9 \times 10$ matrix obtained by stacking the matrices of conditions of~$P_1$, $P_2$, and~$P_3$.
  Then
  \begin{enumerate}
  \item $M$ has rank $\leq 6$;
  \item $M$ has rank $\leq 5$ if and only if one of the points $P_i$ is
    ortogonal to $r$;
  \item $M$ has rank $\leq 5$ if and only if $r$ is tangent to the
    isotropic conic (in one of the points $P_i$);
  \item $M$ has rank $\leq 5$ if and only if it has rank $5$.
  \end{enumerate}
\end{prop}
\begin{proof}
The proof follows from a symbolic computation. See the auxiliary file. (File \verb+contoTrePuntiAllineati.sage+)
\end{proof}


\begin{lemma}
  Let $P_1$, $P_2$ and $P_3$ be three distinct aligned points in the plane.
  Let $r$ be the line passing through these points.
  If $Q = w_1 P_1 + w_1 P_2$ is another point of $r$,
  then the rank of the $12 \times 10$ matrix
  \begin{equation}
  M = \left[
  \Phi(P_1), \Phi(P_2), \Phi(P_3), \Phi(Q)
  \right]
  \end{equation}
is at most $7$; it is less than $7$ if and only if $\sigma(P_1, P_2) = 0$.
\end{lemma}
\begin{proof}
  A direct computation shows that all the maximal minors of $M$ are
  zero, so $\mathrm{rank}\,(M) \leq 7$.
  We take the submatrix $M_1$ of $M$ whose rows are:
  \[
    \phi_1(P_1), \phi_2(P_1), \phi_1(P_2), \phi_2(P_2),
    \phi_1(P_3), \phi_2(P_3), \phi_1(Q)
  \]
  We consider the coordinates of the points as variables and we compute
  the ideal $J_1$, the radical of the ideal of the maximal minors of $M_1$;
  a direct computation shows that $J_1$ is a principal ideal generated by:
  %
  \begin{equation}
  \label{eq:genJ1}
    u_1 u_2 w_1 w_2 (u_1w_2-u_2w_1) A_1 A_2 (u_1A_1+u_2A_2) (A_1B_2-A_2B_1) \sigma(P_1,P_2) \,,
  \end{equation}
  %
  where $P_i = (A_i: B_i, C_i)$ for $i \in \{1,2\}$ and $P3 = u_1 P1 + u_2 P_2$.
  The factors $u_1$, $u_2$, $w_1$, $w_2$, and~$u_1w_2-u_2w_1$ appear due to the fact that if
  $P_3$ (or $Q$) coincides with $P_1$ (or, respectively, with $P_2$),
  the rank of $M_1$ is less than $7$
  and the same is true if $P_3$ and $Q$ coincide.
  As a consequence of
  \Cref{lemma:}, we know that if in the matrix~$M_1$, instead of the two rows
  $\phi_1(P_1)$ and $\phi_2(P_1)$ we take, for instance,
  $\phi_2(P_1), \phi_3(P_1)$, the radical of $J_1$ has the factor
  $C_1$ in place of the factor $A_1$.
  If in $M$, in place of the last row~$\phi_1(Q)$, we take $\phi_2(Q)$ (or $\phi_3(Q)$), the ideal
  $J_1$ has the factor $A_1C_2-A_2C_1$ (respectively, $B_1C_2-B_2C_1$) in
  place of $A_1B_2-A_2B_1$ (these results come from a direct computation
  of the radical of the ideal of the minors).
  Hence, as soon as we know the generator from \Cref{eq:genJ1}, we know all the
  possible generators of the radical of the ideals of all the $7\times 10$
  minors and they are all multiple of $\sigma(P_1, P_2)$. Moreover, if
  all the points are distinct (each given by a triple of not all zero
  coordinates), it is possible to select a minor of order $7$ which
  is zero if and only if $\sigma(P_1, P_2)$ is zero.
  So far, we have considered only the case in which the matrix
  $M_1$ is given by two vectors of $\Phi(P_1)$, two vectors of
  $\Phi(P_2)$, two vectors of $\Phi(P_3)$ and one vector of $\Phi(Q)$.
  To complete the proof, we have to consider the further case in which
  $M_1$ is given by two vectors of $\Phi(Q)$ and therefore only one
  vector taken from $\Phi(P_1)$ or from $\Phi(P_2)$ or from $\Phi(P_3)$.
  This case is similar to the previous ones, since, by
  \Cref{prop:orthogonal_tangent}, we can exchange $Q$ with one of the points
  $P_1$ or $P_2$ or $P_3$.
\end{proof}

\begin{lemma}
  Let $P_1$, $P_2$ and $P_3$ be three distinct aligned points.
  Let $r$ be the line passing through these points.
  If $Q = w_1P_1+w_2P_2$ is another point of $r$,
  then the rank of the $12\times 10$ matrix
\begin{equation}
M = \left[
\Phi(P_1), \Phi(P_2), \Phi(P_3), \Phi(Q)
\right]
\end{equation}
is $\leq 6$ if and only if is $6$.
\end{lemma}
\begin{proof}
  Suppose the rank of $M$ is less than $6$. Then, in particular,
  the rank of the matrix
  \[
    [\Phi(P_1), \Phi(P_2), \Phi(P_3)]
  \]
  is less than $6$ so, by \Cref{prop:rank3points}, has rank $5$ and the
  line $r$ is tangent to the isotropic conic in one of the points
  $P_i$, say in $P_1$. Then consider the matrix
  \[
    [\Phi(P_2), \Phi(P_3), \Phi(Q)]
  \]
  It also has rank $5$ and the line $r$ should be tangent to the isocronic
  conic in one of the points $P_2$ or $P_3$ or $Q$. This is a contradiction.
\end{proof}

Hence we can summarize the two lemmas above with:
\begin{prop}
  Let $P_1, P_2, P_3, Q$ be four collinear points and let
  \[
  M = [\Phi(P_1), \Phi(P_2), \Phi(P_3), \Phi(Q)]
  \]
  Then the following conditions
  are equivalent:
  \begin{enumerate}
  \item $M$ has rank $\leq 7$;
  \item $M$ has rank $\leq 6$ if and only if $r$ is tangent to the
    isotropic conic;
  \item $M$ has rank $\leq 6$ if and only if $\sigma(r) = 0$;
    \item $M$ has rank $\leq 6$ if and only if $M$ has rank $6$.
  \end{enumerate}
  \label{rango67}
\end{prop}

\section{The locus of cubic ternary forms with an aligned triple of eigenpoints}

\begin{definition}
 We set $\mathcal{L} \subseteq \p^9$ to be the closure of the locus of classes of cubic forms $f$ having a reduced zero-dimensional eigenscheme $E(f)$ with an aligned eigentriple.
\end{definition}

\begin{theorem}
The variety~$\mathcal{L}$ is an irreducible hypersurface.
\end{theorem}

\begin{proof}
By the Gallet - Logar construction, for a a general fixed aligned triple of points, the set of cubic forms having such a triple as eigenpoints is a linear system of dimension 3. Since the variety of aligned triples is $5$-dimensional, this proves that $\mathcal{L}$ is a hypersurface.

For the irreducibility, we maybe need the argument on the Geiser map. Is there a simpler one?
\end{proof}

\begin{lemma}
\label{lemma:pencil_one_aligned}
 If $f$ and $g$ are general cubics, then any cubic in the pencil $\lambda f + \mu g$ for $(\lambda: \mu) \in \p^1$ has at most one aligned triple of eigenpoints.
\end{lemma}

This is a corollary of the following result.

\begin{prop}
    The locus of cubics that have at least two aligned triples of eigenpoints has codimension~$2$ in~$\p^9$.
\end{prop}
\begin{proof}
    Conti di Sandro.
\end{proof}

\begin{definition}
 We define $\Delta \subset \mathcal{L}$ to be the closure of the locus of cubics with at least two aligned triples of eigenpoints.
\end{definition}

\begin{prop}
  The variety~$\Delta$ has dimension~$7$ and it is the union of two irreducible components~$\Delta_1$ and~$\Delta_2$.
\end{prop}

In what follows, we shall use the following notation.
Denote by $M_1$, $M_2$ and $M_3$ the three minors of \Cref{eq:def_matrix} relative to a cubic form $f$.
The net of cubics, which base locus is the eigenscheme $E(f)$, will be denoted by $\Lambda_f = \langle M_1, M_2, M_3 \rangle$.
\begin{lemma}
\label{lemma:scroll}
 If $f$ and $g$ are general cubics, then
 %
 \[
   \mathcal{N} := \bigcup_{(\lambda : \mu) \in \p^1} \Lambda_{\lambda f + \mu g} \subset \p^9
 \]
 %
 is an embedding of a rational projective bundle and has degree~$3$.
\end{lemma}
\begin{proof}
Consider the projective bundle given by the family of planes
%
\[
{\mathcal P} := \{ \Lambda_{\lambda f + \mu g} \, : \, (\lambda: \mu)\in \p^1 \} \subset \p^1 \times \p^9
\]
%
Then $\mathcal{N}$ is the projection of~$\mathcal{P}$ on the second factor.
However, the map ${\mathcal P} \to {\mathcal N}$ contracts no subvariety of any plane of ${\mathcal P}$, so either it is an embedding or it contracts some horizontal curve. In the latter case, all the planes of the family should intersect in at least one point. In particular, the two nets $\Lambda_f$ and $\Lambda_g$ should have non-empty intersection.
If we denote by $M_1$, $M_2$ and $M_3$ the $2 \times 2$ minors relative to~$f$, and by $N_1$, $N_2$ and $N_3$ the ones relative to~$g$, the vectorial dimension of the linear span $\left\langle M_1, M_2, M_3, N_1, N_2, N_3 \right\rangle$ should be strictly less than $6$. This can be avoided, since such a condition corresponds to a proper closed subscheme of $\p^9 \times \p^9$.

It follows that if $f$ and $g$ are general enough, then $\mathcal{N}$ is a $3$-dimensional rational normal scroll in $\p(\left\langle M_1, M_2, M_3, N_1, N_2, N_3 \right\rangle) \cong \p^5$.
Being a variety of minimal degree, its degree is $5+1-3 = 3$.
\end{proof}

\begin{theorem}
The degree of $\mathcal L$ is equal to
\[
  \deg \ \mathcal L =  15.
\]
\end{theorem}

\begin{proof}
We start by observing that a reduced $0$-dimensional eigenscheme contains an aligned triple if and only if the net of cubics $\Lambda_f = \langle M_1, M_2, M_3 \rangle$ contains a cubic which splits in three lines, a so called \emph{triangle}. Moreover, if $f$ is general enough, we have exactly one aligned triple and the other $4$ points are in general position; in this case, the net $\Lambda_f$ contains exactly three triangles, namely the unions of the line passing through the aligned triple and the reducible conics through the $4$ points in general position.

To determine the degree of $\mathcal L$ we consider a general pencil of cubic forms $\lambda f + \mu g$, and we will compute the number of elements with associated net $\Lambda_{\lambda f + \mu g}$ containing a triangle.

To this aim, denote by ${\mathcal T} \subset \p^9$ the variety of triangles; it is a classical result that its dimension is $6$ and its degree is $15$,
see for instance \cite[Section 2.2.2]{3264}. We now consider the variety~${\mathcal N}$ from \Cref{lemma:scroll}.
\comment{
given by the union of the nets of cubics of the pencil
$$
{\mathcal N} = \bigcup_{(\lambda : \mu) \in \p^1} \Lambda_{\lambda f + \mu g} \subset \p^9.
$$
Observe that we can assume that ${\mathcal N}$ is an embedding of a rational projective bundle; indeed, it can be seen as an immersion of the $\p^2$-bundle over $\p^1$ given by the family the planes ${\mathcal P}=
\{\Lambda_{\lambda f + \mu g}\ : \ (\lambda:\mu)\in \p^1\} \subset \p^1 \times \p^9$. The map ${\mathcal P} \to {\mathcal N}$ contracts no subvariety of any plane of ${\mathcal P}$, so it is either an embedding or it contracts some horizontal curve. In the latter case, all the planes of the family should intersect in at least one point. In particular, the two nets $\Lambda_f$ and $\Lambda_g$ should have non-empty intersection.
If we denote by $M_1$, $M_2$ and $M_3$ the $2 \times 2$ minors relative to $f$, and by $N_1$, $N_2$ and $N_3$ the ones relative to $g$, the vectorial dimension of the linear span
$\langle M_1,M_2,M_3,N_1,N_2,N_3
\rangle$ should be strictly less than $6$. This can be avoided, since such a condition corresponds to a proper closed subscheme of $\p^9 \times \p^9$.

It follows that if ${\mathcal N}$ is general enough, it is a $3$-dimensional rational normal scroll in $\p(\langle M_1,M_2,M_3,N_1,N_2,N_3
\rangle) \cong \p^5$, and being a variety of minimal degree, its degree is $\deg {\mathcal N}=5+1-3=3$.
}
Note that, since each net containing a triangle, actually contains exactly $3$ of them, the number of nets of ${\mathcal N}$ containing some triangle is given by
%
\[
\frac {{\mathcal T} \cdot {\mathcal N}}{3} =\frac{{15} \cdot {3}}{3}=15 \,.
\]
%
This hence implies that $\deg {\mathcal L} = 15$.
\end{proof}

\begin{es}
The following pencil of cubic forms admits exactly $15$ cubics with an aligned triple of eigenpoints ..
\end{es}


\section{Eigendiscriminants \textbf{togliere?}}
We recall that in \cite[Theorem~4.1, Corollary~4.2]{ASS} the authors prove that the Eigendiscriminant $\Delta_{n,d}\subset \p ((K^n)^{\otimes d})$ parametrizing
all tensors with zero-dimensional non-reduced eigenscheme or with positive dimensional eigenscheme is an irreducible hypersurface of degree $n(n-1)(d-1)^{n-1}$. For $n=3$ and $d=3$
this gives $24$.

In this section we shall consider the analogous question for symmetric tensors seem to be not answered. We shall denote by
\[
\Delta_{3,3, Sym}
\]
the restriction of the discriminant to the subspace of symmetric tensors.

\begin{prop}
Let $f$ be a homogeneous polynomial of degree $3$ in $x_0,x_1,x_2$. If $V(f)$ is a cuspidal curve, then $E(f)$ is not reduced.
\end{prop}

\begin{proof}
We know that $E(f)$ contains the Jacobian subscheme. As such a subscheme is $GL(3)$-invariant, we can assume that $f$ has the normal form $x_0 x_1^2-x_2^3$. Then it is simple to check that the Jacobian subscheme has length~$2$.
\end{proof}

We recall that the hypersurface $\Sigma _{3,3}$ of singular degree three polynomials has degree $12$ (Kapranov) and that the locus of cuspidal cubics is a closed subvariety of codimension~$1$ in~$\Sigma_{3,3}$ and degree~$24$.

\begin{cor}
The discriminant hypersurface~$\Delta_{3,3, Sym}$
intersects the hypersurface~$\Sigma_{3,3}$ along the subvariety of cuspidal cubics.
\end{cor}

\begin{rmk}
The general nodal cubic has reduced zero dimensional eigenscheme. Indeed, this is the case for the cubic $x_0 x_1^2 -x_2^2(x_0+x_1)$.

Note that a totally reducible polynomial
$f=l_1 \, l_2 \, l_3$, which is a product of three linearly independent linear forms, has a reduced Jacobian subscheme, consisting of three distinct simple points. We shall see in what follows that the general form of this type has a zero-dimensional reduced eigenscheme.

On the other hand, if $f$ is a product of three distinct but linearly dependent linear forms, its Jacobian subscheme has length~$4$ and is supported at a point.
\end{rmk}

\section{Further alignments}


\begin{theorem}
\label{theorem:possible_alignments}

\end{theorem}

\begin{table}
\label{table:all_alignments}
\caption{All possible configurations of $7$ points with alignments that can appear as eigenschemes of a ternary cubic form, provided that the eigenscheme is zero-dimensional and reduced.}
\centering
\begin{tabular}{|lll|}\hline
  num & collinear vertices & name\\ \hline
 1& [(1, 2, 3)] & "line" \\
 2& [(1, 2, 3), (1, 4, 5)] & "X shape"\\
 3& [(1, 2, 3), (1, 4, 5), (1, 6, 7)] & "star" \\
  & [(1, 2, 3), (1, 4, 5), (2, 4, 6)] & "triangle" \\
 4& [(1, 2, 3), (1, 4, 5), (1, 6, 7), (2, 4, 6)] & "triangle + altitude" \\
  & [(1, 2, 3), (1, 4, 5), (2, 4, 6), (3, 5, 6)] & "two X shapes" \\
 5& [(1, 2, 3), (1, 4, 5), (1, 6, 7),  & "two stars"\\
  & \phantom{[}(2, 4, 6), (2, 5, 7)] & \\
 6& [(1, 2, 3), (1, 4, 5), (1, 6, 7), & "triangle + three altitudes"\\
  & \phantom{[} (2, 4, 6), (2, 5, 7), (3, 4, 7)] & \\
 7& [(1, 2, 3),
   (1, 4, 5),
   (1, 6, 7),
   (2, 4, 6), & "Fano matroid" \\
  & \phantom{[} (2, 5, 7),
   (3, 4, 7),
   (3, 5, 6)] & \\ \hline
\end{tabular}
\end{table}

It is well-known that case (7) cannot be realized in zero characteristic (see \cite{Whitney1935}), therefore we will not consider it in our analysis.

\begin{description}
 \item[Case (1)]
 \item[Case (2)]
 We get two possible conditions:
 \begin{align}
  d_2 &: \langle P_1, P_1 \rangle\cdot \langle P_2, P_4\rangle -
  \langle P_1, P_2 \rangle \cdot \langle P_1, P_4 \rangle \label{rango9_1} \\
  d_3 &: \langle P_1,P_2 \rangle \cdot \langle P_1,P_3\rangle \cdot
  \langle P_4,P_5\rangle -\langle P_1,P_4\rangle \cdot \langle P_1,P_5\rangle
  \cdot \langle P_2,P_3\rangle  \label{rango9_2}
\end{align}
By bilinearity, one has that $d_2$ is equivalent to any of
\begin{equation}
\begin{aligned}
    & \langle P_1, P_1 \rangle\cdot \langle P_3, P_4\rangle -
  \langle P_1, P_3 \rangle \cdot \langle P_1, P_4 \rangle \,, \\
    & \langle P_1, P_1 \rangle\cdot \langle P_3, P_5\rangle -
  \langle P_1, P_3 \rangle \cdot \langle P_1, P_5 \rangle \,, \\
    & \langle P_1, P_1 \rangle\cdot \langle P_2, P_5\rangle -
  \langle P_1, P_2 \rangle \cdot \langle P_1, P_5 \rangle \,.
\end{aligned}
\end{equation}
From examples, we know that $d_3$ actually implies Case a star configuration as in Case (3).
 \item[Case (3)]

\end{description}


\bibliographystyle{amsalpha}
\bibliography{ooms}

\end{document}


\subsection{The locus of forms with positive-dimensional eigenscheme}
\begin{prop}
Assume that $\dim  E(f)=1$ and that the one-dimensional component of $E(f)$ is a line $L$. Then $L\subseteq {\rm Sing} f$.
\end{prop}

\begin{proof}
Assume by contradiction that
\[
L\subseteq {\overline{R(f)}}.
\]
We observe that if $L \subseteq V(f)$, then $L\subset Q$, where $Q$ is the isotropic quadric, by Proposition .... This is impossible since $Q$ is irreducible.

So consider the scheme-theoretic intersection
\[
Z:= L \cap Q,
\]
where $Q$ is the isotropic quadric. If $Z\subset R(f)$, then $Z\subset V(f)$ by Proposition \ref{pro: relation eigenpoints - tangent space to the isotroquadric}. Since $L \not \subset V(f)$, residually $L$ intersects $V(f)$ in a point. Since such a point is also an eigenpoint and it can't be regular, it is necessarily a singular point of $V(f)$. This gives a contradiction, since every line passing through a singular point has intersection multiplicity $\ge 2$ with $V(f)$.
Finally, assume that $Z=p_1 + p_2$ as a divisor on $Q$, with $p_1 \in R(f)$ and $p_2\in {\overline {R(f)}} \setminus R(f)$. The necessarily $p_2 \in {\rm Sing} V(f)$, so that residually to $Z$ the line $L$ intersects $V(f)$ in $p_2$. Therefore $V(f)$ has a singular point $p_2$ on $Q$ and $V(f)$ intersect tangentially $Q$ in $p_1$. The examples show that this never happens ...
\end{proof}

\begin{prop}\label{p2}
Let $C=V(f)\subset \p^2$ be a cubic curve. Assume that $\dim E(f)=1$ and that the $1$-dimensional component is a line $L$.

Then the residual subscheme $Z={\rm Res}_L(E(f))$ in $E(f)$ with respect to $V(L)$ is  zero dimensional of degree $3$ if $\dim \Lambda =2$, or of degree $4$, if $\dim \Lambda=1$.
\end{prop}

\begin{proof}
Since $G_1$, $G_2$ and $G_3$ have a common linear component $L$, by writing
\[
G_i=L \ H_i, \quad i=1,2,3
\]
we have that the residual subscheme $Z:={\rm Res}_L E(f)$ is a quasi complete intersection
determined by the polynomials $H_1$, $H_2$ and $H_3$; indeed, we recall that $\mathcal {I}_{Z,\p^2} = \mathcal {I}_{E(f),\p^2} (-1)$.

Let us assume $\dim\Lambda=2= \dim \p (\langle H_1,H_2,H_3\rangle)$.

So we have an exact sequence
\[
0\to \mathcal{G} \to\oo_{\p^2} (-2) \oplus \oo_{\p^2} (-2)  \oplus \oo_{\p^2} (-2)  \to \mathcal {I}_{Z,\p^2} \to 0,
\]
where $\mathcal{G}$ is a rank two reflexive sheaf by \cite[Proposition 1]{Hartshorne1980}. But on a smooth surface reflexive implies locally free (see \cite[Example~1.1.6]{Huybrechts2010}), so $\mathcal {G}$ is a rank two vector bundle.

Next we observe that the two independent
syzygies between the generators $G_i$ give rise to the syzygies:
\[
x_2\ H_1 - x_1\ H_2 + x_0\ H_3=0, \qquad \partial_2 f\ H_1 - \partial_1 f \ H_2 +\partial_0 f \ H_3=0,
\]
which occur in degrees $3$ and $4$. We claim that the two relations are independent, again,
otherwise the $G_i$'s would be identically zero. So we apply \cite[Proposition~12]{Ellia2020}, and we have that $\mathcal {G}$ splits. Observe that as $c_1(\mathcal {I}_{Z,\p^2})=0$, we have $c_1(\mathcal {G})=-6$. Moreover, there is no syzygy in degree $2$, since
otherwise the $H_i$ would be linearly dependent, so the $G_i$ would belong to a pencil, which we excluded.

It follows that the splitting of $\mathcal {G}$ is of the form
\[
\mathcal {G}\cong \oo_{\p^2} (-3) \oplus \oo_{\p^2} (-3).
\]
We get the free resolution of the ideal sheaf:
\[
0\to \oo_{\p^2} (-3)\oplus \oo_{\p^2} (-3) \to 3\oo_{\p^2} (-2)\to \mathcal {I}_{Z,\p^2} \to 0.
\]
Moreover
\[
c_2 (\mathcal {I}_{Z,\p^2} )=c_2(3\oo_{\p^2}(-2)) - c_2(\mathcal {G})=12-9=3,
\]
and $h^0(  \mathcal {I}_{Z,\p^2}(1))=0$.

Finally, assume $\dim \Lambda =1 =\dim \p (\langle H_1,H_2,H_3\rangle)$.
Then $Z={\rm Res}_L E(f)$ is the complete intersection of two conics.
\end{proof}


\begin{es}
Consider
\[
f(x_0,x_1,x_2)=x_0^2 (x_1 - x_2)
\]
(Example~\eqref{item: ternary cubic, no regular eigenpoints} with $t=i$).
We have $L=x_0$,
\[
H_1=x_0^2-2x_1^2+2x_1 x_2, \quad H_2= 2x_2^2-x_0^2-2x_1 x_2, \quad H_3= 2x_1x_2-2x_2^2-x_0x_1.
\]
The two syzygies in degree~$3$ are:
\[
x_2\ H_1-x_1\ H_2+x_0\ H_3=0, \quad x_0\ H_2 +2(x_1+x_2)\ H_3+ x_0 H_1=0.
\]
Finally, $Z= \{ (0:1:1),(2:1:-1),(-2:1:-1)\}$. Observe that one point is on the singular line $x_0=0$
and that the Jacobian scheme is non reduced, it consists of a line with an embedded point.
\end{es}

\begin{rmk}
Assume $\dim E(f) =1$ and that the $1$-dimensional component is an irreducible conic $\tilde Q$. Let $Q$ be the isotropic quadric. Then we have the following possible cases:
\begin{itemize}
\item $\tilde Q = Q$. It seems that this never happens.
\item set $Z:= \tilde Q \cap Q$. Then $Z\subset R(f)$ , so $V(f)$ is tangent to $Q$ in four points.
\item $Z$ consists of one singular point for $V(f)$ and one or two regular eigenpoints.
\item $Z$ consists of two singular points for $V(f)$.
\item $Z$ consists of a triple point for $V(f)$, with one common tangent with $Q$.
\end{itemize}
\end{rmk}

\section{Examples}
\begin{es}\label{example: non-determinantal cubic surface}
Cubica:
\[
xy^2 + z^3 + yt^2
\]
Contiene una sola retta, non \`e determinantale, ha un solo punto singolare
di coordinate $(1, 0, 0, 0)$ di molteplicit\`a 6.

Autopunti:
\[
\begin{array}{l}
  (-1/2, -1, 0, 1), \quad
  (-1/2, -1, -2/3, 1), \quad
  (1/2, 1, 0, 1), \\
  (1/2, 1, 2/3, 1),\quad
  (9/2, 9 \sqrt{2}/2, 3, 0), \quad
  (9/2, -9 \sqrt{2}/2, 3, 0),\\
  (0, 0, 1, 0),\quad
  (1, \sqrt{2}, 0, 0), \quad
  (1, -\sqrt{2}, 0, 0),\\
  (1, 0, 0, 0)*
  \end{array}
\]

Questi punti sono con allineamenti e complanarit\`a, secondo questo
schema:\\
terne allineate:
\[
016, 236, 467, 568
\]
quaterne complanari:\\
\[
0278, 0345, 1245
\]
cinquine complanari:\\
\[
01236, 01467, 01568, 23467, 23568, 45678
\]
\end{es}

\begin{es}
Cubica:
\[
xz^2 + xyt + t^3
\]
Caratteristiche: singolare nei punti $(0, 0, 0, 1), (1, 0, 0, 0)$.
Contiene esattamente due rette.

Ci sono 11 autopunti $P_0, P_1, \dots, P_{10}$ le cui coordinate sono,
rispettivamente:
\[
\begin{array}{llll}
(4, 1, \sqrt{30}, 1) &
 (4, 1, -\sqrt{30}, 1) &
 (1, 1, 0, \sqrt{-2}/2)&
 (1, 1, 0, -1\sqrt{-2}/2) \\
 (2, -2, 0, 1)&
 (-2, 2, 0, 1)&
 (0, 1, 0, 0)&
 (1, 0, \sqrt{2}, 0)\\
 (1, 0, -\sqrt{2}, 0)&
 (0, 0, 0, 1)&
 (1, 0, 0, 0)&
\end{array}
\]
Collinearit\`a\\
punti allineati:
\[
(2, 3, 9),  \ (4, 5, 9), \ (7, 8, 10)
\]
quaterne complanari (non contenute in cinquine complanari):
\[
\begin{array}{llll}
(0, 2, 3, 9) &
 (0, 4, 5, 9)&
 (1, 2, 3, 9)&
 (1, 4, 5, 9)\\
 (2, 3, 7, 9)&
 (2, 3, 8, 9)&
 (2, 7, 8, 10)&
 (3, 7, 8, 10)\\
 (4, 5, 7, 9)&
 (4, 5, 8, 9)&
 (4, 7, 8, 10)&
 (5, 7, 8, 10)\\
 (6, 7, 8, 10)&
 (7, 8, 9, 10)& &
\end{array}
\]
cinquine complanari (non contenute in sestuple complanari):
\[
\begin{array}{llll}
(0, 1, 7, 8, 10)&
 (2, 3, 4, 5, 6)&
 (2, 3, 4, 5, 9)&
 (2, 3, 4, 5, 10)\\
 (2, 3, 4, 6, 9)&
 (2, 3, 4, 6, 10)&
 (2, 3, 4, 9, 10)&
 (2, 3, 5, 6, 9)\\
 (2, 3, 5, 6, 10)&
 (2, 3, 5, 9, 10)&
 (2, 3, 6, 9, 10)&
 (2, 4, 5, 6, 9)\\
 (2, 4, 5, 6, 10)&
 (2, 4, 5, 9, 10)&
 (2, 4, 6, 9, 10)&
 (2, 5, 6, 9, 10)\\
 (3, 4, 5, 6, 9)&
 (3, 4, 5, 6, 10)&
 (3, 4, 5, 9, 10)&
 (3, 4, 6, 9, 10)\\
 (3, 5, 6, 9, 10)&
 (4, 5, 6, 9, 10)& &
\end{array}
\]
Non ci sono sestuple complanari\\
C'\`e una settupla di punti complanari che \`e:
\[
(2, 3, 4, 5, 6, 9, 10)
\]
\end{es}

\begin{es}
cubica:
\[
y^3 + xyt - xzt
\]
Caratteristiche. Cubica singolare nei punti: $(0, 0, 0, 1), (1, 0, 0, 0),
(0, 0, 1, 0)$. Contiene esattamente 3 rette.


Ci sono $12$ autopunti $P_0, \dots, P_{11}$ le cui coordinate sono,
rispettivamente:
\[
\begin{array}{ll}
(\sqrt{-\sqrt{-2} + 2} , 1 , \sqrt{-2} , \sqrt{-\sqrt{-2} + 2} ),&
(\sqrt{\sqrt{-2} + 2} , 1 , -\sqrt{-2} , \sqrt{\sqrt{-2} + 2} ),\\
(-\sqrt{-\sqrt{-2} + 2} , 1 , \sqrt{-2} , -\sqrt{-\sqrt{-2} + 2} ),&
(-\sqrt{\sqrt{-2} + 2} , 1 , -\sqrt{-2} , -\sqrt{\sqrt{-2} + 2} ),\\
(-\sqrt{6} , -1 , 2 , \sqrt{6} ),&
(\sqrt{6} , -1 , 2 , -\sqrt{6} ),\\
(-1 , \frac{1}{2} \, \sqrt{2} , \sqrt{2} , 1 ),&
(-1 , -\frac{1}{2} \, \sqrt{2} , -\sqrt{2} , 1 ),\\
(0 , 1 , 0 , 0 ),&
(0 , 0 , 1 , 0 ),\\
(1 , 0 , 0 , 0 ),&
(0 , 0 , 0 , 1)
\end{array}
\]
(gli ultimi 3 sono i punti singolari).

Non sembrano esserci terne allineate

Le quaterne complanari sono:
\[
\begin{array}{llll}
(0, 1, 2, 3) &
 (0, 1, 2, 8) &
 (0, 1, 2, 9) &
 (0, 1, 3, 8) \\
 (0, 1, 3, 9) &
 (0, 1, 8, 9) &
 (0, 2, 3, 8) &
 (0, 2, 3, 9) \\
 (0, 2, 8, 9) &
 (0, 2, 10, 11) &
 (0, 3, 8, 9) &
 (1, 2, 3, 8) \\
 (1, 2, 3, 9) &
 (1, 2, 8, 9) &
 (1, 3, 8, 9) &
 (1, 3, 10, 11) \\
 (2, 3, 8, 9) &
 (4, 5, 6, 7) &
 (4, 5, 6, 8) &
 (4, 5, 6, 9) \\
 (4, 5, 7, 8) &
 (4, 5, 7, 9) &
 (4, 5, 8, 9) &
 (4, 5, 10, 11) \\
 (4, 6, 7, 8) &
 (4, 6, 7, 9) &
 (4, 6, 8, 9) &
 (4, 7, 8, 9) \\
 (5, 6, 7, 8) &
 (5, 6, 7, 9) &
 (5, 6, 8, 9) &
 (5, 7, 8, 9) \\
 (6, 7, 8, 9) &
  (6, 7, 10, 11) & &
\end{array}
\]
Non ci sono cinquine complanari.

Ci sono le seguenti due sestuple complanari:

\[
(0, 1, 2, 3, 8, 9), (4, 5, 6, 7, 8, 9)
\]
\end{es}



Since we shall need some tools used in the proof, we briefly sketch the main ideas.

\begin{definition}
   Let $Z\subset\p^2$ be a set of seven points. If no six points of $Z$ are contained on a conic, then it is easy to see that $\dim I_Z(3)=3$ and
    the base locus of the linear system $\p (I_Z(3))$ is exactly~$Z$. In this case, the \emph{Geiser map} associated with $Z$ is the rational map $\gamma_Z:\p^2\dashrightarrow\p^2$ defined by the linear system $I_Z(3)$.
    %By blowing-up the plane $\p^2$ along $Z$ we get a generically finite morphism
   % \[
   % \widetilde \gamma_Z \colon \Bl_Z \p^2 \to \p^2.
   % \]
    We define $B(Z)$ to be the branch locus of $\gamma_Z$.
\end{definition}



Geiser maps are a classical topic and several of their properties are understood. As an example, $\gamma_Z$ is generically finite of degree~$2$.

%The ramification locus of~$\widetilde{\gamma _Z}$ is given by the \emph{Jacobian locus}~$\Sigma$ defined by the determinant of the Jacobian $\mathrm{Jac}(I_Z)$, that is the locus of singular points of the net (see
%\footnote{Non serve la referenza anche %al libro? Quando citiamo Coolidge nel %seguito, mettiamo anche "Book II".}
%\cite[Chapter~IX, Theorem~25]{Cool}). When $Z$ is general, the Jacobian locus~$\Sigma$ is a curve of degree~$6$ which is singular at $Z$ as illustrated in \cite[Chapter~IX, Theorem~27]{Cool}. We define $B(Z)$ to be the branch locus of $\widetilde{\gamma_Z}$, that is the direct image of~$\Sigma$. For modern references, see for instance \cite[Section~8.7.2]{Dolgachev} and \cite[Section~7]{OS1}, where it is proven that a general Geiser map is branched along a smooth L\"uroth quartic.


\begin{lemma}
\label{lemma:branched_smooth_iff_no_collinear}
Let $Z \subset \p^2$ be a set of seven distinct points such that
no six points lie on a conic, and let $B(Z)\subset\p^2$ be the associated branch locus. If $B(Z)$ is a smooth curve
of degree four, then $Z$ contains no three points on a line.
\end{lemma}

\begin{proof}
See \cite[Lemma 4.3]{BGV}
%{\bf copiata da BGV}
%Let $Y \subset Z$ be any subset of six of the points of~$Z$. Let $\pi \colon \Bl_{Y}\p^2\dashrightarrow \p^2$ be the projection from the seventh point.

%Assume that it is possible to fix a subset~$Y$ consisting of six points of~$Z$, so that they do not contain any collinear triple. Since they do not lie on a conic, the blow up~$\Bl_{Y}\p^2$ is
%isomorphic to a smooth cubic surface~$S$ in~$\p^3$. Let $\pi \colon S \dasharrow \p^2$ denote the projection of~$S$ from the image of the seventh point to~$\p^2$.

%It is known (see \cite[Section~3]{OS2}) that the ramification curve for the projection $\pi$ is a quartic L\"uroth curve, which is singular if and only if the seventh point lies on one of the $27$ lines of~$S$.
%Now recall that such lines correspond to the six exceptional divisors, the six conics passing through five points of~$Y$, and the $15$ lines joining two points
%of~$Y$. The exceptional divisors are excluded in our case, because the seven points are distinct, and the conics are excluded by hypothesis. Therefore $B(Z)$ is singular if and only if the seventh point is collinear with two other points.

%Assume now that for any choice of six points among the points of $Z$, there is always a collinear triple; then it is simple to check that we have at least three alignments. Then the blow-up of $\p^2$ in six points of $Z$ is isomorphic
%to a singular irreducible cubic surface~$S$ in~$\p^3$. As before, denote by $\pi \colon S \dasharrow \p^2$ the projection from the image $A$ of the seventh point. The ramification curve of $\pi$ is given by the points of tangency of the tangent lines to~$S$ passing through $A$. The latter are given by the intersection of~$S$ with the first polar~${\rm Pol}_A$ of~$S$ with respect to~$A$, hence $R_\pi$ has degree~$6$. Since $P_A$ intersects~$S$ tangentially in~$A$, the curve~$R_\pi$ is singular in~$A$. Moreover, since any first polar of a hypersurface contains all the singular points of the hypersurface, the ramification curve~$R_\pi$ has at least one singular point distinct from~$A$ by construction. It may happen that $S$ contains the line through~$A$ and some singular point. In any case,
%the image of~$R_\pi$ under the projection~$\pi$ is either a singular curve of degree~$4$, or a plane curve of degree~$3$ or less, or a finite set points, which contradicts the hypothesis on~$B(Z)$.
\end{proof}

\begin{prop}
\label{pro:plane_cubics_general_position}
If $f\in\C[x_0,x_1,x_2]_3$ is general, then $E(f)$ contains no 3 points on a line.
\end{prop}
\begin{proof}
By the proof of \cite[Theorem~10.4]{OS1}, we have that if $f$ is general, the branch locus of the associated Geiser map is a smooth L\"uroth quartic. Hence by Lemma \ref{lemma:branched_smooth_iff_no_collinear} the eigenscheme~$E(f)$
contains no collinear triples.
\end{proof}


%\begin{es}\label{liscia allineata}
%The eigenscheme of the smooth plane cubic
%\[f=x_0x_2^2+x_0^2x_2-2x_0x_1x_2+x_0^3+x_0^2x_1-x_0x_1^2-x_1^3\]
%has no collinear triples. In this case the Jacobian curve has equation
%{\scriptsize
%\begin{align*}
%& 12x_0^6-18x_0^5x_1-210x_0^4x_1^2+30x_0^3x_1^3+96x_0^2x_1^4-54x_0x_1^5+6x_1^6-36x_0^5x_2-6x_0^4x_1x_2+6x_0^3x_1^2x_2\\
 %   &+294x_0^2x_1^3x_2+78x_0x_1^4x_2-6x_1^5x_2+66x_0^3x_1x_2^2+234x_0^2x_1^2x_2^2+156x_0x_1^3x_2^2-150x_1^4x_2^2+18x_0^3x_2^3\\
 %   &+114x_0^2x_1x_2^3+270x_0x_1^2x_2^3-36x_1^3x_2^3+66x_0^2x_2^4-6x_0x_1x_2^4+60x_1^2x_2^4-6x_0x_2^5+12x_1x_2^5-6x_2^6.
%\end{align*}}
%The branch curve $B(f)$ is the smooth quartic given by the equation
%{\scriptsize
%\begin{align*}
%x_0^4-7x_0^3x_1+5x_0^2x_1^2+4x_0x_1^3+x_1^4-x_0^3x_2+9x_0x_1^2x_2+3x_1^3x_2+14x_0x_1x_2^2-6x_1^2x_2^2-7x_1x_2^3+2x_2^4.
%\end{align*}}

%Moreover, by making experiments with random forms, it is possible to check that if $f$ is a general triangle, then $E(f)$ consists of 7 reduced points, no three of which on a line and no six of which on a conic. The same holds when $f$ is the union of a general conic and a line.
%\end{es}

\begin{prop}\label{allineati_contrae} Let $E(f)$ be zero dimensional.
Then $E(f)$ contains an aligned triple of points on a line $L$
if and only if the Geiser map $\gamma_{E(f)}$ contracts the line $L$.
\end{prop}

\begin{proof}
Let us identify $\p^2 = \p (I_Z(3))^\vee$. For any $P \in \p^2 \setminus E(f)$, the point $\gamma_{E(f)} (P)$ corresponds to the pencil of cubics through $E(f) \cup \{P\}$. If $E(f)$ contains a triple of points on $L$, for any $P \in L \setminus E(f)$
such a pencil of cubics has $L$ as a fixed component and the other component varies in a pencil of conics through the remaining $4$ points, so $\gamma_{E(f)}$ is constant on $L$.

Conversely, if $\gamma_{E(f)}$ contracts a line $L$, for any $P\in L$, the pencils of cubics through $E(f) \cup \{P\}$ is constant. Hence the line $L$ is in the base locus of such pencil. This implies that either $L\subseteq E(f)$ or $E(f)$ has a subscheme of degree three on $L$. The first case can not occur by the assumption $\dim E(f)=0$. {\bf scrivere meglio}
\end{proof}


\begin{lemma}
Let $E(f)$ be zero-dimensional and reduced, and set
$$
g_0:=x_1 \partial _2 f - x_2 \partial _1 f, \quad
g_1:= x_2 \partial _0 f - x_0 \partial _2 f, \quad
g_2:=x_0 \partial _1 f - x_1 \partial _0 f.
$$
Then the Geiser map defined by
$$
\gamma_{E(f)} (P) = (g_0(P):g_1(P):g_2(P))
$$
is surjective, and it fibers over a point $Q=(Q_0:Q_1:Q_2)$ are given by
\begin{equation}\label{fibers}
    \gamma_{E(f)} ^{-1} (Q)= V( Q_0 x_0+ Q_1 x_0+ Q_2 x_2)  \cap   V(Q_0 \partial_0 f+
 Q_1 \partial_1 f+  Q_2 \partial_2 f),
\end{equation},
that is the intersection between the polar line $L_Q$ relative to the isotropic conic and ${\rm Pol}_Q f$, the
first polar curve of $f$ with respect to $Q$.

In particular, the only possible curves contracted by the Geiser map $\gamma _{E(f)}$ are lines.
\end{lemma}

\begin{proof}
We observe that for any point $P=(P_0:P_1:P_2)\in \p^2 \setminus Z$, the homogeneous coordinates
of the image $\gamma_Z(P)=(g_0(P):g_1(P):g_2(P))$ are the homogeneous coordinates of the unique point in the intersection of the two lines
$$
P_0 x_0 + P_1 x_1+ P_2 x_2  = 0, \qquad \partial_0f(P) x_0 +
\partial_1 f(P) x_1+ \partial_2 f(P) x_2 = 0.
$$
So for any $Q=(Q_0:Q_1:Q_2)\in \p^2$, the fiber $\gamma_Z^{-1}(Q)$ consists of the points $P\in \p^2$ such that
\begin{equation}\label{polars}
P_0 Q_0 + P_1 Q_1+ P_2 Q_2  =  \partial_0 f(P) Q_0 +
\partial_1 f(P) Q_1+ \partial_2 f(P) Q_2 = 0,
\end{equation}
which proves that $\gamma_Z$ is surjective and that \eqref{fibers} holds.

%that is the general fiber is the intersection of the polar line $L_Q$ relative to the isotropic conic and ${\rm Pol}_Q f$, the
%first polar curve $Q_0 \partial_0 f  +
%Q_1 \partial_1 f + Q_2 \partial_2 f=0$ of $f$ with respect to $Q$.

To prove that there are no other contracted curves than lines,
set $Z=E(f)$ and consider the blow-up of the plane $\p^2$ along $Z$. We get a generically finite morphism
   \[
    \widetilde \gamma_Z \colon \Bl_Z \p^2 \to \p^2.
    \]
Observe that the fibers of $\widetilde \gamma_Z$ are generically
contained in the divisor $W$ with bihomogenous equation $x_0 y_0 + x_1 y_1 + x_2 y_2=0$. Since both $\Bl_Z \p^2$ and $W$ are irreducible, and since $\widetilde\gamma_Z$ is the restriction of the second projection
$p_2 : \p^2 \times \p^2 \to \p^2$, it follows that $S\subseteq W$; in particular, every fiber of $\widetilde \gamma_Z$ is contained in a line, and by construction the same holds for every fiber of $\gamma_Z$. As a consequence, the map $\gamma_Z$ contracts only lines.
\end{proof}
Next we shall characterize the polynomials $f$ such that there exists a point $Q\in \p^2$ such that the fiber $\gamma_{E(f)}$ is a line.
By Proposition \ref{allineati_contrae}, such polynomials are exactly the ones with an aligned eigentriple.




\begin{prop}
The locus $\sL \subset \p(\C [x_0,x_1,x_2]_3)$ corresponding to cubic polynomials for which the Geiser map contracts at least a line satisfies
\[
{\rm codim} \sL \ge 1.
\]
\end{prop}

\begin{proof}


By the argument above, we see that $\gamma_Z^{-1}(Q)$ is a line if and only if either:
\begin{enumerate}
 %   \item \label{singolari} the
%triple $\partial_0 f (Q)=0,\partial_1 f(Q)=0,\partial_2 f(Q)=0$ is the zero triple, in which case $Q$ is a singular point of the cubic curve $V(f)$;
\item the
polynomials $\partial_0 f,\partial_1 f,\partial_2 f$ are linearly dependent and the coordinates $(Q_0:Q_1:Q_2)$ are precisely the coefficients of a dependancy relation;
this case occurs if and only if $V(f)$ consists of $3$ concurrent lines (see \cite[]{Cool}\footnote{Manca la citazione precisa. Qui forse possiamo usare "Polar Covariants of Plane Cubics and Quartics" di Dolgachev e Kanev, Proposition 3.9?}), or

\item \label{polare riducibile} the line $Q_0 x_0 +Q_1 x_1 + Q_2x_2=0$ is a component of the first polar curve
$\polq (f)=0$.
\end{enumerate}

%identically zero if %and only if the partials of $f$ are %linearly dependent. It is well-known that %this condition occurs if and only if  %$V(f)$ consists
%of $d$ concurrent lines (see, for %instance, \cite[IX, Theorem 30]{Cool}).


%To bound the number of contracted lines, we observe that the ramification divisor of $\lambda_Z$
%the net. Indeed, a point  $P\in \p^2\setminus Z$ is a ramification point if and only if the pencil of degree $d$ curves with base locus $Z \cup \{ P\}$ consists of curves intersecting tangentially in $P$, and there is always a singular curve in the pencil with $P$ as a singular point
%



 To analyse the second case, we
write $L_Q$ in parametric equations with parameters $(t_0:t_1)$. By substituting such expressions in $\polq (f)$, we get a
homogeneous polynomial of degree $2$ in $t_0, t_1$. Therefore the condition $L_Q \subset \polq (f)$ is satisfied if and only if the $3$ coefficients of such a polynomial are zero. These equations are bihomogeneous of bidegree $(1,3)$ in the coefficients of $f$ and the coordinates $Q_i$ of $Q$, so they determine $3$ hypersurfaces in $
\p(\C[x_0,x_1,x_2]_3) \times \p^2$. Let us set
$\widetilde{\mathcal L} \subseteq
\p(\C[x_0,x_1,x_2]_3)\times \p^2$ to be their zero locus. We claim that the $3$ equations are all independent, so that
\begin{equation}\label{codim}
\codim\  \widetilde {\mathcal{L}} = 3.
\end{equation}

Indeed, consider the restriction of the first projection $p_1 \colon \widetilde{{\mathcal L}} \to \p^2$. Such a map is surjective, and all its fibers have codimension $\le 3$ and the codimension is
upper semicontinuous. Hence to determine
$\dim \widetilde {\mathcal P}$ is suffices to
exhibit a specific fiber having codimension~$3$.

Assume $Q=(0:0:1)$. Then
\[
L_Q=(t_0:t_1:0)\mbox{ and }\polq (f)= \partial _2 f,
\]
so the condition $L_Q \subset \polq (f)$ becomes
$
\partial_2 f (t_0,t_1,0)\equiv 0.
$
The last condition is equivalent to
requiring that all the coefficients
of the monomials
$
x_0 ^2 x_2,
x_0 x_1 x_2,  x_1 ^{2}x_2
$
are zero, and the latter are $3$ linearly independent conditions. Hence $\codim \ p_1^{-1}(0:0:1)=3$ and (\ref{codim}) follows, so that
$$
\dim \widetilde{\mathcal L} =8.
$$

Moreover, since $p_1$ is surjective and the fibers of $p_1$ are all linear subspaces, they are all irreducible, so $\widetilde{\mathcal L}$ is irreducible too.

Therefore, if $p_2:
\widetilde{\mathcal L}
\to \p(\C[x_0,x_1,x_2]_3)$ denotes the second projection, by construction we have
$$
\sL=p_2(\widetilde{\mathcal L}).
$$
it follows that
\[
\dim \ \sL \le 8
\]
and $\sL$ being image of an irreducible variety, it is irreducible too.
\end{proof}
 %%%%%%%%%%%%%%%%%%%%%%%%%%%%%%%%%%

 Let $\Lambda =\langle f_0, f_1 \rangle \subset \p^9$ be a general pencil of cubic curves, intersecting $\mathcal L$ properly and transversely, and let $h =s_0 f_0 + s_1 f_1 \in \Lambda$ be its general element.
For a general $Q =(Q_0 :Q_1 :Q_2)\in \p^2$,
consider the polar conic of $h$:
\begin{equation}\label{polare fascio}
{\rm Pol}_Q(h)= Q_0 \partial_0 h + Q_1 \partial_1 h + Q_2 \partial_2 h = s_0 {\rm Pol}_Q(f_0) + s_1 {\rm Pol}_Q(f_1).
\end{equation}

We want to determine the number of elements of the pencil such that there exists a point $Q$ such that
the polar line $L_Q$ to the isotropic quadric is as component of the polar conic ${\rm Pol}_Q(h)$.

By generality we may assume that such
points $Q=(q_0:q_1:q_2)$ satisfy
$$
q_0 \neq 0.
$$
So we may write the general polar line $L_Q$ in parametric equations with parameters $(t_0:t_1)$:
$$
L_Q: \quad (x_0(t_0,t_1), x_1 (t_0, t_1), x_2(t_0,t_1)),
$$
with
$$
\begin{array}{ll}
x_0(t_0,t_1)=& -q_1 t_0 - q_2 t_1\\
x_1(t_0,t_1)=& q_0 t_0\\
x_2(t_0,t_1)=& q_o t_1\\
\end{array}
$$
By substituting such expressions in \eqref{polare fascio},
we get a
homogeneous polynomial
of degree $1$ in $s_0,s_1$, degree $2$ in $t_0, t_1$ and degree $3$ in the coordinates $Q_i$ of $Q$. The condition
\begin{equation}\label{polari}
L_Q \subset {\rm Pol}_Q(h).
\end{equation}
is satisfied if and only if
 the three coefficients of $t_0^2$, $t_0 t_1$ and $t_1^2$ are zero. Such a condition gives three polynomial equations $M_k(q_i, s_j)=0$, $k=1,2,3$, in the $Q_i$ and $s_j$, of bidegree $(1,3)$. The values of $(s_0:s_1)$ for which the three equations have a common root correspond to the cubics of the pencil admitting a point $Q$ such that
\eqref{polari} holds.

%To solve such a problem, we can consider the generalized resultant of $M_k(q_i,s_j)$ in the sense of I.M. Gelfand M.M. Kapranov A.V. Zelevinsky Discriminants, Resultants and Multidimensional Determinants, Chapter 13, Proposition 1.1, page 427 and we find a polynomial of degree $27$ in $s_0,s_1$. Hence

Now observe that the three equations $M_k(q_i, s_j)=0$ determine three divisors
$D_k \subseteq \p^1 \times \p^2$. To determine their class in the Chow ring $A^1(\p^1 \times \p^2)$,
set $p_1 : \p^1 \times \p^2$ and $p_2: \p^1 \times \p^2$ to be the two projections.
If $P\in \p^1$ denotes a point and $L \subset \p^2$ a line, two generators of $A^1(\p^1 \times \p^2)$ are given by
$$
h_1 := p_1^\star P \cong \p^2, \qquad h_2:= p_2^\star L \cong \p^1 \times \p^1,
$$
with the relations:
$$
h_1^2 =0, \quad h_2^2 =1, \quad h_2^3=0, \quad h_1 \cdot h_2^2 =1.
$$
By construction we have
$$
D_k \sim h_1 + 3h_2.
$$
Next we oberve that the intersection $D_1 \cap D_2 \cap D_3$ is not proper.

Indeed, they all contain a curve of class $(h_1+h_2) \cdot h_2 $.

This can be seen by writing explicitly the equations of $D_k$:
if $f_0$ has equation:
$$
a_{000}x^3+a_{001}x^2 y+a_{002} x^2 z+a_{011} x y^2+a_{022} x z^2+a_{012} x y z+a_{111} y^3+a_{112} y^2 z+a_{122} y z^2+a_{222} z^3=0
$$
and $f_1$ is given by
$$
b_{000} x^3+b_{001} x^2 y+b_{002} x^2 z+b_{011} x y^2+b_{022} x z^2+b_{012} x y z+b_{111} y^3+b_{112} y^2 z+b_{122} y z^2+b_{222} z^3=0,
$$
then $D_1$ is given by:
$$
(a_{011} s_{0}+b_{011} s_{1}) q_{0}^3+(-2 a_{001} s_{0}+3 a_{111} s_{0}-2 b_{001} s_{1}+3 b_{111} s_{1}) q_0^2 q_1+
(a_{112} s_0+b_{112} s_1) q_0^2 q_2+
$$
$$
\quad +(3 a_{000} s_0-2 a_{011} s_0+
3 b_{000} s_1-2 b_{011} s_1) q_0 q_1^2+(-a_{012} s_0-b_{012} s_1) q_0 q_1 q_2+(a_{001} s_0+b_{001} s_1) q_1^3+
$$
$$
+(a_{002} s_0+b_{002} s_1) q_1^2 q_2=0,
$$
$D_2$ is equal to
$$
(a_{012} s_0+b_{012} s_1) q0^3+(-2 a_{002} s_0+2 a_{112} s_0-2 b_{002} s_1+2 b_{112} s_1) q_0^2 q_1+
$$
$$
\quad +(-2 a_{001} s_0+
2 a_{122} s_0-2 b_{001} s_1+2 b_{122} s_1) q_2 q_0^2+
(-a_{012} s_0-b_{012} s_1) q_0 q_1^2+
$$
$$
+((6 a_{000} -2 a_{011} -2 a_{022})s_0 +(6 b_{000}-2 b_{011}-2 b_{022})s_1) q_2 q_0 q_1+
$$
$$
\qquad \qquad +(-a_{012} s_0-b_{012} s_1) q_0 q_2^2+
(2 a_{001} s_0+2 b_{001} s1) q_2 q_1^2+(2 a_{002} s_0+2 b_{002} s_1) q_1 q_2^2=0
$$
and $D_3$ has equation
$$
(a_{022} s_0+b_{022} s_1) q_0^3+(a_{122} s_0+b_{122} s_1) q_1 q_0^2+(-2 a_{002} s_0+3 a_{222} s_0-2 b_{002} s_1+
3 b_{222} s_1) q_2 q_0^2+
$$
$$
+(-a_{012} s_0-b_{012} s_1) q_0 q_1 q_2+(3 a_{000} s_0-2 a_{022} s_0+3 b_{000} s_1-2 b_{022} s_1) q_0 q_2^2+
$$
$$
+(a_{001} s_0+b_{001} s_1) q_2^2 q_1+(a_{002} s_0+b_{002} s_1) q_2^3=0.
$$
Recall now that we are looking for the base locus on the open subscheme $q_0 \neq 0$. The restriction of $D_1$, $D_2$ and $D_3$ along the divisor $q_0=0$, whose class in $h_2$, is given by the equations:
$$
q_1^2((a_{001} s_0+b_{001} s_1) q_1+(a_{002} s_0+b_{002} s_1) q_2)=0,
$$
$$
2q_2 q_1((a_{001} s_0+b_{001} s_1) q_1+(a_{002} s_0+b_{002} s_1)q_2)=0,
$$
$$
q_2^2(a_{001} s_0+b_{001} s_1) q_1+(a_{002} s_0+b_{002} s_1) q_2)=0
$$



This concludes the proof.

\end{proof}
%\begin{prop}
%\label{pro: plane cubics, general position}
%If $f\in\C[x_0,x_1,x_2]_3$ is general, then $E(f)$ contains no 3 points on a line and no 6 points on a conic.
%\end{prop}
%\begin{proof}
%We already observed that $E(f)$ does not contain 6 points on a conic by \cite[Lemma 9.1]{OS1}. Having a collinear triple is a closed condition in the space of cubics. In order to conclude, we only need to exhibit an example with no collinear triples. Consider
%\[f=x_0x_2^2+x_0^2x_2-2x_0x_1x_2+x_0^3+x_0^2x_1-x_0x_1^2-x_1^3.\]
%Assume by contradiction that there are three points of $E(f)$ contained in a line $l\subset\p^2$. For every $x\in l$, we get
%\[
%\gamma_{E(f)}(x) =
%\bigl\{ g\in I_{E(f)}(3)\mid g(x)=0 \bigr\} =
%\bigl\{ g\in I_{E(f)}(3)\mid g_{|l}=0 \bigr\},
%\]
%so $l$ is contracted by~$\gamma_{E(f)}$ and therefore $\gamma_{E(f)}(l)$ is a singular point of~$B\bigl(E(f)\bigr)$. On the other hand, an explicit software computation\footnote{Inserire i calcoli} shows that the branch curve $B\bigl(E(f)\bigr)$ is nonsingular.
%\end{proof}

%By making experiments with random forms, it is possible to check that if $f$ is a general triangle, then $E(f)$ consists of 7 reduced points, no three of which on a line and no six of which on a conic. The same holds when $f$ is the union of a general conic and a line.

%\begin{rmk}\label{polar}
%Let $G=x_1 \partial _0 f-x_0 \partial_1 f$, $M=x_0\partial_2 f - x_2 \partial_0 f$, $N=x_2 \partial_1 f -x_1 \partial_2 f$,
%so that the Geiser map $\gamma _{E(f)} \colon \p^2 \dasharrow \p^2$
%is given by $\gamma_{E(f)}(P)=
%(G(P):M(P):N(P))$.

%Observe that for any point $P=(p_0:p_1:p_2)\in \p^2 \setminus Z$, the homogeneous coordinates
%of the image $(G(P):M(P):N(P))$ are also the homogeneous coordinates of the unique point in the intersection of the two polar lines
%\[
%p_0 x_0 + p_1 x_1+ p_2 x_2 =0, \qquad \partial_0 f (P) x_0 +
%\partial_1 f(P) x_1+ \partial_2 f(P) x_2=0.
%\]
%It follows that for any $Q=(q_0:q_1:q_2)\in \p^2$, the fiber of $\lambda_Z$ over $Q\in \p^2$ consists of the points $P\in \p^2$ such that
%\[
%p_0 q_0 + p_1 q_1+ p_2 q_2 =0,
%\qquad \partial_0 f (P) q_0 +
%\partial_1 f(P) q_1+ \partial_2 f(P) q_2=0,
%\]
%hence every fiber is the intersection of the polar line $L_Q$ and the polar curve $f_Q$ with respect to $Q$, where the polar line is relative to the isotropic conic.

%Finally, observe that the polar line $L_Q$ is defined for any $Q\in \p^2$
%and that the polynomial defining the polar curve $f_Q$ is identically zero if and only if the partials of $f$ are linearly dependent. It is well-known that this condition occurs if and only if  $V(f)$ consists
%of $d$ concurrent lines, not necessarily all distinct, and $Q$ is the common point.

%In any case we see that the rational map $\lambda_Z$ is always surjective.
%\end{rmk}


%\[
%f(x_0,x_1,x_2)=x_0 x_2^ 2 -x_0  x_2  \,(a \, x_1+b \, x_0)-c \, x_0^3 - d \, x_0^2 x_1-e \, x_0 x_1^2 - f \, x_1^3.
%\]








