\documentclass[11pt]{article}

    \usepackage[breakable]{tcolorbox}
    \usepackage{parskip} % Stop auto-indenting (to mimic markdown behaviour)
    

    % Basic figure setup, for now with no caption control since it's done
    % automatically by Pandoc (which extracts ![](path) syntax from Markdown).
    \usepackage{graphicx}
    % Maintain compatibility with old templates. Remove in nbconvert 6.0
    \let\Oldincludegraphics\includegraphics
    % Ensure that by default, figures have no caption (until we provide a
    % proper Figure object with a Caption API and a way to capture that
    % in the conversion process - todo).
    \usepackage{caption}
    \DeclareCaptionFormat{nocaption}{}
    \captionsetup{format=nocaption,aboveskip=0pt,belowskip=0pt}

    \usepackage{float}
    \floatplacement{figure}{H} % forces figures to be placed at the correct location
    \usepackage{xcolor} % Allow colors to be defined
    \usepackage{enumerate} % Needed for markdown enumerations to work
    \usepackage{geometry} % Used to adjust the document margins
    \usepackage{amsmath} % Equations
    \usepackage{amssymb} % Equations
    \usepackage{textcomp} % defines textquotesingle
    % Hack from http://tex.stackexchange.com/a/47451/13684:
    \AtBeginDocument{%
        \def\PYZsq{\textquotesingle}% Upright quotes in Pygmentized code
    }
    \usepackage{upquote} % Upright quotes for verbatim code
    \usepackage{eurosym} % defines \euro

    \usepackage{iftex}
    \ifPDFTeX
        \usepackage[T1]{fontenc}
        \IfFileExists{alphabeta.sty}{
              \usepackage{alphabeta}
          }{
              \usepackage[mathletters]{ucs}
              \usepackage[utf8x]{inputenc}
          }
    \else
        \usepackage{fontspec}
        \usepackage{unicode-math}
    \fi

    \usepackage{fancyvrb} % verbatim replacement that allows latex
    \usepackage{grffile} % extends the file name processing of package graphics
                         % to support a larger range
    \makeatletter % fix for old versions of grffile with XeLaTeX
    \@ifpackagelater{grffile}{2019/11/01}
    {
      % Do nothing on new versions
    }
    {
      \def\Gread@@xetex#1{%
        \IfFileExists{"\Gin@base".bb}%
        {\Gread@eps{\Gin@base.bb}}%
        {\Gread@@xetex@aux#1}%
      }
    }
    \makeatother
    \usepackage[Export]{adjustbox} % Used to constrain images to a maximum size
    \adjustboxset{max size={0.9\linewidth}{0.9\paperheight}}

    % The hyperref package gives us a pdf with properly built
    % internal navigation ('pdf bookmarks' for the table of contents,
    % internal cross-reference links, web links for URLs, etc.)
    \usepackage{hyperref}
    % The default LaTeX title has an obnoxious amount of whitespace. By default,
    % titling removes some of it. It also provides customization options.
    \usepackage{titling}
    \usepackage{longtable} % longtable support required by pandoc >1.10
    \usepackage{booktabs}  % table support for pandoc > 1.12.2
    \usepackage{array}     % table support for pandoc >= 2.11.3
    \usepackage{calc}      % table minipage width calculation for pandoc >= 2.11.1
    \usepackage[inline]{enumitem} % IRkernel/repr support (it uses the enumerate* environment)
    \usepackage[normalem]{ulem} % ulem is needed to support strikethroughs (\sout)
                                % normalem makes italics be italics, not underlines
    \usepackage{soul}      % strikethrough (\st) support for pandoc >= 3.0.0
    \usepackage{mathrsfs}
    

    
    % Colors for the hyperref package
    \definecolor{urlcolor}{rgb}{0,.145,.698}
    \definecolor{linkcolor}{rgb}{.71,0.21,0.01}
    \definecolor{citecolor}{rgb}{.12,.54,.11}

    % ANSI colors
    \definecolor{ansi-black}{HTML}{3E424D}
    \definecolor{ansi-black-intense}{HTML}{282C36}
    \definecolor{ansi-red}{HTML}{E75C58}
    \definecolor{ansi-red-intense}{HTML}{B22B31}
    \definecolor{ansi-green}{HTML}{00A250}
    \definecolor{ansi-green-intense}{HTML}{007427}
    \definecolor{ansi-yellow}{HTML}{DDB62B}
    \definecolor{ansi-yellow-intense}{HTML}{B27D12}
    \definecolor{ansi-blue}{HTML}{208FFB}
    \definecolor{ansi-blue-intense}{HTML}{0065CA}
    \definecolor{ansi-magenta}{HTML}{D160C4}
    \definecolor{ansi-magenta-intense}{HTML}{A03196}
    \definecolor{ansi-cyan}{HTML}{60C6C8}
    \definecolor{ansi-cyan-intense}{HTML}{258F8F}
    \definecolor{ansi-white}{HTML}{C5C1B4}
    \definecolor{ansi-white-intense}{HTML}{A1A6B2}
    \definecolor{ansi-default-inverse-fg}{HTML}{FFFFFF}
    \definecolor{ansi-default-inverse-bg}{HTML}{000000}

    % common color for the border for error outputs.
    \definecolor{outerrorbackground}{HTML}{FFDFDF}

    % commands and environments needed by pandoc snippets
    % extracted from the output of `pandoc -s`
    \providecommand{\tightlist}{%
      \setlength{\itemsep}{0pt}\setlength{\parskip}{0pt}}
    \DefineVerbatimEnvironment{Highlighting}{Verbatim}{commandchars=\\\{\}}
    % Add ',fontsize=\small' for more characters per line
    \newenvironment{Shaded}{}{}
    \newcommand{\KeywordTok}[1]{\textcolor[rgb]{0.00,0.44,0.13}{\textbf{{#1}}}}
    \newcommand{\DataTypeTok}[1]{\textcolor[rgb]{0.56,0.13,0.00}{{#1}}}
    \newcommand{\DecValTok}[1]{\textcolor[rgb]{0.25,0.63,0.44}{{#1}}}
    \newcommand{\BaseNTok}[1]{\textcolor[rgb]{0.25,0.63,0.44}{{#1}}}
    \newcommand{\FloatTok}[1]{\textcolor[rgb]{0.25,0.63,0.44}{{#1}}}
    \newcommand{\CharTok}[1]{\textcolor[rgb]{0.25,0.44,0.63}{{#1}}}
    \newcommand{\StringTok}[1]{\textcolor[rgb]{0.25,0.44,0.63}{{#1}}}
    \newcommand{\CommentTok}[1]{\textcolor[rgb]{0.38,0.63,0.69}{\textit{{#1}}}}
    \newcommand{\OtherTok}[1]{\textcolor[rgb]{0.00,0.44,0.13}{{#1}}}
    \newcommand{\AlertTok}[1]{\textcolor[rgb]{1.00,0.00,0.00}{\textbf{{#1}}}}
    \newcommand{\FunctionTok}[1]{\textcolor[rgb]{0.02,0.16,0.49}{{#1}}}
    \newcommand{\RegionMarkerTok}[1]{{#1}}
    \newcommand{\ErrorTok}[1]{\textcolor[rgb]{1.00,0.00,0.00}{\textbf{{#1}}}}
    \newcommand{\NormalTok}[1]{{#1}}

    % Additional commands for more recent versions of Pandoc
    \newcommand{\ConstantTok}[1]{\textcolor[rgb]{0.53,0.00,0.00}{{#1}}}
    \newcommand{\SpecialCharTok}[1]{\textcolor[rgb]{0.25,0.44,0.63}{{#1}}}
    \newcommand{\VerbatimStringTok}[1]{\textcolor[rgb]{0.25,0.44,0.63}{{#1}}}
    \newcommand{\SpecialStringTok}[1]{\textcolor[rgb]{0.73,0.40,0.53}{{#1}}}
    \newcommand{\ImportTok}[1]{{#1}}
    \newcommand{\DocumentationTok}[1]{\textcolor[rgb]{0.73,0.13,0.13}{\textit{{#1}}}}
    \newcommand{\AnnotationTok}[1]{\textcolor[rgb]{0.38,0.63,0.69}{\textbf{\textit{{#1}}}}}
    \newcommand{\CommentVarTok}[1]{\textcolor[rgb]{0.38,0.63,0.69}{\textbf{\textit{{#1}}}}}
    \newcommand{\VariableTok}[1]{\textcolor[rgb]{0.10,0.09,0.49}{{#1}}}
    \newcommand{\ControlFlowTok}[1]{\textcolor[rgb]{0.00,0.44,0.13}{\textbf{{#1}}}}
    \newcommand{\OperatorTok}[1]{\textcolor[rgb]{0.40,0.40,0.40}{{#1}}}
    \newcommand{\BuiltInTok}[1]{{#1}}
    \newcommand{\ExtensionTok}[1]{{#1}}
    \newcommand{\PreprocessorTok}[1]{\textcolor[rgb]{0.74,0.48,0.00}{{#1}}}
    \newcommand{\AttributeTok}[1]{\textcolor[rgb]{0.49,0.56,0.16}{{#1}}}
    \newcommand{\InformationTok}[1]{\textcolor[rgb]{0.38,0.63,0.69}{\textbf{\textit{{#1}}}}}
    \newcommand{\WarningTok}[1]{\textcolor[rgb]{0.38,0.63,0.69}{\textbf{\textit{{#1}}}}}


    % Define a nice break command that doesn't care if a line doesn't already
    % exist.
    \def\br{\hspace*{\fill} \\* }
    % Math Jax compatibility definitions
    \def\gt{>}
    \def\lt{<}
    \let\Oldtex\TeX
    \let\Oldlatex\LaTeX
    \renewcommand{\TeX}{\textrm{\Oldtex}}
    \renewcommand{\LaTeX}{\textrm{\Oldlatex}}
    % Document parameters
    % Document title
    \title{NB.06.F5}
    
    
    
    
    
    
    
% Pygments definitions
\makeatletter
\def\PY@reset{\let\PY@it=\relax \let\PY@bf=\relax%
    \let\PY@ul=\relax \let\PY@tc=\relax%
    \let\PY@bc=\relax \let\PY@ff=\relax}
\def\PY@tok#1{\csname PY@tok@#1\endcsname}
\def\PY@toks#1+{\ifx\relax#1\empty\else%
    \PY@tok{#1}\expandafter\PY@toks\fi}
\def\PY@do#1{\PY@bc{\PY@tc{\PY@ul{%
    \PY@it{\PY@bf{\PY@ff{#1}}}}}}}
\def\PY#1#2{\PY@reset\PY@toks#1+\relax+\PY@do{#2}}

\@namedef{PY@tok@w}{\def\PY@tc##1{\textcolor[rgb]{0.73,0.73,0.73}{##1}}}
\@namedef{PY@tok@c}{\let\PY@it=\textit\def\PY@tc##1{\textcolor[rgb]{0.24,0.48,0.48}{##1}}}
\@namedef{PY@tok@cp}{\def\PY@tc##1{\textcolor[rgb]{0.61,0.40,0.00}{##1}}}
\@namedef{PY@tok@k}{\let\PY@bf=\textbf\def\PY@tc##1{\textcolor[rgb]{0.00,0.50,0.00}{##1}}}
\@namedef{PY@tok@kp}{\def\PY@tc##1{\textcolor[rgb]{0.00,0.50,0.00}{##1}}}
\@namedef{PY@tok@kt}{\def\PY@tc##1{\textcolor[rgb]{0.69,0.00,0.25}{##1}}}
\@namedef{PY@tok@o}{\def\PY@tc##1{\textcolor[rgb]{0.40,0.40,0.40}{##1}}}
\@namedef{PY@tok@ow}{\let\PY@bf=\textbf\def\PY@tc##1{\textcolor[rgb]{0.67,0.13,1.00}{##1}}}
\@namedef{PY@tok@nb}{\def\PY@tc##1{\textcolor[rgb]{0.00,0.50,0.00}{##1}}}
\@namedef{PY@tok@nf}{\def\PY@tc##1{\textcolor[rgb]{0.00,0.00,1.00}{##1}}}
\@namedef{PY@tok@nc}{\let\PY@bf=\textbf\def\PY@tc##1{\textcolor[rgb]{0.00,0.00,1.00}{##1}}}
\@namedef{PY@tok@nn}{\let\PY@bf=\textbf\def\PY@tc##1{\textcolor[rgb]{0.00,0.00,1.00}{##1}}}
\@namedef{PY@tok@ne}{\let\PY@bf=\textbf\def\PY@tc##1{\textcolor[rgb]{0.80,0.25,0.22}{##1}}}
\@namedef{PY@tok@nv}{\def\PY@tc##1{\textcolor[rgb]{0.10,0.09,0.49}{##1}}}
\@namedef{PY@tok@no}{\def\PY@tc##1{\textcolor[rgb]{0.53,0.00,0.00}{##1}}}
\@namedef{PY@tok@nl}{\def\PY@tc##1{\textcolor[rgb]{0.46,0.46,0.00}{##1}}}
\@namedef{PY@tok@ni}{\let\PY@bf=\textbf\def\PY@tc##1{\textcolor[rgb]{0.44,0.44,0.44}{##1}}}
\@namedef{PY@tok@na}{\def\PY@tc##1{\textcolor[rgb]{0.41,0.47,0.13}{##1}}}
\@namedef{PY@tok@nt}{\let\PY@bf=\textbf\def\PY@tc##1{\textcolor[rgb]{0.00,0.50,0.00}{##1}}}
\@namedef{PY@tok@nd}{\def\PY@tc##1{\textcolor[rgb]{0.67,0.13,1.00}{##1}}}
\@namedef{PY@tok@s}{\def\PY@tc##1{\textcolor[rgb]{0.73,0.13,0.13}{##1}}}
\@namedef{PY@tok@sd}{\let\PY@it=\textit\def\PY@tc##1{\textcolor[rgb]{0.73,0.13,0.13}{##1}}}
\@namedef{PY@tok@si}{\let\PY@bf=\textbf\def\PY@tc##1{\textcolor[rgb]{0.64,0.35,0.47}{##1}}}
\@namedef{PY@tok@se}{\let\PY@bf=\textbf\def\PY@tc##1{\textcolor[rgb]{0.67,0.36,0.12}{##1}}}
\@namedef{PY@tok@sr}{\def\PY@tc##1{\textcolor[rgb]{0.64,0.35,0.47}{##1}}}
\@namedef{PY@tok@ss}{\def\PY@tc##1{\textcolor[rgb]{0.10,0.09,0.49}{##1}}}
\@namedef{PY@tok@sx}{\def\PY@tc##1{\textcolor[rgb]{0.00,0.50,0.00}{##1}}}
\@namedef{PY@tok@m}{\def\PY@tc##1{\textcolor[rgb]{0.40,0.40,0.40}{##1}}}
\@namedef{PY@tok@gh}{\let\PY@bf=\textbf\def\PY@tc##1{\textcolor[rgb]{0.00,0.00,0.50}{##1}}}
\@namedef{PY@tok@gu}{\let\PY@bf=\textbf\def\PY@tc##1{\textcolor[rgb]{0.50,0.00,0.50}{##1}}}
\@namedef{PY@tok@gd}{\def\PY@tc##1{\textcolor[rgb]{0.63,0.00,0.00}{##1}}}
\@namedef{PY@tok@gi}{\def\PY@tc##1{\textcolor[rgb]{0.00,0.52,0.00}{##1}}}
\@namedef{PY@tok@gr}{\def\PY@tc##1{\textcolor[rgb]{0.89,0.00,0.00}{##1}}}
\@namedef{PY@tok@ge}{\let\PY@it=\textit}
\@namedef{PY@tok@gs}{\let\PY@bf=\textbf}
\@namedef{PY@tok@ges}{\let\PY@bf=\textbf\let\PY@it=\textit}
\@namedef{PY@tok@gp}{\let\PY@bf=\textbf\def\PY@tc##1{\textcolor[rgb]{0.00,0.00,0.50}{##1}}}
\@namedef{PY@tok@go}{\def\PY@tc##1{\textcolor[rgb]{0.44,0.44,0.44}{##1}}}
\@namedef{PY@tok@gt}{\def\PY@tc##1{\textcolor[rgb]{0.00,0.27,0.87}{##1}}}
\@namedef{PY@tok@err}{\def\PY@bc##1{{\setlength{\fboxsep}{\string -\fboxrule}\fcolorbox[rgb]{1.00,0.00,0.00}{1,1,1}{\strut ##1}}}}
\@namedef{PY@tok@kc}{\let\PY@bf=\textbf\def\PY@tc##1{\textcolor[rgb]{0.00,0.50,0.00}{##1}}}
\@namedef{PY@tok@kd}{\let\PY@bf=\textbf\def\PY@tc##1{\textcolor[rgb]{0.00,0.50,0.00}{##1}}}
\@namedef{PY@tok@kn}{\let\PY@bf=\textbf\def\PY@tc##1{\textcolor[rgb]{0.00,0.50,0.00}{##1}}}
\@namedef{PY@tok@kr}{\let\PY@bf=\textbf\def\PY@tc##1{\textcolor[rgb]{0.00,0.50,0.00}{##1}}}
\@namedef{PY@tok@bp}{\def\PY@tc##1{\textcolor[rgb]{0.00,0.50,0.00}{##1}}}
\@namedef{PY@tok@fm}{\def\PY@tc##1{\textcolor[rgb]{0.00,0.00,1.00}{##1}}}
\@namedef{PY@tok@vc}{\def\PY@tc##1{\textcolor[rgb]{0.10,0.09,0.49}{##1}}}
\@namedef{PY@tok@vg}{\def\PY@tc##1{\textcolor[rgb]{0.10,0.09,0.49}{##1}}}
\@namedef{PY@tok@vi}{\def\PY@tc##1{\textcolor[rgb]{0.10,0.09,0.49}{##1}}}
\@namedef{PY@tok@vm}{\def\PY@tc##1{\textcolor[rgb]{0.10,0.09,0.49}{##1}}}
\@namedef{PY@tok@sa}{\def\PY@tc##1{\textcolor[rgb]{0.73,0.13,0.13}{##1}}}
\@namedef{PY@tok@sb}{\def\PY@tc##1{\textcolor[rgb]{0.73,0.13,0.13}{##1}}}
\@namedef{PY@tok@sc}{\def\PY@tc##1{\textcolor[rgb]{0.73,0.13,0.13}{##1}}}
\@namedef{PY@tok@dl}{\def\PY@tc##1{\textcolor[rgb]{0.73,0.13,0.13}{##1}}}
\@namedef{PY@tok@s2}{\def\PY@tc##1{\textcolor[rgb]{0.73,0.13,0.13}{##1}}}
\@namedef{PY@tok@sh}{\def\PY@tc##1{\textcolor[rgb]{0.73,0.13,0.13}{##1}}}
\@namedef{PY@tok@s1}{\def\PY@tc##1{\textcolor[rgb]{0.73,0.13,0.13}{##1}}}
\@namedef{PY@tok@mb}{\def\PY@tc##1{\textcolor[rgb]{0.40,0.40,0.40}{##1}}}
\@namedef{PY@tok@mf}{\def\PY@tc##1{\textcolor[rgb]{0.40,0.40,0.40}{##1}}}
\@namedef{PY@tok@mh}{\def\PY@tc##1{\textcolor[rgb]{0.40,0.40,0.40}{##1}}}
\@namedef{PY@tok@mi}{\def\PY@tc##1{\textcolor[rgb]{0.40,0.40,0.40}{##1}}}
\@namedef{PY@tok@il}{\def\PY@tc##1{\textcolor[rgb]{0.40,0.40,0.40}{##1}}}
\@namedef{PY@tok@mo}{\def\PY@tc##1{\textcolor[rgb]{0.40,0.40,0.40}{##1}}}
\@namedef{PY@tok@ch}{\let\PY@it=\textit\def\PY@tc##1{\textcolor[rgb]{0.24,0.48,0.48}{##1}}}
\@namedef{PY@tok@cm}{\let\PY@it=\textit\def\PY@tc##1{\textcolor[rgb]{0.24,0.48,0.48}{##1}}}
\@namedef{PY@tok@cpf}{\let\PY@it=\textit\def\PY@tc##1{\textcolor[rgb]{0.24,0.48,0.48}{##1}}}
\@namedef{PY@tok@c1}{\let\PY@it=\textit\def\PY@tc##1{\textcolor[rgb]{0.24,0.48,0.48}{##1}}}
\@namedef{PY@tok@cs}{\let\PY@it=\textit\def\PY@tc##1{\textcolor[rgb]{0.24,0.48,0.48}{##1}}}

\def\PYZbs{\char`\\}
\def\PYZus{\char`\_}
\def\PYZob{\char`\{}
\def\PYZcb{\char`\}}
\def\PYZca{\char`\^}
\def\PYZam{\char`\&}
\def\PYZlt{\char`\<}
\def\PYZgt{\char`\>}
\def\PYZsh{\char`\#}
\def\PYZpc{\char`\%}
\def\PYZdl{\char`\$}
\def\PYZhy{\char`\-}
\def\PYZsq{\char`\'}
\def\PYZdq{\char`\"}
\def\PYZti{\char`\~}
% for compatibility with earlier versions
\def\PYZat{@}
\def\PYZlb{[}
\def\PYZrb{]}
\makeatother


    % For linebreaks inside Verbatim environment from package fancyvrb.
    \makeatletter
        \newbox\Wrappedcontinuationbox
        \newbox\Wrappedvisiblespacebox
        \newcommand*\Wrappedvisiblespace {\textcolor{red}{\textvisiblespace}}
        \newcommand*\Wrappedcontinuationsymbol {\textcolor{red}{\llap{\tiny$\m@th\hookrightarrow$}}}
        \newcommand*\Wrappedcontinuationindent {3ex }
        \newcommand*\Wrappedafterbreak {\kern\Wrappedcontinuationindent\copy\Wrappedcontinuationbox}
        % Take advantage of the already applied Pygments mark-up to insert
        % potential linebreaks for TeX processing.
        %        {, <, #, %, $, ' and ": go to next line.
        %        _, }, ^, &, >, - and ~: stay at end of broken line.
        % Use of \textquotesingle for straight quote.
        \newcommand*\Wrappedbreaksatspecials {%
            \def\PYGZus{\discretionary{\char`\_}{\Wrappedafterbreak}{\char`\_}}%
            \def\PYGZob{\discretionary{}{\Wrappedafterbreak\char`\{}{\char`\{}}%
            \def\PYGZcb{\discretionary{\char`\}}{\Wrappedafterbreak}{\char`\}}}%
            \def\PYGZca{\discretionary{\char`\^}{\Wrappedafterbreak}{\char`\^}}%
            \def\PYGZam{\discretionary{\char`\&}{\Wrappedafterbreak}{\char`\&}}%
            \def\PYGZlt{\discretionary{}{\Wrappedafterbreak\char`\<}{\char`\<}}%
            \def\PYGZgt{\discretionary{\char`\>}{\Wrappedafterbreak}{\char`\>}}%
            \def\PYGZsh{\discretionary{}{\Wrappedafterbreak\char`\#}{\char`\#}}%
            \def\PYGZpc{\discretionary{}{\Wrappedafterbreak\char`\%}{\char`\%}}%
            \def\PYGZdl{\discretionary{}{\Wrappedafterbreak\char`\$}{\char`\$}}%
            \def\PYGZhy{\discretionary{\char`\-}{\Wrappedafterbreak}{\char`\-}}%
            \def\PYGZsq{\discretionary{}{\Wrappedafterbreak\textquotesingle}{\textquotesingle}}%
            \def\PYGZdq{\discretionary{}{\Wrappedafterbreak\char`\"}{\char`\"}}%
            \def\PYGZti{\discretionary{\char`\~}{\Wrappedafterbreak}{\char`\~}}%
        }
        % Some characters . , ; ? ! / are not pygmentized.
        % This macro makes them "active" and they will insert potential linebreaks
        \newcommand*\Wrappedbreaksatpunct {%
            \lccode`\~`\.\lowercase{\def~}{\discretionary{\hbox{\char`\.}}{\Wrappedafterbreak}{\hbox{\char`\.}}}%
            \lccode`\~`\,\lowercase{\def~}{\discretionary{\hbox{\char`\,}}{\Wrappedafterbreak}{\hbox{\char`\,}}}%
            \lccode`\~`\;\lowercase{\def~}{\discretionary{\hbox{\char`\;}}{\Wrappedafterbreak}{\hbox{\char`\;}}}%
            \lccode`\~`\:\lowercase{\def~}{\discretionary{\hbox{\char`\:}}{\Wrappedafterbreak}{\hbox{\char`\:}}}%
            \lccode`\~`\?\lowercase{\def~}{\discretionary{\hbox{\char`\?}}{\Wrappedafterbreak}{\hbox{\char`\?}}}%
            \lccode`\~`\!\lowercase{\def~}{\discretionary{\hbox{\char`\!}}{\Wrappedafterbreak}{\hbox{\char`\!}}}%
            \lccode`\~`\/\lowercase{\def~}{\discretionary{\hbox{\char`\/}}{\Wrappedafterbreak}{\hbox{\char`\/}}}%
            \catcode`\.\active
            \catcode`\,\active
            \catcode`\;\active
            \catcode`\:\active
            \catcode`\?\active
            \catcode`\!\active
            \catcode`\/\active
            \lccode`\~`\~
        }
    \makeatother

    \let\OriginalVerbatim=\Verbatim
    \makeatletter
    \renewcommand{\Verbatim}[1][1]{%
        %\parskip\z@skip
        \sbox\Wrappedcontinuationbox {\Wrappedcontinuationsymbol}%
        \sbox\Wrappedvisiblespacebox {\FV@SetupFont\Wrappedvisiblespace}%
        \def\FancyVerbFormatLine ##1{\hsize\linewidth
            \vtop{\raggedright\hyphenpenalty\z@\exhyphenpenalty\z@
                \doublehyphendemerits\z@\finalhyphendemerits\z@
                \strut ##1\strut}%
        }%
        % If the linebreak is at a space, the latter will be displayed as visible
        % space at end of first line, and a continuation symbol starts next line.
        % Stretch/shrink are however usually zero for typewriter font.
        \def\FV@Space {%
            \nobreak\hskip\z@ plus\fontdimen3\font minus\fontdimen4\font
            \discretionary{\copy\Wrappedvisiblespacebox}{\Wrappedafterbreak}
            {\kern\fontdimen2\font}%
        }%

        % Allow breaks at special characters using \PYG... macros.
        \Wrappedbreaksatspecials
        % Breaks at punctuation characters . , ; ? ! and / need catcode=\active
        \OriginalVerbatim[#1,codes*=\Wrappedbreaksatpunct]%
    }
    \makeatother

    % Exact colors from NB
    \definecolor{incolor}{HTML}{303F9F}
    \definecolor{outcolor}{HTML}{D84315}
    \definecolor{cellborder}{HTML}{CFCFCF}
    \definecolor{cellbackground}{HTML}{F7F7F7}

    % prompt
    \makeatletter
    \newcommand{\boxspacing}{\kern\kvtcb@left@rule\kern\kvtcb@boxsep}
    \makeatother
    \newcommand{\prompt}[4]{
        {\ttfamily\llap{{\color{#2}[#3]:\hspace{3pt}#4}}\vspace{-\baselineskip}}
    }
    

    
    % Prevent overflowing lines due to hard-to-break entities
    \sloppy
    % Setup hyperref package
    \hypersetup{
      breaklinks=true,  % so long urls are correctly broken across lines
      colorlinks=true,
      urlcolor=urlcolor,
      linkcolor=linkcolor,
      citecolor=citecolor,
      }
    % Slightly bigger margins than the latex defaults
    
    \geometry{verbose,tmargin=1in,bmargin=1in,lmargin=1in,rmargin=1in}
    
    

\begin{document}
    
    \maketitle
    
    

    
    \hypertarget{theorem}{%
\section{Theorem}\label{theorem}}

    Suppose that \(\Gamma\) is a conic and let \(C\) be a cubic such that
\(\Gamma \subseteq \mathrm{Eig}(C)\). Then we have three possible cases:
* \(\Gamma = \mathcal{Q}_{\mathrm{iso}}\). This is true if and only if
\(C = \ell \mathcal{Q}_{\mathrm{iso}}\) where \(\ell\) is any line of
the plane; * \(\Gamma\) is bitangent to \(\mathcal{Q}_{\mathrm{iso}}\)
in two distinct points \(P_1\) and \(P_2\). This is true iff
\(C = r(\lambda \mathcal{Q}_{\mathrm{iso}} + \mu r^2)\), where
\(\lambda, \mu \in \mathbb{C}\), \(r = P_1 \vee P_2\). In this case
\(\Gamma = \lambda \mathcal{Q}_{\mathrm{iso}} + 3 \mu r^2\); *
\(\Gamma\) is iperosculating \(\mathcal{Q}_{\mathrm{iso}}\) in a point
\(P\). This is true iff \(C = r(\mathcal{Q}_{\mathrm{iso}}-r^2)\), where
\(r\) is the tangent to \(\mathcal{Q}_{\mathrm{iso}}\) in \(P\). In this
case \(\Gamma = \mathcal{Q}_{\mathrm{iso}}-3r^2\).

    \begin{tcolorbox}[breakable, size=fbox, boxrule=1pt, pad at break*=1mm,colback=cellbackground, colframe=cellborder]
\prompt{In}{incolor}{1}{\boxspacing}
\begin{Verbatim}[commandchars=\\\{\}]
\PY{n}{load}\PY{p}{(}\PY{l+s+s2}{\PYZdq{}}\PY{l+s+s2}{basic\PYZus{}functions.sage}\PY{l+s+s2}{\PYZdq{}}\PY{p}{)}
\end{Verbatim}
\end{tcolorbox}

    If \texttt{do\_long\_computations} is \texttt{True}, we perform all the
computations, if it is \texttt{False}, we load the computations done in
a previous session.

    \begin{tcolorbox}[breakable, size=fbox, boxrule=1pt, pad at break*=1mm,colback=cellbackground, colframe=cellborder]
\prompt{In}{incolor}{2}{\boxspacing}
\begin{Verbatim}[commandchars=\\\{\}]
\PY{n}{do\PYZus{}long\PYZus{}computations} \PY{o}{=} \PY{k+kc}{False}
\end{Verbatim}
\end{tcolorbox}

    \hypertarget{four-distinct-eigenpoints-p_1-dotsc-p_4-on-the-isotropic-conic}{%
\subsection{\texorpdfstring{Four distinct eigenpoints
\(P_1, \dotsc, P_4\) on the isotropic
conic}{Four distinct eigenpoints P\_1, \textbackslash dotsc, P\_4 on the isotropic conic}}\label{four-distinct-eigenpoints-p_1-dotsc-p_4-on-the-isotropic-conic}}

    We assume \(P_1 = (1: i: 0)\) and we define the generic point on the
isotropic conic, which depends on two parameteres: \(l_1\) and \(l_2\).
We can assume \(l_2 \neq 0\):

    \begin{tcolorbox}[breakable, size=fbox, boxrule=1pt, pad at break*=1mm,colback=cellbackground, colframe=cellborder]
\prompt{In}{incolor}{3}{\boxspacing}
\begin{Verbatim}[commandchars=\\\{\}]
\PY{n}{P1} \PY{o}{=} \PY{n}{vector}\PY{p}{(}\PY{n}{S}\PY{p}{,} \PY{p}{(}\PY{l+m+mi}{1}\PY{p}{,} \PY{n}{ii}\PY{p}{,} \PY{l+m+mi}{0}\PY{p}{)}\PY{p}{)}
\PY{n}{Pg} \PY{o}{=} \PY{n}{vector}\PY{p}{(}\PY{n}{S}\PY{p}{,} \PY{p}{(}\PY{p}{(}\PY{o}{\PYZhy{}}\PY{n}{ii}\PY{p}{)}\PY{o}{*}\PY{n}{l1}\PY{o}{\PYZca{}}\PY{l+m+mi}{2} \PY{o}{+} \PY{p}{(}\PY{o}{\PYZhy{}}\PY{n}{ii}\PY{p}{)}\PY{o}{*}\PY{n}{l2}\PY{o}{\PYZca{}}\PY{l+m+mi}{2}\PY{p}{,} \PY{n}{l1}\PY{o}{\PYZca{}}\PY{l+m+mi}{2} \PY{o}{\PYZhy{}} \PY{n}{l2}\PY{o}{\PYZca{}}\PY{l+m+mi}{2}\PY{p}{,} \PY{l+m+mi}{2}\PY{o}{*}\PY{n}{l1}\PY{o}{*}\PY{n}{l2}\PY{p}{)}\PY{p}{)}
\end{Verbatim}
\end{tcolorbox}

    \begin{tcolorbox}[breakable, size=fbox, boxrule=1pt, pad at break*=1mm,colback=cellbackground, colframe=cellborder]
\prompt{In}{incolor}{4}{\boxspacing}
\begin{Verbatim}[commandchars=\\\{\}]
\PY{k}{assert}\PY{p}{(}\PY{n}{matrix}\PY{p}{(}\PY{p}{[}\PY{n}{P1}\PY{p}{,} \PY{n}{Pg}\PY{o}{.}\PY{n}{subs}\PY{p}{(}\PY{n}{l2}\PY{o}{=}\PY{l+m+mi}{0}\PY{p}{)}\PY{p}{]}\PY{p}{)}\PY{o}{.}\PY{n}{rank}\PY{p}{(}\PY{p}{)} \PY{o}{==} \PY{l+m+mi}{1}\PY{p}{)}
\end{Verbatim}
\end{tcolorbox}

    Now we define three points on the isotropic conic and the matrix of
conditions of the four eigenpoints

    \begin{tcolorbox}[breakable, size=fbox, boxrule=1pt, pad at break*=1mm,colback=cellbackground, colframe=cellborder]
\prompt{In}{incolor}{5}{\boxspacing}
\begin{Verbatim}[commandchars=\\\{\}]
\PY{n}{P2} \PY{o}{=} \PY{n}{Pg}\PY{o}{.}\PY{n}{subs}\PY{p}{(}\PY{p}{\PYZob{}}\PY{n}{l1}\PY{p}{:}\PY{n}{u1}\PY{p}{,} \PY{n}{l2}\PY{p}{:}\PY{l+m+mi}{1}\PY{p}{\PYZcb{}}\PY{p}{)}
\PY{n}{P3} \PY{o}{=} \PY{n}{Pg}\PY{o}{.}\PY{n}{subs}\PY{p}{(}\PY{p}{\PYZob{}}\PY{n}{l1}\PY{p}{:}\PY{n}{v1}\PY{p}{,} \PY{n}{l2}\PY{p}{:}\PY{l+m+mi}{1}\PY{p}{\PYZcb{}}\PY{p}{)}
\PY{n}{P4} \PY{o}{=} \PY{n}{Pg}\PY{o}{.}\PY{n}{subs}\PY{p}{(}\PY{p}{\PYZob{}}\PY{n}{l1}\PY{p}{:}\PY{n}{w1}\PY{p}{,} \PY{n}{l2}\PY{p}{:}\PY{l+m+mi}{1}\PY{p}{\PYZcb{}}\PY{p}{)}

\PY{n}{M} \PY{o}{=} \PY{n}{condition\PYZus{}matrix}\PY{p}{(}\PY{p}{[}\PY{n}{P1}\PY{p}{,} \PY{n}{P2}\PY{p}{,} \PY{n}{P3}\PY{p}{,} \PY{n}{P4}\PY{p}{]}\PY{p}{,} \PY{n}{S}\PY{p}{,} \PY{n}{standard}\PY{o}{=}\PY{l+s+s2}{\PYZdq{}}\PY{l+s+s2}{all}\PY{l+s+s2}{\PYZdq{}}\PY{p}{)}
\end{Verbatim}
\end{tcolorbox}

    \hypertarget{first-we-assume-that-none-of-the-four-points-is-1--i-0.}{%
\subsubsection{\texorpdfstring{First, we assume that none of the four
points is
\((1: -i: 0)\).}{First, we assume that none of the four points is (1: -i: 0).}}\label{first-we-assume-that-none-of-the-four-points-is-1--i-0.}}

    Since the point \((1, -i, 0)\) is NOT one of the points
\(P_2, P_3, P_4\), we can assume that \(u_1, v_1, w_1\) are not zero.
The consequence is that the third coordinate of \(P_2, P_3, P_4\) is not
zero. First we see that if \((1: -i:0)\) is one of the points
\(P_2, P_3, P_4\) then we have \(u_1\) or \(v_1\) or \(w_1\) zero.

    \begin{tcolorbox}[breakable, size=fbox, boxrule=1pt, pad at break*=1mm,colback=cellbackground, colframe=cellborder]
\prompt{In}{incolor}{6}{\boxspacing}
\begin{Verbatim}[commandchars=\\\{\}]
\PY{k}{assert}\PY{p}{(}\PY{n}{S}\PY{o}{.}\PY{n}{ideal}\PY{p}{(}\PY{n}{matrix}\PY{p}{(}\PY{p}{[}\PY{p}{(}\PY{l+m+mi}{1}\PY{p}{,} \PY{o}{\PYZhy{}}\PY{n}{ii}\PY{p}{,} \PY{l+m+mi}{0}\PY{p}{)}\PY{p}{,} \PY{n}{P2}\PY{p}{]}\PY{p}{)}\PY{o}{.}\PY{n}{minors}\PY{p}{(}\PY{l+m+mi}{2}\PY{p}{)}\PY{p}{)}\PY{o}{.}\PY{n}{groebner\PYZus{}basis}\PY{p}{(}\PY{p}{)}\PY{o}{==}\PY{p}{[}\PY{n}{u1}\PY{p}{]}\PY{p}{)}
\PY{k}{assert}\PY{p}{(}\PY{n}{S}\PY{o}{.}\PY{n}{ideal}\PY{p}{(}\PY{n}{matrix}\PY{p}{(}\PY{p}{[}\PY{p}{(}\PY{l+m+mi}{1}\PY{p}{,} \PY{o}{\PYZhy{}}\PY{n}{ii}\PY{p}{,} \PY{l+m+mi}{0}\PY{p}{)}\PY{p}{,} \PY{n}{P3}\PY{p}{]}\PY{p}{)}\PY{o}{.}\PY{n}{minors}\PY{p}{(}\PY{l+m+mi}{2}\PY{p}{)}\PY{p}{)}\PY{o}{.}\PY{n}{groebner\PYZus{}basis}\PY{p}{(}\PY{p}{)}\PY{o}{==}\PY{p}{[}\PY{n}{v1}\PY{p}{]}\PY{p}{)}
\PY{k}{assert}\PY{p}{(}\PY{n}{S}\PY{o}{.}\PY{n}{ideal}\PY{p}{(}\PY{n}{matrix}\PY{p}{(}\PY{p}{[}\PY{p}{(}\PY{l+m+mi}{1}\PY{p}{,} \PY{o}{\PYZhy{}}\PY{n}{ii}\PY{p}{,} \PY{l+m+mi}{0}\PY{p}{)}\PY{p}{,} \PY{n}{P4}\PY{p}{]}\PY{p}{)}\PY{o}{.}\PY{n}{minors}\PY{p}{(}\PY{l+m+mi}{2}\PY{p}{)}\PY{p}{)}\PY{o}{.}\PY{n}{groebner\PYZus{}basis}\PY{p}{(}\PY{p}{)}\PY{o}{==}\PY{p}{[}\PY{n}{w1}\PY{p}{]}\PY{p}{)}

\PY{k}{assert}\PY{p}{(}\PY{n}{P2}\PY{p}{[}\PY{l+m+mi}{2}\PY{p}{]} \PY{o}{==} \PY{l+m+mi}{2}\PY{o}{*}\PY{n}{u1}\PY{p}{)}
\PY{k}{assert}\PY{p}{(}\PY{n}{P3}\PY{p}{[}\PY{l+m+mi}{2}\PY{p}{]} \PY{o}{==} \PY{l+m+mi}{2}\PY{o}{*}\PY{n}{v1}\PY{p}{)}
\PY{k}{assert}\PY{p}{(}\PY{n}{P4}\PY{p}{[}\PY{l+m+mi}{2}\PY{p}{]} \PY{o}{==} \PY{l+m+mi}{2}\PY{o}{*}\PY{n}{w1}\PY{p}{)}
\end{Verbatim}
\end{tcolorbox}

    Since \[
P_{2, z} M_{(4)} - P_{2, y} M_{(5)} + P_{2, x} M_{(6)}=0
\] \[
P_{3, z} M_{(7)} - P_{3, y} M_{(8)} + P_{3, x} M_{(9)}]=0
\] \[
P_{4, z} M_{(10)} - P_{4, y} M_{(11)} + P_{4, x} M_{(12)}=0
\] the rows \(M_{(4)}, M_{(7)}, M_{(10)}\) are unnecessary in our
hypothesis; similarly, the row \(M_{(2)}\) is also unnecessary. We
construct the matrix \(M_1\) eraising the rows \(2, 3, 6, 9\) from
\(M\):

    \begin{tcolorbox}[breakable, size=fbox, boxrule=1pt, pad at break*=1mm,colback=cellbackground, colframe=cellborder]
\prompt{In}{incolor}{7}{\boxspacing}
\begin{Verbatim}[commandchars=\\\{\}]
\PY{n}{M1} \PY{o}{=} \PY{n}{M}\PY{o}{.}\PY{n}{matrix\PYZus{}from\PYZus{}rows}\PY{p}{(}\PY{p}{[}\PY{l+m+mi}{0}\PY{p}{,} \PY{l+m+mi}{1}\PY{p}{,} \PY{l+m+mi}{4}\PY{p}{,} \PY{l+m+mi}{5}\PY{p}{,} \PY{l+m+mi}{7}\PY{p}{,} \PY{l+m+mi}{8}\PY{p}{,} \PY{l+m+mi}{10}\PY{p}{,} \PY{l+m+mi}{11}\PY{p}{]}\PY{p}{)} 
\end{Verbatim}
\end{tcolorbox}

    \(M_1\) has always rank \(7\): it cannot have rank \(8\), since all the
order \(8\) minors are zero:

    \begin{tcolorbox}[breakable, size=fbox, boxrule=1pt, pad at break*=1mm,colback=cellbackground, colframe=cellborder]
\prompt{In}{incolor}{8}{\boxspacing}
\begin{Verbatim}[commandchars=\\\{\}]
\PY{k}{assert}\PY{p}{(}\PY{n}{Set}\PY{p}{(}\PY{n}{M1}\PY{o}{.}\PY{n}{minors}\PY{p}{(}\PY{l+m+mi}{8}\PY{p}{)}\PY{p}{)} \PY{o}{==} \PY{n}{Set}\PY{p}{(}\PY{p}{[}\PY{n}{S}\PY{o}{.}\PY{n}{zero}\PY{p}{(}\PY{p}{)}\PY{p}{]}\PY{p}{)}\PY{p}{)}
\end{Verbatim}
\end{tcolorbox}

    If we impose that all the 7-minors of \(M_1\) are zero, we get that (if
the points \(P_1, P_2, P_3, P_4\) are distinct) there are no solutions:

    \begin{tcolorbox}[breakable, size=fbox, boxrule=1pt, pad at break*=1mm,colback=cellbackground, colframe=cellborder]
\prompt{In}{incolor}{9}{\boxspacing}
\begin{Verbatim}[commandchars=\\\{\}]
\PY{n}{mm7} \PY{o}{=} \PY{n}{S}\PY{o}{.}\PY{n}{ideal}\PY{p}{(}\PY{n}{M1}\PY{o}{.}\PY{n}{minors}\PY{p}{(}\PY{l+m+mi}{7}\PY{p}{)}\PY{p}{)}
\PY{n}{mm7} \PY{o}{=} \PY{n}{mm7}\PY{o}{.}\PY{n}{saturation}\PY{p}{(}\PY{n}{S}\PY{o}{.}\PY{n}{ideal}\PY{p}{(}\PY{n}{u1}\PY{o}{*}\PY{n}{v1}\PY{o}{*}\PY{n}{w1}\PY{p}{)}\PY{p}{)}\PY{p}{[}\PY{l+m+mi}{0}\PY{p}{]}
\PY{n}{mm7} \PY{o}{=} \PY{n}{mm7}\PY{o}{.}\PY{n}{saturation}\PY{p}{(}\PY{n}{S}\PY{o}{.}\PY{n}{ideal}\PY{p}{(}\PY{p}{(}\PY{n}{u1}\PY{o}{\PYZhy{}}\PY{n}{v1}\PY{p}{)}\PY{o}{*}\PY{p}{(}\PY{n}{u1}\PY{o}{\PYZhy{}}\PY{n}{w1}\PY{p}{)}\PY{o}{*}\PY{p}{(}\PY{n}{v1}\PY{o}{\PYZhy{}}\PY{n}{w1}\PY{p}{)}\PY{p}{)}\PY{p}{)}\PY{p}{[}\PY{l+m+mi}{0}\PY{p}{]}
\PY{k}{assert}\PY{p}{(}\PY{n}{mm7} \PY{o}{==} \PY{n}{S}\PY{o}{.}\PY{n}{ideal}\PY{p}{(}\PY{n}{S}\PY{p}{(}\PY{l+m+mi}{1}\PY{p}{)}\PY{p}{)}\PY{p}{)}
\end{Verbatim}
\end{tcolorbox}

    All the cubics of the form \(\ell*\mathcal{Q}_{\mathrm{iso}}\) have
\(\mathcal{Q}_{\mathrm{iso}}\) in the eigenscheme:

    \begin{tcolorbox}[breakable, size=fbox, boxrule=1pt, pad at break*=1mm,colback=cellbackground, colframe=cellborder]
\prompt{In}{incolor}{10}{\boxspacing}
\begin{Verbatim}[commandchars=\\\{\}]
\PY{k}{assert}\PY{p}{(}\PY{n}{gcd}\PY{p}{(}\PY{n}{eig}\PY{p}{(}\PY{p}{(}\PY{n}{x}\PY{o}{*}\PY{n}{u2}\PY{o}{+}\PY{n}{y}\PY{o}{*}\PY{n}{v2}\PY{o}{+}\PY{n}{z}\PY{o}{*}\PY{n}{w2}\PY{p}{)}\PY{o}{*}\PY{n}{Ciso}\PY{p}{)}\PY{p}{)}\PY{o}{.}\PY{n}{quo\PYZus{}rem}\PY{p}{(}\PY{n}{Ciso}\PY{p}{)}\PY{p}{[}\PY{l+m+mi}{1}\PY{p}{]}\PY{o}{==} \PY{l+m+mi}{0}\PY{p}{)}
\end{Verbatim}
\end{tcolorbox}

    Conclusion 1: If four distinc points of \(\mathcal{Q}_{\mathrm{iso}}\)
are eigenpoints of a cubic (and if one of the points is not
\((1, -i, 0)\)), then the cubic splits into
\(\mathcal{Q}_{\mathrm{iso}}\) and a line.

    \emph{Proof}: The matrix \(M_1\) has always rank \(7\), hence all the
cubics with \(P_1\), \(P_2\), \(P_3\), \(P_4\) as eigenpoints are a
linear variety in \(\mathbb{P}^9\) of dimension \(2\). However, this
family contains all the cubics which split into
\(\mathcal{Q}_{\mathrm{iso}}\) and a line of the plane, which is of
dimension \(2\), hence the two families coincide.

    \hypertarget{now-we-assume-that-p_2-is-1--i-0}{%
\subsubsection{\texorpdfstring{Now we assume that \(P_2\) is
\((1: -i, 0)\)}{Now we assume that P\_2 is (1: -i, 0)}}\label{now-we-assume-that-p_2-is-1--i-0}}

    \begin{tcolorbox}[breakable, size=fbox, boxrule=1pt, pad at break*=1mm,colback=cellbackground, colframe=cellborder]
\prompt{In}{incolor}{11}{\boxspacing}
\begin{Verbatim}[commandchars=\\\{\}]
\PY{n}{P2} \PY{o}{=} \PY{n}{vector}\PY{p}{(}\PY{n}{S}\PY{p}{,} \PY{p}{(}\PY{l+m+mi}{1}\PY{p}{,} \PY{o}{\PYZhy{}}\PY{n}{ii}\PY{p}{,} \PY{l+m+mi}{0}\PY{p}{)}\PY{p}{)}
\PY{n}{P3} \PY{o}{=} \PY{n}{Pg}\PY{o}{.}\PY{n}{subs}\PY{p}{(}\PY{p}{\PYZob{}}\PY{n}{l1}\PY{p}{:}\PY{n}{v1}\PY{p}{,} \PY{n}{l2}\PY{p}{:}\PY{l+m+mi}{1}\PY{p}{\PYZcb{}}\PY{p}{)}
\PY{n}{P4} \PY{o}{=} \PY{n}{Pg}\PY{o}{.}\PY{n}{subs}\PY{p}{(}\PY{p}{\PYZob{}}\PY{n}{l1}\PY{p}{:}\PY{n}{w1}\PY{p}{,} \PY{n}{l2}\PY{p}{:}\PY{l+m+mi}{1}\PY{p}{\PYZcb{}}\PY{p}{)}

\PY{n}{M} \PY{o}{=} \PY{n}{condition\PYZus{}matrix}\PY{p}{(}\PY{p}{[}\PY{n}{P1}\PY{p}{,} \PY{n}{P2}\PY{p}{,} \PY{n}{P3}\PY{p}{,} \PY{n}{P4}\PY{p}{]}\PY{p}{,} \PY{n}{S}\PY{p}{,} \PY{n}{standard}\PY{o}{=}\PY{l+s+s2}{\PYZdq{}}\PY{l+s+s2}{all}\PY{l+s+s2}{\PYZdq{}}\PY{p}{)}
\end{Verbatim}
\end{tcolorbox}

    As above, we have that \(M_{(3)}\), \(M_{(6)}\), \(M_{(7)}\),
\(M_{(10)}\) are unnecessary so we erase these rows from \(M\):

    \begin{tcolorbox}[breakable, size=fbox, boxrule=1pt, pad at break*=1mm,colback=cellbackground, colframe=cellborder]
\prompt{In}{incolor}{12}{\boxspacing}
\begin{Verbatim}[commandchars=\\\{\}]
\PY{n}{M1} \PY{o}{=} \PY{n}{M}\PY{o}{.}\PY{n}{matrix\PYZus{}from\PYZus{}rows}\PY{p}{(}\PY{p}{[}\PY{l+m+mi}{0}\PY{p}{,} \PY{l+m+mi}{1}\PY{p}{,} \PY{l+m+mi}{3}\PY{p}{,} \PY{l+m+mi}{4}\PY{p}{,} \PY{l+m+mi}{7}\PY{p}{,} \PY{l+m+mi}{8}\PY{p}{,} \PY{l+m+mi}{10}\PY{p}{,} \PY{l+m+mi}{11}\PY{p}{]}\PY{p}{)} 
\end{Verbatim}
\end{tcolorbox}

    \(M_1\) has rank \(7\). It cannot have rank \(8\):

    \begin{tcolorbox}[breakable, size=fbox, boxrule=1pt, pad at break*=1mm,colback=cellbackground, colframe=cellborder]
\prompt{In}{incolor}{13}{\boxspacing}
\begin{Verbatim}[commandchars=\\\{\}]
\PY{k}{assert}\PY{p}{(}\PY{n}{Set}\PY{p}{(}\PY{n}{M1}\PY{o}{.}\PY{n}{minors}\PY{p}{(}\PY{l+m+mi}{8}\PY{p}{)}\PY{p}{)} \PY{o}{==} \PY{n}{Set}\PY{p}{(}\PY{p}{[}\PY{n}{S}\PY{o}{.}\PY{n}{zero}\PY{p}{(}\PY{p}{)}\PY{p}{]}\PY{p}{)}\PY{p}{)}
\end{Verbatim}
\end{tcolorbox}

    \begin{tcolorbox}[breakable, size=fbox, boxrule=1pt, pad at break*=1mm,colback=cellbackground, colframe=cellborder]
\prompt{In}{incolor}{14}{\boxspacing}
\begin{Verbatim}[commandchars=\\\{\}]
\PY{n}{mm7} \PY{o}{=} \PY{n}{S}\PY{o}{.}\PY{n}{ideal}\PY{p}{(}\PY{n}{M1}\PY{o}{.}\PY{n}{minors}\PY{p}{(}\PY{l+m+mi}{7}\PY{p}{)}\PY{p}{)}
\PY{n}{mm7} \PY{o}{=} \PY{n}{mm7}\PY{o}{.}\PY{n}{saturation}\PY{p}{(}\PY{n}{S}\PY{o}{.}\PY{n}{ideal}\PY{p}{(}\PY{n}{v1}\PY{o}{*}\PY{n}{w1}\PY{p}{)}\PY{p}{)}\PY{p}{[}\PY{l+m+mi}{0}\PY{p}{]}
\PY{n}{mm7} \PY{o}{=} \PY{n}{mm7}\PY{o}{.}\PY{n}{saturation}\PY{p}{(}\PY{n}{S}\PY{o}{.}\PY{n}{ideal}\PY{p}{(}\PY{p}{(}\PY{n}{v1}\PY{o}{\PYZhy{}}\PY{n}{w1}\PY{p}{)}\PY{p}{)}\PY{p}{)}\PY{p}{[}\PY{l+m+mi}{0}\PY{p}{]}
\PY{k}{assert}\PY{p}{(}\PY{n}{mm7} \PY{o}{==} \PY{n}{S}\PY{o}{.}\PY{n}{ideal}\PY{p}{(}\PY{n}{S}\PY{o}{.}\PY{n}{one}\PY{p}{(}\PY{p}{)}\PY{p}{)}\PY{p}{)}
\end{Verbatim}
\end{tcolorbox}

    Conclusion 2: Also in case one of the point \((1, -i, 0)\) we obtain the
same conclusion: if four eigenpoints of a cubic are on
\(\mathcal{Q}_{\mathrm{iso}}\), then \(\mathcal{Q}_{\mathrm{iso}}\) is
contained in the eigenscheme of the cubic.

    \hypertarget{case-gamma-tangent-to-mathcalq_mathrmiso-in-p_1-and-passing-through-p_2-and-p_3}{%
\subsection{\texorpdfstring{Case \(\Gamma\) tangent to
\(\mathcal{Q}_{\mathrm{iso}}\) in \(P_1\) and passing through \(P_2\)
and
\(P_3\)}{Case \textbackslash Gamma tangent to \textbackslash mathcal\{Q\}\_\{\textbackslash mathrm\{iso\}\} in P\_1 and passing through P\_2 and P\_3}}\label{case-gamma-tangent-to-mathcalq_mathrmiso-in-p_1-and-passing-through-p_2-and-p_3}}

    We redefine \(P_1=(1:i:0)\), here we define the generic point \(P_g\) on
\(\mathcal{Q}_{\mathrm{iso}}\) and two points \(P_2\) and \(P_3\) on
\(\mathcal{Q}_{\mathrm{iso}}\):

    \begin{tcolorbox}[breakable, size=fbox, boxrule=1pt, pad at break*=1mm,colback=cellbackground, colframe=cellborder]
\prompt{In}{incolor}{15}{\boxspacing}
\begin{Verbatim}[commandchars=\\\{\}]
\PY{n}{P1} \PY{o}{=} \PY{n}{vector}\PY{p}{(}\PY{n}{S}\PY{p}{,} \PY{p}{(}\PY{l+m+mi}{1}\PY{p}{,} \PY{n}{ii}\PY{p}{,} \PY{l+m+mi}{0}\PY{p}{)}\PY{p}{)}
\PY{n}{Pg} \PY{o}{=} \PY{n}{vector}\PY{p}{(}\PY{n}{S}\PY{p}{,} \PY{p}{(}\PY{p}{(}\PY{o}{\PYZhy{}}\PY{n}{ii}\PY{p}{)}\PY{o}{*}\PY{n}{l1}\PY{o}{\PYZca{}}\PY{l+m+mi}{2} \PY{o}{+} \PY{p}{(}\PY{o}{\PYZhy{}}\PY{n}{ii}\PY{p}{)}\PY{o}{*}\PY{n}{l2}\PY{o}{\PYZca{}}\PY{l+m+mi}{2}\PY{p}{,} \PY{n}{l1}\PY{o}{\PYZca{}}\PY{l+m+mi}{2} \PY{o}{\PYZhy{}} \PY{n}{l2}\PY{o}{\PYZca{}}\PY{l+m+mi}{2}\PY{p}{,} \PY{l+m+mi}{2}\PY{o}{*}\PY{n}{l1}\PY{o}{*}\PY{n}{l2}\PY{p}{)}\PY{p}{)}
\PY{k}{assert}\PY{p}{(}\PY{n}{matrix}\PY{p}{(}\PY{p}{[}\PY{n}{P1}\PY{p}{,} \PY{n}{Pg}\PY{o}{.}\PY{n}{subs}\PY{p}{(}\PY{n}{l2}\PY{o}{=}\PY{l+m+mi}{0}\PY{p}{)}\PY{p}{]}\PY{p}{)}\PY{o}{.}\PY{n}{rank}\PY{p}{(}\PY{p}{)} \PY{o}{==} \PY{l+m+mi}{1}\PY{p}{)}

\PY{n}{P2} \PY{o}{=} \PY{n}{Pg}\PY{o}{.}\PY{n}{subs}\PY{p}{(}\PY{p}{\PYZob{}}\PY{n}{l1}\PY{p}{:}\PY{n}{u1}\PY{p}{,} \PY{n}{l2}\PY{p}{:}\PY{l+m+mi}{1}\PY{p}{\PYZcb{}}\PY{p}{)}
\PY{n}{P3} \PY{o}{=} \PY{n}{Pg}\PY{o}{.}\PY{n}{subs}\PY{p}{(}\PY{p}{\PYZob{}}\PY{n}{l1}\PY{p}{:}\PY{n}{v1}\PY{p}{,} \PY{n}{l2}\PY{p}{:}\PY{l+m+mi}{1}\PY{p}{\PYZcb{}}\PY{p}{)}
\end{Verbatim}
\end{tcolorbox}

    Also here we consider two cases: * \((1: -i: 0)\) is not one of the
points \(P_2\), \(P_3\); * \(P_3=(1:-i:0)\).

    \hypertarget{case-p_2-and-p_3-not-1--i-0.}{%
\subsubsection{\texorpdfstring{Case \(P_2\) and \(P_3\) not
\((1: -i: 0)\).}{Case P\_2 and P\_3 not (1: -i: 0).}}\label{case-p_2-and-p_3-not-1--i-0.}}

    Hence, as above, we can assume that \(u_1\) and \(v_1\) are not zero.

    We compute the pencil of conics \(C_g\) passing through
\(P_1, P_2, P_3\) and tangent to \(\mathcal{Q}_{\mathrm{iso}}\) in
\(P_1\):

\(c_2\) is the conic given by the two lines \(P_1 \vee P_2\) and
\(P_1 \vee P_3\)

    \begin{tcolorbox}[breakable, size=fbox, boxrule=1pt, pad at break*=1mm,colback=cellbackground, colframe=cellborder]
\prompt{In}{incolor}{16}{\boxspacing}
\begin{Verbatim}[commandchars=\\\{\}]
\PY{n}{c2} \PY{o}{=} \PY{n}{matrix}\PY{p}{(}\PY{p}{[}\PY{n}{P1}\PY{p}{,} \PY{n}{P2}\PY{p}{,} \PY{p}{[}\PY{n}{x}\PY{p}{,} \PY{n}{y}\PY{p}{,} \PY{n}{z}\PY{p}{]}\PY{p}{]}\PY{p}{)}\PY{o}{.}\PY{n}{det}\PY{p}{(}\PY{p}{)}\PY{o}{*}\PY{n}{matrix}\PY{p}{(}\PY{p}{[}\PY{n}{P1}\PY{p}{,} \PY{n}{P3}\PY{p}{,} \PY{p}{[}\PY{n}{x}\PY{p}{,} \PY{n}{y}\PY{p}{,} \PY{n}{z}\PY{p}{]}\PY{p}{]}\PY{p}{)}\PY{o}{.}\PY{n}{det}\PY{p}{(}\PY{p}{)}
\PY{n}{Cg} \PY{o}{=} \PY{n}{Ciso}\PY{o}{+}\PY{n}{l1}\PY{o}{*}\PY{n}{c2}
\end{Verbatim}
\end{tcolorbox}

    Now we construct a generic point (different from \(P_1\)) on \(C_g\),
depending on the parameter \(w_1\) (\(P_{gn}\) is the generic point on
\(C_g\))

    \begin{tcolorbox}[breakable, size=fbox, boxrule=1pt, pad at break*=1mm,colback=cellbackground, colframe=cellborder]
\prompt{In}{incolor}{17}{\boxspacing}
\begin{Verbatim}[commandchars=\\\{\}]
\PY{n}{foo} \PY{o}{=} \PY{n}{Cg}\PY{o}{.}\PY{n}{subs}\PY{p}{(}\PY{n}{y}\PY{o}{=}\PY{n}{ii}\PY{o}{*}\PY{n}{x}\PY{o}{+}\PY{n}{w1}\PY{o}{*}\PY{n}{z}\PY{p}{)}\PY{o}{.}\PY{n}{factor}\PY{p}{(}\PY{p}{)}\PY{p}{[}\PY{o}{\PYZhy{}}\PY{l+m+mi}{1}\PY{p}{]}\PY{p}{[}\PY{l+m+mi}{0}\PY{p}{]}

\PY{n}{Pgn} \PY{o}{=} \PY{n}{vector}\PY{p}{(}
    \PY{n}{S}\PY{p}{,} 
    \PY{p}{(}
        \PY{n}{foo}\PY{o}{.}\PY{n}{coefficient}\PY{p}{(}\PY{n}{z}\PY{p}{)}\PY{p}{,} 
        \PY{n}{ii}\PY{o}{*}\PY{p}{(}\PY{n}{foo}\PY{o}{.}\PY{n}{coefficient}\PY{p}{(}\PY{n}{z}\PY{p}{)}\PY{p}{)} \PY{o}{+} \PY{n}{w1}\PY{o}{*}\PY{p}{(}\PY{o}{\PYZhy{}}\PY{n}{foo}\PY{o}{.}\PY{n}{coefficient}\PY{p}{(}\PY{n}{x}\PY{p}{)}\PY{p}{)}\PY{p}{,}
        \PY{o}{\PYZhy{}}\PY{n}{foo}\PY{o}{.}\PY{n}{coefficient}\PY{p}{(}\PY{n}{x}\PY{p}{)}
    \PY{p}{)}
\PY{p}{)}

\PY{c+c1}{\PYZsh{}\PYZsh{} Pg is the following:}
\PY{k}{assert}\PY{p}{(}
    \PY{n}{Pgn} \PY{o}{==} 
    \PY{n}{vector}\PY{p}{(}
        \PY{n}{S}\PY{p}{,} 
        \PY{p}{(}
            \PY{n}{u1}\PY{o}{*}\PY{n}{v1}\PY{o}{*}\PY{n}{w1}\PY{o}{\PYZca{}}\PY{l+m+mi}{2}\PY{o}{*}\PY{n}{l1} \PY{o}{+} \PY{n}{u1}\PY{o}{*}\PY{n}{w1}\PY{o}{*}\PY{n}{l1} \PY{o}{+} \PY{n}{v1}\PY{o}{*}\PY{n}{w1}\PY{o}{*}\PY{n}{l1} \PY{o}{+} \PY{l+m+mi}{1}\PY{o}{/}\PY{l+m+mi}{4}\PY{o}{*}\PY{n}{w1}\PY{o}{\PYZca{}}\PY{l+m+mi}{2} \PY{o}{+} \PY{n}{l1} \PY{o}{+} \PY{l+m+mi}{1}\PY{o}{/}\PY{l+m+mi}{4}\PY{p}{,}
            \PY{n}{ii}\PY{o}{*}\PY{n}{u1}\PY{o}{*}\PY{n}{v1}\PY{o}{*}\PY{n}{w1}\PY{o}{\PYZca{}}\PY{l+m+mi}{2}\PY{o}{*}\PY{n}{l1} \PY{o}{+} \PY{n}{ii}\PY{o}{*}\PY{n}{u1}\PY{o}{*}\PY{n}{w1}\PY{o}{*}\PY{n}{l1} \PY{o}{+} \PY{n}{ii}\PY{o}{*}\PY{n}{v1}\PY{o}{*}\PY{n}{w1}\PY{o}{*}\PY{n}{l1} \PY{o}{+} \PY{p}{(}\PY{o}{\PYZhy{}}\PY{l+m+mi}{1}\PY{o}{/}\PY{l+m+mi}{4}\PY{o}{*}\PY{n}{ii}\PY{p}{)}\PY{o}{*}\PY{n}{w1}\PY{o}{\PYZca{}}\PY{l+m+mi}{2} \PY{o}{+} \PY{n}{ii}\PY{o}{*}\PY{n}{l1} \PY{o}{+} \PY{p}{(}\PY{l+m+mi}{1}\PY{o}{/}\PY{l+m+mi}{4}\PY{o}{*}\PY{n}{ii}\PY{p}{)}\PY{p}{,}
            \PY{p}{(}\PY{o}{\PYZhy{}}\PY{l+m+mi}{1}\PY{o}{/}\PY{l+m+mi}{2}\PY{o}{*}\PY{n}{ii}\PY{p}{)}\PY{o}{*}\PY{n}{w1}
        \PY{p}{)}
    \PY{p}{)}
\PY{p}{)}

\PY{n}{Pg1}\PY{p}{,} \PY{n}{Pg2} \PY{o}{=} \PY{n}{Pgn}\PY{o}{.}\PY{n}{subs}\PY{p}{(}\PY{p}{\PYZob{}}\PY{n}{w1}\PY{p}{:}\PY{n}{w1}\PY{p}{\PYZcb{}}\PY{p}{)}\PY{p}{,} \PY{n}{Pgn}\PY{o}{.}\PY{n}{subs}\PY{p}{(}\PY{p}{\PYZob{}}\PY{n}{w1}\PY{p}{:}\PY{n}{w2}\PY{p}{\PYZcb{}}\PY{p}{)}
\end{Verbatim}
\end{tcolorbox}

    \(Pg_1\), \(Pg_2\) are two points on \(C_g\). We can assume \(l_1\),
\(u_1\), \(v_1\), \(w_1\), \(w_2\), \((w_1-w_2)\), \((u_1-v_1)\),
\((v_1 w_2+1)\), \((u_1 w_2+1)\), \((v_1 w_1+1)\), \((u_1 w_1+1)\) all
different from 0.

Indeed, \(l_1 = 0\) gives \(C_g = \mathcal{Q}_{\mathrm{iso}}\)
(condition already considered above) \(u_1 = 0\) or \(v_1 = 0\) gives
that \(P_2\) or \(P_3\) is the point \((1, -i, 0)\), and we are in case
\(1\), so this is not possible;

\(w_1 = 0\) gives \(Pg_1 = P_1\):

\(w_2 = 0\) gives \(Pg_2 = P1\):

\(w_1-w_2 = 0\) gives \(Pg_1 = Pg_2\)

\(u_1-v_1 = 0\) gives \(P_2 = P_3\)

\(v_1 w_2+1 = 0\) gives \(P_3 = Pg_2\):

    \begin{tcolorbox}[breakable, size=fbox, boxrule=1pt, pad at break*=1mm,colback=cellbackground, colframe=cellborder]
\prompt{In}{incolor}{18}{\boxspacing}
\begin{Verbatim}[commandchars=\\\{\}]
\PY{k}{assert}\PY{p}{(}\PY{n}{Set}\PY{p}{(}\PY{n}{matrix}\PY{p}{(}\PY{p}{[}\PY{n}{Pg1}\PY{o}{.}\PY{n}{subs}\PY{p}{(}\PY{n}{w1}\PY{o}{=}\PY{l+m+mi}{0}\PY{p}{)}\PY{p}{,} \PY{n}{P1}\PY{p}{]}\PY{p}{)}\PY{o}{.}\PY{n}{minors}\PY{p}{(}\PY{l+m+mi}{2}\PY{p}{)}\PY{p}{)} \PY{o}{==} \PY{n}{Set}\PY{p}{(}\PY{p}{[}\PY{n}{S}\PY{p}{(}\PY{l+m+mi}{0}\PY{p}{)}\PY{p}{]}\PY{p}{)}\PY{p}{)}
\PY{k}{assert}\PY{p}{(}\PY{n}{Set}\PY{p}{(}\PY{n}{matrix}\PY{p}{(}\PY{p}{[}\PY{n}{Pg2}\PY{o}{.}\PY{n}{subs}\PY{p}{(}\PY{n}{w2}\PY{o}{=}\PY{l+m+mi}{0}\PY{p}{)}\PY{p}{,} \PY{n}{P1}\PY{p}{]}\PY{p}{)}\PY{o}{.}\PY{n}{minors}\PY{p}{(}\PY{l+m+mi}{2}\PY{p}{)}\PY{p}{)} \PY{o}{==} \PY{n}{Set}\PY{p}{(}\PY{p}{[}\PY{n}{S}\PY{p}{(}\PY{l+m+mi}{0}\PY{p}{)}\PY{p}{]}\PY{p}{)}\PY{p}{)}
\PY{k}{assert}\PY{p}{(}
    \PY{n}{S}\PY{o}{.}\PY{n}{ideal}\PY{p}{(}
        \PY{n}{matrix}\PY{p}{(}\PY{p}{[}\PY{n}{P3}\PY{p}{,} \PY{n}{Pg2}\PY{p}{]}\PY{p}{)}\PY{o}{.}\PY{n}{minors}\PY{p}{(}\PY{l+m+mi}{2}\PY{p}{)}
    \PY{p}{)}\PY{o}{.}\PY{n}{saturation}\PY{p}{(}\PY{n}{w2}\PY{p}{)}\PY{p}{[}\PY{l+m+mi}{0}\PY{p}{]} \PY{o}{==} \PY{n}{S}\PY{o}{.}\PY{n}{ideal}\PY{p}{(}\PY{n}{v1}\PY{o}{*}\PY{n}{w2}\PY{o}{+}\PY{l+m+mi}{1}\PY{p}{)}
\PY{p}{)}
\end{Verbatim}
\end{tcolorbox}

    Similarly, \((u_1 w_2+1)\), \((v_1 w_1+1)\), \((u_1 w_1+1)\) give,
respectively: \(P_2 = Pg_2\), \(P_3 = Pg_1\), \(P_2 = Pg_1\). Hence we
define a polynomial \texttt{degenerate\_cases}, which contains all these
degenerate cases and we can saturate our computations w.r.t. this
polynomial. So we construct the polynomial of degenerate cases, in order
to compute saturations.

    \begin{tcolorbox}[breakable, size=fbox, boxrule=1pt, pad at break*=1mm,colback=cellbackground, colframe=cellborder]
\prompt{In}{incolor}{19}{\boxspacing}
\begin{Verbatim}[commandchars=\\\{\}]
\PY{n}{degenerate\PYZus{}cases} \PY{o}{=} \PY{n}{l1}\PY{o}{*}\PY{n}{v1}\PY{o}{*}\PY{n}{u1}\PY{o}{*}\PY{n}{w1}\PY{o}{*}\PY{n}{w2}\PY{o}{*}\PY{p}{(}\PY{n}{w1}\PY{o}{\PYZhy{}}\PY{n}{w2}\PY{p}{)}\PY{o}{*}\PY{p}{(}\PY{n}{u1}\PY{o}{\PYZhy{}}\PY{n}{v1}\PY{p}{)}\PY{o}{*}\PY{p}{(}\PY{n}{v1}\PY{o}{*}\PY{n}{w2}\PY{o}{+}\PY{l+m+mi}{1}\PY{p}{)}\PY{o}{*}\PY{p}{(}\PY{n}{u1}\PY{o}{*}\PY{n}{w2}\PY{o}{+}\PY{l+m+mi}{1}\PY{p}{)}\PY{o}{*}\PY{p}{(}\PY{n}{v1}\PY{o}{*}\PY{n}{w1}\PY{o}{+}\PY{l+m+mi}{1}\PY{p}{)}\PY{o}{*}\PY{p}{(}\PY{n}{u1}\PY{o}{*}\PY{n}{w1}\PY{o}{+}\PY{l+m+mi}{1}\PY{p}{)}
\end{Verbatim}
\end{tcolorbox}

    The points \(P_1\), \(P_2\), \(P_3\), \(Pg_1\), \(Pg_2\) are \(5\)
points on the conic \(C_g\). If they are eigenpoints, the following
matrix must have rank 9 or less:

    \begin{tcolorbox}[breakable, size=fbox, boxrule=1pt, pad at break*=1mm,colback=cellbackground, colframe=cellborder]
\prompt{In}{incolor}{20}{\boxspacing}
\begin{Verbatim}[commandchars=\\\{\}]
\PY{n}{M} \PY{o}{=} \PY{n}{condition\PYZus{}matrix}\PY{p}{(}\PY{p}{[}\PY{n}{P1}\PY{p}{,} \PY{n}{P2}\PY{p}{,} \PY{n}{P3}\PY{p}{,} \PY{n}{Pg1}\PY{p}{,} \PY{n}{Pg2}\PY{p}{]}\PY{p}{,} \PY{n}{S}\PY{p}{,} \PY{n}{standard}\PY{o}{=}\PY{l+s+s2}{\PYZdq{}}\PY{l+s+s2}{all}\PY{l+s+s2}{\PYZdq{}}\PY{p}{)}
\end{Verbatim}
\end{tcolorbox}

    We have \(P_{1, z} M_{(1)} - P_{1, y} M_{(2)} + P_{1, x} M_{(3)} = 0\)
and \(P_{1, x} = 1\) so \(M_{(3)}\) is linearly dependent on \(M_{(1)}\)
and \(M_{(2)}\), and can be omitted.

    \begin{tcolorbox}[breakable, size=fbox, boxrule=1pt, pad at break*=1mm,colback=cellbackground, colframe=cellborder]
\prompt{In}{incolor}{21}{\boxspacing}
\begin{Verbatim}[commandchars=\\\{\}]
\PY{k}{assert}\PY{p}{(}\PY{n}{P1}\PY{p}{[}\PY{l+m+mi}{2}\PY{p}{]}\PY{o}{*}\PY{n}{M}\PY{p}{[}\PY{l+m+mi}{0}\PY{p}{]}\PY{o}{\PYZhy{}}\PY{n}{P1}\PY{p}{[}\PY{l+m+mi}{1}\PY{p}{]}\PY{o}{*}\PY{n}{M}\PY{p}{[}\PY{l+m+mi}{1}\PY{p}{]}\PY{o}{+}\PY{n}{P1}\PY{p}{[}\PY{l+m+mi}{0}\PY{p}{]}\PY{o}{*}\PY{n}{M}\PY{p}{[}\PY{l+m+mi}{2}\PY{p}{]} \PY{o}{==} \PY{l+m+mi}{0}\PY{p}{)}
\PY{k}{assert}\PY{p}{(}\PY{n}{P1}\PY{p}{[}\PY{l+m+mi}{0}\PY{p}{]} \PY{o}{==} \PY{l+m+mi}{1}\PY{p}{)}
\end{Verbatim}
\end{tcolorbox}

    We have \(P_{2, z} M_{(4)} - P_{2, y} M_{(4)} + P_{2, x} M_{(6)} = 0\)
\(P_{2, z} = 2 u_1\) which, under our hypothesis, is always not zero, so
\(M_{(4)}\) can be omitted.

    \begin{tcolorbox}[breakable, size=fbox, boxrule=1pt, pad at break*=1mm,colback=cellbackground, colframe=cellborder]
\prompt{In}{incolor}{22}{\boxspacing}
\begin{Verbatim}[commandchars=\\\{\}]
\PY{k}{assert}\PY{p}{(}\PY{n}{P2}\PY{p}{[}\PY{l+m+mi}{2}\PY{p}{]}\PY{o}{*}\PY{n}{M}\PY{p}{[}\PY{l+m+mi}{3}\PY{p}{]}\PY{o}{\PYZhy{}}\PY{n}{P2}\PY{p}{[}\PY{l+m+mi}{1}\PY{p}{]}\PY{o}{*}\PY{n}{M}\PY{p}{[}\PY{l+m+mi}{4}\PY{p}{]}\PY{o}{+}\PY{n}{P2}\PY{p}{[}\PY{l+m+mi}{0}\PY{p}{]}\PY{o}{*}\PY{n}{M}\PY{p}{[}\PY{l+m+mi}{5}\PY{p}{]} \PY{o}{==} \PY{l+m+mi}{0}\PY{p}{)}
\PY{k}{assert}\PY{p}{(}\PY{n}{P2}\PY{p}{[}\PY{l+m+mi}{2}\PY{p}{]} \PY{o}{==} \PY{l+m+mi}{2}\PY{o}{*}\PY{n}{u1}\PY{p}{)}
\end{Verbatim}
\end{tcolorbox}

    In a similar way, \(M_{(7)}\), \(M_{(10)}\), \(M_{(13)}\), \(_{(16)}\)
can be omitted. Hence we construct the square matrix of order \(10\):

    \begin{tcolorbox}[breakable, size=fbox, boxrule=1pt, pad at break*=1mm,colback=cellbackground, colframe=cellborder]
\prompt{In}{incolor}{23}{\boxspacing}
\begin{Verbatim}[commandchars=\\\{\}]
\PY{n}{MM1} \PY{o}{=} \PY{n}{M}\PY{o}{.}\PY{n}{matrix\PYZus{}from\PYZus{}rows}\PY{p}{(}\PY{p}{[}\PY{l+m+mi}{0}\PY{p}{,} \PY{l+m+mi}{1}\PY{p}{,} \PY{l+m+mi}{4}\PY{p}{,} \PY{l+m+mi}{5}\PY{p}{,} \PY{l+m+mi}{7}\PY{p}{,} \PY{l+m+mi}{8}\PY{p}{,} \PY{l+m+mi}{10}\PY{p}{,} \PY{l+m+mi}{11}\PY{p}{,} \PY{l+m+mi}{13}\PY{p}{,} \PY{l+m+mi}{14}\PY{p}{]}\PY{p}{)}
\end{Verbatim}
\end{tcolorbox}

    The first row of MM1 is \((-3i, 3, 3i, -3, 0, 0, 0, 0, 0, 0)\). The
second row is \((0, 0, 0, 0, 1, i, -1, 0, 0, 0)\)

    \begin{tcolorbox}[breakable, size=fbox, boxrule=1pt, pad at break*=1mm,colback=cellbackground, colframe=cellborder]
\prompt{In}{incolor}{24}{\boxspacing}
\begin{Verbatim}[commandchars=\\\{\}]
\PY{k}{assert}\PY{p}{(}\PY{n}{MM1}\PY{p}{[}\PY{l+m+mi}{0}\PY{p}{]} \PY{o}{==} \PY{n}{vector}\PY{p}{(}\PY{n}{S}\PY{p}{,} \PY{p}{(}\PY{p}{(}\PY{o}{\PYZhy{}}\PY{l+m+mi}{3}\PY{o}{*}\PY{n}{ii}\PY{p}{)}\PY{p}{,} \PY{l+m+mi}{3}\PY{p}{,} \PY{p}{(}\PY{l+m+mi}{3}\PY{o}{*}\PY{n}{ii}\PY{p}{)}\PY{p}{,} \PY{o}{\PYZhy{}}\PY{l+m+mi}{3}\PY{p}{,} \PY{l+m+mi}{0}\PY{p}{,} \PY{l+m+mi}{0}\PY{p}{,} \PY{l+m+mi}{0}\PY{p}{,} \PY{l+m+mi}{0}\PY{p}{,} \PY{l+m+mi}{0}\PY{p}{,} \PY{l+m+mi}{0}\PY{p}{)}\PY{p}{)}\PY{p}{)}
\PY{k}{assert}\PY{p}{(}\PY{n}{MM1}\PY{p}{[}\PY{l+m+mi}{1}\PY{p}{]} \PY{o}{==} \PY{n}{vector}\PY{p}{(}\PY{n}{S}\PY{p}{,} \PY{p}{(}\PY{l+m+mi}{0}\PY{p}{,} \PY{l+m+mi}{0}\PY{p}{,} \PY{l+m+mi}{0}\PY{p}{,} \PY{l+m+mi}{0}\PY{p}{,} \PY{l+m+mi}{1}\PY{p}{,} \PY{n}{ii}\PY{p}{,} \PY{o}{\PYZhy{}}\PY{l+m+mi}{1}\PY{p}{,} \PY{l+m+mi}{0}\PY{p}{,} \PY{l+m+mi}{0}\PY{p}{,} \PY{l+m+mi}{0}\PY{p}{)}\PY{p}{)}\PY{p}{)}
\end{Verbatim}
\end{tcolorbox}

    So with elementary row operations we can simplify MM1:

    \begin{tcolorbox}[breakable, size=fbox, boxrule=1pt, pad at break*=1mm,colback=cellbackground, colframe=cellborder]
\prompt{In}{incolor}{25}{\boxspacing}
\begin{Verbatim}[commandchars=\\\{\}]
\PY{n}{MM1}\PY{o}{.}\PY{n}{rescale\PYZus{}row}\PY{p}{(}\PY{l+m+mi}{0}\PY{p}{,} \PY{l+m+mi}{1}\PY{o}{/}\PY{l+m+mi}{3}\PY{o}{*}\PY{n}{ii}\PY{p}{)}
\PY{k}{for} \PY{n}{i} \PY{o+ow}{in} \PY{n+nb}{range}\PY{p}{(}\PY{l+m+mi}{2}\PY{p}{,} \PY{l+m+mi}{10}\PY{p}{)}\PY{p}{:}
    \PY{n}{MM1}\PY{o}{.}\PY{n}{add\PYZus{}multiple\PYZus{}of\PYZus{}row}\PY{p}{(}\PY{n}{i}\PY{p}{,} \PY{l+m+mi}{0}\PY{p}{,} \PY{o}{\PYZhy{}}\PY{n}{MM1}\PY{p}{[}\PY{n}{i}\PY{p}{]}\PY{p}{[}\PY{l+m+mi}{0}\PY{p}{]}\PY{p}{)}

\PY{k}{for} \PY{n}{i} \PY{o+ow}{in} \PY{n+nb}{range}\PY{p}{(}\PY{l+m+mi}{2}\PY{p}{,} \PY{l+m+mi}{10}\PY{p}{)}\PY{p}{:}
    \PY{n}{MM1}\PY{o}{.}\PY{n}{add\PYZus{}multiple\PYZus{}of\PYZus{}row}\PY{p}{(}\PY{n}{i}\PY{p}{,} \PY{l+m+mi}{1}\PY{p}{,} \PY{o}{\PYZhy{}}\PY{n}{MM1}\PY{p}{[}\PY{n}{i}\PY{p}{]}\PY{p}{[}\PY{l+m+mi}{4}\PY{p}{]}\PY{p}{)}
\end{Verbatim}
\end{tcolorbox}

    Now the 0-th column of MM1 is \((1, 0, \dotsc , 0)\)

and the 4-th column of MM1 is \((0, 1, 0, \dotsc, 0)\)

    \begin{tcolorbox}[breakable, size=fbox, boxrule=1pt, pad at break*=1mm,colback=cellbackground, colframe=cellborder]
\prompt{In}{incolor}{26}{\boxspacing}
\begin{Verbatim}[commandchars=\\\{\}]
\PY{k}{assert}\PY{p}{(}\PY{p}{[}\PY{n}{MM1}\PY{p}{[}\PY{n}{i}\PY{p}{,} \PY{l+m+mi}{0}\PY{p}{]} \PY{k}{for} \PY{n}{i} \PY{o+ow}{in} \PY{n+nb}{range}\PY{p}{(}\PY{l+m+mi}{10}\PY{p}{)}\PY{p}{]} \PY{o}{==} \PY{p}{[}\PY{l+m+mi}{1}\PY{p}{,} \PY{l+m+mi}{0}\PY{p}{,} \PY{l+m+mi}{0}\PY{p}{,} \PY{l+m+mi}{0}\PY{p}{,} \PY{l+m+mi}{0}\PY{p}{,} \PY{l+m+mi}{0}\PY{p}{,} \PY{l+m+mi}{0}\PY{p}{,} \PY{l+m+mi}{0}\PY{p}{,} \PY{l+m+mi}{0}\PY{p}{,} \PY{l+m+mi}{0}\PY{p}{]}\PY{p}{)}
\PY{k}{assert}\PY{p}{(}\PY{p}{[}\PY{n}{MM1}\PY{p}{[}\PY{n}{i}\PY{p}{,} \PY{l+m+mi}{4}\PY{p}{]} \PY{k}{for} \PY{n}{i} \PY{o+ow}{in} \PY{n+nb}{range}\PY{p}{(}\PY{l+m+mi}{10}\PY{p}{)}\PY{p}{]} \PY{o}{==} \PY{p}{[}\PY{l+m+mi}{0}\PY{p}{,} \PY{l+m+mi}{1}\PY{p}{,} \PY{l+m+mi}{0}\PY{p}{,} \PY{l+m+mi}{0}\PY{p}{,} \PY{l+m+mi}{0}\PY{p}{,} \PY{l+m+mi}{0}\PY{p}{,} \PY{l+m+mi}{0}\PY{p}{,} \PY{l+m+mi}{0}\PY{p}{,} \PY{l+m+mi}{0}\PY{p}{,} \PY{l+m+mi}{0}\PY{p}{]}\PY{p}{)}
\end{Verbatim}
\end{tcolorbox}

    We extract from MM1 an order 8 square matrix, extracting the last 8 rows
and the columns of position 1, 2, 3, 5, 6, 7, 8, 9.

We get a new matrix MM1.

This new matrix MM1 has rank \(n\) iff the original MM1 has rank \(n+2\)
(iff M has rank \(n+2\)) In particular, we want to see if MM1 can have
rank \(\leq 6\).

    \begin{tcolorbox}[breakable, size=fbox, boxrule=1pt, pad at break*=1mm,colback=cellbackground, colframe=cellborder]
\prompt{In}{incolor}{27}{\boxspacing}
\begin{Verbatim}[commandchars=\\\{\}]
\PY{n}{MM1} \PY{o}{=} \PY{n}{MM1}\PY{o}{.}\PY{n}{matrix\PYZus{}from\PYZus{}rows\PYZus{}and\PYZus{}columns}\PY{p}{(}
    \PY{p}{[}\PY{l+m+mi}{2}\PY{p}{,} \PY{l+m+mi}{3}\PY{p}{,} \PY{l+m+mi}{4}\PY{p}{,} \PY{l+m+mi}{5}\PY{p}{,} \PY{l+m+mi}{6}\PY{p}{,} \PY{l+m+mi}{7}\PY{p}{,} \PY{l+m+mi}{8}\PY{p}{,} \PY{l+m+mi}{9}\PY{p}{]}\PY{p}{,}
    \PY{p}{[}\PY{l+m+mi}{1}\PY{p}{,} \PY{l+m+mi}{2}\PY{p}{,} \PY{l+m+mi}{3}\PY{p}{,} \PY{l+m+mi}{5}\PY{p}{,} \PY{l+m+mi}{6}\PY{p}{,} \PY{l+m+mi}{7}\PY{p}{,} \PY{l+m+mi}{8}\PY{p}{,} \PY{l+m+mi}{9}\PY{p}{]}
\PY{p}{)}
\end{Verbatim}
\end{tcolorbox}

    \begin{tcolorbox}[breakable, size=fbox, boxrule=1pt, pad at break*=1mm,colback=cellbackground, colframe=cellborder]
\prompt{In}{incolor}{28}{\boxspacing}
\begin{Verbatim}[commandchars=\\\{\}]
\PY{k}{if} \PY{n}{do\PYZus{}long\PYZus{}computations}\PY{p}{:}
   \PY{n}{ttA} \PY{o}{=} \PY{n}{cputime}\PY{p}{(}\PY{p}{)}
   \PY{n}{mm1\PYZus{}7} \PY{o}{=} \PY{n}{MM1}\PY{o}{.}\PY{n}{minors}\PY{p}{(}\PY{l+m+mi}{7}\PY{p}{)}
   \PY{n+nb}{print}\PY{p}{(}\PY{l+s+s2}{\PYZdq{}}\PY{l+s+s2}{Computation of the 64 order 7 minors:}\PY{l+s+s2}{\PYZdq{}}\PY{p}{)}
   \PY{n+nb}{print}\PY{p}{(}\PY{l+s+s2}{\PYZdq{}}\PY{l+s+s2}{time: }\PY{l+s+s2}{\PYZdq{}}\PY{o}{+}\PY{n+nb}{str}\PY{p}{(}\PY{n}{cputime}\PY{p}{(}\PY{p}{)}\PY{o}{\PYZhy{}}\PY{n}{ttA}\PY{p}{)}\PY{p}{)}
   \PY{n}{save}\PY{p}{(}\PY{n}{mm1\PYZus{}7}\PY{p}{,} \PY{l+s+s2}{\PYZdq{}}\PY{l+s+s2}{NB.06.F5\PYZhy{}mm1\PYZus{}7.sobj}\PY{l+s+s2}{\PYZdq{}}\PY{p}{)}
\PY{k}{else}\PY{p}{:}
   \PY{n}{mm1\PYZus{}7} \PY{o}{=} \PY{n}{load}\PY{p}{(}\PY{l+s+s2}{\PYZdq{}}\PY{l+s+s2}{NB.06.F5\PYZhy{}mm1\PYZus{}7.sobj}\PY{l+s+s2}{\PYZdq{}}\PY{p}{)}
\end{Verbatim}
\end{tcolorbox}

    \begin{Verbatim}[commandchars=\\\{\}]
Computation of the 64 order 7 minors:
time: 295.01199800000006
    \end{Verbatim}

    \begin{tcolorbox}[breakable, size=fbox, boxrule=1pt, pad at break*=1mm,colback=cellbackground, colframe=cellborder]
\prompt{In}{incolor}{29}{\boxspacing}
\begin{Verbatim}[commandchars=\\\{\}]
\PY{n}{J7} \PY{o}{=} \PY{n}{S}\PY{o}{.}\PY{n}{ideal}\PY{p}{(}\PY{n}{mm1\PYZus{}7}\PY{p}{)}
\PY{n}{J7} \PY{o}{=} \PY{n}{J7}\PY{o}{.}\PY{n}{saturation}\PY{p}{(}\PY{n}{degenerate\PYZus{}cases}\PY{p}{)}\PY{p}{[}\PY{l+m+mi}{0}\PY{p}{]}
\end{Verbatim}
\end{tcolorbox}

    The above ideal is \((1)\), so it is not possible to have that the
matrix MM1 has rank \(\leq 6\) (hence it is not possible that \(M\) has
rank \(\leq 8\))

    \begin{tcolorbox}[breakable, size=fbox, boxrule=1pt, pad at break*=1mm,colback=cellbackground, colframe=cellborder]
\prompt{In}{incolor}{30}{\boxspacing}
\begin{Verbatim}[commandchars=\\\{\}]
\PY{k}{assert}\PY{p}{(}\PY{n}{J7} \PY{o}{==} \PY{n}{S}\PY{o}{.}\PY{n}{ideal}\PY{p}{(}\PY{n}{S}\PY{p}{(}\PY{l+m+mi}{1}\PY{p}{)}\PY{p}{)}\PY{p}{)}
\end{Verbatim}
\end{tcolorbox}

    In order to have \(P_1\), \(P_2\), \(Pg_1\), \(Pg_2\), \(Pg_3\)
eigenpoints, M must have zero determinant. But \(\det(M) = \det(MM1)\).
So we compute \(\det(MM1)\) and we saturate it w.r.t.
\texttt{degenerate\_cases}.

    \begin{tcolorbox}[breakable, size=fbox, boxrule=1pt, pad at break*=1mm,colback=cellbackground, colframe=cellborder]
\prompt{In}{incolor}{32}{\boxspacing}
\begin{Verbatim}[commandchars=\\\{\}]
\PY{k}{if} \PY{n}{do\PYZus{}long\PYZus{}computations}\PY{p}{:}
   \PY{n}{ttA} \PY{o}{=} \PY{n}{cputime}\PY{p}{(}\PY{p}{)}
   \PY{n}{ddt} \PY{o}{=} \PY{n}{MM1}\PY{o}{.}\PY{n}{det}\PY{p}{(}\PY{p}{)}
   \PY{n}{ddt} \PY{o}{=} \PY{n}{S}\PY{o}{.}\PY{n}{ideal}\PY{p}{(}\PY{n}{ddt}\PY{p}{)}\PY{o}{.}\PY{n}{saturation}\PY{p}{(}\PY{n}{degenerate\PYZus{}cases}\PY{p}{)}\PY{p}{[}\PY{l+m+mi}{0}\PY{p}{]}\PY{o}{.}\PY{n}{gens}\PY{p}{(}\PY{p}{)}\PY{p}{[}\PY{l+m+mi}{0}\PY{p}{]}
   \PY{n+nb}{print}\PY{p}{(}\PY{n}{cputime}\PY{p}{(}\PY{p}{)}\PY{o}{\PYZhy{}}\PY{n}{ttA}\PY{p}{)}
   \PY{n}{sleep}\PY{p}{(}\PY{l+m+mi}{1}\PY{p}{)}
\PY{k}{else}\PY{p}{:}
   \PY{n}{ddt} \PY{o}{=} \PY{p}{(}\PY{l+m+mi}{16}\PY{p}{)} \PY{o}{*} \PY{p}{(}\PY{n}{l1} \PY{o}{+} \PY{l+m+mi}{1}\PY{o}{/}\PY{l+m+mi}{4}\PY{p}{)}\PY{o}{\PYZca{}}\PY{l+m+mi}{2} \PY{o}{*} \PY{p}{(}\PY{n}{u1}\PY{o}{*}\PY{n}{w1}\PY{o}{*}\PY{n}{w2} \PY{o}{+} \PY{n}{v1}\PY{o}{*}\PY{n}{w1}\PY{o}{*}\PY{n}{w2} \PY{o}{+} \PY{n}{w1} \PY{o}{+} \PY{n}{w2}\PY{p}{)}

\PY{k}{assert}\PY{p}{(}\PY{n}{ddt} \PY{o}{==} \PY{p}{(}\PY{l+m+mi}{16}\PY{p}{)} \PY{o}{*} \PY{p}{(}\PY{n}{l1} \PY{o}{+} \PY{l+m+mi}{1}\PY{o}{/}\PY{l+m+mi}{4}\PY{p}{)}\PY{o}{\PYZca{}}\PY{l+m+mi}{2} \PY{o}{*} \PY{p}{(}\PY{n}{u1}\PY{o}{*}\PY{n}{w1}\PY{o}{*}\PY{n}{w2} \PY{o}{+} \PY{n}{v1}\PY{o}{*}\PY{n}{w1}\PY{o}{*}\PY{n}{w2} \PY{o}{+} \PY{n}{w1} \PY{o}{+} \PY{n}{w2}\PY{p}{)}\PY{p}{)}
\end{Verbatim}
\end{tcolorbox}

    \begin{Verbatim}[commandchars=\\\{\}]
0.020036000000004606
    \end{Verbatim}

    Hence we have two possibilities: * \(l_1 + 1/4 = 0\), or *
\(u_1 w_1 w_2 + v_1 w_1 w_2 + w_1 + w_2 = 0\).

    This second condition must be satisfied for every point \(Pg_2\) of the
conic \(C_g\), i.e.~for every \(w_2\), therefore this condition is
impossible.

    Hence the only possible case is \(l_1 = -1/4\). In this case \(C_g\)
splits into two lines: * the line \(x + i y\) * the line
\(x u_1 v_1 + i y u_1 v_1 + i z u_1 + i z v_1 + x - i y\)

The first line is the tangent line to \(\mathcal{Q}_{\mathrm{iso}}\) in
the point \(P_1\) and the second line is the line passing through the
points \(P_2\) and \(P_3\):

    \begin{tcolorbox}[breakable, size=fbox, boxrule=1pt, pad at break*=1mm,colback=cellbackground, colframe=cellborder]
\prompt{In}{incolor}{33}{\boxspacing}
\begin{Verbatim}[commandchars=\\\{\}]
\PY{k}{assert}\PY{p}{(}\PY{n}{S}\PY{o}{.}\PY{n}{ideal}\PY{p}{(}\PY{n}{Ciso}\PY{p}{,} \PY{n}{x}\PY{o}{+}\PY{n}{ii}\PY{o}{*}\PY{n}{y}\PY{p}{)}\PY{o}{.}\PY{n}{groebner\PYZus{}basis}\PY{p}{(}\PY{p}{)} \PY{o}{==} \PY{p}{[}\PY{n}{z}\PY{o}{\PYZca{}}\PY{l+m+mi}{2}\PY{p}{,} \PY{n}{x} \PY{o}{+} \PY{n}{ii}\PY{o}{*}\PY{n}{y}\PY{p}{]}\PY{p}{)}

\PY{k}{assert}\PY{p}{(}
    \PY{n}{Cg}\PY{o}{.}\PY{n}{subs}\PY{p}{(}\PY{n}{l1}\PY{o}{=}\PY{o}{\PYZhy{}}\PY{l+m+mi}{1}\PY{o}{/}\PY{l+m+mi}{4}\PY{p}{)} \PY{o}{==}
    \PY{p}{(}\PY{n}{x}\PY{o}{+}\PY{n}{ii}\PY{o}{*}\PY{n}{y}\PY{p}{)}\PY{o}{*}\PY{n}{S}\PY{p}{(}\PY{n}{matrix}\PY{p}{(}\PY{p}{[}\PY{n}{P2}\PY{p}{,} \PY{n}{P3}\PY{p}{,} \PY{p}{(}\PY{n}{x}\PY{p}{,} \PY{n}{y}\PY{p}{,} \PY{n}{z}\PY{p}{)}\PY{p}{]}\PY{p}{)}\PY{o}{.}\PY{n}{det}\PY{p}{(}\PY{p}{)}\PY{o}{/}\PY{p}{(}\PY{l+m+mi}{2}\PY{o}{*}\PY{p}{(}\PY{n}{u1}\PY{o}{\PYZhy{}}\PY{n}{v1}\PY{p}{)}\PY{p}{)}\PY{p}{)}
\PY{p}{)}
\end{Verbatim}
\end{tcolorbox}

    The matrix
\texttt{M.matrix\_from\_rows({[}0,\ 1,\ 4,\ 5,\ 7,\ 8,\ 10,\ 11,\ 13,\ 14{]})}
i.e., the original matrix extracted from \(M\), with the condition
\(l_1=-1/4\) has rank \(9\):

    \begin{tcolorbox}[breakable, size=fbox, boxrule=1pt, pad at break*=1mm,colback=cellbackground, colframe=cellborder]
\prompt{In}{incolor}{34}{\boxspacing}
\begin{Verbatim}[commandchars=\\\{\}]
\PY{n}{MM2} \PY{o}{=} \PY{n}{M}\PY{o}{.}\PY{n}{matrix\PYZus{}from\PYZus{}rows}\PY{p}{(}\PY{p}{[}\PY{l+m+mi}{0}\PY{p}{,} \PY{l+m+mi}{1}\PY{p}{,} \PY{l+m+mi}{4}\PY{p}{,} \PY{l+m+mi}{5}\PY{p}{,} \PY{l+m+mi}{7}\PY{p}{,} \PY{l+m+mi}{8}\PY{p}{,} \PY{l+m+mi}{10}\PY{p}{,} \PY{l+m+mi}{11}\PY{p}{,} \PY{l+m+mi}{13}\PY{p}{,} \PY{l+m+mi}{14}\PY{p}{]}\PY{p}{)}\PY{o}{.}\PY{n}{subs}\PY{p}{(}\PY{n}{l1} \PY{o}{=} \PY{o}{\PYZhy{}}\PY{l+m+mi}{1}\PY{o}{/}\PY{l+m+mi}{4}\PY{p}{)}
\PY{k}{assert}\PY{p}{(}\PY{n}{MM2}\PY{o}{.}\PY{n}{rank}\PY{p}{(}\PY{p}{)} \PY{o}{==} \PY{l+m+mi}{9}\PY{p}{)}
\end{Verbatim}
\end{tcolorbox}

    and the last row of MM2 is linearly dependent w.r.t. the other rows of
MM2. Indeed, the rank of MM2 without the last row is still 9:

    \begin{tcolorbox}[breakable, size=fbox, boxrule=1pt, pad at break*=1mm,colback=cellbackground, colframe=cellborder]
\prompt{In}{incolor}{35}{\boxspacing}
\begin{Verbatim}[commandchars=\\\{\}]
\PY{n}{MM3} \PY{o}{=} \PY{n}{MM2}\PY{o}{.}\PY{n}{matrix\PYZus{}from\PYZus{}rows}\PY{p}{(}\PY{p}{[}\PY{l+m+mi}{0}\PY{p}{,} \PY{l+m+mi}{1}\PY{p}{,} \PY{l+m+mi}{2}\PY{p}{,} \PY{l+m+mi}{3}\PY{p}{,} \PY{l+m+mi}{4}\PY{p}{,} \PY{l+m+mi}{5}\PY{p}{,} \PY{l+m+mi}{6}\PY{p}{,} \PY{l+m+mi}{7}\PY{p}{,} \PY{l+m+mi}{8}\PY{p}{]}\PY{p}{)}
\PY{k}{assert}\PY{p}{(}\PY{n}{MM3}\PY{o}{.}\PY{n}{rank}\PY{p}{(}\PY{p}{)} \PY{o}{==} \PY{l+m+mi}{9}\PY{p}{)}
\end{Verbatim}
\end{tcolorbox}

    hence the cubic corresponding to the matrix MM2 is:

    \begin{tcolorbox}[breakable, size=fbox, boxrule=1pt, pad at break*=1mm,colback=cellbackground, colframe=cellborder]
\prompt{In}{incolor}{39}{\boxspacing}
\begin{Verbatim}[commandchars=\\\{\}]
\PY{k}{if} \PY{n}{do\PYZus{}long\PYZus{}computations}\PY{p}{:}
    \PY{n}{ttA} \PY{o}{=} \PY{n}{cputime}\PY{p}{(}\PY{p}{)}
    \PY{n}{cbb} \PY{o}{=} \PY{n}{det}\PY{p}{(}\PY{n}{MM3}\PY{o}{.}\PY{n}{stack}\PY{p}{(}\PY{n}{matrix}\PY{p}{(}\PY{p}{[}\PY{n}{mon}\PY{p}{]}\PY{p}{)}\PY{p}{)}\PY{p}{)}
    \PY{n+nb}{print}\PY{p}{(}\PY{n+nb}{str}\PY{p}{(}\PY{n}{cputime}\PY{p}{(}\PY{p}{)}\PY{o}{\PYZhy{}}\PY{n}{ttA}\PY{p}{)} \PY{o}{+} \PY{l+s+s2}{\PYZdq{}}\PY{l+s+s2}{ seconds of computation}\PY{l+s+s2}{\PYZdq{}}\PY{p}{)}
    \PY{n}{sleep}\PY{p}{(}\PY{l+m+mi}{1}\PY{p}{)}
    \PY{n}{ttA} \PY{o}{=} \PY{n}{cputime}\PY{p}{(}\PY{p}{)}
    \PY{n}{sleep}\PY{p}{(}\PY{l+m+mi}{1}\PY{p}{)}
    \PY{n}{cbb} \PY{o}{=} \PY{n}{S}\PY{o}{.}\PY{n}{ideal}\PY{p}{(}\PY{n}{cbb}\PY{p}{)}\PY{o}{.}\PY{n}{saturation}\PY{p}{(}\PY{n}{degenerate\PYZus{}cases}\PY{p}{)}\PY{p}{[}\PY{l+m+mi}{0}\PY{p}{]}\PY{o}{.}\PY{n}{gens}\PY{p}{(}\PY{p}{)}\PY{p}{[}\PY{l+m+mi}{0}\PY{p}{]}
    \PY{n+nb}{print}\PY{p}{(}\PY{n+nb}{str}\PY{p}{(}\PY{n}{cputime}\PY{p}{(}\PY{p}{)}\PY{o}{\PYZhy{}}\PY{n}{ttA}\PY{p}{)} \PY{o}{+} \PY{l+s+s2}{\PYZdq{}}\PY{l+s+s2}{ seconds of computation}\PY{l+s+s2}{\PYZdq{}}\PY{p}{)}
\PY{k}{else}\PY{p}{:}
    \PY{n}{cbb} \PY{o}{=} \PY{p}{(}\PY{n}{x} \PY{o}{+} \PY{n}{ii}\PY{o}{*}\PY{n}{y}\PY{p}{)} \PY{o}{*} \PY{p}{(}\PY{n}{u1}\PY{o}{*}\PY{n}{v1} \PY{o}{\PYZhy{}} \PY{l+m+mi}{1}\PY{p}{)} \PY{o}{*} \PY{p}{(}\PY{n}{x}\PY{o}{*}\PY{n}{u1}\PY{o}{*}\PY{n}{v1} \PY{o}{+} \PY{n}{ii}\PY{o}{*}\PY{n}{y}\PY{o}{*}\PY{n}{u1}\PY{o}{*}\PY{n}{v1} \PY{o}{+} \PY{n}{ii}\PY{o}{*}\PY{n}{z}\PY{o}{*}\PY{n}{u1} \PY{o}{+} \PY{n}{ii}\PY{o}{*}\PY{n}{z}\PY{o}{*}\PY{n}{v1} \PY{o}{+} \PY{n}{x} \PY{o}{+} \PY{p}{(}\PY{o}{\PYZhy{}}\PY{n}{ii}\PY{p}{)}\PY{o}{*}\PY{n}{y}\PY{p}{)}\PY{o}{\PYZca{}}\PY{l+m+mi}{2}

\PY{k}{assert}\PY{p}{(}\PY{n}{cbb} \PY{o}{==} \PY{p}{(}\PY{n}{x} \PY{o}{+} \PY{n}{ii}\PY{o}{*}\PY{n}{y}\PY{p}{)} \PY{o}{*} \PY{p}{(}\PY{n}{u1}\PY{o}{*}\PY{n}{v1} \PY{o}{\PYZhy{}} \PY{l+m+mi}{1}\PY{p}{)} \PY{o}{*} \PY{p}{(}\PY{n}{x}\PY{o}{*}\PY{n}{u1}\PY{o}{*}\PY{n}{v1} \PY{o}{+} \PY{n}{ii}\PY{o}{*}\PY{n}{y}\PY{o}{*}\PY{n}{u1}\PY{o}{*}\PY{n}{v1} \PY{o}{+} \PY{n}{ii}\PY{o}{*}\PY{n}{z}\PY{o}{*}\PY{n}{u1} \PY{o}{+} \PY{n}{ii}\PY{o}{*}\PY{n}{z}\PY{o}{*}\PY{n}{v1} \PY{o}{+} \PY{n}{x} \PY{o}{+} \PY{p}{(}\PY{o}{\PYZhy{}}\PY{n}{ii}\PY{p}{)}\PY{o}{*}\PY{n}{y}\PY{p}{)}\PY{o}{\PYZca{}}\PY{l+m+mi}{2}\PY{p}{)}
\end{Verbatim}
\end{tcolorbox}

    \begin{Verbatim}[commandchars=\\\{\}]
40.211033999999984 seconds of computation
10.736583999999937 seconds of computation
    \end{Verbatim}

    \texttt{cbb} is the cubic obtained from the matrix MM3, hence is the
unique cubic that has \(P_1, P_2, P_3\), \(Pg_1\), \(Pg_2\) as
eigenpoints and \texttt{cbb} splits into the line \((P_1 \vee P_2)^2\)
and the tangent to \(\mathcal{Q}_{\mathrm{iso}}\) in the point \(P_1\).
The eigenpoints are the line \(P_1 \vee P_2\) plus other points and is
not a conic.

This conclude the case \(\Gamma\) tangent to
\(\mathcal{Q}_{\mathrm{iso}}\) in one point in case \(P_2\) and \(P_3\)
are different from the point \((1: -i: 0)\).

    \hypertarget{case-p_3-1--i-0}{%
\subsubsection{\texorpdfstring{Case
\(P_3 = (1: -i: 0)\)}{Case P\_3 = (1: -i: 0)}}\label{case-p_3-1--i-0}}

    \begin{tcolorbox}[breakable, size=fbox, boxrule=1pt, pad at break*=1mm,colback=cellbackground, colframe=cellborder]
\prompt{In}{incolor}{40}{\boxspacing}
\begin{Verbatim}[commandchars=\\\{\}]
\PY{n}{P3} \PY{o}{=} \PY{n}{ii}\PY{o}{*}\PY{n}{P3}\PY{o}{.}\PY{n}{subs}\PY{p}{(}\PY{n}{v1}\PY{o}{=}\PY{l+m+mi}{0}\PY{p}{)}
\PY{k}{assert}\PY{p}{(}\PY{n}{P3} \PY{o}{==} \PY{n}{vector}\PY{p}{(}\PY{n}{S}\PY{p}{,} \PY{p}{(}\PY{l+m+mi}{1}\PY{p}{,} \PY{o}{\PYZhy{}}\PY{n}{ii}\PY{p}{,} \PY{l+m+mi}{0}\PY{p}{)}\PY{p}{)}\PY{p}{)}
\end{Verbatim}
\end{tcolorbox}

    We construct the conic given by two lines \(P_1 \vee P_2\) and
\(P_1 \vee P_3\) and the generic conic tangent to
\(\mathcal{Q}_{\mathrm{iso}}\) in \(P_1\) and passing through \(P_2\)
and \(P_3\) (it depends on the parameter \(l_1\)):

    \begin{tcolorbox}[breakable, size=fbox, boxrule=1pt, pad at break*=1mm,colback=cellbackground, colframe=cellborder]
\prompt{In}{incolor}{41}{\boxspacing}
\begin{Verbatim}[commandchars=\\\{\}]
\PY{n}{c2} \PY{o}{=} \PY{n}{matrix}\PY{p}{(}\PY{p}{[}\PY{n}{P1}\PY{p}{,} \PY{n}{P2}\PY{p}{,} \PY{p}{[}\PY{n}{x}\PY{p}{,} \PY{n}{y}\PY{p}{,} \PY{n}{z}\PY{p}{]}\PY{p}{]}\PY{p}{)}\PY{o}{.}\PY{n}{det}\PY{p}{(}\PY{p}{)}\PY{o}{*}\PY{n}{matrix}\PY{p}{(}\PY{p}{[}\PY{n}{P1}\PY{p}{,} \PY{n}{P3}\PY{p}{,} \PY{p}{[}\PY{n}{x}\PY{p}{,} \PY{n}{y}\PY{p}{,} \PY{n}{z}\PY{p}{]}\PY{p}{]}\PY{p}{)}\PY{o}{.}\PY{n}{det}\PY{p}{(}\PY{p}{)}

\PY{n}{Cg} \PY{o}{=} \PY{n}{Ciso}\PY{o}{+}\PY{n}{l1}\PY{o}{*}\PY{n}{c2}
\end{Verbatim}
\end{tcolorbox}

    Construction of a generic point (different from \((1, i, 0)\)) on
\(C_g\) and generic point of \(C_g\) (depends on the parameter \(w_1\))
and construction of \(Pg_1\) and \(Pg_2\) (two points on \(C_g\))

    \begin{tcolorbox}[breakable, size=fbox, boxrule=1pt, pad at break*=1mm,colback=cellbackground, colframe=cellborder]
\prompt{In}{incolor}{42}{\boxspacing}
\begin{Verbatim}[commandchars=\\\{\}]
\PY{n}{foo} \PY{o}{=} \PY{n}{Cg}\PY{o}{.}\PY{n}{subs}\PY{p}{(}\PY{n}{y}\PY{o}{=}\PY{n}{ii}\PY{o}{*}\PY{n}{x}\PY{o}{+}\PY{n}{w1}\PY{o}{*}\PY{n}{z}\PY{p}{)}\PY{o}{.}\PY{n}{factor}\PY{p}{(}\PY{p}{)}\PY{p}{[}\PY{o}{\PYZhy{}}\PY{l+m+mi}{1}\PY{p}{]}\PY{p}{[}\PY{l+m+mi}{0}\PY{p}{]}

\PY{n}{Pg} \PY{o}{=} \PY{n}{vector}\PY{p}{(}
    \PY{n}{S}\PY{p}{,} 
    \PY{p}{(}
        \PY{n}{foo}\PY{o}{.}\PY{n}{coefficient}\PY{p}{(}\PY{n}{z}\PY{p}{)}\PY{p}{,}
        \PY{n}{ii}\PY{o}{*}\PY{p}{(}\PY{n}{foo}\PY{o}{.}\PY{n}{coefficient}\PY{p}{(}\PY{n}{z}\PY{p}{)}\PY{p}{)} \PY{o}{+} \PY{n}{w1}\PY{o}{*}\PY{p}{(}\PY{o}{\PYZhy{}}\PY{n}{foo}\PY{o}{.}\PY{n}{coefficient}\PY{p}{(}\PY{n}{x}\PY{p}{)}\PY{p}{)}\PY{p}{,}
        \PY{o}{\PYZhy{}}\PY{n}{foo}\PY{o}{.}\PY{n}{coefficient}\PY{p}{(}\PY{n}{x}\PY{p}{)}
    \PY{p}{)}
\PY{p}{)}
\end{Verbatim}
\end{tcolorbox}

    \begin{tcolorbox}[breakable, size=fbox, boxrule=1pt, pad at break*=1mm,colback=cellbackground, colframe=cellborder]
\prompt{In}{incolor}{43}{\boxspacing}
\begin{Verbatim}[commandchars=\\\{\}]
\PY{n}{Pg1}\PY{p}{,} \PY{n}{Pg2} \PY{o}{=} \PY{n}{Pg}\PY{o}{.}\PY{n}{subs}\PY{p}{(}\PY{p}{\PYZob{}}\PY{n}{w1}\PY{p}{:}\PY{n}{w1}\PY{p}{\PYZcb{}}\PY{p}{)}\PY{p}{,} \PY{n}{Pg}\PY{o}{.}\PY{n}{subs}\PY{p}{(}\PY{p}{\PYZob{}}\PY{n}{w1}\PY{p}{:}\PY{n}{w2}\PY{p}{\PYZcb{}}\PY{p}{)}
\end{Verbatim}
\end{tcolorbox}

    Matrix of conditions of \(P_1\), \(P_2\), \(P_3\), \(Pg_1\), \(Pg_2\):

    \begin{tcolorbox}[breakable, size=fbox, boxrule=1pt, pad at break*=1mm,colback=cellbackground, colframe=cellborder]
\prompt{In}{incolor}{44}{\boxspacing}
\begin{Verbatim}[commandchars=\\\{\}]
\PY{n}{M} \PY{o}{=} \PY{n}{condition\PYZus{}matrix}\PY{p}{(}\PY{p}{[}\PY{n}{P1}\PY{p}{,} \PY{n}{P2}\PY{p}{,} \PY{n}{P3}\PY{p}{,} \PY{n}{Pg1}\PY{p}{,} \PY{n}{Pg2}\PY{p}{]}\PY{p}{,} \PY{n}{S}\PY{p}{,} \PY{n}{standard}\PY{o}{=}\PY{l+s+s2}{\PYZdq{}}\PY{l+s+s2}{all}\PY{l+s+s2}{\PYZdq{}}\PY{p}{)}
\end{Verbatim}
\end{tcolorbox}

    We have P1{[}2{]}\emph{M{[}0{]}-P1{[}1{]}}M{[}1{]}+P1{[}0{]}*M{[}2{]} =
0 and P1{[}0{]} = 1 so M{[}2{]} is lin dep of M{[}0{]} and M{[}1{]} and
can be omitted.

We have
P2{[}2{]}\emph{M{[}3{]}-P2{[}1{]}}M{[}4{]}+P2{[}0{]}\emph{M{[}5{]} = 0
and P2{[}2{]} = 2}u1 which, under our hypothesis, is always not zero, so
M{[}3{]} can be omitted.

    \begin{tcolorbox}[breakable, size=fbox, boxrule=1pt, pad at break*=1mm,colback=cellbackground, colframe=cellborder]
\prompt{In}{incolor}{45}{\boxspacing}
\begin{Verbatim}[commandchars=\\\{\}]
\PY{k}{assert}\PY{p}{(}\PY{n}{P1}\PY{p}{[}\PY{l+m+mi}{2}\PY{p}{]}\PY{o}{*}\PY{n}{M}\PY{p}{[}\PY{l+m+mi}{0}\PY{p}{]}\PY{o}{\PYZhy{}}\PY{n}{P1}\PY{p}{[}\PY{l+m+mi}{1}\PY{p}{]}\PY{o}{*}\PY{n}{M}\PY{p}{[}\PY{l+m+mi}{1}\PY{p}{]}\PY{o}{+}\PY{n}{P1}\PY{p}{[}\PY{l+m+mi}{0}\PY{p}{]}\PY{o}{*}\PY{n}{M}\PY{p}{[}\PY{l+m+mi}{2}\PY{p}{]} \PY{o}{==} \PY{l+m+mi}{0}\PY{p}{)}
\PY{k}{assert}\PY{p}{(}\PY{n}{P1}\PY{p}{[}\PY{l+m+mi}{0}\PY{p}{]} \PY{o}{==} \PY{l+m+mi}{1}\PY{p}{)}
\PY{k}{assert}\PY{p}{(} \PY{n}{P2}\PY{p}{[}\PY{l+m+mi}{2}\PY{p}{]}\PY{o}{*}\PY{n}{M}\PY{p}{[}\PY{l+m+mi}{3}\PY{p}{]}\PY{o}{\PYZhy{}}\PY{n}{P2}\PY{p}{[}\PY{l+m+mi}{1}\PY{p}{]}\PY{o}{*}\PY{n}{M}\PY{p}{[}\PY{l+m+mi}{4}\PY{p}{]}\PY{o}{+}\PY{n}{P2}\PY{p}{[}\PY{l+m+mi}{0}\PY{p}{]}\PY{o}{*}\PY{n}{M}\PY{p}{[}\PY{l+m+mi}{5}\PY{p}{]} \PY{o}{==} \PY{l+m+mi}{0}\PY{p}{)}
\PY{k}{assert}\PY{p}{(}\PY{n}{P2}\PY{p}{[}\PY{l+m+mi}{2}\PY{p}{]} \PY{o}{==} \PY{l+m+mi}{2}\PY{o}{*}\PY{n}{u1}\PY{p}{)}
\end{Verbatim}
\end{tcolorbox}

    In a similar way, M{[}9{]}, M{[}12{]}, M{[}15{]} can be omitted.

P3{[}2{]}\emph{M{[}6{]}-P3{[}1{]}}M{[}7{]}+P3{[}0{]}*M{[}8{]} = 0 and
P3{[}0{]} = 1, hence M{[}8{]} can be omitted.

Hence we construct the square matrix of order 10:

    \begin{tcolorbox}[breakable, size=fbox, boxrule=1pt, pad at break*=1mm,colback=cellbackground, colframe=cellborder]
\prompt{In}{incolor}{46}{\boxspacing}
\begin{Verbatim}[commandchars=\\\{\}]
\PY{n}{MM1} \PY{o}{=} \PY{n}{M}\PY{o}{.}\PY{n}{matrix\PYZus{}from\PYZus{}rows}\PY{p}{(}\PY{p}{[}\PY{l+m+mi}{0}\PY{p}{,} \PY{l+m+mi}{1}\PY{p}{,} \PY{l+m+mi}{4}\PY{p}{,} \PY{l+m+mi}{5}\PY{p}{,} \PY{l+m+mi}{6}\PY{p}{,} \PY{l+m+mi}{7}\PY{p}{,} \PY{l+m+mi}{10}\PY{p}{,} \PY{l+m+mi}{11}\PY{p}{,} \PY{l+m+mi}{13}\PY{p}{,} \PY{l+m+mi}{14}\PY{p}{]}\PY{p}{)}
\end{Verbatim}
\end{tcolorbox}

    The first row of MM1 is ((-3\emph{ii), 3, (3}ii), -3, 0, 0, 0, 0, 0, 0)

The second row is (0, 0, 0, 0, 1, ii, -1, 0, 0, 0)

    \begin{tcolorbox}[breakable, size=fbox, boxrule=1pt, pad at break*=1mm,colback=cellbackground, colframe=cellborder]
\prompt{In}{incolor}{47}{\boxspacing}
\begin{Verbatim}[commandchars=\\\{\}]
\PY{k}{assert}\PY{p}{(}\PY{n}{MM1}\PY{p}{[}\PY{l+m+mi}{0}\PY{p}{]} \PY{o}{==} \PY{n}{vector}\PY{p}{(}\PY{n}{S}\PY{p}{,} \PY{p}{(}\PY{p}{(}\PY{o}{\PYZhy{}}\PY{l+m+mi}{3}\PY{o}{*}\PY{n}{ii}\PY{p}{)}\PY{p}{,} \PY{l+m+mi}{3}\PY{p}{,} \PY{p}{(}\PY{l+m+mi}{3}\PY{o}{*}\PY{n}{ii}\PY{p}{)}\PY{p}{,} \PY{o}{\PYZhy{}}\PY{l+m+mi}{3}\PY{p}{,} \PY{l+m+mi}{0}\PY{p}{,} \PY{l+m+mi}{0}\PY{p}{,} \PY{l+m+mi}{0}\PY{p}{,} \PY{l+m+mi}{0}\PY{p}{,} \PY{l+m+mi}{0}\PY{p}{,} \PY{l+m+mi}{0}\PY{p}{)}\PY{p}{)}\PY{p}{)}
\PY{k}{assert}\PY{p}{(}\PY{n}{MM1}\PY{p}{[}\PY{l+m+mi}{1}\PY{p}{]} \PY{o}{==} \PY{n}{vector}\PY{p}{(}\PY{n}{S}\PY{p}{,} \PY{p}{(}\PY{l+m+mi}{0}\PY{p}{,} \PY{l+m+mi}{0}\PY{p}{,} \PY{l+m+mi}{0}\PY{p}{,} \PY{l+m+mi}{0}\PY{p}{,} \PY{l+m+mi}{1}\PY{p}{,} \PY{n}{ii}\PY{p}{,} \PY{o}{\PYZhy{}}\PY{l+m+mi}{1}\PY{p}{,} \PY{l+m+mi}{0}\PY{p}{,} \PY{l+m+mi}{0}\PY{p}{,} \PY{l+m+mi}{0}\PY{p}{)}\PY{p}{)}\PY{p}{)}
\end{Verbatim}
\end{tcolorbox}

    so with elementary row operations we can simplify MM1:

    \begin{tcolorbox}[breakable, size=fbox, boxrule=1pt, pad at break*=1mm,colback=cellbackground, colframe=cellborder]
\prompt{In}{incolor}{48}{\boxspacing}
\begin{Verbatim}[commandchars=\\\{\}]
\PY{n}{MM1}\PY{o}{.}\PY{n}{rescale\PYZus{}row}\PY{p}{(}\PY{l+m+mi}{0}\PY{p}{,} \PY{l+m+mi}{1}\PY{o}{/}\PY{l+m+mi}{3}\PY{o}{*}\PY{n}{ii}\PY{p}{)}
\PY{k}{for} \PY{n}{i} \PY{o+ow}{in} \PY{n+nb}{range}\PY{p}{(}\PY{l+m+mi}{2}\PY{p}{,} \PY{l+m+mi}{10}\PY{p}{)}\PY{p}{:}
    \PY{n}{MM1}\PY{o}{.}\PY{n}{add\PYZus{}multiple\PYZus{}of\PYZus{}row}\PY{p}{(}\PY{n}{i}\PY{p}{,} \PY{l+m+mi}{0}\PY{p}{,} \PY{o}{\PYZhy{}}\PY{n}{MM1}\PY{p}{[}\PY{n}{i}\PY{p}{]}\PY{p}{[}\PY{l+m+mi}{0}\PY{p}{]}\PY{p}{)}

\PY{k}{for} \PY{n}{i} \PY{o+ow}{in} \PY{n+nb}{range}\PY{p}{(}\PY{l+m+mi}{2}\PY{p}{,} \PY{l+m+mi}{10}\PY{p}{)}\PY{p}{:}
    \PY{n}{MM1}\PY{o}{.}\PY{n}{add\PYZus{}multiple\PYZus{}of\PYZus{}row}\PY{p}{(}\PY{n}{i}\PY{p}{,} \PY{l+m+mi}{1}\PY{p}{,} \PY{o}{\PYZhy{}}\PY{n}{MM1}\PY{p}{[}\PY{n}{i}\PY{p}{]}\PY{p}{[}\PY{l+m+mi}{4}\PY{p}{]}\PY{p}{)}
\end{Verbatim}
\end{tcolorbox}

    Now the 0-th column of MM1 is (1, 0, \ldots, 0) and the 4-th column of
MM1 is (0, 1, 0, \ldots, 0)

    \begin{tcolorbox}[breakable, size=fbox, boxrule=1pt, pad at break*=1mm,colback=cellbackground, colframe=cellborder]
\prompt{In}{incolor}{49}{\boxspacing}
\begin{Verbatim}[commandchars=\\\{\}]
\PY{k}{assert}\PY{p}{(}\PY{p}{[}\PY{n}{MM1}\PY{p}{[}\PY{n}{i}\PY{p}{,} \PY{l+m+mi}{0}\PY{p}{]} \PY{k}{for} \PY{n}{i} \PY{o+ow}{in} \PY{n+nb}{range}\PY{p}{(}\PY{l+m+mi}{10}\PY{p}{)}\PY{p}{]} \PY{o}{==} \PY{p}{[}\PY{l+m+mi}{1}\PY{p}{,} \PY{l+m+mi}{0}\PY{p}{,} \PY{l+m+mi}{0}\PY{p}{,} \PY{l+m+mi}{0}\PY{p}{,} \PY{l+m+mi}{0}\PY{p}{,} \PY{l+m+mi}{0}\PY{p}{,} \PY{l+m+mi}{0}\PY{p}{,} \PY{l+m+mi}{0}\PY{p}{,} \PY{l+m+mi}{0}\PY{p}{,} \PY{l+m+mi}{0}\PY{p}{]}\PY{p}{)}
\PY{k}{assert}\PY{p}{(}\PY{p}{[}\PY{n}{MM1}\PY{p}{[}\PY{n}{i}\PY{p}{,} \PY{l+m+mi}{4}\PY{p}{]} \PY{k}{for} \PY{n}{i} \PY{o+ow}{in} \PY{n+nb}{range}\PY{p}{(}\PY{l+m+mi}{10}\PY{p}{)}\PY{p}{]} \PY{o}{==} \PY{p}{[}\PY{l+m+mi}{0}\PY{p}{,} \PY{l+m+mi}{1}\PY{p}{,} \PY{l+m+mi}{0}\PY{p}{,} \PY{l+m+mi}{0}\PY{p}{,} \PY{l+m+mi}{0}\PY{p}{,} \PY{l+m+mi}{0}\PY{p}{,} \PY{l+m+mi}{0}\PY{p}{,} \PY{l+m+mi}{0}\PY{p}{,} \PY{l+m+mi}{0}\PY{p}{,} \PY{l+m+mi}{0}\PY{p}{]}\PY{p}{)}
\end{Verbatim}
\end{tcolorbox}

    We extract from MM1 an order 8 square matrix, extracting the last 8 rows
and the columns of position 1, 2, 3, 5, 6, 7, 8, 9.

    \begin{tcolorbox}[breakable, size=fbox, boxrule=1pt, pad at break*=1mm,colback=cellbackground, colframe=cellborder]
\prompt{In}{incolor}{50}{\boxspacing}
\begin{Verbatim}[commandchars=\\\{\}]
\PY{n}{MM1} \PY{o}{=} \PY{n}{MM1}\PY{o}{.}\PY{n}{matrix\PYZus{}from\PYZus{}rows\PYZus{}and\PYZus{}columns}\PY{p}{(}
    \PY{p}{[}\PY{l+m+mi}{2}\PY{p}{,} \PY{l+m+mi}{3}\PY{p}{,} \PY{l+m+mi}{4}\PY{p}{,} \PY{l+m+mi}{5}\PY{p}{,} \PY{l+m+mi}{6}\PY{p}{,} \PY{l+m+mi}{7}\PY{p}{,} \PY{l+m+mi}{8}\PY{p}{,} \PY{l+m+mi}{9}\PY{p}{]}\PY{p}{,}
    \PY{p}{[}\PY{l+m+mi}{1}\PY{p}{,} \PY{l+m+mi}{2}\PY{p}{,} \PY{l+m+mi}{3}\PY{p}{,} \PY{l+m+mi}{5}\PY{p}{,} \PY{l+m+mi}{6}\PY{p}{,} \PY{l+m+mi}{7}\PY{p}{,} \PY{l+m+mi}{8}\PY{p}{,} \PY{l+m+mi}{9}\PY{p}{]}
\PY{p}{)}
\end{Verbatim}
\end{tcolorbox}

    This new matrix MM1 has rank n iff the original MM1 has rank n+2 (iff M
has rank n+2) In particular, we want to see if MM1 can have rank
\textless= 6.

    \begin{tcolorbox}[breakable, size=fbox, boxrule=1pt, pad at break*=1mm,colback=cellbackground, colframe=cellborder]
\prompt{In}{incolor}{51}{\boxspacing}
\begin{Verbatim}[commandchars=\\\{\}]
\PY{k}{if} \PY{n}{do\PYZus{}long\PYZus{}computations}\PY{p}{:}
   \PY{n}{ttA} \PY{o}{=} \PY{n}{cputime}\PY{p}{(}\PY{p}{)}
   \PY{n}{mm1\PYZus{}7b} \PY{o}{=} \PY{n}{MM1}\PY{o}{.}\PY{n}{minors}\PY{p}{(}\PY{l+m+mi}{7}\PY{p}{)}
   \PY{n+nb}{print}\PY{p}{(}\PY{l+s+s2}{\PYZdq{}}\PY{l+s+s2}{Computation of the 64 order 7 minors:}\PY{l+s+s2}{\PYZdq{}}\PY{p}{)}
   \PY{n}{sleep}\PY{p}{(}\PY{l+m+mi}{1}\PY{p}{)}
   \PY{n+nb}{print}\PY{p}{(}\PY{l+s+s2}{\PYZdq{}}\PY{l+s+s2}{time: }\PY{l+s+s2}{\PYZdq{}} \PY{o}{+} \PY{n+nb}{str}\PY{p}{(}\PY{n}{cputime}\PY{p}{(}\PY{p}{)}\PY{o}{\PYZhy{}}\PY{n}{ttA}\PY{p}{)}\PY{p}{)}
   \PY{n}{save}\PY{p}{(}\PY{n}{mm1\PYZus{}7b}\PY{p}{,} \PY{l+s+s2}{\PYZdq{}}\PY{l+s+s2}{NB.06.F5\PYZhy{}mm1\PYZus{}7b.sobj}\PY{l+s+s2}{\PYZdq{}}\PY{p}{)}
\PY{k}{else}\PY{p}{:}
   \PY{n}{mm1\PYZus{}7b} \PY{o}{=} \PY{n}{load}\PY{p}{(}\PY{l+s+s2}{\PYZdq{}}\PY{l+s+s2}{NB.06.F5\PYZhy{}mm1\PYZus{}7b.sobj}\PY{l+s+s2}{\PYZdq{}}\PY{p}{)}
\end{Verbatim}
\end{tcolorbox}

    \begin{Verbatim}[commandchars=\\\{\}]
Computation of the 64 order 7 minors:
time: 14.641388000000006
    \end{Verbatim}

    \begin{tcolorbox}[breakable, size=fbox, boxrule=1pt, pad at break*=1mm,colback=cellbackground, colframe=cellborder]
\prompt{In}{incolor}{52}{\boxspacing}
\begin{Verbatim}[commandchars=\\\{\}]
\PY{n}{dgnCs} \PY{o}{=} \PY{n}{l1}\PY{o}{*}\PY{n}{u1}\PY{o}{*}\PY{n}{w1}\PY{o}{*}\PY{n}{w2}\PY{o}{*}\PY{p}{(}\PY{n}{w1}\PY{o}{\PYZhy{}}\PY{n}{w2}\PY{p}{)}\PY{o}{*}\PY{p}{(}\PY{n}{u1}\PY{o}{\PYZhy{}}\PY{n}{v1}\PY{p}{)}\PY{o}{*}\PY{p}{(}\PY{n}{v1}\PY{o}{*}\PY{n}{w2}\PY{o}{+}\PY{l+m+mi}{1}\PY{p}{)}\PY{o}{*}\PY{p}{(}\PY{n}{u1}\PY{o}{*}\PY{n}{w2}\PY{o}{+}\PY{l+m+mi}{1}\PY{p}{)}\PY{o}{*}\PY{p}{(}\PY{n}{v1}\PY{o}{*}\PY{n}{w1}\PY{o}{+}\PY{l+m+mi}{1}\PY{p}{)}\PY{o}{*}\PY{p}{(}\PY{n}{u1}\PY{o}{*}\PY{n}{w1}\PY{o}{+}\PY{l+m+mi}{1}\PY{p}{)}
\end{Verbatim}
\end{tcolorbox}

    \begin{tcolorbox}[breakable, size=fbox, boxrule=1pt, pad at break*=1mm,colback=cellbackground, colframe=cellborder]
\prompt{In}{incolor}{53}{\boxspacing}
\begin{Verbatim}[commandchars=\\\{\}]
\PY{n}{J7} \PY{o}{=} \PY{n}{S}\PY{o}{.}\PY{n}{ideal}\PY{p}{(}\PY{n}{mm1\PYZus{}7b}\PY{p}{)}
\PY{n}{J7} \PY{o}{=} \PY{n}{J7}\PY{o}{.}\PY{n}{saturation}\PY{p}{(}\PY{n}{dgnCs}\PY{p}{)}\PY{p}{[}\PY{l+m+mi}{0}\PY{p}{]}
\end{Verbatim}
\end{tcolorbox}

    The above ideal is (1), so it is not possible to have that the matrix
MM1 has rank \textless= 6 (hence it is not possible that M has rank
\textless= 8)

    \begin{tcolorbox}[breakable, size=fbox, boxrule=1pt, pad at break*=1mm,colback=cellbackground, colframe=cellborder]
\prompt{In}{incolor}{54}{\boxspacing}
\begin{Verbatim}[commandchars=\\\{\}]
\PY{k}{assert}\PY{p}{(}\PY{n}{J7} \PY{o}{==} \PY{n}{S}\PY{o}{.}\PY{n}{ideal}\PY{p}{(}\PY{n}{S}\PY{p}{(}\PY{l+m+mi}{1}\PY{p}{)}\PY{p}{)}\PY{p}{)}
\end{Verbatim}
\end{tcolorbox}

    We want now to compute the determinant of the original MM1 (the order 10
matrix) which is equal to the determiant of the order 8 matrix MM1:

    \begin{tcolorbox}[breakable, size=fbox, boxrule=1pt, pad at break*=1mm,colback=cellbackground, colframe=cellborder]
\prompt{In}{incolor}{55}{\boxspacing}
\begin{Verbatim}[commandchars=\\\{\}]
\PY{n}{dt2} \PY{o}{=} \PY{n}{MM1}\PY{o}{.}\PY{n}{det}\PY{p}{(}\PY{p}{)}
\PY{n}{dt2} \PY{o}{=} \PY{n}{S}\PY{o}{.}\PY{n}{ideal}\PY{p}{(}\PY{n}{dt2}\PY{p}{)}\PY{o}{.}\PY{n}{saturation}\PY{p}{(}\PY{n}{dgnCs}\PY{p}{)}\PY{p}{[}\PY{l+m+mi}{0}\PY{p}{]}\PY{o}{.}\PY{n}{gens}\PY{p}{(}\PY{p}{)}\PY{p}{[}\PY{l+m+mi}{0}\PY{p}{]}
\end{Verbatim}
\end{tcolorbox}

    we get that dt2 is:

\begin{enumerate}
\def\labelenumi{(\arabic{enumi})}
\setcounter{enumi}{15}
\item
  \begin{itemize}
  \tightlist
  \item
    (l1 + (-1/4\emph{ii))\^{}2 } (u1\emph{w1}w2 + w1 + w2)
  \end{itemize}
\end{enumerate}

    \begin{tcolorbox}[breakable, size=fbox, boxrule=1pt, pad at break*=1mm,colback=cellbackground, colframe=cellborder]
\prompt{In}{incolor}{56}{\boxspacing}
\begin{Verbatim}[commandchars=\\\{\}]
\PY{k}{assert}\PY{p}{(}\PY{n}{dt2} \PY{o}{==} \PY{p}{(}\PY{l+m+mi}{16}\PY{p}{)} \PY{o}{*} \PY{p}{(}\PY{n}{l1} \PY{o}{+} \PY{p}{(}\PY{o}{\PYZhy{}}\PY{l+m+mi}{1}\PY{o}{/}\PY{l+m+mi}{4}\PY{o}{*}\PY{n}{ii}\PY{p}{)}\PY{p}{)}\PY{o}{\PYZca{}}\PY{l+m+mi}{2} \PY{o}{*} \PY{p}{(}\PY{n}{u1}\PY{o}{*}\PY{n}{w1}\PY{o}{*}\PY{n}{w2} \PY{o}{+} \PY{n}{w1} \PY{o}{+} \PY{n}{w2}\PY{p}{)}\PY{p}{)}
\end{Verbatim}
\end{tcolorbox}

    We have u1\emph{w1}w2+w1+w2 = 0 and, as in the previous case, since w2
is generic, does not give solutions.

The case l1 = 1/4\emph{ii In this case Cg splits into the line x+ii}y
(tangent line to Cg in P1) and the line P2+P3:

    \begin{tcolorbox}[breakable, size=fbox, boxrule=1pt, pad at break*=1mm,colback=cellbackground, colframe=cellborder]
\prompt{In}{incolor}{57}{\boxspacing}
\begin{Verbatim}[commandchars=\\\{\}]
\PY{k}{assert}\PY{p}{(}
    \PY{p}{(}\PY{o}{\PYZhy{}}\PY{l+m+mi}{2}\PY{p}{)}\PY{o}{*}\PY{n}{u1}\PY{o}{*}\PY{n}{Cg}\PY{o}{.}\PY{n}{subs}\PY{p}{(}\PY{n}{l1}\PY{o}{=}\PY{l+m+mi}{1}\PY{o}{/}\PY{l+m+mi}{4}\PY{o}{*}\PY{n}{ii}\PY{p}{)} 
    \PY{o}{==} \PY{n}{ii}\PY{o}{*}\PY{p}{(}\PY{n}{x}\PY{o}{+}\PY{n}{ii}\PY{o}{*}\PY{n}{y}\PY{p}{)}\PY{o}{*}\PY{n}{matrix}\PY{p}{(}\PY{p}{[}\PY{n}{P2}\PY{p}{,} \PY{n}{P3}\PY{p}{,} \PY{p}{(}\PY{n}{x}\PY{p}{,} \PY{n}{y}\PY{p}{,} \PY{n}{z}\PY{p}{)}\PY{p}{]}\PY{p}{)}\PY{o}{.}\PY{n}{det}\PY{p}{(}\PY{p}{)}
\PY{p}{)}
\end{Verbatim}
\end{tcolorbox}

    We consider the substitution of l1=1/4*ii in the original MM, i.e.~in
MM1 = M.matrix\_from\_rows({[}0, 1, 4, 5, 6, 7, 10, 11, 13, 14{]})

    \begin{tcolorbox}[breakable, size=fbox, boxrule=1pt, pad at break*=1mm,colback=cellbackground, colframe=cellborder]
\prompt{In}{incolor}{58}{\boxspacing}
\begin{Verbatim}[commandchars=\\\{\}]
\PY{n}{MM2} \PY{o}{=} \PY{n}{M}\PY{o}{.}\PY{n}{matrix\PYZus{}from\PYZus{}rows}\PY{p}{(}\PY{p}{[}\PY{l+m+mi}{0}\PY{p}{,} \PY{l+m+mi}{1}\PY{p}{,} \PY{l+m+mi}{4}\PY{p}{,} \PY{l+m+mi}{5}\PY{p}{,} \PY{l+m+mi}{6}\PY{p}{,} \PY{l+m+mi}{7}\PY{p}{,} \PY{l+m+mi}{10}\PY{p}{,} \PY{l+m+mi}{11}\PY{p}{,} \PY{l+m+mi}{13}\PY{p}{,} \PY{l+m+mi}{14}\PY{p}{]}\PY{p}{)}\PY{o}{.}\PY{n}{subs}\PY{p}{(}\PY{n}{l1}\PY{o}{=}\PY{l+m+mi}{1}\PY{o}{/}\PY{l+m+mi}{4}\PY{o}{*}\PY{n}{ii}\PY{p}{)}
\end{Verbatim}
\end{tcolorbox}

    Also here the last row of MM2 is unnecessary:

    \begin{tcolorbox}[breakable, size=fbox, boxrule=1pt, pad at break*=1mm,colback=cellbackground, colframe=cellborder]
\prompt{In}{incolor}{59}{\boxspacing}
\begin{Verbatim}[commandchars=\\\{\}]
\PY{k}{assert}\PY{p}{(}\PY{n}{MM2}\PY{o}{.}\PY{n}{rank}\PY{p}{(}\PY{p}{)} \PY{o}{==} \PY{n}{MM2}\PY{o}{.}\PY{n}{matrix\PYZus{}from\PYZus{}rows}\PY{p}{(}\PY{n+nb}{range}\PY{p}{(}\PY{l+m+mi}{9}\PY{p}{)}\PY{p}{)}\PY{o}{.}\PY{n}{rank}\PY{p}{(}\PY{p}{)}\PY{p}{)}
\end{Verbatim}
\end{tcolorbox}

    Hence we compute the cubic:

    \begin{tcolorbox}[breakable, size=fbox, boxrule=1pt, pad at break*=1mm,colback=cellbackground, colframe=cellborder]
\prompt{In}{incolor}{60}{\boxspacing}
\begin{Verbatim}[commandchars=\\\{\}]
\PY{n}{M3} \PY{o}{=} \PY{p}{(}\PY{n}{MM2}\PY{o}{.}\PY{n}{matrix\PYZus{}from\PYZus{}rows}\PY{p}{(}\PY{n+nb}{range}\PY{p}{(}\PY{l+m+mi}{9}\PY{p}{)}\PY{p}{)}\PY{p}{)}\PY{o}{.}\PY{n}{stack}\PY{p}{(}\PY{n}{matrix}\PY{p}{(}\PY{p}{[}\PY{n}{mon}\PY{p}{]}\PY{p}{)}\PY{p}{)}
\PY{n}{cb2} \PY{o}{=} \PY{n}{M3}\PY{o}{.}\PY{n}{det}\PY{p}{(}\PY{p}{)}
\end{Verbatim}
\end{tcolorbox}

    The cubic splits into the line tangent to \(\mathcal{Q}_{\mathrm{iso}}\)
in \(P_1\) and the reducible conic given by \((P_2 \vee P_3)^2\)

    \begin{tcolorbox}[breakable, size=fbox, boxrule=1pt, pad at break*=1mm,colback=cellbackground, colframe=cellborder]
\prompt{In}{incolor}{61}{\boxspacing}
\begin{Verbatim}[commandchars=\\\{\}]
\PY{k}{assert}\PY{p}{(}
    \PY{n}{S}\PY{o}{.}\PY{n}{ideal}\PY{p}{(}\PY{n}{cb2}\PY{p}{)}\PY{o}{.}\PY{n}{saturation}\PY{p}{(}\PY{n}{dgnCs}\PY{p}{)}\PY{p}{[}\PY{l+m+mi}{0}\PY{p}{]}\PY{o}{.}\PY{n}{gens}\PY{p}{(}\PY{p}{)}\PY{p}{[}\PY{l+m+mi}{0}\PY{p}{]} 
    \PY{o}{==} \PY{p}{(}\PY{n}{x} \PY{o}{+} \PY{n}{ii}\PY{o}{*}\PY{n}{y}\PY{p}{)} \PY{o}{*} \PY{p}{(}\PY{n}{z}\PY{o}{*}\PY{n}{u1} \PY{o}{+} \PY{p}{(}\PY{o}{\PYZhy{}}\PY{n}{ii}\PY{p}{)}\PY{o}{*}\PY{n}{x} \PY{o}{\PYZhy{}} \PY{n}{y}\PY{p}{)}\PY{o}{\PYZca{}}\PY{l+m+mi}{2}
\PY{p}{)}
\end{Verbatim}
\end{tcolorbox}

    also in this case, we do not get a cubic which has a conic among its
eigenpoints.

This conclude the case of \(\Gamma\) tangent to
\(\mathcal{Q}_{\mathrm{iso}}\)

    \hypertarget{gamma-bitangent-to-mathcalq_mathrmiso}{%
\subsection{\texorpdfstring{\(\Gamma\) bitangent to
\(\mathcal{Q}_{\mathrm{iso}}\)}{\textbackslash Gamma bitangent to \textbackslash mathcal\{Q\}\_\{\textbackslash mathrm\{iso\}\}}}\label{gamma-bitangent-to-mathcalq_mathrmiso}}

    Now we consider the case in which among the eigenpoints of a cubic, we
have a concic \(\Gamma\) which is bitangent to
\(\mathcal{Q}_{\mathrm{iso}}\) in two points \(P_1\) and \(P_2\). We
restart the computations:

    \begin{tcolorbox}[breakable, size=fbox, boxrule=1pt, pad at break*=1mm,colback=cellbackground, colframe=cellborder]
\prompt{In}{incolor}{62}{\boxspacing}
\begin{Verbatim}[commandchars=\\\{\}]
\PY{n}{do\PYZus{}long\PYZus{}computations} \PY{o}{=} \PY{k+kc}{False}
\end{Verbatim}
\end{tcolorbox}

    Construction of a generic point Pg on the isotropic conic:

    \begin{tcolorbox}[breakable, size=fbox, boxrule=1pt, pad at break*=1mm,colback=cellbackground, colframe=cellborder]
\prompt{In}{incolor}{63}{\boxspacing}
\begin{Verbatim}[commandchars=\\\{\}]
\PY{n}{P1} \PY{o}{=} \PY{n}{vector}\PY{p}{(}\PY{n}{S}\PY{p}{,} \PY{p}{(}\PY{l+m+mi}{1}\PY{p}{,} \PY{n}{ii}\PY{p}{,} \PY{l+m+mi}{0}\PY{p}{)}\PY{p}{)}
\PY{n}{rt1} \PY{o}{=} \PY{n}{l1}\PY{o}{*}\PY{p}{(}\PY{n}{y}\PY{o}{\PYZhy{}}\PY{n}{ii}\PY{o}{*}\PY{n}{x}\PY{p}{)}\PY{o}{+}\PY{n}{l2}\PY{o}{*}\PY{n}{z}

\PY{n}{rt1}\PY{o}{.}\PY{n}{subs}\PY{p}{(}\PY{p}{\PYZob{}}\PY{n}{x}\PY{p}{:}\PY{n}{P1}\PY{p}{[}\PY{l+m+mi}{0}\PY{p}{]}\PY{p}{,} \PY{n}{y}\PY{p}{:}\PY{n}{P1}\PY{p}{[}\PY{l+m+mi}{1}\PY{p}{]}\PY{p}{,} \PY{n}{z}\PY{p}{:}\PY{n}{P1}\PY{p}{[}\PY{l+m+mi}{2}\PY{p}{]}\PY{p}{\PYZcb{}}\PY{p}{)}

\PY{n}{scndP} \PY{o}{=} \PY{n}{S}\PY{o}{.}\PY{n}{ideal}\PY{p}{(}\PY{n}{Ciso}\PY{p}{,} \PY{n}{rt1}\PY{p}{)}\PY{o}{.}\PY{n}{radical}\PY{p}{(}\PY{p}{)}\PY{o}{.}\PY{n}{primary\PYZus{}decomposition}\PY{p}{(}\PY{p}{)}\PY{p}{[}\PY{l+m+mi}{1}\PY{p}{]}
\PY{n}{aux} \PY{o}{=} \PY{n}{scndP}\PY{o}{.}\PY{n}{gens}\PY{p}{(}\PY{p}{)}\PY{p}{[}\PY{p}{:}\PY{l+m+mi}{2}\PY{p}{]}
\PY{n}{mm2} \PY{o}{=} \PY{n}{matrix}\PY{p}{(}
    \PY{p}{[}
        \PY{p}{[}\PY{n}{aux}\PY{p}{[}\PY{l+m+mi}{0}\PY{p}{]}\PY{o}{.}\PY{n}{coefficient}\PY{p}{(}\PY{n}{x}\PY{p}{)}\PY{p}{,} \PY{n}{aux}\PY{p}{[}\PY{l+m+mi}{0}\PY{p}{]}\PY{o}{.}\PY{n}{coefficient}\PY{p}{(}\PY{n}{y}\PY{p}{)}\PY{p}{,} \PY{n}{aux}\PY{p}{[}\PY{l+m+mi}{0}\PY{p}{]}\PY{o}{.}\PY{n}{coefficient}\PY{p}{(}\PY{n}{z}\PY{p}{)}\PY{p}{]}\PY{p}{,}
        \PY{p}{[}\PY{n}{aux}\PY{p}{[}\PY{l+m+mi}{1}\PY{p}{]}\PY{o}{.}\PY{n}{coefficient}\PY{p}{(}\PY{n}{x}\PY{p}{)}\PY{p}{,} \PY{n}{aux}\PY{p}{[}\PY{l+m+mi}{1}\PY{p}{]}\PY{o}{.}\PY{n}{coefficient}\PY{p}{(}\PY{n}{y}\PY{p}{)}\PY{p}{,} \PY{n}{aux}\PY{p}{[}\PY{l+m+mi}{1}\PY{p}{]}\PY{o}{.}\PY{n}{coefficient}\PY{p}{(}\PY{n}{z}\PY{p}{)}\PY{p}{]}
    \PY{p}{]}
\PY{p}{)}\PY{o}{.}\PY{n}{minors}\PY{p}{(}\PY{l+m+mi}{2}\PY{p}{)}
\end{Verbatim}
\end{tcolorbox}

    \begin{tcolorbox}[breakable, size=fbox, boxrule=1pt, pad at break*=1mm,colback=cellbackground, colframe=cellborder]
\prompt{In}{incolor}{64}{\boxspacing}
\begin{Verbatim}[commandchars=\\\{\}]
\PY{n}{Pg} \PY{o}{=} \PY{n}{vector}\PY{p}{(}\PY{n}{S}\PY{p}{,} \PY{p}{(}\PY{n}{mm2}\PY{p}{[}\PY{l+m+mi}{2}\PY{p}{]}\PY{p}{,} \PY{o}{\PYZhy{}}\PY{n}{mm2}\PY{p}{[}\PY{l+m+mi}{1}\PY{p}{]}\PY{p}{,} \PY{n}{mm2}\PY{p}{[}\PY{l+m+mi}{0}\PY{p}{]}\PY{p}{)}\PY{p}{)}
\end{Verbatim}
\end{tcolorbox}

    \begin{tcolorbox}[breakable, size=fbox, boxrule=1pt, pad at break*=1mm,colback=cellbackground, colframe=cellborder]
\prompt{In}{incolor}{65}{\boxspacing}
\begin{Verbatim}[commandchars=\\\{\}]
\PY{k}{assert}\PY{p}{(}\PY{n}{scndP}\PY{o}{.}\PY{n}{subs}\PY{p}{(}\PY{p}{\PYZob{}}\PY{n}{x}\PY{p}{:}\PY{n}{Pg}\PY{p}{[}\PY{l+m+mi}{0}\PY{p}{]}\PY{p}{,} \PY{n}{y}\PY{p}{:}\PY{n}{Pg}\PY{p}{[}\PY{l+m+mi}{1}\PY{p}{]}\PY{p}{,} \PY{n}{z}\PY{p}{:}\PY{n}{Pg}\PY{p}{[}\PY{l+m+mi}{2}\PY{p}{]}\PY{p}{\PYZcb{}}\PY{p}{)} \PY{o}{==} \PY{n}{S}\PY{o}{.}\PY{n}{ideal}\PY{p}{(}\PY{n}{S}\PY{p}{(}\PY{l+m+mi}{0}\PY{p}{)}\PY{p}{)}\PY{p}{)}
\PY{k}{assert}\PY{p}{(}\PY{n}{Ciso}\PY{o}{.}\PY{n}{subs}\PY{p}{(}\PY{p}{\PYZob{}}\PY{n}{x}\PY{p}{:}\PY{n}{Pg}\PY{p}{[}\PY{l+m+mi}{0}\PY{p}{]}\PY{p}{,} \PY{n}{y}\PY{p}{:}\PY{n}{Pg}\PY{p}{[}\PY{l+m+mi}{1}\PY{p}{]}\PY{p}{,} \PY{n}{z}\PY{p}{:}\PY{n}{Pg}\PY{p}{[}\PY{l+m+mi}{2}\PY{p}{]}\PY{p}{\PYZcb{}}\PY{p}{)} \PY{o}{==} \PY{n}{S}\PY{p}{(}\PY{l+m+mi}{0}\PY{p}{)}\PY{p}{)}
\end{Verbatim}
\end{tcolorbox}

    We can always assume that l2 != 0, since l2 = 0 gives that Pg = P1

    \begin{tcolorbox}[breakable, size=fbox, boxrule=1pt, pad at break*=1mm,colback=cellbackground, colframe=cellborder]
\prompt{In}{incolor}{66}{\boxspacing}
\begin{Verbatim}[commandchars=\\\{\}]
\PY{k}{assert}\PY{p}{(}\PY{n}{matrix}\PY{p}{(}\PY{p}{[}\PY{n}{P1}\PY{p}{,} \PY{n}{Pg}\PY{o}{.}\PY{n}{subs}\PY{p}{(}\PY{n}{l2}\PY{o}{=}\PY{l+m+mi}{0}\PY{p}{)}\PY{p}{]}\PY{p}{)}\PY{o}{.}\PY{n}{rank}\PY{p}{(}\PY{p}{)} \PY{o}{==} \PY{l+m+mi}{1}\PY{p}{)}
\end{Verbatim}
\end{tcolorbox}

    We construct P2, a generic point on \(\mathcal{Q}_{\mathrm{iso}}\) (it
depends on u1)

    \begin{tcolorbox}[breakable, size=fbox, boxrule=1pt, pad at break*=1mm,colback=cellbackground, colframe=cellborder]
\prompt{In}{incolor}{67}{\boxspacing}
\begin{Verbatim}[commandchars=\\\{\}]
\PY{n}{P2} \PY{o}{=} \PY{n}{Pg}\PY{o}{.}\PY{n}{subs}\PY{p}{(}\PY{p}{\PYZob{}}\PY{n}{l1}\PY{p}{:}\PY{n}{u1}\PY{p}{,} \PY{n}{l2}\PY{p}{:}\PY{l+m+mi}{1}\PY{p}{\PYZcb{}}\PY{p}{)}
\end{Verbatim}
\end{tcolorbox}

    And now we construct the tangent line to \(\mathcal{Q}_{\mathrm{iso}}\)
in P1, the tangent line to \(\mathcal{Q}_{\mathrm{iso}}\) in P2, the
pencil of conics tangent to \(\mathcal{Q}_{\mathrm{iso}}\) in P1 and P2
(it depends on l1)

    \begin{tcolorbox}[breakable, size=fbox, boxrule=1pt, pad at break*=1mm,colback=cellbackground, colframe=cellborder]
\prompt{In}{incolor}{68}{\boxspacing}
\begin{Verbatim}[commandchars=\\\{\}]
\PY{n}{rtg1} \PY{o}{=} \PY{n}{scalar\PYZus{}product}\PY{p}{(}\PY{n}{P1}\PY{p}{,} \PY{n}{vector}\PY{p}{(}\PY{p}{(}\PY{n}{x}\PY{p}{,} \PY{n}{y}\PY{p}{,} \PY{n}{z}\PY{p}{)}\PY{p}{)}\PY{o}{\PYZhy{}}\PY{n}{P1}\PY{p}{)}
\PY{n}{rtg2} \PY{o}{=} \PY{n}{scalar\PYZus{}product}\PY{p}{(}\PY{n}{P2}\PY{p}{,} \PY{n}{vector}\PY{p}{(}\PY{p}{(}\PY{n}{x}\PY{p}{,} \PY{n}{y}\PY{p}{,} \PY{n}{z}\PY{p}{)}\PY{p}{)}\PY{o}{\PYZhy{}}\PY{n}{P2}\PY{p}{)}\PY{o}{*}\PY{n}{ii}

\PY{n}{Cg} \PY{o}{=} \PY{n}{Ciso} \PY{o}{+} \PY{n}{l1}\PY{o}{*}\PY{n}{rtg1}\PY{o}{*}\PY{n}{rtg2}
\end{Verbatim}
\end{tcolorbox}

    If l1 = -1, Cg is the conic given by (P1+P2)\^{}2, so we can assume l1+1
!= 0

    \begin{tcolorbox}[breakable, size=fbox, boxrule=1pt, pad at break*=1mm,colback=cellbackground, colframe=cellborder]
\prompt{In}{incolor}{69}{\boxspacing}
\begin{Verbatim}[commandchars=\\\{\}]
\PY{k}{assert}\PY{p}{(}\PY{l+m+mi}{4}\PY{o}{*}\PY{n}{Cg}\PY{o}{.}\PY{n}{subs}\PY{p}{(}\PY{n}{l1}\PY{o}{=}\PY{o}{\PYZhy{}}\PY{l+m+mi}{1}\PY{p}{)} \PY{o}{==} \PY{n}{det}\PY{p}{(}\PY{n}{matrix}\PY{p}{(}\PY{p}{[}\PY{n}{P1}\PY{p}{,} \PY{n}{P2}\PY{p}{,} \PY{p}{(}\PY{n}{x}\PY{p}{,} \PY{n}{y}\PY{p}{,} \PY{n}{z}\PY{p}{)}\PY{p}{]}\PY{p}{)}\PY{p}{)}\PY{o}{\PYZca{}}\PY{l+m+mi}{2}\PY{p}{)}
\end{Verbatim}
\end{tcolorbox}

    construction of a generic point (different from (1, ii, 0)) on Cg:

    \begin{tcolorbox}[breakable, size=fbox, boxrule=1pt, pad at break*=1mm,colback=cellbackground, colframe=cellborder]
\prompt{In}{incolor}{70}{\boxspacing}
\begin{Verbatim}[commandchars=\\\{\}]
\PY{n}{foo} \PY{o}{=} \PY{n}{Cg}\PY{o}{.}\PY{n}{subs}\PY{p}{(}\PY{n}{y}\PY{o}{=}\PY{n}{ii}\PY{o}{*}\PY{n}{x}\PY{o}{+}\PY{n}{w1}\PY{o}{*}\PY{n}{z}\PY{p}{)}\PY{o}{.}\PY{n}{factor}\PY{p}{(}\PY{p}{)}\PY{p}{[}\PY{o}{\PYZhy{}}\PY{l+m+mi}{1}\PY{p}{]}\PY{p}{[}\PY{l+m+mi}{0}\PY{p}{]}
\end{Verbatim}
\end{tcolorbox}

    Generic point of Cg (depends on the parameter w1):

    \begin{tcolorbox}[breakable, size=fbox, boxrule=1pt, pad at break*=1mm,colback=cellbackground, colframe=cellborder]
\prompt{In}{incolor}{71}{\boxspacing}
\begin{Verbatim}[commandchars=\\\{\}]
\PY{n}{Pg2} \PY{o}{=} \PY{n}{vector}\PY{p}{(}
    \PY{n}{S}\PY{p}{,} 
    \PY{p}{(}
        \PY{n}{foo}\PY{o}{.}\PY{n}{coefficient}\PY{p}{(}\PY{n}{z}\PY{p}{)}\PY{p}{,}
        \PY{n}{ii}\PY{o}{*}\PY{p}{(}\PY{n}{foo}\PY{o}{.}\PY{n}{coefficient}\PY{p}{(}\PY{n}{z}\PY{p}{)}\PY{p}{)} \PY{o}{+} \PY{n}{w1}\PY{o}{*}\PY{p}{(}\PY{o}{\PYZhy{}}\PY{n}{foo}\PY{o}{.}\PY{n}{coefficient}\PY{p}{(}\PY{n}{x}\PY{p}{)}\PY{p}{)}\PY{p}{,}
        \PY{o}{\PYZhy{}}\PY{n}{foo}\PY{o}{.}\PY{n}{coefficient}\PY{p}{(}\PY{n}{x}\PY{p}{)}
    \PY{p}{)}
\PY{p}{)}
\end{Verbatim}
\end{tcolorbox}

    the last coordinate of Pg2 is ((2\emph{ii)) } (l1 + 1) * w1. If w1 = 0,
then Pg2 = P1, hence we can assume w1 != 0.

    \begin{tcolorbox}[breakable, size=fbox, boxrule=1pt, pad at break*=1mm,colback=cellbackground, colframe=cellborder]
\prompt{In}{incolor}{72}{\boxspacing}
\begin{Verbatim}[commandchars=\\\{\}]
\PY{k}{assert}\PY{p}{(}\PY{n}{matrix}\PY{p}{(}\PY{p}{[}\PY{n}{P1}\PY{p}{,} \PY{n}{Pg2}\PY{o}{.}\PY{n}{subs}\PY{p}{(}\PY{n}{w1}\PY{o}{=}\PY{l+m+mi}{0}\PY{p}{)}\PY{p}{]}\PY{p}{)}\PY{o}{.}\PY{n}{rank}\PY{p}{(}\PY{p}{)} \PY{o}{==} \PY{l+m+mi}{1}\PY{p}{)}
\end{Verbatim}
\end{tcolorbox}

    Now we define three points on Cg:

    \begin{tcolorbox}[breakable, size=fbox, boxrule=1pt, pad at break*=1mm,colback=cellbackground, colframe=cellborder]
\prompt{In}{incolor}{73}{\boxspacing}
\begin{Verbatim}[commandchars=\\\{\}]
\PY{n}{Pa1}\PY{p}{,} \PY{n}{Pa2} \PY{o}{=} \PY{n}{Pg2}\PY{o}{.}\PY{n}{subs}\PY{p}{(}\PY{p}{\PYZob{}}\PY{n}{w1}\PY{p}{:}\PY{n}{w1}\PY{p}{\PYZcb{}}\PY{p}{)}\PY{p}{,} \PY{n}{Pg2}\PY{o}{.}\PY{n}{subs}\PY{p}{(}\PY{p}{\PYZob{}}\PY{n}{w1}\PY{p}{:}\PY{n}{w2}\PY{p}{\PYZcb{}}\PY{p}{)}
\PY{n}{Pa3} \PY{o}{=} \PY{n}{Pg2}\PY{o}{.}\PY{n}{subs}\PY{p}{(}\PY{p}{\PYZob{}}\PY{n}{w1}\PY{p}{:}\PY{n}{m1}\PY{p}{\PYZcb{}}\PY{p}{)}
\end{Verbatim}
\end{tcolorbox}

    and we can assume w1, w2, m1 != 0.

the following matrix must have rank \textless= 9:

    \begin{tcolorbox}[breakable, size=fbox, boxrule=1pt, pad at break*=1mm,colback=cellbackground, colframe=cellborder]
\prompt{In}{incolor}{74}{\boxspacing}
\begin{Verbatim}[commandchars=\\\{\}]
\PY{n}{M} \PY{o}{=} \PY{n}{condition\PYZus{}matrix}\PY{p}{(}\PY{p}{[}\PY{n}{P1}\PY{p}{,} \PY{n}{P2}\PY{p}{,} \PY{n}{Pa1}\PY{p}{,} \PY{n}{Pa2}\PY{p}{,} \PY{n}{Pa3}\PY{p}{]}\PY{p}{,} \PY{n}{S}\PY{p}{,} \PY{n}{standard}\PY{o}{=}\PY{l+s+s2}{\PYZdq{}}\PY{l+s+s2}{all}\PY{l+s+s2}{\PYZdq{}}\PY{p}{)}
\end{Verbatim}
\end{tcolorbox}

    \hypertarget{we-assume-that-u_1-not-0-so-p_2-is-not-the-point-1--i-0.}{%
\subsubsection{\texorpdfstring{We assume that \(u_1 \not= 0\), so
\(P_2\) is not the point
\((1: -i: 0)\).}{We assume that u\_1 \textbackslash not= 0, so P\_2 is not the point (1: -i: 0).}}\label{we-assume-that-u_1-not-0-so-p_2-is-not-the-point-1--i-0.}}

    Under this hypothesis, we have that we can extract from M the rows: 0,
1; 4, 5; 7, 8; 10, 11; 13, 14. (Remember that w1, w2, m1 are not zero).

    \begin{tcolorbox}[breakable, size=fbox, boxrule=1pt, pad at break*=1mm,colback=cellbackground, colframe=cellborder]
\prompt{In}{incolor}{75}{\boxspacing}
\begin{Verbatim}[commandchars=\\\{\}]
\PY{n}{MM1} \PY{o}{=} \PY{n}{M}\PY{o}{.}\PY{n}{matrix\PYZus{}from\PYZus{}rows}\PY{p}{(}\PY{p}{[}\PY{l+m+mi}{0}\PY{p}{,} \PY{l+m+mi}{1}\PY{p}{,} \PY{l+m+mi}{4}\PY{p}{,} \PY{l+m+mi}{5}\PY{p}{,} \PY{l+m+mi}{7}\PY{p}{,} \PY{l+m+mi}{8}\PY{p}{,} \PY{l+m+mi}{10}\PY{p}{,} \PY{l+m+mi}{11}\PY{p}{,} \PY{l+m+mi}{13}\PY{p}{,} \PY{l+m+mi}{14}\PY{p}{]}\PY{p}{)}
\end{Verbatim}
\end{tcolorbox}

    \begin{tcolorbox}[breakable, size=fbox, boxrule=1pt, pad at break*=1mm,colback=cellbackground, colframe=cellborder]
\prompt{In}{incolor}{76}{\boxspacing}
\begin{Verbatim}[commandchars=\\\{\}]
\PY{c+c1}{\PYZsh{} we make a copy of MM1}
\PY{n}{MM2} \PY{o}{=} \PY{n}{M}\PY{o}{.}\PY{n}{matrix\PYZus{}from\PYZus{}rows}\PY{p}{(}\PY{p}{[}\PY{l+m+mi}{0}\PY{p}{,} \PY{l+m+mi}{1}\PY{p}{,} \PY{l+m+mi}{4}\PY{p}{,} \PY{l+m+mi}{5}\PY{p}{,} \PY{l+m+mi}{7}\PY{p}{,} \PY{l+m+mi}{8}\PY{p}{,} \PY{l+m+mi}{10}\PY{p}{,} \PY{l+m+mi}{11}\PY{p}{,} \PY{l+m+mi}{13}\PY{p}{,} \PY{l+m+mi}{14}\PY{p}{]}\PY{p}{)}
\end{Verbatim}
\end{tcolorbox}

    Using the fact that the first two rows are good (MM2{[}0,0{]} is a non
zero constant, and MM2{[}1, 4{]} is 1) we can reduce MM2 with elementary
rows and columns operations.

    \begin{tcolorbox}[breakable, size=fbox, boxrule=1pt, pad at break*=1mm,colback=cellbackground, colframe=cellborder]
\prompt{In}{incolor}{77}{\boxspacing}
\begin{Verbatim}[commandchars=\\\{\}]
\PY{n}{MM2}\PY{o}{.}\PY{n}{rescale\PYZus{}row}\PY{p}{(}\PY{l+m+mi}{0}\PY{p}{,} \PY{n}{ii}\PY{o}{/}\PY{l+m+mi}{3}\PY{p}{)}
\PY{k}{for} \PY{n}{i} \PY{o+ow}{in} \PY{n+nb}{range}\PY{p}{(}\PY{l+m+mi}{2}\PY{p}{,} \PY{l+m+mi}{10}\PY{p}{)}\PY{p}{:}
    \PY{n}{MM2}\PY{o}{.}\PY{n}{add\PYZus{}multiple\PYZus{}of\PYZus{}row}\PY{p}{(}\PY{n}{i}\PY{p}{,} \PY{l+m+mi}{0}\PY{p}{,} \PY{o}{\PYZhy{}}\PY{n}{MM2}\PY{p}{[}\PY{n}{i}\PY{p}{]}\PY{p}{[}\PY{l+m+mi}{0}\PY{p}{]}\PY{p}{)}

\PY{k}{for} \PY{n}{i} \PY{o+ow}{in} \PY{n+nb}{range}\PY{p}{(}\PY{l+m+mi}{2}\PY{p}{,} \PY{l+m+mi}{10}\PY{p}{)}\PY{p}{:}
    \PY{n}{MM2}\PY{o}{.}\PY{n}{add\PYZus{}multiple\PYZus{}of\PYZus{}row}\PY{p}{(}\PY{n}{i}\PY{p}{,} \PY{l+m+mi}{1}\PY{p}{,} \PY{o}{\PYZhy{}}\PY{n}{MM2}\PY{p}{[}\PY{n}{i}\PY{p}{]}\PY{p}{[}\PY{l+m+mi}{4}\PY{p}{]}\PY{p}{)}
\end{Verbatim}
\end{tcolorbox}

    We extract from MM2 an order 8 square matrix, extracting the last 8 rows
and the columns of position 1, 2, 3, 5, 6, 7, 8, 9.

    \begin{tcolorbox}[breakable, size=fbox, boxrule=1pt, pad at break*=1mm,colback=cellbackground, colframe=cellborder]
\prompt{In}{incolor}{78}{\boxspacing}
\begin{Verbatim}[commandchars=\\\{\}]
\PY{n}{MM2} \PY{o}{=} \PY{n}{MM2}\PY{o}{.}\PY{n}{matrix\PYZus{}from\PYZus{}rows\PYZus{}and\PYZus{}columns}\PY{p}{(}
    \PY{p}{[}\PY{l+m+mi}{2}\PY{p}{,} \PY{l+m+mi}{3}\PY{p}{,} \PY{l+m+mi}{4}\PY{p}{,} \PY{l+m+mi}{5}\PY{p}{,} \PY{l+m+mi}{6}\PY{p}{,} \PY{l+m+mi}{7}\PY{p}{,} \PY{l+m+mi}{8}\PY{p}{,} \PY{l+m+mi}{9}\PY{p}{]}\PY{p}{,}
    \PY{p}{[}\PY{l+m+mi}{1}\PY{p}{,} \PY{l+m+mi}{2}\PY{p}{,} \PY{l+m+mi}{3}\PY{p}{,} \PY{l+m+mi}{5}\PY{p}{,} \PY{l+m+mi}{6}\PY{p}{,} \PY{l+m+mi}{7}\PY{p}{,} \PY{l+m+mi}{8}\PY{p}{,} \PY{l+m+mi}{9}\PY{p}{]}
\PY{p}{)}
\end{Verbatim}
\end{tcolorbox}

    The computation of det(MM2) gives 0 (time of computation: 3' 20'\,')

    \begin{tcolorbox}[breakable, size=fbox, boxrule=1pt, pad at break*=1mm,colback=cellbackground, colframe=cellborder]
\prompt{In}{incolor}{79}{\boxspacing}
\begin{Verbatim}[commandchars=\\\{\}]
\PY{k}{if} \PY{n}{do\PYZus{}long\PYZus{}computations}\PY{p}{:}
   \PY{n}{ttA} \PY{o}{=} \PY{n}{cputime}\PY{p}{(}\PY{p}{)}
   \PY{n}{dtMM2} \PY{o}{=} \PY{n}{MM2}\PY{o}{.}\PY{n}{det}\PY{p}{(}\PY{p}{)}
   \PY{n+nb}{print}\PY{p}{(}\PY{l+s+s2}{\PYZdq{}}\PY{l+s+s2}{Computation of the determinant of MM2:}\PY{l+s+s2}{\PYZdq{}}\PY{p}{)}
   \PY{n+nb}{print}\PY{p}{(}\PY{l+s+s2}{\PYZdq{}}\PY{l+s+s2}{time: }\PY{l+s+s2}{\PYZdq{}}\PY{o}{+}\PY{n+nb}{str}\PY{p}{(}\PY{n}{cputime}\PY{p}{(}\PY{p}{)}\PY{o}{\PYZhy{}}\PY{n}{ttA}\PY{p}{)}\PY{p}{)}
   \PY{n}{sleep}\PY{p}{(}\PY{l+m+mi}{1}\PY{p}{)}
\PY{k}{else}\PY{p}{:}
   \PY{n}{dtMM2} \PY{o}{=} \PY{n}{S}\PY{o}{.}\PY{n}{zero}\PY{p}{(}\PY{p}{)}

\PY{k}{assert}\PY{p}{(}\PY{n}{dtMM2} \PY{o}{==} \PY{n}{S}\PY{o}{.}\PY{n}{zero}\PY{p}{(}\PY{p}{)}\PY{p}{)}
\end{Verbatim}
\end{tcolorbox}

    \begin{Verbatim}[commandchars=\\\{\}]
Computation of the determinant of MM2:
time: 103.08253500000006
    \end{Verbatim}

    Hence M and MM1 have rank \textless= 9. We want to see when M has rank
\textless=8, i.e.~when MM2 has rank \textless=6 hence we compute the
ideal of the order 7-minors of MM2. Time of computation: 35'

If doLongComputations=false, it takes some seconds to load the file

    \begin{tcolorbox}[breakable, size=fbox, boxrule=1pt, pad at break*=1mm,colback=cellbackground, colframe=cellborder]
\prompt{In}{incolor}{80}{\boxspacing}
\begin{Verbatim}[commandchars=\\\{\}]
\PY{k}{if} \PY{n}{do\PYZus{}long\PYZus{}computations}\PY{p}{:}
   \PY{n}{ttA} \PY{o}{=} \PY{n}{cputime}\PY{p}{(}\PY{p}{)}
   \PY{n}{mm2\PYZus{}7} \PY{o}{=} \PY{n}{MM2}\PY{o}{.}\PY{n}{minors}\PY{p}{(}\PY{l+m+mi}{7}\PY{p}{)}
   \PY{n+nb}{print}\PY{p}{(}\PY{l+s+s2}{\PYZdq{}}\PY{l+s+s2}{Computation of the order 7 minors:}\PY{l+s+s2}{\PYZdq{}}\PY{p}{)}
   \PY{n+nb}{print}\PY{p}{(}\PY{l+s+s2}{\PYZdq{}}\PY{l+s+s2}{time: }\PY{l+s+s2}{\PYZdq{}}\PY{o}{+}\PY{n+nb}{str}\PY{p}{(}\PY{n}{cputime}\PY{p}{(}\PY{p}{)}\PY{o}{\PYZhy{}}\PY{n}{ttA}\PY{p}{)}\PY{p}{)}
   \PY{n}{save}\PY{p}{(}\PY{n}{mm2\PYZus{}7}\PY{p}{,} \PY{l+s+s2}{\PYZdq{}}\PY{l+s+s2}{NB.06.F5\PYZhy{}mm2\PYZus{}7.sobj}\PY{l+s+s2}{\PYZdq{}}\PY{p}{)}
   \PY{n}{sleep}\PY{p}{(}\PY{l+m+mi}{1}\PY{p}{)}
\PY{k}{else}\PY{p}{:}
   \PY{n}{mm2\PYZus{}7} \PY{o}{=} \PY{n}{load}\PY{p}{(}\PY{l+s+s2}{\PYZdq{}}\PY{l+s+s2}{NB.06.F5\PYZhy{}mm2\PYZus{}7.sobj}\PY{l+s+s2}{\PYZdq{}}\PY{p}{)}
\end{Verbatim}
\end{tcolorbox}

    \begin{Verbatim}[commandchars=\\\{\}]
Computation of the order 7 minors:
time: 1000.0280200000001
    \end{Verbatim}

    here is a list of factors that cannot be zero:

    \begin{tcolorbox}[breakable, size=fbox, boxrule=1pt, pad at break*=1mm,colback=cellbackground, colframe=cellborder]
\prompt{In}{incolor}{81}{\boxspacing}
\begin{Verbatim}[commandchars=\\\{\}]
\PY{n}{nonZeroFt} \PY{o}{=} \PY{p}{[}
    \PY{n}{l1}\PY{p}{,}\PY{n}{v1}\PY{p}{,}\PY{n}{u1}\PY{p}{,}\PY{n}{w1}\PY{p}{,}\PY{n}{w2}\PY{p}{,}\PY{n}{m1}\PY{p}{,}\PY{n}{l1}\PY{o}{+}\PY{l+m+mi}{1}\PY{p}{,} 
    \PY{p}{(}\PY{n}{w1}\PY{o}{\PYZhy{}}\PY{n}{w2}\PY{p}{)}\PY{p}{,}
    \PY{p}{(}\PY{n}{u1}\PY{o}{\PYZhy{}}\PY{n}{v1}\PY{p}{)}\PY{p}{,}
    \PY{p}{(}\PY{n}{w2}\PY{o}{\PYZhy{}}\PY{n}{m1}\PY{p}{)}\PY{p}{,}
    \PY{p}{(}\PY{n}{w1}\PY{o}{\PYZhy{}}\PY{n}{m1}\PY{p}{)}\PY{p}{,}
    \PY{p}{(}\PY{n}{v1}\PY{o}{*}\PY{n}{w2}\PY{o}{+}\PY{l+m+mi}{1}\PY{p}{)}\PY{p}{,}
    \PY{p}{(}\PY{n}{u1}\PY{o}{*}\PY{n}{w2}\PY{o}{+}\PY{l+m+mi}{1}\PY{p}{)}\PY{p}{,}
    \PY{p}{(}\PY{n}{v1}\PY{o}{*}\PY{n}{w1}\PY{o}{+}\PY{l+m+mi}{1}\PY{p}{)}\PY{p}{,}
    \PY{p}{(}\PY{n}{u1}\PY{o}{*}\PY{n}{w1}\PY{o}{+}\PY{l+m+mi}{1}\PY{p}{)}\PY{p}{,}
    \PY{p}{(}\PY{n}{u1}\PY{o}{*}\PY{n}{m1}\PY{o}{+}\PY{l+m+mi}{1}\PY{p}{)}
\PY{p}{]}
\end{Verbatim}
\end{tcolorbox}

    \begin{tcolorbox}[breakable, size=fbox, boxrule=1pt, pad at break*=1mm,colback=cellbackground, colframe=cellborder]
\prompt{In}{incolor}{82}{\boxspacing}
\begin{Verbatim}[commandchars=\\\{\}]
\PY{c+c1}{\PYZsh{} an auxiliary function:}
\PY{c+c1}{\PYZsh{} input: a polynomial pol and a list of polynomials listFt:}
\PY{c+c1}{\PYZsh{} output: a list of polynomials obtained deleting from the factors of pol}
\PY{c+c1}{\PYZsh{} those factors which are contained in the list listFt. Each factor is taken }
\PY{c+c1}{\PYZsh{} with exponent 1.}
\PY{c+c1}{\PYZsh{} example:}
\PY{c+c1}{\PYZsh{} erase\PYZus{}superfluous\PYZus{}factors(x\PYZca{}3*(x+1)\PYZca{}4*(z+y)\PYZca{}3*(x+y+z)\PYZca{}5, [x, z+y, z])}
\PY{c+c1}{\PYZsh{} answer: [x+1, x+y+z]}
\end{Verbatim}
\end{tcolorbox}

    \begin{tcolorbox}[breakable, size=fbox, boxrule=1pt, pad at break*=1mm,colback=cellbackground, colframe=cellborder]
\prompt{In}{incolor}{83}{\boxspacing}
\begin{Verbatim}[commandchars=\\\{\}]
\PY{k}{def} \PY{n+nf}{erase\PYZus{}superfluous\PYZus{}factors}\PY{p}{(}\PY{n}{pol}\PY{p}{,} \PY{n}{list\PYZus{}factors}\PY{p}{)}\PY{p}{:}
    \PY{k}{if} \PY{n}{pol} \PY{o}{==} \PY{n}{S}\PY{o}{.}\PY{n}{zero}\PY{p}{(}\PY{p}{)}\PY{p}{:}
        \PY{k}{return}\PY{p}{(}\PY{n}{pol}\PY{p}{)}
    \PY{n}{ftOK} \PY{o}{=} \PY{p}{[}\PY{p}{]}
    \PY{k}{for} \PY{n}{ft} \PY{o+ow}{in} \PY{n+nb}{list}\PY{p}{(}\PY{n}{pol}\PY{o}{.}\PY{n}{factor}\PY{p}{(}\PY{p}{)}\PY{p}{)}\PY{p}{:}
        \PY{k}{if} \PY{o+ow}{not} \PY{n}{ft}\PY{p}{[}\PY{l+m+mi}{0}\PY{p}{]} \PY{o+ow}{in} \PY{n}{list\PYZus{}factors}\PY{p}{:}
            \PY{n}{ftOK}\PY{o}{.}\PY{n}{append}\PY{p}{(}\PY{n}{ft}\PY{p}{[}\PY{l+m+mi}{0}\PY{p}{]}\PY{p}{)}
    \PY{k}{return}\PY{p}{(}\PY{n}{ftOK}\PY{p}{)}
\end{Verbatim}
\end{tcolorbox}

    here from each order 7-minor of MM2 (i.e.~every element of mm2\_7) we
clear off the superfluous factors and we construct the ideal J7 of these
polynomials (7'\,' of computation)

    \begin{tcolorbox}[breakable, size=fbox, boxrule=1pt, pad at break*=1mm,colback=cellbackground, colframe=cellborder]
\prompt{In}{incolor}{84}{\boxspacing}
\begin{Verbatim}[commandchars=\\\{\}]
\PY{n}{J7} \PY{o}{=} \PY{p}{[}\PY{p}{]}
\PY{k}{for} \PY{n}{ff} \PY{o+ow}{in} \PY{n}{mm2\PYZus{}7}\PY{p}{:}
    \PY{n}{J7}\PY{o}{.}\PY{n}{append}\PY{p}{(}\PY{n}{prod}\PY{p}{(}\PY{n}{erase\PYZus{}superfluous\PYZus{}factors}\PY{p}{(}\PY{n}{ff}\PY{p}{,} \PY{n}{nonZeroFt}\PY{p}{)}\PY{p}{)}\PY{p}{)}
\end{Verbatim}
\end{tcolorbox}

    since the ideal J7 (after suitable saturation) is (1), we have that the
matrix MM2 cannot have rank 6 or smaller, hence M cannot have rank 8 or
smaller:

    \begin{tcolorbox}[breakable, size=fbox, boxrule=1pt, pad at break*=1mm,colback=cellbackground, colframe=cellborder]
\prompt{In}{incolor}{85}{\boxspacing}
\begin{Verbatim}[commandchars=\\\{\}]
\PY{k}{assert}\PY{p}{(}\PY{n}{S}\PY{o}{.}\PY{n}{ideal}\PY{p}{(}\PY{n}{J7}\PY{p}{)}\PY{o}{.}\PY{n}{saturation}\PY{p}{(}\PY{p}{(}\PY{n}{l1}\PY{o}{+}\PY{l+m+mi}{1}\PY{p}{)}\PY{o}{*}\PY{p}{(}\PY{n}{w2}\PY{o}{\PYZhy{}}\PY{n}{m1}\PY{p}{)}\PY{p}{)}\PY{p}{[}\PY{l+m+mi}{0}\PY{p}{]} \PY{o}{==} \PY{n}{S}\PY{o}{.}\PY{n}{ideal}\PY{p}{(}\PY{n}{S}\PY{p}{(}\PY{l+m+mi}{1}\PY{p}{)}\PY{p}{)}\PY{p}{)}
\end{Verbatim}
\end{tcolorbox}

    The above computations show that the matrix \(M\) has rank 9, there is
thereofre only one cubic given by \(M\) and is obtained from the
determinant of the matrix Mc below, given when we stack to MM1 the row
ginven by the list mon.

The next block constructs the cubic Cbc, which is the unique cubic given
by the matrix M.

    \begin{tcolorbox}[breakable, size=fbox, boxrule=1pt, pad at break*=1mm,colback=cellbackground, colframe=cellborder]
\prompt{In}{incolor}{86}{\boxspacing}
\begin{Verbatim}[commandchars=\\\{\}]
\PY{n}{Mc} \PY{o}{=} \PY{n}{MM1}\PY{o}{.}\PY{n}{matrix\PYZus{}from\PYZus{}rows}\PY{p}{(}\PY{n+nb}{range}\PY{p}{(}\PY{l+m+mi}{9}\PY{p}{)}\PY{p}{)}  
\PY{n}{Mc} \PY{o}{=} \PY{n}{Mc}\PY{o}{.}\PY{n}{stack}\PY{p}{(}\PY{n}{matrix}\PY{p}{(}\PY{p}{[}\PY{n}{mon}\PY{p}{]}\PY{p}{)}\PY{p}{)}
\end{Verbatim}
\end{tcolorbox}

    \begin{tcolorbox}[breakable, size=fbox, boxrule=1pt, pad at break*=1mm,colback=cellbackground, colframe=cellborder]
\prompt{In}{incolor}{87}{\boxspacing}
\begin{Verbatim}[commandchars=\\\{\}]
\PY{k}{if} \PY{n}{do\PYZus{}long\PYZus{}computations}\PY{p}{:}
   \PY{n}{ttA} \PY{o}{=} \PY{n}{cputime}\PY{p}{(}\PY{p}{)}
   \PY{n}{dtMc} \PY{o}{=} \PY{n}{Mc}\PY{o}{.}\PY{n}{det}\PY{p}{(}\PY{p}{)}
   \PY{n+nb}{print}\PY{p}{(}\PY{l+s+s2}{\PYZdq{}}\PY{l+s+s2}{Computation of the cubic:}\PY{l+s+s2}{\PYZdq{}}\PY{p}{)}
   \PY{n+nb}{print}\PY{p}{(}\PY{l+s+s2}{\PYZdq{}}\PY{l+s+s2}{time: }\PY{l+s+s2}{\PYZdq{}}\PY{o}{+}\PY{n+nb}{str}\PY{p}{(}\PY{n}{cputime}\PY{p}{(}\PY{p}{)}\PY{o}{\PYZhy{}}\PY{n}{ttA}\PY{p}{)}\PY{p}{)}
   \PY{n}{save}\PY{p}{(}\PY{n}{dtMc}\PY{p}{,} \PY{l+s+s2}{\PYZdq{}}\PY{l+s+s2}{NB.06.F5\PYZhy{}dtMc.sobj}\PY{l+s+s2}{\PYZdq{}}\PY{p}{)}
   \PY{n}{sleep}\PY{p}{(}\PY{l+m+mi}{1}\PY{p}{)}
\PY{k}{else}\PY{p}{:}
   \PY{n}{dtMc} \PY{o}{=} \PY{n}{load}\PY{p}{(}\PY{l+s+s2}{\PYZdq{}}\PY{l+s+s2}{NB.06.F5\PYZhy{}dtMc.sobj}\PY{l+s+s2}{\PYZdq{}}\PY{p}{)}
\end{Verbatim}
\end{tcolorbox}

    \begin{Verbatim}[commandchars=\\\{\}]
Computation of the cubic:
time: 1077.124312
    \end{Verbatim}

    \begin{tcolorbox}[breakable, size=fbox, boxrule=1pt, pad at break*=1mm,colback=cellbackground, colframe=cellborder]
\prompt{In}{incolor}{88}{\boxspacing}
\begin{Verbatim}[commandchars=\\\{\}]
\PY{c+c1}{\PYZsh{}\PYZsh{} from dtMc we erase useless factors:}

\PY{n}{dt1} \PY{o}{=} \PY{n}{erase\PYZus{}superfluous\PYZus{}factors}\PY{p}{(}\PY{n}{dtMc}\PY{p}{,} \PY{n}{nonZeroFt}\PY{p}{)}
\end{Verbatim}
\end{tcolorbox}

    We get that dt1 is a list of 4 elements, which are:

    \begin{tcolorbox}[breakable, size=fbox, boxrule=1pt, pad at break*=1mm,colback=cellbackground, colframe=cellborder]
\prompt{In}{incolor}{89}{\boxspacing}
\begin{Verbatim}[commandchars=\\\{\}]
\PY{n}{Ls} \PY{o}{=} \PY{p}{[}
    \PY{n}{x}\PY{o}{*}\PY{n}{u1} \PY{o}{+} \PY{n}{ii}\PY{o}{*}\PY{n}{y}\PY{o}{*}\PY{n}{u1} \PY{o}{+} \PY{n}{ii}\PY{o}{*}\PY{n}{z}\PY{p}{,}
    \PY{n}{u1}\PY{o}{*}\PY{n}{w1}\PY{o}{*}\PY{n}{w2} \PY{o}{+} \PY{l+m+mi}{1}\PY{o}{/}\PY{l+m+mi}{2}\PY{o}{*}\PY{n}{w1} \PY{o}{+} \PY{l+m+mi}{1}\PY{o}{/}\PY{l+m+mi}{2}\PY{o}{*}\PY{n}{w2}\PY{p}{,}
    \PY{n}{u1}\PY{o}{\PYZca{}}\PY{l+m+mi}{2}\PY{o}{*}\PY{n}{l1}\PY{o}{*}\PY{n}{m1}\PY{o}{\PYZca{}}\PY{l+m+mi}{2} \PY{o}{\PYZhy{}} \PY{n}{l1}\PY{o}{*}\PY{n}{m1}\PY{o}{\PYZca{}}\PY{l+m+mi}{2} \PY{o}{\PYZhy{}} \PY{l+m+mi}{2}\PY{o}{*}\PY{n}{u1}\PY{o}{*}\PY{n}{m1} \PY{o}{\PYZhy{}} \PY{n}{m1}\PY{o}{\PYZca{}}\PY{l+m+mi}{2} \PY{o}{\PYZhy{}} \PY{l+m+mi}{1}\PY{p}{,}
    \PY{n}{x}\PY{o}{\PYZca{}}\PY{l+m+mi}{2}\PY{o}{*}\PY{n}{u1}\PY{o}{\PYZca{}}\PY{l+m+mi}{2}\PY{o}{*}\PY{n}{l1} \PY{o}{+} \PY{p}{(}\PY{l+m+mi}{2}\PY{o}{*}\PY{n}{ii}\PY{p}{)}\PY{o}{*}\PY{n}{x}\PY{o}{*}\PY{n}{y}\PY{o}{*}\PY{n}{u1}\PY{o}{\PYZca{}}\PY{l+m+mi}{2}\PY{o}{*}\PY{n}{l1} \PY{o}{\PYZhy{}} \PY{n}{y}\PY{o}{\PYZca{}}\PY{l+m+mi}{2}\PY{o}{*}\PY{n}{u1}\PY{o}{\PYZca{}}\PY{l+m+mi}{2}\PY{o}{*}\PY{n}{l1} \PY{o}{+} \PY{p}{(}\PY{l+m+mi}{2}\PY{o}{*}\PY{n}{ii}\PY{p}{)}\PY{o}{*}\PY{n}{x}\PY{o}{*}\PY{n}{z}\PY{o}{*}\PY{n}{u1}\PY{o}{*}\PY{n}{l1} 
    \PY{o}{\PYZhy{}} \PY{l+m+mi}{2}\PY{o}{*}\PY{n}{y}\PY{o}{*}\PY{n}{z}\PY{o}{*}\PY{n}{u1}\PY{o}{*}\PY{n}{l1} \PY{o}{+} \PY{l+m+mi}{3}\PY{o}{*}\PY{n}{x}\PY{o}{\PYZca{}}\PY{l+m+mi}{2}\PY{o}{*}\PY{n}{l1} \PY{o}{+} \PY{l+m+mi}{3}\PY{o}{*}\PY{n}{y}\PY{o}{\PYZca{}}\PY{l+m+mi}{2}\PY{o}{*}\PY{n}{l1} \PY{o}{+} \PY{l+m+mi}{2}\PY{o}{*}\PY{n}{z}\PY{o}{\PYZca{}}\PY{l+m+mi}{2}\PY{o}{*}\PY{n}{l1} \PY{o}{+} \PY{l+m+mi}{3}\PY{o}{*}\PY{n}{x}\PY{o}{\PYZca{}}\PY{l+m+mi}{2} \PY{o}{+} \PY{l+m+mi}{3}\PY{o}{*}\PY{n}{y}\PY{o}{\PYZca{}}\PY{l+m+mi}{2} \PY{o}{+} \PY{l+m+mi}{3}\PY{o}{*}\PY{n}{z}\PY{o}{\PYZca{}}\PY{l+m+mi}{2}
\PY{p}{]}
\end{Verbatim}
\end{tcolorbox}

    \begin{tcolorbox}[breakable, size=fbox, boxrule=1pt, pad at break*=1mm,colback=cellbackground, colframe=cellborder]
\prompt{In}{incolor}{90}{\boxspacing}
\begin{Verbatim}[commandchars=\\\{\}]
\PY{k}{assert}\PY{p}{(}\PY{n}{dt1} \PY{o}{==} \PY{n}{Ls}\PY{p}{)}
\end{Verbatim}
\end{tcolorbox}

    We select the two factors which are a line and a conic and whose product
is the desired cubic.

The factor dt1{[}0{]} is the line r12 given by P1+P2:

    \begin{tcolorbox}[breakable, size=fbox, boxrule=1pt, pad at break*=1mm,colback=cellbackground, colframe=cellborder]
\prompt{In}{incolor}{91}{\boxspacing}
\begin{Verbatim}[commandchars=\\\{\}]
\PY{n}{r12} \PY{o}{=} \PY{n}{det}\PY{p}{(}\PY{n}{matrix}\PY{p}{(}\PY{p}{[}\PY{n}{P1}\PY{p}{,} \PY{n}{P2}\PY{p}{,} \PY{p}{(}\PY{n}{x}\PY{p}{,} \PY{n}{y}\PY{p}{,} \PY{n}{z}\PY{p}{)}\PY{p}{]}\PY{p}{)}\PY{p}{)}\PY{o}{/}\PY{p}{(}\PY{l+m+mi}{2}\PY{o}{*}\PY{n}{ii}\PY{p}{)}
\end{Verbatim}
\end{tcolorbox}

    \begin{tcolorbox}[breakable, size=fbox, boxrule=1pt, pad at break*=1mm,colback=cellbackground, colframe=cellborder]
\prompt{In}{incolor}{92}{\boxspacing}
\begin{Verbatim}[commandchars=\\\{\}]
\PY{k}{assert}\PY{p}{(}\PY{n}{dt1}\PY{p}{[}\PY{l+m+mi}{0}\PY{p}{]} \PY{o}{==} \PY{n}{r12}\PY{p}{)}
\end{Verbatim}
\end{tcolorbox}

    The factor Lt{[}3{]} is the l1\emph{r12\^{}2+3}(l1+1)*Ciso:

    \begin{tcolorbox}[breakable, size=fbox, boxrule=1pt, pad at break*=1mm,colback=cellbackground, colframe=cellborder]
\prompt{In}{incolor}{93}{\boxspacing}
\begin{Verbatim}[commandchars=\\\{\}]
\PY{k}{assert}\PY{p}{(}\PY{n}{dt1}\PY{p}{[}\PY{l+m+mi}{3}\PY{p}{]} \PY{o}{==} \PY{n}{l1}\PY{o}{*}\PY{n}{r12}\PY{o}{\PYZca{}}\PY{l+m+mi}{2}\PY{o}{+}\PY{l+m+mi}{3}\PY{o}{*}\PY{p}{(}\PY{n}{l1}\PY{o}{+}\PY{l+m+mi}{1}\PY{p}{)}\PY{o}{*}\PY{n}{Ciso}\PY{p}{)}
\end{Verbatim}
\end{tcolorbox}

    The cubic is therefore:

    \begin{tcolorbox}[breakable, size=fbox, boxrule=1pt, pad at break*=1mm,colback=cellbackground, colframe=cellborder]
\prompt{In}{incolor}{94}{\boxspacing}
\begin{Verbatim}[commandchars=\\\{\}]
\PY{n}{Cbc} \PY{o}{=} \PY{n}{dt1}\PY{p}{[}\PY{l+m+mi}{0}\PY{p}{]}\PY{o}{*}\PY{n}{dt1}\PY{p}{[}\PY{l+m+mi}{3}\PY{p}{]}
\end{Verbatim}
\end{tcolorbox}

    The cubic Cbc is of the form r12\emph{(u}r12\^{}2+v*Ciso) for suitable
u, v

    \begin{tcolorbox}[breakable, size=fbox, boxrule=1pt, pad at break*=1mm,colback=cellbackground, colframe=cellborder]
\prompt{In}{incolor}{95}{\boxspacing}
\begin{Verbatim}[commandchars=\\\{\}]
\PY{k}{assert}\PY{p}{(}\PY{n}{Cbc} \PY{o}{==} \PY{n}{r12}\PY{o}{*}\PY{p}{(}\PY{n}{l1}\PY{o}{*}\PY{n}{r12}\PY{o}{\PYZca{}}\PY{l+m+mi}{2}\PY{o}{+}\PY{l+m+mi}{3}\PY{o}{*}\PY{p}{(}\PY{n}{l1}\PY{o}{+}\PY{l+m+mi}{1}\PY{p}{)}\PY{o}{*}\PY{n}{Ciso}\PY{p}{)}\PY{p}{)}
\end{Verbatim}
\end{tcolorbox}

    now we compute the eigenpoints of Cbc:

    \begin{tcolorbox}[breakable, size=fbox, boxrule=1pt, pad at break*=1mm,colback=cellbackground, colframe=cellborder]
\prompt{In}{incolor}{96}{\boxspacing}
\begin{Verbatim}[commandchars=\\\{\}]
\PY{n}{Je} \PY{o}{=} \PY{n}{S}\PY{o}{.}\PY{n}{ideal}\PY{p}{(}
    \PY{n}{matrix}\PY{p}{(}
        \PY{p}{[}
            \PY{p}{[}
                \PY{n}{Cbc}\PY{o}{.}\PY{n}{derivative}\PY{p}{(}\PY{n}{x}\PY{p}{)}\PY{p}{,}
                \PY{n}{Cbc}\PY{o}{.}\PY{n}{derivative}\PY{p}{(}\PY{n}{y}\PY{p}{)}\PY{p}{,}
                \PY{n}{Cbc}\PY{o}{.}\PY{n}{derivative}\PY{p}{(}\PY{n}{z}\PY{p}{)}
            \PY{p}{]}\PY{p}{,}
            \PY{p}{[}\PY{n}{x}\PY{p}{,} \PY{n}{y}\PY{p}{,} \PY{n}{z}\PY{p}{]}
        \PY{p}{]}
    \PY{p}{)}\PY{o}{.}\PY{n}{minors}\PY{p}{(}\PY{l+m+mi}{2}\PY{p}{)}
\PY{p}{)}
\end{Verbatim}
\end{tcolorbox}

    we get two components, which are: the ideal generated by rtg1, rtg2
(hence is a point) and the principal ideal generated by
Ciso+l1\emph{rtg1}rtg2

    \begin{tcolorbox}[breakable, size=fbox, boxrule=1pt, pad at break*=1mm,colback=cellbackground, colframe=cellborder]
\prompt{In}{incolor}{97}{\boxspacing}
\begin{Verbatim}[commandchars=\\\{\}]
\PY{n}{PDe} \PY{o}{=} \PY{n}{Je}\PY{o}{.}\PY{n}{primary\PYZus{}decomposition}\PY{p}{(}\PY{p}{)}
\end{Verbatim}
\end{tcolorbox}

    \begin{tcolorbox}[breakable, size=fbox, boxrule=1pt, pad at break*=1mm,colback=cellbackground, colframe=cellborder]
\prompt{In}{incolor}{98}{\boxspacing}
\begin{Verbatim}[commandchars=\\\{\}]
\PY{k}{assert}\PY{p}{(}\PY{n+nb}{len}\PY{p}{(}\PY{n}{PDe}\PY{p}{)} \PY{o}{==} \PY{l+m+mi}{2}\PY{p}{)}
\PY{k}{assert}\PY{p}{(}\PY{n}{PDe}\PY{p}{[}\PY{l+m+mi}{0}\PY{p}{]} \PY{o}{==} \PY{n}{S}\PY{o}{.}\PY{n}{ideal}\PY{p}{(}\PY{n}{rtg1}\PY{p}{,} \PY{n}{rtg2}\PY{p}{)}\PY{p}{)}
\PY{k}{assert}\PY{p}{(}\PY{n}{PDe}\PY{p}{[}\PY{l+m+mi}{1}\PY{p}{]} \PY{o}{==} \PY{n}{S}\PY{o}{.}\PY{n}{ideal}\PY{p}{(}\PY{n}{Ciso}\PY{o}{+}\PY{n}{l1}\PY{o}{*}\PY{n}{rtg1}\PY{o}{*}\PY{n}{rtg2}\PY{p}{)}\PY{p}{)}
\end{Verbatim}
\end{tcolorbox}

    Conversely, we assume here that the cubic C is of the form C1 =
(l1\emph{Ciso+l2}(u1\emph{x+v1}y+w1\emph{z)\^{}2)}(u1\emph{x+v1}y+w1\emph{z)
where u1}x+v1\emph{y+w1}z is a generic line of the plane and l1 and l2
are parameters.

    \begin{tcolorbox}[breakable, size=fbox, boxrule=1pt, pad at break*=1mm,colback=cellbackground, colframe=cellborder]
\prompt{In}{incolor}{99}{\boxspacing}
\begin{Verbatim}[commandchars=\\\{\}]
\PY{n}{C1} \PY{o}{=} \PY{p}{(}\PY{n}{l1}\PY{o}{*}\PY{n}{Ciso}\PY{o}{+}\PY{n}{l2}\PY{o}{*}\PY{p}{(}\PY{n}{u1}\PY{o}{*}\PY{n}{x}\PY{o}{+}\PY{n}{v1}\PY{o}{*}\PY{n}{y}\PY{o}{+}\PY{n}{w1}\PY{o}{*}\PY{n}{z}\PY{p}{)}\PY{o}{\PYZca{}}\PY{l+m+mi}{2}\PY{p}{)}\PY{o}{*}\PY{p}{(}\PY{n}{u1}\PY{o}{*}\PY{n}{x}\PY{o}{+}\PY{n}{v1}\PY{o}{*}\PY{n}{y}\PY{o}{+}\PY{n}{w1}\PY{o}{*}\PY{n}{z}\PY{p}{)}
\end{Verbatim}
\end{tcolorbox}

    We compute the eigenpoints of C1: and the primary decomposition of the
ideal of the eigenpoints. We get two components:

The first is:

Ideal (z\emph{v1 - y}w1, z\emph{u1 - x}w1, y\emph{u1 - x}v1)

i.e.~the point (u1, v1, w1)

and the second component is the conic

l1\emph{Ciso+3}l2\emph{(u1}x+v1\emph{y+w1}z)\^{}2

    \begin{tcolorbox}[breakable, size=fbox, boxrule=1pt, pad at break*=1mm,colback=cellbackground, colframe=cellborder]
\prompt{In}{incolor}{100}{\boxspacing}
\begin{Verbatim}[commandchars=\\\{\}]
\PY{n}{JJ} \PY{o}{=} \PY{n}{S}\PY{o}{.}\PY{n}{ideal}\PY{p}{(}
    \PY{n}{matrix}\PY{p}{(}
        \PY{p}{[}
            \PY{p}{[}
                \PY{n}{C1}\PY{o}{.}\PY{n}{derivative}\PY{p}{(}\PY{n}{x}\PY{p}{)}\PY{p}{,}
                \PY{n}{C1}\PY{o}{.}\PY{n}{derivative}\PY{p}{(}\PY{n}{y}\PY{p}{)}\PY{p}{,}
                \PY{n}{C1}\PY{o}{.}\PY{n}{derivative}\PY{p}{(}\PY{n}{z}\PY{p}{)}
            \PY{p}{]}\PY{p}{,} 
            \PY{p}{[}\PY{n}{x}\PY{p}{,} \PY{n}{y}\PY{p}{,} \PY{n}{z}\PY{p}{]}
        \PY{p}{]}
    \PY{p}{)}\PY{o}{.}\PY{n}{minors}\PY{p}{(}\PY{l+m+mi}{2}\PY{p}{)}
\PY{p}{)}
\end{Verbatim}
\end{tcolorbox}

    \begin{tcolorbox}[breakable, size=fbox, boxrule=1pt, pad at break*=1mm,colback=cellbackground, colframe=cellborder]
\prompt{In}{incolor}{101}{\boxspacing}
\begin{Verbatim}[commandchars=\\\{\}]
\PY{n}{pdJJ} \PY{o}{=} \PY{n}{JJ}\PY{o}{.}\PY{n}{primary\PYZus{}decomposition}\PY{p}{(}\PY{p}{)}
\end{Verbatim}
\end{tcolorbox}

    \begin{tcolorbox}[breakable, size=fbox, boxrule=1pt, pad at break*=1mm,colback=cellbackground, colframe=cellborder]
\prompt{In}{incolor}{102}{\boxspacing}
\begin{Verbatim}[commandchars=\\\{\}]
\PY{k}{assert}\PY{p}{(}\PY{n}{pdJJ}\PY{p}{[}\PY{l+m+mi}{0}\PY{p}{]} \PY{o}{==} \PY{n}{S}\PY{o}{.}\PY{n}{ideal} \PY{p}{(}\PY{n}{z}\PY{o}{*}\PY{n}{v1} \PY{o}{\PYZhy{}} \PY{n}{y}\PY{o}{*}\PY{n}{w1}\PY{p}{,} \PY{n}{z}\PY{o}{*}\PY{n}{u1} \PY{o}{\PYZhy{}} \PY{n}{x}\PY{o}{*}\PY{n}{w1}\PY{p}{,} \PY{n}{y}\PY{o}{*}\PY{n}{u1} \PY{o}{\PYZhy{}} \PY{n}{x}\PY{o}{*}\PY{n}{v1}\PY{p}{)}\PY{p}{)}
\PY{k}{assert}\PY{p}{(}\PY{n}{pdJJ}\PY{p}{[}\PY{l+m+mi}{1}\PY{p}{]} \PY{o}{==} \PY{n}{S}\PY{o}{.}\PY{n}{ideal}\PY{p}{(}\PY{n}{l1}\PY{o}{*}\PY{n}{Ciso}\PY{o}{+}\PY{l+m+mi}{3}\PY{o}{*}\PY{n}{l2}\PY{o}{*}\PY{p}{(}\PY{n}{u1}\PY{o}{*}\PY{n}{x}\PY{o}{+}\PY{n}{v1}\PY{o}{*}\PY{n}{y}\PY{o}{+}\PY{n}{w1}\PY{o}{*}\PY{n}{z}\PY{p}{)}\PY{o}{\PYZca{}}\PY{l+m+mi}{2}\PY{p}{)}\PY{p}{)}
\end{Verbatim}
\end{tcolorbox}

    In case \(P_2 \not= (1: -i:0)\) we have:

A cubic C of the plane has in its eigenpoints a conic bitanget to
\(\mathcal{Q}_{\mathrm{iso}}\) if and only if C is of the form
(l1\emph{Ciso+l2}(u1\emph{x+v1}y+w1\emph{z)\^{}2)}(u1\emph{x+v1}y+w1*z)

Moreover (for any \(P_2\)), if a cubic is of the above form, the conic
l1\emph{Ciso+3}l2\emph{(u1}x+v1\emph{y+w1}z)\^{}2 is contained in its
eigenpoints.

    \hypertarget{case-p_2-1--i-0.}{%
\subsubsection{\texorpdfstring{Case
\(P_2 = (1: -i: 0)\).}{Case P\_2 = (1: -i: 0).}}\label{case-p_2-1--i-0.}}

    We start as before, so we define the points \(P_1, P_2\), the tangent
lines to \(\mathcal{Q}_{\mathrm{iso}}\) in \(P_1\) and \(P_2\), the
pencil of conics Cg bitangent to \(\mathcal{Q}_{\mathrm{iso}}\) in
\(P_1\) and \(P_2\), we construct three points, here called Pg1, Pg2 and
Pg3 on Cg. We observe that we can assume that some used parameters are
not 0

    \begin{tcolorbox}[breakable, size=fbox, boxrule=1pt, pad at break*=1mm,colback=cellbackground, colframe=cellborder]
\prompt{In}{incolor}{103}{\boxspacing}
\begin{Verbatim}[commandchars=\\\{\}]
\PY{n}{P1} \PY{o}{=} \PY{n}{vector}\PY{p}{(}\PY{n}{S}\PY{p}{,} \PY{p}{(}\PY{l+m+mi}{1}\PY{p}{,} \PY{n}{ii}\PY{p}{,} \PY{l+m+mi}{0}\PY{p}{)}\PY{p}{)}
\PY{n}{P2} \PY{o}{=} \PY{n}{vector}\PY{p}{(}\PY{n}{S}\PY{p}{,} \PY{p}{(}\PY{l+m+mi}{1}\PY{p}{,} \PY{o}{\PYZhy{}}\PY{n}{ii}\PY{p}{,} \PY{l+m+mi}{0}\PY{p}{)}\PY{p}{)}
\end{Verbatim}
\end{tcolorbox}

    \begin{tcolorbox}[breakable, size=fbox, boxrule=1pt, pad at break*=1mm,colback=cellbackground, colframe=cellborder]
\prompt{In}{incolor}{104}{\boxspacing}
\begin{Verbatim}[commandchars=\\\{\}]
\PY{c+c1}{\PYZsh{}\PYZsh{} Tangent line to Ciso in P1:}
\PY{n}{rtg1} \PY{o}{=} \PY{n}{scalar\PYZus{}product}\PY{p}{(}\PY{n}{P1}\PY{p}{,} \PY{n}{vector}\PY{p}{(}\PY{p}{(}\PY{n}{x}\PY{p}{,} \PY{n}{y}\PY{p}{,} \PY{n}{z}\PY{p}{)}\PY{p}{)}\PY{o}{\PYZhy{}}\PY{n}{P1}\PY{p}{)}


\PY{c+c1}{\PYZsh{}\PYZsh{} Tangent line to Ciso in P2:}
\PY{n}{rtg2} \PY{o}{=} \PY{n}{scalar\PYZus{}product}\PY{p}{(}\PY{n}{P2}\PY{p}{,} \PY{n}{vector}\PY{p}{(}\PY{p}{(}\PY{n}{x}\PY{p}{,} \PY{n}{y}\PY{p}{,} \PY{n}{z}\PY{p}{)}\PY{p}{)}\PY{o}{\PYZhy{}}\PY{n}{P2}\PY{p}{)}

\PY{c+c1}{\PYZsh{}\PYZsh{} Pencil of conics tangent to P1 and P2 to Ciso:}
\PY{n}{Cg} \PY{o}{=} \PY{n}{Ciso} \PY{o}{+} \PY{n}{l1}\PY{o}{*}\PY{n}{rtg1}\PY{o}{*}\PY{n}{rtg2}

\PY{c+c1}{\PYZsh{}\PYZsh{} If l1 = \PYZhy{}1, Cg is the conic given by (P1+P2)\PYZca{}2, so we can assume}
\PY{c+c1}{\PYZsh{}\PYZsh{} l1+1 != 0}
\PY{k}{assert}\PY{p}{(}\PY{o}{\PYZhy{}}\PY{l+m+mi}{4}\PY{o}{*}\PY{n}{Cg}\PY{o}{.}\PY{n}{subs}\PY{p}{(}\PY{n}{l1}\PY{o}{=}\PY{o}{\PYZhy{}}\PY{l+m+mi}{1}\PY{p}{)} \PY{o}{==} \PY{n}{det}\PY{p}{(}\PY{n}{matrix}\PY{p}{(}\PY{p}{[}\PY{n}{P1}\PY{p}{,} \PY{n}{P2}\PY{p}{,} \PY{p}{(}\PY{n}{x}\PY{p}{,} \PY{n}{y}\PY{p}{,} \PY{n}{z}\PY{p}{)}\PY{p}{]}\PY{p}{)}\PY{p}{)}\PY{o}{\PYZca{}}\PY{l+m+mi}{2}\PY{p}{)}

\PY{c+c1}{\PYZsh{}\PYZsh{} construction of a generic point (different from (1, ii, 0)) on Cg:}

\PY{n}{foo} \PY{o}{=} \PY{n}{Cg}\PY{o}{.}\PY{n}{subs}\PY{p}{(}\PY{n}{y}\PY{o}{=}\PY{n}{ii}\PY{o}{*}\PY{n}{x}\PY{o}{+}\PY{n}{w1}\PY{o}{*}\PY{n}{z}\PY{p}{)}\PY{o}{.}\PY{n}{factor}\PY{p}{(}\PY{p}{)}\PY{p}{[}\PY{o}{\PYZhy{}}\PY{l+m+mi}{1}\PY{p}{]}\PY{p}{[}\PY{l+m+mi}{0}\PY{p}{]}

\PY{c+c1}{\PYZsh{}\PYZsh{} generic point of Cg (depends on the parameter w1):}
\PY{n}{Pg} \PY{o}{=} \PY{n}{vector}\PY{p}{(}\PY{n}{S}\PY{p}{,} \PY{p}{(}\PY{n}{foo}\PY{o}{.}\PY{n}{coefficient}\PY{p}{(}\PY{n}{z}\PY{p}{)}\PY{p}{,} \PY{n}{ii}\PY{o}{*}\PY{p}{(}\PY{n}{foo}\PY{o}{.}\PY{n}{coefficient}\PY{p}{(}\PY{n}{z}\PY{p}{)}\PY{p}{)}\PY{o}{+}\PYZbs{}
                     \PY{n}{w1}\PY{o}{*}\PY{p}{(}\PY{o}{\PYZhy{}}\PY{n}{foo}\PY{o}{.}\PY{n}{coefficient}\PY{p}{(}\PY{n}{x}\PY{p}{)}\PY{p}{)}\PY{p}{,} \PY{o}{\PYZhy{}}\PY{n}{foo}\PY{o}{.}\PY{n}{coefficient}\PY{p}{(}\PY{n}{x}\PY{p}{)}\PY{p}{)}\PY{p}{)}

\PY{c+c1}{\PYZsh{}\PYZsh{} the last coordinate of Pg is ((\PYZhy{}2*ii)) * (l1 + 1) * w1. }
\PY{c+c1}{\PYZsh{}\PYZsh{} If w1 = 0, then Pg = P1, hence we can assume w1 != 0.}
\PY{k}{assert}\PY{p}{(}\PY{n}{matrix}\PY{p}{(}\PY{p}{[}\PY{n}{P1}\PY{p}{,} \PY{n}{Pg}\PY{o}{.}\PY{n}{subs}\PY{p}{(}\PY{n}{w1}\PY{o}{=}\PY{l+m+mi}{0}\PY{p}{)}\PY{p}{]}\PY{p}{)}\PY{o}{.}\PY{n}{rank}\PY{p}{(}\PY{p}{)} \PY{o}{==} \PY{l+m+mi}{1}\PY{p}{)}

\PY{c+c1}{\PYZsh{}\PYZsh{} Now we define three points on Cg:}

\PY{n}{Pg1}\PY{p}{,} \PY{n}{Pg2} \PY{o}{=} \PY{n}{Pg}\PY{o}{.}\PY{n}{subs}\PY{p}{(}\PY{p}{\PYZob{}}\PY{n}{w1}\PY{p}{:}\PY{n}{w1}\PY{p}{\PYZcb{}}\PY{p}{)}\PY{p}{,} \PY{n}{Pg}\PY{o}{.}\PY{n}{subs}\PY{p}{(}\PY{p}{\PYZob{}}\PY{n}{w1}\PY{p}{:}\PY{n}{w2}\PY{p}{\PYZcb{}}\PY{p}{)}
\PY{n}{Pg3} \PY{o}{=} \PY{n}{Pg}\PY{o}{.}\PY{n}{subs}\PY{p}{(}\PY{p}{\PYZob{}}\PY{n}{w1}\PY{p}{:}\PY{n}{m1}\PY{p}{\PYZcb{}}\PY{p}{)}

\PY{c+c1}{\PYZsh{}\PYZsh{} and we can assume w1, w2, m1 != 0.}
\end{Verbatim}
\end{tcolorbox}

    Now we construct the matrix of conditions of the points P1, P2, Pg1,
Pg2, Pg3 and we study its rank. We see that the matrix \(M\) has rank
\textless=9.

    \begin{tcolorbox}[breakable, size=fbox, boxrule=1pt, pad at break*=1mm,colback=cellbackground, colframe=cellborder]
\prompt{In}{incolor}{105}{\boxspacing}
\begin{Verbatim}[commandchars=\\\{\}]
\PY{c+c1}{\PYZsh{}\PYZsh{} the following matrix must have rank \PYZlt{}= 9:}

\PY{n}{M} \PY{o}{=} \PY{n}{condition\PYZus{}matrix}\PY{p}{(}\PY{p}{[}\PY{n}{P1}\PY{p}{,} \PY{n}{P2}\PY{p}{,} \PY{n}{Pg1}\PY{p}{,} \PY{n}{Pg2}\PY{p}{,} \PY{n}{Pg3}\PY{p}{]}\PY{p}{,} \PY{n}{S}\PY{p}{,} \PY{n}{standard}\PY{o}{=}\PY{l+s+s2}{\PYZdq{}}\PY{l+s+s2}{all}\PY{l+s+s2}{\PYZdq{}}\PY{p}{)}

\PY{c+c1}{\PYZsh{}\PYZsh{} Under this hypothesis, we have that we can extract from M}
\PY{c+c1}{\PYZsh{}\PYZsh{} the rows: 0, 1; 3, 4; 7, 8; 10, 11; 13, 14.}
\PY{c+c1}{\PYZsh{}\PYZsh{} (Remember that w1, w2, m1 are not zero).}

\PY{n}{MM1} \PY{o}{=} \PY{n}{M}\PY{o}{.}\PY{n}{matrix\PYZus{}from\PYZus{}rows}\PY{p}{(}\PY{p}{[}\PY{l+m+mi}{0}\PY{p}{,} \PY{l+m+mi}{1}\PY{p}{,} \PY{l+m+mi}{3}\PY{p}{,} \PY{l+m+mi}{4}\PY{p}{,} \PY{l+m+mi}{7}\PY{p}{,} \PY{l+m+mi}{8}\PY{p}{,} \PY{l+m+mi}{10}\PY{p}{,} \PY{l+m+mi}{11}\PY{p}{,} \PY{l+m+mi}{13}\PY{p}{,} \PY{l+m+mi}{14}\PY{p}{]}\PY{p}{)}

\PY{c+c1}{\PYZsh{}\PYZsh{} we make a copy of MM1}
\PY{n}{MM2} \PY{o}{=} \PY{n}{M}\PY{o}{.}\PY{n}{matrix\PYZus{}from\PYZus{}rows}\PY{p}{(}\PY{p}{[}\PY{l+m+mi}{0}\PY{p}{,} \PY{l+m+mi}{1}\PY{p}{,} \PY{l+m+mi}{3}\PY{p}{,} \PY{l+m+mi}{4}\PY{p}{,} \PY{l+m+mi}{7}\PY{p}{,} \PY{l+m+mi}{8}\PY{p}{,} \PY{l+m+mi}{10}\PY{p}{,} \PY{l+m+mi}{11}\PY{p}{,} \PY{l+m+mi}{13}\PY{p}{,} \PY{l+m+mi}{14}\PY{p}{]}\PY{p}{)}

\PY{c+c1}{\PYZsh{}\PYZsh{} Using the fact that the first two rows are good}
\PY{c+c1}{\PYZsh{}\PYZsh{} (MM2[0,0] is a non zero constant, and MM2[1, 4] is 1)}
\PY{c+c1}{\PYZsh{}\PYZsh{} we can reduce MM2 with elementary rows and columns operations.}

\PY{n}{MM2}\PY{o}{.}\PY{n}{rescale\PYZus{}row}\PY{p}{(}\PY{l+m+mi}{0}\PY{p}{,} \PY{n}{ii}\PY{o}{/}\PY{l+m+mi}{3}\PY{p}{)}
\PY{k}{for} \PY{n}{i} \PY{o+ow}{in} \PY{n+nb}{range}\PY{p}{(}\PY{l+m+mi}{2}\PY{p}{,} \PY{l+m+mi}{10}\PY{p}{)}\PY{p}{:}
    \PY{n}{MM2}\PY{o}{.}\PY{n}{add\PYZus{}multiple\PYZus{}of\PYZus{}row}\PY{p}{(}\PY{n}{i}\PY{p}{,} \PY{l+m+mi}{0}\PY{p}{,} \PY{o}{\PYZhy{}}\PY{n}{MM2}\PY{p}{[}\PY{n}{i}\PY{p}{]}\PY{p}{[}\PY{l+m+mi}{0}\PY{p}{]}\PY{p}{)}


\PY{k}{for} \PY{n}{i} \PY{o+ow}{in} \PY{n+nb}{range}\PY{p}{(}\PY{l+m+mi}{2}\PY{p}{,} \PY{l+m+mi}{10}\PY{p}{)}\PY{p}{:}
    \PY{n}{MM2}\PY{o}{.}\PY{n}{add\PYZus{}multiple\PYZus{}of\PYZus{}row}\PY{p}{(}\PY{n}{i}\PY{p}{,} \PY{l+m+mi}{1}\PY{p}{,} \PY{o}{\PYZhy{}}\PY{n}{MM2}\PY{p}{[}\PY{n}{i}\PY{p}{]}\PY{p}{[}\PY{l+m+mi}{4}\PY{p}{]}\PY{p}{)}

\PY{c+c1}{\PYZsh{}\PYZsh{} We extract from MM2 an order 8 square matrix, }
\PY{c+c1}{\PYZsh{}\PYZsh{} extracting the last 8 rows and the columns of position}
\PY{c+c1}{\PYZsh{}\PYZsh{} 1, 2, 3, 5, 6, 7, 8, 9. }
\PY{n}{MM2} \PY{o}{=} \PY{n}{MM2}\PY{o}{.}\PY{n}{matrix\PYZus{}from\PYZus{}rows\PYZus{}and\PYZus{}columns}\PY{p}{(}\PY{p}{[}\PY{l+m+mi}{2}\PY{p}{,} \PY{l+m+mi}{3}\PY{p}{,} \PY{l+m+mi}{4}\PY{p}{,} \PY{l+m+mi}{5}\PY{p}{,} \PY{l+m+mi}{6}\PY{p}{,} \PY{l+m+mi}{7}\PY{p}{,} \PY{l+m+mi}{8}\PY{p}{,} \PY{l+m+mi}{9}\PY{p}{]}\PY{p}{,} \PYZbs{}
                                       \PY{p}{[}\PY{l+m+mi}{1}\PY{p}{,} \PY{l+m+mi}{2}\PY{p}{,} \PY{l+m+mi}{3}\PY{p}{,} \PY{l+m+mi}{5}\PY{p}{,} \PY{l+m+mi}{6}\PY{p}{,} \PY{l+m+mi}{7}\PY{p}{,} \PY{l+m+mi}{8}\PY{p}{,} \PY{l+m+mi}{9}\PY{p}{]}\PY{p}{)}

\PY{c+c1}{\PYZsh{}\PYZsh{} The computation of det(MM2) gives 0 }
\PY{n}{dtMM2} \PY{o}{=} \PY{n}{MM2}\PY{o}{.}\PY{n}{det}\PY{p}{(}\PY{p}{)}
\PY{k}{assert}\PY{p}{(}\PY{n}{dtMM2} \PY{o}{==} \PY{n}{S}\PY{p}{(}\PY{l+m+mi}{0}\PY{p}{)}\PY{p}{)}

\PY{c+c1}{\PYZsh{}\PYZsh{} Hence M and MM1 have rank \PYZlt{}= 9.}
\end{Verbatim}
\end{tcolorbox}

    Now we want to see if \(M\) can have rank \textless{} 9. We will see
that \(M\) cannot have rank 8 or smaller:

    \begin{tcolorbox}[breakable, size=fbox, boxrule=1pt, pad at break*=1mm,colback=cellbackground, colframe=cellborder]
\prompt{In}{incolor}{106}{\boxspacing}
\begin{Verbatim}[commandchars=\\\{\}]
\PY{c+c1}{\PYZsh{}\PYZsh{} We want to see when M has rank \PYZlt{}=8, i.e. when MM2 has rank \PYZlt{}=6}
\PY{c+c1}{\PYZsh{}\PYZsh{} hence we compute the ideal of the order 7\PYZhy{}minors of MM2.}
\PY{c+c1}{\PYZsh{}\PYZsh{} Time of computation: 15\PYZsq{}\PYZsq{}}

\PY{n}{mm2\PYZus{}7} \PY{o}{=} \PY{n}{MM2}\PY{o}{.}\PY{n}{minors}\PY{p}{(}\PY{l+m+mi}{7}\PY{p}{)}
    
\PY{c+c1}{\PYZsh{}\PYZsh{} here is a list of factors that cannot be zero:}
\PY{n}{nonZeroFt} \PY{o}{=} \PY{p}{[}\PY{n}{l1}\PY{p}{,} \PY{n}{l1}\PY{o}{+}\PY{l+m+mi}{1}\PY{p}{,} \PY{n}{w1}\PY{p}{,} \PY{n}{w2}\PY{p}{,} \PY{n}{m1}\PY{p}{,} \PY{n}{w1}\PY{o}{\PYZhy{}}\PY{n}{w2}\PY{p}{,} \PY{n}{w1}\PY{o}{\PYZhy{}}\PY{n}{m1}\PY{p}{,} \PY{n}{w2}\PY{o}{\PYZhy{}}\PY{n}{m1}\PY{p}{]}


\PY{c+c1}{\PYZsh{}\PYZsh{} here from each order 7\PYZhy{}minor of MM2 (i.e. every element }
\PY{c+c1}{\PYZsh{}\PYZsh{} of mm2\PYZus{}7) we clear off the superfluous factors and }
\PY{c+c1}{\PYZsh{}\PYZsh{} we construct the ideal J7 of these polynomials (7\PYZsq{}\PYZsq{} of computation)}

\PY{n}{J7} \PY{o}{=} \PY{p}{[}\PY{p}{]}
\PY{k}{for} \PY{n}{ff} \PY{o+ow}{in} \PY{n}{mm2\PYZus{}7}\PY{p}{:}
    \PY{n}{J7}\PY{o}{.}\PY{n}{append}\PY{p}{(}\PY{n}{prod}\PY{p}{(}\PY{n}{erase\PYZus{}superfluous\PYZus{}factors}\PY{p}{(}\PY{n}{ff}\PY{p}{,} \PY{n}{nonZeroFt}\PY{p}{)}\PY{p}{)}\PY{p}{)}

\PY{c+c1}{\PYZsh{}\PYZsh{} since the ideal J7 is (1), we have }
\PY{c+c1}{\PYZsh{}\PYZsh{} that the matrix MM2 cannot have rank 6 or smaller, hence}
\PY{c+c1}{\PYZsh{}\PYZsh{} M cannot have rank 8 or smaller:}
\PY{k}{assert}\PY{p}{(}\PY{n}{S}\PY{o}{.}\PY{n}{ideal}\PY{p}{(}\PY{n}{J7}\PY{p}{)} \PY{o}{==} \PY{n}{S}\PY{o}{.}\PY{n}{ideal}\PY{p}{(}\PY{n}{S}\PY{p}{(}\PY{l+m+mi}{1}\PY{p}{)}\PY{p}{)}\PY{p}{)}
\end{Verbatim}
\end{tcolorbox}

    The above computation show that the matrix \(M\) has rank 9. Therefore
there is only one cubic given by \(M\) and is obtained from the
determinant of the matrix Mc below. We get therefore the cubic Cbc
below.

    \begin{tcolorbox}[breakable, size=fbox, boxrule=1pt, pad at break*=1mm,colback=cellbackground, colframe=cellborder]
\prompt{In}{incolor}{107}{\boxspacing}
\begin{Verbatim}[commandchars=\\\{\}]
\PY{n}{Mc} \PY{o}{=} \PY{n}{MM1}\PY{o}{.}\PY{n}{matrix\PYZus{}from\PYZus{}rows}\PY{p}{(}\PY{n+nb}{range}\PY{p}{(}\PY{l+m+mi}{9}\PY{p}{)}\PY{p}{)}  
\PY{n}{Mc} \PY{o}{=} \PY{n}{Mc}\PY{o}{.}\PY{n}{stack}\PY{p}{(}\PY{n}{matrix}\PY{p}{(}\PY{p}{[}\PY{n}{mon}\PY{p}{]}\PY{p}{)}\PY{p}{)}

\PY{n}{dtMc} \PY{o}{=} \PY{n}{Mc}\PY{o}{.}\PY{n}{det}\PY{p}{(}\PY{p}{)}

\PY{c+c1}{\PYZsh{}\PYZsh{} from dtMc we erase useless factors:}

\PY{n}{dt1} \PY{o}{=} \PY{n}{erase\PYZus{}superfluous\PYZus{}factors}\PY{p}{(}\PY{n}{dtMc}\PY{p}{,} \PY{n}{nonZeroFt}\PY{p}{)}

\PY{c+c1}{\PYZsh{}\PYZsh{} We get that dt1 is a list of 4 elements, which are:}
\PY{n}{Ls} \PY{o}{=} \PY{p}{[}\PY{n}{w1} \PY{o}{+} \PY{n}{w2}\PY{p}{,} \PY{n}{z}\PY{p}{,} \PY{n}{l1}\PY{o}{*}\PY{n}{m1}\PY{o}{\PYZca{}}\PY{l+m+mi}{2} \PY{o}{+} \PY{n}{m1}\PY{o}{\PYZca{}}\PY{l+m+mi}{2} \PY{o}{+} \PY{l+m+mi}{1}\PY{p}{,} \PYZbs{}
      \PY{n}{x}\PY{o}{\PYZca{}}\PY{l+m+mi}{2}\PY{o}{*}\PY{n}{l1} \PY{o}{+} \PY{n}{y}\PY{o}{\PYZca{}}\PY{l+m+mi}{2}\PY{o}{*}\PY{n}{l1} \PY{o}{+} \PY{l+m+mi}{2}\PY{o}{/}\PY{l+m+mi}{3}\PY{o}{*}\PY{n}{z}\PY{o}{\PYZca{}}\PY{l+m+mi}{2}\PY{o}{*}\PY{n}{l1} \PY{o}{+} \PY{n}{x}\PY{o}{\PYZca{}}\PY{l+m+mi}{2} \PY{o}{+} \PY{n}{y}\PY{o}{\PYZca{}}\PY{l+m+mi}{2} \PY{o}{+} \PY{n}{z}\PY{o}{\PYZca{}}\PY{l+m+mi}{2}\PY{p}{]}

\PY{k}{assert}\PY{p}{(}\PY{n}{dt1} \PY{o}{==} \PY{n}{Ls}\PY{p}{)}

\PY{c+c1}{\PYZsh{}\PYZsh{} we select the two factors which are a line and a conic and whose }
\PY{c+c1}{\PYZsh{}\PYZsh{} product is the desired cubic.}

\PY{c+c1}{\PYZsh{}\PYZsh{} The factor dt1[1] is the line r12 given by P1+P2:}

\PY{n}{r12} \PY{o}{=} \PY{o}{\PYZhy{}}\PY{n}{det}\PY{p}{(}\PY{n}{matrix}\PY{p}{(}\PY{p}{[}\PY{n}{P1}\PY{p}{,} \PY{n}{P2}\PY{p}{,} \PY{p}{(}\PY{n}{x}\PY{p}{,} \PY{n}{y}\PY{p}{,} \PY{n}{z}\PY{p}{)}\PY{p}{]}\PY{p}{)}\PY{p}{)}\PY{o}{/}\PY{p}{(}\PY{l+m+mi}{2}\PY{o}{*}\PY{n}{ii}\PY{p}{)}

\PY{k}{assert}\PY{p}{(}\PY{n}{dt1}\PY{p}{[}\PY{l+m+mi}{1}\PY{p}{]} \PY{o}{==} \PY{n}{r12}\PY{p}{)}

\PY{c+c1}{\PYZsh{}\PYZsh{} The factor Lt[3] is the l1*r12\PYZca{}2+3*(l1+1)*Ciso:}

\PY{k}{assert}\PY{p}{(}\PY{n}{dt1}\PY{p}{[}\PY{l+m+mi}{3}\PY{p}{]} \PY{o}{==} \PY{p}{(}\PY{n}{l1}\PY{o}{+}\PY{l+m+mi}{1}\PY{p}{)}\PY{o}{*}\PY{n}{Ciso}\PY{o}{\PYZhy{}}\PY{l+m+mi}{1}\PY{o}{/}\PY{l+m+mi}{3}\PY{o}{*}\PY{n}{l1}\PY{o}{*}\PY{n}{r12}\PY{o}{\PYZca{}}\PY{l+m+mi}{2}\PY{p}{)}

\PY{c+c1}{\PYZsh{}\PYZsh{} The cubic is therefore:}

\PY{n}{Cbc} \PY{o}{=} \PY{n}{dt1}\PY{p}{[}\PY{l+m+mi}{1}\PY{p}{]}\PY{o}{*}\PY{n}{dt1}\PY{p}{[}\PY{l+m+mi}{3}\PY{p}{]}

\PY{k}{assert}\PY{p}{(}\PY{n}{Cbc} \PY{o}{==} \PY{n}{r12}\PY{o}{*}\PY{p}{(}\PY{p}{(}\PY{n}{l1}\PY{o}{+}\PY{l+m+mi}{1}\PY{p}{)}\PY{o}{*}\PY{n}{Ciso}\PY{o}{\PYZhy{}}\PY{l+m+mi}{1}\PY{o}{/}\PY{l+m+mi}{3}\PY{o}{*}\PY{n}{l1}\PY{o}{*}\PY{n}{r12}\PY{o}{\PYZca{}}\PY{l+m+mi}{2}\PY{p}{)}\PY{p}{)}
\end{Verbatim}
\end{tcolorbox}

    The above cubic is of the form desired, i.e.~is of the form
r\emph{(u}r\^{}2+v*Ciso) for suitable u, v in K. Now we compute the
eigenpoints of Cbc and we see that they are of the desired form:

    \begin{tcolorbox}[breakable, size=fbox, boxrule=1pt, pad at break*=1mm,colback=cellbackground, colframe=cellborder]
\prompt{In}{incolor}{108}{\boxspacing}
\begin{Verbatim}[commandchars=\\\{\}]
\PY{n}{Je} \PY{o}{=} \PY{n}{S}\PY{o}{.}\PY{n}{ideal}\PY{p}{(}
    \PY{n}{matrix}\PY{p}{(}
        \PY{p}{[}
            \PY{p}{[}\PY{n}{Cbc}\PY{o}{.}\PY{n}{derivative}\PY{p}{(}\PY{n}{x}\PY{p}{)}\PY{p}{,} \PY{n}{Cbc}\PY{o}{.}\PY{n}{derivative}\PY{p}{(}\PY{n}{y}\PY{p}{)}\PY{p}{,} \PY{n}{Cbc}\PY{o}{.}\PY{n}{derivative}\PY{p}{(}\PY{n}{z}\PY{p}{)}\PY{p}{]}\PY{p}{,} 
            \PY{p}{[}\PY{n}{x}\PY{p}{,} \PY{n}{y}\PY{p}{,} \PY{n}{z}\PY{p}{]}
        \PY{p}{]}
    \PY{p}{)}\PY{o}{.}\PY{n}{minors}\PY{p}{(}\PY{l+m+mi}{2}\PY{p}{)}
\PY{p}{)}

\PY{c+c1}{\PYZsh{}\PYZsh{} we get two components: }
\PY{n}{PDe} \PY{o}{=} \PY{n}{Je}\PY{o}{.}\PY{n}{primary\PYZus{}decomposition}\PY{p}{(}\PY{p}{)}

\PY{k}{assert}\PY{p}{(}\PY{n+nb}{len}\PY{p}{(}\PY{n}{PDe}\PY{p}{)} \PY{o}{==} \PY{l+m+mi}{2}\PY{p}{)}

\PY{k}{assert}\PY{p}{(}\PY{n}{PDe}\PY{p}{[}\PY{l+m+mi}{0}\PY{p}{]} \PY{o}{==} \PY{n}{S}\PY{o}{.}\PY{n}{ideal}\PY{p}{(}\PY{n}{rtg1}\PY{p}{,} \PY{n}{rtg2}\PY{p}{)}\PY{p}{)}

\PY{k}{assert}\PY{p}{(}\PY{n}{PDe}\PY{p}{[}\PY{l+m+mi}{1}\PY{p}{]} \PY{o}{==} \PY{n}{S}\PY{o}{.}\PY{n}{ideal}\PY{p}{(}\PY{n}{Ciso}\PY{o}{+}\PY{n}{l1}\PY{o}{*}\PY{n}{rtg1}\PY{o}{*}\PY{n}{rtg2}\PY{p}{)}\PY{p}{)}
\end{Verbatim}
\end{tcolorbox}

    This concludes the case \(P_2 =(1:-i:0)\) and there are no differences
w.r.t. the case \(P_2 \not= (1:-i:0)\)

    \hypertarget{gamma-osculating-the-isotropic-conic-in-p_11i0.}{%
\subsection{\texorpdfstring{\(\Gamma\) osculating the isotropic conic in
\(P_1=(1:i:0)\).}{\textbackslash Gamma osculating the isotropic conic in P\_1=(1:i:0).}}\label{gamma-osculating-the-isotropic-conic-in-p_11i0.}}

    Time of computations: about 800 seconds.

Here we want to see the case in which a cubic has among its eigenpoints,
a conic which is osculating \(\mathcal{Q}_{\mathrm{iso}}\) in the point
\(P_1=(1:i:0)\). We define \(P_2\) a generic point on
\(\mathcal{Q}_{\mathrm{iso}}\) different from \(P_1\) (\(P_2\) depends
on the parameter \(u_1\)). We define the tangent lines to
\(\mathcal{Q}_{\mathrm{iso}}\) in \(P_1\) and \(P_2\), we define the
pencil of conics Cg that pass through \(P_2\) and are osculating
\(\mathcal{Q}_{\mathrm{iso}}\) in \(P_1\). Then we define 6 ``random''
points \(P_3, \dots, P_8\) on Cg

    \begin{tcolorbox}[breakable, size=fbox, boxrule=1pt, pad at break*=1mm,colback=cellbackground, colframe=cellborder]
\prompt{In}{incolor}{109}{\boxspacing}
\begin{Verbatim}[commandchars=\\\{\}]
\PY{n}{P1} \PY{o}{=} \PY{n}{vector}\PY{p}{(}\PY{n}{S}\PY{p}{,} \PY{p}{(}\PY{l+m+mi}{1}\PY{p}{,} \PY{n}{ii}\PY{p}{,} \PY{l+m+mi}{0}\PY{p}{)}\PY{p}{)}
\PY{n}{P2} \PY{o}{=} \PY{n}{vector}\PY{p}{(}\PY{n}{S}\PY{p}{,} \PY{p}{(}\PY{p}{(}\PY{o}{\PYZhy{}}\PY{n}{ii}\PY{p}{)}\PY{o}{*}\PY{n}{u1}\PY{o}{\PYZca{}}\PY{l+m+mi}{2} \PY{o}{+} \PY{p}{(}\PY{o}{\PYZhy{}}\PY{n}{ii}\PY{p}{)}\PY{p}{,} \PY{n}{u1}\PY{o}{\PYZca{}}\PY{l+m+mi}{2} \PY{o}{\PYZhy{}} \PY{l+m+mi}{1}\PY{p}{,} \PY{l+m+mi}{2}\PY{o}{*}\PY{n}{u1}\PY{p}{)}\PY{p}{)}

\PY{c+c1}{\PYZsh{}\PYZsh{} Tangent line to Ciso in P1:}
\PY{n}{rtg} \PY{o}{=} \PY{n}{scalar\PYZus{}product}\PY{p}{(}\PY{n}{P1}\PY{p}{,} \PY{n}{vector}\PY{p}{(}\PY{p}{(}\PY{n}{x}\PY{p}{,} \PY{n}{y}\PY{p}{,} \PY{n}{z}\PY{p}{)}\PY{p}{)}\PY{o}{\PYZhy{}}\PY{n}{P1}\PY{p}{)}

\PY{c+c1}{\PYZsh{}\PYZsh{} line P1 + P2}
\PY{n}{r12} \PY{o}{=} \PY{n}{matrix}\PY{p}{(}\PY{p}{[}\PY{n}{P1}\PY{p}{,} \PY{n}{P2}\PY{p}{,} \PY{p}{(}\PY{n}{x}\PY{p}{,} \PY{n}{y}\PY{p}{,} \PY{n}{z}\PY{p}{)}\PY{p}{]}\PY{p}{)}\PY{o}{.}\PY{n}{det}\PY{p}{(}\PY{p}{)}

\PY{c+c1}{\PYZsh{}\PYZsh{} Pencil of conics osculating Ciso in P1 and passing through P2:}
\PY{n}{Cg} \PY{o}{=} \PY{n}{Ciso} \PY{o}{+} \PY{n}{l1}\PY{o}{*}\PY{n}{rtg}\PY{o}{*}\PY{n}{r12}

\PY{c+c1}{\PYZsh{}\PYZsh{} If l1 = 0, Cg is Ciso, so we can assume}
\PY{c+c1}{\PYZsh{}\PYZsh{} l1 != 0}
\PY{k}{assert}\PY{p}{(}\PY{n}{Cg}\PY{o}{.}\PY{n}{subs}\PY{p}{(}\PY{n}{l1}\PY{o}{=}\PY{l+m+mi}{0}\PY{p}{)} \PY{o}{==} \PY{n}{Ciso}\PY{p}{)}

\PY{c+c1}{\PYZsh{}\PYZsh{} construction of a generic point (different from (1, ii, 0)) on Cg:}

\PY{n}{foo} \PY{o}{=} \PY{n}{Cg}\PY{o}{.}\PY{n}{subs}\PY{p}{(}\PY{n}{y}\PY{o}{=}\PY{n}{ii}\PY{o}{*}\PY{n}{x}\PY{o}{+}\PY{n}{w1}\PY{o}{*}\PY{n}{z}\PY{p}{)}\PY{o}{.}\PY{n}{factor}\PY{p}{(}\PY{p}{)}\PY{p}{[}\PY{o}{\PYZhy{}}\PY{l+m+mi}{1}\PY{p}{]}\PY{p}{[}\PY{l+m+mi}{0}\PY{p}{]}

\PY{c+c1}{\PYZsh{}\PYZsh{} generic point of Cg (it depends on the parameter w1):}
\PY{n}{Pg} \PY{o}{=} \PY{n}{vector}\PY{p}{(}
    \PY{n}{S}\PY{p}{,} 
    \PY{p}{(}
        \PY{n}{foo}\PY{o}{.}\PY{n}{coefficient}\PY{p}{(}\PY{n}{z}\PY{p}{)}\PY{p}{,} 
        \PY{n}{ii}\PY{o}{*}\PY{p}{(}\PY{n}{foo}\PY{o}{.}\PY{n}{coefficient}\PY{p}{(}\PY{n}{z}\PY{p}{)}\PY{p}{)}\PY{o}{+}\PY{n}{w1}\PY{o}{*}\PY{p}{(}\PY{o}{\PYZhy{}}\PY{n}{foo}\PY{o}{.}\PY{n}{coefficient}\PY{p}{(}\PY{n}{x}\PY{p}{)}\PY{p}{)}\PY{p}{,} 
        \PY{o}{\PYZhy{}}\PY{n}{foo}\PY{o}{.}\PY{n}{coefficient}\PY{p}{(}\PY{n}{x}\PY{p}{)}
    \PY{p}{)}
\PY{p}{)}

\PY{c+c1}{\PYZsh{}\PYZsh{} Pg is on Cg:}
\PY{k}{assert}\PY{p}{(}\PY{n}{Cg}\PY{o}{.}\PY{n}{subs}\PY{p}{(}\PY{n}{substitution}\PY{p}{(}\PY{n}{Pg}\PY{p}{)}\PY{p}{)} \PY{o}{==} \PY{l+m+mi}{0}\PY{p}{)}


\PY{c+c1}{\PYZsh{}\PYZsh{} If w1 = 0, then Pg = P1, hence we can assume w1 != 0.}
\PY{k}{assert}\PY{p}{(}\PY{n}{matrix}\PY{p}{(}\PY{p}{[}\PY{n}{P1}\PY{p}{,} \PY{n}{Pg}\PY{o}{.}\PY{n}{subs}\PY{p}{(}\PY{n}{w1}\PY{o}{=}\PY{l+m+mi}{0}\PY{p}{)}\PY{p}{]}\PY{p}{)}\PY{o}{.}\PY{n}{rank}\PY{p}{(}\PY{p}{)} \PY{o}{==} \PY{l+m+mi}{1}\PY{p}{)}

\PY{c+c1}{\PYZsh{}\PYZsh{} Now we define six specific points on Cg:}

\PY{n}{P3} \PY{o}{=} \PY{n}{Pg}\PY{o}{.}\PY{n}{subs}\PY{p}{(}\PY{n}{w1}\PY{o}{=}\PY{o}{\PYZhy{}}\PY{l+m+mi}{1}\PY{p}{)}
\PY{n}{P4} \PY{o}{=} \PY{n}{Pg}\PY{o}{.}\PY{n}{subs}\PY{p}{(}\PY{n}{w1}\PY{o}{=}\PY{o}{\PYZhy{}}\PY{l+m+mi}{2}\PY{p}{)}
\PY{n}{P5} \PY{o}{=} \PY{n}{Pg}\PY{o}{.}\PY{n}{subs}\PY{p}{(}\PY{n}{w1}\PY{o}{=}\PY{o}{\PYZhy{}}\PY{l+m+mi}{3}\PY{p}{)}
\PY{n}{P6} \PY{o}{=} \PY{n}{Pg}\PY{o}{.}\PY{n}{subs}\PY{p}{(}\PY{n}{w1}\PY{o}{=}\PY{l+m+mi}{4}\PY{p}{)}
\PY{n}{P7} \PY{o}{=} \PY{n}{Pg}\PY{o}{.}\PY{n}{subs}\PY{p}{(}\PY{n}{w1}\PY{o}{=}\PY{o}{\PYZhy{}}\PY{l+m+mi}{4}\PY{p}{)}
\PY{n}{P8} \PY{o}{=} \PY{n}{Pg}\PY{o}{.}\PY{n}{subs}\PY{p}{(}\PY{n}{w1}\PY{o}{=}\PY{o}{\PYZhy{}}\PY{l+m+mi}{6}\PY{p}{)}
\end{Verbatim}
\end{tcolorbox}

    Then we construct two matrices, MT1 and MT2 of conditions that must have
rank \(\leq 9\). We manipulate these matrices and at the end of the
block below, we have two matrices (called again MT1 and MT2) that, if
\(P_1, \dotsc, P_8\) are eigenpoints, must have rank \(\leq 8\).

    \begin{tcolorbox}[breakable, size=fbox, boxrule=1pt, pad at break*=1mm,colback=cellbackground, colframe=cellborder]
\prompt{In}{incolor}{110}{\boxspacing}
\begin{Verbatim}[commandchars=\\\{\}]
\PY{n}{MT1} \PY{o}{=} \PY{n}{condition\PYZus{}matrix}\PY{p}{(}\PY{p}{[}\PY{n}{P1}\PY{p}{,} \PY{n}{P2}\PY{p}{,} \PY{n}{P3}\PY{p}{,} \PY{n}{P4}\PY{p}{,} \PY{n}{P5}\PY{p}{]}\PY{p}{,} \PY{n}{S}\PY{p}{,} \PY{n}{standard}\PY{o}{=}\PY{l+s+s2}{\PYZdq{}}\PY{l+s+s2}{all}\PY{l+s+s2}{\PYZdq{}}\PY{p}{)}
\PY{n}{MT2} \PY{o}{=} \PY{n}{condition\PYZus{}matrix}\PY{p}{(}\PY{p}{[}\PY{n}{P1}\PY{p}{,} \PY{n}{P2}\PY{p}{,} \PY{n}{P6}\PY{p}{,} \PY{n}{P7}\PY{p}{,} \PY{n}{P8}\PY{p}{]}\PY{p}{,} \PY{n}{S}\PY{p}{,} \PY{n}{standard}\PY{o}{=}\PY{l+s+s2}{\PYZdq{}}\PY{l+s+s2}{all}\PY{l+s+s2}{\PYZdq{}}\PY{p}{)}

\PY{c+c1}{\PYZsh{}\PYZsh{} Since the 0\PYZhy{}row of MT1 and MT2 is numeric and also the 1\PYZhy{}row }
\PY{c+c1}{\PYZsh{}\PYZsh{} of MT1 and MT2 is numeric, we can use them to simplify MT1 and MT2. }

\PY{k}{assert}\PY{p}{(}\PY{n}{MT1}\PY{p}{[}\PY{l+m+mi}{0}\PY{p}{]} \PY{o}{==} \PY{n}{vector}\PY{p}{(}\PY{n}{S}\PY{p}{,} \PY{p}{(}\PY{p}{(}\PY{o}{\PYZhy{}}\PY{l+m+mi}{3}\PY{o}{*}\PY{n}{ii}\PY{p}{)}\PY{p}{,} \PY{l+m+mi}{3}\PY{p}{,} \PY{p}{(}\PY{l+m+mi}{3}\PY{o}{*}\PY{n}{ii}\PY{p}{)}\PY{p}{,} \PY{o}{\PYZhy{}}\PY{l+m+mi}{3}\PY{p}{,} \PY{l+m+mi}{0}\PY{p}{,} \PY{l+m+mi}{0}\PY{p}{,} \PY{l+m+mi}{0}\PY{p}{,} \PY{l+m+mi}{0}\PY{p}{,} \PY{l+m+mi}{0}\PY{p}{,} \PY{l+m+mi}{0}\PY{p}{)}\PY{p}{)}\PY{p}{)}
\PY{k}{assert}\PY{p}{(}\PY{n}{MT2}\PY{p}{[}\PY{l+m+mi}{0}\PY{p}{]} \PY{o}{==} \PY{n}{vector}\PY{p}{(}\PY{n}{S}\PY{p}{,} \PY{p}{(}\PY{p}{(}\PY{o}{\PYZhy{}}\PY{l+m+mi}{3}\PY{o}{*}\PY{n}{ii}\PY{p}{)}\PY{p}{,} \PY{l+m+mi}{3}\PY{p}{,} \PY{p}{(}\PY{l+m+mi}{3}\PY{o}{*}\PY{n}{ii}\PY{p}{)}\PY{p}{,} \PY{o}{\PYZhy{}}\PY{l+m+mi}{3}\PY{p}{,} \PY{l+m+mi}{0}\PY{p}{,} \PY{l+m+mi}{0}\PY{p}{,} \PY{l+m+mi}{0}\PY{p}{,} \PY{l+m+mi}{0}\PY{p}{,} \PY{l+m+mi}{0}\PY{p}{,} \PY{l+m+mi}{0}\PY{p}{)}\PY{p}{)}\PY{p}{)}
\PY{k}{assert}\PY{p}{(}\PY{n}{MT1}\PY{p}{[}\PY{l+m+mi}{1}\PY{p}{]} \PY{o}{==} \PY{n}{vector}\PY{p}{(}\PY{n}{S}\PY{p}{,} \PY{p}{(}\PY{l+m+mi}{0}\PY{p}{,} \PY{l+m+mi}{0}\PY{p}{,} \PY{l+m+mi}{0}\PY{p}{,} \PY{l+m+mi}{0}\PY{p}{,} \PY{l+m+mi}{1}\PY{p}{,} \PY{n}{ii}\PY{p}{,} \PY{o}{\PYZhy{}}\PY{l+m+mi}{1}\PY{p}{,} \PY{l+m+mi}{0}\PY{p}{,} \PY{l+m+mi}{0}\PY{p}{,} \PY{l+m+mi}{0}\PY{p}{)}\PY{p}{)}\PY{p}{)}
\PY{k}{assert}\PY{p}{(}\PY{n}{MT2}\PY{p}{[}\PY{l+m+mi}{1}\PY{p}{]} \PY{o}{==} \PY{n}{vector}\PY{p}{(}\PY{n}{S}\PY{p}{,} \PY{p}{(}\PY{l+m+mi}{0}\PY{p}{,} \PY{l+m+mi}{0}\PY{p}{,} \PY{l+m+mi}{0}\PY{p}{,} \PY{l+m+mi}{0}\PY{p}{,} \PY{l+m+mi}{1}\PY{p}{,} \PY{n}{ii}\PY{p}{,} \PY{o}{\PYZhy{}}\PY{l+m+mi}{1}\PY{p}{,} \PY{l+m+mi}{0}\PY{p}{,} \PY{l+m+mi}{0}\PY{p}{,} \PY{l+m+mi}{0}\PY{p}{)}\PY{p}{)}\PY{p}{)}

\PY{n}{MT1}\PY{o}{.}\PY{n}{rescale\PYZus{}row}\PY{p}{(}\PY{l+m+mi}{0}\PY{p}{,} \PY{n}{ii}\PY{o}{/}\PY{l+m+mi}{3}\PY{p}{)}
\PY{k}{for} \PY{n}{i} \PY{o+ow}{in} \PY{n+nb}{range}\PY{p}{(}\PY{l+m+mi}{2}\PY{p}{,} \PY{l+m+mi}{10}\PY{p}{)}\PY{p}{:}
    \PY{n}{MT1}\PY{o}{.}\PY{n}{add\PYZus{}multiple\PYZus{}of\PYZus{}row}\PY{p}{(}\PY{n}{i}\PY{p}{,} \PY{l+m+mi}{0}\PY{p}{,} \PY{o}{\PYZhy{}}\PY{n}{MT1}\PY{p}{[}\PY{n}{i}\PY{p}{]}\PY{p}{[}\PY{l+m+mi}{0}\PY{p}{]}\PY{p}{)}


\PY{n}{MT1}\PY{o}{.}\PY{n}{rescale\PYZus{}row}\PY{p}{(}\PY{l+m+mi}{1}\PY{p}{,} \PY{l+m+mi}{1}\PY{p}{)}
\PY{k}{for} \PY{n}{i} \PY{o+ow}{in} \PY{n+nb}{range}\PY{p}{(}\PY{l+m+mi}{2}\PY{p}{,} \PY{l+m+mi}{10}\PY{p}{)}\PY{p}{:}
    \PY{n}{MT1}\PY{o}{.}\PY{n}{add\PYZus{}multiple\PYZus{}of\PYZus{}row}\PY{p}{(}\PY{n}{i}\PY{p}{,} \PY{l+m+mi}{1}\PY{p}{,} \PY{o}{\PYZhy{}}\PY{n}{MT1}\PY{p}{[}\PY{n}{i}\PY{p}{]}\PY{p}{[}\PY{l+m+mi}{4}\PY{p}{]}\PY{p}{)}


\PY{n}{MT1} \PY{o}{=} \PY{n}{MT1}\PY{o}{.}\PY{n}{matrix\PYZus{}from\PYZus{}columns}\PY{p}{(}\PY{p}{[}\PY{l+m+mi}{1}\PY{p}{,} \PY{l+m+mi}{2}\PY{p}{,} \PY{l+m+mi}{3}\PY{p}{,} \PY{l+m+mi}{5}\PY{p}{,} \PY{l+m+mi}{6}\PY{p}{,} \PY{l+m+mi}{7}\PY{p}{,} \PY{l+m+mi}{8}\PY{p}{,} \PY{l+m+mi}{9}\PY{p}{]}\PY{p}{)}
\PY{n}{MT1} \PY{o}{=} \PY{n}{MT1}\PY{o}{.}\PY{n}{matrix\PYZus{}from\PYZus{}rows}\PY{p}{(}\PY{n+nb}{range}\PY{p}{(}\PY{l+m+mi}{3}\PY{p}{,} \PY{l+m+mi}{15}\PY{p}{)}\PY{p}{)}

\PY{c+c1}{\PYZsh{}\PYZsh{} same for MT2:}

\PY{n}{MT2}\PY{o}{.}\PY{n}{rescale\PYZus{}row}\PY{p}{(}\PY{l+m+mi}{0}\PY{p}{,} \PY{n}{ii}\PY{o}{/}\PY{l+m+mi}{3}\PY{p}{)}
\PY{k}{for} \PY{n}{i} \PY{o+ow}{in} \PY{n+nb}{range}\PY{p}{(}\PY{l+m+mi}{2}\PY{p}{,} \PY{l+m+mi}{10}\PY{p}{)}\PY{p}{:}
    \PY{n}{MT2}\PY{o}{.}\PY{n}{add\PYZus{}multiple\PYZus{}of\PYZus{}row}\PY{p}{(}\PY{n}{i}\PY{p}{,} \PY{l+m+mi}{0}\PY{p}{,} \PY{o}{\PYZhy{}}\PY{n}{MT2}\PY{p}{[}\PY{n}{i}\PY{p}{]}\PY{p}{[}\PY{l+m+mi}{0}\PY{p}{]}\PY{p}{)}


\PY{n}{MT2}\PY{o}{.}\PY{n}{rescale\PYZus{}row}\PY{p}{(}\PY{l+m+mi}{1}\PY{p}{,} \PY{l+m+mi}{1}\PY{p}{)}
\PY{k}{for} \PY{n}{i} \PY{o+ow}{in} \PY{n+nb}{range}\PY{p}{(}\PY{l+m+mi}{2}\PY{p}{,} \PY{l+m+mi}{10}\PY{p}{)}\PY{p}{:}
    \PY{n}{MT2}\PY{o}{.}\PY{n}{add\PYZus{}multiple\PYZus{}of\PYZus{}row}\PY{p}{(}\PY{n}{i}\PY{p}{,} \PY{l+m+mi}{1}\PY{p}{,} \PY{o}{\PYZhy{}}\PY{n}{MT2}\PY{p}{[}\PY{n}{i}\PY{p}{]}\PY{p}{[}\PY{l+m+mi}{4}\PY{p}{]}\PY{p}{)}


\PY{n}{MT2} \PY{o}{=} \PY{n}{MT2}\PY{o}{.}\PY{n}{matrix\PYZus{}from\PYZus{}columns}\PY{p}{(}\PY{p}{[}\PY{l+m+mi}{1}\PY{p}{,} \PY{l+m+mi}{2}\PY{p}{,} \PY{l+m+mi}{3}\PY{p}{,} \PY{l+m+mi}{5}\PY{p}{,} \PY{l+m+mi}{6}\PY{p}{,} \PY{l+m+mi}{7}\PY{p}{,} \PY{l+m+mi}{8}\PY{p}{,} \PY{l+m+mi}{9}\PY{p}{]}\PY{p}{)}
\PY{n}{MT2} \PY{o}{=} \PY{n}{MT2}\PY{o}{.}\PY{n}{matrix\PYZus{}from\PYZus{}rows}\PY{p}{(}\PY{n+nb}{range}\PY{p}{(}\PY{l+m+mi}{3}\PY{p}{,} \PY{l+m+mi}{15}\PY{p}{)}\PY{p}{)}

\PY{c+c1}{\PYZsh{}\PYZsh{} If P1, P2, P3, P4, P5, P6, P7, P8 are eigenpoints, }
\PY{c+c1}{\PYZsh{}\PYZsh{} the order 8 minors of MT1 and of MT2 must all be zero.}
\end{Verbatim}
\end{tcolorbox}

    Now we compute the two ideals of the minors of order 8 of MT1 and MT2.
The next computations require about 200 seconds. At the end of the
computations below, we have two ideals, j1 and j2 which are such that
when the parameters are a zero of j1+j2, then MT1 and MT2 have rank
\textless=8

    \begin{tcolorbox}[breakable, size=fbox, boxrule=1pt, pad at break*=1mm,colback=cellbackground, colframe=cellborder]
\prompt{In}{incolor}{111}{\boxspacing}
\begin{Verbatim}[commandchars=\\\{\}]
\PY{c+c1}{\PYZsh{}\PYZsh{} this computation requires about 63\PYZdq{}:}
\PY{n+nb}{print}\PY{p}{(}\PY{l+s+s2}{\PYZdq{}}\PY{l+s+s2}{About 63 seconds of computation...}\PY{l+s+s2}{\PYZdq{}}\PY{p}{)}
\PY{n}{sleep}\PY{p}{(}\PY{l+m+mi}{1}\PY{p}{)}
\PY{n}{ttA} \PY{o}{=} \PY{n}{cputime}\PY{p}{(}\PY{p}{)}
\PY{n}{mt1\PYZus{}8} \PY{o}{=} \PY{n}{MT1}\PY{o}{.}\PY{n}{minors}\PY{p}{(}\PY{l+m+mi}{8}\PY{p}{)}
\PY{n+nb}{print}\PY{p}{(}\PY{n}{cputime}\PY{p}{(}\PY{p}{)}\PY{o}{\PYZhy{}}\PY{n}{ttA}\PY{p}{)}


\PY{c+c1}{\PYZsh{}\PYZsh{} this computation requires about 80\PYZdq{}}
\PY{n+nb}{print}\PY{p}{(}\PY{l+s+s2}{\PYZdq{}}\PY{l+s+s2}{About 80 seconds of computation...}\PY{l+s+s2}{\PYZdq{}}\PY{p}{)}
\PY{n}{sleep}\PY{p}{(}\PY{l+m+mi}{1}\PY{p}{)}
\PY{n}{ttA} \PY{o}{=} \PY{n}{cputime}\PY{p}{(}\PY{p}{)}
\PY{n}{mt2\PYZus{}8} \PY{o}{=} \PY{n}{MT2}\PY{o}{.}\PY{n}{minors}\PY{p}{(}\PY{l+m+mi}{8}\PY{p}{)}
\PY{n+nb}{print}\PY{p}{(}\PY{n}{cputime}\PY{p}{(}\PY{p}{)}\PY{o}{\PYZhy{}}\PY{n}{ttA}\PY{p}{)}


\PY{c+c1}{\PYZsh{}\PYZsh{} To speed up the computations, we keep separated the two ideals}

\PY{c+c1}{\PYZsh{}\PYZsh{} The following block requires 36 seconds.}
\PY{n+nb}{print}\PY{p}{(}\PY{l+s+s2}{\PYZdq{}}\PY{l+s+s2}{Now 36 seconds of computations}\PY{l+s+s2}{\PYZdq{}}\PY{p}{)}
\PY{n}{sleep}\PY{p}{(}\PY{l+m+mi}{1}\PY{p}{)}
\PY{n}{JJ1} \PY{o}{=} \PY{n}{S}\PY{o}{.}\PY{n}{ideal}\PY{p}{(}\PY{n}{mt1\PYZus{}8}\PY{p}{)}
\PY{n}{JJ2} \PY{o}{=} \PY{n}{S}\PY{o}{.}\PY{n}{ideal}\PY{p}{(}\PY{n}{mt2\PYZus{}8}\PY{p}{)}


\PY{n}{ttA} \PY{o}{=} \PY{n}{cputime}\PY{p}{(}\PY{p}{)}
\PY{n}{j1} \PY{o}{=} \PY{n}{S}\PY{o}{.}\PY{n}{ideal}\PY{p}{(}\PY{n}{JJ1}\PY{p}{)}\PY{o}{.}\PY{n}{saturation}\PY{p}{(}\PY{n}{l1}\PY{p}{)}\PY{p}{[}\PY{l+m+mi}{0}\PY{p}{]}

\PY{c+c1}{\PYZsh{}\PYZsh{} we saturate w.r.t. the conditions that P2 and P3, P4, P5}
\PY{c+c1}{\PYZsh{}\PYZsh{} are different:}

\PY{n}{plSat} \PY{o}{=} \PY{n}{prod}\PY{p}{(}
    \PY{p}{[}
        \PY{n}{S}\PY{o}{.}\PY{n}{ideal}\PY{p}{(}\PY{n}{matrix}\PY{p}{(}\PY{p}{[}\PY{n}{P2}\PY{p}{,} \PY{n}{pp}\PY{p}{]}\PY{p}{)}\PY{o}{.}\PY{n}{minors}\PY{p}{(}\PY{l+m+mi}{2}\PY{p}{)}\PY{p}{)}\PY{o}{.}\PY{n}{groebner\PYZus{}basis}\PY{p}{(}\PY{p}{)}\PY{p}{[}\PY{l+m+mi}{0}\PY{p}{]}
        \PY{k}{for} \PY{n}{pp} \PY{o+ow}{in} \PY{p}{[}\PY{n}{P3}\PY{p}{,} \PY{n}{P4}\PY{p}{,} \PY{n}{P5}\PY{p}{]}
    \PY{p}{]}
\PY{p}{)}

\PY{n}{j1} \PY{o}{=} \PY{n}{j1}\PY{o}{.}\PY{n}{saturation}\PY{p}{(}\PY{n}{plSat}\PY{p}{)}\PY{p}{[}\PY{l+m+mi}{0}\PY{p}{]}

\PY{c+c1}{\PYZsh{}\PYZsh{} similarly for JJ2.}
\PY{c+c1}{\PYZsh{}\PYZsh{} we saturate w.r.t. the conditions that P2 and P6, P7, P8}
\PY{c+c1}{\PYZsh{}\PYZsh{} are different:}

\PY{n}{j2} \PY{o}{=} \PY{n}{S}\PY{o}{.}\PY{n}{ideal}\PY{p}{(}\PY{n}{JJ2}\PY{p}{)}\PY{o}{.}\PY{n}{saturation}\PY{p}{(}\PY{n}{l1}\PY{p}{)}\PY{p}{[}\PY{l+m+mi}{0}\PY{p}{]}

\PY{n}{plSat} \PY{o}{=} \PY{n}{prod}\PY{p}{(}
    \PY{p}{[}
        \PY{n}{S}\PY{o}{.}\PY{n}{ideal}\PY{p}{(}\PY{n}{matrix}\PY{p}{(}\PY{p}{[}\PY{n}{P2}\PY{p}{,} \PY{n}{pp}\PY{p}{]}\PY{p}{)}\PY{o}{.}\PY{n}{minors}\PY{p}{(}\PY{l+m+mi}{2}\PY{p}{)}\PY{p}{)}\PY{o}{.}\PY{n}{groebner\PYZus{}basis}\PY{p}{(}\PY{p}{)}\PY{p}{[}\PY{l+m+mi}{0}\PY{p}{]}
        \PY{k}{for} \PY{n}{pp} \PY{o+ow}{in} \PY{p}{[}\PY{n}{P6}\PY{p}{,} \PY{n}{P7}\PY{p}{,} \PY{n}{P8}\PY{p}{]}
    \PY{p}{]}
\PY{p}{)}

\PY{n}{j2} \PY{o}{=} \PY{n}{j2}\PY{o}{.}\PY{n}{saturation}\PY{p}{(}\PY{n}{plSat}\PY{p}{)}\PY{p}{[}\PY{l+m+mi}{0}\PY{p}{]}

\PY{n+nb}{print}\PY{p}{(}\PY{n}{cputime}\PY{p}{(}\PY{p}{)}\PY{o}{\PYZhy{}}\PY{n}{ttA}\PY{p}{)}
\end{Verbatim}
\end{tcolorbox}

    \begin{Verbatim}[commandchars=\\\{\}]
About 63 seconds of computation{\ldots}
41.41182299999991
About 80 seconds of computation{\ldots}
46.731866999999966
Now 36 seconds of computations
25.96072499999991
    \end{Verbatim}

    In order to study the zeros of j1 and j2, we compute their radical, then
we sum the two radicals and we get the ideal (1). This means that the
matrices MT1 and MT2 cannot have rank 8 Time of computations: about 146
seconds.

    \begin{tcolorbox}[breakable, size=fbox, boxrule=1pt, pad at break*=1mm,colback=cellbackground, colframe=cellborder]
\prompt{In}{incolor}{112}{\boxspacing}
\begin{Verbatim}[commandchars=\\\{\}]
\PY{c+c1}{\PYZsh{}\PYZsh{} We compute the radical of j1 and j2:}
\PY{n+nb}{print}\PY{p}{(}\PY{l+s+s2}{\PYZdq{}}\PY{l+s+s2}{Computation of a radical: 54 seconds}\PY{l+s+s2}{\PYZdq{}}\PY{p}{)}
\PY{n}{sleep}\PY{p}{(}\PY{l+m+mi}{1}\PY{p}{)}
\PY{n}{ttA} \PY{o}{=} \PY{n}{cputime}\PY{p}{(}\PY{p}{)}
\PY{n}{rJ1} \PY{o}{=} \PY{n}{j1}\PY{o}{.}\PY{n}{radical}\PY{p}{(}\PY{p}{)}
\PY{n+nb}{print}\PY{p}{(}\PY{n}{cputime}\PY{p}{(}\PY{p}{)}\PY{o}{\PYZhy{}}\PY{n}{ttA}\PY{p}{)}

\PY{n+nb}{print}\PY{p}{(}\PY{l+s+s2}{\PYZdq{}}\PY{l+s+s2}{computation of the second radical: 73 seconds}\PY{l+s+s2}{\PYZdq{}}\PY{p}{)}
\PY{n}{sleep}\PY{p}{(}\PY{l+m+mi}{1}\PY{p}{)}
\PY{n}{ttA} \PY{o}{=} \PY{n}{cputime}\PY{p}{(}\PY{p}{)}
\PY{n}{rJ2} \PY{o}{=} \PY{n}{j2}\PY{o}{.}\PY{n}{radical}\PY{p}{(}\PY{p}{)}
\PY{n+nb}{print}\PY{p}{(}\PY{n}{cputime}\PY{p}{(}\PY{p}{)}\PY{o}{\PYZhy{}}\PY{n}{ttA}\PY{p}{)}


\PY{c+c1}{\PYZsh{}\PYZsh{} now we sum the two ideals and we get }
\PY{c+c1}{\PYZsh{}\PYZsh{} the ideal (1).}

\PY{n+nb}{print}\PY{p}{(}\PY{l+s+s2}{\PYZdq{}}\PY{l+s+s2}{about 15 seconds}\PY{l+s+s2}{\PYZdq{}}\PY{p}{)}
\PY{n}{sleep}\PY{p}{(}\PY{l+m+mi}{1}\PY{p}{)}
\PY{n}{ttA} \PY{o}{=} \PY{n}{cputime}\PY{p}{(}\PY{p}{)}
\PY{k}{assert}\PY{p}{(}\PY{n}{rJ1}\PY{o}{+}\PY{n}{rJ2} \PY{o}{==} \PY{n}{S}\PY{o}{.}\PY{n}{ideal}\PY{p}{(}\PY{l+m+mi}{1}\PY{p}{)}\PY{p}{)}
\PY{n+nb}{print}\PY{p}{(}\PY{n}{cputime}\PY{p}{(}\PY{p}{)}\PY{o}{\PYZhy{}}\PY{n}{ttA}\PY{p}{)}
\end{Verbatim}
\end{tcolorbox}

    \begin{Verbatim}[commandchars=\\\{\}]
Computation of a radical: 54 seconds
39.453751999999895
computation of the second radical: 73 seconds
52.07700100000011
about 15 seconds
10.753017999999884
    \end{Verbatim}

    The above computations show that it is almost sure that it is not
possible to have \(P_1, \dotsc, P_8\) eigenpoints. In order to be sure,
we have to repeat the computation for other points and to see if we get
again that also these points on Cs cannot be eigenpoint for a cubic. The
next block repeat therefore the computations for other points. The time
of computation is 420 seconds.

    \begin{tcolorbox}[breakable, size=fbox, boxrule=1pt, pad at break*=1mm,colback=cellbackground, colframe=cellborder]
\prompt{In}{incolor}{113}{\boxspacing}
\begin{Verbatim}[commandchars=\\\{\}]
\PY{n+nb}{print}\PY{p}{(}\PY{l+s+s2}{\PYZdq{}}\PY{l+s+s2}{Half of the computation. Now the same computations with other points}\PY{l+s+s2}{\PYZdq{}}\PY{p}{)}

\PY{n}{P3} \PY{o}{=} \PY{n}{Pg}\PY{o}{.}\PY{n}{subs}\PY{p}{(}\PY{n}{w1}\PY{o}{=}\PY{l+m+mi}{1}\PY{p}{)}
\PY{n}{P4} \PY{o}{=} \PY{n}{Pg}\PY{o}{.}\PY{n}{subs}\PY{p}{(}\PY{n}{w1}\PY{o}{=}\PY{l+m+mi}{2}\PY{p}{)}
\PY{n}{P5} \PY{o}{=} \PY{n}{Pg}\PY{o}{.}\PY{n}{subs}\PY{p}{(}\PY{n}{w1}\PY{o}{=}\PY{l+m+mi}{3}\PY{p}{)}
\PY{n}{P6} \PY{o}{=} \PY{n}{Pg}\PY{o}{.}\PY{n}{subs}\PY{p}{(}\PY{n}{w1}\PY{o}{=}\PY{l+m+mi}{5}\PY{p}{)}
\PY{n}{P7} \PY{o}{=} \PY{n}{Pg}\PY{o}{.}\PY{n}{subs}\PY{p}{(}\PY{n}{w1}\PY{o}{=}\PY{o}{\PYZhy{}}\PY{l+m+mi}{5}\PY{p}{)}
\PY{n}{P8} \PY{o}{=} \PY{n}{Pg}\PY{o}{.}\PY{n}{subs}\PY{p}{(}\PY{n}{w1}\PY{o}{=}\PY{l+m+mi}{6}\PY{p}{)}



\PY{c+c1}{\PYZsh{}\PYZsh{} the following two matrices must have rank \PYZlt{}= 9:}

\PY{n}{MT1} \PY{o}{=} \PY{n}{condition\PYZus{}matrix}\PY{p}{(}\PY{p}{[}\PY{n}{P1}\PY{p}{,} \PY{n}{P2}\PY{p}{,} \PY{n}{P3}\PY{p}{,} \PY{n}{P4}\PY{p}{,} \PY{n}{P5}\PY{p}{]}\PY{p}{,} \PY{n}{S}\PY{p}{,} \PY{n}{standard}\PY{o}{=}\PY{l+s+s2}{\PYZdq{}}\PY{l+s+s2}{all}\PY{l+s+s2}{\PYZdq{}}\PY{p}{)}
\PY{n}{MT2} \PY{o}{=} \PY{n}{condition\PYZus{}matrix}\PY{p}{(}\PY{p}{[}\PY{n}{P1}\PY{p}{,} \PY{n}{P2}\PY{p}{,} \PY{n}{P6}\PY{p}{,} \PY{n}{P7}\PY{p}{,} \PY{n}{P8}\PY{p}{]}\PY{p}{,} \PY{n}{S}\PY{p}{,} \PY{n}{standard}\PY{o}{=}\PY{l+s+s2}{\PYZdq{}}\PY{l+s+s2}{all}\PY{l+s+s2}{\PYZdq{}}\PY{p}{)}

\PY{c+c1}{\PYZsh{}\PYZsh{} Since the 0\PYZhy{}row of MT1 and MT2 is numeric and also the 1\PYZhy{}row }
\PY{c+c1}{\PYZsh{}\PYZsh{} of MT1 and MT2 is numeric, we can use them to simplify MT1 and MT2. }

\PY{c+c1}{\PYZsh{}\PYZsh{} assert(MT1[0] == }
\PY{n}{MT1}\PY{o}{.}\PY{n}{rescale\PYZus{}row}\PY{p}{(}\PY{l+m+mi}{0}\PY{p}{,} \PY{n}{ii}\PY{o}{/}\PY{l+m+mi}{3}\PY{p}{)}
\PY{k}{for} \PY{n}{i} \PY{o+ow}{in} \PY{n+nb}{range}\PY{p}{(}\PY{l+m+mi}{2}\PY{p}{,} \PY{l+m+mi}{10}\PY{p}{)}\PY{p}{:}
    \PY{n}{MT1}\PY{o}{.}\PY{n}{add\PYZus{}multiple\PYZus{}of\PYZus{}row}\PY{p}{(}\PY{n}{i}\PY{p}{,} \PY{l+m+mi}{0}\PY{p}{,} \PY{o}{\PYZhy{}}\PY{n}{MT1}\PY{p}{[}\PY{n}{i}\PY{p}{]}\PY{p}{[}\PY{l+m+mi}{0}\PY{p}{]}\PY{p}{)}


\PY{n}{MT1}\PY{o}{.}\PY{n}{rescale\PYZus{}row}\PY{p}{(}\PY{l+m+mi}{1}\PY{p}{,} \PY{l+m+mi}{1}\PY{p}{)}
\PY{k}{for} \PY{n}{i} \PY{o+ow}{in} \PY{n+nb}{range}\PY{p}{(}\PY{l+m+mi}{2}\PY{p}{,} \PY{l+m+mi}{10}\PY{p}{)}\PY{p}{:}
    \PY{n}{MT1}\PY{o}{.}\PY{n}{add\PYZus{}multiple\PYZus{}of\PYZus{}row}\PY{p}{(}\PY{n}{i}\PY{p}{,} \PY{l+m+mi}{1}\PY{p}{,} \PY{o}{\PYZhy{}}\PY{n}{MT1}\PY{p}{[}\PY{n}{i}\PY{p}{]}\PY{p}{[}\PY{l+m+mi}{4}\PY{p}{]}\PY{p}{)}


\PY{n}{MT1} \PY{o}{=} \PY{n}{MT1}\PY{o}{.}\PY{n}{matrix\PYZus{}from\PYZus{}columns}\PY{p}{(}\PY{p}{[}\PY{l+m+mi}{1}\PY{p}{,} \PY{l+m+mi}{2}\PY{p}{,} \PY{l+m+mi}{3}\PY{p}{,} \PY{l+m+mi}{5}\PY{p}{,} \PY{l+m+mi}{6}\PY{p}{,} \PY{l+m+mi}{7}\PY{p}{,} \PY{l+m+mi}{8}\PY{p}{,} \PY{l+m+mi}{9}\PY{p}{]}\PY{p}{)}
\PY{n}{MT1} \PY{o}{=} \PY{n}{MT1}\PY{o}{.}\PY{n}{matrix\PYZus{}from\PYZus{}rows}\PY{p}{(}\PY{n+nb}{range}\PY{p}{(}\PY{l+m+mi}{3}\PY{p}{,} \PY{l+m+mi}{15}\PY{p}{)}\PY{p}{)}

\PY{c+c1}{\PYZsh{}\PYZsh{} same for MT2:}
\PY{c+c1}{\PYZsh{}\PYZsh{}assert(MT1[0] == MT2[0])}
\PY{c+c1}{\PYZsh{}\PYZsh{}assert(MT1[1] == MT2[1])}

\PY{n}{MT2}\PY{o}{.}\PY{n}{rescale\PYZus{}row}\PY{p}{(}\PY{l+m+mi}{0}\PY{p}{,} \PY{n}{ii}\PY{o}{/}\PY{l+m+mi}{3}\PY{p}{)}
\PY{k}{for} \PY{n}{i} \PY{o+ow}{in} \PY{n+nb}{range}\PY{p}{(}\PY{l+m+mi}{2}\PY{p}{,} \PY{l+m+mi}{10}\PY{p}{)}\PY{p}{:}
    \PY{n}{MT2}\PY{o}{.}\PY{n}{add\PYZus{}multiple\PYZus{}of\PYZus{}row}\PY{p}{(}\PY{n}{i}\PY{p}{,} \PY{l+m+mi}{0}\PY{p}{,} \PY{o}{\PYZhy{}}\PY{n}{MT2}\PY{p}{[}\PY{n}{i}\PY{p}{]}\PY{p}{[}\PY{l+m+mi}{0}\PY{p}{]}\PY{p}{)}


\PY{n}{MT2}\PY{o}{.}\PY{n}{rescale\PYZus{}row}\PY{p}{(}\PY{l+m+mi}{1}\PY{p}{,} \PY{l+m+mi}{1}\PY{p}{)}
\PY{k}{for} \PY{n}{i} \PY{o+ow}{in} \PY{n+nb}{range}\PY{p}{(}\PY{l+m+mi}{2}\PY{p}{,} \PY{l+m+mi}{10}\PY{p}{)}\PY{p}{:}
    \PY{n}{MT2}\PY{o}{.}\PY{n}{add\PYZus{}multiple\PYZus{}of\PYZus{}row}\PY{p}{(}\PY{n}{i}\PY{p}{,} \PY{l+m+mi}{1}\PY{p}{,} \PY{o}{\PYZhy{}}\PY{n}{MT2}\PY{p}{[}\PY{n}{i}\PY{p}{]}\PY{p}{[}\PY{l+m+mi}{4}\PY{p}{]}\PY{p}{)}


\PY{n}{MT2} \PY{o}{=} \PY{n}{MT2}\PY{o}{.}\PY{n}{matrix\PYZus{}from\PYZus{}columns}\PY{p}{(}\PY{p}{[}\PY{l+m+mi}{1}\PY{p}{,} \PY{l+m+mi}{2}\PY{p}{,} \PY{l+m+mi}{3}\PY{p}{,} \PY{l+m+mi}{5}\PY{p}{,} \PY{l+m+mi}{6}\PY{p}{,} \PY{l+m+mi}{7}\PY{p}{,} \PY{l+m+mi}{8}\PY{p}{,} \PY{l+m+mi}{9}\PY{p}{]}\PY{p}{)}
\PY{n}{MT2} \PY{o}{=} \PY{n}{MT2}\PY{o}{.}\PY{n}{matrix\PYZus{}from\PYZus{}rows}\PY{p}{(}\PY{n+nb}{range}\PY{p}{(}\PY{l+m+mi}{3}\PY{p}{,} \PY{l+m+mi}{15}\PY{p}{)}\PY{p}{)}
\end{Verbatim}
\end{tcolorbox}

    \begin{Verbatim}[commandchars=\\\{\}]
Half of the computation. Now the same computations with other points
    \end{Verbatim}

    \begin{tcolorbox}[breakable, size=fbox, boxrule=1pt, pad at break*=1mm,colback=cellbackground, colframe=cellborder]
\prompt{In}{incolor}{114}{\boxspacing}
\begin{Verbatim}[commandchars=\\\{\}]
\PY{c+c1}{\PYZsh{}\PYZsh{} If P1, P2, P3, P4, P5, P6, P7, P8 are eigenpoints, }
\PY{c+c1}{\PYZsh{}\PYZsh{} the order 8 minors of MT1 and of MT2 must all be zero.}

\PY{c+c1}{\PYZsh{}\PYZsh{} this computation requires about 62\PYZdq{}:}
\PY{n+nb}{print}\PY{p}{(}\PY{l+s+s2}{\PYZdq{}}\PY{l+s+s2}{About 62 seconds of computation...}\PY{l+s+s2}{\PYZdq{}}\PY{p}{)}
\PY{n}{sleep}\PY{p}{(}\PY{l+m+mi}{1}\PY{p}{)}
\PY{n}{ttA} \PY{o}{=} \PY{n}{cputime}\PY{p}{(}\PY{p}{)}
\PY{n}{mt1\PYZus{}8} \PY{o}{=} \PY{n}{MT1}\PY{o}{.}\PY{n}{minors}\PY{p}{(}\PY{l+m+mi}{8}\PY{p}{)}
\PY{n+nb}{print}\PY{p}{(}\PY{n}{cputime}\PY{p}{(}\PY{p}{)}\PY{o}{\PYZhy{}}\PY{n}{ttA}\PY{p}{)}

\PY{c+c1}{\PYZsh{}\PYZsh{} this computation requires about 97\PYZdq{}}
\PY{n+nb}{print}\PY{p}{(}\PY{l+s+s2}{\PYZdq{}}\PY{l+s+s2}{About 97 seconds of computation...}\PY{l+s+s2}{\PYZdq{}}\PY{p}{)}
\PY{n}{sleep}\PY{p}{(}\PY{l+m+mi}{1}\PY{p}{)}
\PY{n}{ttA} \PY{o}{=} \PY{n}{cputime}\PY{p}{(}\PY{p}{)}
\PY{n}{mt2\PYZus{}8} \PY{o}{=} \PY{n}{MT2}\PY{o}{.}\PY{n}{minors}\PY{p}{(}\PY{l+m+mi}{8}\PY{p}{)}
\PY{n+nb}{print}\PY{p}{(}\PY{n}{cputime}\PY{p}{(}\PY{p}{)}\PY{o}{\PYZhy{}}\PY{n}{ttA}\PY{p}{)}
\end{Verbatim}
\end{tcolorbox}

    \begin{Verbatim}[commandchars=\\\{\}]
About 62 seconds of computation{\ldots}
35.11327500000016
About 97 seconds of computation{\ldots}
51.91341700000021
    \end{Verbatim}

    \begin{tcolorbox}[breakable, size=fbox, boxrule=1pt, pad at break*=1mm,colback=cellbackground, colframe=cellborder]
\prompt{In}{incolor}{115}{\boxspacing}
\begin{Verbatim}[commandchars=\\\{\}]
\PY{c+c1}{\PYZsh{}\PYZsh{} To speed up the computations, we keep separated the two ideals}

\PY{c+c1}{\PYZsh{}\PYZsh{} The following block requires 46 seconds.}
\PY{n+nb}{print}\PY{p}{(}\PY{l+s+s2}{\PYZdq{}}\PY{l+s+s2}{Now 46 seconds of computations}\PY{l+s+s2}{\PYZdq{}}\PY{p}{)}
\PY{n}{sleep}\PY{p}{(}\PY{l+m+mi}{1}\PY{p}{)}
\PY{n}{JJ1} \PY{o}{=} \PY{n}{S}\PY{o}{.}\PY{n}{ideal}\PY{p}{(}\PY{n}{mt1\PYZus{}8}\PY{p}{)}
\PY{n}{JJ2} \PY{o}{=} \PY{n}{S}\PY{o}{.}\PY{n}{ideal}\PY{p}{(}\PY{n}{mt2\PYZus{}8}\PY{p}{)}

\PY{n}{ttA} \PY{o}{=} \PY{n}{cputime}\PY{p}{(}\PY{p}{)}
\PY{n}{j1} \PY{o}{=} \PY{n}{S}\PY{o}{.}\PY{n}{ideal}\PY{p}{(}\PY{n}{JJ1}\PY{p}{)}\PY{o}{.}\PY{n}{saturation}\PY{p}{(}\PY{n}{l1}\PY{p}{)}\PY{p}{[}\PY{l+m+mi}{0}\PY{p}{]}
\end{Verbatim}
\end{tcolorbox}

    \begin{Verbatim}[commandchars=\\\{\}]
Now 46 seconds of computations
    \end{Verbatim}

    \begin{tcolorbox}[breakable, size=fbox, boxrule=1pt, pad at break*=1mm,colback=cellbackground, colframe=cellborder]
\prompt{In}{incolor}{116}{\boxspacing}
\begin{Verbatim}[commandchars=\\\{\}]
\PY{c+c1}{\PYZsh{}\PYZsh{} we saturate w.r.t. the conditions that P2 and P3, P4, P5}
\PY{c+c1}{\PYZsh{}\PYZsh{} are different:}

\PY{n}{plSat} \PY{o}{=} \PY{n}{prod}\PY{p}{(}
    \PY{p}{[}
        \PY{n}{S}\PY{o}{.}\PY{n}{ideal}\PY{p}{(}\PY{n}{matrix}\PY{p}{(}\PY{p}{[}\PY{n}{P2}\PY{p}{,} \PY{n}{pp}\PY{p}{]}\PY{p}{)}\PY{o}{.}\PY{n}{minors}\PY{p}{(}\PY{l+m+mi}{2}\PY{p}{)}\PY{p}{)}\PY{o}{.}\PY{n}{groebner\PYZus{}basis}\PY{p}{(}\PY{p}{)}\PY{p}{[}\PY{l+m+mi}{0}\PY{p}{]}
        \PY{k}{for} \PY{n}{pp} \PY{o+ow}{in} \PY{p}{[}\PY{n}{P3}\PY{p}{,} \PY{n}{P4}\PY{p}{,} \PY{n}{P5}\PY{p}{]}
    \PY{p}{]}
\PY{p}{)}

\PY{n}{j1} \PY{o}{=} \PY{n}{j1}\PY{o}{.}\PY{n}{saturation}\PY{p}{(}\PY{n}{plSat}\PY{p}{)}\PY{p}{[}\PY{l+m+mi}{0}\PY{p}{]}
\end{Verbatim}
\end{tcolorbox}

    \begin{tcolorbox}[breakable, size=fbox, boxrule=1pt, pad at break*=1mm,colback=cellbackground, colframe=cellborder]
\prompt{In}{incolor}{117}{\boxspacing}
\begin{Verbatim}[commandchars=\\\{\}]
\PY{c+c1}{\PYZsh{}\PYZsh{} similarly for JJ2.}
\PY{c+c1}{\PYZsh{}\PYZsh{} we saturate w.r.t. the conditions that P2 and P6, P7, P8}
\PY{c+c1}{\PYZsh{}\PYZsh{} are different:}

\PY{n}{j2} \PY{o}{=} \PY{n}{S}\PY{o}{.}\PY{n}{ideal}\PY{p}{(}\PY{n}{JJ2}\PY{p}{)}\PY{o}{.}\PY{n}{saturation}\PY{p}{(}\PY{n}{l1}\PY{p}{)}\PY{p}{[}\PY{l+m+mi}{0}\PY{p}{]}

\PY{n}{plSat} \PY{o}{=} \PY{n}{prod}\PY{p}{(}
    \PY{p}{[}
        \PY{n}{S}\PY{o}{.}\PY{n}{ideal}\PY{p}{(}
            \PY{n}{matrix}\PY{p}{(}\PY{p}{[}\PY{n}{P2}\PY{p}{,} \PY{n}{pp}\PY{p}{]}\PY{p}{)}\PY{o}{.}\PY{n}{minors}\PY{p}{(}\PY{l+m+mi}{2}\PY{p}{)}
        \PY{p}{)}\PY{o}{.}\PY{n}{groebner\PYZus{}basis}\PY{p}{(}\PY{p}{)}\PY{p}{[}\PY{l+m+mi}{0}\PY{p}{]}
        \PY{k}{for} \PY{n}{pp} \PY{o+ow}{in} \PY{p}{[}\PY{n}{P6}\PY{p}{,} \PY{n}{P7}\PY{p}{,} \PY{n}{P8}\PY{p}{]}
    \PY{p}{]}
\PY{p}{)}

\PY{n}{j2} \PY{o}{=} \PY{n}{j2}\PY{o}{.}\PY{n}{saturation}\PY{p}{(}\PY{n}{plSat}\PY{p}{)}\PY{p}{[}\PY{l+m+mi}{0}\PY{p}{]}

\PY{n+nb}{print}\PY{p}{(}\PY{n}{cputime}\PY{p}{(}\PY{p}{)}\PY{o}{\PYZhy{}}\PY{n}{ttA}\PY{p}{)}
\end{Verbatim}
\end{tcolorbox}

    \begin{Verbatim}[commandchars=\\\{\}]
32.323909000000185
    \end{Verbatim}

    \begin{tcolorbox}[breakable, size=fbox, boxrule=1pt, pad at break*=1mm,colback=cellbackground, colframe=cellborder]
\prompt{In}{incolor}{118}{\boxspacing}
\begin{Verbatim}[commandchars=\\\{\}]
\PY{c+c1}{\PYZsh{}\PYZsh{} We compute the radical of j1 and j2:}
\PY{n+nb}{print}\PY{p}{(}\PY{l+s+s2}{\PYZdq{}}\PY{l+s+s2}{Computation of a radical: 55 seconds}\PY{l+s+s2}{\PYZdq{}}\PY{p}{)}
\PY{n}{sleep}\PY{p}{(}\PY{l+m+mi}{1}\PY{p}{)}
\PY{n}{ttA} \PY{o}{=} \PY{n}{cputime}\PY{p}{(}\PY{p}{)}
\PY{n}{rJ1} \PY{o}{=} \PY{n}{j1}\PY{o}{.}\PY{n}{radical}\PY{p}{(}\PY{p}{)}
\PY{n+nb}{print}\PY{p}{(}\PY{n}{cputime}\PY{p}{(}\PY{p}{)}\PY{o}{\PYZhy{}}\PY{n}{ttA}\PY{p}{)}

\PY{n+nb}{print}\PY{p}{(}\PY{l+s+s2}{\PYZdq{}}\PY{l+s+s2}{computation of the second radical: 116 seconds}\PY{l+s+s2}{\PYZdq{}}\PY{p}{)}
\PY{n}{sleep}\PY{p}{(}\PY{l+m+mi}{1}\PY{p}{)}

\PY{n}{ttA} \PY{o}{=} \PY{n}{cputime}\PY{p}{(}\PY{p}{)}
\PY{n}{rJ2} \PY{o}{=} \PY{n}{j2}\PY{o}{.}\PY{n}{radical}\PY{p}{(}\PY{p}{)}
\PY{n+nb}{print}\PY{p}{(}\PY{n}{cputime}\PY{p}{(}\PY{p}{)}\PY{o}{\PYZhy{}}\PY{n}{ttA}\PY{p}{)}
\end{Verbatim}
\end{tcolorbox}

    \begin{Verbatim}[commandchars=\\\{\}]
Computation of a radical: 55 seconds
38.51320700000042
computation of the second radical: 116 seconds
79.28385499999968
    \end{Verbatim}

    \begin{tcolorbox}[breakable, size=fbox, boxrule=1pt, pad at break*=1mm,colback=cellbackground, colframe=cellborder]
\prompt{In}{incolor}{119}{\boxspacing}
\begin{Verbatim}[commandchars=\\\{\}]
\PY{c+c1}{\PYZsh{}\PYZsh{} now we sum the two ideals and we get }
\PY{c+c1}{\PYZsh{}\PYZsh{} the ideal (1).}

\PY{n+nb}{print}\PY{p}{(}\PY{l+s+s2}{\PYZdq{}}\PY{l+s+s2}{about 20 seconds}\PY{l+s+s2}{\PYZdq{}}\PY{p}{)}
\PY{n}{sleep}\PY{p}{(}\PY{l+m+mi}{1}\PY{p}{)}
\PY{n}{ttA} \PY{o}{=} \PY{n}{cputime}\PY{p}{(}\PY{p}{)}
\PY{k}{assert}\PY{p}{(}\PY{n}{rJ1}\PY{o}{+}\PY{n}{rJ2} \PY{o}{==} \PY{n}{S}\PY{o}{.}\PY{n}{ideal}\PY{p}{(}\PY{l+m+mi}{1}\PY{p}{)}\PY{p}{)}
\PY{n+nb}{print}\PY{p}{(}\PY{n}{cputime}\PY{p}{(}\PY{p}{)}\PY{o}{\PYZhy{}}\PY{n}{ttA}\PY{p}{)}
\end{Verbatim}
\end{tcolorbox}

    \begin{Verbatim}[commandchars=\\\{\}]
about 20 seconds
13.807926000000407
    \end{Verbatim}

    Conclusion of the above comptations: it is not possible to have a cubic
whose eigenpoints contain a conic which is osculating the isotropic
conic in a point.

    \hypertarget{gamma-iperosculating-the-isotropic-conic-mathcalq_mathrmiso-in-p_1-1i0}{%
\subsection{\texorpdfstring{\(\Gamma\) iperosculating the isotropic
conic \(\mathcal{Q}_{\mathrm{iso}}\) in
\(P_1 = (1:i:0)\)}{\textbackslash Gamma iperosculating the isotropic conic \textbackslash mathcal\{Q\}\_\{\textbackslash mathrm\{iso\}\} in P\_1 = (1:i:0)}}\label{gamma-iperosculating-the-isotropic-conic-mathcalq_mathrmiso-in-p_1-1i0}}

    We want to see now if there are cubics such that among their eigenpoints
have a conic which is iperosculating \(\mathcal{Q}_{\mathrm{iso}}\) in a
point that, as usual, can be the point \(P_1=(1:i:0)\). The computations
are similar to the computations in the previous case:

We construct the pencil of conics Cg iperosculating
\(\mathcal{Q}_{\mathrm{iso}}\) in \(P_1\), we define 8 further points
``random'' on Cg and we construct two matrices MT1 and MT2 whose rank
must be \(\leq 9\). We study the ideal obtained from these conditions.

    \begin{tcolorbox}[breakable, size=fbox, boxrule=1pt, pad at break*=1mm,colback=cellbackground, colframe=cellborder]
\prompt{In}{incolor}{120}{\boxspacing}
\begin{Verbatim}[commandchars=\\\{\}]
\PY{n}{P1} \PY{o}{=} \PY{n}{vector}\PY{p}{(}\PY{n}{S}\PY{p}{,} \PY{p}{(}\PY{l+m+mi}{1}\PY{p}{,} \PY{n}{ii}\PY{p}{,} \PY{l+m+mi}{0}\PY{p}{)}\PY{p}{)}
\end{Verbatim}
\end{tcolorbox}

    \begin{tcolorbox}[breakable, size=fbox, boxrule=1pt, pad at break*=1mm,colback=cellbackground, colframe=cellborder]
\prompt{In}{incolor}{121}{\boxspacing}
\begin{Verbatim}[commandchars=\\\{\}]
\PY{c+c1}{\PYZsh{}\PYZsh{} Tangent line to Ciso in P1:}
\PY{n}{rtg} \PY{o}{=} \PY{n}{scalar\PYZus{}product}\PY{p}{(}\PY{n}{P1}\PY{p}{,} \PY{n}{vector}\PY{p}{(}\PY{p}{(}\PY{n}{x}\PY{p}{,} \PY{n}{y}\PY{p}{,} \PY{n}{z}\PY{p}{)}\PY{p}{)}\PY{o}{\PYZhy{}}\PY{n}{P1}\PY{p}{)}
\end{Verbatim}
\end{tcolorbox}

    \begin{tcolorbox}[breakable, size=fbox, boxrule=1pt, pad at break*=1mm,colback=cellbackground, colframe=cellborder]
\prompt{In}{incolor}{122}{\boxspacing}
\begin{Verbatim}[commandchars=\\\{\}]
\PY{c+c1}{\PYZsh{}\PYZsh{} Pencil of conics iperosculating Ciso in P1}
\PY{n}{Cg} \PY{o}{=} \PY{n}{Ciso} \PY{o}{+} \PY{n}{l1}\PY{o}{*}\PY{n}{rtg}\PY{o}{\PYZca{}}\PY{l+m+mi}{2}
\end{Verbatim}
\end{tcolorbox}

    \begin{tcolorbox}[breakable, size=fbox, boxrule=1pt, pad at break*=1mm,colback=cellbackground, colframe=cellborder]
\prompt{In}{incolor}{123}{\boxspacing}
\begin{Verbatim}[commandchars=\\\{\}]
\PY{c+c1}{\PYZsh{}\PYZsh{} If l1 = 0, Cg is Ciso, so we can assume}
\PY{c+c1}{\PYZsh{}\PYZsh{} l1 != 0}
\PY{k}{assert}\PY{p}{(}\PY{n}{Cg}\PY{o}{.}\PY{n}{subs}\PY{p}{(}\PY{n}{l1}\PY{o}{=}\PY{l+m+mi}{0}\PY{p}{)} \PY{o}{==} \PY{n}{Ciso}\PY{p}{)}
\end{Verbatim}
\end{tcolorbox}

    \begin{tcolorbox}[breakable, size=fbox, boxrule=1pt, pad at break*=1mm,colback=cellbackground, colframe=cellborder]
\prompt{In}{incolor}{124}{\boxspacing}
\begin{Verbatim}[commandchars=\\\{\}]
\PY{c+c1}{\PYZsh{}\PYZsh{} construction of a generic point (different from (1, ii, 0)) on Cg:}

\PY{n}{foo} \PY{o}{=} \PY{n}{Cg}\PY{o}{.}\PY{n}{subs}\PY{p}{(}\PY{n}{y}\PY{o}{=}\PY{n}{ii}\PY{o}{*}\PY{n}{x}\PY{o}{+}\PY{n}{w1}\PY{o}{*}\PY{n}{z}\PY{p}{)}\PY{o}{.}\PY{n}{factor}\PY{p}{(}\PY{p}{)}\PY{p}{[}\PY{o}{\PYZhy{}}\PY{l+m+mi}{1}\PY{p}{]}\PY{p}{[}\PY{l+m+mi}{0}\PY{p}{]}
\end{Verbatim}
\end{tcolorbox}

    \begin{tcolorbox}[breakable, size=fbox, boxrule=1pt, pad at break*=1mm,colback=cellbackground, colframe=cellborder]
\prompt{In}{incolor}{125}{\boxspacing}
\begin{Verbatim}[commandchars=\\\{\}]
\PY{c+c1}{\PYZsh{}\PYZsh{} generic point of Cg (it depends on the parameter w1):}
\PY{n}{Pg} \PY{o}{=} \PY{n}{vector}\PY{p}{(}
    \PY{n}{S}\PY{p}{,} 
    \PY{p}{(}
        \PY{n}{foo}\PY{o}{.}\PY{n}{coefficient}\PY{p}{(}\PY{n}{z}\PY{p}{)}\PY{p}{,}
        \PY{n}{ii}\PY{o}{*}\PY{p}{(}\PY{n}{foo}\PY{o}{.}\PY{n}{coefficient}\PY{p}{(}\PY{n}{z}\PY{p}{)}\PY{p}{)} \PY{o}{+} \PY{n}{w1}\PY{o}{*}\PY{p}{(}\PY{o}{\PYZhy{}}\PY{n}{foo}\PY{o}{.}\PY{n}{coefficient}\PY{p}{(}\PY{n}{x}\PY{p}{)}\PY{p}{)}\PY{p}{,}
        \PY{o}{\PYZhy{}}\PY{n}{foo}\PY{o}{.}\PY{n}{coefficient}\PY{p}{(}\PY{n}{x}\PY{p}{)}
    \PY{p}{)}
\PY{p}{)}
\end{Verbatim}
\end{tcolorbox}

    \begin{tcolorbox}[breakable, size=fbox, boxrule=1pt, pad at break*=1mm,colback=cellbackground, colframe=cellborder]
\prompt{In}{incolor}{126}{\boxspacing}
\begin{Verbatim}[commandchars=\\\{\}]
\PY{c+c1}{\PYZsh{}\PYZsh{} Pg is on Cg:}
\PY{k}{assert}\PY{p}{(}\PY{n}{Cg}\PY{o}{.}\PY{n}{subs}\PY{p}{(}\PY{n}{substitution}\PY{p}{(}\PY{n}{Pg}\PY{p}{)}\PY{p}{)} \PY{o}{==} \PY{l+m+mi}{0}\PY{p}{)}
\end{Verbatim}
\end{tcolorbox}

    \begin{tcolorbox}[breakable, size=fbox, boxrule=1pt, pad at break*=1mm,colback=cellbackground, colframe=cellborder]
\prompt{In}{incolor}{127}{\boxspacing}
\begin{Verbatim}[commandchars=\\\{\}]
\PY{c+c1}{\PYZsh{}\PYZsh{} If w1 = 0, then Pg = P1, hence we can assume w1 != 0.}
\PY{k}{assert}\PY{p}{(}\PY{n}{matrix}\PY{p}{(}\PY{p}{[}\PY{n}{P1}\PY{p}{,} \PY{n}{Pg}\PY{o}{.}\PY{n}{subs}\PY{p}{(}\PY{n}{w1}\PY{o}{=}\PY{l+m+mi}{0}\PY{p}{)}\PY{p}{]}\PY{p}{)}\PY{o}{.}\PY{n}{rank}\PY{p}{(}\PY{p}{)} \PY{o}{==} \PY{l+m+mi}{1}\PY{p}{)}
\end{Verbatim}
\end{tcolorbox}

    \begin{tcolorbox}[breakable, size=fbox, boxrule=1pt, pad at break*=1mm,colback=cellbackground, colframe=cellborder]
\prompt{In}{incolor}{128}{\boxspacing}
\begin{Verbatim}[commandchars=\\\{\}]
\PY{c+c1}{\PYZsh{}\PYZsh{} Now we define eight specific points on Cg:}
\end{Verbatim}
\end{tcolorbox}

    \begin{tcolorbox}[breakable, size=fbox, boxrule=1pt, pad at break*=1mm,colback=cellbackground, colframe=cellborder]
\prompt{In}{incolor}{129}{\boxspacing}
\begin{Verbatim}[commandchars=\\\{\}]
\PY{n}{P2} \PY{o}{=} \PY{n}{Pg}\PY{o}{.}\PY{n}{subs}\PY{p}{(}\PY{n}{w1}\PY{o}{=}\PY{o}{\PYZhy{}}\PY{l+m+mi}{1}\PY{p}{)}
\PY{n}{P3} \PY{o}{=} \PY{n}{Pg}\PY{o}{.}\PY{n}{subs}\PY{p}{(}\PY{n}{w1}\PY{o}{=}\PY{o}{\PYZhy{}}\PY{l+m+mi}{2}\PY{p}{)}
\PY{n}{P4} \PY{o}{=} \PY{n}{Pg}\PY{o}{.}\PY{n}{subs}\PY{p}{(}\PY{n}{w1}\PY{o}{=}\PY{o}{\PYZhy{}}\PY{l+m+mi}{3}\PY{p}{)}
\PY{n}{P5} \PY{o}{=} \PY{n}{Pg}\PY{o}{.}\PY{n}{subs}\PY{p}{(}\PY{n}{w1}\PY{o}{=}\PY{l+m+mi}{3}\PY{p}{)}
\PY{n}{P6} \PY{o}{=} \PY{n}{Pg}\PY{o}{.}\PY{n}{subs}\PY{p}{(}\PY{n}{w1}\PY{o}{=}\PY{l+m+mi}{1}\PY{p}{)}
\PY{n}{P7} \PY{o}{=} \PY{n}{Pg}\PY{o}{.}\PY{n}{subs}\PY{p}{(}\PY{n}{w1}\PY{o}{=}\PY{l+m+mi}{2}\PY{p}{)}
\PY{n}{P8} \PY{o}{=} \PY{n}{Pg}\PY{o}{.}\PY{n}{subs}\PY{p}{(}\PY{n}{w1}\PY{o}{=}\PY{l+m+mi}{5}\PY{p}{)}
\PY{n}{P9} \PY{o}{=} \PY{n}{Pg}\PY{o}{.}\PY{n}{subs}\PY{p}{(}\PY{n}{w1}\PY{o}{=}\PY{l+m+mi}{7}\PY{p}{)}
\end{Verbatim}
\end{tcolorbox}

    \begin{tcolorbox}[breakable, size=fbox, boxrule=1pt, pad at break*=1mm,colback=cellbackground, colframe=cellborder]
\prompt{In}{incolor}{130}{\boxspacing}
\begin{Verbatim}[commandchars=\\\{\}]
\PY{c+c1}{\PYZsh{}\PYZsh{} the following two matrices must have rank \PYZlt{}= 9:}

\PY{n}{MT1} \PY{o}{=} \PY{n}{condition\PYZus{}matrix}\PY{p}{(}\PY{p}{[}\PY{n}{P1}\PY{p}{,} \PY{n}{P2}\PY{p}{,} \PY{n}{P3}\PY{p}{,} \PY{n}{P4}\PY{p}{,} \PY{n}{P5}\PY{p}{]}\PY{p}{,} \PY{n}{S}\PY{p}{,} \PY{n}{standard}\PY{o}{=}\PY{l+s+s2}{\PYZdq{}}\PY{l+s+s2}{all}\PY{l+s+s2}{\PYZdq{}}\PY{p}{)}
\PY{n}{MT2} \PY{o}{=} \PY{n}{condition\PYZus{}matrix}\PY{p}{(}\PY{p}{[}\PY{n}{P1}\PY{p}{,} \PY{n}{P6}\PY{p}{,} \PY{n}{P7}\PY{p}{,} \PY{n}{P8}\PY{p}{,} \PY{n}{P9}\PY{p}{]}\PY{p}{,} \PY{n}{S}\PY{p}{,} \PY{n}{standard}\PY{o}{=}\PY{l+s+s2}{\PYZdq{}}\PY{l+s+s2}{all}\PY{l+s+s2}{\PYZdq{}}\PY{p}{)}
\end{Verbatim}
\end{tcolorbox}

    We manipulate MT1 and MT2 in order to see if it is possible that their
rank is \(\leq 9\). At the end of these computations, we see that we
have to study the condition given by the groebner basis gb which is
l1*(l1 + 3) (about 10 seconds of computations)

    \begin{tcolorbox}[breakable, size=fbox, boxrule=1pt, pad at break*=1mm,colback=cellbackground, colframe=cellborder]
\prompt{In}{incolor}{131}{\boxspacing}
\begin{Verbatim}[commandchars=\\\{\}]
\PY{n}{MT1}\PY{o}{.}\PY{n}{rescale\PYZus{}row}\PY{p}{(}\PY{l+m+mi}{0}\PY{p}{,} \PY{n}{ii}\PY{o}{/}\PY{l+m+mi}{3}\PY{p}{)}
\PY{k}{for} \PY{n}{i} \PY{o+ow}{in} \PY{n+nb}{range}\PY{p}{(}\PY{l+m+mi}{2}\PY{p}{,} \PY{l+m+mi}{10}\PY{p}{)}\PY{p}{:}
    \PY{n}{MT1}\PY{o}{.}\PY{n}{add\PYZus{}multiple\PYZus{}of\PYZus{}row}\PY{p}{(}\PY{n}{i}\PY{p}{,} \PY{l+m+mi}{0}\PY{p}{,} \PY{o}{\PYZhy{}}\PY{n}{MT1}\PY{p}{[}\PY{n}{i}\PY{p}{]}\PY{p}{[}\PY{l+m+mi}{0}\PY{p}{]}\PY{p}{)}


\PY{n}{MT1}\PY{o}{.}\PY{n}{rescale\PYZus{}row}\PY{p}{(}\PY{l+m+mi}{1}\PY{p}{,} \PY{l+m+mi}{1}\PY{p}{)}
\PY{k}{for} \PY{n}{i} \PY{o+ow}{in} \PY{n+nb}{range}\PY{p}{(}\PY{l+m+mi}{2}\PY{p}{,} \PY{l+m+mi}{10}\PY{p}{)}\PY{p}{:}
    \PY{n}{MT1}\PY{o}{.}\PY{n}{add\PYZus{}multiple\PYZus{}of\PYZus{}row}\PY{p}{(}\PY{n}{i}\PY{p}{,} \PY{l+m+mi}{1}\PY{p}{,} \PY{o}{\PYZhy{}}\PY{n}{MT1}\PY{p}{[}\PY{n}{i}\PY{p}{]}\PY{p}{[}\PY{l+m+mi}{4}\PY{p}{]}\PY{p}{)}
    
\PY{n}{MT1} \PY{o}{=} \PY{n}{MT1}\PY{o}{.}\PY{n}{matrix\PYZus{}from\PYZus{}columns}\PY{p}{(}\PY{p}{[}\PY{l+m+mi}{1}\PY{p}{,} \PY{l+m+mi}{2}\PY{p}{,} \PY{l+m+mi}{3}\PY{p}{,} \PY{l+m+mi}{5}\PY{p}{,} \PY{l+m+mi}{6}\PY{p}{,} \PY{l+m+mi}{7}\PY{p}{,} \PY{l+m+mi}{8}\PY{p}{,} \PY{l+m+mi}{9}\PY{p}{]}\PY{p}{)}
\PY{n}{MT1} \PY{o}{=} \PY{n}{MT1}\PY{o}{.}\PY{n}{matrix\PYZus{}from\PYZus{}rows}\PY{p}{(}\PY{n+nb}{range}\PY{p}{(}\PY{l+m+mi}{3}\PY{p}{,} \PY{l+m+mi}{15}\PY{p}{)}\PY{p}{)}

\PY{n}{m8} \PY{o}{=} \PY{n}{MT1}\PY{o}{.}\PY{n}{minors}\PY{p}{(}\PY{l+m+mi}{8}\PY{p}{)}

\PY{n}{I} \PY{o}{=} \PY{n}{S}\PY{o}{.}\PY{n}{ideal}\PY{p}{(}\PY{n}{m8}\PY{p}{)}

\PY{n}{MT2}\PY{o}{.}\PY{n}{rescale\PYZus{}row}\PY{p}{(}\PY{l+m+mi}{0}\PY{p}{,} \PY{n}{ii}\PY{o}{/}\PY{l+m+mi}{3}\PY{p}{)}
\PY{k}{for} \PY{n}{i} \PY{o+ow}{in} \PY{n+nb}{range}\PY{p}{(}\PY{l+m+mi}{2}\PY{p}{,} \PY{l+m+mi}{10}\PY{p}{)}\PY{p}{:}
    \PY{n}{MT2}\PY{o}{.}\PY{n}{add\PYZus{}multiple\PYZus{}of\PYZus{}row}\PY{p}{(}\PY{n}{i}\PY{p}{,} \PY{l+m+mi}{0}\PY{p}{,} \PY{o}{\PYZhy{}}\PY{n}{MT2}\PY{p}{[}\PY{n}{i}\PY{p}{]}\PY{p}{[}\PY{l+m+mi}{0}\PY{p}{]}\PY{p}{)}


\PY{n}{MT2}\PY{o}{.}\PY{n}{rescale\PYZus{}row}\PY{p}{(}\PY{l+m+mi}{1}\PY{p}{,} \PY{l+m+mi}{1}\PY{p}{)}
\PY{k}{for} \PY{n}{i} \PY{o+ow}{in} \PY{n+nb}{range}\PY{p}{(}\PY{l+m+mi}{2}\PY{p}{,} \PY{l+m+mi}{10}\PY{p}{)}\PY{p}{:}
    \PY{n}{MT2}\PY{o}{.}\PY{n}{add\PYZus{}multiple\PYZus{}of\PYZus{}row}\PY{p}{(}\PY{n}{i}\PY{p}{,} \PY{l+m+mi}{1}\PY{p}{,} \PY{o}{\PYZhy{}}\PY{n}{MT2}\PY{p}{[}\PY{n}{i}\PY{p}{]}\PY{p}{[}\PY{l+m+mi}{4}\PY{p}{]}\PY{p}{)}
    
\PY{n}{MT2} \PY{o}{=} \PY{n}{MT2}\PY{o}{.}\PY{n}{matrix\PYZus{}from\PYZus{}columns}\PY{p}{(}\PY{p}{[}\PY{l+m+mi}{1}\PY{p}{,} \PY{l+m+mi}{2}\PY{p}{,} \PY{l+m+mi}{3}\PY{p}{,} \PY{l+m+mi}{5}\PY{p}{,} \PY{l+m+mi}{6}\PY{p}{,} \PY{l+m+mi}{7}\PY{p}{,} \PY{l+m+mi}{8}\PY{p}{,} \PY{l+m+mi}{9}\PY{p}{]}\PY{p}{)}
\PY{n}{MT2} \PY{o}{=} \PY{n}{MT2}\PY{o}{.}\PY{n}{matrix\PYZus{}from\PYZus{}rows}\PY{p}{(}\PY{n+nb}{range}\PY{p}{(}\PY{l+m+mi}{3}\PY{p}{,} \PY{l+m+mi}{15}\PY{p}{)}\PY{p}{)}

\PY{n}{m8} \PY{o}{=} \PY{n}{MT2}\PY{o}{.}\PY{n}{minors}\PY{p}{(}\PY{l+m+mi}{8}\PY{p}{)}

\PY{n}{J} \PY{o}{=} \PY{n}{S}\PY{o}{.}\PY{n}{ideal}\PY{p}{(}\PY{n}{m8}\PY{p}{)}

\PY{n}{gb} \PY{o}{=} \PY{p}{(}\PY{n}{I}\PY{o}{+}\PY{n}{J}\PY{p}{)}\PY{o}{.}\PY{n}{groebner\PYZus{}basis}\PY{p}{(}\PY{p}{)}

\PY{k}{assert}\PY{p}{(}\PY{n}{gb} \PY{o}{==} \PY{p}{[}\PY{n}{l1}\PY{o}{*}\PY{p}{(}\PY{n}{l1} \PY{o}{+} \PY{l+m+mi}{3}\PY{p}{)}\PY{p}{]}\PY{p}{)}
\end{Verbatim}
\end{tcolorbox}

    Here we see that we have to study the case l1 = -3. We study the cubic
we obtain and we see it is of the form \(t(\mathcal{Q}_{iso}-t^2)\),
while the eigenpoints are given by \(\mathcal{Q}_{iso} -3t^2\) (\(t\) is
the tangent line)

    \begin{tcolorbox}[breakable, size=fbox, boxrule=1pt, pad at break*=1mm,colback=cellbackground, colframe=cellborder]
\prompt{In}{incolor}{132}{\boxspacing}
\begin{Verbatim}[commandchars=\\\{\}]
\PY{c+c1}{\PYZsh{} we particularize Cg  and we define Cs:}

\PY{n}{Cs} \PY{o}{=} \PY{n}{Cg}\PY{o}{.}\PY{n}{subs}\PY{p}{(}\PY{p}{\PYZob{}}\PY{n}{l1}\PY{p}{:}\PY{o}{\PYZhy{}}\PY{l+m+mi}{3}\PY{p}{\PYZcb{}}\PY{p}{)}
\PY{c+c1}{\PYZsh{}\PYZsh{} we define the matrix of conditions for the case l1 = \PYZhy{}3:}
\PY{n}{M} \PY{o}{=} \PY{n}{condition\PYZus{}matrix}\PY{p}{(}\PY{p}{[}\PY{n}{P1}\PY{p}{,} \PY{n}{P2}\PY{p}{,} \PY{n}{P3}\PY{p}{,} \PY{n}{P4}\PY{p}{,} \PY{n}{P5}\PY{p}{]}\PY{p}{,} \PY{n}{S}\PY{p}{,} \PY{n}{standard} \PY{o}{=} \PY{l+s+s1}{\PYZsq{}}\PY{l+s+s1}{all}\PY{l+s+s1}{\PYZsq{}}\PY{p}{)}\PY{o}{.}\PY{n}{subs}\PY{p}{(}\PY{p}{\PYZob{}}\PY{n}{l1}\PY{p}{:}\PY{o}{\PYZhy{}}\PY{l+m+mi}{3}\PY{p}{\PYZcb{}}\PY{p}{)}
\PY{k}{assert}\PY{p}{(}\PY{n}{M}\PY{o}{.}\PY{n}{rank}\PY{p}{(}\PY{p}{)} \PY{o}{==} \PY{l+m+mi}{9}\PY{p}{)}

\PY{c+c1}{\PYZsh{} select a suitable 9x10 submatrix of rank 9}
\PY{n}{N} \PY{o}{=} \PY{n}{M}\PY{o}{.}\PY{n}{matrix\PYZus{}from\PYZus{}rows}\PY{p}{(}\PY{p}{[}\PY{l+m+mi}{0}\PY{p}{,}\PY{l+m+mi}{1}\PY{p}{,}\PY{l+m+mi}{3}\PY{p}{,}\PY{l+m+mi}{4}\PY{p}{,}\PY{l+m+mi}{6}\PY{p}{,}\PY{l+m+mi}{7}\PY{p}{,}\PY{l+m+mi}{9}\PY{p}{,}\PY{l+m+mi}{10}\PY{p}{,}\PY{l+m+mi}{12}\PY{p}{]}\PY{p}{)}
\PY{k}{assert}\PY{p}{(}\PY{n}{N}\PY{o}{.}\PY{n}{rank}\PY{p}{(}\PY{p}{)} \PY{o}{==} \PY{l+m+mi}{9}\PY{p}{)}

\PY{c+c1}{\PYZsh{} determine the cubic from M (and N)}
\PY{n}{C} \PY{o}{=} \PY{n}{N}\PY{o}{.}\PY{n}{stack}\PY{p}{(}\PY{n}{vector}\PY{p}{(}\PY{n}{S}\PY{p}{,} \PY{n}{mon}\PY{p}{)}\PY{p}{)}\PY{o}{.}\PY{n}{det}\PY{p}{(}\PY{p}{)}

\PY{n}{pd} \PY{o}{=} \PY{p}{(}\PY{n}{S}\PY{o}{.}\PY{n}{ideal}\PY{p}{(}\PY{n+nb}{list}\PY{p}{(}\PY{n}{eig}\PY{p}{(}\PY{n}{C}\PY{p}{)}\PY{p}{)}\PY{p}{)}\PY{p}{)}\PY{o}{.}\PY{n}{primary\PYZus{}decomposition}\PY{p}{(}\PY{p}{)}

\PY{c+c1}{\PYZsh{}\PYZsh{} we get two ideals. One is the double point P\PYZus{}1}

\PY{k}{assert}\PY{p}{(}\PY{n+nb}{len}\PY{p}{(}\PY{n}{pd}\PY{p}{)} \PY{o}{==} \PY{l+m+mi}{2}\PY{p}{)}

\PY{k}{assert}\PY{p}{(}\PY{n}{pd}\PY{p}{[}\PY{l+m+mi}{1}\PY{p}{]} \PY{o}{==} \PY{n}{S}\PY{o}{.}\PY{n}{ideal}\PY{p}{(}\PY{n}{z}\PY{p}{,} \PY{n}{x}\PY{o}{\PYZca{}}\PY{l+m+mi}{2} \PY{o}{+} \PY{p}{(}\PY{l+m+mi}{2}\PY{o}{*}\PY{n}{ii}\PY{p}{)}\PY{o}{*}\PY{n}{x}\PY{o}{*}\PY{n}{y} \PY{o}{\PYZhy{}} \PY{n}{y}\PY{o}{\PYZca{}}\PY{l+m+mi}{2}\PY{p}{)}\PY{p}{)}

\PY{c+c1}{\PYZsh{}\PYZsh{} the other ideal is the conic Cs:}
\PY{k}{assert}\PY{p}{(}\PY{n}{pd}\PY{p}{[}\PY{l+m+mi}{0}\PY{p}{]} \PY{o}{==} \PY{n}{S}\PY{o}{.}\PY{n}{ideal}\PY{p}{(}\PY{n}{Cs}\PY{p}{)}\PY{p}{)}

\PY{c+c1}{\PYZsh{}\PYZsh{} The cubic C is of the form rtg*(Ciso\PYZhy{}rtg\PYZca{}2)}
\PY{c+c1}{\PYZsh{}\PYZsh{} the conic Cs is of the form Ciso\PYZhy{}3*rtg\PYZca{}2:}
\PY{k}{assert}\PY{p}{(}\PY{n}{S}\PY{o}{.}\PY{n}{ideal}\PY{p}{(}\PY{n}{C}\PY{p}{)} \PY{o}{==} \PY{n}{S}\PY{o}{.}\PY{n}{ideal}\PY{p}{(}\PY{n}{rtg}\PY{o}{*}\PY{p}{(}\PY{n}{Ciso}\PY{o}{\PYZhy{}}\PY{n}{rtg}\PY{o}{\PYZca{}}\PY{l+m+mi}{2}\PY{p}{)}\PY{p}{)}\PY{p}{)}
\PY{k}{assert}\PY{p}{(}\PY{n}{S}\PY{o}{.}\PY{n}{ideal}\PY{p}{(}\PY{n}{Cs}\PY{p}{)} \PY{o}{==} \PY{n}{S}\PY{o}{.}\PY{n}{ideal}\PY{p}{(}\PY{n}{Ciso}\PY{o}{\PYZhy{}}\PY{l+m+mi}{3}\PY{o}{*}\PY{n}{rtg}\PY{o}{\PYZca{}}\PY{l+m+mi}{2}\PY{p}{)}\PY{p}{)}

\PY{c+c1}{\PYZsh{}\PYZsh{} Moreover, rtg*(1/2*Cs + Ciso)) and rtg*(Ciso \PYZhy{} rtg\PYZca{}2)}
\PY{c+c1}{\PYZsh{}\PYZsh{} are the same cubic}
\PY{k}{assert}\PY{p}{(}\PY{n}{S}\PY{o}{.}\PY{n}{ideal}\PY{p}{(}\PY{n}{rtg}\PY{o}{*}\PY{p}{(}\PY{l+m+mi}{1}\PY{o}{/}\PY{l+m+mi}{2}\PY{o}{*}\PY{n}{Cs} \PY{o}{+} \PY{n}{Ciso}\PY{p}{)}\PY{p}{)} \PY{o}{==} \PY{n}{S}\PY{o}{.}\PY{n}{ideal}\PY{p}{(}\PY{n}{rtg}\PY{o}{*}\PY{p}{(}\PY{n}{Ciso} \PY{o}{\PYZhy{}} \PY{n}{rtg}\PY{o}{\PYZca{}}\PY{l+m+mi}{2}\PY{p}{)}\PY{p}{)}\PY{p}{)}
\end{Verbatim}
\end{tcolorbox}

    The conclusion of this computation is that if a cubic \(C\) has among
the eigenpoints a conic which is iperosculating
\(\mathcal{Q}_{\mathrm{iso}}\) in a point \(P_1\), then the cubic is of
the form \(t(\mathcal{Q}_{iso}-t^2)\) and the conic of eigenpoints is
\(\mathcal{Q}_{iso}-3t^2\). The converse is also true and is immediately
verified, because is given by the computation of the ideal pd above,
which is the ideal of the eigenpoints of the cubic
\(t(\mathcal{Q}_{iso}-t^2)\).

    This concludes the proof of the theorem.


    % Add a bibliography block to the postdoc
    
    
    
\end{document}
