\documentclass[11pt]{article}

    \usepackage[breakable]{tcolorbox}
    \usepackage{parskip} % Stop auto-indenting (to mimic markdown behaviour)
    

    % Basic figure setup, for now with no caption control since it's done
    % automatically by Pandoc (which extracts ![](path) syntax from Markdown).
    \usepackage{graphicx}
    % Maintain compatibility with old templates. Remove in nbconvert 6.0
    \let\Oldincludegraphics\includegraphics
    % Ensure that by default, figures have no caption (until we provide a
    % proper Figure object with a Caption API and a way to capture that
    % in the conversion process - todo).
    \usepackage{caption}
    \DeclareCaptionFormat{nocaption}{}
    \captionsetup{format=nocaption,aboveskip=0pt,belowskip=0pt}

    \usepackage{float}
    \floatplacement{figure}{H} % forces figures to be placed at the correct location
    \usepackage{xcolor} % Allow colors to be defined
    \usepackage{enumerate} % Needed for markdown enumerations to work
    \usepackage{geometry} % Used to adjust the document margins
    \usepackage{amsmath} % Equations
    \usepackage{amssymb} % Equations
    \usepackage{textcomp} % defines textquotesingle
    % Hack from http://tex.stackexchange.com/a/47451/13684:
    \AtBeginDocument{%
        \def\PYZsq{\textquotesingle}% Upright quotes in Pygmentized code
    }
    \usepackage{upquote} % Upright quotes for verbatim code
    \usepackage{eurosym} % defines \euro

    \usepackage{iftex}
    \ifPDFTeX
        \usepackage[T1]{fontenc}
        \IfFileExists{alphabeta.sty}{
              \usepackage{alphabeta}
          }{
              \usepackage[mathletters]{ucs}
              \usepackage[utf8x]{inputenc}
          }
    \else
        \usepackage{fontspec}
        \usepackage{unicode-math}
    \fi

    \usepackage{fancyvrb} % verbatim replacement that allows latex
    \usepackage{grffile} % extends the file name processing of package graphics
                         % to support a larger range
    \makeatletter % fix for old versions of grffile with XeLaTeX
    \@ifpackagelater{grffile}{2019/11/01}
    {
      % Do nothing on new versions
    }
    {
      \def\Gread@@xetex#1{%
        \IfFileExists{"\Gin@base".bb}%
        {\Gread@eps{\Gin@base.bb}}%
        {\Gread@@xetex@aux#1}%
      }
    }
    \makeatother
    \usepackage[Export]{adjustbox} % Used to constrain images to a maximum size
    \adjustboxset{max size={0.9\linewidth}{0.9\paperheight}}

    % The hyperref package gives us a pdf with properly built
    % internal navigation ('pdf bookmarks' for the table of contents,
    % internal cross-reference links, web links for URLs, etc.)
    \usepackage{hyperref}
    % The default LaTeX title has an obnoxious amount of whitespace. By default,
    % titling removes some of it. It also provides customization options.
    \usepackage{titling}
    \usepackage{longtable} % longtable support required by pandoc >1.10
    \usepackage{booktabs}  % table support for pandoc > 1.12.2
    \usepackage{array}     % table support for pandoc >= 2.11.3
    \usepackage{calc}      % table minipage width calculation for pandoc >= 2.11.1
    \usepackage[inline]{enumitem} % IRkernel/repr support (it uses the enumerate* environment)
    \usepackage[normalem]{ulem} % ulem is needed to support strikethroughs (\sout)
                                % normalem makes italics be italics, not underlines
    \usepackage{soul}      % strikethrough (\st) support for pandoc >= 3.0.0
    \usepackage{mathrsfs}
    

    
    % Colors for the hyperref package
    \definecolor{urlcolor}{rgb}{0,.145,.698}
    \definecolor{linkcolor}{rgb}{.71,0.21,0.01}
    \definecolor{citecolor}{rgb}{.12,.54,.11}

    % ANSI colors
    \definecolor{ansi-black}{HTML}{3E424D}
    \definecolor{ansi-black-intense}{HTML}{282C36}
    \definecolor{ansi-red}{HTML}{E75C58}
    \definecolor{ansi-red-intense}{HTML}{B22B31}
    \definecolor{ansi-green}{HTML}{00A250}
    \definecolor{ansi-green-intense}{HTML}{007427}
    \definecolor{ansi-yellow}{HTML}{DDB62B}
    \definecolor{ansi-yellow-intense}{HTML}{B27D12}
    \definecolor{ansi-blue}{HTML}{208FFB}
    \definecolor{ansi-blue-intense}{HTML}{0065CA}
    \definecolor{ansi-magenta}{HTML}{D160C4}
    \definecolor{ansi-magenta-intense}{HTML}{A03196}
    \definecolor{ansi-cyan}{HTML}{60C6C8}
    \definecolor{ansi-cyan-intense}{HTML}{258F8F}
    \definecolor{ansi-white}{HTML}{C5C1B4}
    \definecolor{ansi-white-intense}{HTML}{A1A6B2}
    \definecolor{ansi-default-inverse-fg}{HTML}{FFFFFF}
    \definecolor{ansi-default-inverse-bg}{HTML}{000000}

    % common color for the border for error outputs.
    \definecolor{outerrorbackground}{HTML}{FFDFDF}

    % commands and environments needed by pandoc snippets
    % extracted from the output of `pandoc -s`
    \providecommand{\tightlist}{%
      \setlength{\itemsep}{0pt}\setlength{\parskip}{0pt}}
    \DefineVerbatimEnvironment{Highlighting}{Verbatim}{commandchars=\\\{\}}
    % Add ',fontsize=\small' for more characters per line
    \newenvironment{Shaded}{}{}
    \newcommand{\KeywordTok}[1]{\textcolor[rgb]{0.00,0.44,0.13}{\textbf{{#1}}}}
    \newcommand{\DataTypeTok}[1]{\textcolor[rgb]{0.56,0.13,0.00}{{#1}}}
    \newcommand{\DecValTok}[1]{\textcolor[rgb]{0.25,0.63,0.44}{{#1}}}
    \newcommand{\BaseNTok}[1]{\textcolor[rgb]{0.25,0.63,0.44}{{#1}}}
    \newcommand{\FloatTok}[1]{\textcolor[rgb]{0.25,0.63,0.44}{{#1}}}
    \newcommand{\CharTok}[1]{\textcolor[rgb]{0.25,0.44,0.63}{{#1}}}
    \newcommand{\StringTok}[1]{\textcolor[rgb]{0.25,0.44,0.63}{{#1}}}
    \newcommand{\CommentTok}[1]{\textcolor[rgb]{0.38,0.63,0.69}{\textit{{#1}}}}
    \newcommand{\OtherTok}[1]{\textcolor[rgb]{0.00,0.44,0.13}{{#1}}}
    \newcommand{\AlertTok}[1]{\textcolor[rgb]{1.00,0.00,0.00}{\textbf{{#1}}}}
    \newcommand{\FunctionTok}[1]{\textcolor[rgb]{0.02,0.16,0.49}{{#1}}}
    \newcommand{\RegionMarkerTok}[1]{{#1}}
    \newcommand{\ErrorTok}[1]{\textcolor[rgb]{1.00,0.00,0.00}{\textbf{{#1}}}}
    \newcommand{\NormalTok}[1]{{#1}}

    % Additional commands for more recent versions of Pandoc
    \newcommand{\ConstantTok}[1]{\textcolor[rgb]{0.53,0.00,0.00}{{#1}}}
    \newcommand{\SpecialCharTok}[1]{\textcolor[rgb]{0.25,0.44,0.63}{{#1}}}
    \newcommand{\VerbatimStringTok}[1]{\textcolor[rgb]{0.25,0.44,0.63}{{#1}}}
    \newcommand{\SpecialStringTok}[1]{\textcolor[rgb]{0.73,0.40,0.53}{{#1}}}
    \newcommand{\ImportTok}[1]{{#1}}
    \newcommand{\DocumentationTok}[1]{\textcolor[rgb]{0.73,0.13,0.13}{\textit{{#1}}}}
    \newcommand{\AnnotationTok}[1]{\textcolor[rgb]{0.38,0.63,0.69}{\textbf{\textit{{#1}}}}}
    \newcommand{\CommentVarTok}[1]{\textcolor[rgb]{0.38,0.63,0.69}{\textbf{\textit{{#1}}}}}
    \newcommand{\VariableTok}[1]{\textcolor[rgb]{0.10,0.09,0.49}{{#1}}}
    \newcommand{\ControlFlowTok}[1]{\textcolor[rgb]{0.00,0.44,0.13}{\textbf{{#1}}}}
    \newcommand{\OperatorTok}[1]{\textcolor[rgb]{0.40,0.40,0.40}{{#1}}}
    \newcommand{\BuiltInTok}[1]{{#1}}
    \newcommand{\ExtensionTok}[1]{{#1}}
    \newcommand{\PreprocessorTok}[1]{\textcolor[rgb]{0.74,0.48,0.00}{{#1}}}
    \newcommand{\AttributeTok}[1]{\textcolor[rgb]{0.49,0.56,0.16}{{#1}}}
    \newcommand{\InformationTok}[1]{\textcolor[rgb]{0.38,0.63,0.69}{\textbf{\textit{{#1}}}}}
    \newcommand{\WarningTok}[1]{\textcolor[rgb]{0.38,0.63,0.69}{\textbf{\textit{{#1}}}}}


    % Define a nice break command that doesn't care if a line doesn't already
    % exist.
    \def\br{\hspace*{\fill} \\* }
    % Math Jax compatibility definitions
    \def\gt{>}
    \def\lt{<}
    \let\Oldtex\TeX
    \let\Oldlatex\LaTeX
    \renewcommand{\TeX}{\textrm{\Oldtex}}
    \renewcommand{\LaTeX}{\textrm{\Oldlatex}}
    % Document parameters
    % Document title
    \title{NB.04.F7}
    
    
    
    
    
    
    
% Pygments definitions
\makeatletter
\def\PY@reset{\let\PY@it=\relax \let\PY@bf=\relax%
    \let\PY@ul=\relax \let\PY@tc=\relax%
    \let\PY@bc=\relax \let\PY@ff=\relax}
\def\PY@tok#1{\csname PY@tok@#1\endcsname}
\def\PY@toks#1+{\ifx\relax#1\empty\else%
    \PY@tok{#1}\expandafter\PY@toks\fi}
\def\PY@do#1{\PY@bc{\PY@tc{\PY@ul{%
    \PY@it{\PY@bf{\PY@ff{#1}}}}}}}
\def\PY#1#2{\PY@reset\PY@toks#1+\relax+\PY@do{#2}}

\@namedef{PY@tok@w}{\def\PY@tc##1{\textcolor[rgb]{0.73,0.73,0.73}{##1}}}
\@namedef{PY@tok@c}{\let\PY@it=\textit\def\PY@tc##1{\textcolor[rgb]{0.24,0.48,0.48}{##1}}}
\@namedef{PY@tok@cp}{\def\PY@tc##1{\textcolor[rgb]{0.61,0.40,0.00}{##1}}}
\@namedef{PY@tok@k}{\let\PY@bf=\textbf\def\PY@tc##1{\textcolor[rgb]{0.00,0.50,0.00}{##1}}}
\@namedef{PY@tok@kp}{\def\PY@tc##1{\textcolor[rgb]{0.00,0.50,0.00}{##1}}}
\@namedef{PY@tok@kt}{\def\PY@tc##1{\textcolor[rgb]{0.69,0.00,0.25}{##1}}}
\@namedef{PY@tok@o}{\def\PY@tc##1{\textcolor[rgb]{0.40,0.40,0.40}{##1}}}
\@namedef{PY@tok@ow}{\let\PY@bf=\textbf\def\PY@tc##1{\textcolor[rgb]{0.67,0.13,1.00}{##1}}}
\@namedef{PY@tok@nb}{\def\PY@tc##1{\textcolor[rgb]{0.00,0.50,0.00}{##1}}}
\@namedef{PY@tok@nf}{\def\PY@tc##1{\textcolor[rgb]{0.00,0.00,1.00}{##1}}}
\@namedef{PY@tok@nc}{\let\PY@bf=\textbf\def\PY@tc##1{\textcolor[rgb]{0.00,0.00,1.00}{##1}}}
\@namedef{PY@tok@nn}{\let\PY@bf=\textbf\def\PY@tc##1{\textcolor[rgb]{0.00,0.00,1.00}{##1}}}
\@namedef{PY@tok@ne}{\let\PY@bf=\textbf\def\PY@tc##1{\textcolor[rgb]{0.80,0.25,0.22}{##1}}}
\@namedef{PY@tok@nv}{\def\PY@tc##1{\textcolor[rgb]{0.10,0.09,0.49}{##1}}}
\@namedef{PY@tok@no}{\def\PY@tc##1{\textcolor[rgb]{0.53,0.00,0.00}{##1}}}
\@namedef{PY@tok@nl}{\def\PY@tc##1{\textcolor[rgb]{0.46,0.46,0.00}{##1}}}
\@namedef{PY@tok@ni}{\let\PY@bf=\textbf\def\PY@tc##1{\textcolor[rgb]{0.44,0.44,0.44}{##1}}}
\@namedef{PY@tok@na}{\def\PY@tc##1{\textcolor[rgb]{0.41,0.47,0.13}{##1}}}
\@namedef{PY@tok@nt}{\let\PY@bf=\textbf\def\PY@tc##1{\textcolor[rgb]{0.00,0.50,0.00}{##1}}}
\@namedef{PY@tok@nd}{\def\PY@tc##1{\textcolor[rgb]{0.67,0.13,1.00}{##1}}}
\@namedef{PY@tok@s}{\def\PY@tc##1{\textcolor[rgb]{0.73,0.13,0.13}{##1}}}
\@namedef{PY@tok@sd}{\let\PY@it=\textit\def\PY@tc##1{\textcolor[rgb]{0.73,0.13,0.13}{##1}}}
\@namedef{PY@tok@si}{\let\PY@bf=\textbf\def\PY@tc##1{\textcolor[rgb]{0.64,0.35,0.47}{##1}}}
\@namedef{PY@tok@se}{\let\PY@bf=\textbf\def\PY@tc##1{\textcolor[rgb]{0.67,0.36,0.12}{##1}}}
\@namedef{PY@tok@sr}{\def\PY@tc##1{\textcolor[rgb]{0.64,0.35,0.47}{##1}}}
\@namedef{PY@tok@ss}{\def\PY@tc##1{\textcolor[rgb]{0.10,0.09,0.49}{##1}}}
\@namedef{PY@tok@sx}{\def\PY@tc##1{\textcolor[rgb]{0.00,0.50,0.00}{##1}}}
\@namedef{PY@tok@m}{\def\PY@tc##1{\textcolor[rgb]{0.40,0.40,0.40}{##1}}}
\@namedef{PY@tok@gh}{\let\PY@bf=\textbf\def\PY@tc##1{\textcolor[rgb]{0.00,0.00,0.50}{##1}}}
\@namedef{PY@tok@gu}{\let\PY@bf=\textbf\def\PY@tc##1{\textcolor[rgb]{0.50,0.00,0.50}{##1}}}
\@namedef{PY@tok@gd}{\def\PY@tc##1{\textcolor[rgb]{0.63,0.00,0.00}{##1}}}
\@namedef{PY@tok@gi}{\def\PY@tc##1{\textcolor[rgb]{0.00,0.52,0.00}{##1}}}
\@namedef{PY@tok@gr}{\def\PY@tc##1{\textcolor[rgb]{0.89,0.00,0.00}{##1}}}
\@namedef{PY@tok@ge}{\let\PY@it=\textit}
\@namedef{PY@tok@gs}{\let\PY@bf=\textbf}
\@namedef{PY@tok@ges}{\let\PY@bf=\textbf\let\PY@it=\textit}
\@namedef{PY@tok@gp}{\let\PY@bf=\textbf\def\PY@tc##1{\textcolor[rgb]{0.00,0.00,0.50}{##1}}}
\@namedef{PY@tok@go}{\def\PY@tc##1{\textcolor[rgb]{0.44,0.44,0.44}{##1}}}
\@namedef{PY@tok@gt}{\def\PY@tc##1{\textcolor[rgb]{0.00,0.27,0.87}{##1}}}
\@namedef{PY@tok@err}{\def\PY@bc##1{{\setlength{\fboxsep}{\string -\fboxrule}\fcolorbox[rgb]{1.00,0.00,0.00}{1,1,1}{\strut ##1}}}}
\@namedef{PY@tok@kc}{\let\PY@bf=\textbf\def\PY@tc##1{\textcolor[rgb]{0.00,0.50,0.00}{##1}}}
\@namedef{PY@tok@kd}{\let\PY@bf=\textbf\def\PY@tc##1{\textcolor[rgb]{0.00,0.50,0.00}{##1}}}
\@namedef{PY@tok@kn}{\let\PY@bf=\textbf\def\PY@tc##1{\textcolor[rgb]{0.00,0.50,0.00}{##1}}}
\@namedef{PY@tok@kr}{\let\PY@bf=\textbf\def\PY@tc##1{\textcolor[rgb]{0.00,0.50,0.00}{##1}}}
\@namedef{PY@tok@bp}{\def\PY@tc##1{\textcolor[rgb]{0.00,0.50,0.00}{##1}}}
\@namedef{PY@tok@fm}{\def\PY@tc##1{\textcolor[rgb]{0.00,0.00,1.00}{##1}}}
\@namedef{PY@tok@vc}{\def\PY@tc##1{\textcolor[rgb]{0.10,0.09,0.49}{##1}}}
\@namedef{PY@tok@vg}{\def\PY@tc##1{\textcolor[rgb]{0.10,0.09,0.49}{##1}}}
\@namedef{PY@tok@vi}{\def\PY@tc##1{\textcolor[rgb]{0.10,0.09,0.49}{##1}}}
\@namedef{PY@tok@vm}{\def\PY@tc##1{\textcolor[rgb]{0.10,0.09,0.49}{##1}}}
\@namedef{PY@tok@sa}{\def\PY@tc##1{\textcolor[rgb]{0.73,0.13,0.13}{##1}}}
\@namedef{PY@tok@sb}{\def\PY@tc##1{\textcolor[rgb]{0.73,0.13,0.13}{##1}}}
\@namedef{PY@tok@sc}{\def\PY@tc##1{\textcolor[rgb]{0.73,0.13,0.13}{##1}}}
\@namedef{PY@tok@dl}{\def\PY@tc##1{\textcolor[rgb]{0.73,0.13,0.13}{##1}}}
\@namedef{PY@tok@s2}{\def\PY@tc##1{\textcolor[rgb]{0.73,0.13,0.13}{##1}}}
\@namedef{PY@tok@sh}{\def\PY@tc##1{\textcolor[rgb]{0.73,0.13,0.13}{##1}}}
\@namedef{PY@tok@s1}{\def\PY@tc##1{\textcolor[rgb]{0.73,0.13,0.13}{##1}}}
\@namedef{PY@tok@mb}{\def\PY@tc##1{\textcolor[rgb]{0.40,0.40,0.40}{##1}}}
\@namedef{PY@tok@mf}{\def\PY@tc##1{\textcolor[rgb]{0.40,0.40,0.40}{##1}}}
\@namedef{PY@tok@mh}{\def\PY@tc##1{\textcolor[rgb]{0.40,0.40,0.40}{##1}}}
\@namedef{PY@tok@mi}{\def\PY@tc##1{\textcolor[rgb]{0.40,0.40,0.40}{##1}}}
\@namedef{PY@tok@il}{\def\PY@tc##1{\textcolor[rgb]{0.40,0.40,0.40}{##1}}}
\@namedef{PY@tok@mo}{\def\PY@tc##1{\textcolor[rgb]{0.40,0.40,0.40}{##1}}}
\@namedef{PY@tok@ch}{\let\PY@it=\textit\def\PY@tc##1{\textcolor[rgb]{0.24,0.48,0.48}{##1}}}
\@namedef{PY@tok@cm}{\let\PY@it=\textit\def\PY@tc##1{\textcolor[rgb]{0.24,0.48,0.48}{##1}}}
\@namedef{PY@tok@cpf}{\let\PY@it=\textit\def\PY@tc##1{\textcolor[rgb]{0.24,0.48,0.48}{##1}}}
\@namedef{PY@tok@c1}{\let\PY@it=\textit\def\PY@tc##1{\textcolor[rgb]{0.24,0.48,0.48}{##1}}}
\@namedef{PY@tok@cs}{\let\PY@it=\textit\def\PY@tc##1{\textcolor[rgb]{0.24,0.48,0.48}{##1}}}

\def\PYZbs{\char`\\}
\def\PYZus{\char`\_}
\def\PYZob{\char`\{}
\def\PYZcb{\char`\}}
\def\PYZca{\char`\^}
\def\PYZam{\char`\&}
\def\PYZlt{\char`\<}
\def\PYZgt{\char`\>}
\def\PYZsh{\char`\#}
\def\PYZpc{\char`\%}
\def\PYZdl{\char`\$}
\def\PYZhy{\char`\-}
\def\PYZsq{\char`\'}
\def\PYZdq{\char`\"}
\def\PYZti{\char`\~}
% for compatibility with earlier versions
\def\PYZat{@}
\def\PYZlb{[}
\def\PYZrb{]}
\makeatother


    % For linebreaks inside Verbatim environment from package fancyvrb.
    \makeatletter
        \newbox\Wrappedcontinuationbox
        \newbox\Wrappedvisiblespacebox
        \newcommand*\Wrappedvisiblespace {\textcolor{red}{\textvisiblespace}}
        \newcommand*\Wrappedcontinuationsymbol {\textcolor{red}{\llap{\tiny$\m@th\hookrightarrow$}}}
        \newcommand*\Wrappedcontinuationindent {3ex }
        \newcommand*\Wrappedafterbreak {\kern\Wrappedcontinuationindent\copy\Wrappedcontinuationbox}
        % Take advantage of the already applied Pygments mark-up to insert
        % potential linebreaks for TeX processing.
        %        {, <, #, %, $, ' and ": go to next line.
        %        _, }, ^, &, >, - and ~: stay at end of broken line.
        % Use of \textquotesingle for straight quote.
        \newcommand*\Wrappedbreaksatspecials {%
            \def\PYGZus{\discretionary{\char`\_}{\Wrappedafterbreak}{\char`\_}}%
            \def\PYGZob{\discretionary{}{\Wrappedafterbreak\char`\{}{\char`\{}}%
            \def\PYGZcb{\discretionary{\char`\}}{\Wrappedafterbreak}{\char`\}}}%
            \def\PYGZca{\discretionary{\char`\^}{\Wrappedafterbreak}{\char`\^}}%
            \def\PYGZam{\discretionary{\char`\&}{\Wrappedafterbreak}{\char`\&}}%
            \def\PYGZlt{\discretionary{}{\Wrappedafterbreak\char`\<}{\char`\<}}%
            \def\PYGZgt{\discretionary{\char`\>}{\Wrappedafterbreak}{\char`\>}}%
            \def\PYGZsh{\discretionary{}{\Wrappedafterbreak\char`\#}{\char`\#}}%
            \def\PYGZpc{\discretionary{}{\Wrappedafterbreak\char`\%}{\char`\%}}%
            \def\PYGZdl{\discretionary{}{\Wrappedafterbreak\char`\$}{\char`\$}}%
            \def\PYGZhy{\discretionary{\char`\-}{\Wrappedafterbreak}{\char`\-}}%
            \def\PYGZsq{\discretionary{}{\Wrappedafterbreak\textquotesingle}{\textquotesingle}}%
            \def\PYGZdq{\discretionary{}{\Wrappedafterbreak\char`\"}{\char`\"}}%
            \def\PYGZti{\discretionary{\char`\~}{\Wrappedafterbreak}{\char`\~}}%
        }
        % Some characters . , ; ? ! / are not pygmentized.
        % This macro makes them "active" and they will insert potential linebreaks
        \newcommand*\Wrappedbreaksatpunct {%
            \lccode`\~`\.\lowercase{\def~}{\discretionary{\hbox{\char`\.}}{\Wrappedafterbreak}{\hbox{\char`\.}}}%
            \lccode`\~`\,\lowercase{\def~}{\discretionary{\hbox{\char`\,}}{\Wrappedafterbreak}{\hbox{\char`\,}}}%
            \lccode`\~`\;\lowercase{\def~}{\discretionary{\hbox{\char`\;}}{\Wrappedafterbreak}{\hbox{\char`\;}}}%
            \lccode`\~`\:\lowercase{\def~}{\discretionary{\hbox{\char`\:}}{\Wrappedafterbreak}{\hbox{\char`\:}}}%
            \lccode`\~`\?\lowercase{\def~}{\discretionary{\hbox{\char`\?}}{\Wrappedafterbreak}{\hbox{\char`\?}}}%
            \lccode`\~`\!\lowercase{\def~}{\discretionary{\hbox{\char`\!}}{\Wrappedafterbreak}{\hbox{\char`\!}}}%
            \lccode`\~`\/\lowercase{\def~}{\discretionary{\hbox{\char`\/}}{\Wrappedafterbreak}{\hbox{\char`\/}}}%
            \catcode`\.\active
            \catcode`\,\active
            \catcode`\;\active
            \catcode`\:\active
            \catcode`\?\active
            \catcode`\!\active
            \catcode`\/\active
            \lccode`\~`\~
        }
    \makeatother

    \let\OriginalVerbatim=\Verbatim
    \makeatletter
    \renewcommand{\Verbatim}[1][1]{%
        %\parskip\z@skip
        \sbox\Wrappedcontinuationbox {\Wrappedcontinuationsymbol}%
        \sbox\Wrappedvisiblespacebox {\FV@SetupFont\Wrappedvisiblespace}%
        \def\FancyVerbFormatLine ##1{\hsize\linewidth
            \vtop{\raggedright\hyphenpenalty\z@\exhyphenpenalty\z@
                \doublehyphendemerits\z@\finalhyphendemerits\z@
                \strut ##1\strut}%
        }%
        % If the linebreak is at a space, the latter will be displayed as visible
        % space at end of first line, and a continuation symbol starts next line.
        % Stretch/shrink are however usually zero for typewriter font.
        \def\FV@Space {%
            \nobreak\hskip\z@ plus\fontdimen3\font minus\fontdimen4\font
            \discretionary{\copy\Wrappedvisiblespacebox}{\Wrappedafterbreak}
            {\kern\fontdimen2\font}%
        }%

        % Allow breaks at special characters using \PYG... macros.
        \Wrappedbreaksatspecials
        % Breaks at punctuation characters . , ; ? ! and / need catcode=\active
        \OriginalVerbatim[#1,codes*=\Wrappedbreaksatpunct]%
    }
    \makeatother

    % Exact colors from NB
    \definecolor{incolor}{HTML}{303F9F}
    \definecolor{outcolor}{HTML}{D84315}
    \definecolor{cellborder}{HTML}{CFCFCF}
    \definecolor{cellbackground}{HTML}{F7F7F7}

    % prompt
    \makeatletter
    \newcommand{\boxspacing}{\kern\kvtcb@left@rule\kern\kvtcb@boxsep}
    \makeatother
    \newcommand{\prompt}[4]{
        {\ttfamily\llap{{\color{#2}[#3]:\hspace{3pt}#4}}\vspace{-\baselineskip}}
    }
    

    
    % Prevent overflowing lines due to hard-to-break entities
    \sloppy
    % Setup hyperref package
    \hypersetup{
      breaklinks=true,  % so long urls are correctly broken across lines
      colorlinks=true,
      urlcolor=urlcolor,
      linkcolor=linkcolor,
      citecolor=citecolor,
      }
    % Slightly bigger margins than the latex defaults
    
    \geometry{verbose,tmargin=1in,bmargin=1in,lmargin=1in,rmargin=1in}
    
    

\begin{document}
    
    \maketitle
    
    

    
    \hypertarget{proposition}{%
\section{Proposition}\label{proposition}}

    The generic cubic of the family of cubics satisfying \[
  \sigma(P_1, P_2) = \sigma(P_1, P_4) = 0 \ \ \mbox{and} \ \ s_{22} = s_{44} = 0
\] has seven eigenpoints with the alignments: \[
  (P_1, P_2, P_3), \ (P_1, P_4, P_5), \ (P_1, P_6, P_7)
\] Among these points we have the relation
\(\left\langle P_1 \times P_6, P_3\times P_5 \right\rangle=0\) (i.e.,
the lines \(P_1 \vee P_6\) and \(P_3 \vee P_5\) are orthogonal). In the
family there is a sub-family of cubics whose eigenpoints have the
following alignments: \[
  (P_1, P_2, P_3),\ (P_1, P_4, P_5),\ (P_1, P_6, P_7),\ (P_2, P_4, P_6).
\] In this case the points \(P_6\) and \(P_7\) are given by the
formulas: \[
P_6 = s_{15}s_{34}\, P_2 + s_{13}s_{25}\, P_4, \quad 
P_7 = s_{15}(s_{26}s_{46}+s_{24}s_{66})\, P_1+ s_{11}s_{24}s_{56}\, P_6
\] and a sub-family whose eigenpoints have the following alignments: \[
  (P_1, P_2, P_3),\ (P_1, P_4, P_5), \ 
  (P_1, P_6, P_7),\ (P_2, P_5, P_6), \ 
  (P_3, P_4, P_6),\ (P_3, P_5, P_7)
\] In this case, the point\textasciitilde{}\(P_6\) (given, for instance,
by the formula \(P_6 = s_{15} \, P_3 + s_{13} \, P_5\)) is obtained as
the intersection \((P_2 \vee P_5) \cap (P_3 \vee P_4)\) and consequently
\(P_7 = (P_1 \vee P_6) \cap (P_3 \vee P_5)\).

No other collinearities among the eigenpoints are possible.

    \begin{tcolorbox}[breakable, size=fbox, boxrule=1pt, pad at break*=1mm,colback=cellbackground, colframe=cellborder]
\prompt{In}{incolor}{1}{\boxspacing}
\begin{Verbatim}[commandchars=\\\{\}]
\PY{n}{load}\PY{p}{(}\PY{l+s+s2}{\PYZdq{}}\PY{l+s+s2}{basic\PYZus{}functions.sage}\PY{l+s+s2}{\PYZdq{}}\PY{p}{)}
\end{Verbatim}
\end{tcolorbox}

    \begin{tcolorbox}[breakable, size=fbox, boxrule=1pt, pad at break*=1mm,colback=cellbackground, colframe=cellborder]
\prompt{In}{incolor}{2}{\boxspacing}
\begin{Verbatim}[commandchars=\\\{\}]
\PY{n}{P1} \PY{o}{=} \PY{n}{vector}\PY{p}{(}\PY{n}{S}\PY{p}{,} \PY{p}{(}\PY{l+m+mi}{1}\PY{p}{,} \PY{l+m+mi}{0}\PY{p}{,} \PY{l+m+mi}{0}\PY{p}{)}\PY{p}{)}
\PY{n}{P2} \PY{o}{=} \PY{n}{vector}\PY{p}{(}\PY{n}{S}\PY{p}{,} \PY{p}{(}\PY{l+m+mi}{0}\PY{p}{,} \PY{n}{ii}\PY{p}{,} \PY{l+m+mi}{1}\PY{p}{)}\PY{p}{)}
\PY{n}{P4} \PY{o}{=} \PY{n}{vector}\PY{p}{(}\PY{n}{S}\PY{p}{,} \PY{p}{(}\PY{l+m+mi}{0}\PY{p}{,} \PY{o}{\PYZhy{}}\PY{n}{ii}\PY{p}{,} \PY{l+m+mi}{1}\PY{p}{)}\PY{p}{)}
\PY{n}{P3} \PY{o}{=} \PY{n}{u1}\PY{o}{*}\PY{n}{P1} \PY{o}{+} \PY{n}{u2}\PY{o}{*}\PY{n}{P2}
\PY{n}{P5} \PY{o}{=} \PY{n}{v1}\PY{o}{*}\PY{n}{P1} \PY{o}{+} \PY{n}{v2}\PY{o}{*}\PY{n}{P4}
\end{Verbatim}
\end{tcolorbox}

    a remark on \(\delta_1\) and \(\delta_2\): \(\delta_1(P_1, P_2, P_4)\)
is not zero, while \(\delta_2(P_1, P_2, P_3, P_4, P_5)\) is 0.

    \begin{tcolorbox}[breakable, size=fbox, boxrule=1pt, pad at break*=1mm,colback=cellbackground, colframe=cellborder]
\prompt{In}{incolor}{3}{\boxspacing}
\begin{Verbatim}[commandchars=\\\{\}]
\PY{k}{assert}\PY{p}{(}\PY{n}{delta1}\PY{p}{(}\PY{n}{P1}\PY{p}{,} \PY{n}{P2}\PY{p}{,} \PY{n}{P4}\PY{p}{)} \PY{o}{!=} \PY{l+m+mi}{0}\PY{p}{)}
\PY{k}{assert}\PY{p}{(}\PY{n}{delta2}\PY{p}{(}\PY{n}{P1}\PY{p}{,} \PY{n}{P2}\PY{p}{,} \PY{n}{P3}\PY{p}{,} \PY{n}{P4}\PY{p}{,} \PY{n}{P5}\PY{p}{)} \PY{o}{==} \PY{l+m+mi}{0}\PY{p}{)}
\end{Verbatim}
\end{tcolorbox}

    In this configuration, the line \(P_1 \vee P_2\) is tangent to the
isotropic conic in \(P_2\) and the line \(P_1 \vee P_4\) is tangent to
the isotropic conic in \(P_4\) (for all \(P_3\) and \(P_5\))

    \begin{tcolorbox}[breakable, size=fbox, boxrule=1pt, pad at break*=1mm,colback=cellbackground, colframe=cellborder]
\prompt{In}{incolor}{4}{\boxspacing}
\begin{Verbatim}[commandchars=\\\{\}]
\PY{n}{M} \PY{o}{=} \PY{n}{condition\PYZus{}matrix}\PY{p}{(}\PY{p}{[}\PY{n}{P1}\PY{p}{,} \PY{n}{P2}\PY{p}{,} \PY{n}{P3}\PY{p}{,} \PY{n}{P4}\PY{p}{,} \PY{n}{P5}\PY{p}{]}\PY{p}{,} \PY{n}{S}\PY{p}{,} \PY{n}{standard}\PY{o}{=}\PY{l+s+s2}{\PYZdq{}}\PY{l+s+s2}{all}\PY{l+s+s2}{\PYZdq{}}\PY{p}{)}
\PY{k}{assert}\PY{p}{(}\PY{n}{M}\PY{o}{.}\PY{n}{rank}\PY{p}{(}\PY{p}{)} \PY{o}{==} \PY{l+m+mi}{8}\PY{p}{)}
\end{Verbatim}
\end{tcolorbox}

    The next computation requires about 90 seconds:

    \begin{tcolorbox}[breakable, size=fbox, boxrule=1pt, pad at break*=1mm,colback=cellbackground, colframe=cellborder]
\prompt{In}{incolor}{5}{\boxspacing}
\begin{Verbatim}[commandchars=\\\{\}]
\PY{n}{ttA} \PY{o}{=} \PY{n}{cputime}\PY{p}{(}\PY{p}{)}
\PY{n}{m8} \PY{o}{=} \PY{n}{M}\PY{o}{.}\PY{n}{minors}\PY{p}{(}\PY{l+m+mi}{8}\PY{p}{)}
\PY{n+nb}{print}\PY{p}{(}\PY{n}{cputime}\PY{p}{(}\PY{p}{)}\PY{o}{\PYZhy{}}\PY{n}{ttA}\PY{p}{)}
\end{Verbatim}
\end{tcolorbox}

    \begin{Verbatim}[commandchars=\\\{\}]
52.118775
    \end{Verbatim}

    The linear system \(\Lambda(M)\) is the same of the linear system
\(\Lambda(M_e)\), where \(M_e\) is the row echelon form of \(M\)

In the matrix \(M_e\) the rows 8, 9, 10, 11, 14 are zero, the row 7 is a
multiple of the row 6 and the row 13 is a multiple of the row 12.

    \begin{tcolorbox}[breakable, size=fbox, boxrule=1pt, pad at break*=1mm,colback=cellbackground, colframe=cellborder]
\prompt{In}{incolor}{6}{\boxspacing}
\begin{Verbatim}[commandchars=\\\{\}]
\PY{n}{Me} \PY{o}{=} \PY{n}{M}\PY{o}{.}\PY{n}{echelon\PYZus{}form}\PY{p}{(}\PY{p}{)}
\PY{k}{assert}\PY{p}{(}\PY{n}{Me}\PY{p}{[}\PY{l+m+mi}{8}\PY{p}{]} \PY{o}{==} \PY{n}{vector}\PY{p}{(}\PY{n}{S}\PY{p}{,} \PY{p}{[}\PY{l+m+mi}{0} \PY{k}{for} \PY{n}{\PYZus{}} \PY{o+ow}{in} \PY{n+nb}{range}\PY{p}{(}\PY{l+m+mi}{10}\PY{p}{)}\PY{p}{]}\PY{p}{)}\PY{p}{)}
\PY{k}{assert}\PY{p}{(}\PY{n}{Me}\PY{p}{[}\PY{l+m+mi}{9}\PY{p}{]} \PY{o}{==} \PY{n}{vector}\PY{p}{(}\PY{n}{S}\PY{p}{,} \PY{p}{[}\PY{l+m+mi}{0} \PY{k}{for} \PY{n}{\PYZus{}} \PY{o+ow}{in} \PY{n+nb}{range}\PY{p}{(}\PY{l+m+mi}{10}\PY{p}{)}\PY{p}{]}\PY{p}{)}\PY{p}{)}
\PY{k}{assert}\PY{p}{(}\PY{n}{Me}\PY{p}{[}\PY{l+m+mi}{10}\PY{p}{]} \PY{o}{==} \PY{n}{vector}\PY{p}{(}\PY{n}{S}\PY{p}{,} \PY{p}{[}\PY{l+m+mi}{0} \PY{k}{for} \PY{n}{\PYZus{}} \PY{o+ow}{in} \PY{n+nb}{range}\PY{p}{(}\PY{l+m+mi}{10}\PY{p}{)}\PY{p}{]}\PY{p}{)}\PY{p}{)}
\PY{k}{assert}\PY{p}{(}\PY{n}{Me}\PY{p}{[}\PY{l+m+mi}{11}\PY{p}{]} \PY{o}{==} \PY{n}{vector}\PY{p}{(}\PY{n}{S}\PY{p}{,} \PY{p}{[}\PY{l+m+mi}{0} \PY{k}{for} \PY{n}{\PYZus{}} \PY{o+ow}{in} \PY{n+nb}{range}\PY{p}{(}\PY{l+m+mi}{10}\PY{p}{)}\PY{p}{]}\PY{p}{)}\PY{p}{)}
\PY{k}{assert}\PY{p}{(}\PY{n}{Me}\PY{p}{[}\PY{l+m+mi}{14}\PY{p}{]} \PY{o}{==} \PY{n}{vector}\PY{p}{(}\PY{n}{S}\PY{p}{,} \PY{p}{[}\PY{l+m+mi}{0} \PY{k}{for} \PY{n}{\PYZus{}} \PY{o+ow}{in} \PY{n+nb}{range}\PY{p}{(}\PY{l+m+mi}{10}\PY{p}{)}\PY{p}{]}\PY{p}{)}\PY{p}{)}
\PY{k}{assert}\PY{p}{(}\PY{n}{Me}\PY{p}{[}\PY{l+m+mi}{7}\PY{p}{]}\PY{o}{+}\PY{n}{ii}\PY{o}{*}\PY{n}{Me}\PY{p}{[}\PY{l+m+mi}{6}\PY{p}{]} \PY{o}{==} \PY{n}{vector}\PY{p}{(}\PY{n}{S}\PY{p}{,} \PY{p}{[}\PY{l+m+mi}{0} \PY{k}{for} \PY{n}{\PYZus{}} \PY{o+ow}{in} \PY{n+nb}{range}\PY{p}{(}\PY{l+m+mi}{10}\PY{p}{)}\PY{p}{]}\PY{p}{)}\PY{p}{)}
\PY{k}{assert}\PY{p}{(}\PY{n}{Me}\PY{p}{[}\PY{l+m+mi}{13}\PY{p}{]}\PY{o}{\PYZhy{}}\PY{n}{ii}\PY{o}{*}\PY{n}{Me}\PY{p}{[}\PY{l+m+mi}{12}\PY{p}{]} \PY{o}{==} \PY{n}{vector}\PY{p}{(}\PY{n}{S}\PY{p}{,} \PY{p}{[}\PY{l+m+mi}{0} \PY{k}{for} \PY{n}{\PYZus{}} \PY{o+ow}{in} \PY{n+nb}{range}\PY{p}{(}\PY{l+m+mi}{10}\PY{p}{)}\PY{p}{]}\PY{p}{)}\PY{p}{)}
\end{Verbatim}
\end{tcolorbox}

    Hence, the family of cubics which have eigenpoints \(P_1, \dots, P_5\)
is obtained from the matrix M1 below:

    \begin{tcolorbox}[breakable, size=fbox, boxrule=1pt, pad at break*=1mm,colback=cellbackground, colframe=cellborder]
\prompt{In}{incolor}{7}{\boxspacing}
\begin{Verbatim}[commandchars=\\\{\}]
\PY{n}{M1} \PY{o}{=} \PY{n}{Me}\PY{o}{.}\PY{n}{matrix\PYZus{}from\PYZus{}rows}\PY{p}{(}\PY{p}{[}\PY{l+m+mi}{0}\PY{p}{,} \PY{l+m+mi}{1}\PY{p}{,} \PY{l+m+mi}{2}\PY{p}{,} \PY{l+m+mi}{3}\PY{p}{,} \PY{l+m+mi}{4}\PY{p}{,} \PY{l+m+mi}{5}\PY{p}{,} \PY{l+m+mi}{6}\PY{p}{,} \PY{l+m+mi}{13}\PY{p}{]}\PY{p}{)}
\end{Verbatim}
\end{tcolorbox}

    \(M_1\) has rank 8, as it should be:

    \begin{tcolorbox}[breakable, size=fbox, boxrule=1pt, pad at break*=1mm,colback=cellbackground, colframe=cellborder]
\prompt{In}{incolor}{8}{\boxspacing}
\begin{Verbatim}[commandchars=\\\{\}]
\PY{k}{assert}\PY{p}{(}\PY{n}{M1}\PY{o}{.}\PY{n}{rank}\PY{p}{(}\PY{p}{)} \PY{o}{==} \PY{l+m+mi}{8}\PY{p}{)}
\end{Verbatim}
\end{tcolorbox}

    \hypertarget{now-we-construct-the-pencil-of-cubics-obtained-from-m_1}{%
\subsubsection{\texorpdfstring{Now we construct the pencil of cubics
obtained from
\(M_1\)}{Now we construct the pencil of cubics obtained from M\_1}}\label{now-we-construct-the-pencil-of-cubics-obtained-from-m_1}}

    first we compute two random cubics with \(P_1, \dots, P_5\) eigenpoints:

    \begin{tcolorbox}[breakable, size=fbox, boxrule=1pt, pad at break*=1mm,colback=cellbackground, colframe=cellborder]
\prompt{In}{incolor}{9}{\boxspacing}
\begin{Verbatim}[commandchars=\\\{\}]
\PY{n}{ff1} \PY{o}{=} \PY{n}{M1}\PY{o}{.}\PY{n}{stack}\PY{p}{(}\PY{n}{vector}\PY{p}{(}\PY{n}{S}\PY{p}{,} \PY{p}{[}\PY{l+m+mi}{0}\PY{p}{,} \PY{l+m+mi}{1}\PY{p}{,} \PY{l+m+mi}{0}\PY{p}{,} \PY{l+m+mi}{2}\PY{p}{,} \PY{o}{\PYZhy{}}\PY{l+m+mi}{1}\PY{p}{,} \PY{l+m+mi}{3}\PY{p}{,} \PY{l+m+mi}{1}\PY{p}{,} \PY{l+m+mi}{4}\PY{p}{,} \PY{o}{\PYZhy{}}\PY{l+m+mi}{2}\PY{p}{,} \PY{l+m+mi}{3}\PY{p}{]}\PY{p}{)}\PY{p}{)}\PY{o}{.}\PY{n}{stack}\PY{p}{(}\PY{n}{vector}\PY{p}{(}\PY{n}{S}\PY{p}{,} \PY{n}{mon}\PY{p}{)}\PY{p}{)}\PY{o}{.}\PY{n}{det}\PY{p}{(}\PY{p}{)}
\PY{n}{ff2} \PY{o}{=} \PY{n}{M1}\PY{o}{.}\PY{n}{stack}\PY{p}{(}\PY{n}{vector}\PY{p}{(}\PY{n}{S}\PY{p}{,} \PY{p}{[}\PY{l+m+mi}{1}\PY{p}{,} \PY{l+m+mi}{2}\PY{p}{,} \PY{l+m+mi}{0}\PY{p}{,} \PY{l+m+mi}{1}\PY{p}{,} \PY{l+m+mi}{5}\PY{p}{,} \PY{l+m+mi}{0}\PY{p}{,} \PY{l+m+mi}{1}\PY{p}{,} \PY{l+m+mi}{3}\PY{p}{,} \PY{o}{\PYZhy{}}\PY{l+m+mi}{1}\PY{p}{,} \PY{l+m+mi}{1}\PY{p}{]}\PY{p}{)}\PY{p}{)}\PY{o}{.}\PY{n}{stack}\PY{p}{(}\PY{n}{vector}\PY{p}{(}\PY{n}{S}\PY{p}{,} \PY{n}{mon}\PY{p}{)}\PY{p}{)}\PY{o}{.}\PY{n}{det}\PY{p}{(}\PY{p}{)}
\end{Verbatim}
\end{tcolorbox}

    we erase non zero factors from the two polynomials:

    \begin{tcolorbox}[breakable, size=fbox, boxrule=1pt, pad at break*=1mm,colback=cellbackground, colframe=cellborder]
\prompt{In}{incolor}{10}{\boxspacing}
\begin{Verbatim}[commandchars=\\\{\}]
\PY{n}{ff1} \PY{o}{=} \PY{n}{S}\PY{o}{.}\PY{n}{ideal}\PY{p}{(}\PY{n}{ff1}\PY{p}{)}\PY{o}{.}\PY{n}{saturation}\PY{p}{(}\PY{n}{u1}\PY{o}{*}\PY{n}{u2}\PY{o}{*}\PY{n}{v1}\PY{o}{*}\PY{n}{v2}\PY{p}{)}\PY{p}{[}\PY{l+m+mi}{0}\PY{p}{]}\PY{o}{.}\PY{n}{gens}\PY{p}{(}\PY{p}{)}\PY{p}{[}\PY{l+m+mi}{0}\PY{p}{]}
\PY{n}{ff2} \PY{o}{=} \PY{n}{S}\PY{o}{.}\PY{n}{ideal}\PY{p}{(}\PY{n}{ff2}\PY{p}{)}\PY{o}{.}\PY{n}{saturation}\PY{p}{(}\PY{n}{u1}\PY{o}{*}\PY{n}{u2}\PY{o}{*}\PY{n}{v1}\PY{o}{*}\PY{n}{v2}\PY{p}{)}\PY{p}{[}\PY{l+m+mi}{0}\PY{p}{]}\PY{o}{.}\PY{n}{gens}\PY{p}{(}\PY{p}{)}\PY{p}{[}\PY{l+m+mi}{0}\PY{p}{]}
\end{Verbatim}
\end{tcolorbox}

    Now we verify that the pencil of cubics \(\langle ff_1, ff_2 \rangle\)
is also generated by the two other simpler polynomials \(f_1\) and
\(f_2\) below.

    \begin{tcolorbox}[breakable, size=fbox, boxrule=1pt, pad at break*=1mm,colback=cellbackground, colframe=cellborder]
\prompt{In}{incolor}{11}{\boxspacing}
\begin{Verbatim}[commandchars=\\\{\}]
\PY{n}{f1} \PY{o}{=} \PY{n}{x}\PY{o}{*}\PY{p}{(}\PY{n}{x}\PY{o}{\PYZca{}}\PY{l+m+mi}{2}\PY{o}{+}\PY{l+m+mi}{3}\PY{o}{/}\PY{l+m+mi}{2}\PY{o}{*}\PY{n}{y}\PY{o}{\PYZca{}}\PY{l+m+mi}{2}\PY{o}{+}\PY{l+m+mi}{3}\PY{o}{/}\PY{l+m+mi}{2}\PY{o}{*}\PY{n}{z}\PY{o}{\PYZca{}}\PY{l+m+mi}{2}\PY{p}{)}
\PY{n}{f2} \PY{o}{=} \PY{p}{(}\PY{n}{y} \PY{o}{+} \PY{n}{ii}\PY{o}{*}\PY{n}{z}\PY{p}{)} \PY{o}{*} \PY{p}{(}\PY{n}{y} \PY{o}{+} \PY{p}{(}\PY{o}{\PYZhy{}}\PY{n}{ii}\PY{p}{)}\PY{o}{*}\PY{n}{z}\PY{p}{)} \PY{o}{*} \PY{p}{(}\PY{n}{y}\PY{o}{*}\PY{n}{u2}\PY{o}{*}\PY{n}{v1} \PY{o}{+} \PY{p}{(}\PY{o}{\PYZhy{}}\PY{n}{ii}\PY{p}{)}\PY{o}{*}\PY{n}{z}\PY{o}{*}\PY{n}{u2}\PY{o}{*}\PY{n}{v1} \PY{o}{\PYZhy{}} \PY{n}{y}\PY{o}{*}\PY{n}{u1}\PY{o}{*}\PY{n}{v2} \PY{o}{+} \PY{p}{(}\PY{o}{\PYZhy{}}\PY{n}{ii}\PY{p}{)}\PY{o}{*}\PY{n}{z}\PY{o}{*}\PY{n}{u1}\PY{o}{*}\PY{n}{v2} \PY{o}{+} \PY{p}{(}\PY{l+m+mi}{2}\PY{o}{*}\PY{n}{ii}\PY{p}{)}\PY{o}{*}\PY{n}{x}\PY{o}{*}\PY{n}{u2}\PY{o}{*}\PY{n}{v2}\PY{p}{)}
\end{Verbatim}
\end{tcolorbox}

    The computations below show that
\(\langle ff_1, ff_2 \rangle = \langle f_1, f_2 \rangle\) (we convert
the polynomials in 10-components vectors and we verify that the two
vector associated to \(ff_1\) is a linear combination of the two vectors
associated to \(f_1\) and \(f_2\) (same for \(ff_2\))

    \begin{tcolorbox}[breakable, size=fbox, boxrule=1pt, pad at break*=1mm,colback=cellbackground, colframe=cellborder]
\prompt{In}{incolor}{12}{\boxspacing}
\begin{Verbatim}[commandchars=\\\{\}]
\PY{k}{assert}\PY{p}{(}\PY{n}{matrix}\PY{p}{(}\PY{p}{[}\PY{p}{[}\PY{n}{f1}\PY{o}{.}\PY{n}{coefficient}\PY{p}{(}\PY{n}{mm}\PY{p}{)} \PY{k}{for} \PY{n}{mm} \PY{o+ow}{in} \PY{n}{mon}\PY{p}{]}\PY{p}{,} \PY{p}{[}\PY{n}{f2}\PY{o}{.}\PY{n}{coefficient}\PY{p}{(}\PY{n}{mm}\PY{p}{)} \PY{k}{for} \PY{n}{mm} \PY{o+ow}{in} \PY{n}{mon}\PY{p}{]}\PY{p}{]}\PY{p}{)}\PY{o}{.}\PY{n}{rank}\PY{p}{(}\PY{p}{)} \PY{o}{==} \PY{l+m+mi}{2}\PY{p}{)}
\PY{k}{assert}\PY{p}{(}\PY{n}{matrix}\PY{p}{(}\PY{p}{[}\PY{p}{[}\PY{n}{f1}\PY{o}{.}\PY{n}{coefficient}\PY{p}{(}\PY{n}{mm}\PY{p}{)} \PY{k}{for} \PY{n}{mm} \PY{o+ow}{in} \PY{n}{mon}\PY{p}{]}\PY{p}{,} \PY{p}{[}\PY{n}{f2}\PY{o}{.}\PY{n}{coefficient}\PY{p}{(}\PY{n}{mm}\PY{p}{)} \PY{k}{for} \PY{n}{mm} \PY{o+ow}{in} \PY{n}{mon}\PY{p}{]}\PY{p}{,} \PY{p}{[}\PY{n}{ff1}\PY{o}{.}\PY{n}{coefficient}\PY{p}{(}\PY{n}{mm}\PY{p}{)} \PY{k}{for} \PY{n}{mm} \PY{o+ow}{in} \PY{n}{mon}\PY{p}{]}\PY{p}{]}\PY{p}{)}\PY{o}{.}\PY{n}{rank}\PY{p}{(}\PY{p}{)} \PY{o}{==} \PY{l+m+mi}{2}\PY{p}{)}
\PY{k}{assert}\PY{p}{(}\PY{n}{matrix}\PY{p}{(}\PY{p}{[}\PY{p}{[}\PY{n}{f1}\PY{o}{.}\PY{n}{coefficient}\PY{p}{(}\PY{n}{mm}\PY{p}{)} \PY{k}{for} \PY{n}{mm} \PY{o+ow}{in} \PY{n}{mon}\PY{p}{]}\PY{p}{,} \PY{p}{[}\PY{n}{f2}\PY{o}{.}\PY{n}{coefficient}\PY{p}{(}\PY{n}{mm}\PY{p}{)} \PY{k}{for} \PY{n}{mm} \PY{o+ow}{in} \PY{n}{mon}\PY{p}{]}\PY{p}{,} \PY{p}{[}\PY{n}{ff2}\PY{o}{.}\PY{n}{coefficient}\PY{p}{(}\PY{n}{mm}\PY{p}{)} \PY{k}{for} \PY{n}{mm} \PY{o+ow}{in} \PY{n}{mon}\PY{p}{]}\PY{p}{]}\PY{p}{)}\PY{o}{.}\PY{n}{rank}\PY{p}{(}\PY{p}{)} \PY{o}{==} \PY{l+m+mi}{2}\PY{p}{)}
\end{Verbatim}
\end{tcolorbox}

    hence the pencil of cubics which have eigenpoints \(P_1, \dots, P_5\)
is:

    \begin{tcolorbox}[breakable, size=fbox, boxrule=1pt, pad at break*=1mm,colback=cellbackground, colframe=cellborder]
\prompt{In}{incolor}{13}{\boxspacing}
\begin{Verbatim}[commandchars=\\\{\}]
\PY{n}{cb} \PY{o}{=} \PY{n}{l1}\PY{o}{*}\PY{n}{f1}\PY{o}{+}\PY{n}{l2}\PY{o}{*}\PY{n}{f2}
\end{Verbatim}
\end{tcolorbox}

    The above is the equation of ALL the cubics with eigenpoints
\(P_1, P_2, P_3, P_4, P_5\) (it depends on \(u_1, u_2, v_1, v_2\) and
\(l_1, l_2\))

    the polynomial \(f_1\) has the lines of equation \(x+iy=0\) and
\(x-iy=0\) of eigenpoints, hence \(f_1\) does not have a finite number
of eigenpoints. This means that in the study of all the cubics given by
cb we can assume \(l_2\not = 0\)

    \begin{tcolorbox}[breakable, size=fbox, boxrule=1pt, pad at break*=1mm,colback=cellbackground, colframe=cellborder]
\prompt{In}{incolor}{14}{\boxspacing}
\begin{Verbatim}[commandchars=\\\{\}]
\PY{k}{assert}\PY{p}{(}\PY{n}{S}\PY{o}{.}\PY{n}{ideal}\PY{p}{(}\PY{n+nb}{list}\PY{p}{(}\PY{n}{eig}\PY{p}{(}\PY{n}{f1}\PY{p}{)}\PY{p}{)}\PY{p}{)}\PY{o}{.}\PY{n}{subs}\PY{p}{(}\PY{n}{y} \PY{o}{=} \PY{o}{\PYZhy{}}\PY{n}{ii}\PY{o}{*}\PY{n}{z}\PY{p}{)} \PY{o}{==} \PY{l+m+mi}{0}\PY{p}{)}
\PY{k}{assert}\PY{p}{(}\PY{n}{S}\PY{o}{.}\PY{n}{ideal}\PY{p}{(}\PY{n+nb}{list}\PY{p}{(}\PY{n}{eig}\PY{p}{(}\PY{n}{f1}\PY{p}{)}\PY{p}{)}\PY{p}{)}\PY{o}{.}\PY{n}{subs}\PY{p}{(}\PY{n}{y} \PY{o}{=} \PY{n}{ii}\PY{o}{*}\PY{n}{z}\PY{p}{)} \PY{o}{==} \PY{l+m+mi}{0}\PY{p}{)}
\end{Verbatim}
\end{tcolorbox}

    We compute now the iegenpoints of \(cb\)

    \begin{tcolorbox}[breakable, size=fbox, boxrule=1pt, pad at break*=1mm,colback=cellbackground, colframe=cellborder]
\prompt{In}{incolor}{15}{\boxspacing}
\begin{Verbatim}[commandchars=\\\{\}]
\PY{n}{ej} \PY{o}{=} \PY{n}{S}\PY{o}{.}\PY{n}{ideal}\PY{p}{(}\PY{n+nb}{list}\PY{p}{(}\PY{n}{eig}\PY{p}{(}\PY{n}{cb}\PY{p}{)}\PY{p}{)}\PY{p}{)}
\end{Verbatim}
\end{tcolorbox}

    we erase the known eigenpoints from ej:

    \begin{tcolorbox}[breakable, size=fbox, boxrule=1pt, pad at break*=1mm,colback=cellbackground, colframe=cellborder]
\prompt{In}{incolor}{16}{\boxspacing}
\begin{Verbatim}[commandchars=\\\{\}]
\PY{k}{for} \PY{n}{pp} \PY{o+ow}{in} \PY{p}{[}\PY{n}{P1}\PY{p}{,} \PY{n}{P2}\PY{p}{,} \PY{n}{P3}\PY{p}{,} \PY{n}{P4}\PY{p}{,} \PY{n}{P5}\PY{p}{]}\PY{p}{:}
    \PY{n}{ej} \PY{o}{=} \PY{n}{ej}\PY{o}{.}\PY{n}{saturation}\PY{p}{(}\PY{n}{S}\PY{o}{.}\PY{n}{ideal}\PY{p}{(}\PY{n}{matrix}\PY{p}{(}\PY{p}{[}\PY{n}{pp}\PY{p}{,} \PY{p}{(}\PY{n}{x}\PY{p}{,} \PY{n}{y}\PY{p}{,} \PY{n}{z}\PY{p}{)}\PY{p}{]}\PY{p}{)}\PY{o}{.}\PY{n}{minors}\PY{p}{(}\PY{l+m+mi}{2}\PY{p}{)}\PY{p}{)}\PY{p}{)}\PY{p}{[}\PY{l+m+mi}{0}\PY{p}{]}
\end{Verbatim}
\end{tcolorbox}

    \begin{tcolorbox}[breakable, size=fbox, boxrule=1pt, pad at break*=1mm,colback=cellbackground, colframe=cellborder]
\prompt{In}{incolor}{17}{\boxspacing}
\begin{Verbatim}[commandchars=\\\{\}]
\PY{n}{ej} \PY{o}{=} \PY{n}{ej}\PY{o}{.}\PY{n}{saturation}\PY{p}{(}\PY{n}{u1}\PY{o}{*}\PY{n}{u2}\PY{o}{*}\PY{n}{v1}\PY{o}{*}\PY{n}{v2}\PY{o}{*}\PY{n}{l2}\PY{p}{)}\PY{p}{[}\PY{l+m+mi}{0}\PY{p}{]}
\end{Verbatim}
\end{tcolorbox}

    \begin{tcolorbox}[breakable, size=fbox, boxrule=1pt, pad at break*=1mm,colback=cellbackground, colframe=cellborder]
\prompt{In}{incolor}{18}{\boxspacing}
\begin{Verbatim}[commandchars=\\\{\}]
\PY{n}{ej} \PY{o}{=} \PY{n}{ej}\PY{o}{.}\PY{n}{radical}\PY{p}{(}\PY{p}{)}
\PY{k}{assert}\PY{p}{(}\PY{n}{ej}\PY{o}{.}\PY{n}{is\PYZus{}prime}\PY{p}{(}\PY{p}{)}\PY{p}{)}
\end{Verbatim}
\end{tcolorbox}

    Now \(ej\) is a prime ideal with two generators: a polynomial \(rt\) of
degree 1 in \(x, y, z\) and a polynomial of degree 2 in \(x, y, z\).
Since \(ej\) gives the remaining eigenpoints of \(cb\), we see that they
are the intersection of a line \(rt\) and a conic \(qn\)

    \begin{tcolorbox}[breakable, size=fbox, boxrule=1pt, pad at break*=1mm,colback=cellbackground, colframe=cellborder]
\prompt{In}{incolor}{19}{\boxspacing}
\begin{Verbatim}[commandchars=\\\{\}]
\PY{n}{rt} \PY{o}{=} \PY{n}{y}\PY{o}{*}\PY{n}{u2}\PY{o}{*}\PY{n}{v1} \PY{o}{+} \PY{p}{(}\PY{o}{\PYZhy{}}\PY{n}{ii}\PY{p}{)}\PY{o}{*}\PY{n}{z}\PY{o}{*}\PY{n}{u2}\PY{o}{*}\PY{n}{v1} \PY{o}{+} \PY{n}{y}\PY{o}{*}\PY{n}{u1}\PY{o}{*}\PY{n}{v2} \PY{o}{+} \PY{n}{ii}\PY{o}{*}\PY{n}{z}\PY{o}{*}\PY{n}{u1}\PY{o}{*}\PY{n}{v2} 

\PY{n}{qn} \PY{o}{=} \PY{l+m+mi}{12}\PY{o}{*}\PY{n}{x}\PY{o}{*}\PY{n}{y}\PY{o}{*}\PY{n}{u1}\PY{o}{*}\PY{n}{v2}\PY{o}{*}\PY{n}{l2} \PY{o}{+} \PY{p}{(}\PY{l+m+mi}{12}\PY{o}{*}\PY{n}{ii}\PY{p}{)}\PY{o}{*}\PY{n}{x}\PY{o}{*}\PY{n}{z}\PY{o}{*}\PY{n}{u1}\PY{o}{*}\PY{n}{v2}\PY{o}{*}\PY{n}{l2} \PY{o}{+} \PY{p}{(}\PY{o}{\PYZhy{}}\PY{l+m+mi}{8}\PY{o}{*}\PY{n}{ii}\PY{p}{)}\PY{o}{*}\PY{n}{x}\PY{o}{\PYZca{}}\PY{l+m+mi}{2}\PY{o}{*}\PY{n}{u2}\PY{o}{*}\PY{n}{v2}\PY{o}{*}\PY{n}{l2} \PY{o}{+} \PY{p}{(}\PY{l+m+mi}{4}\PY{o}{*}\PY{n}{ii}\PY{p}{)}\PY{o}{*}\PY{n}{y}\PY{o}{\PYZca{}}\PY{l+m+mi}{2}\PY{o}{*}\PY{n}{u2}\PY{o}{*}\PY{n}{v2}\PY{o}{*}\PY{n}{l2} \PY{o}{+} \PY{p}{(}\PY{l+m+mi}{4}\PY{o}{*}\PY{n}{ii}\PY{p}{)}\PY{o}{*}\PY{n}{z}\PY{o}{\PYZca{}}\PY{l+m+mi}{2}\PY{o}{*}\PY{n}{u2}\PY{o}{*}\PY{n}{v2}\PY{o}{*}\PY{n}{l2} \PY{o}{+} \PY{l+m+mi}{3}\PY{o}{*}\PY{n}{y}\PY{o}{\PYZca{}}\PY{l+m+mi}{2}\PY{o}{*}\PY{n}{l1} \PY{o}{+} \PY{l+m+mi}{3}\PY{o}{*}\PY{n}{z}\PY{o}{\PYZca{}}\PY{l+m+mi}{2}\PY{o}{*}\PY{n}{l1}

\PY{k}{assert}\PY{p}{(}\PY{n}{ej} \PY{o}{==} \PY{n}{S}\PY{o}{.}\PY{n}{ideal}\PY{p}{(}\PY{n}{rt}\PY{p}{,} \PY{n}{qn}\PY{p}{)}\PY{o}{.}\PY{n}{saturation}\PY{p}{(}\PY{n}{u1}\PY{o}{*}\PY{n}{u2}\PY{o}{*}\PY{n}{v1}\PY{o}{*}\PY{n}{v2}\PY{p}{)}\PY{p}{[}\PY{l+m+mi}{0}\PY{p}{]}\PY{p}{)}
\end{Verbatim}
\end{tcolorbox}

    The line \(rt\) contains therefore the points \(P_6\) and \(P_7\) and it
passes through \(P_1\):

    \begin{tcolorbox}[breakable, size=fbox, boxrule=1pt, pad at break*=1mm,colback=cellbackground, colframe=cellborder]
\prompt{In}{incolor}{20}{\boxspacing}
\begin{Verbatim}[commandchars=\\\{\}]
\PY{k}{assert}\PY{p}{(}\PY{n}{rt}\PY{o}{.}\PY{n}{subs}\PY{p}{(}\PY{n}{substitution}\PY{p}{(}\PY{n}{P1}\PY{p}{)}\PY{p}{)}\PY{o}{==} \PY{l+m+mi}{0}\PY{p}{)}
\end{Verbatim}
\end{tcolorbox}

    The line \(rt\) is orthogonal to the line \(P_3 \vee P_5\), as follows
from these computations:

    \begin{tcolorbox}[breakable, size=fbox, boxrule=1pt, pad at break*=1mm,colback=cellbackground, colframe=cellborder]
\prompt{In}{incolor}{21}{\boxspacing}
\begin{Verbatim}[commandchars=\\\{\}]
\PY{n}{r35} \PY{o}{=} \PY{n}{matrix}\PY{p}{(}\PY{p}{[}\PY{n}{P3}\PY{p}{,} \PY{n}{P5}\PY{p}{,} \PY{p}{(}\PY{n}{x}\PY{p}{,} \PY{n}{y}\PY{p}{,} \PY{n}{z}\PY{p}{)}\PY{p}{]}\PY{p}{)}\PY{o}{.}\PY{n}{det}\PY{p}{(}\PY{p}{)}
\end{Verbatim}
\end{tcolorbox}

    r35 and rt are orthogonal:

    \begin{tcolorbox}[breakable, size=fbox, boxrule=1pt, pad at break*=1mm,colback=cellbackground, colframe=cellborder]
\prompt{In}{incolor}{22}{\boxspacing}
\begin{Verbatim}[commandchars=\\\{\}]
\PY{k}{assert}\PY{p}{(}
    \PY{n}{scalar\PYZus{}product}\PY{p}{(}
        \PY{p}{[}\PY{n}{r35}\PY{o}{.}\PY{n}{coefficient}\PY{p}{(}\PY{n}{xx}\PY{p}{)} \PY{k}{for} \PY{n}{xx} \PY{o+ow}{in} \PY{p}{(}\PY{n}{x}\PY{p}{,} \PY{n}{y}\PY{p}{,} \PY{n}{z}\PY{p}{)}\PY{p}{]}\PY{p}{,}
        \PY{p}{[}\PY{n}{rt}\PY{o}{.}\PY{n}{coefficient}\PY{p}{(}\PY{n}{xx}\PY{p}{)} \PY{k}{for} \PY{n}{xx} \PY{o+ow}{in} \PY{p}{(}\PY{n}{x}\PY{p}{,} \PY{n}{y}\PY{p}{,} \PY{n}{z}\PY{p}{)}\PY{p}{]}
    \PY{p}{)}\PY{o}{.}\PY{n}{is\PYZus{}zero}\PY{p}{(}\PY{p}{)}
\PY{p}{)}
\end{Verbatim}
\end{tcolorbox}

    The points \(P_6\) and \(P_7\), which are the itersecton point of \(rt\)
and conic, do not have coordinates in \(\mathbb{Q}[i][A, B, C, u, v]\),
since the ideal \(ej\) is prime.

    \hypertarget{subcases}{%
\subsection{Subcases:}\label{subcases}}

We want to see if there are other alignments among the points. We have
to consider three cases:

\begin{itemize}
\tightlist
\item
  Case 1: \(P_2, P_4, P_6\) aligned
\item
  Case 2: \(P_2, P_5, P_6\) aligned
\item
  Case 3: \(P_3, P_5, P_7\) (or \(P_3, P_5, P_6\)) aligned
\end{itemize}

    \hypertarget{case-1-p_2-p_4-p_6-aligned.}{%
\subsection{\texorpdfstring{Case 1: \(P_2, P_4, P_6\)
aligned.}{Case 1: P\_2, P\_4, P\_6 aligned.}}\label{case-1-p_2-p_4-p_6-aligned.}}

    The line \(P_2 \vee P_4\) is \(r24 = x\):

    \begin{tcolorbox}[breakable, size=fbox, boxrule=1pt, pad at break*=1mm,colback=cellbackground, colframe=cellborder]
\prompt{In}{incolor}{23}{\boxspacing}
\begin{Verbatim}[commandchars=\\\{\}]
\PY{n}{r24} \PY{o}{=} \PY{n}{x}
\PY{k}{assert}\PY{p}{(}\PY{n}{matrix}\PY{p}{(}\PY{p}{[}\PY{n}{P2}\PY{p}{,} \PY{n}{P4}\PY{p}{,} \PY{p}{(}\PY{n}{x}\PY{p}{,} \PY{n}{y}\PY{p}{,} \PY{n}{z}\PY{p}{)}\PY{p}{]}\PY{p}{)}\PY{o}{.}\PY{n}{det}\PY{p}{(}\PY{p}{)} \PY{o}{==} \PY{l+m+mi}{2}\PY{o}{*}\PY{n}{ii}\PY{o}{*}\PY{n}{x}\PY{p}{)}
\end{Verbatim}
\end{tcolorbox}

    Construction of the point \(P_6\) as intersection of the lines rt and
r24:

    \begin{tcolorbox}[breakable, size=fbox, boxrule=1pt, pad at break*=1mm,colback=cellbackground, colframe=cellborder]
\prompt{In}{incolor}{24}{\boxspacing}
\begin{Verbatim}[commandchars=\\\{\}]
\PY{n}{P6} \PY{o}{=} \PY{n}{vector}\PY{p}{(}\PY{n}{S}\PY{p}{,} \PY{p}{(}\PY{l+m+mi}{0}\PY{p}{,} \PY{n}{rt}\PY{o}{.}\PY{n}{coefficient}\PY{p}{(}\PY{n}{z}\PY{p}{)}\PY{p}{,} \PY{o}{\PYZhy{}}\PY{n}{rt}\PY{o}{.}\PY{n}{coefficient}\PY{p}{(}\PY{n}{y}\PY{p}{)}\PY{p}{)}\PY{p}{)}
\end{Verbatim}
\end{tcolorbox}

    \begin{tcolorbox}[breakable, size=fbox, boxrule=1pt, pad at break*=1mm,colback=cellbackground, colframe=cellborder]
\prompt{In}{incolor}{25}{\boxspacing}
\begin{Verbatim}[commandchars=\\\{\}]
\PY{k}{assert}\PY{p}{(}\PY{n}{rt}\PY{o}{.}\PY{n}{subs}\PY{p}{(}\PY{n}{substitution}\PY{p}{(}\PY{n}{P6}\PY{p}{)}\PY{p}{)} \PY{o}{==} \PY{l+m+mi}{0}\PY{p}{)}
\PY{k}{assert}\PY{p}{(}\PY{n}{r24}\PY{o}{.}\PY{n}{subs}\PY{p}{(}\PY{n}{substitution}\PY{p}{(}\PY{n}{P6}\PY{p}{)}\PY{p}{)} \PY{o}{==} \PY{l+m+mi}{0}\PY{p}{)}
\end{Verbatim}
\end{tcolorbox}

    \(P_6\) always exists:

    \begin{tcolorbox}[breakable, size=fbox, boxrule=1pt, pad at break*=1mm,colback=cellbackground, colframe=cellborder]
\prompt{In}{incolor}{26}{\boxspacing}
\begin{Verbatim}[commandchars=\\\{\}]
\PY{k}{assert}\PY{p}{(}\PY{n}{S}\PY{o}{.}\PY{n}{ideal}\PY{p}{(}\PY{n+nb}{list}\PY{p}{(}\PY{n}{P6}\PY{p}{)}\PY{p}{)}\PY{o}{.}\PY{n}{saturation}\PY{p}{(}\PY{n}{u1}\PY{o}{*}\PY{n}{u2}\PY{o}{*}\PY{n}{v1}\PY{o}{*}\PY{n}{v2}\PY{p}{)}\PY{p}{[}\PY{l+m+mi}{0}\PY{p}{]} \PY{o}{==} \PY{n}{S}\PY{o}{.}\PY{n}{ideal}\PY{p}{(}\PY{l+m+mi}{1}\PY{p}{)}\PY{p}{)}
\end{Verbatim}
\end{tcolorbox}

    If \(P_6\) is an eigenpoint, it must be a point of the conic qn. This
condition gives that \(l_1 = u_2v_2\), \(l_2 = 3/4i\):

    \begin{tcolorbox}[breakable, size=fbox, boxrule=1pt, pad at break*=1mm,colback=cellbackground, colframe=cellborder]
\prompt{In}{incolor}{27}{\boxspacing}
\begin{Verbatim}[commandchars=\\\{\}]
\PY{k}{assert}\PY{p}{(}\PY{n}{qn}\PY{o}{.}\PY{n}{subs}\PY{p}{(}\PY{n}{substitution}\PY{p}{(}\PY{n}{P6}\PY{p}{)}\PY{p}{)} \PY{o}{==} \PY{p}{(}\PY{p}{(}\PY{l+m+mi}{16}\PY{o}{*}\PY{n}{ii}\PY{p}{)}\PY{p}{)}\PY{o}{*}\PY{n}{v2}\PY{o}{*}\PY{n}{v1}\PY{o}{*}\PY{n}{u2}\PY{o}{*}\PY{n}{u1}\PY{o}{*}\PY{p}{(}\PY{n}{u2}\PY{o}{*}\PY{n}{v2}\PY{o}{*}\PY{n}{l2} \PY{o}{+} \PY{p}{(}\PY{o}{\PYZhy{}}\PY{l+m+mi}{3}\PY{o}{/}\PY{l+m+mi}{4}\PY{o}{*}\PY{n}{ii}\PY{p}{)}\PY{o}{*}\PY{n}{l1}\PY{p}{)}\PY{p}{)}
\end{Verbatim}
\end{tcolorbox}

    the family of cubics in which \(P_6\) is aligned with \(P_2\) and
\(P_4\) is therefore obtained from cb as follows:

    \begin{tcolorbox}[breakable, size=fbox, boxrule=1pt, pad at break*=1mm,colback=cellbackground, colframe=cellborder]
\prompt{In}{incolor}{28}{\boxspacing}
\begin{Verbatim}[commandchars=\\\{\}]
\PY{n}{cb1} \PY{o}{=} \PY{n}{S}\PY{p}{(}\PY{n}{cb}\PY{o}{.}\PY{n}{subs}\PY{p}{(}\PY{p}{\PYZob{}}\PY{n}{l1}\PY{p}{:} \PY{n}{u2}\PY{o}{*}\PY{n}{v2}\PY{p}{,} \PY{n}{l2}\PY{p}{:} \PY{l+m+mi}{3}\PY{o}{/}\PY{l+m+mi}{4}\PY{o}{*}\PY{n}{ii}\PY{p}{\PYZcb{}}\PY{p}{)}\PY{p}{)}
\end{Verbatim}
\end{tcolorbox}

    cb1 is smooth:

    \begin{tcolorbox}[breakable, size=fbox, boxrule=1pt, pad at break*=1mm,colback=cellbackground, colframe=cellborder]
\prompt{In}{incolor}{29}{\boxspacing}
\begin{Verbatim}[commandchars=\\\{\}]
\PY{k}{assert}\PY{p}{(}\PY{n}{S}\PY{o}{.}\PY{n}{ideal}\PY{p}{(}\PY{n+nb}{list}\PY{p}{(}\PY{n}{gdn}\PY{p}{(}\PY{n}{cb1}\PY{p}{)}\PY{p}{)}\PY{p}{)}\PY{o}{.}\PY{n}{saturation}\PY{p}{(}\PY{n}{u1}\PY{o}{*}\PY{n}{u2}\PY{o}{*}\PY{n}{v1}\PY{o}{*}\PY{n}{v2}\PY{p}{)}\PY{p}{[}\PY{l+m+mi}{0}\PY{p}{]}\PY{o}{.}\PY{n}{radical}\PY{p}{(}\PY{p}{)} \PY{o}{==} \PY{n}{S}\PY{o}{.}\PY{n}{ideal}\PY{p}{(}\PY{n}{x}\PY{p}{,} \PY{n}{y}\PY{p}{,} \PY{n}{z}\PY{p}{)}\PY{p}{)}
\end{Verbatim}
\end{tcolorbox}

    Construction of \(P_7\). It is the second intersection of rt and conic,
and these two polynomials give the ideal ej: first we evaluate ej on
\(l_1 = u_2v_2\), \(l_2 = 3/4i\), then we saturate the ideal, we divide
it by the ideal of \(P_6\), we get the ideal pp7 which gives the
coordinates of the point \(P_7\)

    \begin{tcolorbox}[breakable, size=fbox, boxrule=1pt, pad at break*=1mm,colback=cellbackground, colframe=cellborder]
\prompt{In}{incolor}{30}{\boxspacing}
\begin{Verbatim}[commandchars=\\\{\}]
\PY{n}{ej1} \PY{o}{=} \PY{n}{ej}\PY{o}{.}\PY{n}{subs}\PY{p}{(}\PY{p}{\PYZob{}}\PY{n}{l1}\PY{p}{:} \PY{n}{u2}\PY{o}{*}\PY{n}{v2}\PY{p}{,} \PY{n}{l2}\PY{p}{:} \PY{l+m+mi}{3}\PY{o}{/}\PY{l+m+mi}{4}\PY{o}{*}\PY{n}{ii}\PY{p}{\PYZcb{}}\PY{p}{)}
\end{Verbatim}
\end{tcolorbox}

    \begin{tcolorbox}[breakable, size=fbox, boxrule=1pt, pad at break*=1mm,colback=cellbackground, colframe=cellborder]
\prompt{In}{incolor}{31}{\boxspacing}
\begin{Verbatim}[commandchars=\\\{\}]
\PY{n}{ej1} \PY{o}{=} \PY{n}{ej1}\PY{o}{.}\PY{n}{saturation}\PY{p}{(}\PY{n}{u1}\PY{o}{*}\PY{n}{u2}\PY{o}{*}\PY{n}{v1}\PY{o}{*}\PY{n}{v2}\PY{p}{)}\PY{p}{[}\PY{l+m+mi}{0}\PY{p}{]}
\end{Verbatim}
\end{tcolorbox}

    \begin{tcolorbox}[breakable, size=fbox, boxrule=1pt, pad at break*=1mm,colback=cellbackground, colframe=cellborder]
\prompt{In}{incolor}{32}{\boxspacing}
\begin{Verbatim}[commandchars=\\\{\}]
\PY{n}{pp7} \PY{o}{=} \PY{n}{ej1}\PY{o}{.}\PY{n}{saturation}\PY{p}{(}\PY{n}{S}\PY{o}{.}\PY{n}{ideal}\PY{p}{(}\PY{n}{matrix}\PY{p}{(}\PY{p}{[}\PY{n}{P6}\PY{p}{,} \PY{p}{(}\PY{n}{x}\PY{p}{,} \PY{n}{y}\PY{p}{,} \PY{n}{z}\PY{p}{)}\PY{p}{]}\PY{p}{)}\PY{o}{.}\PY{n}{minors}\PY{p}{(}\PY{l+m+mi}{2}\PY{p}{)}\PY{p}{)}\PY{p}{)}\PY{p}{[}\PY{l+m+mi}{0}\PY{p}{]}
\end{Verbatim}
\end{tcolorbox}

    \begin{tcolorbox}[breakable, size=fbox, boxrule=1pt, pad at break*=1mm,colback=cellbackground, colframe=cellborder]
\prompt{In}{incolor}{33}{\boxspacing}
\begin{Verbatim}[commandchars=\\\{\}]
\PY{n}{p7coord} \PY{o}{=} \PY{n}{matrix}\PY{p}{(}
    \PY{p}{[}
        \PY{p}{[}
            \PY{n}{pp7}\PY{o}{.}\PY{n}{gens}\PY{p}{(}\PY{p}{)}\PY{p}{[}\PY{l+m+mi}{0}\PY{p}{]}\PY{o}{.}\PY{n}{coefficient}\PY{p}{(}\PY{n}{xx}\PY{p}{)} \PY{k}{for} \PY{n}{xx} \PY{o+ow}{in} \PY{p}{[}\PY{n}{x}\PY{p}{,} \PY{n}{y}\PY{p}{,} \PY{n}{z}\PY{p}{]}
        \PY{p}{]}\PY{p}{,} 
        \PY{p}{[}
            \PY{n}{pp7}\PY{o}{.}\PY{n}{gens}\PY{p}{(}\PY{p}{)}\PY{p}{[}\PY{l+m+mi}{1}\PY{p}{]}\PY{o}{.}\PY{n}{coefficient}\PY{p}{(}\PY{n}{xx}\PY{p}{)} \PY{k}{for} \PY{n}{xx} \PY{o+ow}{in} \PY{p}{[}\PY{n}{x}\PY{p}{,} \PY{n}{y}\PY{p}{,} \PY{n}{z}\PY{p}{]}
        \PY{p}{]}
    \PY{p}{]}
\PY{p}{)}\PY{o}{.}\PY{n}{minors}\PY{p}{(}\PY{l+m+mi}{2}\PY{p}{)}
\end{Verbatim}
\end{tcolorbox}

    \begin{tcolorbox}[breakable, size=fbox, boxrule=1pt, pad at break*=1mm,colback=cellbackground, colframe=cellborder]
\prompt{In}{incolor}{34}{\boxspacing}
\begin{Verbatim}[commandchars=\\\{\}]
\PY{n}{P7} \PY{o}{=} \PY{n}{vector}\PY{p}{(}\PY{n}{S}\PY{p}{,} \PY{p}{(}\PY{n}{p7coord}\PY{p}{[}\PY{l+m+mi}{2}\PY{p}{]}\PY{p}{,} \PY{o}{\PYZhy{}}\PY{n}{p7coord}\PY{p}{[}\PY{l+m+mi}{1}\PY{p}{]}\PY{p}{,} \PY{n}{p7coord}\PY{p}{[}\PY{l+m+mi}{0}\PY{p}{]}\PY{p}{)}\PY{p}{)}
\end{Verbatim}
\end{tcolorbox}

    P7 is a common point of rt and conic (i.e.~a zero of ej1) and \(P_7\)
always exists:

    \begin{tcolorbox}[breakable, size=fbox, boxrule=1pt, pad at break*=1mm,colback=cellbackground, colframe=cellborder]
\prompt{In}{incolor}{35}{\boxspacing}
\begin{Verbatim}[commandchars=\\\{\}]
\PY{k}{assert}\PY{p}{(}\PY{n}{ej1}\PY{o}{.}\PY{n}{subs}\PY{p}{(}\PY{n}{substitution}\PY{p}{(}\PY{n}{P7}\PY{p}{)}\PY{p}{)} \PY{o}{==} \PY{n}{S}\PY{o}{.}\PY{n}{ideal}\PY{p}{(}\PY{l+m+mi}{0}\PY{p}{)}\PY{p}{)}
\end{Verbatim}
\end{tcolorbox}

    \begin{tcolorbox}[breakable, size=fbox, boxrule=1pt, pad at break*=1mm,colback=cellbackground, colframe=cellborder]
\prompt{In}{incolor}{36}{\boxspacing}
\begin{Verbatim}[commandchars=\\\{\}]
\PY{k}{assert}\PY{p}{(}\PY{n}{S}\PY{o}{.}\PY{n}{ideal}\PY{p}{(}\PY{n+nb}{list}\PY{p}{(}\PY{n}{P7}\PY{p}{)}\PY{p}{)}\PY{o}{.}\PY{n}{saturation}\PY{p}{(}\PY{n}{u1}\PY{o}{*}\PY{n}{u2}\PY{o}{*}\PY{n}{v1}\PY{o}{*}\PY{n}{v2}\PY{p}{)}\PY{p}{[}\PY{l+m+mi}{0}\PY{p}{]} \PY{o}{==} \PY{n}{S}\PY{o}{.}\PY{n}{ideal}\PY{p}{(}\PY{l+m+mi}{1}\PY{p}{)}\PY{p}{)}
\end{Verbatim}
\end{tcolorbox}

    \(P_6\) is obtained from the formula:
\(P_6 = s_{11}s_{15}P_3-2s_{13}s_{15}P_1+s_{11}s_{15}P_5\)

    \begin{tcolorbox}[breakable, size=fbox, boxrule=1pt, pad at break*=1mm,colback=cellbackground, colframe=cellborder]
\prompt{In}{incolor}{37}{\boxspacing}
\begin{Verbatim}[commandchars=\\\{\}]
\PY{n}{P6a} \PY{o}{=} \PY{p}{(}
    \PY{n}{scalar\PYZus{}product}\PY{p}{(}\PY{n}{P1}\PY{p}{,} \PY{n}{P1}\PY{p}{)}\PY{o}{*}\PY{n}{scalar\PYZus{}product}\PY{p}{(}\PY{n}{P1}\PY{p}{,} \PY{n}{P5}\PY{p}{)}\PY{o}{*}\PY{n}{P3}
    \PY{o}{\PYZhy{}} \PY{l+m+mi}{2}\PY{o}{*}\PY{n}{scalar\PYZus{}product}\PY{p}{(}\PY{n}{P1}\PY{p}{,} \PY{n}{P3}\PY{p}{)}\PY{o}{*}\PY{n}{scalar\PYZus{}product}\PY{p}{(}\PY{n}{P1}\PY{p}{,} \PY{n}{P5}\PY{p}{)}\PY{o}{*}\PY{n}{P1}
    \PY{o}{+} \PY{n}{scalar\PYZus{}product}\PY{p}{(}\PY{n}{P1}\PY{p}{,} \PY{n}{P1}\PY{p}{)}\PY{o}{*}\PY{n}{scalar\PYZus{}product}\PY{p}{(}\PY{n}{P1}\PY{p}{,} \PY{n}{P3}\PY{p}{)}\PY{o}{*}\PY{n}{P5}
\PY{p}{)}

\PY{k}{assert}\PY{p}{(}\PY{n}{S}\PY{o}{.}\PY{n}{ideal}\PY{p}{(}\PY{n}{matrix}\PY{p}{(}\PY{p}{[}\PY{n}{P6}\PY{p}{,} \PY{n}{P6a}\PY{p}{]}\PY{p}{)}\PY{o}{.}\PY{n}{minors}\PY{p}{(}\PY{l+m+mi}{2}\PY{p}{)}\PY{p}{)} \PY{o}{==} \PY{n}{S}\PY{o}{.}\PY{n}{ideal}\PY{p}{(}\PY{l+m+mi}{0}\PY{p}{)}\PY{p}{)}
\end{Verbatim}
\end{tcolorbox}

    \(P_7\) is obtained from the formula:
\(P_7 = s_{11}s_{15}P_3+s_{13}s_{15}P_1+s_{11}s_{15}P_5\)

    \begin{tcolorbox}[breakable, size=fbox, boxrule=1pt, pad at break*=1mm,colback=cellbackground, colframe=cellborder]
\prompt{In}{incolor}{38}{\boxspacing}
\begin{Verbatim}[commandchars=\\\{\}]
\PY{n}{P7a} \PY{o}{=} \PY{p}{(}
    \PY{n}{scalar\PYZus{}product}\PY{p}{(}\PY{n}{P1}\PY{p}{,} \PY{n}{P1}\PY{p}{)}\PY{o}{*}\PY{n}{scalar\PYZus{}product}\PY{p}{(}\PY{n}{P1}\PY{p}{,} \PY{n}{P5}\PY{p}{)}\PY{o}{*}\PY{n}{P3}
    \PY{o}{+} \PY{n}{scalar\PYZus{}product}\PY{p}{(}\PY{n}{P1}\PY{p}{,} \PY{n}{P3}\PY{p}{)}\PY{o}{*}\PY{n}{scalar\PYZus{}product}\PY{p}{(}\PY{n}{P1}\PY{p}{,} \PY{n}{P5}\PY{p}{)}\PY{o}{*}\PY{n}{P1}
    \PY{o}{+} \PY{n}{scalar\PYZus{}product}\PY{p}{(}\PY{n}{P1}\PY{p}{,} \PY{n}{P1}\PY{p}{)}\PY{o}{*}\PY{n}{scalar\PYZus{}product}\PY{p}{(}\PY{n}{P1}\PY{p}{,} \PY{n}{P3}\PY{p}{)}\PY{o}{*}\PY{n}{P5}
\PY{p}{)}

\PY{k}{assert}\PY{p}{(}\PY{n}{S}\PY{o}{.}\PY{n}{ideal}\PY{p}{(}\PY{n}{matrix}\PY{p}{(}\PY{p}{[}\PY{n}{P7}\PY{p}{,} \PY{n}{P7a}\PY{p}{]}\PY{p}{)}\PY{o}{.}\PY{n}{minors}\PY{p}{(}\PY{l+m+mi}{2}\PY{p}{)}\PY{p}{)} \PY{o}{==} \PY{n}{S}\PY{o}{.}\PY{n}{ideal}\PY{p}{(}\PY{l+m+mi}{0}\PY{p}{)}\PY{p}{)}
\end{Verbatim}
\end{tcolorbox}

    The eigenpoints \(P_1, \dots, P_7\) have the following alignments:
\([(1, 2, 3), (1, 4, 5), (1, 6, 7), (2, 4, 6)]\)

    \begin{tcolorbox}[breakable, size=fbox, boxrule=1pt, pad at break*=1mm,colback=cellbackground, colframe=cellborder]
\prompt{In}{incolor}{39}{\boxspacing}
\begin{Verbatim}[commandchars=\\\{\}]
\PY{k}{assert}\PY{p}{(}\PY{n}{alignments}\PY{p}{(}\PY{p}{[}\PY{n}{P1}\PY{p}{,} \PY{n}{P2}\PY{p}{,} \PY{n}{P3}\PY{p}{,} \PY{n}{P4}\PY{p}{,} \PY{n}{P5}\PY{p}{,} \PY{n}{P6}\PY{p}{,} \PY{n}{P7}\PY{p}{]}\PY{p}{)} \PY{o}{==} \PY{p}{[}\PY{p}{(}\PY{l+m+mi}{1}\PY{p}{,} \PY{l+m+mi}{2}\PY{p}{,} \PY{l+m+mi}{3}\PY{p}{)}\PY{p}{,} \PY{p}{(}\PY{l+m+mi}{1}\PY{p}{,} \PY{l+m+mi}{4}\PY{p}{,} \PY{l+m+mi}{5}\PY{p}{)}\PY{p}{,} \PY{p}{(}\PY{l+m+mi}{1}\PY{p}{,} \PY{l+m+mi}{6}\PY{p}{,} \PY{l+m+mi}{7}\PY{p}{)}\PY{p}{,} \PY{p}{(}\PY{l+m+mi}{2}\PY{p}{,} \PY{l+m+mi}{4}\PY{p}{,} \PY{l+m+mi}{6}\PY{p}{)}\PY{p}{]}\PY{p}{)}
\end{Verbatim}
\end{tcolorbox}

    There are no subcases in which the seven eigenpints have other
alignments:

    \begin{tcolorbox}[breakable, size=fbox, boxrule=1pt, pad at break*=1mm,colback=cellbackground, colframe=cellborder]
\prompt{In}{incolor}{40}{\boxspacing}
\begin{Verbatim}[commandchars=\\\{\}]
\PY{n}{lp} \PY{o}{=} \PY{p}{[}\PY{n}{P1}\PY{p}{,} \PY{n}{P2}\PY{p}{,} \PY{n}{P3}\PY{p}{,} \PY{n}{P4}\PY{p}{,} \PY{n}{P5}\PY{p}{,} \PY{n}{P6}\PY{p}{,} \PY{n}{P7}\PY{p}{]}

\PY{k}{for} \PY{n}{i} \PY{o+ow}{in} \PY{n+nb}{range}\PY{p}{(}\PY{l+m+mi}{5}\PY{p}{)}\PY{p}{:}
    \PY{k}{for} \PY{n}{j} \PY{o+ow}{in} \PY{n+nb}{range}\PY{p}{(}\PY{n}{i}\PY{o}{+}\PY{l+m+mi}{1}\PY{p}{,} \PY{l+m+mi}{6}\PY{p}{)}\PY{p}{:}
        \PY{k}{for} \PY{n}{k} \PY{o+ow}{in} \PY{n+nb}{range}\PY{p}{(}\PY{n}{j}\PY{o}{+}\PY{l+m+mi}{1}\PY{p}{,} \PY{l+m+mi}{7}\PY{p}{)}\PY{p}{:}
            \PY{k}{if} \PY{o+ow}{not} \PY{p}{(}\PY{p}{(}\PY{n}{i}\PY{o}{+}\PY{l+m+mi}{1}\PY{p}{,} \PY{n}{j}\PY{o}{+}\PY{l+m+mi}{1}\PY{p}{,} \PY{n}{k}\PY{o}{+}\PY{l+m+mi}{1}\PY{p}{)} \PY{o+ow}{in} \PY{p}{[}\PY{p}{(}\PY{l+m+mi}{1}\PY{p}{,} \PY{l+m+mi}{2}\PY{p}{,} \PY{l+m+mi}{3}\PY{p}{)}\PY{p}{,} \PY{p}{(}\PY{l+m+mi}{1}\PY{p}{,} \PY{l+m+mi}{4}\PY{p}{,} \PY{l+m+mi}{5}\PY{p}{)}\PY{p}{,} \PY{p}{(}\PY{l+m+mi}{1}\PY{p}{,} \PY{l+m+mi}{6}\PY{p}{,} \PY{l+m+mi}{7}\PY{p}{)}\PY{p}{,} \PY{p}{(}\PY{l+m+mi}{2}\PY{p}{,} \PY{l+m+mi}{4}\PY{p}{,} \PY{l+m+mi}{6}\PY{p}{)}\PY{p}{]}\PY{p}{)}\PY{p}{:}
                \PY{k}{assert}\PY{p}{(}\PY{n}{S}\PY{o}{.}\PY{n}{ideal}\PY{p}{(}\PY{n}{matrix}\PY{p}{(}\PY{p}{[}\PY{n}{lp}\PY{p}{[}\PY{n}{i}\PY{p}{]}\PY{p}{,} \PY{n}{lp}\PY{p}{[}\PY{n}{j}\PY{p}{]}\PY{p}{,} \PY{n}{lp}\PY{p}{[}\PY{n}{k}\PY{p}{]}\PY{p}{]}\PY{p}{)}\PY{o}{.}\PY{n}{det}\PY{p}{(}\PY{p}{)}\PY{p}{)}\PY{o}{.}\PY{n}{saturation}\PY{p}{(}\PY{n}{u1}\PY{o}{*}\PY{n}{u2}\PY{o}{*}\PY{n}{v1}\PY{o}{*}\PY{n}{v2}\PY{p}{)}\PY{p}{[}\PY{l+m+mi}{0}\PY{p}{]} \PY{o}{==} \PY{n}{S}\PY{o}{.}\PY{n}{ideal}\PY{p}{(}\PY{n}{S}\PY{o}{.}\PY{n}{one}\PY{p}{(}\PY{p}{)}\PY{p}{)}\PY{p}{)}
\end{Verbatim}
\end{tcolorbox}

    We have the following orthogonalities: (\(P_1 \vee P_2\) ort
\(P_2 \vee P_4\)), (\(P_1 \vee P_6\) ort \(P_2 \vee P_4\)),
(\(P_1 \vee P_4\) ort \(P_2 \vee P_4\)), (\(P_1 \vee P_6\) ort
\(P_3 \vee P_5\))

    \begin{tcolorbox}[breakable, size=fbox, boxrule=1pt, pad at break*=1mm,colback=cellbackground, colframe=cellborder]
\prompt{In}{incolor}{41}{\boxspacing}
\begin{Verbatim}[commandchars=\\\{\}]
\PY{k}{assert}\PY{p}{(}\PY{n}{scalar\PYZus{}product}\PY{p}{(}\PY{n}{wedge\PYZus{}product}\PY{p}{(}\PY{n}{P1}\PY{p}{,} \PY{n}{P2}\PY{p}{)}\PY{p}{,} \PY{n}{wedge\PYZus{}product}\PY{p}{(}\PY{n}{P2}\PY{p}{,} \PY{n}{P4}\PY{p}{)}\PY{p}{)} \PY{o}{==} \PY{l+m+mi}{0}\PY{p}{)}
\PY{k}{assert}\PY{p}{(}\PY{n}{scalar\PYZus{}product}\PY{p}{(}\PY{n}{wedge\PYZus{}product}\PY{p}{(}\PY{n}{P1}\PY{p}{,} \PY{n}{P6}\PY{p}{)}\PY{p}{,} \PY{n}{wedge\PYZus{}product}\PY{p}{(}\PY{n}{P2}\PY{p}{,} \PY{n}{P4}\PY{p}{)}\PY{p}{)} \PY{o}{==} \PY{l+m+mi}{0}\PY{p}{)}
\PY{k}{assert}\PY{p}{(}\PY{n}{scalar\PYZus{}product}\PY{p}{(}\PY{n}{wedge\PYZus{}product}\PY{p}{(}\PY{n}{P1}\PY{p}{,} \PY{n}{P4}\PY{p}{)}\PY{p}{,} \PY{n}{wedge\PYZus{}product}\PY{p}{(}\PY{n}{P2}\PY{p}{,} \PY{n}{P4}\PY{p}{)}\PY{p}{)} \PY{o}{==} \PY{l+m+mi}{0}\PY{p}{)}
\PY{k}{assert}\PY{p}{(}\PY{n}{scalar\PYZus{}product}\PY{p}{(}\PY{n}{wedge\PYZus{}product}\PY{p}{(}\PY{n}{P1}\PY{p}{,} \PY{n}{P6}\PY{p}{)}\PY{p}{,} \PY{n}{wedge\PYZus{}product}\PY{p}{(}\PY{n}{P3}\PY{p}{,} \PY{n}{P5}\PY{p}{)}\PY{p}{)} \PY{o}{==} \PY{l+m+mi}{0}\PY{p}{)}
\end{Verbatim}
\end{tcolorbox}

    In particular, we have that \(P_1 = P_2 \times P_4\) and the
configuration is \((C_5)\)

The formulas for \(P_6\) and \(P_7\) will be considered later.

This concludes case 1

    \begin{tcolorbox}[breakable, size=fbox, boxrule=1pt, pad at break*=1mm,colback=cellbackground, colframe=cellborder]
\prompt{In}{incolor}{42}{\boxspacing}
\begin{Verbatim}[commandchars=\\\{\}]
\PY{k}{assert}\PY{p}{(}\PY{n}{condition\PYZus{}matrix}\PY{p}{(}\PY{p}{[}\PY{n}{P1}\PY{p}{,} \PY{n}{P2}\PY{p}{,} \PY{n}{P3}\PY{p}{,} \PY{n}{P4}\PY{p}{,} \PY{n}{P5}\PY{p}{]}\PY{p}{,} \PY{n}{S}\PY{p}{,} \PY{n}{standard} \PY{o}{=} \PY{l+s+s2}{\PYZdq{}}\PY{l+s+s2}{all}\PY{l+s+s2}{\PYZdq{}}\PY{p}{)}\PY{o}{.}\PY{n}{rank}\PY{p}{(}\PY{p}{)} \PY{o}{==} \PY{l+m+mi}{8}\PY{p}{)}
\end{Verbatim}
\end{tcolorbox}

    \hypertarget{case-2-p_2-p_5-p_6-aligned}{%
\subsection{\texorpdfstring{Case 2: \(P_2, P_5, P_6\)
aligned}{Case 2: P\_2, P\_5, P\_6 aligned}}\label{case-2-p_2-p_5-p_6-aligned}}

    equation line \(P_2 \vee P_5\): \(r25 = yv_1 -izv_1 + 2ixv_2\)

    \begin{tcolorbox}[breakable, size=fbox, boxrule=1pt, pad at break*=1mm,colback=cellbackground, colframe=cellborder]
\prompt{In}{incolor}{43}{\boxspacing}
\begin{Verbatim}[commandchars=\\\{\}]
\PY{n}{r25} \PY{o}{=} \PY{n}{matrix}\PY{p}{(}\PY{p}{[}\PY{n}{P2}\PY{p}{,} \PY{n}{P5}\PY{p}{,} \PY{p}{(}\PY{n}{x}\PY{p}{,} \PY{n}{y}\PY{p}{,} \PY{n}{z}\PY{p}{)}\PY{p}{]}\PY{p}{)}\PY{o}{.}\PY{n}{det}\PY{p}{(}\PY{p}{)}
\PY{k}{assert}\PY{p}{(}\PY{n}{r25} \PY{o}{==} \PY{n}{y}\PY{o}{*}\PY{n}{v1} \PY{o}{+} \PY{p}{(}\PY{o}{\PYZhy{}}\PY{n}{ii}\PY{p}{)}\PY{o}{*}\PY{n}{z}\PY{o}{*}\PY{n}{v1} \PY{o}{+} \PY{p}{(}\PY{l+m+mi}{2}\PY{o}{*}\PY{n}{ii}\PY{p}{)}\PY{o}{*}\PY{n}{x}\PY{o}{*}\PY{n}{v2}\PY{p}{)}
\end{Verbatim}
\end{tcolorbox}

    construction of the point \(P_6\) intersection of the lines rt and r25:

    \begin{tcolorbox}[breakable, size=fbox, boxrule=1pt, pad at break*=1mm,colback=cellbackground, colframe=cellborder]
\prompt{In}{incolor}{44}{\boxspacing}
\begin{Verbatim}[commandchars=\\\{\}]
\PY{n}{slz6} \PY{o}{=} \PY{n}{matrix}\PY{p}{(}
    \PY{p}{[}
        \PY{p}{[}\PY{n}{r25}\PY{o}{.}\PY{n}{coefficient}\PY{p}{(}\PY{n}{xx}\PY{p}{)} \PY{k}{for} \PY{n}{xx} \PY{o+ow}{in} \PY{p}{(}\PY{n}{x}\PY{p}{,} \PY{n}{y}\PY{p}{,} \PY{n}{z}\PY{p}{)}\PY{p}{]}\PY{p}{,} 
        \PY{p}{[}\PY{n}{rt}\PY{o}{.}\PY{n}{coefficient}\PY{p}{(}\PY{n}{xx}\PY{p}{)} \PY{k}{for} \PY{n}{xx} \PY{o+ow}{in} \PY{p}{(}\PY{n}{x}\PY{p}{,} \PY{n}{y}\PY{p}{,} \PY{n}{z}\PY{p}{)}\PY{p}{]}
    \PY{p}{]}
\PY{p}{)}\PY{o}{.}\PY{n}{minors}\PY{p}{(}\PY{l+m+mi}{2}\PY{p}{)}

\PY{n}{P6} \PY{o}{=} \PY{n}{vector}\PY{p}{(}\PY{n}{S}\PY{p}{,} \PY{p}{(}\PY{n}{slz6}\PY{p}{[}\PY{l+m+mi}{2}\PY{p}{]}\PY{p}{,} \PY{o}{\PYZhy{}}\PY{n}{slz6}\PY{p}{[}\PY{l+m+mi}{1}\PY{p}{]}\PY{p}{,} \PY{n}{slz6}\PY{p}{[}\PY{l+m+mi}{0}\PY{p}{]}\PY{p}{)}\PY{p}{)}
\end{Verbatim}
\end{tcolorbox}

    check that \(P_6\) is \(rt \cap r25\):

    \begin{tcolorbox}[breakable, size=fbox, boxrule=1pt, pad at break*=1mm,colback=cellbackground, colframe=cellborder]
\prompt{In}{incolor}{45}{\boxspacing}
\begin{Verbatim}[commandchars=\\\{\}]
\PY{k}{assert}\PY{p}{(}\PY{n}{r25}\PY{o}{.}\PY{n}{subs}\PY{p}{(}\PY{n}{substitution}\PY{p}{(}\PY{n}{P6}\PY{p}{)}\PY{p}{)}\PY{o}{.}\PY{n}{is\PYZus{}zero}\PY{p}{(}\PY{p}{)} \PY{o+ow}{and}  \PY{n}{rt}\PY{o}{.}\PY{n}{subs}\PY{p}{(}\PY{n}{substitution}\PY{p}{(}\PY{n}{P6}\PY{p}{)}\PY{p}{)}\PY{o}{.}\PY{n}{is\PYZus{}zero}\PY{p}{(}\PY{p}{)}\PY{p}{)}
\end{Verbatim}
\end{tcolorbox}

    \(P_6\) always exists:

    \begin{tcolorbox}[breakable, size=fbox, boxrule=1pt, pad at break*=1mm,colback=cellbackground, colframe=cellborder]
\prompt{In}{incolor}{46}{\boxspacing}
\begin{Verbatim}[commandchars=\\\{\}]
\PY{k}{assert}\PY{p}{(}\PY{n}{S}\PY{o}{.}\PY{n}{ideal}\PY{p}{(}\PY{n+nb}{list}\PY{p}{(}\PY{n}{P6}\PY{p}{)}\PY{p}{)}\PY{o}{.}\PY{n}{saturation}\PY{p}{(}\PY{n}{u1}\PY{o}{*}\PY{n}{u2}\PY{o}{*}\PY{n}{v1}\PY{o}{*}\PY{n}{v2}\PY{p}{)}\PY{p}{[}\PY{l+m+mi}{0}\PY{p}{]} \PY{o}{==} \PY{n}{S}\PY{o}{.}\PY{n}{ideal}\PY{p}{(}\PY{n}{S}\PY{o}{.}\PY{n}{one}\PY{p}{(}\PY{p}{)}\PY{p}{)}\PY{p}{)}
\end{Verbatim}
\end{tcolorbox}

    if \(P_6\) is an eigenpoint, it must be a point of the conic. This
condiiton gives that\\
\(u_1v_1l_2 + u_2v_2l_2 -3/4il_1 = 0\):

    \begin{tcolorbox}[breakable, size=fbox, boxrule=1pt, pad at break*=1mm,colback=cellbackground, colframe=cellborder]
\prompt{In}{incolor}{47}{\boxspacing}
\begin{Verbatim}[commandchars=\\\{\}]
\PY{k}{assert}\PY{p}{(}
    \PY{n}{qn}\PY{o}{.}\PY{n}{subs}\PY{p}{(}\PY{n}{substitution}\PY{p}{(}\PY{n}{P6}\PY{p}{)}\PY{p}{)} 
    \PY{o}{==} \PY{p}{(}\PY{p}{(}\PY{o}{\PYZhy{}}\PY{l+m+mi}{64}\PY{o}{*}\PY{n}{ii}\PY{p}{)}\PY{p}{)} \PY{o}{*} \PY{n}{v1} \PY{o}{*} \PY{n}{u2} \PY{o}{*} \PY{n}{u1} \PY{o}{*} \PY{n}{v2}\PY{o}{\PYZca{}}\PY{l+m+mi}{3} \PY{o}{*} \PY{p}{(}\PY{n}{u1}\PY{o}{*}\PY{n}{v1}\PY{o}{*}\PY{n}{l2} \PY{o}{+} \PY{n}{u2}\PY{o}{*}\PY{n}{v2}\PY{o}{*}\PY{n}{l2} \PY{o}{+} \PY{p}{(}\PY{o}{\PYZhy{}}\PY{l+m+mi}{3}\PY{o}{/}\PY{l+m+mi}{4}\PY{o}{*}\PY{n}{ii}\PY{p}{)}\PY{o}{*}\PY{n}{l1}\PY{p}{)}
\PY{p}{)}
\end{Verbatim}
\end{tcolorbox}

    Hence the values for \(l_1\) and \(l_2\) is:
\(l_1: u_1v_1+u_2v_2: l_2: 3/4i\). This gives a cubic cb1

    \begin{tcolorbox}[breakable, size=fbox, boxrule=1pt, pad at break*=1mm,colback=cellbackground, colframe=cellborder]
\prompt{In}{incolor}{48}{\boxspacing}
\begin{Verbatim}[commandchars=\\\{\}]
\PY{n}{cb1} \PY{o}{=} \PY{n}{S}\PY{p}{(}\PY{n}{cb}\PY{o}{.}\PY{n}{subs}\PY{p}{(}\PY{p}{\PYZob{}}\PY{n}{l1}\PY{p}{:} \PY{n}{u1}\PY{o}{*}\PY{n}{v1}\PY{o}{+}\PY{n}{u2}\PY{o}{*}\PY{n}{v2}\PY{p}{,} \PY{n}{l2}\PY{p}{:} \PY{l+m+mi}{3}\PY{o}{/}\PY{l+m+mi}{4}\PY{o}{*}\PY{n}{ii}\PY{p}{\PYZcb{}}\PY{p}{)}\PY{p}{)}
\end{Verbatim}
\end{tcolorbox}

    If we assume
\((2u_1v_1+3u_2v_2)(u_1v_1+u_2v_2)(u_1v_1+3u_2v_2) \not = 0\), then cb1
is smooth:

    \begin{tcolorbox}[breakable, size=fbox, boxrule=1pt, pad at break*=1mm,colback=cellbackground, colframe=cellborder]
\prompt{In}{incolor}{49}{\boxspacing}
\begin{Verbatim}[commandchars=\\\{\}]
\PY{n}{irr} \PY{o}{=} \PY{n}{S}\PY{o}{.}\PY{n}{ideal}\PY{p}{(}\PY{n+nb}{list}\PY{p}{(}\PY{n}{gdn}\PY{p}{(}\PY{n}{cb1}\PY{p}{)}\PY{p}{)}\PY{p}{)}\PY{o}{.}\PY{n}{saturation}\PY{p}{(}\PY{n}{u1}\PY{o}{*}\PY{n}{u2}\PY{o}{*}\PY{n}{v1}\PY{o}{*}\PY{n}{v2}\PY{p}{)}\PY{p}{[}\PY{l+m+mi}{0}\PY{p}{]}
\PY{n}{irr} \PY{o}{=} \PY{n}{irr}\PY{o}{.}\PY{n}{saturation}\PY{p}{(}\PY{l+m+mi}{2}\PY{o}{*}\PY{n}{u1}\PY{o}{*}\PY{n}{v1}\PY{o}{+}\PY{l+m+mi}{3}\PY{o}{*}\PY{n}{u2}\PY{o}{*}\PY{n}{v2}\PY{p}{)}\PY{p}{[}\PY{l+m+mi}{0}\PY{p}{]}
\PY{n}{irr} \PY{o}{=} \PY{n}{irr}\PY{o}{.}\PY{n}{saturation}\PY{p}{(}\PY{n}{u1}\PY{o}{*}\PY{n}{v1}\PY{o}{+}\PY{n}{u2}\PY{o}{*}\PY{n}{v2}\PY{p}{)}\PY{p}{[}\PY{l+m+mi}{0}\PY{p}{]}
\PY{n}{irr} \PY{o}{=} \PY{n}{irr}\PY{o}{.}\PY{n}{saturation}\PY{p}{(}\PY{n}{u1}\PY{o}{*}\PY{n}{v1}\PY{o}{+}\PY{l+m+mi}{3}\PY{o}{*}\PY{n}{u2}\PY{o}{*}\PY{n}{v2}\PY{p}{)}\PY{p}{[}\PY{l+m+mi}{0}\PY{p}{]}
\PY{k}{assert}\PY{p}{(}\PY{n}{irr}\PY{o}{.}\PY{n}{radical}\PY{p}{(}\PY{p}{)} \PY{o}{==} \PY{n}{S}\PY{o}{.}\PY{n}{ideal}\PY{p}{(}\PY{n}{x}\PY{p}{,} \PY{n}{y}\PY{p}{,} \PY{n}{z}\PY{p}{)}\PY{p}{)}
\end{Verbatim}
\end{tcolorbox}

    In particular we see that the cubic cb1 is not necessarily smooth.

Now we construct the point \(P_7\). We compute the ideal e\_cb1 of the
iegenpoints of cb1 and we saturate it w.r.t. the eigenpoints
\(P_1, \dots, P_6\):

    \begin{tcolorbox}[breakable, size=fbox, boxrule=1pt, pad at break*=1mm,colback=cellbackground, colframe=cellborder]
\prompt{In}{incolor}{50}{\boxspacing}
\begin{Verbatim}[commandchars=\\\{\}]
\PY{n}{e\PYZus{}cb1} \PY{o}{=} \PY{n}{S}\PY{o}{.}\PY{n}{ideal}\PY{p}{(}\PY{n+nb}{list}\PY{p}{(}\PY{n}{eig}\PY{p}{(}\PY{n}{cb1}\PY{p}{)}\PY{p}{)}\PY{p}{)}\PY{o}{.}\PY{n}{saturation}\PY{p}{(}\PY{n}{u1}\PY{o}{*}\PY{n}{u2}\PY{o}{*}\PY{n}{v1}\PY{o}{*}\PY{n}{v2}\PY{p}{)}\PY{p}{[}\PY{l+m+mi}{0}\PY{p}{]}

\PY{k}{for} \PY{n}{pp} \PY{o+ow}{in} \PY{p}{[}\PY{n}{P1}\PY{p}{,} \PY{n}{P2}\PY{p}{,} \PY{n}{P3}\PY{p}{,} \PY{n}{P4}\PY{p}{,} \PY{n}{P5}\PY{p}{,} \PY{n}{P6}\PY{p}{]}\PY{p}{:}
    \PY{n}{e\PYZus{}cb1} \PY{o}{=} \PY{n}{e\PYZus{}cb1}\PY{o}{.}\PY{n}{saturation}\PY{p}{(}\PY{n}{S}\PY{o}{.}\PY{n}{ideal}\PY{p}{(}\PY{n}{matrix}\PY{p}{(}\PY{p}{[}\PY{n}{pp}\PY{p}{,} \PY{p}{(}\PY{n}{x}\PY{p}{,} \PY{n}{y}\PY{p}{,} \PY{n}{z}\PY{p}{)}\PY{p}{]}\PY{p}{)}\PY{o}{.}\PY{n}{minors}\PY{p}{(}\PY{l+m+mi}{2}\PY{p}{)}\PY{p}{)}\PY{p}{)}\PY{p}{[}\PY{l+m+mi}{0}\PY{p}{]}
\end{Verbatim}
\end{tcolorbox}

    Now the ideal e-cb1 is the ideal of the point \(P_7\). It is given by
three polynomials, but it is generated by the first two polynoials:

    \begin{tcolorbox}[breakable, size=fbox, boxrule=1pt, pad at break*=1mm,colback=cellbackground, colframe=cellborder]
\prompt{In}{incolor}{51}{\boxspacing}
\begin{Verbatim}[commandchars=\\\{\}]
\PY{k}{assert}\PY{p}{(}\PY{n+nb}{len}\PY{p}{(}\PY{n}{e\PYZus{}cb1}\PY{o}{.}\PY{n}{gens}\PY{p}{(}\PY{p}{)}\PY{p}{)} \PY{o}{==} \PY{l+m+mi}{3}\PY{p}{)}
\PY{n}{pl1}\PY{p}{,} \PY{n}{pl2} \PY{o}{=} \PY{n+nb}{tuple}\PY{p}{(}\PY{n}{e\PYZus{}cb1}\PY{o}{.}\PY{n}{gens}\PY{p}{(}\PY{p}{)}\PY{p}{[}\PY{p}{:}\PY{l+m+mi}{2}\PY{p}{]}\PY{p}{)}
\PY{k}{assert}\PY{p}{(}\PY{n}{e\PYZus{}cb1} \PY{o}{==} \PY{n}{S}\PY{o}{.}\PY{n}{ideal}\PY{p}{(}\PY{n}{pl1}\PY{p}{,} \PY{n}{pl2}\PY{p}{)}\PY{p}{)}
\end{Verbatim}
\end{tcolorbox}

    We use the two polynomials pl1 and pl2, which are linear in \(x, y, z\),
to construct \(P_7\):

    \begin{tcolorbox}[breakable, size=fbox, boxrule=1pt, pad at break*=1mm,colback=cellbackground, colframe=cellborder]
\prompt{In}{incolor}{52}{\boxspacing}
\begin{Verbatim}[commandchars=\\\{\}]
\PY{n}{slz} \PY{o}{=} \PY{n}{matrix}\PY{p}{(}
    \PY{p}{[}
        \PY{p}{[}\PY{n}{pl1}\PY{o}{.}\PY{n}{coefficient}\PY{p}{(}\PY{n}{xx}\PY{p}{)} \PY{k}{for} \PY{n}{xx} \PY{o+ow}{in} \PY{p}{(}\PY{n}{x}\PY{p}{,} \PY{n}{y}\PY{p}{,} \PY{n}{z}\PY{p}{)}\PY{p}{]}\PY{p}{,}
        \PY{p}{[}\PY{n}{pl2}\PY{o}{.}\PY{n}{coefficient}\PY{p}{(}\PY{n}{xx}\PY{p}{)} \PY{k}{for} \PY{n}{xx} \PY{o+ow}{in} \PY{p}{(}\PY{n}{x}\PY{p}{,} \PY{n}{y}\PY{p}{,} \PY{n}{z}\PY{p}{)}\PY{p}{]}
    \PY{p}{]}
\PY{p}{)}\PY{o}{.}\PY{n}{minors}\PY{p}{(}\PY{l+m+mi}{2}\PY{p}{)}

\PY{n}{P7} \PY{o}{=} \PY{n}{vector}\PY{p}{(}\PY{n}{S}\PY{p}{,} \PY{p}{[}\PY{n}{slz}\PY{p}{[}\PY{l+m+mi}{2}\PY{p}{]}\PY{p}{,} \PY{o}{\PYZhy{}}\PY{n}{slz}\PY{p}{[}\PY{l+m+mi}{1}\PY{p}{]}\PY{p}{,} \PY{n}{slz}\PY{p}{[}\PY{l+m+mi}{0}\PY{p}{]}\PY{p}{]}\PY{p}{)}
\end{Verbatim}
\end{tcolorbox}

    \(P_7\) is an eigenpoint of cb1 and it always exists:

    \begin{tcolorbox}[breakable, size=fbox, boxrule=1pt, pad at break*=1mm,colback=cellbackground, colframe=cellborder]
\prompt{In}{incolor}{53}{\boxspacing}
\begin{Verbatim}[commandchars=\\\{\}]
\PY{k}{assert}\PY{p}{(}\PY{n}{eig}\PY{p}{(}\PY{n}{cb1}\PY{p}{)}\PY{o}{.}\PY{n}{subs}\PY{p}{(}\PY{n}{substitution}\PY{p}{(}\PY{n}{P7}\PY{p}{)}\PY{p}{)}\PY{o}{==}\PY{n}{vector}\PY{p}{(}\PY{n}{S}\PY{p}{,} \PY{p}{(}\PY{l+m+mi}{0}\PY{p}{,} \PY{l+m+mi}{0}\PY{p}{,} \PY{l+m+mi}{0}\PY{p}{)}\PY{p}{)}\PY{p}{)}

\PY{k}{assert}\PY{p}{(}\PY{n}{S}\PY{o}{.}\PY{n}{ideal}\PY{p}{(}\PY{n+nb}{list}\PY{p}{(}\PY{n}{P7}\PY{p}{)}\PY{p}{)}\PY{o}{.}\PY{n}{saturation}\PY{p}{(}\PY{n}{u1}\PY{o}{*}\PY{n}{u2}\PY{o}{*}\PY{n}{v1}\PY{o}{*}\PY{n}{v2}\PY{p}{)}\PY{p}{[}\PY{l+m+mi}{0}\PY{p}{]} \PY{o}{==} \PY{n}{S}\PY{o}{.}\PY{n}{ideal}\PY{p}{(}\PY{n}{S}\PY{o}{.}\PY{n}{one}\PY{p}{(}\PY{p}{)}\PY{p}{)}\PY{p}{)}
\end{Verbatim}
\end{tcolorbox}

    The seven eigenpoints have the following alignments: \[
[(1, 2, 3), (1, 4, 5), (1, 6, 7), (2, 5, 6), (3, 4, 6), (3, 5, 7)]
\] Hence we get the configuration \((C_8)\)

    \begin{tcolorbox}[breakable, size=fbox, boxrule=1pt, pad at break*=1mm,colback=cellbackground, colframe=cellborder]
\prompt{In}{incolor}{54}{\boxspacing}
\begin{Verbatim}[commandchars=\\\{\}]
\PY{k}{assert}\PY{p}{(}\PY{n}{alignments}\PY{p}{(}\PY{p}{[}\PY{n}{P1}\PY{p}{,} \PY{n}{P2}\PY{p}{,} \PY{n}{P3}\PY{p}{,} \PY{n}{P4}\PY{p}{,} \PY{n}{P5}\PY{p}{,} \PY{n}{P6}\PY{p}{,} \PY{n}{P7}\PY{p}{]}\PY{p}{)} \PY{o}{==} \PY{p}{[}\PY{p}{(}\PY{l+m+mi}{1}\PY{p}{,} \PY{l+m+mi}{2}\PY{p}{,} \PY{l+m+mi}{3}\PY{p}{)}\PY{p}{,} \PY{p}{(}\PY{l+m+mi}{1}\PY{p}{,} \PY{l+m+mi}{4}\PY{p}{,} \PY{l+m+mi}{5}\PY{p}{)}\PY{p}{,} \PY{p}{(}\PY{l+m+mi}{1}\PY{p}{,} \PY{l+m+mi}{6}\PY{p}{,} \PY{l+m+mi}{7}\PY{p}{)}\PY{p}{,} \PY{p}{(}\PY{l+m+mi}{2}\PY{p}{,} \PY{l+m+mi}{5}\PY{p}{,} \PY{l+m+mi}{6}\PY{p}{)}\PY{p}{,} \PY{p}{(}\PY{l+m+mi}{3}\PY{p}{,} \PY{l+m+mi}{4}\PY{p}{,} \PY{l+m+mi}{6}\PY{p}{)}\PY{p}{,} \PY{p}{(}\PY{l+m+mi}{3}\PY{p}{,} \PY{l+m+mi}{5}\PY{p}{,} \PY{l+m+mi}{7}\PY{p}{)}\PY{p}{]}\PY{p}{)}
\end{Verbatim}
\end{tcolorbox}

    We also have the relation \(P_3 = Q_3\), where \(Q_3\) is given by the
formula above (is the orthocenter):

    \begin{tcolorbox}[breakable, size=fbox, boxrule=1pt, pad at break*=1mm,colback=cellbackground, colframe=cellborder]
\prompt{In}{incolor}{55}{\boxspacing}
\begin{Verbatim}[commandchars=\\\{\}]
\PY{n}{Q3} \PY{o}{=} \PY{p}{(}
    \PY{n}{wedge\PYZus{}product}\PY{p}{(}\PY{n}{P1}\PY{p}{,} \PY{n}{P5}\PY{p}{)}\PY{o}{*}\PY{n}{scalar\PYZus{}product}\PY{p}{(}\PY{n}{P1}\PY{p}{,} \PY{n}{P6}\PY{p}{)}\PY{o}{*}\PY{n}{scalar\PYZus{}product}\PY{p}{(}\PY{n}{P5}\PY{p}{,} \PY{n}{P6}\PY{p}{)}
    \PY{o}{\PYZhy{}} \PY{n}{scalar\PYZus{}product}\PY{p}{(}\PY{n}{P1}\PY{p}{,} \PY{n}{P5}\PY{p}{)}\PY{o}{*}\PY{n}{wedge\PYZus{}product}\PY{p}{(}\PY{n}{P1}\PY{p}{,} \PY{n}{P6}\PY{p}{)}\PY{o}{*}\PY{n}{scalar\PYZus{}product}\PY{p}{(}\PY{n}{P5}\PY{p}{,} \PY{n}{P6}\PY{p}{)}
    \PY{o}{+} \PY{n}{scalar\PYZus{}product}\PY{p}{(}\PY{n}{P1}\PY{p}{,} \PY{n}{P5}\PY{p}{)}\PY{o}{*}\PY{n}{scalar\PYZus{}product}\PY{p}{(}\PY{n}{P1}\PY{p}{,} \PY{n}{P6}\PY{p}{)}\PY{o}{*}\PY{n}{wedge\PYZus{}product}\PY{p}{(}\PY{n}{P5}\PY{p}{,} \PY{n}{P6}\PY{p}{)}
\PY{p}{)}

\PY{k}{assert}\PY{p}{(}\PY{n}{Q3} \PY{o}{!=} \PY{n}{vector}\PY{p}{(}\PY{n}{S}\PY{p}{,} \PY{p}{(}\PY{l+m+mi}{0}\PY{p}{,} \PY{l+m+mi}{0}\PY{p}{,} \PY{l+m+mi}{0}\PY{p}{)}\PY{p}{)}\PY{p}{)}
\PY{k}{assert}\PY{p}{(}\PY{n}{matrix}\PY{p}{(}\PY{p}{[}\PY{n}{P3}\PY{p}{,} \PY{n}{Q3}\PY{p}{]}\PY{p}{)}\PY{o}{.}\PY{n}{minors}\PY{p}{(}\PY{l+m+mi}{2}\PY{p}{)} \PY{o}{==} \PY{p}{[}\PY{l+m+mi}{0}\PY{p}{,} \PY{l+m+mi}{0}\PY{p}{,} \PY{l+m+mi}{0}\PY{p}{]}\PY{p}{)}
\end{Verbatim}
\end{tcolorbox}

    and we have some orthogonalities of the lines joining the eigenpoints:

    \begin{tcolorbox}[breakable, size=fbox, boxrule=1pt, pad at break*=1mm,colback=cellbackground, colframe=cellborder]
\prompt{In}{incolor}{56}{\boxspacing}
\begin{Verbatim}[commandchars=\\\{\}]
\PY{k}{assert}\PY{p}{(}\PY{n}{scalar\PYZus{}product}\PY{p}{(}\PY{n}{wedge\PYZus{}product}\PY{p}{(}\PY{n}{P1}\PY{p}{,} \PY{n}{P4}\PY{p}{)}\PY{p}{,} \PY{n}{wedge\PYZus{}product}\PY{p}{(}\PY{n}{P3}\PY{p}{,} \PY{n}{P4}\PY{p}{)}\PY{p}{)}\PY{o}{==}\PY{l+m+mi}{0}\PY{p}{)}
\PY{k}{assert}\PY{p}{(}\PY{n}{scalar\PYZus{}product}\PY{p}{(}\PY{n}{wedge\PYZus{}product}\PY{p}{(}\PY{n}{P1}\PY{p}{,} \PY{n}{P2}\PY{p}{)}\PY{p}{,} \PY{n}{wedge\PYZus{}product}\PY{p}{(}\PY{n}{P2}\PY{p}{,} \PY{n}{P5}\PY{p}{)}\PY{p}{)}\PY{o}{==}\PY{l+m+mi}{0}\PY{p}{)}
\PY{k}{assert}\PY{p}{(}\PY{n}{scalar\PYZus{}product}\PY{p}{(}\PY{n}{wedge\PYZus{}product}\PY{p}{(}\PY{n}{P3}\PY{p}{,} \PY{n}{P5}\PY{p}{)}\PY{p}{,} \PY{n}{wedge\PYZus{}product}\PY{p}{(}\PY{n}{P1}\PY{p}{,} \PY{n}{P6}\PY{p}{)}\PY{p}{)}\PY{o}{==}\PY{l+m+mi}{0}\PY{p}{)}
\end{Verbatim}
\end{tcolorbox}

    \hypertarget{case-3-p_3-p_5-p_7-aligned}{%
\subsection{\texorpdfstring{Case 3: \(P_3, P_5, P_7\)
aligned}{Case 3: P\_3, P\_5, P\_7 aligned}}\label{case-3-p_3-p_5-p_7-aligned}}

    Here we call P7 the point of rt which is aligned with \(P_3\) and
\(P_5\)

equation line r35 = \(P_3 \vee P_5\). It turns out to be:

\(y u_2 v_1 + (-i) z u_2 v_1 - y u_1 v_2 + (-i) z u_1 v_2 + (2 i) x u_2 v_2\)

    \begin{tcolorbox}[breakable, size=fbox, boxrule=1pt, pad at break*=1mm,colback=cellbackground, colframe=cellborder]
\prompt{In}{incolor}{57}{\boxspacing}
\begin{Verbatim}[commandchars=\\\{\}]
\PY{n}{r35} \PY{o}{=} \PY{n}{matrix}\PY{p}{(}\PY{p}{[}\PY{n}{P3}\PY{p}{,} \PY{n}{P5}\PY{p}{,} \PY{p}{(}\PY{n}{x}\PY{p}{,} \PY{n}{y}\PY{p}{,} \PY{n}{z}\PY{p}{)}\PY{p}{]}\PY{p}{)}\PY{o}{.}\PY{n}{det}\PY{p}{(}\PY{p}{)}
\PY{k}{assert}\PY{p}{(}\PY{n}{r35} \PY{o}{==} \PY{n}{y}\PY{o}{*}\PY{n}{u2}\PY{o}{*}\PY{n}{v1} \PY{o}{+} \PY{p}{(}\PY{o}{\PYZhy{}}\PY{n}{ii}\PY{p}{)}\PY{o}{*}\PY{n}{z}\PY{o}{*}\PY{n}{u2}\PY{o}{*}\PY{n}{v1} \PY{o}{\PYZhy{}} \PY{n}{y}\PY{o}{*}\PY{n}{u1}\PY{o}{*}\PY{n}{v2} \PY{o}{+} \PY{p}{(}\PY{o}{\PYZhy{}}\PY{n}{ii}\PY{p}{)}\PY{o}{*}\PY{n}{z}\PY{o}{*}\PY{n}{u1}\PY{o}{*}\PY{n}{v2} \PY{o}{+} \PY{p}{(}\PY{l+m+mi}{2}\PY{o}{*}\PY{n}{ii}\PY{p}{)}\PY{o}{*}\PY{n}{x}\PY{o}{*}\PY{n}{u2}\PY{o}{*}\PY{n}{v2}\PY{p}{)}
\end{Verbatim}
\end{tcolorbox}

    Construction of the point \(P_7\) intersection of the lines rt and r35:

    \begin{tcolorbox}[breakable, size=fbox, boxrule=1pt, pad at break*=1mm,colback=cellbackground, colframe=cellborder]
\prompt{In}{incolor}{58}{\boxspacing}
\begin{Verbatim}[commandchars=\\\{\}]
\PY{n}{slz7} \PY{o}{=} \PY{n}{matrix}\PY{p}{(}
    \PY{p}{[}
        \PY{p}{[}\PY{n}{r35}\PY{o}{.}\PY{n}{coefficient}\PY{p}{(}\PY{n}{xx}\PY{p}{)} \PY{k}{for} \PY{n}{xx} \PY{o+ow}{in} \PY{p}{(}\PY{n}{x}\PY{p}{,} \PY{n}{y}\PY{p}{,} \PY{n}{z}\PY{p}{)}\PY{p}{]}\PY{p}{,} 
        \PY{p}{[}\PY{n}{rt}\PY{o}{.}\PY{n}{coefficient}\PY{p}{(}\PY{n}{xx}\PY{p}{)} \PY{k}{for} \PY{n}{xx} \PY{o+ow}{in} \PY{p}{(}\PY{n}{x}\PY{p}{,} \PY{n}{y}\PY{p}{,} \PY{n}{z}\PY{p}{)}\PY{p}{]}
    \PY{p}{]}
\PY{p}{)}\PY{o}{.}\PY{n}{minors}\PY{p}{(}\PY{l+m+mi}{2}\PY{p}{)}
\PY{n}{P7} \PY{o}{=} \PY{n}{vector}\PY{p}{(}\PY{n}{S}\PY{p}{,} \PY{p}{(}\PY{n}{slz7}\PY{p}{[}\PY{l+m+mi}{2}\PY{p}{]}\PY{p}{,} \PY{o}{\PYZhy{}}\PY{n}{slz7}\PY{p}{[}\PY{l+m+mi}{1}\PY{p}{]}\PY{p}{,} \PY{n}{slz7}\PY{p}{[}\PY{l+m+mi}{0}\PY{p}{]}\PY{p}{)}\PY{p}{)}

\PY{k}{assert}\PY{p}{(}\PY{n}{r35}\PY{o}{.}\PY{n}{subs}\PY{p}{(}\PY{n}{substitution}\PY{p}{(}\PY{n}{P7}\PY{p}{)}\PY{p}{)}\PY{o}{.}\PY{n}{is\PYZus{}zero}\PY{p}{(}\PY{p}{)} \PY{o+ow}{and}  \PY{n}{rt}\PY{o}{.}\PY{n}{subs}\PY{p}{(}\PY{n}{substitution}\PY{p}{(}\PY{n}{P7}\PY{p}{)}\PY{p}{)}\PY{o}{.}\PY{n}{is\PYZus{}zero}\PY{p}{(}\PY{p}{)}\PY{p}{)}
\end{Verbatim}
\end{tcolorbox}

    \(P_7\) always exists:

    \begin{tcolorbox}[breakable, size=fbox, boxrule=1pt, pad at break*=1mm,colback=cellbackground, colframe=cellborder]
\prompt{In}{incolor}{59}{\boxspacing}
\begin{Verbatim}[commandchars=\\\{\}]
\PY{k}{assert}\PY{p}{(}\PY{n}{S}\PY{o}{.}\PY{n}{ideal}\PY{p}{(}\PY{n+nb}{list}\PY{p}{(}\PY{n}{P7}\PY{p}{)}\PY{p}{)}\PY{o}{.}\PY{n}{saturation}\PY{p}{(}\PY{n}{u1}\PY{o}{*}\PY{n}{u2}\PY{o}{*}\PY{n}{v1}\PY{o}{*}\PY{n}{v2}\PY{p}{)}\PY{p}{[}\PY{l+m+mi}{0}\PY{p}{]} \PY{o}{==} \PY{n}{S}\PY{o}{.}\PY{n}{ideal}\PY{p}{(}\PY{n}{S}\PY{o}{.}\PY{n}{one}\PY{p}{(}\PY{p}{)}\PY{p}{)}\PY{p}{)}
\end{Verbatim}
\end{tcolorbox}

    If \(P_7\) is an eigenpoint, it must be a point of the conic. Hence

u1\emph{v1}l2 + u2\emph{v2}l2 + (-3/4\emph{ii)}l1

indeed:

    \begin{tcolorbox}[breakable, size=fbox, boxrule=1pt, pad at break*=1mm,colback=cellbackground, colframe=cellborder]
\prompt{In}{incolor}{60}{\boxspacing}
\begin{Verbatim}[commandchars=\\\{\}]
\PY{k}{assert}\PY{p}{(}
    \PY{n}{qn}\PY{o}{.}\PY{n}{subs}\PY{p}{(}\PY{n}{substitution}\PY{p}{(}\PY{n}{P7}\PY{p}{)}\PY{p}{)} 
    \PY{o}{==} \PY{p}{(}\PY{p}{(}\PY{o}{\PYZhy{}}\PY{l+m+mi}{64}\PY{o}{*}\PY{n}{ii}\PY{p}{)}\PY{p}{)} \PY{o}{*} \PY{n}{v1} \PY{o}{*} \PY{n}{u1} \PY{o}{*} \PY{n}{v2}\PY{o}{\PYZca{}}\PY{l+m+mi}{3} \PY{o}{*} \PY{n}{u2}\PY{o}{\PYZca{}}\PY{l+m+mi}{3} \PY{o}{*} \PY{p}{(}\PY{n}{u1}\PY{o}{*}\PY{n}{v1}\PY{o}{*}\PY{n}{l2} \PY{o}{+} \PY{n}{u2}\PY{o}{*}\PY{n}{v2}\PY{o}{*}\PY{n}{l2} \PY{o}{+} \PY{p}{(}\PY{o}{\PYZhy{}}\PY{l+m+mi}{3}\PY{o}{/}\PY{l+m+mi}{4}\PY{o}{*}\PY{n}{ii}\PY{p}{)}\PY{o}{*}\PY{n}{l1}\PY{p}{)}
\PY{p}{)}
\end{Verbatim}
\end{tcolorbox}

    Hence the substitution into cb is: \{l1: u1\emph{v1+u2}v2, l2: 3/4*ii\}

    \begin{tcolorbox}[breakable, size=fbox, boxrule=1pt, pad at break*=1mm,colback=cellbackground, colframe=cellborder]
\prompt{In}{incolor}{61}{\boxspacing}
\begin{Verbatim}[commandchars=\\\{\}]
\PY{n}{cb1bis} \PY{o}{=} \PY{n}{S}\PY{p}{(}\PY{n}{cb}\PY{o}{.}\PY{n}{subs}\PY{p}{(}\PY{p}{\PYZob{}}\PY{n}{l1}\PY{p}{:} \PY{n}{u1}\PY{o}{*}\PY{n}{v1}\PY{o}{+}\PY{n}{u2}\PY{o}{*}\PY{n}{v2}\PY{p}{,} \PY{n}{l2}\PY{p}{:} \PY{l+m+mi}{3}\PY{o}{/}\PY{l+m+mi}{4}\PY{o}{*}\PY{n}{ii}\PY{p}{\PYZcb{}}\PY{p}{)}\PY{p}{)}
\end{Verbatim}
\end{tcolorbox}

    We obtain that the new cubic is the same the cubic cb1 of the previous
case, so this case is already studied above.

    \begin{tcolorbox}[breakable, size=fbox, boxrule=1pt, pad at break*=1mm,colback=cellbackground, colframe=cellborder]
\prompt{In}{incolor}{62}{\boxspacing}
\begin{Verbatim}[commandchars=\\\{\}]
\PY{k}{assert}\PY{p}{(}\PY{n}{cb1bis} \PY{o}{==} \PY{n}{cb1}\PY{p}{)}
\end{Verbatim}
\end{tcolorbox}

    In conclusion, we have two possible configurations: \((C_5)\) and
\((C_8)\) (and a case of two lines of eigenpoints) the line
\(P_3 \vee P_5\) and the line \(P_1 \vee P_6 \vee P_7\) are always
orthogonal.

    \hypertarget{final-computations-formulas-for-p_6-and-p_7}{%
\subsection{\texorpdfstring{Final computations: formulas for \(P_6\) and
\(P_7\)}{Final computations: formulas for P\_6 and P\_7}}\label{final-computations-formulas-for-p_6-and-p_7}}

We want to find the formulas for \(P_6\) and \(P_7\) given above (in
Case 1: \(P_2, P_6, P_6\) collinear)

This part, if doLongComputations is true, requires about 30 minutes.

    We define 5 generic points such that: \(P_2\) and \(P_4\) are on the
isotropic conic, \(P_1 \vee P_2\) and \(P_1 \vee P_4\) are tangent to
the isotropic conic and \(P_3\) is a point on \(P_1 \vee P_2\) and
\(P_5\) is a point on \(P_1 \vee P_4\) (parameteres: u1, u2, l1, l2, v1,
v2, m1, m2):

    \begin{tcolorbox}[breakable, size=fbox, boxrule=1pt, pad at break*=1mm,colback=cellbackground, colframe=cellborder]
\prompt{In}{incolor}{63}{\boxspacing}
\begin{Verbatim}[commandchars=\\\{\}]
\PY{n}{P1} \PY{o}{=} \PY{n}{vector}\PY{p}{(}\PY{n}{S}\PY{p}{,} \PY{p}{(}\PY{l+m+mi}{2}\PY{o}{*}\PY{n}{u1}\PY{o}{*}\PY{n}{v1} \PY{o}{+} \PY{l+m+mi}{2}\PY{o}{*}\PY{n}{u2}\PY{o}{*}\PY{n}{v2}\PY{p}{,} \PY{p}{(}\PY{l+m+mi}{2}\PY{o}{*}\PY{n}{ii}\PY{p}{)}\PY{o}{*}\PY{n}{u1}\PY{o}{*}\PY{n}{v1} \PY{o}{+} \PY{p}{(}\PY{o}{\PYZhy{}}\PY{l+m+mi}{2}\PY{o}{*}\PY{n}{ii}\PY{p}{)}\PY{o}{*}\PY{n}{u2}\PY{o}{*}\PY{n}{v2}\PY{p}{,} \PY{p}{(}\PY{l+m+mi}{2}\PY{o}{*}\PY{n}{ii}\PY{p}{)}\PY{o}{*}\PY{n}{u2}\PY{o}{*}\PY{n}{v1} \PY{o}{+} \PY{p}{(}\PY{l+m+mi}{2}\PY{o}{*}\PY{n}{ii}\PY{p}{)}\PY{o}{*}\PY{n}{u1}\PY{o}{*}\PY{n}{v2}\PY{p}{)}\PY{p}{)}
\PY{n}{P2} \PY{o}{=} \PY{n}{vector}\PY{p}{(}\PY{n}{S}\PY{p}{,} \PY{p}{(}\PY{p}{(}\PY{o}{\PYZhy{}}\PY{n}{ii}\PY{p}{)}\PY{o}{*}\PY{n}{u1}\PY{o}{\PYZca{}}\PY{l+m+mi}{2} \PY{o}{+} \PY{p}{(}\PY{o}{\PYZhy{}}\PY{n}{ii}\PY{p}{)}\PY{o}{*}\PY{n}{u2}\PY{o}{\PYZca{}}\PY{l+m+mi}{2}\PY{p}{,} \PY{n}{u1}\PY{o}{\PYZca{}}\PY{l+m+mi}{2} \PY{o}{\PYZhy{}} \PY{n}{u2}\PY{o}{\PYZca{}}\PY{l+m+mi}{2}\PY{p}{,} \PY{l+m+mi}{2}\PY{o}{*}\PY{n}{u1}\PY{o}{*}\PY{n}{u2}\PY{p}{)}\PY{p}{)}
\PY{n}{P3} \PY{o}{=} \PY{n}{vector}\PY{p}{(}
    \PY{n}{S}\PY{p}{,} 
    \PY{p}{(}
        \PY{l+m+mi}{2}\PY{o}{*}\PY{n}{u1}\PY{o}{*}\PY{n}{v1}\PY{o}{*}\PY{n}{l1} \PY{o}{+} \PY{l+m+mi}{2}\PY{o}{*}\PY{n}{u2}\PY{o}{*}\PY{n}{v2}\PY{o}{*}\PY{n}{l1} \PY{o}{+} \PY{p}{(}\PY{o}{\PYZhy{}}\PY{n}{ii}\PY{p}{)}\PY{o}{*}\PY{n}{u1}\PY{o}{\PYZca{}}\PY{l+m+mi}{2}\PY{o}{*}\PY{n}{l2} \PY{o}{+} \PY{p}{(}\PY{o}{\PYZhy{}}\PY{n}{ii}\PY{p}{)}\PY{o}{*}\PY{n}{u2}\PY{o}{\PYZca{}}\PY{l+m+mi}{2}\PY{o}{*}\PY{n}{l2}\PY{p}{,}
        \PY{p}{(}\PY{l+m+mi}{2}\PY{o}{*}\PY{n}{ii}\PY{p}{)}\PY{o}{*}\PY{n}{u1}\PY{o}{*}\PY{n}{v1}\PY{o}{*}\PY{n}{l1} \PY{o}{+} \PY{p}{(}\PY{o}{\PYZhy{}}\PY{l+m+mi}{2}\PY{o}{*}\PY{n}{ii}\PY{p}{)}\PY{o}{*}\PY{n}{u2}\PY{o}{*}\PY{n}{v2}\PY{o}{*}\PY{n}{l1} \PY{o}{+} \PY{n}{u1}\PY{o}{\PYZca{}}\PY{l+m+mi}{2}\PY{o}{*}\PY{n}{l2} \PY{o}{\PYZhy{}} \PY{n}{u2}\PY{o}{\PYZca{}}\PY{l+m+mi}{2}\PY{o}{*}\PY{n}{l2}\PY{p}{,} 
        \PY{p}{(}\PY{l+m+mi}{2}\PY{o}{*}\PY{n}{ii}\PY{p}{)}\PY{o}{*}\PY{n}{u2}\PY{o}{*}\PY{n}{v1}\PY{o}{*}\PY{n}{l1} \PY{o}{+} \PY{p}{(}\PY{l+m+mi}{2}\PY{o}{*}\PY{n}{ii}\PY{p}{)}\PY{o}{*}\PY{n}{u1}\PY{o}{*}\PY{n}{v2}\PY{o}{*}\PY{n}{l1} \PY{o}{+} \PY{l+m+mi}{2}\PY{o}{*}\PY{n}{u1}\PY{o}{*}\PY{n}{u2}\PY{o}{*}\PY{n}{l2}
    \PY{p}{)}
\PY{p}{)}
\PY{n}{P4} \PY{o}{=} \PY{n}{vector}\PY{p}{(}\PY{n}{S}\PY{p}{,} \PY{p}{(}\PY{p}{(}\PY{o}{\PYZhy{}}\PY{n}{ii}\PY{p}{)}\PY{o}{*}\PY{n}{v1}\PY{o}{\PYZca{}}\PY{l+m+mi}{2} \PY{o}{+} \PY{p}{(}\PY{o}{\PYZhy{}}\PY{n}{ii}\PY{p}{)}\PY{o}{*}\PY{n}{v2}\PY{o}{\PYZca{}}\PY{l+m+mi}{2}\PY{p}{,} \PY{n}{v1}\PY{o}{\PYZca{}}\PY{l+m+mi}{2} \PY{o}{\PYZhy{}} \PY{n}{v2}\PY{o}{\PYZca{}}\PY{l+m+mi}{2}\PY{p}{,} \PY{l+m+mi}{2}\PY{o}{*}\PY{n}{v1}\PY{o}{*}\PY{n}{v2}\PY{p}{)}\PY{p}{)}
\PY{n}{P5} \PY{o}{=} \PY{n}{vector}\PY{p}{(}
    \PY{n}{S}\PY{p}{,} 
    \PY{p}{(}
        \PY{l+m+mi}{2}\PY{o}{*}\PY{n}{u1}\PY{o}{*}\PY{n}{v1}\PY{o}{*}\PY{n}{m1} \PY{o}{+} \PY{l+m+mi}{2}\PY{o}{*}\PY{n}{u2}\PY{o}{*}\PY{n}{v2}\PY{o}{*}\PY{n}{m1} \PY{o}{+} \PY{p}{(}\PY{o}{\PYZhy{}}\PY{n}{ii}\PY{p}{)}\PY{o}{*}\PY{n}{v1}\PY{o}{\PYZca{}}\PY{l+m+mi}{2}\PY{o}{*}\PY{n}{m2} \PY{o}{+} \PY{p}{(}\PY{o}{\PYZhy{}}\PY{n}{ii}\PY{p}{)}\PY{o}{*}\PY{n}{v2}\PY{o}{\PYZca{}}\PY{l+m+mi}{2}\PY{o}{*}\PY{n}{m2}\PY{p}{,}
        \PY{p}{(}\PY{l+m+mi}{2}\PY{o}{*}\PY{n}{ii}\PY{p}{)}\PY{o}{*}\PY{n}{u1}\PY{o}{*}\PY{n}{v1}\PY{o}{*}\PY{n}{m1} \PY{o}{+} \PY{p}{(}\PY{o}{\PYZhy{}}\PY{l+m+mi}{2}\PY{o}{*}\PY{n}{ii}\PY{p}{)}\PY{o}{*}\PY{n}{u2}\PY{o}{*}\PY{n}{v2}\PY{o}{*}\PY{n}{m1} \PY{o}{+} \PY{n}{v1}\PY{o}{\PYZca{}}\PY{l+m+mi}{2}\PY{o}{*}\PY{n}{m2} \PY{o}{\PYZhy{}} \PY{n}{v2}\PY{o}{\PYZca{}}\PY{l+m+mi}{2}\PY{o}{*}\PY{n}{m2}\PY{p}{,} 
        \PY{p}{(}\PY{l+m+mi}{2}\PY{o}{*}\PY{n}{ii}\PY{p}{)}\PY{o}{*}\PY{n}{u2}\PY{o}{*}\PY{n}{v1}\PY{o}{*}\PY{n}{m1} \PY{o}{+} \PY{p}{(}\PY{l+m+mi}{2}\PY{o}{*}\PY{n}{ii}\PY{p}{)}\PY{o}{*}\PY{n}{u1}\PY{o}{*}\PY{n}{v2}\PY{o}{*}\PY{n}{m1} \PY{o}{+} \PY{l+m+mi}{2}\PY{o}{*}\PY{n}{v1}\PY{o}{*}\PY{n}{v2}\PY{o}{*}\PY{n}{m2}
    \PY{p}{)}
\PY{p}{)}
\end{Verbatim}
\end{tcolorbox}

    \begin{tcolorbox}[breakable, size=fbox, boxrule=1pt, pad at break*=1mm,colback=cellbackground, colframe=cellborder]
\prompt{In}{incolor}{64}{\boxspacing}
\begin{Verbatim}[commandchars=\\\{\}]
\PY{k}{assert}\PY{p}{(}\PY{n}{Ciso}\PY{o}{.}\PY{n}{subs}\PY{p}{(}\PY{n}{substitution}\PY{p}{(}\PY{n}{P2}\PY{p}{)}\PY{p}{)} \PY{o}{==} \PY{l+m+mi}{0}\PY{p}{)}
\PY{k}{assert}\PY{p}{(}\PY{n}{Ciso}\PY{o}{.}\PY{n}{subs}\PY{p}{(}\PY{n}{substitution}\PY{p}{(}\PY{n}{P4}\PY{p}{)}\PY{p}{)} \PY{o}{==} \PY{l+m+mi}{0}\PY{p}{)}
\PY{k}{assert}\PY{p}{(}\PY{n}{alignments}\PY{p}{(}\PY{p}{[}\PY{n}{P1}\PY{p}{,} \PY{n}{P2}\PY{p}{,} \PY{n}{P3}\PY{p}{,} \PY{n}{P4}\PY{p}{,} \PY{n}{P5}\PY{p}{]}\PY{p}{)} \PY{o}{==} \PY{p}{[}\PY{p}{(}\PY{l+m+mi}{1}\PY{p}{,} \PY{l+m+mi}{2}\PY{p}{,} \PY{l+m+mi}{3}\PY{p}{)}\PY{p}{,} \PY{p}{(}\PY{l+m+mi}{1}\PY{p}{,} \PY{l+m+mi}{4}\PY{p}{,} \PY{l+m+mi}{5}\PY{p}{)}\PY{p}{]}\PY{p}{)}
\end{Verbatim}
\end{tcolorbox}

    It holds: \[ 
P_1 = P_2 \times P_4
\]

    \begin{tcolorbox}[breakable, size=fbox, boxrule=1pt, pad at break*=1mm,colback=cellbackground, colframe=cellborder]
\prompt{In}{incolor}{65}{\boxspacing}
\begin{Verbatim}[commandchars=\\\{\}]
\PY{k}{assert}\PY{p}{(}\PY{n}{matrix}\PY{p}{(}\PY{p}{[}\PY{n}{P1}\PY{p}{,} \PY{n}{wedge\PYZus{}product}\PY{p}{(}\PY{n}{P2}\PY{p}{,} \PY{n}{P4}\PY{p}{)}\PY{p}{]}\PY{p}{)}\PY{o}{.}\PY{n}{minors}\PY{p}{(}\PY{l+m+mi}{2}\PY{p}{)} \PY{o}{==} \PY{p}{[}\PY{l+m+mi}{0}\PY{p}{,} \PY{l+m+mi}{0}\PY{p}{,} \PY{l+m+mi}{0}\PY{p}{]}\PY{p}{)}
\end{Verbatim}
\end{tcolorbox}

    We know that \(\sigma(P_1, P_2) = 0\), \(\sigma(P_1, P_4) = 0\) and
\(P_2, P_4\) are on Ciso, hence we have: \[
s_{12} = 0, s_{14} = 0, s_{22} = 0, s_{44} = 0, s_{23} = 0, s_{45} = 0
\]

    \begin{tcolorbox}[breakable, size=fbox, boxrule=1pt, pad at break*=1mm,colback=cellbackground, colframe=cellborder]
\prompt{In}{incolor}{66}{\boxspacing}
\begin{Verbatim}[commandchars=\\\{\}]
\PY{k}{assert}\PY{p}{(}\PY{n}{scalar\PYZus{}product}\PY{p}{(}\PY{n}{P1}\PY{p}{,} \PY{n}{P2}\PY{p}{)} \PY{o}{==} \PY{l+m+mi}{0}\PY{p}{)}
\PY{k}{assert}\PY{p}{(}\PY{n}{scalar\PYZus{}product}\PY{p}{(}\PY{n}{P1}\PY{p}{,} \PY{n}{P4}\PY{p}{)} \PY{o}{==} \PY{l+m+mi}{0}\PY{p}{)}
\PY{k}{assert}\PY{p}{(}\PY{n}{scalar\PYZus{}product}\PY{p}{(}\PY{n}{P2}\PY{p}{,} \PY{n}{P2}\PY{p}{)} \PY{o}{==} \PY{l+m+mi}{0}\PY{p}{)}
\PY{k}{assert}\PY{p}{(}\PY{n}{scalar\PYZus{}product}\PY{p}{(}\PY{n}{P4}\PY{p}{,} \PY{n}{P4}\PY{p}{)} \PY{o}{==} \PY{l+m+mi}{0}\PY{p}{)}
\PY{k}{assert}\PY{p}{(}\PY{n}{scalar\PYZus{}product}\PY{p}{(}\PY{n}{P2}\PY{p}{,} \PY{n}{P3}\PY{p}{)} \PY{o}{==} \PY{l+m+mi}{0}\PY{p}{)}
\PY{k}{assert}\PY{p}{(}\PY{n}{scalar\PYZus{}product}\PY{p}{(}\PY{n}{P4}\PY{p}{,} \PY{n}{P5}\PY{p}{)} \PY{o}{==} \PY{l+m+mi}{0}\PY{p}{)}
\end{Verbatim}
\end{tcolorbox}

    We know that the \(V\) - configuration \(P_1, \dots, P_5\) has rank 8

    \begin{tcolorbox}[breakable, size=fbox, boxrule=1pt, pad at break*=1mm,colback=cellbackground, colframe=cellborder]
\prompt{In}{incolor}{67}{\boxspacing}
\begin{Verbatim}[commandchars=\\\{\}]
\PY{c+c1}{\PYZsh{} The following computation if executed, requires about 1\PYZsq{} 30\PYZsq{}\PYZsq{}}
\PY{c+c1}{\PYZsh{} assert(matrixEigenpoints([P1, P2, P3, P4, P5]).rank() == 8)}
\end{Verbatim}
\end{tcolorbox}

    We want to see when a point \(P_6\) on the line \(P_2 \vee P_4\) is an
eigenpoint. Therefore we define \(P_6\) and we try to see when it is an
eigenpoint, i.e.~when \(\Phi(P_1, \dots, P_6)\) has rank \(\leq 9\).

    \begin{tcolorbox}[breakable, size=fbox, boxrule=1pt, pad at break*=1mm,colback=cellbackground, colframe=cellborder]
\prompt{In}{incolor}{68}{\boxspacing}
\begin{Verbatim}[commandchars=\\\{\}]
\PY{n}{P6} \PY{o}{=} \PY{n}{w1}\PY{o}{*}\PY{n}{P2}\PY{o}{+}\PY{n}{w2}\PY{o}{*}\PY{n}{P4}
\end{Verbatim}
\end{tcolorbox}

    As a consequence of its definition, we have that \(s_{16} = 0\).

Now we construct the matrix of conditions:

    \begin{tcolorbox}[breakable, size=fbox, boxrule=1pt, pad at break*=1mm,colback=cellbackground, colframe=cellborder]
\prompt{In}{incolor}{69}{\boxspacing}
\begin{Verbatim}[commandchars=\\\{\}]
\PY{n}{MM} \PY{o}{=} \PY{n}{condition\PYZus{}matrix}\PY{p}{(}\PY{p}{[}\PY{n}{P1}\PY{p}{,} \PY{n}{P2}\PY{p}{,} \PY{n}{P3}\PY{p}{,} \PY{n}{P4}\PY{p}{,} \PY{n}{P5}\PY{p}{,} \PY{n}{P6}\PY{p}{]}\PY{p}{,} \PY{n}{S}\PY{p}{,} \PY{n}{standard} \PY{o}{=} \PY{l+s+s2}{\PYZdq{}}\PY{l+s+s2}{all}\PY{l+s+s2}{\PYZdq{}}\PY{p}{)}
\end{Verbatim}
\end{tcolorbox}

    The next computations show that \(P_6\) is an eigenpoint iff
w2\emph{l2}m1 - w1\emph{l1}m2 = 0. This result requires 25 minutes.

We select some order 10-minors of MM in such a way that the ideal they
generate gives precisely the conditions for which \(P_6\) is an
eigenpoint. Remember that we can assume (u2\emph{v1-u1}v2) != 0, since
\(P_2\) and \(P_4\) are distinct. It turns out that all the determinants
of the order 10-minors of MM can be divided by (u2\emph{v1-u1}v2)\^{}24.

The next block contains these computations:

    \begin{tcolorbox}[breakable, size=fbox, boxrule=1pt, pad at break*=1mm,colback=cellbackground, colframe=cellborder]
\prompt{In}{incolor}{72}{\boxspacing}
\begin{Verbatim}[commandchars=\\\{\}]
\PY{n}{doLongComputations} \PY{o}{=} \PY{k+kc}{False}
\end{Verbatim}
\end{tcolorbox}

    \begin{tcolorbox}[breakable, size=fbox, boxrule=1pt, pad at break*=1mm,colback=cellbackground, colframe=cellborder]
\prompt{In}{incolor}{73}{\boxspacing}
\begin{Verbatim}[commandchars=\\\{\}]
\PY{k}{if} \PY{n}{doLongComputations}\PY{p}{:}
    \PY{n}{Lrw} \PY{o}{=} \PY{p}{[}
        \PY{p}{[}\PY{l+m+mi}{0}\PY{p}{,} \PY{l+m+mi}{1}\PY{p}{,} \PY{l+m+mi}{3}\PY{p}{,} \PY{l+m+mi}{4}\PY{p}{,} \PY{l+m+mi}{6}\PY{p}{,} \PY{l+m+mi}{9}\PY{p}{,} \PY{l+m+mi}{10}\PY{p}{,} \PY{l+m+mi}{12}\PY{p}{,} \PY{l+m+mi}{15}\PY{p}{,} \PY{l+m+mi}{16}\PY{p}{]}\PY{p}{,}
        \PY{p}{[}\PY{l+m+mi}{0}\PY{p}{,} \PY{l+m+mi}{1}\PY{p}{,} \PY{l+m+mi}{3}\PY{p}{,} \PY{l+m+mi}{4}\PY{p}{,} \PY{l+m+mi}{6}\PY{p}{,} \PY{l+m+mi}{9}\PY{p}{,} \PY{l+m+mi}{10}\PY{p}{,} \PY{l+m+mi}{12}\PY{p}{,} \PY{l+m+mi}{15}\PY{p}{,} \PY{l+m+mi}{17}\PY{p}{]}\PY{p}{,}
        \PY{p}{[}\PY{l+m+mi}{0}\PY{p}{,} \PY{l+m+mi}{2}\PY{p}{,} \PY{l+m+mi}{3}\PY{p}{,} \PY{l+m+mi}{4}\PY{p}{,} \PY{l+m+mi}{6}\PY{p}{,} \PY{l+m+mi}{9}\PY{p}{,} \PY{l+m+mi}{10}\PY{p}{,} \PY{l+m+mi}{12}\PY{p}{,} \PY{l+m+mi}{15}\PY{p}{,} \PY{l+m+mi}{16}\PY{p}{]}\PY{p}{,}   
        \PY{p}{[}\PY{l+m+mi}{0}\PY{p}{,} \PY{l+m+mi}{1}\PY{p}{,} \PY{l+m+mi}{3}\PY{p}{,} \PY{l+m+mi}{4}\PY{p}{,} \PY{l+m+mi}{7}\PY{p}{,} \PY{l+m+mi}{9}\PY{p}{,} \PY{l+m+mi}{10}\PY{p}{,} \PY{l+m+mi}{12}\PY{p}{,} \PY{l+m+mi}{15}\PY{p}{,} \PY{l+m+mi}{16}\PY{p}{]}\PY{p}{,}
        \PY{p}{[}\PY{l+m+mi}{0}\PY{p}{,} \PY{l+m+mi}{1}\PY{p}{,} \PY{l+m+mi}{3}\PY{p}{,} \PY{l+m+mi}{4}\PY{p}{,} \PY{l+m+mi}{6}\PY{p}{,} \PY{l+m+mi}{9}\PY{p}{,} \PY{l+m+mi}{10}\PY{p}{,} \PY{l+m+mi}{13}\PY{p}{,} \PY{l+m+mi}{15}\PY{p}{,} \PY{l+m+mi}{16}\PY{p}{]}\PY{p}{,}
        \PY{p}{[}\PY{l+m+mi}{0}\PY{p}{,} \PY{l+m+mi}{1}\PY{p}{,} \PY{l+m+mi}{3}\PY{p}{,} \PY{l+m+mi}{4}\PY{p}{,} \PY{l+m+mi}{6}\PY{p}{,} \PY{l+m+mi}{9}\PY{p}{,} \PY{l+m+mi}{11}\PY{p}{,} \PY{l+m+mi}{12}\PY{p}{,} \PY{l+m+mi}{15}\PY{p}{,} \PY{l+m+mi}{16}\PY{p}{]}\PY{p}{,}
        \PY{p}{[}\PY{l+m+mi}{0}\PY{p}{,} \PY{l+m+mi}{1}\PY{p}{,} \PY{l+m+mi}{3}\PY{p}{,} \PY{l+m+mi}{5}\PY{p}{,} \PY{l+m+mi}{6}\PY{p}{,} \PY{l+m+mi}{9}\PY{p}{,} \PY{l+m+mi}{10}\PY{p}{,} \PY{l+m+mi}{12}\PY{p}{,} \PY{l+m+mi}{15}\PY{p}{,} \PY{l+m+mi}{16}\PY{p}{]}\PY{p}{,}
        \PY{p}{[}\PY{l+m+mi}{1}\PY{p}{,} \PY{l+m+mi}{2}\PY{p}{,} \PY{l+m+mi}{3}\PY{p}{,} \PY{l+m+mi}{4}\PY{p}{,} \PY{l+m+mi}{7}\PY{p}{,} \PY{l+m+mi}{9}\PY{p}{,} \PY{l+m+mi}{10}\PY{p}{,} \PY{l+m+mi}{13}\PY{p}{,} \PY{l+m+mi}{15}\PY{p}{,} \PY{l+m+mi}{16}\PY{p}{]}\PY{p}{,}
        \PY{p}{[}\PY{l+m+mi}{0}\PY{p}{,} \PY{l+m+mi}{2}\PY{p}{,} \PY{l+m+mi}{3}\PY{p}{,} \PY{l+m+mi}{4}\PY{p}{,} \PY{l+m+mi}{6}\PY{p}{,} \PY{l+m+mi}{9}\PY{p}{,} \PY{l+m+mi}{10}\PY{p}{,} \PY{l+m+mi}{12}\PY{p}{,} \PY{l+m+mi}{15}\PY{p}{,} \PY{l+m+mi}{17}\PY{p}{]}\PY{p}{,}
        \PY{p}{[}\PY{l+m+mi}{0}\PY{p}{,} \PY{l+m+mi}{1}\PY{p}{,} \PY{l+m+mi}{3}\PY{p}{,} \PY{l+m+mi}{4}\PY{p}{,} \PY{l+m+mi}{7}\PY{p}{,} \PY{l+m+mi}{9}\PY{p}{,} \PY{l+m+mi}{11}\PY{p}{,} \PY{l+m+mi}{12}\PY{p}{,} \PY{l+m+mi}{15}\PY{p}{,} \PY{l+m+mi}{16}\PY{p}{]}\PY{p}{,}
        \PY{p}{[}\PY{l+m+mi}{0}\PY{p}{,} \PY{l+m+mi}{1}\PY{p}{,} \PY{l+m+mi}{3}\PY{p}{,} \PY{l+m+mi}{5}\PY{p}{,} \PY{l+m+mi}{6}\PY{p}{,} \PY{l+m+mi}{9}\PY{p}{,} \PY{l+m+mi}{11}\PY{p}{,} \PY{l+m+mi}{12}\PY{p}{,} \PY{l+m+mi}{15}\PY{p}{,} \PY{l+m+mi}{17}\PY{p}{]}\PY{p}{,}
        \PY{p}{[}\PY{l+m+mi}{0}\PY{p}{,} \PY{l+m+mi}{1}\PY{p}{,} \PY{l+m+mi}{3}\PY{p}{,} \PY{l+m+mi}{5}\PY{p}{,} \PY{l+m+mi}{6}\PY{p}{,} \PY{l+m+mi}{9}\PY{p}{,} \PY{l+m+mi}{10}\PY{p}{,} \PY{l+m+mi}{13}\PY{p}{,} \PY{l+m+mi}{15}\PY{p}{,} \PY{l+m+mi}{16}\PY{p}{]}\PY{p}{,}
        \PY{p}{[}\PY{l+m+mi}{0}\PY{p}{,} \PY{l+m+mi}{1}\PY{p}{,} \PY{l+m+mi}{3}\PY{p}{,} \PY{l+m+mi}{4}\PY{p}{,} \PY{l+m+mi}{7}\PY{p}{,} \PY{l+m+mi}{9}\PY{p}{,} \PY{l+m+mi}{10}\PY{p}{,} \PY{l+m+mi}{12}\PY{p}{,} \PY{l+m+mi}{15}\PY{p}{,} \PY{l+m+mi}{17}\PY{p}{]}\PY{p}{,}
        \PY{p}{[}\PY{l+m+mi}{0}\PY{p}{,} \PY{l+m+mi}{1}\PY{p}{,} \PY{l+m+mi}{3}\PY{p}{,} \PY{l+m+mi}{4}\PY{p}{,} \PY{l+m+mi}{6}\PY{p}{,} \PY{l+m+mi}{9}\PY{p}{,} \PY{l+m+mi}{10}\PY{p}{,} \PY{l+m+mi}{13}\PY{p}{,} \PY{l+m+mi}{15}\PY{p}{,} \PY{l+m+mi}{17}\PY{p}{]}\PY{p}{,}
        \PY{p}{[}\PY{l+m+mi}{1}\PY{p}{,} \PY{l+m+mi}{2}\PY{p}{,} \PY{l+m+mi}{3}\PY{p}{,} \PY{l+m+mi}{4}\PY{p}{,} \PY{l+m+mi}{7}\PY{p}{,} \PY{l+m+mi}{9}\PY{p}{,} \PY{l+m+mi}{10}\PY{p}{,} \PY{l+m+mi}{13}\PY{p}{,} \PY{l+m+mi}{15}\PY{p}{,} \PY{l+m+mi}{17}\PY{p}{]}\PY{p}{,}
        \PY{p}{[}\PY{l+m+mi}{0}\PY{p}{,} \PY{l+m+mi}{1}\PY{p}{,} \PY{l+m+mi}{3}\PY{p}{,} \PY{l+m+mi}{4}\PY{p}{,} \PY{l+m+mi}{6}\PY{p}{,} \PY{l+m+mi}{9}\PY{p}{,} \PY{l+m+mi}{10}\PY{p}{,} \PY{l+m+mi}{12}\PY{p}{,} \PY{l+m+mi}{16}\PY{p}{,} \PY{l+m+mi}{17}\PY{p}{]}\PY{p}{,}
        \PY{p}{[}\PY{l+m+mi}{0}\PY{p}{,} \PY{l+m+mi}{1}\PY{p}{,} \PY{l+m+mi}{3}\PY{p}{,} \PY{l+m+mi}{5}\PY{p}{,} \PY{l+m+mi}{6}\PY{p}{,} \PY{l+m+mi}{9}\PY{p}{,} \PY{l+m+mi}{11}\PY{p}{,} \PY{l+m+mi}{12}\PY{p}{,} \PY{l+m+mi}{16}\PY{p}{,} \PY{l+m+mi}{17}\PY{p}{]}\PY{p}{,}
        \PY{p}{[}\PY{l+m+mi}{0}\PY{p}{,} \PY{l+m+mi}{2}\PY{p}{,} \PY{l+m+mi}{3}\PY{p}{,} \PY{l+m+mi}{4}\PY{p}{,} \PY{l+m+mi}{6}\PY{p}{,} \PY{l+m+mi}{9}\PY{p}{,} \PY{l+m+mi}{10}\PY{p}{,} \PY{l+m+mi}{12}\PY{p}{,} \PY{l+m+mi}{16}\PY{p}{,} \PY{l+m+mi}{17}\PY{p}{]}
    \PY{p}{]}

    \PY{n}{Jb} \PY{o}{=} \PY{n}{S}\PY{o}{.}\PY{n}{ideal}\PY{p}{(}\PY{l+m+mi}{0}\PY{p}{)}
    \PY{n}{flag} \PY{o}{=} \PY{l+m+mi}{1}
    \PY{k}{for} \PY{n}{ll} \PY{o+ow}{in} \PY{n}{Lrw}\PY{p}{:}
        \PY{n+nb}{print}\PY{p}{(}\PY{n}{ll}\PY{p}{)}
        \PY{n}{sleep}\PY{p}{(}\PY{l+m+mi}{1}\PY{p}{)}
        \PY{n}{ttA} \PY{o}{=} \PY{n}{cputime}\PY{p}{(}\PY{p}{)}
        \PY{n}{Mx} \PY{o}{=} \PY{n}{MM}\PY{o}{.}\PY{n}{matrix\PYZus{}from\PYZus{}rows}\PY{p}{(}\PY{n}{ll}\PY{p}{)}
        \PY{n}{ddt} \PY{o}{=} \PY{n}{Mx}\PY{o}{.}\PY{n}{det}\PY{p}{(}\PY{p}{)}
        \PY{n+nb}{print}\PY{p}{(}\PY{l+s+s2}{\PYZdq{}}\PY{l+s+s2}{computed long determinant}\PY{l+s+s2}{\PYZdq{}}\PY{p}{)}
        \PY{n}{sleep}\PY{p}{(}\PY{l+m+mi}{1}\PY{p}{)}
        \PY{n}{ddtDiv} \PY{o}{=} \PY{n}{ddt}\PY{o}{.}\PY{n}{quo\PYZus{}rem}\PY{p}{(}\PY{p}{(}\PY{n}{u2}\PY{o}{*}\PY{n}{v1}\PY{o}{\PYZhy{}}\PY{n}{u1}\PY{o}{*}\PY{n}{v2}\PY{p}{)}\PY{o}{\PYZca{}}\PY{l+m+mi}{24}\PY{p}{)} \PY{c+c1}{\PYZsh{}\PYZsh{} it turns out this happens}
        \PY{k}{if} \PY{n}{ddtDiv}\PY{p}{[}\PY{l+m+mi}{1}\PY{p}{]} \PY{o}{==} \PY{l+m+mi}{0}\PY{p}{:}
            \PY{n}{ff} \PY{o}{=} \PY{n}{ddtDiv}\PY{p}{[}\PY{l+m+mi}{0}\PY{p}{]}
        \PY{k}{else}\PY{p}{:}
            \PY{n}{ff} \PY{o}{=} \PY{n}{ddt}
            \PY{n+nb}{print}\PY{p}{(}\PY{l+s+s2}{\PYZdq{}}\PY{l+s+s2}{just in case...}\PY{l+s+s2}{\PYZdq{}}\PY{p}{)} \PY{c+c1}{\PYZsh{}\PYZsh{} In practise, this never happens}
        \PY{n+nb}{print}\PY{p}{(}\PY{n}{cputime}\PY{p}{(}\PY{p}{)}\PY{o}{\PYZhy{}}\PY{n}{ttA}\PY{p}{)}
        \PY{n}{sleep}\PY{p}{(}\PY{l+m+mi}{1}\PY{p}{)}
        \PY{n}{ffId} \PY{o}{=} \PY{n}{S}\PY{o}{.}\PY{n}{ideal}\PY{p}{(}\PY{n}{ff}\PY{p}{)}
        \PY{c+c1}{\PYZsh{}\PYZsh{} we saturate w.r.t. conditions that are surely satisfied.}
        \PY{n}{ffId} \PY{o}{=} \PY{n}{ffId}\PY{o}{.}\PY{n}{saturation}\PY{p}{(}\PY{p}{(}\PY{n}{u2}\PY{o}{*}\PY{n}{v1}\PY{o}{\PYZhy{}}\PY{n}{u1}\PY{o}{*}\PY{n}{v2}\PY{p}{)}\PY{o}{*}\PY{n}{m1}\PY{o}{*}\PY{n}{m2}\PY{o}{*}\PY{n}{l1}\PY{o}{*}\PY{n}{l2}\PY{o}{*}\PY{n}{w1}\PY{o}{*}\PY{n}{w2}\PY{p}{)}\PY{p}{[}\PY{l+m+mi}{0}\PY{p}{]}
        \PY{n}{Jb} \PY{o}{=} \PY{n}{Jb} \PY{o}{+} \PY{n}{ffId}
        \PY{n}{Jb} \PY{o}{=} \PY{n}{Jb}\PY{o}{.}\PY{n}{saturation}\PY{p}{(}\PY{p}{(}\PY{n}{u2}\PY{o}{*}\PY{n}{v1}\PY{o}{\PYZhy{}}\PY{n}{u1}\PY{o}{*}\PY{n}{v2}\PY{p}{)}\PY{o}{*}\PY{n}{m1}\PY{o}{*}\PY{n}{m2}\PY{o}{*}\PY{n}{l1}\PY{o}{*}\PY{n}{l2}\PY{o}{*}\PY{n}{w1}\PY{o}{*}\PY{n}{w2}\PY{p}{)}\PY{p}{[}\PY{l+m+mi}{0}\PY{p}{]}
        \PY{n+nb}{print}\PY{p}{(}\PY{l+s+s2}{\PYZdq{}}\PY{l+s+s2}{computation n. : }\PY{l+s+s2}{\PYZdq{}}\PY{o}{+}\PY{n+nb}{str}\PY{p}{(}\PY{n}{flag}\PY{p}{)}\PY{o}{+}\PY{l+s+s2}{\PYZdq{}}\PY{l+s+s2}{ over }\PY{l+s+s2}{\PYZdq{}}\PY{o}{+} \PY{n+nb}{str}\PY{p}{(}\PY{n+nb}{len}\PY{p}{(}\PY{n}{Lrw}\PY{p}{)}\PY{p}{)}\PY{p}{)}
        \PY{n+nb}{print}\PY{p}{(}\PY{l+s+s2}{\PYZdq{}}\PY{l+s+s2}{\PYZdq{}}\PY{p}{)}
        \PY{n}{flag} \PY{o}{+}\PY{o}{=} \PY{l+m+mi}{1}
        \PY{n}{sleep}\PY{p}{(}\PY{l+m+mi}{1}\PY{p}{)}
\PY{k}{else}\PY{p}{:}
    \PY{n}{Jb} \PY{o}{=} \PY{n}{S}\PY{o}{.}\PY{n}{ideal}\PY{p}{(}\PY{n}{w2}\PY{o}{*}\PY{n}{l2}\PY{o}{*}\PY{n}{m1} \PY{o}{\PYZhy{}} \PY{n}{w1}\PY{o}{*}\PY{n}{l1}\PY{o}{*}\PY{n}{m2}\PY{p}{)}
\end{Verbatim}
\end{tcolorbox}

    The only condition we get is: w2\emph{l2}m1 - w1\emph{l1}m2 = 0

    \begin{tcolorbox}[breakable, size=fbox, boxrule=1pt, pad at break*=1mm,colback=cellbackground, colframe=cellborder]
\prompt{In}{incolor}{74}{\boxspacing}
\begin{Verbatim}[commandchars=\\\{\}]
\PY{k}{assert}\PY{p}{(}\PY{n}{Jb} \PY{o}{==} \PY{n}{S}\PY{o}{.}\PY{n}{ideal}\PY{p}{(}\PY{n}{w2}\PY{o}{*}\PY{n}{l2}\PY{o}{*}\PY{n}{m1} \PY{o}{\PYZhy{}} \PY{n}{w1}\PY{o}{*}\PY{n}{l1}\PY{o}{*}\PY{n}{m2}\PY{p}{)}\PY{p}{)}
\end{Verbatim}
\end{tcolorbox}

    We construct a matrix whose determinant is the cubic which has
\(P_1, P_2, P_3, P_4, P_5\) as eigenpoints and a row of \(\phi(P_6)\):

    We construct \(P_6\) with this condition:

    \begin{tcolorbox}[breakable, size=fbox, boxrule=1pt, pad at break*=1mm,colback=cellbackground, colframe=cellborder]
\prompt{In}{incolor}{75}{\boxspacing}
\begin{Verbatim}[commandchars=\\\{\}]
\PY{n}{PP6} \PY{o}{=} \PY{n}{P6}\PY{o}{.}\PY{n}{subs}\PY{p}{(}\PY{p}{\PYZob{}}\PY{n}{w1}\PY{p}{:}\PY{n}{l2}\PY{o}{*}\PY{n}{m1}\PY{p}{,} \PY{n}{w2}\PY{p}{:} \PY{n}{l1}\PY{o}{*}\PY{n}{m2}\PY{p}{\PYZcb{}}\PY{p}{)}
\end{Verbatim}
\end{tcolorbox}

    It holds: \[
P_6 = s_{11}s_{15}P_3-2s_{13}s_{15}P_1+s_{11}s_{13}P_5
\] and also: \[
P_6 = s_{15}s_{34}P_2+s_{13}s_{25}P_4
\]

    \begin{tcolorbox}[breakable, size=fbox, boxrule=1pt, pad at break*=1mm,colback=cellbackground, colframe=cellborder]
\prompt{In}{incolor}{76}{\boxspacing}
\begin{Verbatim}[commandchars=\\\{\}]
\PY{n}{PP6a} \PY{o}{=} \PY{p}{(}
    \PY{n}{scalar\PYZus{}product}\PY{p}{(}\PY{n}{P1}\PY{p}{,} \PY{n}{P1}\PY{p}{)}\PY{o}{*}\PY{n}{scalar\PYZus{}product}\PY{p}{(}\PY{n}{P1}\PY{p}{,} \PY{n}{P5}\PY{p}{)}\PY{o}{*}\PY{n}{P3}
    \PY{o}{\PYZhy{}} \PY{l+m+mi}{2}\PY{o}{*}\PY{n}{scalar\PYZus{}product}\PY{p}{(}\PY{n}{P1}\PY{p}{,} \PY{n}{P3}\PY{p}{)}\PY{o}{*}\PY{n}{scalar\PYZus{}product}\PY{p}{(}\PY{n}{P1}\PY{p}{,} \PY{n}{P5}\PY{p}{)}\PY{o}{*}\PY{n}{P1}
    \PY{o}{+} \PY{n}{scalar\PYZus{}product}\PY{p}{(}\PY{n}{P1}\PY{p}{,} \PY{n}{P1}\PY{p}{)}\PY{o}{*}\PY{n}{scalar\PYZus{}product}\PY{p}{(}\PY{n}{P1}\PY{p}{,} \PY{n}{P3}\PY{p}{)}\PY{o}{*}\PY{n}{P5}
\PY{p}{)}

\PY{k}{assert}\PY{p}{(}\PY{n}{S}\PY{o}{.}\PY{n}{ideal}\PY{p}{(}\PY{n}{matrix}\PY{p}{(}\PY{p}{[}\PY{n}{PP6}\PY{p}{,} \PY{n}{PP6a}\PY{p}{]}\PY{p}{)}\PY{o}{.}\PY{n}{minors}\PY{p}{(}\PY{l+m+mi}{2}\PY{p}{)}\PY{p}{)} \PY{o}{==} \PY{n}{S}\PY{o}{.}\PY{n}{ideal}\PY{p}{(}\PY{l+m+mi}{0}\PY{p}{)}\PY{p}{)}

\PY{n}{PP6b} \PY{o}{=} \PY{n}{scalar\PYZus{}product}\PY{p}{(}\PY{n}{P1}\PY{p}{,} \PY{n}{P5}\PY{p}{)}\PY{o}{*}\PY{n}{scalar\PYZus{}product}\PY{p}{(}\PY{n}{P3}\PY{p}{,} \PY{n}{P4}\PY{p}{)}\PY{o}{*}\PY{n}{P2}\PY{o}{+}\PY{n}{scalar\PYZus{}product}\PY{p}{(}\PY{n}{P1}\PY{p}{,} \PY{n}{P3}\PY{p}{)}\PY{o}{*}\PY{n}{scalar\PYZus{}product}\PY{p}{(}\PY{n}{P2}\PY{p}{,} \PY{n}{P5}\PY{p}{)}\PY{o}{*}\PY{n}{P4}
\PY{k}{assert}\PY{p}{(}\PY{n}{S}\PY{o}{.}\PY{n}{ideal}\PY{p}{(}\PY{n}{matrix}\PY{p}{(}\PY{p}{[}\PY{n}{PP6}\PY{p}{,} \PY{n}{PP6b}\PY{p}{]}\PY{p}{)}\PY{o}{.}\PY{n}{minors}\PY{p}{(}\PY{l+m+mi}{2}\PY{p}{)}\PY{p}{)} \PY{o}{==} \PY{n}{S}\PY{o}{.}\PY{n}{ideal}\PY{p}{(}\PY{l+m+mi}{0}\PY{p}{)}\PY{p}{)}
\end{Verbatim}
\end{tcolorbox}

    We construct the cubic whose eigenpoints are \(P_1, \dots, P_6\):

    \begin{tcolorbox}[breakable, size=fbox, boxrule=1pt, pad at break*=1mm,colback=cellbackground, colframe=cellborder]
\prompt{In}{incolor}{77}{\boxspacing}
\begin{Verbatim}[commandchars=\\\{\}]
\PY{n}{MM1} \PY{o}{=} \PY{n}{condition\PYZus{}matrix}\PY{p}{(}\PY{p}{[}\PY{n}{P1}\PY{p}{,} \PY{n}{P2}\PY{p}{,} \PY{n}{P3}\PY{p}{,} \PY{n}{P4}\PY{p}{,} \PY{n}{P5}\PY{p}{,} \PY{n}{PP6}\PY{p}{]}\PY{p}{,} \PY{n}{S}\PY{p}{,} \PY{n}{standard} \PY{o}{=} \PY{l+s+s2}{\PYZdq{}}\PY{l+s+s2}{all}\PY{l+s+s2}{\PYZdq{}}\PY{p}{)}\PY{o}{.}\PY{n}{matrix\PYZus{}from\PYZus{}rows}\PY{p}{(}\PY{p}{[}\PY{l+m+mi}{0}\PY{p}{,} \PY{l+m+mi}{1}\PY{p}{,} \PY{l+m+mi}{3}\PY{p}{,} \PY{l+m+mi}{4}\PY{p}{,} \PY{l+m+mi}{6}\PY{p}{,} \PY{l+m+mi}{9}\PY{p}{,} \PY{l+m+mi}{10}\PY{p}{,} \PY{l+m+mi}{12}\PY{p}{,} \PY{l+m+mi}{16}\PY{p}{]}\PY{p}{)}
\PY{n}{MM1} \PY{o}{=} \PY{n}{MM1}\PY{o}{.}\PY{n}{stack}\PY{p}{(}\PY{n}{vector}\PY{p}{(}\PY{n}{mon}\PY{p}{)}\PY{p}{)}
\end{Verbatim}
\end{tcolorbox}

    \begin{tcolorbox}[breakable, size=fbox, boxrule=1pt, pad at break*=1mm,colback=cellbackground, colframe=cellborder]
\prompt{In}{incolor}{78}{\boxspacing}
\begin{Verbatim}[commandchars=\\\{\}]
\PY{c+c1}{\PYZsh{}\PYZsh{} the following computation requires 146 seconds:}
\PY{n}{cbc} \PY{o}{=} \PY{n}{MM1}\PY{o}{.}\PY{n}{det}\PY{p}{(}\PY{p}{)}
\end{Verbatim}
\end{tcolorbox}

    We can find that \(P_7\) is given by the formula: \[
s_{11}s_{15}P_3+s_{13}s_{15}P_1+s_{11}s_{13}P_5
\] and also by: \[
s_{15}(s_{26}s_{46}+s_{24}s_{66})P_1+s_{11}s_{24}s_{56}P_6
\]

    \begin{tcolorbox}[breakable, size=fbox, boxrule=1pt, pad at break*=1mm,colback=cellbackground, colframe=cellborder]
\prompt{In}{incolor}{79}{\boxspacing}
\begin{Verbatim}[commandchars=\\\{\}]
\PY{n}{PP7b} \PY{o}{=} \PY{p}{(}
    \PY{n}{scalar\PYZus{}product}\PY{p}{(}\PY{n}{P1}\PY{p}{,} \PY{n}{P5}\PY{p}{)}
    \PY{o}{*} \PY{p}{(}\PY{n}{scalar\PYZus{}product}\PY{p}{(}\PY{n}{P2}\PY{p}{,} \PY{n}{PP6}\PY{p}{)}\PY{o}{*}\PY{n}{scalar\PYZus{}product}\PY{p}{(}\PY{n}{P4}\PY{p}{,} \PY{n}{PP6}\PY{p}{)}\PY{o}{+} \PY{n}{scalar\PYZus{}product}\PY{p}{(}\PY{n}{P2}\PY{p}{,} \PY{n}{P4}\PY{p}{)}\PY{o}{*}\PY{n}{scalar\PYZus{}product}\PY{p}{(}\PY{n}{PP6}\PY{p}{,} \PY{n}{PP6}\PY{p}{)}\PY{p}{)}\PY{o}{*}\PY{n}{P1}
    \PY{o}{+} \PY{n}{scalar\PYZus{}product}\PY{p}{(}\PY{n}{P1}\PY{p}{,} \PY{n}{P1}\PY{p}{)}\PY{o}{*}\PY{n}{scalar\PYZus{}product}\PY{p}{(}\PY{n}{P2}\PY{p}{,} \PY{n}{P4}\PY{p}{)}\PY{o}{*}\PY{n}{scalar\PYZus{}product}\PY{p}{(}\PY{n}{P5}\PY{p}{,} \PY{n}{PP6}\PY{p}{)}\PY{o}{*}\PY{n}{PP6}
\PY{p}{)}

\PY{n}{PP7} \PY{o}{=} \PY{p}{(}
    \PY{n}{scalar\PYZus{}product}\PY{p}{(}\PY{n}{P1}\PY{p}{,} \PY{n}{P1}\PY{p}{)}\PY{o}{*}\PY{n}{scalar\PYZus{}product}\PY{p}{(}\PY{n}{P1}\PY{p}{,} \PY{n}{P5}\PY{p}{)}\PY{o}{*}\PY{n}{P3}
    \PY{o}{+} \PY{n}{scalar\PYZus{}product}\PY{p}{(}\PY{n}{P1}\PY{p}{,} \PY{n}{P3}\PY{p}{)}\PY{o}{*}\PY{n}{scalar\PYZus{}product}\PY{p}{(}\PY{n}{P1}\PY{p}{,} \PY{n}{P5}\PY{p}{)}\PY{o}{*}\PY{n}{P1}
    \PY{o}{+} \PY{n}{scalar\PYZus{}product}\PY{p}{(}\PY{n}{P1}\PY{p}{,} \PY{n}{P1}\PY{p}{)}\PY{o}{*}\PY{n}{scalar\PYZus{}product}\PY{p}{(}\PY{n}{P1}\PY{p}{,} \PY{n}{P3}\PY{p}{)}\PY{o}{*}\PY{n}{P5}
\PY{p}{)}
\end{Verbatim}
\end{tcolorbox}

    PP7 and PP7b are the same point:

    \begin{tcolorbox}[breakable, size=fbox, boxrule=1pt, pad at break*=1mm,colback=cellbackground, colframe=cellborder]
\prompt{In}{incolor}{80}{\boxspacing}
\begin{Verbatim}[commandchars=\\\{\}]
\PY{k}{assert}\PY{p}{(}\PY{n}{matrix}\PY{p}{(}\PY{p}{[}\PY{n}{PP7}\PY{p}{,} \PY{n}{PP7b}\PY{p}{]}\PY{p}{)}\PY{o}{.}\PY{n}{minors}\PY{p}{(}\PY{l+m+mi}{2}\PY{p}{)} \PY{o}{==} \PY{p}{[}\PY{l+m+mi}{0}\PY{p}{,} \PY{l+m+mi}{0}\PY{p}{,} \PY{l+m+mi}{0}\PY{p}{]}\PY{p}{)}
\end{Verbatim}
\end{tcolorbox}

    \(P_7\) is an eigenpoint (about 40 seconds of computations):

    \begin{tcolorbox}[breakable, size=fbox, boxrule=1pt, pad at break*=1mm,colback=cellbackground, colframe=cellborder]
\prompt{In}{incolor}{81}{\boxspacing}
\begin{Verbatim}[commandchars=\\\{\}]
\PY{k}{assert}\PY{p}{(}\PY{n}{S}\PY{o}{.}\PY{n}{ideal}\PY{p}{(}\PY{n+nb}{list}\PY{p}{(}\PY{n}{eig}\PY{p}{(}\PY{n}{cbc}\PY{p}{)}\PY{p}{)}\PY{p}{)}\PY{o}{.}\PY{n}{subs}\PY{p}{(}\PY{n}{substitution}\PY{p}{(}\PY{n}{PP7}\PY{p}{)}\PY{p}{)} \PY{o}{==} \PY{n}{S}\PY{o}{.}\PY{n}{ideal}\PY{p}{(}\PY{n}{S}\PY{o}{.}\PY{n}{zero}\PY{p}{(}\PY{p}{)}\PY{p}{)}\PY{p}{)}
\end{Verbatim}
\end{tcolorbox}


    % Add a bibliography block to the postdoc
    
    
    
\end{document}
