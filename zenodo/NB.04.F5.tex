\documentclass[11pt]{article}

    \usepackage[breakable]{tcolorbox}
    \usepackage{parskip} % Stop auto-indenting (to mimic markdown behaviour)
    

    % Basic figure setup, for now with no caption control since it's done
    % automatically by Pandoc (which extracts ![](path) syntax from Markdown).
    \usepackage{graphicx}
    % Maintain compatibility with old templates. Remove in nbconvert 6.0
    \let\Oldincludegraphics\includegraphics
    % Ensure that by default, figures have no caption (until we provide a
    % proper Figure object with a Caption API and a way to capture that
    % in the conversion process - todo).
    \usepackage{caption}
    \DeclareCaptionFormat{nocaption}{}
    \captionsetup{format=nocaption,aboveskip=0pt,belowskip=0pt}

    \usepackage{float}
    \floatplacement{figure}{H} % forces figures to be placed at the correct location
    \usepackage{xcolor} % Allow colors to be defined
    \usepackage{enumerate} % Needed for markdown enumerations to work
    \usepackage{geometry} % Used to adjust the document margins
    \usepackage{amsmath} % Equations
    \usepackage{amssymb} % Equations
    \usepackage{textcomp} % defines textquotesingle
    % Hack from http://tex.stackexchange.com/a/47451/13684:
    \AtBeginDocument{%
        \def\PYZsq{\textquotesingle}% Upright quotes in Pygmentized code
    }
    \usepackage{upquote} % Upright quotes for verbatim code
    \usepackage{eurosym} % defines \euro

    \usepackage{iftex}
    \ifPDFTeX
        \usepackage[T1]{fontenc}
        \IfFileExists{alphabeta.sty}{
              \usepackage{alphabeta}
          }{
              \usepackage[mathletters]{ucs}
              \usepackage[utf8x]{inputenc}
          }
    \else
        \usepackage{fontspec}
        \usepackage{unicode-math}
    \fi

    \usepackage{fancyvrb} % verbatim replacement that allows latex
    \usepackage{grffile} % extends the file name processing of package graphics
                         % to support a larger range
    \makeatletter % fix for old versions of grffile with XeLaTeX
    \@ifpackagelater{grffile}{2019/11/01}
    {
      % Do nothing on new versions
    }
    {
      \def\Gread@@xetex#1{%
        \IfFileExists{"\Gin@base".bb}%
        {\Gread@eps{\Gin@base.bb}}%
        {\Gread@@xetex@aux#1}%
      }
    }
    \makeatother
    \usepackage[Export]{adjustbox} % Used to constrain images to a maximum size
    \adjustboxset{max size={0.9\linewidth}{0.9\paperheight}}

    % The hyperref package gives us a pdf with properly built
    % internal navigation ('pdf bookmarks' for the table of contents,
    % internal cross-reference links, web links for URLs, etc.)
    \usepackage{hyperref}
    % The default LaTeX title has an obnoxious amount of whitespace. By default,
    % titling removes some of it. It also provides customization options.
    \usepackage{titling}
    \usepackage{longtable} % longtable support required by pandoc >1.10
    \usepackage{booktabs}  % table support for pandoc > 1.12.2
    \usepackage{array}     % table support for pandoc >= 2.11.3
    \usepackage{calc}      % table minipage width calculation for pandoc >= 2.11.1
    \usepackage[inline]{enumitem} % IRkernel/repr support (it uses the enumerate* environment)
    \usepackage[normalem]{ulem} % ulem is needed to support strikethroughs (\sout)
                                % normalem makes italics be italics, not underlines
    \usepackage{soul}      % strikethrough (\st) support for pandoc >= 3.0.0
    \usepackage{mathrsfs}
    

    
    % Colors for the hyperref package
    \definecolor{urlcolor}{rgb}{0,.145,.698}
    \definecolor{linkcolor}{rgb}{.71,0.21,0.01}
    \definecolor{citecolor}{rgb}{.12,.54,.11}

    % ANSI colors
    \definecolor{ansi-black}{HTML}{3E424D}
    \definecolor{ansi-black-intense}{HTML}{282C36}
    \definecolor{ansi-red}{HTML}{E75C58}
    \definecolor{ansi-red-intense}{HTML}{B22B31}
    \definecolor{ansi-green}{HTML}{00A250}
    \definecolor{ansi-green-intense}{HTML}{007427}
    \definecolor{ansi-yellow}{HTML}{DDB62B}
    \definecolor{ansi-yellow-intense}{HTML}{B27D12}
    \definecolor{ansi-blue}{HTML}{208FFB}
    \definecolor{ansi-blue-intense}{HTML}{0065CA}
    \definecolor{ansi-magenta}{HTML}{D160C4}
    \definecolor{ansi-magenta-intense}{HTML}{A03196}
    \definecolor{ansi-cyan}{HTML}{60C6C8}
    \definecolor{ansi-cyan-intense}{HTML}{258F8F}
    \definecolor{ansi-white}{HTML}{C5C1B4}
    \definecolor{ansi-white-intense}{HTML}{A1A6B2}
    \definecolor{ansi-default-inverse-fg}{HTML}{FFFFFF}
    \definecolor{ansi-default-inverse-bg}{HTML}{000000}

    % common color for the border for error outputs.
    \definecolor{outerrorbackground}{HTML}{FFDFDF}

    % commands and environments needed by pandoc snippets
    % extracted from the output of `pandoc -s`
    \providecommand{\tightlist}{%
      \setlength{\itemsep}{0pt}\setlength{\parskip}{0pt}}
    \DefineVerbatimEnvironment{Highlighting}{Verbatim}{commandchars=\\\{\}}
    % Add ',fontsize=\small' for more characters per line
    \newenvironment{Shaded}{}{}
    \newcommand{\KeywordTok}[1]{\textcolor[rgb]{0.00,0.44,0.13}{\textbf{{#1}}}}
    \newcommand{\DataTypeTok}[1]{\textcolor[rgb]{0.56,0.13,0.00}{{#1}}}
    \newcommand{\DecValTok}[1]{\textcolor[rgb]{0.25,0.63,0.44}{{#1}}}
    \newcommand{\BaseNTok}[1]{\textcolor[rgb]{0.25,0.63,0.44}{{#1}}}
    \newcommand{\FloatTok}[1]{\textcolor[rgb]{0.25,0.63,0.44}{{#1}}}
    \newcommand{\CharTok}[1]{\textcolor[rgb]{0.25,0.44,0.63}{{#1}}}
    \newcommand{\StringTok}[1]{\textcolor[rgb]{0.25,0.44,0.63}{{#1}}}
    \newcommand{\CommentTok}[1]{\textcolor[rgb]{0.38,0.63,0.69}{\textit{{#1}}}}
    \newcommand{\OtherTok}[1]{\textcolor[rgb]{0.00,0.44,0.13}{{#1}}}
    \newcommand{\AlertTok}[1]{\textcolor[rgb]{1.00,0.00,0.00}{\textbf{{#1}}}}
    \newcommand{\FunctionTok}[1]{\textcolor[rgb]{0.02,0.16,0.49}{{#1}}}
    \newcommand{\RegionMarkerTok}[1]{{#1}}
    \newcommand{\ErrorTok}[1]{\textcolor[rgb]{1.00,0.00,0.00}{\textbf{{#1}}}}
    \newcommand{\NormalTok}[1]{{#1}}

    % Additional commands for more recent versions of Pandoc
    \newcommand{\ConstantTok}[1]{\textcolor[rgb]{0.53,0.00,0.00}{{#1}}}
    \newcommand{\SpecialCharTok}[1]{\textcolor[rgb]{0.25,0.44,0.63}{{#1}}}
    \newcommand{\VerbatimStringTok}[1]{\textcolor[rgb]{0.25,0.44,0.63}{{#1}}}
    \newcommand{\SpecialStringTok}[1]{\textcolor[rgb]{0.73,0.40,0.53}{{#1}}}
    \newcommand{\ImportTok}[1]{{#1}}
    \newcommand{\DocumentationTok}[1]{\textcolor[rgb]{0.73,0.13,0.13}{\textit{{#1}}}}
    \newcommand{\AnnotationTok}[1]{\textcolor[rgb]{0.38,0.63,0.69}{\textbf{\textit{{#1}}}}}
    \newcommand{\CommentVarTok}[1]{\textcolor[rgb]{0.38,0.63,0.69}{\textbf{\textit{{#1}}}}}
    \newcommand{\VariableTok}[1]{\textcolor[rgb]{0.10,0.09,0.49}{{#1}}}
    \newcommand{\ControlFlowTok}[1]{\textcolor[rgb]{0.00,0.44,0.13}{\textbf{{#1}}}}
    \newcommand{\OperatorTok}[1]{\textcolor[rgb]{0.40,0.40,0.40}{{#1}}}
    \newcommand{\BuiltInTok}[1]{{#1}}
    \newcommand{\ExtensionTok}[1]{{#1}}
    \newcommand{\PreprocessorTok}[1]{\textcolor[rgb]{0.74,0.48,0.00}{{#1}}}
    \newcommand{\AttributeTok}[1]{\textcolor[rgb]{0.49,0.56,0.16}{{#1}}}
    \newcommand{\InformationTok}[1]{\textcolor[rgb]{0.38,0.63,0.69}{\textbf{\textit{{#1}}}}}
    \newcommand{\WarningTok}[1]{\textcolor[rgb]{0.38,0.63,0.69}{\textbf{\textit{{#1}}}}}


    % Define a nice break command that doesn't care if a line doesn't already
    % exist.
    \def\br{\hspace*{\fill} \\* }
    % Math Jax compatibility definitions
    \def\gt{>}
    \def\lt{<}
    \let\Oldtex\TeX
    \let\Oldlatex\LaTeX
    \renewcommand{\TeX}{\textrm{\Oldtex}}
    \renewcommand{\LaTeX}{\textrm{\Oldlatex}}
    % Document parameters
    % Document title
    \title{NB.04.F5}
    
    
    
    
    
    
    
% Pygments definitions
\makeatletter
\def\PY@reset{\let\PY@it=\relax \let\PY@bf=\relax%
    \let\PY@ul=\relax \let\PY@tc=\relax%
    \let\PY@bc=\relax \let\PY@ff=\relax}
\def\PY@tok#1{\csname PY@tok@#1\endcsname}
\def\PY@toks#1+{\ifx\relax#1\empty\else%
    \PY@tok{#1}\expandafter\PY@toks\fi}
\def\PY@do#1{\PY@bc{\PY@tc{\PY@ul{%
    \PY@it{\PY@bf{\PY@ff{#1}}}}}}}
\def\PY#1#2{\PY@reset\PY@toks#1+\relax+\PY@do{#2}}

\@namedef{PY@tok@w}{\def\PY@tc##1{\textcolor[rgb]{0.73,0.73,0.73}{##1}}}
\@namedef{PY@tok@c}{\let\PY@it=\textit\def\PY@tc##1{\textcolor[rgb]{0.24,0.48,0.48}{##1}}}
\@namedef{PY@tok@cp}{\def\PY@tc##1{\textcolor[rgb]{0.61,0.40,0.00}{##1}}}
\@namedef{PY@tok@k}{\let\PY@bf=\textbf\def\PY@tc##1{\textcolor[rgb]{0.00,0.50,0.00}{##1}}}
\@namedef{PY@tok@kp}{\def\PY@tc##1{\textcolor[rgb]{0.00,0.50,0.00}{##1}}}
\@namedef{PY@tok@kt}{\def\PY@tc##1{\textcolor[rgb]{0.69,0.00,0.25}{##1}}}
\@namedef{PY@tok@o}{\def\PY@tc##1{\textcolor[rgb]{0.40,0.40,0.40}{##1}}}
\@namedef{PY@tok@ow}{\let\PY@bf=\textbf\def\PY@tc##1{\textcolor[rgb]{0.67,0.13,1.00}{##1}}}
\@namedef{PY@tok@nb}{\def\PY@tc##1{\textcolor[rgb]{0.00,0.50,0.00}{##1}}}
\@namedef{PY@tok@nf}{\def\PY@tc##1{\textcolor[rgb]{0.00,0.00,1.00}{##1}}}
\@namedef{PY@tok@nc}{\let\PY@bf=\textbf\def\PY@tc##1{\textcolor[rgb]{0.00,0.00,1.00}{##1}}}
\@namedef{PY@tok@nn}{\let\PY@bf=\textbf\def\PY@tc##1{\textcolor[rgb]{0.00,0.00,1.00}{##1}}}
\@namedef{PY@tok@ne}{\let\PY@bf=\textbf\def\PY@tc##1{\textcolor[rgb]{0.80,0.25,0.22}{##1}}}
\@namedef{PY@tok@nv}{\def\PY@tc##1{\textcolor[rgb]{0.10,0.09,0.49}{##1}}}
\@namedef{PY@tok@no}{\def\PY@tc##1{\textcolor[rgb]{0.53,0.00,0.00}{##1}}}
\@namedef{PY@tok@nl}{\def\PY@tc##1{\textcolor[rgb]{0.46,0.46,0.00}{##1}}}
\@namedef{PY@tok@ni}{\let\PY@bf=\textbf\def\PY@tc##1{\textcolor[rgb]{0.44,0.44,0.44}{##1}}}
\@namedef{PY@tok@na}{\def\PY@tc##1{\textcolor[rgb]{0.41,0.47,0.13}{##1}}}
\@namedef{PY@tok@nt}{\let\PY@bf=\textbf\def\PY@tc##1{\textcolor[rgb]{0.00,0.50,0.00}{##1}}}
\@namedef{PY@tok@nd}{\def\PY@tc##1{\textcolor[rgb]{0.67,0.13,1.00}{##1}}}
\@namedef{PY@tok@s}{\def\PY@tc##1{\textcolor[rgb]{0.73,0.13,0.13}{##1}}}
\@namedef{PY@tok@sd}{\let\PY@it=\textit\def\PY@tc##1{\textcolor[rgb]{0.73,0.13,0.13}{##1}}}
\@namedef{PY@tok@si}{\let\PY@bf=\textbf\def\PY@tc##1{\textcolor[rgb]{0.64,0.35,0.47}{##1}}}
\@namedef{PY@tok@se}{\let\PY@bf=\textbf\def\PY@tc##1{\textcolor[rgb]{0.67,0.36,0.12}{##1}}}
\@namedef{PY@tok@sr}{\def\PY@tc##1{\textcolor[rgb]{0.64,0.35,0.47}{##1}}}
\@namedef{PY@tok@ss}{\def\PY@tc##1{\textcolor[rgb]{0.10,0.09,0.49}{##1}}}
\@namedef{PY@tok@sx}{\def\PY@tc##1{\textcolor[rgb]{0.00,0.50,0.00}{##1}}}
\@namedef{PY@tok@m}{\def\PY@tc##1{\textcolor[rgb]{0.40,0.40,0.40}{##1}}}
\@namedef{PY@tok@gh}{\let\PY@bf=\textbf\def\PY@tc##1{\textcolor[rgb]{0.00,0.00,0.50}{##1}}}
\@namedef{PY@tok@gu}{\let\PY@bf=\textbf\def\PY@tc##1{\textcolor[rgb]{0.50,0.00,0.50}{##1}}}
\@namedef{PY@tok@gd}{\def\PY@tc##1{\textcolor[rgb]{0.63,0.00,0.00}{##1}}}
\@namedef{PY@tok@gi}{\def\PY@tc##1{\textcolor[rgb]{0.00,0.52,0.00}{##1}}}
\@namedef{PY@tok@gr}{\def\PY@tc##1{\textcolor[rgb]{0.89,0.00,0.00}{##1}}}
\@namedef{PY@tok@ge}{\let\PY@it=\textit}
\@namedef{PY@tok@gs}{\let\PY@bf=\textbf}
\@namedef{PY@tok@ges}{\let\PY@bf=\textbf\let\PY@it=\textit}
\@namedef{PY@tok@gp}{\let\PY@bf=\textbf\def\PY@tc##1{\textcolor[rgb]{0.00,0.00,0.50}{##1}}}
\@namedef{PY@tok@go}{\def\PY@tc##1{\textcolor[rgb]{0.44,0.44,0.44}{##1}}}
\@namedef{PY@tok@gt}{\def\PY@tc##1{\textcolor[rgb]{0.00,0.27,0.87}{##1}}}
\@namedef{PY@tok@err}{\def\PY@bc##1{{\setlength{\fboxsep}{\string -\fboxrule}\fcolorbox[rgb]{1.00,0.00,0.00}{1,1,1}{\strut ##1}}}}
\@namedef{PY@tok@kc}{\let\PY@bf=\textbf\def\PY@tc##1{\textcolor[rgb]{0.00,0.50,0.00}{##1}}}
\@namedef{PY@tok@kd}{\let\PY@bf=\textbf\def\PY@tc##1{\textcolor[rgb]{0.00,0.50,0.00}{##1}}}
\@namedef{PY@tok@kn}{\let\PY@bf=\textbf\def\PY@tc##1{\textcolor[rgb]{0.00,0.50,0.00}{##1}}}
\@namedef{PY@tok@kr}{\let\PY@bf=\textbf\def\PY@tc##1{\textcolor[rgb]{0.00,0.50,0.00}{##1}}}
\@namedef{PY@tok@bp}{\def\PY@tc##1{\textcolor[rgb]{0.00,0.50,0.00}{##1}}}
\@namedef{PY@tok@fm}{\def\PY@tc##1{\textcolor[rgb]{0.00,0.00,1.00}{##1}}}
\@namedef{PY@tok@vc}{\def\PY@tc##1{\textcolor[rgb]{0.10,0.09,0.49}{##1}}}
\@namedef{PY@tok@vg}{\def\PY@tc##1{\textcolor[rgb]{0.10,0.09,0.49}{##1}}}
\@namedef{PY@tok@vi}{\def\PY@tc##1{\textcolor[rgb]{0.10,0.09,0.49}{##1}}}
\@namedef{PY@tok@vm}{\def\PY@tc##1{\textcolor[rgb]{0.10,0.09,0.49}{##1}}}
\@namedef{PY@tok@sa}{\def\PY@tc##1{\textcolor[rgb]{0.73,0.13,0.13}{##1}}}
\@namedef{PY@tok@sb}{\def\PY@tc##1{\textcolor[rgb]{0.73,0.13,0.13}{##1}}}
\@namedef{PY@tok@sc}{\def\PY@tc##1{\textcolor[rgb]{0.73,0.13,0.13}{##1}}}
\@namedef{PY@tok@dl}{\def\PY@tc##1{\textcolor[rgb]{0.73,0.13,0.13}{##1}}}
\@namedef{PY@tok@s2}{\def\PY@tc##1{\textcolor[rgb]{0.73,0.13,0.13}{##1}}}
\@namedef{PY@tok@sh}{\def\PY@tc##1{\textcolor[rgb]{0.73,0.13,0.13}{##1}}}
\@namedef{PY@tok@s1}{\def\PY@tc##1{\textcolor[rgb]{0.73,0.13,0.13}{##1}}}
\@namedef{PY@tok@mb}{\def\PY@tc##1{\textcolor[rgb]{0.40,0.40,0.40}{##1}}}
\@namedef{PY@tok@mf}{\def\PY@tc##1{\textcolor[rgb]{0.40,0.40,0.40}{##1}}}
\@namedef{PY@tok@mh}{\def\PY@tc##1{\textcolor[rgb]{0.40,0.40,0.40}{##1}}}
\@namedef{PY@tok@mi}{\def\PY@tc##1{\textcolor[rgb]{0.40,0.40,0.40}{##1}}}
\@namedef{PY@tok@il}{\def\PY@tc##1{\textcolor[rgb]{0.40,0.40,0.40}{##1}}}
\@namedef{PY@tok@mo}{\def\PY@tc##1{\textcolor[rgb]{0.40,0.40,0.40}{##1}}}
\@namedef{PY@tok@ch}{\let\PY@it=\textit\def\PY@tc##1{\textcolor[rgb]{0.24,0.48,0.48}{##1}}}
\@namedef{PY@tok@cm}{\let\PY@it=\textit\def\PY@tc##1{\textcolor[rgb]{0.24,0.48,0.48}{##1}}}
\@namedef{PY@tok@cpf}{\let\PY@it=\textit\def\PY@tc##1{\textcolor[rgb]{0.24,0.48,0.48}{##1}}}
\@namedef{PY@tok@c1}{\let\PY@it=\textit\def\PY@tc##1{\textcolor[rgb]{0.24,0.48,0.48}{##1}}}
\@namedef{PY@tok@cs}{\let\PY@it=\textit\def\PY@tc##1{\textcolor[rgb]{0.24,0.48,0.48}{##1}}}

\def\PYZbs{\char`\\}
\def\PYZus{\char`\_}
\def\PYZob{\char`\{}
\def\PYZcb{\char`\}}
\def\PYZca{\char`\^}
\def\PYZam{\char`\&}
\def\PYZlt{\char`\<}
\def\PYZgt{\char`\>}
\def\PYZsh{\char`\#}
\def\PYZpc{\char`\%}
\def\PYZdl{\char`\$}
\def\PYZhy{\char`\-}
\def\PYZsq{\char`\'}
\def\PYZdq{\char`\"}
\def\PYZti{\char`\~}
% for compatibility with earlier versions
\def\PYZat{@}
\def\PYZlb{[}
\def\PYZrb{]}
\makeatother


    % For linebreaks inside Verbatim environment from package fancyvrb.
    \makeatletter
        \newbox\Wrappedcontinuationbox
        \newbox\Wrappedvisiblespacebox
        \newcommand*\Wrappedvisiblespace {\textcolor{red}{\textvisiblespace}}
        \newcommand*\Wrappedcontinuationsymbol {\textcolor{red}{\llap{\tiny$\m@th\hookrightarrow$}}}
        \newcommand*\Wrappedcontinuationindent {3ex }
        \newcommand*\Wrappedafterbreak {\kern\Wrappedcontinuationindent\copy\Wrappedcontinuationbox}
        % Take advantage of the already applied Pygments mark-up to insert
        % potential linebreaks for TeX processing.
        %        {, <, #, %, $, ' and ": go to next line.
        %        _, }, ^, &, >, - and ~: stay at end of broken line.
        % Use of \textquotesingle for straight quote.
        \newcommand*\Wrappedbreaksatspecials {%
            \def\PYGZus{\discretionary{\char`\_}{\Wrappedafterbreak}{\char`\_}}%
            \def\PYGZob{\discretionary{}{\Wrappedafterbreak\char`\{}{\char`\{}}%
            \def\PYGZcb{\discretionary{\char`\}}{\Wrappedafterbreak}{\char`\}}}%
            \def\PYGZca{\discretionary{\char`\^}{\Wrappedafterbreak}{\char`\^}}%
            \def\PYGZam{\discretionary{\char`\&}{\Wrappedafterbreak}{\char`\&}}%
            \def\PYGZlt{\discretionary{}{\Wrappedafterbreak\char`\<}{\char`\<}}%
            \def\PYGZgt{\discretionary{\char`\>}{\Wrappedafterbreak}{\char`\>}}%
            \def\PYGZsh{\discretionary{}{\Wrappedafterbreak\char`\#}{\char`\#}}%
            \def\PYGZpc{\discretionary{}{\Wrappedafterbreak\char`\%}{\char`\%}}%
            \def\PYGZdl{\discretionary{}{\Wrappedafterbreak\char`\$}{\char`\$}}%
            \def\PYGZhy{\discretionary{\char`\-}{\Wrappedafterbreak}{\char`\-}}%
            \def\PYGZsq{\discretionary{}{\Wrappedafterbreak\textquotesingle}{\textquotesingle}}%
            \def\PYGZdq{\discretionary{}{\Wrappedafterbreak\char`\"}{\char`\"}}%
            \def\PYGZti{\discretionary{\char`\~}{\Wrappedafterbreak}{\char`\~}}%
        }
        % Some characters . , ; ? ! / are not pygmentized.
        % This macro makes them "active" and they will insert potential linebreaks
        \newcommand*\Wrappedbreaksatpunct {%
            \lccode`\~`\.\lowercase{\def~}{\discretionary{\hbox{\char`\.}}{\Wrappedafterbreak}{\hbox{\char`\.}}}%
            \lccode`\~`\,\lowercase{\def~}{\discretionary{\hbox{\char`\,}}{\Wrappedafterbreak}{\hbox{\char`\,}}}%
            \lccode`\~`\;\lowercase{\def~}{\discretionary{\hbox{\char`\;}}{\Wrappedafterbreak}{\hbox{\char`\;}}}%
            \lccode`\~`\:\lowercase{\def~}{\discretionary{\hbox{\char`\:}}{\Wrappedafterbreak}{\hbox{\char`\:}}}%
            \lccode`\~`\?\lowercase{\def~}{\discretionary{\hbox{\char`\?}}{\Wrappedafterbreak}{\hbox{\char`\?}}}%
            \lccode`\~`\!\lowercase{\def~}{\discretionary{\hbox{\char`\!}}{\Wrappedafterbreak}{\hbox{\char`\!}}}%
            \lccode`\~`\/\lowercase{\def~}{\discretionary{\hbox{\char`\/}}{\Wrappedafterbreak}{\hbox{\char`\/}}}%
            \catcode`\.\active
            \catcode`\,\active
            \catcode`\;\active
            \catcode`\:\active
            \catcode`\?\active
            \catcode`\!\active
            \catcode`\/\active
            \lccode`\~`\~
        }
    \makeatother

    \let\OriginalVerbatim=\Verbatim
    \makeatletter
    \renewcommand{\Verbatim}[1][1]{%
        %\parskip\z@skip
        \sbox\Wrappedcontinuationbox {\Wrappedcontinuationsymbol}%
        \sbox\Wrappedvisiblespacebox {\FV@SetupFont\Wrappedvisiblespace}%
        \def\FancyVerbFormatLine ##1{\hsize\linewidth
            \vtop{\raggedright\hyphenpenalty\z@\exhyphenpenalty\z@
                \doublehyphendemerits\z@\finalhyphendemerits\z@
                \strut ##1\strut}%
        }%
        % If the linebreak is at a space, the latter will be displayed as visible
        % space at end of first line, and a continuation symbol starts next line.
        % Stretch/shrink are however usually zero for typewriter font.
        \def\FV@Space {%
            \nobreak\hskip\z@ plus\fontdimen3\font minus\fontdimen4\font
            \discretionary{\copy\Wrappedvisiblespacebox}{\Wrappedafterbreak}
            {\kern\fontdimen2\font}%
        }%

        % Allow breaks at special characters using \PYG... macros.
        \Wrappedbreaksatspecials
        % Breaks at punctuation characters . , ; ? ! and / need catcode=\active
        \OriginalVerbatim[#1,codes*=\Wrappedbreaksatpunct]%
    }
    \makeatother

    % Exact colors from NB
    \definecolor{incolor}{HTML}{303F9F}
    \definecolor{outcolor}{HTML}{D84315}
    \definecolor{cellborder}{HTML}{CFCFCF}
    \definecolor{cellbackground}{HTML}{F7F7F7}

    % prompt
    \makeatletter
    \newcommand{\boxspacing}{\kern\kvtcb@left@rule\kern\kvtcb@boxsep}
    \makeatother
    \newcommand{\prompt}[4]{
        {\ttfamily\llap{{\color{#2}[#3]:\hspace{3pt}#4}}\vspace{-\baselineskip}}
    }
    

    
    % Prevent overflowing lines due to hard-to-break entities
    \sloppy
    % Setup hyperref package
    \hypersetup{
      breaklinks=true,  % so long urls are correctly broken across lines
      colorlinks=true,
      urlcolor=urlcolor,
      linkcolor=linkcolor,
      citecolor=citecolor,
      }
    % Slightly bigger margins than the latex defaults
    
    \geometry{verbose,tmargin=1in,bmargin=1in,lmargin=1in,rmargin=1in}
    
    

\begin{document}
    
    \maketitle
    
    

    
    \hypertarget{proposition}{%
\section{Proposition}\label{proposition}}

    If five points \(P_1, \dots, P_5\) in a \(V\)-configuration satisfy \[
  \delta_1(P_1, P_2, P_4) = 
  \bar{\delta}_1(P_1, P_2, P_3) =
  \bar{\delta}_1(P_1, P_4, P_5) = 0
\] then, in \(\Lambda \bigl( \Phi(P_1, \dotsc, P_5)\bigr)\) there is a
cubic curve with \(7\) eigenpoints with the following three alignments:
\[
 (P_1, P_2, P_3) \,, \quad (P_1, P_4, P_5) \,, \quad \text{and} \quad (P_1, P_6, P_7) \,.
\] No choices of \(P_1, \dots, P_5\) allow one to obtain further
alignments of the \(7\) eigenpoints.

    \begin{tcolorbox}[breakable, size=fbox, boxrule=1pt, pad at break*=1mm,colback=cellbackground, colframe=cellborder]
\prompt{In}{incolor}{1}{\boxspacing}
\begin{Verbatim}[commandchars=\\\{\}]
\PY{n}{load}\PY{p}{(}\PY{l+s+s2}{\PYZdq{}}\PY{l+s+s2}{basic\PYZus{}functions.sage}\PY{l+s+s2}{\PYZdq{}}\PY{p}{)}
\end{Verbatim}
\end{tcolorbox}

    We define three points \(P_1, \dotsc, P_5\) so that they form a
\(V\)-configuration

    \begin{tcolorbox}[breakable, size=fbox, boxrule=1pt, pad at break*=1mm,colback=cellbackground, colframe=cellborder]
\prompt{In}{incolor}{2}{\boxspacing}
\begin{Verbatim}[commandchars=\\\{\}]
\PY{n}{P1} \PY{o}{=} \PY{n}{vector}\PY{p}{(}\PY{p}{(}\PY{l+m+mi}{1}\PY{p}{,} \PY{l+m+mi}{0}\PY{p}{,} \PY{l+m+mi}{0}\PY{p}{)}\PY{p}{)}
\PY{n}{P2} \PY{o}{=} \PY{n}{vector}\PY{p}{(}\PY{n}{S}\PY{p}{,} \PY{p}{(}\PY{n}{A2}\PY{p}{,} \PY{n}{B2}\PY{p}{,} \PY{n}{C2}\PY{p}{)}\PY{p}{)}
\PY{n}{P3} \PY{o}{=} \PY{n}{u1}\PY{o}{*}\PY{n}{P1}\PY{o}{+}\PY{n}{u2}\PY{o}{*}\PY{n}{P2}
\PY{n}{P4} \PY{o}{=} \PY{n}{vector}\PY{p}{(}\PY{n}{S}\PY{p}{,} \PY{p}{(}\PY{n}{A4}\PY{p}{,} \PY{n}{B4}\PY{p}{,} \PY{n}{C4}\PY{p}{)}\PY{p}{)}
\PY{n}{P5} \PY{o}{=} \PY{n}{v1}\PY{o}{*}\PY{n}{P1}\PY{o}{+}\PY{n}{v2}\PY{o}{*}\PY{n}{P4}
\end{Verbatim}
\end{tcolorbox}

    We want to impose \(\delta_1(P_1, P_2, P_4) = 0\). We check that
\(\delta_1(P_1, P_2, P_4) = 0\) is \(B_2 B_4 + C_2 C_4\):

    \begin{tcolorbox}[breakable, size=fbox, boxrule=1pt, pad at break*=1mm,colback=cellbackground, colframe=cellborder]
\prompt{In}{incolor}{3}{\boxspacing}
\begin{Verbatim}[commandchars=\\\{\}]
\PY{k}{assert}\PY{p}{(}\PY{n}{delta1}\PY{p}{(}\PY{n}{P1}\PY{p}{,} \PY{n}{P2}\PY{p}{,} \PY{n}{P4}\PY{p}{)} \PY{o}{==} \PY{n}{B2}\PY{o}{*}\PY{n}{B4}\PY{o}{+}\PY{n}{C2}\PY{o}{*}\PY{n}{C4}\PY{p}{)}
\end{Verbatim}
\end{tcolorbox}

    It is not possible to have \(C_2 = C_4 = 0\).

If \(C_2 = 0\), necessarily \(B_2 \neq 0\).

If \(C_2 \neq 0\) and \(B_4 = 0\), we get \(C_4 = 0\), but
\(P_4 \neq P_1\), so if \(C_2 \neq 0\), we can assume \(B_4 \neq 0\).

Hence, the other situation to consider is \(C_2 = B_4 = 0\), in which
case the points are \(p_1 = (1, 0, 0)\), \(p_2 = (A_2, B_2, 0)\),
\(p_4 = (A_4, 0, C_4)\).

    \hypertarget{we-assume-c_2-neq-0}{%
\subsection{\texorpdfstring{We assume
\(C_2 \neq 0\)}{We assume C\_2 \textbackslash neq 0}}\label{we-assume-c_2-neq-0}}

    \begin{tcolorbox}[breakable, size=fbox, boxrule=1pt, pad at break*=1mm,colback=cellbackground, colframe=cellborder]
\prompt{In}{incolor}{4}{\boxspacing}
\begin{Verbatim}[commandchars=\\\{\}]
\PY{n}{st1} \PY{o}{=} \PY{p}{\PYZob{}}\PY{n}{A4}\PY{p}{:} \PY{n}{A4}\PY{o}{*}\PY{n}{C2}\PY{p}{,} \PY{n}{B4}\PY{p}{:} \PY{n}{B4}\PY{o}{*}\PY{n}{C2}\PY{p}{,} \PY{n}{C4}\PY{p}{:}\PY{o}{\PYZhy{}}\PY{n}{B2}\PY{o}{*}\PY{n}{B4}\PY{p}{\PYZcb{}}
\PY{n}{p1} \PY{o}{=} \PY{n}{P1}\PY{o}{.}\PY{n}{subs}\PY{p}{(}\PY{n}{st1}\PY{p}{)}
\PY{n}{p2} \PY{o}{=} \PY{n}{P2}\PY{o}{.}\PY{n}{subs}\PY{p}{(}\PY{n}{st1}\PY{p}{)}
\PY{n}{p4} \PY{o}{=} \PY{n}{P4}\PY{o}{.}\PY{n}{subs}\PY{p}{(}\PY{n}{st1}\PY{p}{)}
\PY{k}{assert}\PY{p}{(}\PY{n}{delta1}\PY{p}{(}\PY{n}{p1}\PY{p}{,} \PY{n}{p2}\PY{p}{,} \PY{n}{p4}\PY{p}{)} \PY{o}{==} \PY{l+m+mi}{0}\PY{p}{)}
\end{Verbatim}
\end{tcolorbox}

    We define \(p_3\) and \(p_5\) according to the known formulas:

    \begin{tcolorbox}[breakable, size=fbox, boxrule=1pt, pad at break*=1mm,colback=cellbackground, colframe=cellborder]
\prompt{In}{incolor}{5}{\boxspacing}
\begin{Verbatim}[commandchars=\\\{\}]
\PY{n}{p3} \PY{o}{=} \PY{p}{(}\PY{n}{scalar\PYZus{}product}\PY{p}{(}\PY{n}{p1}\PY{p}{,} \PY{n}{p2}\PY{p}{)}\PY{o}{\PYZca{}}\PY{l+m+mi}{2}\PY{o}{+}\PY{n}{scalar\PYZus{}product}\PY{p}{(}\PY{n}{p1}\PY{p}{,} \PY{n}{p1}\PY{p}{)}\PY{o}{*}\PY{n}{scalar\PYZus{}product}\PY{p}{(}\PY{n}{p2}\PY{p}{,} \PY{n}{p2}\PY{p}{)}\PY{p}{)}\PY{o}{*}\PY{n}{p1}\PY{o}{\PYZhy{}}\PY{l+m+mi}{2}\PY{o}{*}\PY{p}{(}\PY{n}{scalar\PYZus{}product}\PY{p}{(}\PY{n}{p1}\PY{p}{,} \PY{n}{p1}\PY{p}{)}\PY{o}{*}\PY{n}{scalar\PYZus{}product}\PY{p}{(}\PY{n}{p1}\PY{p}{,} \PY{n}{p2}\PY{p}{)}\PY{p}{)}\PY{o}{*}\PY{n}{p2}
\PY{n}{p5} \PY{o}{=} \PY{p}{(}\PY{n}{scalar\PYZus{}product}\PY{p}{(}\PY{n}{p1}\PY{p}{,} \PY{n}{p4}\PY{p}{)}\PY{o}{\PYZca{}}\PY{l+m+mi}{2}\PY{o}{+}\PY{n}{scalar\PYZus{}product}\PY{p}{(}\PY{n}{p1}\PY{p}{,} \PY{n}{p1}\PY{p}{)}\PY{o}{*}\PY{n}{scalar\PYZus{}product}\PY{p}{(}\PY{n}{p4}\PY{p}{,} \PY{n}{p4}\PY{p}{)}\PY{p}{)}\PY{o}{*}\PY{n}{p1}\PY{o}{\PYZhy{}}\PY{l+m+mi}{2}\PY{o}{*}\PY{p}{(}\PY{n}{scalar\PYZus{}product}\PY{p}{(}\PY{n}{p1}\PY{p}{,} \PY{n}{p1}\PY{p}{)}\PY{o}{*}\PY{n}{scalar\PYZus{}product}\PY{p}{(}\PY{n}{p1}\PY{p}{,} \PY{n}{p4}\PY{p}{)}\PY{p}{)}\PY{o}{*}\PY{n}{p4}
\end{Verbatim}
\end{tcolorbox}

    The entries of \(p_5\) can be divided by \(B_4\) (which is, as said, not
0)

    \begin{tcolorbox}[breakable, size=fbox, boxrule=1pt, pad at break*=1mm,colback=cellbackground, colframe=cellborder]
\prompt{In}{incolor}{6}{\boxspacing}
\begin{Verbatim}[commandchars=\\\{\}]
\PY{k}{assert}\PY{p}{(}\PY{n}{gcd}\PY{p}{(}\PY{n+nb}{list}\PY{p}{(}\PY{n}{p5}\PY{p}{)}\PY{p}{)} \PY{o}{==} \PY{n}{B4}\PY{p}{)}
\end{Verbatim}
\end{tcolorbox}

    \begin{tcolorbox}[breakable, size=fbox, boxrule=1pt, pad at break*=1mm,colback=cellbackground, colframe=cellborder]
\prompt{In}{incolor}{7}{\boxspacing}
\begin{Verbatim}[commandchars=\\\{\}]
\PY{n}{p5} \PY{o}{=} \PY{n}{vector}\PY{p}{(}\PY{n}{S}\PY{p}{,} \PY{p}{[}\PY{n}{px}\PY{o}{.}\PY{n}{quo\PYZus{}rem}\PY{p}{(}\PY{n}{B4}\PY{p}{)}\PY{p}{[}\PY{l+m+mi}{0}\PY{p}{]} \PY{k}{for} \PY{n}{px} \PY{o+ow}{in} \PY{n+nb}{list}\PY{p}{(}\PY{n}{p5}\PY{p}{)}\PY{p}{]}\PY{p}{)}
\end{Verbatim}
\end{tcolorbox}

    \begin{tcolorbox}[breakable, size=fbox, boxrule=1pt, pad at break*=1mm,colback=cellbackground, colframe=cellborder]
\prompt{In}{incolor}{8}{\boxspacing}
\begin{Verbatim}[commandchars=\\\{\}]
\PY{k}{assert}\PY{p}{(}\PY{n}{delta1b}\PY{p}{(}\PY{n}{p1}\PY{p}{,} \PY{n}{p2}\PY{p}{,} \PY{n}{p3}\PY{p}{)} \PY{o}{==} \PY{l+m+mi}{0}\PY{p}{)}
\PY{k}{assert}\PY{p}{(}\PY{n}{delta1b}\PY{p}{(}\PY{n}{p1}\PY{p}{,} \PY{n}{p4}\PY{p}{,} \PY{n}{p5}\PY{p}{)} \PY{o}{==} \PY{l+m+mi}{0}\PY{p}{)}
\PY{c+c1}{\PYZsh{}\PYZsh{} Incidentally, delta2 is also 0:}
\PY{k}{assert}\PY{p}{(}\PY{n}{delta2}\PY{p}{(}\PY{n}{p1}\PY{p}{,} \PY{n}{p2}\PY{p}{,} \PY{n}{p3}\PY{p}{,} \PY{n}{p4}\PY{p}{,} \PY{n}{p5}\PY{p}{)} \PY{o}{==} \PY{l+m+mi}{0}\PY{p}{)}
\end{Verbatim}
\end{tcolorbox}

    An example shows that, in general, \(p_6\) and \(p_7\) are not aligned
with \(p_1\). Here is an example.

    \begin{tcolorbox}[breakable, size=fbox, boxrule=1pt, pad at break*=1mm,colback=cellbackground, colframe=cellborder]
\prompt{In}{incolor}{9}{\boxspacing}
\begin{Verbatim}[commandchars=\\\{\}]
\PY{n}{ss3} \PY{o}{=} \PY{p}{\PYZob{}}\PY{n}{A2}\PY{p}{:}\PY{l+m+mi}{1}\PY{p}{,} \PY{n}{B2}\PY{p}{:}\PY{o}{\PYZhy{}}\PY{l+m+mi}{5}\PY{p}{,} \PY{n}{C2}\PY{p}{:}\PY{o}{\PYZhy{}}\PY{l+m+mi}{3}\PY{p}{,} \PY{n}{A4}\PY{p}{:}\PY{l+m+mi}{7}\PY{p}{,} \PY{n}{B4}\PY{p}{:}\PY{o}{\PYZhy{}}\PY{l+m+mi}{5}\PY{p}{\PYZcb{}}
\PY{n}{pp1} \PY{o}{=} \PY{n}{p1}\PY{o}{.}\PY{n}{subs}\PY{p}{(}\PY{n}{ss3}\PY{p}{)}
\PY{n}{pp2} \PY{o}{=} \PY{n}{p2}\PY{o}{.}\PY{n}{subs}\PY{p}{(}\PY{n}{ss3}\PY{p}{)}
\PY{n}{pp3} \PY{o}{=} \PY{n}{p3}\PY{o}{.}\PY{n}{subs}\PY{p}{(}\PY{n}{ss3}\PY{p}{)}
\PY{n}{pp4} \PY{o}{=} \PY{n}{p4}\PY{o}{.}\PY{n}{subs}\PY{p}{(}\PY{n}{ss3}\PY{p}{)}
\PY{n}{pp5} \PY{o}{=} \PY{n}{p5}\PY{o}{.}\PY{n}{subs}\PY{p}{(}\PY{n}{ss3}\PY{p}{)}
\PY{n}{cb} \PY{o}{=} \PY{n}{cubic\PYZus{}from\PYZus{}matrix}\PY{p}{(}
    \PY{n}{condition\PYZus{}matrix}\PY{p}{(}
        \PY{p}{[}\PY{n}{pp1}\PY{p}{,} \PY{n}{pp2}\PY{p}{,} \PY{n}{pp3}\PY{p}{,} \PY{n}{pp4}\PY{p}{,} \PY{n}{pp5}\PY{p}{]}\PY{p}{,}
         \PY{n}{S}\PY{p}{,} 
        \PY{n}{standard}\PY{o}{=}\PY{l+s+s2}{\PYZdq{}}\PY{l+s+s2}{all}\PY{l+s+s2}{\PYZdq{}}
    \PY{p}{)}\PY{o}{.}\PY{n}{stack}\PY{p}{(}
        \PY{n}{matrix}\PY{p}{(}
            \PY{p}{[}
                \PY{p}{[}\PY{l+m+mi}{2}\PY{p}{,} \PY{l+m+mi}{3}\PY{p}{,} \PY{l+m+mi}{4}\PY{p}{,} \PY{l+m+mi}{5}\PY{p}{,} \PY{l+m+mi}{6}\PY{p}{,} \PY{l+m+mi}{7}\PY{p}{,} \PY{l+m+mi}{8}\PY{p}{,} \PY{l+m+mi}{9}\PY{p}{,} \PY{l+m+mi}{1}\PY{p}{,} \PY{l+m+mi}{2}\PY{p}{]}
            \PY{p}{]}
        \PY{p}{)}
    \PY{p}{)}
\PY{p}{)}
\end{Verbatim}
\end{tcolorbox}

    NOW WE WANT TO SEE WHAT HAPPENS IF WE IMPOSE THE ALIGNMENT \(p_1\),
\(p_6\), \(p_7\).

We start with \(p_1\), \(p_2\), \(p_3\), \(p_4\), \(p_5\) as above, such
that \(\delta_1(p_1, p_2, p_4)\), \(\bar{\delta}_1(p_1, p_2, p_3)\) and
\(\bar{\delta}_1(p_1, p_4, p_5)\) are \(0\)

    The matrix \(\Phi(p_1, p_2, p_3, p_4, p_5)\) has rank 8.

    \begin{tcolorbox}[breakable, size=fbox, boxrule=1pt, pad at break*=1mm,colback=cellbackground, colframe=cellborder]
\prompt{In}{incolor}{10}{\boxspacing}
\begin{Verbatim}[commandchars=\\\{\}]
\PY{n}{M} \PY{o}{=} \PY{n}{condition\PYZus{}matrix}\PY{p}{(}\PY{p}{[}\PY{n}{p1}\PY{p}{,} \PY{n}{p2}\PY{p}{,} \PY{n}{p3}\PY{p}{,} \PY{n}{p4}\PY{p}{,} \PY{n}{p5}\PY{p}{]}\PY{p}{,} \PY{n}{S}\PY{p}{,} \PY{n}{standard}\PY{o}{=}\PY{l+s+s2}{\PYZdq{}}\PY{l+s+s2}{all}\PY{l+s+s2}{\PYZdq{}}\PY{p}{)}
\PY{k}{assert}\PY{p}{(}\PY{n}{M}\PY{o}{.}\PY{n}{rank}\PY{p}{(}\PY{p}{)} \PY{o}{==} \PY{l+m+mi}{8}\PY{p}{)}
\end{Verbatim}
\end{tcolorbox}

    We select 8 linearly independent rows:

    \begin{tcolorbox}[breakable, size=fbox, boxrule=1pt, pad at break*=1mm,colback=cellbackground, colframe=cellborder]
\prompt{In}{incolor}{11}{\boxspacing}
\begin{Verbatim}[commandchars=\\\{\}]
\PY{n}{mm} \PY{o}{=} \PY{n}{M}\PY{o}{.}\PY{n}{matrix\PYZus{}from\PYZus{}rows}\PY{p}{(}\PY{p}{[}\PY{l+m+mi}{0}\PY{p}{,} \PY{l+m+mi}{1}\PY{p}{,} \PY{l+m+mi}{4}\PY{p}{,} \PY{l+m+mi}{5}\PY{p}{,} \PY{l+m+mi}{7}\PY{p}{,} \PY{l+m+mi}{8}\PY{p}{,} \PY{l+m+mi}{12}\PY{p}{,} \PY{l+m+mi}{14}\PY{p}{]}\PY{p}{)}
\end{Verbatim}
\end{tcolorbox}

    \begin{tcolorbox}[breakable, size=fbox, boxrule=1pt, pad at break*=1mm,colback=cellbackground, colframe=cellborder]
\prompt{In}{incolor}{12}{\boxspacing}
\begin{Verbatim}[commandchars=\\\{\}]
\PY{c+c1}{\PYZsh{} in general, mm has rank 8}
\PY{k}{assert}\PY{p}{(}\PY{n}{mm}\PY{o}{.}\PY{n}{rank}\PY{p}{(}\PY{p}{)} \PY{o}{==} \PY{l+m+mi}{8}\PY{p}{)}  
\end{Verbatim}
\end{tcolorbox}

    \begin{tcolorbox}[breakable, size=fbox, boxrule=1pt, pad at break*=1mm,colback=cellbackground, colframe=cellborder]
\prompt{In}{incolor}{13}{\boxspacing}
\begin{Verbatim}[commandchars=\\\{\}]
\PY{c+c1}{\PYZsh{} let us see when the rank is not 8:}
\PY{n}{hj} \PY{o}{=} \PY{n}{S}\PY{o}{.}\PY{n}{ideal}\PY{p}{(}\PY{n}{mm}\PY{o}{.}\PY{n}{minors}\PY{p}{(}\PY{l+m+mi}{8}\PY{p}{)}\PY{p}{)}
\PY{n}{hj} \PY{o}{=} \PY{n}{hj}\PY{o}{.}\PY{n}{saturation}\PY{p}{(}\PY{n}{S}\PY{o}{.}\PY{n}{ideal}\PY{p}{(}\PY{n}{matrix}\PY{p}{(}\PY{p}{[}\PY{n}{p3}\PY{p}{,} \PY{n}{p5}\PY{p}{]}\PY{p}{)}\PY{o}{.}\PY{n}{minors}\PY{p}{(}\PY{l+m+mi}{2}\PY{p}{)}\PY{p}{)}\PY{p}{)}\PY{p}{[}\PY{l+m+mi}{0}\PY{p}{]}
\PY{n}{hj} \PY{o}{=} \PY{n}{hj}\PY{o}{.}\PY{n}{saturation}\PY{p}{(}\PY{n}{S}\PY{o}{.}\PY{n}{ideal}\PY{p}{(}\PY{n}{matrix}\PY{p}{(}\PY{p}{[}\PY{n}{p1}\PY{p}{,} \PY{n}{p3}\PY{p}{]}\PY{p}{)}\PY{o}{.}\PY{n}{minors}\PY{p}{(}\PY{l+m+mi}{2}\PY{p}{)}\PY{p}{)}\PY{p}{)}\PY{p}{[}\PY{l+m+mi}{0}\PY{p}{]}
\PY{n}{hj} \PY{o}{=} \PY{n}{hj}\PY{o}{.}\PY{n}{saturation}\PY{p}{(}\PY{n}{S}\PY{o}{.}\PY{n}{ideal}\PY{p}{(}\PY{n}{matrix}\PY{p}{(}\PY{p}{[}\PY{n}{p1}\PY{p}{,} \PY{n}{p5}\PY{p}{]}\PY{p}{)}\PY{o}{.}\PY{n}{minors}\PY{p}{(}\PY{l+m+mi}{2}\PY{p}{)}\PY{p}{)}\PY{p}{)}\PY{p}{[}\PY{l+m+mi}{0}\PY{p}{]}
\PY{n}{hj} \PY{o}{=} \PY{n}{hj}\PY{o}{.}\PY{n}{saturation}\PY{p}{(}\PY{n}{S}\PY{o}{.}\PY{n}{ideal}\PY{p}{(}\PY{n}{matrix}\PY{p}{(}\PY{p}{[}\PY{n}{p2}\PY{p}{,} \PY{n}{p3}\PY{p}{]}\PY{p}{)}\PY{o}{.}\PY{n}{minors}\PY{p}{(}\PY{l+m+mi}{2}\PY{p}{)}\PY{p}{)}\PY{p}{)}\PY{p}{[}\PY{l+m+mi}{0}\PY{p}{]}
\PY{c+c1}{\PYZsh{} }
\PY{c+c1}{\PYZsh{} it holds: hj = (1)}
\PY{k}{assert}\PY{p}{(}\PY{n}{hj} \PY{o}{==} \PY{n}{S}\PY{o}{.}\PY{n}{ideal}\PY{p}{(}\PY{l+m+mi}{1}\PY{p}{)}\PY{p}{)}
\PY{c+c1}{\PYZsh{} hence the order 8 minor mm has always rank 8.}
\end{Verbatim}
\end{tcolorbox}

    The cubics of \(\mathbb{P}^9\) that have \(p_1, \dotsc, p_5\) as
eigenpoints are a line \(L\) of \(\mathbb{P}^9\).

We want to see if, among the points of this line, we can find some which
give cubic curves with the eigenpoints \(p_1\), \(p_6\), \(p_7\)
collinear. Here is the way in which we procede. We fix two 10-components
vectors like: \((1, 2, 5, 6, 0, 2, 3, 4, 9, 11)\) and
\((-1, 3, 6, 5, 0, 1, 3, 7, 9, -5)\) and we consider the following
matrices: mmA and mmB (we put an index ``1'' because later we shall
repeat the computation):

    \begin{tcolorbox}[breakable, size=fbox, boxrule=1pt, pad at break*=1mm,colback=cellbackground, colframe=cellborder]
\prompt{In}{incolor}{14}{\boxspacing}
\begin{Verbatim}[commandchars=\\\{\}]
\PY{n}{mmA\PYZus{}1} \PY{o}{=} \PY{n}{mm}\PY{o}{.}\PY{n}{stack}\PY{p}{(}\PY{n}{matrix}\PY{p}{(}\PY{p}{[}\PY{l+m+mi}{1}\PY{p}{,} \PY{l+m+mi}{2}\PY{p}{,} \PY{l+m+mi}{5}\PY{p}{,} \PY{l+m+mi}{6}\PY{p}{,} \PY{l+m+mi}{0}\PY{p}{,} \PY{l+m+mi}{2}\PY{p}{,} \PY{l+m+mi}{3}\PY{p}{,} \PY{l+m+mi}{4}\PY{p}{,} \PY{l+m+mi}{9}\PY{p}{,} \PY{l+m+mi}{11}\PY{p}{]}\PY{p}{)}\PY{p}{)}
\PY{n}{mmB\PYZus{}1} \PY{o}{=} \PY{n}{mm}\PY{o}{.}\PY{n}{stack}\PY{p}{(}\PY{n}{matrix}\PY{p}{(}\PY{p}{[}\PY{o}{\PYZhy{}}\PY{l+m+mi}{1}\PY{p}{,} \PY{l+m+mi}{3}\PY{p}{,} \PY{l+m+mi}{6}\PY{p}{,} \PY{l+m+mi}{5}\PY{p}{,} \PY{l+m+mi}{0}\PY{p}{,} \PY{l+m+mi}{1}\PY{p}{,} \PY{l+m+mi}{3}\PY{p}{,} \PY{l+m+mi}{7}\PY{p}{,} \PY{l+m+mi}{9}\PY{p}{,} \PY{o}{\PYZhy{}}\PY{l+m+mi}{5}\PY{p}{]}\PY{p}{)}\PY{p}{)}
\end{Verbatim}
\end{tcolorbox}

    mmA and mmB have rank 9 (if not, take two other points!) From mmA we get
one cubic curve cbA, from mmB we get one cubic cbB and they are two
points of the line \(L\) of \(\mathbb{P}^9\). Both cbA and cbB have,
among the eigenpoints, \(p_1\), \(p_2\), \(p_3\), \(_4\), \(p_5\).

Given the generic point \(P = (x: y: z)\) of \(\mathbb{P}^2\), consider
the three matrices obtained from mmA\_1 given by * mmA plus the row
\(\Phi(P)_{(1)}\), * mmA plus the row \(\Phi(P)_{(2)}\), and * mmA plus
the row \(\Phi(P)_{(3)}\).

We get three square matrices with determinant GA1, GA2, GA3 such that
\(p_{1, z} GA1 - p_{1, y} GA2 + p_{1, x} GA3\) splits, as a polynomial
in x, y, z, into three linear factors, one corresponds to the line
\(p_1 \vee p_2\), the other to the line \(p_1 \vee p_4\) and the third
to the line passing through the two remaining eigenpoints of cbA. A
simplification comes from the fact that \(p_1\) is the point
\((1: 0: 0)\), hence \(p_{1, z} GA1 - p_{1, y} GA2 + p_{1, x} GA3\) is
\(GA3\). Hence we have to factorize it.

    The same construction can be done for mmB and we get a polynomial GB3
which factorizes into three linear factors in x, y, z, the first two
correspond again to the lines \(p_1 \vee p_2\) and \(p_1 \vee p_4\), the
third corresponds to the line through the two remaining eigenpoints of
cbB.

A point of the line \(L\) corresponds to the cubic cb =
w1\emph{cbA+w2}cbB, where w1 and w2 are parameters.

The cubic cb has \(p_1\), \(p_2\), \(p_3\), \(p_4\), \(p_5\) as
eigenpoints and two other eigenpints, say p6 and p7, that are obtained
from the factorization of w1\emph{GA3+w2}GB3.

Now the explicit computations (we repeat the construction twice):

    \begin{tcolorbox}[breakable, size=fbox, boxrule=1pt, pad at break*=1mm,colback=cellbackground, colframe=cellborder]
\prompt{In}{incolor}{15}{\boxspacing}
\begin{Verbatim}[commandchars=\\\{\}]
\PY{n}{GA3\PYZus{}1} \PY{o}{=} \PY{n}{mmA\PYZus{}1}\PY{o}{.}\PY{n}{stack}\PY{p}{(}\PY{n}{matrix}\PY{p}{(}\PY{p}{[}\PY{n}{phi}\PY{p}{(}\PY{p}{(}\PY{n}{x}\PY{p}{,} \PY{n}{y}\PY{p}{,} \PY{n}{z}\PY{p}{)}\PY{p}{,} \PY{n}{S}\PY{p}{)}\PY{p}{[}\PY{l+m+mi}{2}\PY{p}{]}\PY{p}{]}\PY{p}{)}\PY{p}{)}\PY{o}{.}\PY{n}{det}\PY{p}{(}\PY{p}{)}
\PY{n}{GB3\PYZus{}1} \PY{o}{=} \PY{n}{mmB\PYZus{}1}\PY{o}{.}\PY{n}{stack}\PY{p}{(}\PY{n}{matrix}\PY{p}{(}\PY{p}{[}\PY{n}{phi}\PY{p}{(}\PY{p}{(}\PY{n}{x}\PY{p}{,} \PY{n}{y}\PY{p}{,} \PY{n}{z}\PY{p}{)}\PY{p}{,} \PY{n}{S}\PY{p}{)}\PY{p}{[}\PY{l+m+mi}{2}\PY{p}{]}\PY{p}{]}\PY{p}{)}\PY{p}{)}\PY{o}{.}\PY{n}{det}\PY{p}{(}\PY{p}{)}

\PY{n}{rr3\PYZus{}1} \PY{o}{=} \PY{n+nb}{list}\PY{p}{(}
    \PY{n+nb}{filter}\PY{p}{(}
        \PY{k}{lambda} \PY{n}{uu}\PY{p}{:} \PY{n}{w1} \PY{o+ow}{in} \PY{n}{uu}\PY{p}{[}\PY{l+m+mi}{0}\PY{p}{]}\PY{o}{.}\PY{n}{variables}\PY{p}{(}\PY{p}{)}\PY{p}{,}
        \PY{n+nb}{list}\PY{p}{(}\PY{n}{factor}\PY{p}{(}\PY{n}{w1}\PY{o}{*}\PY{n}{GA3\PYZus{}1}\PY{o}{+}\PY{n}{w2}\PY{o}{*}\PY{n}{GB3\PYZus{}1}\PY{p}{)}\PY{p}{)}
    \PY{p}{)}
\PY{p}{)}\PY{p}{[}\PY{l+m+mi}{0}\PY{p}{]}\PY{p}{[}\PY{l+m+mi}{0}\PY{p}{]}
\end{Verbatim}
\end{tcolorbox}

    rr3\_1 is a polynomial of degree 1 in x, y, z which gives the line
passing through the eigenpoints p6 and p7 of cb (it depends of w1 and
w2).

    We want to see if, among the cubics cb = cb(w1, w2), it is possible to
find a cubic with p1, p6, p7 aligned. Hence p1 must be a point of
rr3\_1, so the following polynomial must be zero:

    \begin{tcolorbox}[breakable, size=fbox, boxrule=1pt, pad at break*=1mm,colback=cellbackground, colframe=cellborder]
\prompt{In}{incolor}{16}{\boxspacing}
\begin{Verbatim}[commandchars=\\\{\}]
\PY{n}{hh1} \PY{o}{=} \PY{n}{rr3\PYZus{}1}\PY{o}{.}\PY{n}{subs}\PY{p}{(}\PY{p}{\PYZob{}}\PY{n}{x}\PY{p}{:}\PY{l+m+mi}{1}\PY{p}{,} \PY{n}{y}\PY{p}{:}\PY{l+m+mi}{0}\PY{p}{,} \PY{n}{z}\PY{p}{:}\PY{l+m+mi}{0}\PY{p}{\PYZcb{}}\PY{p}{)}
\end{Verbatim}
\end{tcolorbox}

    In order to get rid of the choice of the two rows above (i.e.~the choice
of two points of L), we repeat the construction for two other random
rows:

    \begin{tcolorbox}[breakable, size=fbox, boxrule=1pt, pad at break*=1mm,colback=cellbackground, colframe=cellborder]
\prompt{In}{incolor}{17}{\boxspacing}
\begin{Verbatim}[commandchars=\\\{\}]
\PY{n}{mmA\PYZus{}2} \PY{o}{=} \PY{n}{mm}\PY{o}{.}\PY{n}{stack}\PY{p}{(}\PY{n}{matrix}\PY{p}{(}\PY{p}{[}\PY{l+m+mi}{1}\PY{p}{,} \PY{o}{\PYZhy{}}\PY{l+m+mi}{5}\PY{p}{,} \PY{l+m+mi}{1}\PY{p}{,} \PY{l+m+mi}{2}\PY{p}{,} \PY{l+m+mi}{0}\PY{p}{,} \PY{l+m+mi}{1}\PY{p}{,} \PY{o}{\PYZhy{}}\PY{l+m+mi}{2}\PY{p}{,} \PY{l+m+mi}{1}\PY{p}{,} \PY{l+m+mi}{3}\PY{p}{,} \PY{l+m+mi}{7}\PY{p}{]}\PY{p}{)}\PY{p}{)}
\PY{n}{mmB\PYZus{}2} \PY{o}{=} \PY{n}{mm}\PY{o}{.}\PY{n}{stack}\PY{p}{(}\PY{n}{matrix}\PY{p}{(}\PY{p}{[}\PY{o}{\PYZhy{}}\PY{l+m+mi}{1}\PY{p}{,} \PY{o}{\PYZhy{}}\PY{l+m+mi}{1}\PY{p}{,} \PY{l+m+mi}{0}\PY{p}{,} \PY{l+m+mi}{4}\PY{p}{,} \PY{l+m+mi}{0}\PY{p}{,} \PY{l+m+mi}{1}\PY{p}{,} \PY{l+m+mi}{0}\PY{p}{,} \PY{l+m+mi}{1}\PY{p}{,} \PY{l+m+mi}{0}\PY{p}{,} \PY{o}{\PYZhy{}}\PY{l+m+mi}{5}\PY{p}{]}\PY{p}{)}\PY{p}{)}

\PY{n}{GA3\PYZus{}2} \PY{o}{=} \PY{n}{mmA\PYZus{}2}\PY{o}{.}\PY{n}{stack}\PY{p}{(}\PY{n}{matrix}\PY{p}{(}\PY{p}{[}\PY{n}{phi}\PY{p}{(}\PY{p}{(}\PY{n}{x}\PY{p}{,} \PY{n}{y}\PY{p}{,} \PY{n}{z}\PY{p}{)}\PY{p}{,} \PY{n}{S}\PY{p}{)}\PY{p}{[}\PY{l+m+mi}{2}\PY{p}{]}\PY{p}{]}\PY{p}{)}\PY{p}{)}\PY{o}{.}\PY{n}{det}\PY{p}{(}\PY{p}{)}
\PY{n}{GB3\PYZus{}2} \PY{o}{=} \PY{n}{mmB\PYZus{}2}\PY{o}{.}\PY{n}{stack}\PY{p}{(}\PY{n}{matrix}\PY{p}{(}\PY{p}{[}\PY{n}{phi}\PY{p}{(}\PY{p}{(}\PY{n}{x}\PY{p}{,} \PY{n}{y}\PY{p}{,} \PY{n}{z}\PY{p}{)}\PY{p}{,} \PY{n}{S}\PY{p}{)}\PY{p}{[}\PY{l+m+mi}{2}\PY{p}{]}\PY{p}{]}\PY{p}{)}\PY{p}{)}\PY{o}{.}\PY{n}{det}\PY{p}{(}\PY{p}{)}

\PY{n}{rr3\PYZus{}2} \PY{o}{=} \PY{n+nb}{list}\PY{p}{(}
    \PY{n+nb}{filter}\PY{p}{(}
        \PY{k}{lambda} \PY{n}{uu}\PY{p}{:} \PY{n}{w1} \PY{o+ow}{in} \PY{n}{uu}\PY{p}{[}\PY{l+m+mi}{0}\PY{p}{]}\PY{o}{.}\PY{n}{variables}\PY{p}{(}\PY{p}{)}\PY{p}{,}
        \PY{n+nb}{list}\PY{p}{(}\PY{n}{factor}\PY{p}{(}\PY{n}{w1}\PY{o}{*}\PY{n}{GA3\PYZus{}2}\PY{o}{+}\PY{n}{w2}\PY{o}{*}\PY{n}{GB3\PYZus{}2}\PY{p}{)}\PY{p}{)}
    \PY{p}{)}
\PY{p}{)}\PY{p}{[}\PY{l+m+mi}{0}\PY{p}{]}\PY{p}{[}\PY{l+m+mi}{0}\PY{p}{]}
\end{Verbatim}
\end{tcolorbox}

    \begin{tcolorbox}[breakable, size=fbox, boxrule=1pt, pad at break*=1mm,colback=cellbackground, colframe=cellborder]
\prompt{In}{incolor}{18}{\boxspacing}
\begin{Verbatim}[commandchars=\\\{\}]
\PY{c+c1}{\PYZsh{}\PYZsh{} if p1, p6, p7 are alligned, also the following polynomial must be zero}
\PY{n}{hh2} \PY{o}{=} \PY{n}{rr3\PYZus{}2}\PY{o}{.}\PY{n}{subs}\PY{p}{(}\PY{p}{\PYZob{}}\PY{n}{x}\PY{p}{:}\PY{l+m+mi}{1}\PY{p}{,} \PY{n}{y}\PY{p}{:}\PY{l+m+mi}{0}\PY{p}{,} \PY{n}{z}\PY{p}{:}\PY{l+m+mi}{0}\PY{p}{\PYZcb{}}\PY{p}{)}
\end{Verbatim}
\end{tcolorbox}

    hh1 (and hh2) are of the form w1\emph{()+w2}(). Hence hh1 (and hh2) is
zero iff w1 and w2 are chosen as solution of the equation
w1\emph{()+w2}() = 0 or if the coefficients of w1 and w2 are both zero.
In the first case, we construct r3\_1 and r3\_2 as follows:

    \begin{tcolorbox}[breakable, size=fbox, boxrule=1pt, pad at break*=1mm,colback=cellbackground, colframe=cellborder]
\prompt{In}{incolor}{23}{\boxspacing}
\begin{Verbatim}[commandchars=\\\{\}]
\PY{n}{r3\PYZus{}1} \PY{o}{=} \PY{p}{(}\PY{n}{w1}\PY{o}{*}\PY{n}{GA3\PYZus{}1}\PY{o}{+}\PY{n}{w2}\PY{o}{*}\PY{n}{GB3\PYZus{}1}\PY{p}{)}\PY{o}{.}\PY{n}{subs}\PY{p}{(}
    \PY{p}{\PYZob{}}
        \PY{n}{w1}\PY{p}{:} \PY{n}{hh1}\PY{o}{.}\PY{n}{coefficient}\PY{p}{(}\PY{n}{w2}\PY{p}{)}\PY{p}{,}
        \PY{n}{w2}\PY{p}{:} \PY{o}{\PYZhy{}}\PY{n}{hh1}\PY{o}{.}\PY{n}{coefficient}\PY{p}{(}\PY{n}{w1}\PY{p}{)}
    \PY{p}{\PYZcb{}}
\PY{p}{)}\PY{o}{.}\PY{n}{factor}\PY{p}{(}\PY{p}{)}\PY{p}{[}\PY{o}{\PYZhy{}}\PY{l+m+mi}{2}\PY{p}{]}\PY{p}{[}\PY{l+m+mi}{0}\PY{p}{]}
\PY{n}{r3\PYZus{}2} \PY{o}{=} \PY{p}{(}\PY{n}{w1}\PY{o}{*}\PY{n}{GA3\PYZus{}2}\PY{o}{+}\PY{n}{w2}\PY{o}{*}\PY{n}{GB3\PYZus{}2}\PY{p}{)}\PY{o}{.}\PY{n}{subs}\PY{p}{(}
    \PY{p}{\PYZob{}}
        \PY{n}{w1}\PY{p}{:} \PY{n}{hh2}\PY{o}{.}\PY{n}{coefficient}\PY{p}{(}\PY{n}{w2}\PY{p}{)}\PY{p}{,}
        \PY{n}{w2}\PY{p}{:} \PY{o}{\PYZhy{}}\PY{n}{hh2}\PY{o}{.}\PY{n}{coefficient}\PY{p}{(}\PY{n}{w1}\PY{p}{)}
    \PY{p}{\PYZcb{}}
\PY{p}{)}\PY{o}{.}\PY{n}{factor}\PY{p}{(}\PY{p}{)}\PY{p}{[}\PY{o}{\PYZhy{}}\PY{l+m+mi}{2}\PY{p}{]}\PY{p}{[}\PY{l+m+mi}{0}\PY{p}{]}
\end{Verbatim}
\end{tcolorbox}

    (i.e.~r3\_1 and r3\_2 are the line passing through p1, p6, p7. They
should be equal, because they should not depend of the two points of the
line L chosen. Indeed, they are equal:

    \begin{tcolorbox}[breakable, size=fbox, boxrule=1pt, pad at break*=1mm,colback=cellbackground, colframe=cellborder]
\prompt{In}{incolor}{24}{\boxspacing}
\begin{Verbatim}[commandchars=\\\{\}]
\PY{k}{assert}\PY{p}{(}\PY{n}{r3\PYZus{}1} \PY{o}{==} \PY{n}{r3\PYZus{}2}\PY{p}{)}
\end{Verbatim}
\end{tcolorbox}

    Now we have to consider the case in which r3\_1 and r3\_2 are not
defined, i.e.~when the coefficients of w1 and w2 in hh1 and hh2 are all
zero:

    \begin{tcolorbox}[breakable, size=fbox, boxrule=1pt, pad at break*=1mm,colback=cellbackground, colframe=cellborder]
\prompt{In}{incolor}{25}{\boxspacing}
\begin{Verbatim}[commandchars=\\\{\}]
\PY{n}{HH} \PY{o}{=} \PY{p}{[}\PY{n}{hh1}\PY{p}{,} \PY{n}{hh2}\PY{p}{]}
\PY{n}{JJ} \PY{o}{=} \PY{n}{S}\PY{o}{.}\PY{n}{ideal}\PY{p}{(}\PY{p}{[}\PY{n}{hh}\PY{o}{.}\PY{n}{coefficient}\PY{p}{(}\PY{n}{w1}\PY{p}{)} \PY{k}{for} \PY{n}{hh} \PY{o+ow}{in} \PY{n}{HH}\PY{p}{]} \PY{o}{+} \PY{p}{[}\PY{n}{hh}\PY{o}{.}\PY{n}{coefficient}\PY{p}{(}\PY{n}{w2}\PY{p}{)} \PY{k}{for} \PY{n}{hh} \PY{o+ow}{in} \PY{n}{HH}\PY{p}{]}\PY{p}{)}
\end{Verbatim}
\end{tcolorbox}

    If we have some values of the points p1, \ldots, p5 such that give a
zero of JJ, we have to study that case.

    But JJ, after saturation, is (1):

    \begin{tcolorbox}[breakable, size=fbox, boxrule=1pt, pad at break*=1mm,colback=cellbackground, colframe=cellborder]
\prompt{In}{incolor}{26}{\boxspacing}
\begin{Verbatim}[commandchars=\\\{\}]
\PY{n}{JJ} \PY{o}{=} \PY{n}{JJ}\PY{o}{.}\PY{n}{saturation}\PY{p}{(}\PY{n}{S}\PY{o}{.}\PY{n}{ideal}\PY{p}{(}\PY{n}{matrix}\PY{p}{(}\PY{p}{[}\PY{n}{p1}\PY{p}{,} \PY{n}{p2}\PY{p}{]}\PY{p}{)}\PY{o}{.}\PY{n}{minors}\PY{p}{(}\PY{l+m+mi}{2}\PY{p}{)}\PY{p}{)}\PY{p}{)}\PY{p}{[}\PY{l+m+mi}{0}\PY{p}{]}
\PY{n}{JJ} \PY{o}{=} \PY{n}{JJ}\PY{o}{.}\PY{n}{saturation}\PY{p}{(}\PY{n}{S}\PY{o}{.}\PY{n}{ideal}\PY{p}{(}\PY{n}{matrix}\PY{p}{(}\PY{p}{[}\PY{n}{p1}\PY{p}{,} \PY{n}{p3}\PY{p}{]}\PY{p}{)}\PY{o}{.}\PY{n}{minors}\PY{p}{(}\PY{l+m+mi}{2}\PY{p}{)}\PY{p}{)}\PY{p}{)}\PY{p}{[}\PY{l+m+mi}{0}\PY{p}{]}
\PY{n}{JJ} \PY{o}{=} \PY{n}{JJ}\PY{o}{.}\PY{n}{saturation}\PY{p}{(}\PY{n}{S}\PY{o}{.}\PY{n}{ideal}\PY{p}{(}\PY{n}{matrix}\PY{p}{(}\PY{p}{[}\PY{n}{p1}\PY{p}{,} \PY{n}{p4}\PY{p}{]}\PY{p}{)}\PY{o}{.}\PY{n}{minors}\PY{p}{(}\PY{l+m+mi}{2}\PY{p}{)}\PY{p}{)}\PY{p}{)}\PY{p}{[}\PY{l+m+mi}{0}\PY{p}{]}
\PY{n}{JJ} \PY{o}{=} \PY{n}{JJ}\PY{o}{.}\PY{n}{saturation}\PY{p}{(}\PY{n}{S}\PY{o}{.}\PY{n}{ideal}\PY{p}{(}\PY{n}{matrix}\PY{p}{(}\PY{p}{[}\PY{n}{p1}\PY{p}{,} \PY{n}{p5}\PY{p}{]}\PY{p}{)}\PY{o}{.}\PY{n}{minors}\PY{p}{(}\PY{l+m+mi}{2}\PY{p}{)}\PY{p}{)}\PY{p}{)}\PY{p}{[}\PY{l+m+mi}{0}\PY{p}{]}
\PY{n}{JJ} \PY{o}{=} \PY{n}{JJ}\PY{o}{.}\PY{n}{saturation}\PY{p}{(}\PY{n}{S}\PY{o}{.}\PY{n}{ideal}\PY{p}{(}\PY{n}{matrix}\PY{p}{(}\PY{p}{[}\PY{n}{p3}\PY{p}{,} \PY{n}{p5}\PY{p}{]}\PY{p}{)}\PY{o}{.}\PY{n}{minors}\PY{p}{(}\PY{l+m+mi}{2}\PY{p}{)}\PY{p}{)}\PY{p}{)}\PY{p}{[}\PY{l+m+mi}{0}\PY{p}{]}
\end{Verbatim}
\end{tcolorbox}

    \begin{tcolorbox}[breakable, size=fbox, boxrule=1pt, pad at break*=1mm,colback=cellbackground, colframe=cellborder]
\prompt{In}{incolor}{27}{\boxspacing}
\begin{Verbatim}[commandchars=\\\{\}]
\PY{k}{assert}\PY{p}{(}\PY{n}{JJ} \PY{o}{==} \PY{n}{S}\PY{o}{.}\PY{n}{ideal}\PY{p}{(}\PY{l+m+mi}{1}\PY{p}{)}\PY{p}{)}
\end{Verbatim}
\end{tcolorbox}

    This computation shows that there are no exceptions to consider.

On the line \(L\) of \(\mathbb{P}^9\) there is a point that corresponds
to a cubic which has the alignments \(p_1\), \(p_2\), \(p_3\), and
\(p_1\), \(p_4\), \(p_5\), and \(p_1\), \(p_6\), \(p_7\) among the
eigenpoints.

We can determine the cubic:

    \begin{tcolorbox}[breakable, size=fbox, boxrule=1pt, pad at break*=1mm,colback=cellbackground, colframe=cellborder]
\prompt{In}{incolor}{28}{\boxspacing}
\begin{Verbatim}[commandchars=\\\{\}]
\PY{n}{MM\PYZus{}1} \PY{o}{=} \PY{p}{(}\PY{n}{w1}\PY{o}{*}\PY{n}{mmA\PYZus{}1}\PY{o}{+}\PY{n}{w2}\PY{o}{*}\PY{n}{mmB\PYZus{}1}\PY{p}{)}\PY{o}{.}\PY{n}{subs}\PY{p}{(}
    \PY{p}{\PYZob{}}
        \PY{n}{w1}\PY{p}{:} \PY{n}{hh1}\PY{o}{.}\PY{n}{coefficient}\PY{p}{(}\PY{n}{w2}\PY{p}{)}\PY{p}{,}
        \PY{n}{w2}\PY{p}{:} \PY{o}{\PYZhy{}}\PY{n}{hh1}\PY{o}{.}\PY{n}{coefficient}\PY{p}{(}\PY{n}{w1}\PY{p}{)}
    \PY{p}{\PYZcb{}}
\PY{p}{)}

\PY{n}{Mcb1} \PY{o}{=} \PY{n}{MM\PYZus{}1}\PY{o}{.}\PY{n}{stack}\PY{p}{(}\PY{n}{vector}\PY{p}{(}\PY{n}{S}\PY{p}{,} \PY{n}{mon}\PY{p}{)}\PY{p}{)}

\PY{n}{MM\PYZus{}2} \PY{o}{=} \PY{p}{(}\PY{n}{w1}\PY{o}{*}\PY{n}{mmA\PYZus{}2}\PY{o}{+}\PY{n}{w2}\PY{o}{*}\PY{n}{mmB\PYZus{}2}\PY{p}{)}\PY{o}{.}\PY{n}{subs}\PY{p}{(}
    \PY{p}{\PYZob{}}
        \PY{n}{w1}\PY{p}{:} \PY{n}{hh2}\PY{o}{.}\PY{n}{coefficient}\PY{p}{(}\PY{n}{w2}\PY{p}{)}\PY{p}{,}
        \PY{n}{w2}\PY{p}{:} \PY{o}{\PYZhy{}}\PY{n}{hh2}\PY{o}{.}\PY{n}{coefficient}\PY{p}{(}\PY{n}{w1}\PY{p}{)}
    \PY{p}{\PYZcb{}}
\PY{p}{)}

\PY{n}{Mcb2} \PY{o}{=} \PY{n}{MM\PYZus{}2}\PY{o}{.}\PY{n}{stack}\PY{p}{(}\PY{n}{vector}\PY{p}{(}\PY{n}{S}\PY{p}{,} \PY{n}{mon}\PY{p}{)}\PY{p}{)}
\end{Verbatim}
\end{tcolorbox}

    The cubic is the determinant of Mcb1 (or of Mcb2).

    The following computations require, respectively, about 3000 and 4000
seconds. We omit them but we define below the cubic cb which is
obtained:

    \begin{tcolorbox}[breakable, size=fbox, boxrule=1pt, pad at break*=1mm,colback=cellbackground, colframe=cellborder]
\prompt{In}{incolor}{29}{\boxspacing}
\begin{Verbatim}[commandchars=\\\{\}]
\PY{c+c1}{\PYZsh{} ttA = cputime()}
\PY{c+c1}{\PYZsh{} cb1 = Mcb1.det()}
\PY{c+c1}{\PYZsh{} print(cputime()\PYZhy{}ttA)}
\PY{c+c1}{\PYZsh{} }
\PY{c+c1}{\PYZsh{} ttA = cputime()}
\PY{c+c1}{\PYZsh{} cb2 = Mcb2.det()}
\PY{c+c1}{\PYZsh{} print(cputime()\PYZhy{}ttA)}
\PY{c+c1}{\PYZsh{} }
\PY{c+c1}{\PYZsh{} assert(cb1.factor()[\PYZhy{}1][0]  == cb2.factor()[\PYZhy{}1][0])}
\end{Verbatim}
\end{tcolorbox}

    \begin{tcolorbox}[breakable, size=fbox, boxrule=1pt, pad at break*=1mm,colback=cellbackground, colframe=cellborder]
\prompt{In}{incolor}{30}{\boxspacing}
\begin{Verbatim}[commandchars=\\\{\}]
\PY{n}{cb} \PY{o}{=} \PY{p}{(}
    \PY{n}{z}\PY{o}{\PYZca{}}\PY{l+m+mi}{3}\PY{o}{*}\PY{n}{A2}\PY{o}{*}\PY{n}{B2}\PY{o}{\PYZca{}}\PY{l+m+mi}{3}\PY{o}{*}\PY{n}{C2}\PY{o}{\PYZca{}}\PY{l+m+mi}{2}\PY{o}{*}\PY{n}{A4}\PY{o}{\PYZca{}}\PY{l+m+mi}{2} \PY{o}{\PYZhy{}} \PY{l+m+mi}{3}\PY{o}{*}\PY{n}{y}\PY{o}{*}\PY{n}{z}\PY{o}{\PYZca{}}\PY{l+m+mi}{2}\PY{o}{*}\PY{n}{A2}\PY{o}{*}\PY{n}{B2}\PY{o}{\PYZca{}}\PY{l+m+mi}{2}\PY{o}{*}\PY{n}{C2}\PY{o}{\PYZca{}}\PY{l+m+mi}{3}\PY{o}{*}\PY{n}{A4}\PY{o}{\PYZca{}}\PY{l+m+mi}{2} \PY{o}{+} \PY{l+m+mi}{3}\PY{o}{*}\PY{n}{y}\PY{o}{\PYZca{}}\PY{l+m+mi}{2}\PY{o}{*}\PY{n}{z}\PY{o}{*}\PY{n}{A2}\PY{o}{*}\PY{n}{B2}\PY{o}{*}\PY{n}{C2}\PY{o}{\PYZca{}}\PY{l+m+mi}{4}\PY{o}{*}\PY{n}{A4}\PY{o}{\PYZca{}}\PY{l+m+mi}{2} 
    \PY{o}{\PYZhy{}} \PY{n}{y}\PY{o}{\PYZca{}}\PY{l+m+mi}{3}\PY{o}{*}\PY{n}{A2}\PY{o}{*}\PY{n}{C2}\PY{o}{\PYZca{}}\PY{l+m+mi}{5}\PY{o}{*}\PY{n}{A4}\PY{o}{\PYZca{}}\PY{l+m+mi}{2} \PY{o}{\PYZhy{}} \PY{n}{y}\PY{o}{\PYZca{}}\PY{l+m+mi}{3}\PY{o}{*}\PY{n}{A2}\PY{o}{\PYZca{}}\PY{l+m+mi}{2}\PY{o}{*}\PY{n}{B2}\PY{o}{\PYZca{}}\PY{l+m+mi}{3}\PY{o}{*}\PY{n}{C2}\PY{o}{*}\PY{n}{A4}\PY{o}{*}\PY{n}{B4} \PY{o}{+} \PY{l+m+mi}{3}\PY{o}{/}\PY{l+m+mi}{2}\PY{o}{*}\PY{n}{x}\PY{o}{*}\PY{n}{y}\PY{o}{\PYZca{}}\PY{l+m+mi}{2}\PY{o}{*}\PY{n}{A2}\PY{o}{*}\PY{n}{B2}\PY{o}{\PYZca{}}\PY{l+m+mi}{4}\PY{o}{*}\PY{n}{C2}\PY{o}{*}\PY{n}{A4}\PY{o}{*}\PY{n}{B4} 
    \PY{o}{+} \PY{l+m+mi}{3}\PY{o}{/}\PY{l+m+mi}{2}\PY{o}{*}\PY{n}{x}\PY{o}{*}\PY{n}{z}\PY{o}{\PYZca{}}\PY{l+m+mi}{2}\PY{o}{*}\PY{n}{A2}\PY{o}{*}\PY{n}{B2}\PY{o}{\PYZca{}}\PY{l+m+mi}{4}\PY{o}{*}\PY{n}{C2}\PY{o}{*}\PY{n}{A4}\PY{o}{*}\PY{n}{B4} \PY{o}{+} \PY{l+m+mi}{1}\PY{o}{/}\PY{l+m+mi}{2}\PY{o}{*}\PY{n}{y}\PY{o}{\PYZca{}}\PY{l+m+mi}{3}\PY{o}{*}\PY{n}{B2}\PY{o}{\PYZca{}}\PY{l+m+mi}{5}\PY{o}{*}\PY{n}{C2}\PY{o}{*}\PY{n}{A4}\PY{o}{*}\PY{n}{B4} \PY{o}{\PYZhy{}} \PY{l+m+mi}{3}\PY{o}{*}\PY{n}{y}\PY{o}{\PYZca{}}\PY{l+m+mi}{2}\PY{o}{*}\PY{n}{z}\PY{o}{*}\PY{n}{A2}\PY{o}{\PYZca{}}\PY{l+m+mi}{2}\PY{o}{*}\PY{n}{B2}\PY{o}{\PYZca{}}\PY{l+m+mi}{2}\PY{o}{*}\PY{n}{C2}\PY{o}{\PYZca{}}\PY{l+m+mi}{2}\PY{o}{*}\PY{n}{A4}\PY{o}{*}\PY{n}{B4} 
    \PY{o}{+} \PY{l+m+mi}{3}\PY{o}{/}\PY{l+m+mi}{2}\PY{o}{*}\PY{n}{y}\PY{o}{\PYZca{}}\PY{l+m+mi}{2}\PY{o}{*}\PY{n}{z}\PY{o}{*}\PY{n}{B2}\PY{o}{\PYZca{}}\PY{l+m+mi}{4}\PY{o}{*}\PY{n}{C2}\PY{o}{\PYZca{}}\PY{l+m+mi}{2}\PY{o}{*}\PY{n}{A4}\PY{o}{*}\PY{n}{B4} \PY{o}{\PYZhy{}} \PY{l+m+mi}{3}\PY{o}{*}\PY{n}{y}\PY{o}{*}\PY{n}{z}\PY{o}{\PYZca{}}\PY{l+m+mi}{2}\PY{o}{*}\PY{n}{A2}\PY{o}{\PYZca{}}\PY{l+m+mi}{2}\PY{o}{*}\PY{n}{B2}\PY{o}{*}\PY{n}{C2}\PY{o}{\PYZca{}}\PY{l+m+mi}{3}\PY{o}{*}\PY{n}{A4}\PY{o}{*}\PY{n}{B4} \PY{o}{+} \PY{l+m+mi}{3}\PY{o}{*}\PY{n}{x}\PY{o}{*}\PY{n}{y}\PY{o}{\PYZca{}}\PY{l+m+mi}{2}\PY{o}{*}\PY{n}{A2}\PY{o}{*}\PY{n}{B2}\PY{o}{\PYZca{}}\PY{l+m+mi}{2}\PY{o}{*}\PY{n}{C2}\PY{o}{\PYZca{}}\PY{l+m+mi}{3}\PY{o}{*}\PY{n}{A4}\PY{o}{*}\PY{n}{B4} 
    \PY{o}{+} \PY{l+m+mi}{3}\PY{o}{*}\PY{n}{x}\PY{o}{*}\PY{n}{z}\PY{o}{\PYZca{}}\PY{l+m+mi}{2}\PY{o}{*}\PY{n}{A2}\PY{o}{*}\PY{n}{B2}\PY{o}{\PYZca{}}\PY{l+m+mi}{2}\PY{o}{*}\PY{n}{C2}\PY{o}{\PYZca{}}\PY{l+m+mi}{3}\PY{o}{*}\PY{n}{A4}\PY{o}{*}\PY{n}{B4} \PY{o}{+} \PY{l+m+mi}{1}\PY{o}{/}\PY{l+m+mi}{2}\PY{o}{*}\PY{n}{y}\PY{o}{\PYZca{}}\PY{l+m+mi}{3}\PY{o}{*}\PY{n}{B2}\PY{o}{\PYZca{}}\PY{l+m+mi}{3}\PY{o}{*}\PY{n}{C2}\PY{o}{\PYZca{}}\PY{l+m+mi}{3}\PY{o}{*}\PY{n}{A4}\PY{o}{*}\PY{n}{B4} \PY{o}{+} \PY{l+m+mi}{3}\PY{o}{/}\PY{l+m+mi}{2}\PY{o}{*}\PY{n}{y}\PY{o}{*}\PY{n}{z}\PY{o}{\PYZca{}}\PY{l+m+mi}{2}\PY{o}{*}\PY{n}{B2}\PY{o}{\PYZca{}}\PY{l+m+mi}{3}\PY{o}{*}\PY{n}{C2}\PY{o}{\PYZca{}}\PY{l+m+mi}{3}\PY{o}{*}\PY{n}{A4}\PY{o}{*}\PY{n}{B4} 
    \PY{o}{\PYZhy{}} \PY{n}{z}\PY{o}{\PYZca{}}\PY{l+m+mi}{3}\PY{o}{*}\PY{n}{A2}\PY{o}{\PYZca{}}\PY{l+m+mi}{2}\PY{o}{*}\PY{n}{C2}\PY{o}{\PYZca{}}\PY{l+m+mi}{4}\PY{o}{*}\PY{n}{A4}\PY{o}{*}\PY{n}{B4} \PY{o}{+} \PY{l+m+mi}{3}\PY{o}{/}\PY{l+m+mi}{2}\PY{o}{*}\PY{n}{y}\PY{o}{\PYZca{}}\PY{l+m+mi}{2}\PY{o}{*}\PY{n}{z}\PY{o}{*}\PY{n}{B2}\PY{o}{\PYZca{}}\PY{l+m+mi}{2}\PY{o}{*}\PY{n}{C2}\PY{o}{\PYZca{}}\PY{l+m+mi}{4}\PY{o}{*}\PY{n}{A4}\PY{o}{*}\PY{n}{B4} \PY{o}{+} \PY{l+m+mi}{1}\PY{o}{/}\PY{l+m+mi}{2}\PY{o}{*}\PY{n}{z}\PY{o}{\PYZca{}}\PY{l+m+mi}{3}\PY{o}{*}\PY{n}{B2}\PY{o}{\PYZca{}}\PY{l+m+mi}{2}\PY{o}{*}\PY{n}{C2}\PY{o}{\PYZca{}}\PY{l+m+mi}{4}\PY{o}{*}\PY{n}{A4}\PY{o}{*}\PY{n}{B4} 
    \PY{o}{+} \PY{l+m+mi}{3}\PY{o}{/}\PY{l+m+mi}{2}\PY{o}{*}\PY{n}{x}\PY{o}{*}\PY{n}{y}\PY{o}{\PYZca{}}\PY{l+m+mi}{2}\PY{o}{*}\PY{n}{A2}\PY{o}{*}\PY{n}{C2}\PY{o}{\PYZca{}}\PY{l+m+mi}{5}\PY{o}{*}\PY{n}{A4}\PY{o}{*}\PY{n}{B4} \PY{o}{+} \PY{l+m+mi}{3}\PY{o}{/}\PY{l+m+mi}{2}\PY{o}{*}\PY{n}{x}\PY{o}{*}\PY{n}{z}\PY{o}{\PYZca{}}\PY{l+m+mi}{2}\PY{o}{*}\PY{n}{A2}\PY{o}{*}\PY{n}{C2}\PY{o}{\PYZca{}}\PY{l+m+mi}{5}\PY{o}{*}\PY{n}{A4}\PY{o}{*}\PY{n}{B4} \PY{o}{+} \PY{l+m+mi}{3}\PY{o}{/}\PY{l+m+mi}{2}\PY{o}{*}\PY{n}{y}\PY{o}{*}\PY{n}{z}\PY{o}{\PYZca{}}\PY{l+m+mi}{2}\PY{o}{*}\PY{n}{B2}\PY{o}{*}\PY{n}{C2}\PY{o}{\PYZca{}}\PY{l+m+mi}{5}\PY{o}{*}\PY{n}{A4}\PY{o}{*}\PY{n}{B4} 
    \PY{o}{+} \PY{l+m+mi}{1}\PY{o}{/}\PY{l+m+mi}{2}\PY{o}{*}\PY{n}{z}\PY{o}{\PYZca{}}\PY{l+m+mi}{3}\PY{o}{*}\PY{n}{C2}\PY{o}{\PYZca{}}\PY{l+m+mi}{6}\PY{o}{*}\PY{n}{A4}\PY{o}{*}\PY{n}{B4} \PY{o}{\PYZhy{}} \PY{l+m+mi}{1}\PY{o}{/}\PY{l+m+mi}{2}\PY{o}{*}\PY{n}{z}\PY{o}{\PYZca{}}\PY{l+m+mi}{3}\PY{o}{*}\PY{n}{A2}\PY{o}{*}\PY{n}{B2}\PY{o}{\PYZca{}}\PY{l+m+mi}{5}\PY{o}{*}\PY{n}{B4}\PY{o}{\PYZca{}}\PY{l+m+mi}{2} \PY{o}{+} \PY{l+m+mi}{3}\PY{o}{/}\PY{l+m+mi}{2}\PY{o}{*}\PY{n}{y}\PY{o}{*}\PY{n}{z}\PY{o}{\PYZca{}}\PY{l+m+mi}{2}\PY{o}{*}\PY{n}{A2}\PY{o}{*}\PY{n}{B2}\PY{o}{\PYZca{}}\PY{l+m+mi}{4}\PY{o}{*}\PY{n}{C2}\PY{o}{*}\PY{n}{B4}\PY{o}{\PYZca{}}\PY{l+m+mi}{2} 
    \PY{o}{\PYZhy{}} \PY{l+m+mi}{3}\PY{o}{/}\PY{l+m+mi}{2}\PY{o}{*}\PY{n}{y}\PY{o}{\PYZca{}}\PY{l+m+mi}{2}\PY{o}{*}\PY{n}{z}\PY{o}{*}\PY{n}{A2}\PY{o}{*}\PY{n}{B2}\PY{o}{\PYZca{}}\PY{l+m+mi}{3}\PY{o}{*}\PY{n}{C2}\PY{o}{\PYZca{}}\PY{l+m+mi}{2}\PY{o}{*}\PY{n}{B4}\PY{o}{\PYZca{}}\PY{l+m+mi}{2} \PY{o}{\PYZhy{}} \PY{l+m+mi}{1}\PY{o}{/}\PY{l+m+mi}{2}\PY{o}{*}\PY{n}{z}\PY{o}{\PYZca{}}\PY{l+m+mi}{3}\PY{o}{*}\PY{n}{A2}\PY{o}{*}\PY{n}{B2}\PY{o}{\PYZca{}}\PY{l+m+mi}{3}\PY{o}{*}\PY{n}{C2}\PY{o}{\PYZca{}}\PY{l+m+mi}{2}\PY{o}{*}\PY{n}{B4}\PY{o}{\PYZca{}}\PY{l+m+mi}{2} \PY{o}{+} \PY{l+m+mi}{1}\PY{o}{/}\PY{l+m+mi}{2}\PY{o}{*}\PY{n}{y}\PY{o}{\PYZca{}}\PY{l+m+mi}{3}\PY{o}{*}\PY{n}{A2}\PY{o}{*}\PY{n}{B2}\PY{o}{\PYZca{}}\PY{l+m+mi}{2}\PY{o}{*}\PY{n}{C2}\PY{o}{\PYZca{}}\PY{l+m+mi}{3}\PY{o}{*}\PY{n}{B4}\PY{o}{\PYZca{}}\PY{l+m+mi}{2} 
    \PY{o}{+} \PY{l+m+mi}{3}\PY{o}{/}\PY{l+m+mi}{2}\PY{o}{*}\PY{n}{y}\PY{o}{*}\PY{n}{z}\PY{o}{\PYZca{}}\PY{l+m+mi}{2}\PY{o}{*}\PY{n}{A2}\PY{o}{*}\PY{n}{B2}\PY{o}{\PYZca{}}\PY{l+m+mi}{2}\PY{o}{*}\PY{n}{C2}\PY{o}{\PYZca{}}\PY{l+m+mi}{3}\PY{o}{*}\PY{n}{B4}\PY{o}{\PYZca{}}\PY{l+m+mi}{2} \PY{o}{\PYZhy{}} \PY{l+m+mi}{3}\PY{o}{/}\PY{l+m+mi}{2}\PY{o}{*}\PY{n}{y}\PY{o}{\PYZca{}}\PY{l+m+mi}{2}\PY{o}{*}\PY{n}{z}\PY{o}{*}\PY{n}{A2}\PY{o}{*}\PY{n}{B2}\PY{o}{*}\PY{n}{C2}\PY{o}{\PYZca{}}\PY{l+m+mi}{4}\PY{o}{*}\PY{n}{B4}\PY{o}{\PYZca{}}\PY{l+m+mi}{2} \PY{o}{+} \PY{l+m+mi}{1}\PY{o}{/}\PY{l+m+mi}{2}\PY{o}{*}\PY{n}{y}\PY{o}{\PYZca{}}\PY{l+m+mi}{3}\PY{o}{*}\PY{n}{A2}\PY{o}{*}\PY{n}{C2}\PY{o}{\PYZca{}}\PY{l+m+mi}{5}\PY{o}{*}\PY{n}{B4}\PY{o}{\PYZca{}}\PY{l+m+mi}{2}
\PY{p}{)}
\end{Verbatim}
\end{tcolorbox}

    We remark however that cb can be computed in 12 seconds using the fact
that the rows of Mcb1 have big common factors:

    \begin{tcolorbox}[breakable, size=fbox, boxrule=1pt, pad at break*=1mm,colback=cellbackground, colframe=cellborder]
\prompt{In}{incolor}{31}{\boxspacing}
\begin{Verbatim}[commandchars=\\\{\}]
\PY{n}{Ms} \PY{o}{=} \PY{p}{[}\PY{p}{]}
\PY{k}{for} \PY{n}{i} \PY{o+ow}{in} \PY{n+nb}{range}\PY{p}{(}\PY{l+m+mi}{10}\PY{p}{)}\PY{p}{:}
    \PY{n}{gd} \PY{o}{=} \PY{n}{gcd}\PY{p}{(}\PY{p}{[}\PY{n}{Mcb1}\PY{p}{[}\PY{n}{i}\PY{p}{,}\PY{n}{j}\PY{p}{]} \PY{k}{for} \PY{n}{j} \PY{o+ow}{in} \PY{n+nb}{range}\PY{p}{(}\PY{l+m+mi}{10}\PY{p}{)}\PY{p}{]}\PY{p}{)}
    \PY{n}{Ms}\PY{o}{.}\PY{n}{append}\PY{p}{(}\PY{p}{[}\PY{n}{Mcb1}\PY{p}{[}\PY{n}{i}\PY{p}{,}\PY{n}{j}\PY{p}{]}\PY{o}{.}\PY{n}{quo\PYZus{}rem}\PY{p}{(}\PY{n}{gd}\PY{p}{)}\PY{p}{[}\PY{l+m+mi}{0}\PY{p}{]} \PY{k}{for} \PY{n}{j} \PY{o+ow}{in} \PY{n+nb}{range}\PY{p}{(}\PY{l+m+mi}{10}\PY{p}{)}\PY{p}{]}\PY{p}{)}

\PY{n}{cb\PYZus{}alt} \PY{o}{=} \PY{n}{matrix}\PY{p}{(}\PY{n}{Ms}\PY{p}{)}\PY{o}{.}\PY{n}{det}\PY{p}{(}\PY{p}{)}\PY{o}{.}\PY{n}{factor}\PY{p}{(}\PY{p}{)}\PY{p}{[}\PY{o}{\PYZhy{}}\PY{l+m+mi}{1}\PY{p}{]}\PY{p}{[}\PY{l+m+mi}{0}\PY{p}{]}
\end{Verbatim}
\end{tcolorbox}

    \begin{tcolorbox}[breakable, size=fbox, boxrule=1pt, pad at break*=1mm,colback=cellbackground, colframe=cellborder]
\prompt{In}{incolor}{32}{\boxspacing}
\begin{Verbatim}[commandchars=\\\{\}]
\PY{k}{assert}\PY{p}{(}\PY{n}{cb\PYZus{}alt} \PY{o}{==} \PY{n}{cb}\PY{p}{)}
\end{Verbatim}
\end{tcolorbox}

    An example shows that cb, in general, has the following aligned
eigenpoints: p1, p2, p3 and p1, p4, p5 and p1, p6, p7. The cubic is
irreducible and singular in p1.

    \begin{tcolorbox}[breakable, size=fbox, boxrule=1pt, pad at break*=1mm,colback=cellbackground, colframe=cellborder]
\prompt{In}{incolor}{33}{\boxspacing}
\begin{Verbatim}[commandchars=\\\{\}]
\PY{n}{ccb} \PY{o}{=} \PY{n}{cb}\PY{o}{.}\PY{n}{subs}\PY{p}{(}\PY{p}{\PYZob{}}\PY{n}{A2}\PY{p}{:}\PY{l+m+mi}{5}\PY{p}{,} \PY{n}{B2}\PY{p}{:}\PY{o}{\PYZhy{}}\PY{l+m+mi}{3}\PY{p}{,} \PY{n}{C2}\PY{p}{:}\PY{o}{\PYZhy{}}\PY{l+m+mi}{1}\PY{p}{,} \PY{n}{A4}\PY{p}{:}\PY{l+m+mi}{2}\PY{p}{,} \PY{n}{B4}\PY{p}{:}\PY{o}{\PYZhy{}}\PY{l+m+mi}{7}\PY{p}{\PYZcb{}}\PY{p}{)}
\end{Verbatim}
\end{tcolorbox}

    HERE WE CONCLUDE THE FIRST PART OF THE COMPUTATION:

In case \(C_2 \neq 0\), IT IS POSSIBLE TO HAVE THREE ALIGNMENTS: (1, 2,
3), (1, 4, 5), (1, 6, 7)

    Now we want to see if it is possible to have more then three alignments
(We continue to assume C2 != 0)

Recall that cb is our cubic.

Recall that r3\_1 (= r3\_2) is the line through p6 and p7.

Up to a permutation of the indices of the points, if there is another
alignment among the eigenpoints, p6 must be on the line p2+p4. Hence we
can find it, since is the intersection or p2+p4 and r3.

    \begin{tcolorbox}[breakable, size=fbox, boxrule=1pt, pad at break*=1mm,colback=cellbackground, colframe=cellborder]
\prompt{In}{incolor}{34}{\boxspacing}
\begin{Verbatim}[commandchars=\\\{\}]
\PY{n}{r24} \PY{o}{=} \PY{n}{matrix}\PY{p}{(}\PY{p}{[}\PY{n}{p2}\PY{p}{,} \PY{n}{p4}\PY{p}{,} \PY{p}{(}\PY{n}{x}\PY{p}{,} \PY{n}{y}\PY{p}{,} \PY{n}{z}\PY{p}{)}\PY{p}{]}\PY{p}{)}\PY{o}{.}\PY{n}{det}\PY{p}{(}\PY{p}{)}

\PY{n}{E1} \PY{o}{=} \PY{n}{matrix}\PY{p}{(}
    \PY{p}{[}
        \PY{p}{[}\PY{n}{r3\PYZus{}1}\PY{o}{.}\PY{n}{coefficient}\PY{p}{(}\PY{n}{xx}\PY{p}{)} \PY{k}{for} \PY{n}{xx} \PY{o+ow}{in} \PY{p}{[}\PY{n}{x}\PY{p}{,} \PY{n}{y}\PY{p}{,} \PY{n}{z}\PY{p}{]}\PY{p}{]}\PY{p}{,}
        \PY{p}{[}\PY{n}{r24}\PY{o}{.}\PY{n}{coefficient}\PY{p}{(}\PY{n}{xx}\PY{p}{)} \PY{k}{for} \PY{n}{xx} \PY{o+ow}{in} \PY{p}{[}\PY{n}{x}\PY{p}{,} \PY{n}{y}\PY{p}{,} \PY{n}{z}\PY{p}{]}\PY{p}{]}
    \PY{p}{]}
\PY{p}{)}\PY{o}{.}\PY{n}{minors}\PY{p}{(}\PY{l+m+mi}{2}\PY{p}{)}
\end{Verbatim}
\end{tcolorbox}

    All the entries of E1 can be divided by \(B_2^2 B_4 + C_2^2 B_4\) and we
know that \(B_2^2 B_4 + C_2^2 B_4 = 0\) implies \(p_3 = p_5\), hence we
divide with no problems.

    \begin{tcolorbox}[breakable, size=fbox, boxrule=1pt, pad at break*=1mm,colback=cellbackground, colframe=cellborder]
\prompt{In}{incolor}{35}{\boxspacing}
\begin{Verbatim}[commandchars=\\\{\}]
\PY{n}{p6} \PY{o}{=} \PY{n}{vector}\PY{p}{(}
    \PY{n}{S}\PY{p}{,}
    \PY{p}{(}
        \PY{n}{E1}\PY{p}{[}\PY{l+m+mi}{2}\PY{p}{]}\PY{o}{/}\PY{p}{(}\PY{n}{B2}\PY{o}{\PYZca{}}\PY{l+m+mi}{2}\PY{o}{*}\PY{n}{B4} \PY{o}{+} \PY{n}{C2}\PY{o}{\PYZca{}}\PY{l+m+mi}{2}\PY{o}{*}\PY{n}{B4}\PY{p}{)}\PY{p}{,} 
        \PY{o}{\PYZhy{}}\PY{n}{E1}\PY{p}{[}\PY{l+m+mi}{1}\PY{p}{]}\PY{o}{/}\PY{p}{(}\PY{n}{B2}\PY{o}{\PYZca{}}\PY{l+m+mi}{2}\PY{o}{*}\PY{n}{B4} \PY{o}{+} \PY{n}{C2}\PY{o}{\PYZca{}}\PY{l+m+mi}{2}\PY{o}{*}\PY{n}{B4}\PY{p}{)}\PY{p}{,}
        \PY{n}{E1}\PY{p}{[}\PY{l+m+mi}{0}\PY{p}{]}\PY{o}{/}\PY{p}{(}\PY{n}{B2}\PY{o}{\PYZca{}}\PY{l+m+mi}{2}\PY{o}{*}\PY{n}{B4} \PY{o}{+} \PY{n}{C2}\PY{o}{\PYZca{}}\PY{l+m+mi}{2}\PY{o}{*}\PY{n}{B4}\PY{p}{)}
    \PY{p}{)}
\PY{p}{)}
\end{Verbatim}
\end{tcolorbox}

    Now we compute the ideal kJ of the eigenpoints of cb and we saturate it
as much as possible (in particular, we saturate it w.r.t. the ideals of
the points p1, p2, p3, p4, p5):

    \begin{tcolorbox}[breakable, size=fbox, boxrule=1pt, pad at break*=1mm,colback=cellbackground, colframe=cellborder]
\prompt{In}{incolor}{36}{\boxspacing}
\begin{Verbatim}[commandchars=\\\{\}]
\PY{n}{kJ} \PY{o}{=} \PY{n}{S}\PY{o}{.}\PY{n}{ideal}\PY{p}{(}
    \PY{n}{matrix}\PY{p}{(}
        \PY{p}{[}
            \PY{p}{[}\PY{n}{x}\PY{p}{,} \PY{n}{y}\PY{p}{,} \PY{n}{z}\PY{p}{]}\PY{p}{,}
            \PY{p}{[}\PY{n}{cb}\PY{o}{.}\PY{n}{derivative}\PY{p}{(}\PY{n}{x}\PY{p}{)}\PY{p}{,} \PY{n}{cb}\PY{o}{.}\PY{n}{derivative}\PY{p}{(}\PY{n}{y}\PY{p}{)}\PY{p}{,} \PY{n}{cb}\PY{o}{.}\PY{n}{derivative}\PY{p}{(}\PY{n}{z}\PY{p}{)}\PY{p}{]}
        \PY{p}{]}
    \PY{p}{)}\PY{o}{.}\PY{n}{minors}\PY{p}{(}\PY{l+m+mi}{2}\PY{p}{)}
\PY{p}{)}
\end{Verbatim}
\end{tcolorbox}

    \begin{tcolorbox}[breakable, size=fbox, boxrule=1pt, pad at break*=1mm,colback=cellbackground, colframe=cellborder]
\prompt{In}{incolor}{37}{\boxspacing}
\begin{Verbatim}[commandchars=\\\{\}]
\PY{n}{kJ} \PY{o}{=} \PY{n}{kJ}\PY{o}{.}\PY{n}{saturation}\PY{p}{(}\PY{n}{B2}\PY{o}{\PYZca{}}\PY{l+m+mi}{2}\PY{o}{*}\PY{n}{B4} \PY{o}{+} \PY{n}{C2}\PY{o}{\PYZca{}}\PY{l+m+mi}{2}\PY{o}{*}\PY{n}{B4}\PY{p}{)}\PY{p}{[}\PY{l+m+mi}{0}\PY{p}{]}
\PY{n}{kJ} \PY{o}{=} \PY{n}{kJ}\PY{o}{.}\PY{n}{saturation}\PY{p}{(}\PY{n}{S}\PY{o}{.}\PY{n}{ideal}\PY{p}{(}\PY{n}{matrix}\PY{p}{(}\PY{p}{[}\PY{n}{p1}\PY{p}{,} \PY{n}{p3}\PY{p}{]}\PY{p}{)}\PY{o}{.}\PY{n}{minors}\PY{p}{(}\PY{l+m+mi}{2}\PY{p}{)}\PY{p}{)}\PY{p}{)}\PY{p}{[}\PY{l+m+mi}{0}\PY{p}{]}
\PY{n}{kJ} \PY{o}{=} \PY{n}{kJ}\PY{o}{.}\PY{n}{saturation}\PY{p}{(}\PY{n}{S}\PY{o}{.}\PY{n}{ideal}\PY{p}{(}\PY{n}{matrix}\PY{p}{(}\PY{p}{[}\PY{n}{p1}\PY{p}{,} \PY{n}{p5}\PY{p}{]}\PY{p}{)}\PY{o}{.}\PY{n}{minors}\PY{p}{(}\PY{l+m+mi}{2}\PY{p}{)}\PY{p}{)}\PY{p}{)}\PY{p}{[}\PY{l+m+mi}{0}\PY{p}{]}
\PY{n}{kJ} \PY{o}{=} \PY{n}{kJ}\PY{o}{.}\PY{n}{saturation}\PY{p}{(}\PY{n}{S}\PY{o}{.}\PY{n}{ideal}\PY{p}{(}\PY{n}{z}\PY{p}{,} \PY{n}{y}\PY{p}{)}\PY{p}{)}\PY{p}{[}\PY{l+m+mi}{0}\PY{p}{]}  \PY{c+c1}{\PYZsh{}\PYZsh{}p1}
\PY{n}{kJ} \PY{o}{=} \PY{n}{kJ}\PY{o}{.}\PY{n}{saturation}\PY{p}{(}\PY{n}{S}\PY{o}{.}\PY{n}{ideal}\PY{p}{(}\PY{n}{p2}\PY{p}{[}\PY{l+m+mi}{0}\PY{p}{]}\PY{o}{*}\PY{n}{y}\PY{o}{\PYZhy{}}\PY{n}{p2}\PY{p}{[}\PY{l+m+mi}{1}\PY{p}{]}\PY{o}{*}\PY{n}{x}\PY{p}{,} \PY{n}{p2}\PY{p}{[}\PY{l+m+mi}{0}\PY{p}{]}\PY{o}{*}\PY{n}{z}\PY{o}{\PYZhy{}}\PY{n}{p2}\PY{p}{[}\PY{l+m+mi}{2}\PY{p}{]}\PY{o}{*}\PY{n}{x}\PY{p}{,} \PY{n}{p2}\PY{p}{[}\PY{l+m+mi}{1}\PY{p}{]}\PY{o}{*}\PY{n}{z}\PY{o}{\PYZhy{}}\PY{n}{p2}\PY{p}{[}\PY{l+m+mi}{2}\PY{p}{]}\PY{o}{*}\PY{n}{y}\PY{p}{)}\PY{p}{)}\PY{p}{[}\PY{l+m+mi}{0}\PY{p}{]} \PY{c+c1}{\PYZsh{}\PYZsh{} p2}
\PY{n}{kJ} \PY{o}{=} \PY{n}{kJ}\PY{o}{.}\PY{n}{saturation}\PY{p}{(}\PY{n}{S}\PY{o}{.}\PY{n}{ideal}\PY{p}{(}\PY{n}{p3}\PY{p}{[}\PY{l+m+mi}{0}\PY{p}{]}\PY{o}{*}\PY{n}{y}\PY{o}{\PYZhy{}}\PY{n}{p3}\PY{p}{[}\PY{l+m+mi}{1}\PY{p}{]}\PY{o}{*}\PY{n}{x}\PY{p}{,} \PY{n}{p3}\PY{p}{[}\PY{l+m+mi}{0}\PY{p}{]}\PY{o}{*}\PY{n}{z}\PY{o}{\PYZhy{}}\PY{n}{p3}\PY{p}{[}\PY{l+m+mi}{2}\PY{p}{]}\PY{o}{*}\PY{n}{x}\PY{p}{,} \PY{n}{p3}\PY{p}{[}\PY{l+m+mi}{1}\PY{p}{]}\PY{o}{*}\PY{n}{z}\PY{o}{\PYZhy{}}\PY{n}{p3}\PY{p}{[}\PY{l+m+mi}{2}\PY{p}{]}\PY{o}{*}\PY{n}{y}\PY{p}{)}\PY{p}{)}\PY{p}{[}\PY{l+m+mi}{0}\PY{p}{]} \PY{c+c1}{\PYZsh{}\PYZsh{} p3}
\PY{n}{kJ} \PY{o}{=} \PY{n}{kJ}\PY{o}{.}\PY{n}{saturation}\PY{p}{(}\PY{n}{S}\PY{o}{.}\PY{n}{ideal}\PY{p}{(}\PY{n}{p4}\PY{p}{[}\PY{l+m+mi}{0}\PY{p}{]}\PY{o}{*}\PY{n}{y}\PY{o}{\PYZhy{}}\PY{n}{p4}\PY{p}{[}\PY{l+m+mi}{1}\PY{p}{]}\PY{o}{*}\PY{n}{x}\PY{p}{,} \PY{n}{p4}\PY{p}{[}\PY{l+m+mi}{0}\PY{p}{]}\PY{o}{*}\PY{n}{z}\PY{o}{\PYZhy{}}\PY{n}{p4}\PY{p}{[}\PY{l+m+mi}{2}\PY{p}{]}\PY{o}{*}\PY{n}{x}\PY{p}{,} \PY{n}{p4}\PY{p}{[}\PY{l+m+mi}{1}\PY{p}{]}\PY{o}{*}\PY{n}{z}\PY{o}{\PYZhy{}}\PY{n}{p4}\PY{p}{[}\PY{l+m+mi}{2}\PY{p}{]}\PY{o}{*}\PY{n}{y}\PY{p}{)}\PY{p}{)}\PY{p}{[}\PY{l+m+mi}{0}\PY{p}{]} \PY{c+c1}{\PYZsh{}\PYZsh{} p4}
\PY{n}{kJ} \PY{o}{=} \PY{n}{kJ}\PY{o}{.}\PY{n}{saturation}\PY{p}{(}\PY{n}{S}\PY{o}{.}\PY{n}{ideal}\PY{p}{(}\PY{n}{p5}\PY{p}{[}\PY{l+m+mi}{0}\PY{p}{]}\PY{o}{*}\PY{n}{y}\PY{o}{\PYZhy{}}\PY{n}{p5}\PY{p}{[}\PY{l+m+mi}{1}\PY{p}{]}\PY{o}{*}\PY{n}{x}\PY{p}{,} \PY{n}{p5}\PY{p}{[}\PY{l+m+mi}{0}\PY{p}{]}\PY{o}{*}\PY{n}{z}\PY{o}{\PYZhy{}}\PY{n}{p5}\PY{p}{[}\PY{l+m+mi}{2}\PY{p}{]}\PY{o}{*}\PY{n}{x}\PY{p}{,} \PY{n}{p5}\PY{p}{[}\PY{l+m+mi}{1}\PY{p}{]}\PY{o}{*}\PY{n}{z}\PY{o}{\PYZhy{}}\PY{n}{p5}\PY{p}{[}\PY{l+m+mi}{2}\PY{p}{]}\PY{o}{*}\PY{n}{y}\PY{p}{)}\PY{p}{)}\PY{p}{[}\PY{l+m+mi}{0}\PY{p}{]} \PY{c+c1}{\PYZsh{}\PYZsh{} p5}
\end{Verbatim}
\end{tcolorbox}

    After these computations, kJ is the ideal of the two reminining
eigenpoints. we want that p1 defined above is an eigenpoint, so the
ideal kkJ here defined must be zero:

    \begin{tcolorbox}[breakable, size=fbox, boxrule=1pt, pad at break*=1mm,colback=cellbackground, colframe=cellborder]
\prompt{In}{incolor}{39}{\boxspacing}
\begin{Verbatim}[commandchars=\\\{\}]
\PY{n}{kkJ} \PY{o}{=} \PY{n}{kJ}\PY{o}{.}\PY{n}{subs}\PY{p}{(}\PY{p}{\PYZob{}}\PY{n}{x}\PY{p}{:}\PY{n}{p6}\PY{p}{[}\PY{l+m+mi}{0}\PY{p}{]}\PY{p}{,} \PY{n}{y}\PY{p}{:}\PY{n}{p6}\PY{p}{[}\PY{l+m+mi}{1}\PY{p}{]}\PY{p}{,} \PY{n}{z}\PY{p}{:}\PY{n}{p6}\PY{p}{[}\PY{l+m+mi}{2}\PY{p}{]}\PY{p}{\PYZcb{}}\PY{p}{)}\PY{o}{.}\PY{n}{radical}\PY{p}{(}\PY{p}{)}
\end{Verbatim}
\end{tcolorbox}

    We saturate kkJ and we get a primary decomposiiton:

    \begin{tcolorbox}[breakable, size=fbox, boxrule=1pt, pad at break*=1mm,colback=cellbackground, colframe=cellborder]
\prompt{In}{incolor}{40}{\boxspacing}
\begin{Verbatim}[commandchars=\\\{\}]
\PY{n}{kkJ} \PY{o}{=} \PY{n}{kkJ}\PY{o}{.}\PY{n}{saturation}\PY{p}{(}\PY{n}{S}\PY{o}{.}\PY{n}{ideal}\PY{p}{(}\PY{n}{matrix}\PY{p}{(}\PY{p}{[}\PY{n}{p1}\PY{p}{,} \PY{n}{p3}\PY{p}{]}\PY{p}{)}\PY{o}{.}\PY{n}{minors}\PY{p}{(}\PY{l+m+mi}{2}\PY{p}{)}\PY{p}{)}\PY{p}{)}\PY{p}{[}\PY{l+m+mi}{0}\PY{p}{]}
\PY{n}{kkJ} \PY{o}{=} \PY{n}{kkJ}\PY{o}{.}\PY{n}{saturation}\PY{p}{(}\PY{n}{S}\PY{o}{.}\PY{n}{ideal}\PY{p}{(}\PY{n}{matrix}\PY{p}{(}\PY{p}{[}\PY{n}{p1}\PY{p}{,} \PY{n}{p4}\PY{p}{]}\PY{p}{)}\PY{o}{.}\PY{n}{minors}\PY{p}{(}\PY{l+m+mi}{2}\PY{p}{)}\PY{p}{)}\PY{p}{)}\PY{p}{[}\PY{l+m+mi}{0}\PY{p}{]}
\PY{n}{kkJ} \PY{o}{=} \PY{n}{kkJ}\PY{o}{.}\PY{n}{saturation}\PY{p}{(}\PY{n}{S}\PY{o}{.}\PY{n}{ideal}\PY{p}{(}\PY{n}{matrix}\PY{p}{(}\PY{p}{[}\PY{n}{p1}\PY{p}{,} \PY{n}{p5}\PY{p}{]}\PY{p}{)}\PY{o}{.}\PY{n}{minors}\PY{p}{(}\PY{l+m+mi}{2}\PY{p}{)}\PY{p}{)}\PY{p}{)}\PY{p}{[}\PY{l+m+mi}{0}\PY{p}{]}
\PY{n}{kkJ} \PY{o}{=} \PY{n}{kkJ}\PY{o}{.}\PY{n}{saturation}\PY{p}{(}\PY{n}{S}\PY{o}{.}\PY{n}{ideal}\PY{p}{(}\PY{n}{matrix}\PY{p}{(}\PY{p}{[}\PY{n}{p3}\PY{p}{,} \PY{n}{p5}\PY{p}{]}\PY{p}{)}\PY{o}{.}\PY{n}{minors}\PY{p}{(}\PY{l+m+mi}{2}\PY{p}{)}\PY{p}{)}\PY{p}{)}\PY{p}{[}\PY{l+m+mi}{0}\PY{p}{]}
\PY{n}{kkJ} \PY{o}{=} \PY{n}{kkJ}\PY{o}{.}\PY{n}{saturation}\PY{p}{(}\PY{n}{S}\PY{o}{.}\PY{n}{ideal}\PY{p}{(}\PY{n}{matrix}\PY{p}{(}\PY{p}{[}\PY{n}{p2}\PY{p}{,} \PY{n}{p6}\PY{p}{]}\PY{p}{)}\PY{o}{.}\PY{n}{minors}\PY{p}{(}\PY{l+m+mi}{2}\PY{p}{)}\PY{p}{)}\PY{p}{)}\PY{p}{[}\PY{l+m+mi}{0}\PY{p}{]}
\PY{n}{kkJ} \PY{o}{=} \PY{n}{kkJ}\PY{o}{.}\PY{n}{saturation}\PY{p}{(}\PY{n}{S}\PY{o}{.}\PY{n}{ideal}\PY{p}{(}\PY{n}{matrix}\PY{p}{(}\PY{p}{[}\PY{n}{p4}\PY{p}{,} \PY{n}{p6}\PY{p}{]}\PY{p}{)}\PY{o}{.}\PY{n}{minors}\PY{p}{(}\PY{l+m+mi}{2}\PY{p}{)}\PY{p}{)}\PY{p}{)}\PY{p}{[}\PY{l+m+mi}{0}\PY{p}{]}
\end{Verbatim}
\end{tcolorbox}

    kkJ is (1), so there are no solutions:

    \begin{tcolorbox}[breakable, size=fbox, boxrule=1pt, pad at break*=1mm,colback=cellbackground, colframe=cellborder]
\prompt{In}{incolor}{41}{\boxspacing}
\begin{Verbatim}[commandchars=\\\{\}]
\PY{k}{assert}\PY{p}{(}\PY{n}{kkJ} \PY{o}{==} \PY{n}{S}\PY{o}{.}\PY{n}{ideal}\PY{p}{(}\PY{l+m+mi}{1}\PY{p}{)}\PY{p}{)}
\end{Verbatim}
\end{tcolorbox}

    CONCLUSION (for the case \(C_2 \neq 0\)) when the three deltas are zero,
(hence rank of the matrix of the five points is 8), we have also
\(\delta_2 = 0\) and we have the collinearities (1, 2, 3) (1, 4, 5). We
have a sub-case in which there is the further collinearity (1, 6, 7). No
other collinearities among the 7 eigenpoints are possible.

    \hypertarget{we-assume-c_2-0}{%
\subsection{\texorpdfstring{We assume
\(C_2 = 0\)}{We assume C\_2 = 0}}\label{we-assume-c_2-0}}

    \begin{tcolorbox}[breakable, size=fbox, boxrule=1pt, pad at break*=1mm,colback=cellbackground, colframe=cellborder]
\prompt{In}{incolor}{68}{\boxspacing}
\begin{Verbatim}[commandchars=\\\{\}]
\PY{n}{p1} \PY{o}{=} \PY{n}{vector}\PY{p}{(}\PY{n}{S}\PY{p}{,} \PY{p}{(}\PY{l+m+mi}{1}\PY{p}{,} \PY{l+m+mi}{0}\PY{p}{,} \PY{l+m+mi}{0}\PY{p}{)}\PY{p}{)}
\PY{n}{p2} \PY{o}{=} \PY{n}{vector}\PY{p}{(}\PY{n}{S}\PY{p}{,} \PY{p}{(}\PY{n}{A2}\PY{p}{,} \PY{n}{B2}\PY{p}{,} \PY{l+m+mi}{0}\PY{p}{)}\PY{p}{)}
\PY{n}{p4} \PY{o}{=} \PY{n}{vector}\PY{p}{(}\PY{n}{S}\PY{p}{,} \PY{p}{(}\PY{n}{A4}\PY{p}{,} \PY{l+m+mi}{0}\PY{p}{,} \PY{n}{C4}\PY{p}{)}\PY{p}{)}
\end{Verbatim}
\end{tcolorbox}

    We are sure that \(B_2 \neq 0\) (since \(p_2 \neq p_1\)) and
\(C_4 \neq 0\) (since \(p_4 \neq p_1\))

    \begin{tcolorbox}[breakable, size=fbox, boxrule=1pt, pad at break*=1mm,colback=cellbackground, colframe=cellborder]
\prompt{In}{incolor}{69}{\boxspacing}
\begin{Verbatim}[commandchars=\\\{\}]
\PY{n}{p3} \PY{o}{=} \PY{p}{(}
    \PY{p}{(}\PY{n}{scalar\PYZus{}product}\PY{p}{(}\PY{n}{p1}\PY{p}{,} \PY{n}{p2}\PY{p}{)}\PY{o}{\PYZca{}}\PY{l+m+mi}{2} \PY{o}{+} \PY{n}{scalar\PYZus{}product}\PY{p}{(}\PY{n}{p1}\PY{p}{,} \PY{n}{p1}\PY{p}{)}\PY{o}{*}\PY{n}{scalar\PYZus{}product}\PY{p}{(}\PY{n}{p2}\PY{p}{,} \PY{n}{p2}\PY{p}{)}\PY{p}{)}\PY{o}{*}\PY{n}{p1}
    \PY{o}{\PYZhy{}}\PY{l+m+mi}{2}\PY{o}{*}\PY{p}{(}\PY{n}{scalar\PYZus{}product}\PY{p}{(}\PY{n}{p1}\PY{p}{,} \PY{n}{p1}\PY{p}{)}\PY{o}{*}\PY{n}{scalar\PYZus{}product}\PY{p}{(}\PY{n}{p1}\PY{p}{,} \PY{n}{p2}\PY{p}{)}\PY{p}{)}\PY{o}{*}\PY{n}{p2}
\PY{p}{)}
\PY{n}{p5} \PY{o}{=} \PY{p}{(}
    \PY{p}{(}\PY{n}{scalar\PYZus{}product}\PY{p}{(}\PY{n}{p1}\PY{p}{,} \PY{n}{p4}\PY{p}{)}\PY{o}{\PYZca{}}\PY{l+m+mi}{2} \PY{o}{+} \PY{n}{scalar\PYZus{}product}\PY{p}{(}\PY{n}{p1}\PY{p}{,} \PY{n}{p1}\PY{p}{)}\PY{o}{*}\PY{n}{scalar\PYZus{}product}\PY{p}{(}\PY{n}{p4}\PY{p}{,} \PY{n}{p4}\PY{p}{)}\PY{p}{)}\PY{o}{*}\PY{n}{p1}
    \PY{o}{\PYZhy{}}\PY{l+m+mi}{2}\PY{o}{*}\PY{p}{(}\PY{n}{scalar\PYZus{}product}\PY{p}{(}\PY{n}{p1}\PY{p}{,} \PY{n}{p1}\PY{p}{)}\PY{o}{*}\PY{n}{scalar\PYZus{}product}\PY{p}{(}\PY{n}{p1}\PY{p}{,} \PY{n}{p4}\PY{p}{)}\PY{p}{)}\PY{o}{*}\PY{n}{p4}
\PY{p}{)}
\end{Verbatim}
\end{tcolorbox}

    We redefine the points, since \(B_2\) and \(C_4\) are not 0.

    \begin{tcolorbox}[breakable, size=fbox, boxrule=1pt, pad at break*=1mm,colback=cellbackground, colframe=cellborder]
\prompt{In}{incolor}{70}{\boxspacing}
\begin{Verbatim}[commandchars=\\\{\}]
\PY{n}{p3}\PY{p}{,} \PY{n}{p5} \PY{o}{=} \PY{n}{p3}\PY{o}{/}\PY{n}{B2}\PY{p}{,} \PY{n}{p5}\PY{o}{/}\PY{n}{C4}
\end{Verbatim}
\end{tcolorbox}

    \begin{tcolorbox}[breakable, size=fbox, boxrule=1pt, pad at break*=1mm,colback=cellbackground, colframe=cellborder]
\prompt{In}{incolor}{71}{\boxspacing}
\begin{Verbatim}[commandchars=\\\{\}]
\PY{k}{assert}\PY{p}{(}\PY{n}{delta1b}\PY{p}{(}\PY{n}{p1}\PY{p}{,} \PY{n}{p2}\PY{p}{,} \PY{n}{p3}\PY{p}{)} \PY{o}{==} \PY{l+m+mi}{0}\PY{p}{)}
\PY{k}{assert}\PY{p}{(}\PY{n}{delta1b}\PY{p}{(}\PY{n}{p1}\PY{p}{,} \PY{n}{p4}\PY{p}{,} \PY{n}{p5}\PY{p}{)} \PY{o}{==} \PY{l+m+mi}{0}\PY{p}{)}
\PY{c+c1}{\PYZsh{}\PYZsh{} Incidentally, delta2 is also 0:}
\PY{k}{assert}\PY{p}{(}\PY{n}{delta2}\PY{p}{(}\PY{n}{p1}\PY{p}{,} \PY{n}{p2}\PY{p}{,} \PY{n}{p3}\PY{p}{,} \PY{n}{p4}\PY{p}{,} \PY{n}{p5}\PY{p}{)} \PY{o}{==} \PY{l+m+mi}{0}\PY{p}{)}
\end{Verbatim}
\end{tcolorbox}

    An example shows that, in general, \(p_6\) and \(p_7\) are not aligned
with \(p_1\). Here is an example.

    \begin{tcolorbox}[breakable, size=fbox, boxrule=1pt, pad at break*=1mm,colback=cellbackground, colframe=cellborder]
\prompt{In}{incolor}{72}{\boxspacing}
\begin{Verbatim}[commandchars=\\\{\}]
\PY{n}{ss3} \PY{o}{=} \PY{p}{\PYZob{}}\PY{n}{A2}\PY{p}{:}\PY{l+m+mi}{1}\PY{p}{,} \PY{n}{B2}\PY{p}{:}\PY{o}{\PYZhy{}}\PY{l+m+mi}{5}\PY{p}{,} \PY{n}{A4}\PY{p}{:}\PY{l+m+mi}{7}\PY{p}{,} \PY{n}{C4}\PY{p}{:}\PY{o}{\PYZhy{}}\PY{l+m+mi}{5}\PY{p}{\PYZcb{}}
\PY{n}{pp1} \PY{o}{=} \PY{n}{p1}\PY{o}{.}\PY{n}{subs}\PY{p}{(}\PY{n}{ss3}\PY{p}{)}
\PY{n}{pp2} \PY{o}{=} \PY{n}{p2}\PY{o}{.}\PY{n}{subs}\PY{p}{(}\PY{n}{ss3}\PY{p}{)}
\PY{n}{pp3} \PY{o}{=} \PY{n}{p3}\PY{o}{.}\PY{n}{subs}\PY{p}{(}\PY{n}{ss3}\PY{p}{)}
\PY{n}{pp4} \PY{o}{=} \PY{n}{p4}\PY{o}{.}\PY{n}{subs}\PY{p}{(}\PY{n}{ss3}\PY{p}{)}
\PY{n}{pp5} \PY{o}{=} \PY{n}{p5}\PY{o}{.}\PY{n}{subs}\PY{p}{(}\PY{n}{ss3}\PY{p}{)}
\PY{n}{cb} \PY{o}{=} \PY{n}{cubic\PYZus{}from\PYZus{}matrix}\PY{p}{(}
    \PY{n}{condition\PYZus{}matrix}\PY{p}{(}
        \PY{p}{[}\PY{n}{pp1}\PY{p}{,} \PY{n}{pp2}\PY{p}{,} \PY{n}{pp3}\PY{p}{,} \PY{n}{pp4}\PY{p}{,} \PY{n}{pp5}\PY{p}{]}\PY{p}{,}
         \PY{n}{S}\PY{p}{,} 
        \PY{n}{standard}\PY{o}{=}\PY{l+s+s2}{\PYZdq{}}\PY{l+s+s2}{all}\PY{l+s+s2}{\PYZdq{}}
    \PY{p}{)}\PY{o}{.}\PY{n}{stack}\PY{p}{(}
        \PY{n}{matrix}\PY{p}{(}
            \PY{p}{[}
                \PY{p}{[}\PY{l+m+mi}{2}\PY{p}{,} \PY{l+m+mi}{3}\PY{p}{,} \PY{l+m+mi}{4}\PY{p}{,} \PY{l+m+mi}{5}\PY{p}{,} \PY{l+m+mi}{6}\PY{p}{,} \PY{l+m+mi}{7}\PY{p}{,} \PY{l+m+mi}{8}\PY{p}{,} \PY{l+m+mi}{9}\PY{p}{,} \PY{l+m+mi}{1}\PY{p}{,} \PY{l+m+mi}{2}\PY{p}{]}
            \PY{p}{]}
        \PY{p}{)}
    \PY{p}{)}
\PY{p}{)}
\end{Verbatim}
\end{tcolorbox}

    NOW WE WANT TO SEE WHAT HAPPENS IF WE IMPOSE THE ALIGNMENT \(p_1\),
\(p_6\), \(p_7\).

We start with \(p_1\), \(p_2\), \(p_3\), \(p_4\), \(p_5\) as above, such
that \(\delta_1(p_1, p_2, p_4)\), \(\bar{\delta}_1(p_1, p_2, p_3)\) and
\(\bar{\delta}_1(p_1, p_4, p_5)\) are \(0\)

    The matrix \(\Phi(p_1, p_2, p_3, p_4, p_5)\) has rank 8.

    \begin{tcolorbox}[breakable, size=fbox, boxrule=1pt, pad at break*=1mm,colback=cellbackground, colframe=cellborder]
\prompt{In}{incolor}{73}{\boxspacing}
\begin{Verbatim}[commandchars=\\\{\}]
\PY{n}{M} \PY{o}{=} \PY{n}{condition\PYZus{}matrix}\PY{p}{(}\PY{p}{[}\PY{n}{p1}\PY{p}{,} \PY{n}{p2}\PY{p}{,} \PY{n}{p3}\PY{p}{,} \PY{n}{p4}\PY{p}{,} \PY{n}{p5}\PY{p}{]}\PY{p}{,} \PY{n}{S}\PY{p}{,} \PY{n}{standard}\PY{o}{=}\PY{l+s+s2}{\PYZdq{}}\PY{l+s+s2}{all}\PY{l+s+s2}{\PYZdq{}}\PY{p}{)}
\PY{k}{assert}\PY{p}{(}\PY{n}{M}\PY{o}{.}\PY{n}{rank}\PY{p}{(}\PY{p}{)} \PY{o}{==} \PY{l+m+mi}{8}\PY{p}{)}
\end{Verbatim}
\end{tcolorbox}

    We select 8 linearly independent rows:

    \begin{tcolorbox}[breakable, size=fbox, boxrule=1pt, pad at break*=1mm,colback=cellbackground, colframe=cellborder]
\prompt{In}{incolor}{74}{\boxspacing}
\begin{Verbatim}[commandchars=\\\{\}]
\PY{n}{mm} \PY{o}{=} \PY{n}{M}\PY{o}{.}\PY{n}{matrix\PYZus{}from\PYZus{}rows}\PY{p}{(}\PY{p}{[}\PY{l+m+mi}{0}\PY{p}{,} \PY{l+m+mi}{1}\PY{p}{,} \PY{l+m+mi}{3}\PY{p}{,} \PY{l+m+mi}{4}\PY{p}{,} \PY{l+m+mi}{6}\PY{p}{,} \PY{l+m+mi}{7}\PY{p}{,} \PY{l+m+mi}{9}\PY{p}{,} \PY{l+m+mi}{10}\PY{p}{]}\PY{p}{)}
\end{Verbatim}
\end{tcolorbox}

    \begin{tcolorbox}[breakable, size=fbox, boxrule=1pt, pad at break*=1mm,colback=cellbackground, colframe=cellborder]
\prompt{In}{incolor}{75}{\boxspacing}
\begin{Verbatim}[commandchars=\\\{\}]
\PY{c+c1}{\PYZsh{} in general, mm has rank 8}
\PY{k}{assert}\PY{p}{(}\PY{n}{mm}\PY{o}{.}\PY{n}{rank}\PY{p}{(}\PY{p}{)} \PY{o}{==} \PY{l+m+mi}{8}\PY{p}{)}  
\end{Verbatim}
\end{tcolorbox}

    \begin{tcolorbox}[breakable, size=fbox, boxrule=1pt, pad at break*=1mm,colback=cellbackground, colframe=cellborder]
\prompt{In}{incolor}{76}{\boxspacing}
\begin{Verbatim}[commandchars=\\\{\}]
\PY{c+c1}{\PYZsh{} let us see when it is not 8:}
\PY{n}{hj} \PY{o}{=} \PY{n}{S}\PY{o}{.}\PY{n}{ideal}\PY{p}{(}\PY{n}{mm}\PY{o}{.}\PY{n}{minors}\PY{p}{(}\PY{l+m+mi}{8}\PY{p}{)}\PY{p}{)}
\end{Verbatim}
\end{tcolorbox}

    \begin{tcolorbox}[breakable, size=fbox, boxrule=1pt, pad at break*=1mm,colback=cellbackground, colframe=cellborder]
\prompt{In}{incolor}{77}{\boxspacing}
\begin{Verbatim}[commandchars=\\\{\}]
\PY{n}{hj} \PY{o}{=} \PY{n}{hj}\PY{o}{.}\PY{n}{saturation}\PY{p}{(}\PY{n}{S}\PY{o}{.}\PY{n}{ideal}\PY{p}{(}\PY{n}{matrix}\PY{p}{(}\PY{p}{[}\PY{n}{p1}\PY{p}{,} \PY{n}{p2}\PY{p}{]}\PY{p}{)}\PY{o}{.}\PY{n}{minors}\PY{p}{(}\PY{l+m+mi}{2}\PY{p}{)}\PY{p}{)}\PY{p}{)}\PY{p}{[}\PY{l+m+mi}{0}\PY{p}{]}
\PY{n}{hj} \PY{o}{=} \PY{n}{hj}\PY{o}{.}\PY{n}{saturation}\PY{p}{(}\PY{n}{S}\PY{o}{.}\PY{n}{ideal}\PY{p}{(}\PY{n}{matrix}\PY{p}{(}\PY{p}{[}\PY{n}{p1}\PY{p}{,} \PY{n}{p3}\PY{p}{]}\PY{p}{)}\PY{o}{.}\PY{n}{minors}\PY{p}{(}\PY{l+m+mi}{2}\PY{p}{)}\PY{p}{)}\PY{p}{)}\PY{p}{[}\PY{l+m+mi}{0}\PY{p}{]}
\PY{n}{hj} \PY{o}{=} \PY{n}{hj}\PY{o}{.}\PY{n}{saturation}\PY{p}{(}\PY{n}{S}\PY{o}{.}\PY{n}{ideal}\PY{p}{(}\PY{n}{matrix}\PY{p}{(}\PY{p}{[}\PY{n}{p1}\PY{p}{,} \PY{n}{p4}\PY{p}{]}\PY{p}{)}\PY{o}{.}\PY{n}{minors}\PY{p}{(}\PY{l+m+mi}{2}\PY{p}{)}\PY{p}{)}\PY{p}{)}\PY{p}{[}\PY{l+m+mi}{0}\PY{p}{]}
\PY{n}{hj} \PY{o}{=} \PY{n}{hj}\PY{o}{.}\PY{n}{saturation}\PY{p}{(}\PY{n}{S}\PY{o}{.}\PY{n}{ideal}\PY{p}{(}\PY{n}{matrix}\PY{p}{(}\PY{p}{[}\PY{n}{p1}\PY{p}{,} \PY{n}{p5}\PY{p}{]}\PY{p}{)}\PY{o}{.}\PY{n}{minors}\PY{p}{(}\PY{l+m+mi}{2}\PY{p}{)}\PY{p}{)}\PY{p}{)}\PY{p}{[}\PY{l+m+mi}{0}\PY{p}{]}
\PY{n}{hj} \PY{o}{=} \PY{n}{hj}\PY{o}{.}\PY{n}{saturation}\PY{p}{(}\PY{n}{S}\PY{o}{.}\PY{n}{ideal}\PY{p}{(}\PY{n}{matrix}\PY{p}{(}\PY{p}{[}\PY{n}{p2}\PY{p}{,} \PY{n}{p3}\PY{p}{]}\PY{p}{)}\PY{o}{.}\PY{n}{minors}\PY{p}{(}\PY{l+m+mi}{2}\PY{p}{)}\PY{p}{)}\PY{p}{)}\PY{p}{[}\PY{l+m+mi}{0}\PY{p}{]}
\end{Verbatim}
\end{tcolorbox}

    hj is (1), so mm has always rank 8.

    \begin{tcolorbox}[breakable, size=fbox, boxrule=1pt, pad at break*=1mm,colback=cellbackground, colframe=cellborder]
\prompt{In}{incolor}{82}{\boxspacing}
\begin{Verbatim}[commandchars=\\\{\}]
\PY{k}{assert}\PY{p}{(}\PY{n}{hj} \PY{o}{==} \PY{n}{S}\PY{o}{.}\PY{n}{ideal}\PY{p}{(}\PY{n}{S}\PY{o}{.}\PY{n}{one}\PY{p}{(}\PY{p}{)}\PY{p}{)}\PY{p}{)}
\end{Verbatim}
\end{tcolorbox}

    Hence the order 8 minor mm has always rank 8. As above, we construct
mmA\_1 and mmB\_1. The construction is the same as above.

    \begin{tcolorbox}[breakable, size=fbox, boxrule=1pt, pad at break*=1mm,colback=cellbackground, colframe=cellborder]
\prompt{In}{incolor}{83}{\boxspacing}
\begin{Verbatim}[commandchars=\\\{\}]
\PY{n}{mmA\PYZus{}1} \PY{o}{=} \PY{n}{mm}\PY{o}{.}\PY{n}{stack}\PY{p}{(}\PY{n}{matrix}\PY{p}{(}\PY{p}{[}\PY{l+m+mi}{1}\PY{p}{,} \PY{l+m+mi}{2}\PY{p}{,} \PY{l+m+mi}{5}\PY{p}{,} \PY{l+m+mi}{6}\PY{p}{,} \PY{l+m+mi}{0}\PY{p}{,} \PY{l+m+mi}{2}\PY{p}{,} \PY{l+m+mi}{3}\PY{p}{,} \PY{l+m+mi}{4}\PY{p}{,} \PY{l+m+mi}{9}\PY{p}{,} \PY{l+m+mi}{11}\PY{p}{]}\PY{p}{)}\PY{p}{)}
\PY{n}{mmB\PYZus{}1} \PY{o}{=} \PY{n}{mm}\PY{o}{.}\PY{n}{stack}\PY{p}{(}\PY{n}{matrix}\PY{p}{(}\PY{p}{[}\PY{o}{\PYZhy{}}\PY{l+m+mi}{1}\PY{p}{,} \PY{l+m+mi}{3}\PY{p}{,} \PY{l+m+mi}{6}\PY{p}{,} \PY{l+m+mi}{5}\PY{p}{,} \PY{l+m+mi}{0}\PY{p}{,} \PY{l+m+mi}{1}\PY{p}{,} \PY{l+m+mi}{3}\PY{p}{,} \PY{l+m+mi}{7}\PY{p}{,} \PY{l+m+mi}{9}\PY{p}{,} \PY{o}{\PYZhy{}}\PY{l+m+mi}{5}\PY{p}{]}\PY{p}{)}\PY{p}{)}
\end{Verbatim}
\end{tcolorbox}

    These two matrices have rank 9

    \begin{tcolorbox}[breakable, size=fbox, boxrule=1pt, pad at break*=1mm,colback=cellbackground, colframe=cellborder]
\prompt{In}{incolor}{84}{\boxspacing}
\begin{Verbatim}[commandchars=\\\{\}]
\PY{k}{assert}\PY{p}{(}\PY{n}{mmA\PYZus{}1}\PY{o}{.}\PY{n}{rank}\PY{p}{(}\PY{p}{)} \PY{o}{==} \PY{l+m+mi}{9}\PY{p}{)}
\PY{k}{assert}\PY{p}{(}\PY{n}{mmB\PYZus{}1}\PY{o}{.}\PY{n}{rank}\PY{p}{(}\PY{p}{)} \PY{o}{==} \PY{l+m+mi}{9}\PY{p}{)}
\end{Verbatim}
\end{tcolorbox}

    \begin{tcolorbox}[breakable, size=fbox, boxrule=1pt, pad at break*=1mm,colback=cellbackground, colframe=cellborder]
\prompt{In}{incolor}{87}{\boxspacing}
\begin{Verbatim}[commandchars=\\\{\}]
\PY{n}{GA3\PYZus{}1} \PY{o}{=} \PY{n}{mmA\PYZus{}1}\PY{o}{.}\PY{n}{stack}\PY{p}{(}\PY{n}{matrix}\PY{p}{(}\PY{p}{[}\PY{n}{phi}\PY{p}{(}\PY{p}{(}\PY{n}{x}\PY{p}{,} \PY{n}{y}\PY{p}{,} \PY{n}{z}\PY{p}{)}\PY{p}{,} \PY{n}{S}\PY{p}{)}\PY{p}{[}\PY{l+m+mi}{2}\PY{p}{]}\PY{p}{]}\PY{p}{)}\PY{p}{)}\PY{o}{.}\PY{n}{det}\PY{p}{(}\PY{p}{)}
\PY{n}{GB3\PYZus{}1} \PY{o}{=} \PY{n}{mmB\PYZus{}1}\PY{o}{.}\PY{n}{stack}\PY{p}{(}\PY{n}{matrix}\PY{p}{(}\PY{p}{[}\PY{n}{phi}\PY{p}{(}\PY{p}{(}\PY{n}{x}\PY{p}{,} \PY{n}{y}\PY{p}{,} \PY{n}{z}\PY{p}{)}\PY{p}{,} \PY{n}{S}\PY{p}{)}\PY{p}{[}\PY{l+m+mi}{2}\PY{p}{]}\PY{p}{]}\PY{p}{)}\PY{p}{)}\PY{o}{.}\PY{n}{det}\PY{p}{(}\PY{p}{)}
\end{Verbatim}
\end{tcolorbox}

    \begin{tcolorbox}[breakable, size=fbox, boxrule=1pt, pad at break*=1mm,colback=cellbackground, colframe=cellborder]
\prompt{In}{incolor}{88}{\boxspacing}
\begin{Verbatim}[commandchars=\\\{\}]
\PY{n}{rr3\PYZus{}1} \PY{o}{=} \PY{n+nb}{list}\PY{p}{(}
    \PY{n+nb}{filter}\PY{p}{(}
        \PY{k}{lambda} \PY{n}{uu}\PY{p}{:} \PY{n}{w1} \PY{o+ow}{in} \PY{n}{uu}\PY{p}{[}\PY{l+m+mi}{0}\PY{p}{]}\PY{o}{.}\PY{n}{variables}\PY{p}{(}\PY{p}{)}\PY{p}{,}
        \PY{n+nb}{list}\PY{p}{(}\PY{n}{factor}\PY{p}{(}\PY{n}{w1}\PY{o}{*}\PY{n}{GA3\PYZus{}1}\PY{o}{+}\PY{n}{w2}\PY{o}{*}\PY{n}{GB3\PYZus{}1}\PY{p}{)}\PY{p}{)}
    \PY{p}{)}
\PY{p}{)}\PY{p}{[}\PY{l+m+mi}{0}\PY{p}{]}\PY{p}{[}\PY{l+m+mi}{0}\PY{p}{]}

\PY{n}{hh1} \PY{o}{=} \PY{n}{rr3\PYZus{}1}\PY{o}{.}\PY{n}{subs}\PY{p}{(}\PY{p}{\PYZob{}}\PY{n}{x}\PY{p}{:}\PY{l+m+mi}{1}\PY{p}{,} \PY{n}{y}\PY{p}{:}\PY{l+m+mi}{0}\PY{p}{,} \PY{n}{z}\PY{p}{:}\PY{l+m+mi}{0}\PY{p}{\PYZcb{}}\PY{p}{)}
\end{Verbatim}
\end{tcolorbox}

    \begin{tcolorbox}[breakable, size=fbox, boxrule=1pt, pad at break*=1mm,colback=cellbackground, colframe=cellborder]
\prompt{In}{incolor}{91}{\boxspacing}
\begin{Verbatim}[commandchars=\\\{\}]
\PY{n}{mmA\PYZus{}2} \PY{o}{=} \PY{n}{mm}\PY{o}{.}\PY{n}{stack}\PY{p}{(}\PY{n}{matrix}\PY{p}{(}\PY{p}{[}\PY{l+m+mi}{1}\PY{p}{,} \PY{o}{\PYZhy{}}\PY{l+m+mi}{5}\PY{p}{,} \PY{l+m+mi}{1}\PY{p}{,} \PY{l+m+mi}{2}\PY{p}{,} \PY{l+m+mi}{0}\PY{p}{,} \PY{l+m+mi}{1}\PY{p}{,} \PY{o}{\PYZhy{}}\PY{l+m+mi}{2}\PY{p}{,} \PY{l+m+mi}{1}\PY{p}{,} \PY{l+m+mi}{3}\PY{p}{,} \PY{l+m+mi}{7}\PY{p}{]}\PY{p}{)}\PY{p}{)}
\PY{n}{mmB\PYZus{}2} \PY{o}{=} \PY{n}{mm}\PY{o}{.}\PY{n}{stack}\PY{p}{(}\PY{n}{matrix}\PY{p}{(}\PY{p}{[}\PY{o}{\PYZhy{}}\PY{l+m+mi}{1}\PY{p}{,} \PY{o}{\PYZhy{}}\PY{l+m+mi}{1}\PY{p}{,} \PY{l+m+mi}{0}\PY{p}{,} \PY{l+m+mi}{4}\PY{p}{,} \PY{l+m+mi}{0}\PY{p}{,} \PY{l+m+mi}{1}\PY{p}{,} \PY{l+m+mi}{0}\PY{p}{,} \PY{l+m+mi}{1}\PY{p}{,} \PY{l+m+mi}{0}\PY{p}{,} \PY{o}{\PYZhy{}}\PY{l+m+mi}{5}\PY{p}{]}\PY{p}{)}\PY{p}{)}

\PY{n}{GA3\PYZus{}2} \PY{o}{=} \PY{n}{mmA\PYZus{}2}\PY{o}{.}\PY{n}{stack}\PY{p}{(}\PY{n}{matrix}\PY{p}{(}\PY{p}{[}\PY{n}{phi}\PY{p}{(}\PY{p}{(}\PY{n}{x}\PY{p}{,} \PY{n}{y}\PY{p}{,} \PY{n}{z}\PY{p}{)}\PY{p}{,} \PY{n}{S}\PY{p}{)}\PY{p}{[}\PY{l+m+mi}{2}\PY{p}{]}\PY{p}{]}\PY{p}{)}\PY{p}{)}\PY{o}{.}\PY{n}{det}\PY{p}{(}\PY{p}{)}
\PY{n}{GB3\PYZus{}2} \PY{o}{=} \PY{n}{mmB\PYZus{}2}\PY{o}{.}\PY{n}{stack}\PY{p}{(}\PY{n}{matrix}\PY{p}{(}\PY{p}{[}\PY{n}{phi}\PY{p}{(}\PY{p}{(}\PY{n}{x}\PY{p}{,} \PY{n}{y}\PY{p}{,} \PY{n}{z}\PY{p}{)}\PY{p}{,} \PY{n}{S}\PY{p}{)}\PY{p}{[}\PY{l+m+mi}{2}\PY{p}{]}\PY{p}{]}\PY{p}{)}\PY{p}{)}\PY{o}{.}\PY{n}{det}\PY{p}{(}\PY{p}{)}

\PY{n}{rr3\PYZus{}2} \PY{o}{=} \PY{n+nb}{list}\PY{p}{(}
    \PY{n+nb}{filter}\PY{p}{(}
        \PY{k}{lambda} \PY{n}{uu}\PY{p}{:} \PY{n}{w1} \PY{o+ow}{in} \PY{n}{uu}\PY{p}{[}\PY{l+m+mi}{0}\PY{p}{]}\PY{o}{.}\PY{n}{variables}\PY{p}{(}\PY{p}{)}\PY{p}{,}
        \PY{n+nb}{list}\PY{p}{(}\PY{n}{factor}\PY{p}{(}\PY{n}{w1}\PY{o}{*}\PY{n}{GA3\PYZus{}2}\PY{o}{+}\PY{n}{w2}\PY{o}{*}\PY{n}{GB3\PYZus{}2}\PY{p}{)}\PY{p}{)}
    \PY{p}{)}
\PY{p}{)}\PY{p}{[}\PY{l+m+mi}{0}\PY{p}{]}\PY{p}{[}\PY{l+m+mi}{0}\PY{p}{]}
\end{Verbatim}
\end{tcolorbox}

    If \(p_1\), \(p_6\), \(p_7\) are aligned, also the following polynomial
must be zero

    \begin{tcolorbox}[breakable, size=fbox, boxrule=1pt, pad at break*=1mm,colback=cellbackground, colframe=cellborder]
\prompt{In}{incolor}{92}{\boxspacing}
\begin{Verbatim}[commandchars=\\\{\}]
\PY{n}{hh2} \PY{o}{=} \PY{n}{rr3\PYZus{}2}\PY{o}{.}\PY{n}{subs}\PY{p}{(}\PY{p}{\PYZob{}}\PY{n}{x}\PY{p}{:}\PY{l+m+mi}{1}\PY{p}{,} \PY{n}{y}\PY{p}{:}\PY{l+m+mi}{0}\PY{p}{,} \PY{n}{z}\PY{p}{:}\PY{l+m+mi}{0}\PY{p}{\PYZcb{}}\PY{p}{)}
\end{Verbatim}
\end{tcolorbox}

    \begin{tcolorbox}[breakable, size=fbox, boxrule=1pt, pad at break*=1mm,colback=cellbackground, colframe=cellborder]
\prompt{In}{incolor}{93}{\boxspacing}
\begin{Verbatim}[commandchars=\\\{\}]
\PY{n}{r3\PYZus{}1} \PY{o}{=} \PY{p}{(}\PY{n}{w1}\PY{o}{*}\PY{n}{GA3\PYZus{}1}\PY{o}{+}\PY{n}{w2}\PY{o}{*}\PY{n}{GB3\PYZus{}1}\PY{p}{)}\PY{o}{.}\PY{n}{subs}\PY{p}{(}
    \PY{p}{\PYZob{}}
        \PY{n}{w1}\PY{p}{:} \PY{n}{hh1}\PY{o}{.}\PY{n}{coefficient}\PY{p}{(}\PY{n}{w2}\PY{p}{)}\PY{p}{,}
        \PY{n}{w2}\PY{p}{:} \PY{o}{\PYZhy{}}\PY{n}{hh1}\PY{o}{.}\PY{n}{coefficient}\PY{p}{(}\PY{n}{w1}\PY{p}{)}
    \PY{p}{\PYZcb{}}
\PY{p}{)}\PY{o}{.}\PY{n}{factor}\PY{p}{(}\PY{p}{)}\PY{p}{[}\PY{o}{\PYZhy{}}\PY{l+m+mi}{1}\PY{p}{]}\PY{p}{[}\PY{l+m+mi}{0}\PY{p}{]}

\PY{n}{r3\PYZus{}2} \PY{o}{=} \PY{p}{(}\PY{n}{w1}\PY{o}{*}\PY{n}{GA3\PYZus{}2}\PY{o}{+}\PY{n}{w2}\PY{o}{*}\PY{n}{GB3\PYZus{}2}\PY{p}{)}\PY{o}{.}\PY{n}{subs}\PY{p}{(}
    \PY{p}{\PYZob{}}
        \PY{n}{w1}\PY{p}{:} \PY{n}{hh2}\PY{o}{.}\PY{n}{coefficient}\PY{p}{(}\PY{n}{w2}\PY{p}{)}\PY{p}{,}
        \PY{n}{w2}\PY{p}{:} \PY{o}{\PYZhy{}}\PY{n}{hh2}\PY{o}{.}\PY{n}{coefficient}\PY{p}{(}\PY{n}{w1}\PY{p}{)}
    \PY{p}{\PYZcb{}}
\PY{p}{)}\PY{o}{.}\PY{n}{factor}\PY{p}{(}\PY{p}{)}\PY{p}{[}\PY{o}{\PYZhy{}}\PY{l+m+mi}{1}\PY{p}{]}\PY{p}{[}\PY{l+m+mi}{0}\PY{p}{]}
\end{Verbatim}
\end{tcolorbox}

    (i.e.~r3\_1 and r3\_2 are the line passing through p1, p6, p7. They
should be equal, because they should not depend of the two points of the
line L chosen. Indeed, they are equal:

    \begin{tcolorbox}[breakable, size=fbox, boxrule=1pt, pad at break*=1mm,colback=cellbackground, colframe=cellborder]
\prompt{In}{incolor}{96}{\boxspacing}
\begin{Verbatim}[commandchars=\\\{\}]
\PY{k}{assert}\PY{p}{(}\PY{n}{r3\PYZus{}1} \PY{o}{==} \PY{n}{r3\PYZus{}2}\PY{p}{)}
\end{Verbatim}
\end{tcolorbox}

    \begin{tcolorbox}[breakable, size=fbox, boxrule=1pt, pad at break*=1mm,colback=cellbackground, colframe=cellborder]
\prompt{In}{incolor}{97}{\boxspacing}
\begin{Verbatim}[commandchars=\\\{\}]
\PY{n}{HH} \PY{o}{=} \PY{p}{[}\PY{n}{hh1}\PY{p}{,} \PY{n}{hh2}\PY{p}{]}
\PY{n}{JJ} \PY{o}{=} \PY{n}{S}\PY{o}{.}\PY{n}{ideal}\PY{p}{(}\PY{p}{[}\PY{n}{hh}\PY{o}{.}\PY{n}{coefficient}\PY{p}{(}\PY{n}{w1}\PY{p}{)} \PY{k}{for} \PY{n}{hh} \PY{o+ow}{in} \PY{n}{HH}\PY{p}{]}\PY{o}{+}\PY{p}{[}\PY{n}{hh}\PY{o}{.}\PY{n}{coefficient}\PY{p}{(}\PY{n}{w2}\PY{p}{)} \PY{k}{for} \PY{n}{hh} \PY{o+ow}{in} \PY{n}{HH}\PY{p}{]}\PY{p}{)}
\end{Verbatim}
\end{tcolorbox}

    \begin{tcolorbox}[breakable, size=fbox, boxrule=1pt, pad at break*=1mm,colback=cellbackground, colframe=cellborder]
\prompt{In}{incolor}{98}{\boxspacing}
\begin{Verbatim}[commandchars=\\\{\}]
\PY{n}{JJ} \PY{o}{=} \PY{n}{JJ}\PY{o}{.}\PY{n}{saturation}\PY{p}{(}\PY{n}{S}\PY{o}{.}\PY{n}{ideal}\PY{p}{(}\PY{n}{matrix}\PY{p}{(}\PY{p}{[}\PY{n}{p1}\PY{p}{,} \PY{n}{p2}\PY{p}{]}\PY{p}{)}\PY{o}{.}\PY{n}{minors}\PY{p}{(}\PY{l+m+mi}{2}\PY{p}{)}\PY{p}{)}\PY{p}{)}\PY{p}{[}\PY{l+m+mi}{0}\PY{p}{]}
\PY{n}{JJ} \PY{o}{=} \PY{n}{JJ}\PY{o}{.}\PY{n}{saturation}\PY{p}{(}\PY{n}{S}\PY{o}{.}\PY{n}{ideal}\PY{p}{(}\PY{n}{matrix}\PY{p}{(}\PY{p}{[}\PY{n}{p1}\PY{p}{,} \PY{n}{p3}\PY{p}{]}\PY{p}{)}\PY{o}{.}\PY{n}{minors}\PY{p}{(}\PY{l+m+mi}{2}\PY{p}{)}\PY{p}{)}\PY{p}{)}\PY{p}{[}\PY{l+m+mi}{0}\PY{p}{]}
\PY{n}{JJ} \PY{o}{=} \PY{n}{JJ}\PY{o}{.}\PY{n}{saturation}\PY{p}{(}\PY{n}{S}\PY{o}{.}\PY{n}{ideal}\PY{p}{(}\PY{n}{matrix}\PY{p}{(}\PY{p}{[}\PY{n}{p1}\PY{p}{,} \PY{n}{p4}\PY{p}{]}\PY{p}{)}\PY{o}{.}\PY{n}{minors}\PY{p}{(}\PY{l+m+mi}{2}\PY{p}{)}\PY{p}{)}\PY{p}{)}\PY{p}{[}\PY{l+m+mi}{0}\PY{p}{]}
\end{Verbatim}
\end{tcolorbox}

    \begin{tcolorbox}[breakable, size=fbox, boxrule=1pt, pad at break*=1mm,colback=cellbackground, colframe=cellborder]
\prompt{In}{incolor}{100}{\boxspacing}
\begin{Verbatim}[commandchars=\\\{\}]
\PY{k}{assert}\PY{p}{(}\PY{n}{JJ} \PY{o}{==} \PY{n}{S}\PY{o}{.}\PY{n}{ideal}\PY{p}{(}\PY{n}{S}\PY{o}{.}\PY{n}{one}\PY{p}{(}\PY{p}{)}\PY{p}{)}\PY{p}{)}
\end{Verbatim}
\end{tcolorbox}

    This computation shows that there are no exceptions to consider.

    \begin{tcolorbox}[breakable, size=fbox, boxrule=1pt, pad at break*=1mm,colback=cellbackground, colframe=cellborder]
\prompt{In}{incolor}{ }{\boxspacing}
\begin{Verbatim}[commandchars=\\\{\}]
\PY{n}{MM\PYZus{}1} \PY{o}{=} \PY{p}{(}\PY{n}{w1}\PY{o}{*}\PY{n}{mmA\PYZus{}1}\PY{o}{+}\PY{n}{w2}\PY{o}{*}\PY{n}{mmB\PYZus{}1}\PY{p}{)}\PY{o}{.}\PY{n}{subs}\PY{p}{(}
    \PY{p}{\PYZob{}}
        \PY{n}{w1}\PY{p}{:} \PY{n}{hh1}\PY{o}{.}\PY{n}{coefficient}\PY{p}{(}\PY{n}{w2}\PY{p}{)}\PY{p}{,}
        \PY{n}{w2}\PY{p}{:} \PY{o}{\PYZhy{}}\PY{n}{hh1}\PY{o}{.}\PY{n}{coefficient}\PY{p}{(}\PY{n}{w1}\PY{p}{)}
    \PY{p}{\PYZcb{}}
\PY{p}{)}

\PY{n}{Mcb1} \PY{o}{=} \PY{n}{MM\PYZus{}1}\PY{o}{.}\PY{n}{stack}\PY{p}{(}\PY{n}{vector}\PY{p}{(}\PY{n}{S}\PY{p}{,} \PY{n}{mon}\PY{p}{)}\PY{p}{)}

\PY{n}{MM\PYZus{}2} \PY{o}{=} \PY{p}{(}\PY{n}{w1}\PY{o}{*}\PY{n}{mmA\PYZus{}2}\PY{o}{+}\PY{n}{w2}\PY{o}{*}\PY{n}{mmB\PYZus{}2}\PY{p}{)}\PY{o}{.}\PY{n}{subs}\PY{p}{(}
    \PY{p}{\PYZob{}}
        \PY{n}{w1}\PY{p}{:} \PY{n}{hh2}\PY{o}{.}\PY{n}{coefficient}\PY{p}{(}\PY{n}{w2}\PY{p}{)}\PY{p}{,}
        \PY{n}{w2}\PY{p}{:} \PY{o}{\PYZhy{}}\PY{n}{hh2}\PY{o}{.}\PY{n}{coefficient}\PY{p}{(}\PY{n}{w1}\PY{p}{)}
    \PY{p}{\PYZcb{}}
\PY{p}{)}

\PY{n}{Mcb2} \PY{o}{=} \PY{n}{MM\PYZus{}2}\PY{o}{.}\PY{n}{stack}\PY{p}{(}\PY{n}{vector}\PY{p}{(}\PY{n}{S}\PY{p}{,} \PY{n}{mon}\PY{p}{)}\PY{p}{)}
\end{Verbatim}
\end{tcolorbox}

    The cubic is the determinant of Mcb1 (or of Mcb2). one possibility is
the following computation (not long)

    \begin{tcolorbox}[breakable, size=fbox, boxrule=1pt, pad at break*=1mm,colback=cellbackground, colframe=cellborder]
\prompt{In}{incolor}{101}{\boxspacing}
\begin{Verbatim}[commandchars=\\\{\}]
\PY{n}{cb1} \PY{o}{=} \PY{n}{Mcb1}\PY{o}{.}\PY{n}{det}\PY{p}{(}\PY{p}{)}\PY{o}{.}\PY{n}{factor}\PY{p}{(}\PY{p}{)}\PY{p}{[}\PY{o}{\PYZhy{}}\PY{l+m+mi}{1}\PY{p}{]}\PY{p}{[}\PY{l+m+mi}{0}\PY{p}{]}
\PY{n}{cb2} \PY{o}{=} \PY{n}{Mcb2}\PY{o}{.}\PY{n}{det}\PY{p}{(}\PY{p}{)}\PY{o}{.}\PY{n}{factor}\PY{p}{(}\PY{p}{)}\PY{p}{[}\PY{o}{\PYZhy{}}\PY{l+m+mi}{1}\PY{p}{]}\PY{p}{[}\PY{l+m+mi}{0}\PY{p}{]}
\end{Verbatim}
\end{tcolorbox}

    \begin{Verbatim}[commandchars=\\\{\}, frame=single, framerule=2mm, rulecolor=\color{outerrorbackground}]
\textcolor{ansi-red-intense}{\textbf{---------------------------------------------------------------------------}}
\textcolor{ansi-red-intense}{\textbf{KeyboardInterrupt}}                         Traceback (most recent call last)
Cell \textcolor{ansi-green-intense}{\textbf{In[101], line 1}}
\textcolor{ansi-green-intense}{\textbf{----> 1}} cb1 \def\tcRGB{\textcolor[RGB]}\expandafter\tcRGB\expandafter{\detokenize{98,98,98}}{=} \setlength{\fboxsep}{0pt}\colorbox{ansi-yellow}{Mcb1\strut}\def\tcRGB{\textcolor[RGB]}\expandafter\tcRGB\expandafter{\detokenize{98,98,98}}{\setlength{\fboxsep}{0pt}\colorbox{ansi-yellow}{.\strut}}\setlength{\fboxsep}{0pt}\colorbox{ansi-yellow}{det\strut}\setlength{\fboxsep}{0pt}\colorbox{ansi-yellow}{(\strut}\setlength{\fboxsep}{0pt}\colorbox{ansi-yellow}{)\strut}\def\tcRGB{\textcolor[RGB]}\expandafter\tcRGB\expandafter{\detokenize{98,98,98}}{.}factor()[\def\tcRGB{\textcolor[RGB]}\expandafter\tcRGB\expandafter{\detokenize{98,98,98}}{-}Integer(\def\tcRGB{\textcolor[RGB]}\expandafter\tcRGB\expandafter{\detokenize{98,98,98}}{1})][Integer(\def\tcRGB{\textcolor[RGB]}\expandafter\tcRGB\expandafter{\detokenize{98,98,98}}{0})]
\textcolor{ansi-green}{      2} cb2 \def\tcRGB{\textcolor[RGB]}\expandafter\tcRGB\expandafter{\detokenize{98,98,98}}{=} Mcb2\def\tcRGB{\textcolor[RGB]}\expandafter\tcRGB\expandafter{\detokenize{98,98,98}}{.}det()\def\tcRGB{\textcolor[RGB]}\expandafter\tcRGB\expandafter{\detokenize{98,98,98}}{.}factor()[\def\tcRGB{\textcolor[RGB]}\expandafter\tcRGB\expandafter{\detokenize{98,98,98}}{-}Integer(\def\tcRGB{\textcolor[RGB]}\expandafter\tcRGB\expandafter{\detokenize{98,98,98}}{1})][Integer(\def\tcRGB{\textcolor[RGB]}\expandafter\tcRGB\expandafter{\detokenize{98,98,98}}{0})]

File \textcolor{ansi-green-intense}{\textbf{\textasciitilde{}/miniforge3/lib/python3.9/site-packages/sage/matrix/matrix2.pyx:1984}}, in \textcolor{ansi-cyan}{sage.matrix.matrix2.Matrix.det (build/cythonized/sage/matrix/matrix2.c:23316)}\textcolor{ansi-blue-intense}{\textbf{()}}
\textcolor{ansi-green}{   1982}         6
\textcolor{ansi-green}{   1983}     """
\textcolor{ansi-green-intense}{\textbf{-> 1984}}     return self.determinant(*args, **kwds)
\textcolor{ansi-green}{   1985} 
\textcolor{ansi-green}{   1986} def determinant(self, algorithm=None):

File \textcolor{ansi-green-intense}{\textbf{\textasciitilde{}/miniforge3/lib/python3.9/site-packages/sage/matrix/matrix\_mpolynomial\_dense.pyx:610}}, in \textcolor{ansi-cyan}{sage.matrix.matrix\_mpolynomial\_dense.Matrix\_mpolynomial\_dense.determinant (build/cythonized/sage/matrix/matrix\_mpolynomial\_dense.cpp:12320)}\textcolor{ansi-blue-intense}{\textbf{()}}
\textcolor{ansi-green}{    608} if isinstance(R, MPolynomialRing\_libsingular) and R.base\_ring().is\_field():
\textcolor{ansi-green}{    609}     singular\_det = singular\_function("det")
\textcolor{ansi-green-intense}{\textbf{--> 610}}     d = singular\_det(self)
\textcolor{ansi-green}{    611} 
\textcolor{ansi-green}{    612} elif can\_convert\_to\_singular(self.base\_ring()):

File \textcolor{ansi-green-intense}{\textbf{\textasciitilde{}/miniforge3/lib/python3.9/site-packages/sage/libs/singular/function.pyx:1298}}, in \textcolor{ansi-cyan}{sage.libs.singular.function.SingularFunction.\_\_call\_\_ (build/cythonized/sage/libs/singular/function.cpp:21339)}\textcolor{ansi-blue-intense}{\textbf{()}}
\textcolor{ansi-green}{   1296}     if not (isinstance(ring, MPolynomialRing\_libsingular) or isinstance(ring, NCPolynomialRing\_plural)):
\textcolor{ansi-green}{   1297}         raise TypeError("cannot call Singular function '\%s' with ring parameter of type '\%s'" \% (self.\_name,type(ring)))
\textcolor{ansi-green-intense}{\textbf{-> 1298}}     return call\_function(self, args, ring, interruptible, attributes)
\textcolor{ansi-green}{   1299} 
\textcolor{ansi-green}{   1300} def \_instancedoc\_(self):

File \textcolor{ansi-green-intense}{\textbf{\textasciitilde{}/miniforge3/lib/python3.9/site-packages/sage/libs/singular/function.pyx:1477}}, in \textcolor{ansi-cyan}{sage.libs.singular.function.call\_function (build/cythonized/sage/libs/singular/function.cpp:23326)}\textcolor{ansi-blue-intense}{\textbf{()}}
\textcolor{ansi-green}{   1475}     error\_messages.pop()
\textcolor{ansi-green}{   1476} 
\textcolor{ansi-green-intense}{\textbf{-> 1477}} with opt\_ctx: \# we are preserving the global options state here
\textcolor{ansi-green}{   1478}     if signal\_handler:
\textcolor{ansi-green}{   1479}         sig\_on()

File \textcolor{ansi-green-intense}{\textbf{\textasciitilde{}/miniforge3/lib/python3.9/site-packages/sage/libs/singular/function.pyx:1479}}, in \textcolor{ansi-cyan}{sage.libs.singular.function.call\_function (build/cythonized/sage/libs/singular/function.cpp:23238)}\textcolor{ansi-blue-intense}{\textbf{()}}
\textcolor{ansi-green}{   1477} with opt\_ctx: \# we are preserving the global options state here
\textcolor{ansi-green}{   1478}     if signal\_handler:
\textcolor{ansi-green-intense}{\textbf{-> 1479}}         sig\_on()
\textcolor{ansi-green}{   1480}         \_res = self.call\_handler.handle\_call(argument\_list, si\_ring)
\textcolor{ansi-green}{   1481}         sig\_off()

\textcolor{ansi-red-intense}{\textbf{KeyboardInterrupt}}: 
    \end{Verbatim}

    \begin{tcolorbox}[breakable, size=fbox, boxrule=1pt, pad at break*=1mm,colback=cellbackground, colframe=cellborder]
\prompt{In}{incolor}{102}{\boxspacing}
\begin{Verbatim}[commandchars=\\\{\}]
\PY{c+c1}{\PYZsh{} here is an alternative:}
\PY{n}{Ms} \PY{o}{=} \PY{p}{[}\PY{p}{]}
\PY{k}{for} \PY{n}{i} \PY{o+ow}{in} \PY{n+nb}{range}\PY{p}{(}\PY{l+m+mi}{10}\PY{p}{)}\PY{p}{:}
    \PY{n}{gd} \PY{o}{=} \PY{n}{gcd}\PY{p}{(}\PY{p}{[}\PY{n}{Mcb1}\PY{p}{[}\PY{n}{i}\PY{p}{,}\PY{n}{j}\PY{p}{]} \PY{k}{for} \PY{n}{j} \PY{o+ow}{in} \PY{n+nb}{range}\PY{p}{(}\PY{l+m+mi}{10}\PY{p}{)}\PY{p}{]}\PY{p}{)}
    \PY{n}{Ms}\PY{o}{.}\PY{n}{append}\PY{p}{(}\PY{p}{[}\PY{n}{Mcb1}\PY{p}{[}\PY{n}{i}\PY{p}{,}\PY{n}{j}\PY{p}{]}\PY{o}{.}\PY{n}{quo\PYZus{}rem}\PY{p}{(}\PY{n}{gd}\PY{p}{)}\PY{p}{[}\PY{l+m+mi}{0}\PY{p}{]} \PY{k}{for} \PY{n}{j} \PY{o+ow}{in} \PY{n+nb}{range}\PY{p}{(}\PY{l+m+mi}{10}\PY{p}{)}\PY{p}{]}\PY{p}{)}

\PY{n}{cb\PYZus{}alt} \PY{o}{=} \PY{n}{matrix}\PY{p}{(}\PY{n}{Ms}\PY{p}{)}\PY{o}{.}\PY{n}{det}\PY{p}{(}\PY{p}{)}\PY{o}{.}\PY{n}{factor}\PY{p}{(}\PY{p}{)}\PY{p}{[}\PY{o}{\PYZhy{}}\PY{l+m+mi}{1}\PY{p}{]}\PY{p}{[}\PY{l+m+mi}{0}\PY{p}{]}
\end{Verbatim}
\end{tcolorbox}

    We have: cb1 = cb2 and cb2 = cb\_alt:

    \begin{tcolorbox}[breakable, size=fbox, boxrule=1pt, pad at break*=1mm,colback=cellbackground, colframe=cellborder]
\prompt{In}{incolor}{103}{\boxspacing}
\begin{Verbatim}[commandchars=\\\{\}]
\PY{k}{assert}\PY{p}{(}\PY{n}{cb1} \PY{o}{==} \PY{n}{cb2}\PY{p}{)}
\PY{k}{assert}\PY{p}{(}\PY{n}{cb1} \PY{o}{==} \PY{n}{cb\PYZus{}alt}\PY{p}{)}
\end{Verbatim}
\end{tcolorbox}

    \begin{Verbatim}[commandchars=\\\{\}, frame=single, framerule=2mm, rulecolor=\color{outerrorbackground}]
\textcolor{ansi-red-intense}{\textbf{---------------------------------------------------------------------------}}
\textcolor{ansi-red-intense}{\textbf{NameError}}                                 Traceback (most recent call last)
Cell \textcolor{ansi-green-intense}{\textbf{In[103], line 1}}
\textcolor{ansi-green-intense}{\textbf{----> 1}} \def\tcRGB{\textcolor[RGB]}\expandafter\tcRGB\expandafter{\detokenize{0,135,0}}{\textbf{assert}}(\setlength{\fboxsep}{0pt}\colorbox{ansi-yellow}{cb1\strut} \def\tcRGB{\textcolor[RGB]}\expandafter\tcRGB\expandafter{\detokenize{98,98,98}}{==} cb2)
\textcolor{ansi-green}{      2} \def\tcRGB{\textcolor[RGB]}\expandafter\tcRGB\expandafter{\detokenize{0,135,0}}{\textbf{assert}}(cb1 \def\tcRGB{\textcolor[RGB]}\expandafter\tcRGB\expandafter{\detokenize{98,98,98}}{==} cb\_alt)

\textcolor{ansi-red-intense}{\textbf{NameError}}: name 'cb1' is not defined
    \end{Verbatim}

    The following example shows that in general there are only the
collinearities (1, 2, 3), (1, 4, 5), (1, 6, 7)

    \begin{tcolorbox}[breakable, size=fbox, boxrule=1pt, pad at break*=1mm,colback=cellbackground, colframe=cellborder]
\prompt{In}{incolor}{ }{\boxspacing}
\begin{Verbatim}[commandchars=\\\{\}]
\PY{n}{ccb} \PY{o}{=} \PY{n}{cb\PYZus{}alt}\PY{o}{.}\PY{n}{subs}\PY{p}{(}\PY{p}{\PYZob{}}\PY{n}{A2}\PY{p}{:}\PY{l+m+mi}{5}\PY{p}{,} \PY{n}{B2}\PY{p}{:}\PY{o}{\PYZhy{}}\PY{l+m+mi}{3}\PY{p}{,} \PY{n}{A4}\PY{p}{:}\PY{l+m+mi}{2}\PY{p}{,} \PY{n}{C4}\PY{p}{:}\PY{o}{\PYZhy{}}\PY{l+m+mi}{7}\PY{p}{\PYZcb{}}\PY{p}{)}
\PY{n}{pp1} \PY{o}{=} \PY{n}{p1}\PY{o}{.}\PY{n}{subs}\PY{p}{(}\PY{p}{\PYZob{}}\PY{n}{A2}\PY{p}{:}\PY{l+m+mi}{5}\PY{p}{,} \PY{n}{B2}\PY{p}{:}\PY{o}{\PYZhy{}}\PY{l+m+mi}{3}\PY{p}{,} \PY{n}{A4}\PY{p}{:}\PY{l+m+mi}{2}\PY{p}{,} \PY{n}{C4}\PY{p}{:}\PY{o}{\PYZhy{}}\PY{l+m+mi}{7}\PY{p}{\PYZcb{}}\PY{p}{)}
\PY{n}{pp2} \PY{o}{=} \PY{n}{p2}\PY{o}{.}\PY{n}{subs}\PY{p}{(}\PY{p}{\PYZob{}}\PY{n}{A2}\PY{p}{:}\PY{l+m+mi}{5}\PY{p}{,} \PY{n}{B2}\PY{p}{:}\PY{o}{\PYZhy{}}\PY{l+m+mi}{3}\PY{p}{,} \PY{n}{A4}\PY{p}{:}\PY{l+m+mi}{2}\PY{p}{,} \PY{n}{C4}\PY{p}{:}\PY{o}{\PYZhy{}}\PY{l+m+mi}{7}\PY{p}{\PYZcb{}}\PY{p}{)}
\PY{n}{pp3} \PY{o}{=} \PY{n}{p3}\PY{o}{.}\PY{n}{subs}\PY{p}{(}\PY{p}{\PYZob{}}\PY{n}{A2}\PY{p}{:}\PY{l+m+mi}{5}\PY{p}{,} \PY{n}{B2}\PY{p}{:}\PY{o}{\PYZhy{}}\PY{l+m+mi}{3}\PY{p}{,} \PY{n}{A4}\PY{p}{:}\PY{l+m+mi}{2}\PY{p}{,} \PY{n}{C4}\PY{p}{:}\PY{o}{\PYZhy{}}\PY{l+m+mi}{7}\PY{p}{\PYZcb{}}\PY{p}{)}
\PY{n}{pp4} \PY{o}{=} \PY{n}{p4}\PY{o}{.}\PY{n}{subs}\PY{p}{(}\PY{p}{\PYZob{}}\PY{n}{A2}\PY{p}{:}\PY{l+m+mi}{5}\PY{p}{,} \PY{n}{B2}\PY{p}{:}\PY{o}{\PYZhy{}}\PY{l+m+mi}{3}\PY{p}{,} \PY{n}{A4}\PY{p}{:}\PY{l+m+mi}{2}\PY{p}{,} \PY{n}{C4}\PY{p}{:}\PY{o}{\PYZhy{}}\PY{l+m+mi}{7}\PY{p}{\PYZcb{}}\PY{p}{)}
\PY{n}{pp5} \PY{o}{=} \PY{n}{p5}\PY{o}{.}\PY{n}{subs}\PY{p}{(}\PY{p}{\PYZob{}}\PY{n}{A2}\PY{p}{:}\PY{l+m+mi}{5}\PY{p}{,} \PY{n}{B2}\PY{p}{:}\PY{o}{\PYZhy{}}\PY{l+m+mi}{3}\PY{p}{,} \PY{n}{A4}\PY{p}{:}\PY{l+m+mi}{2}\PY{p}{,} \PY{n}{C4}\PY{p}{:}\PY{o}{\PYZhy{}}\PY{l+m+mi}{7}\PY{p}{\PYZcb{}}\PY{p}{)}
\end{Verbatim}
\end{tcolorbox}

    HERE WE CONCLUDE THE FIRST PART OF THE COMPUTATION:

In case \(C_2 = 0\), IT IS POSSIBLE TO HAVE THREE ALIGNMENTS: (1, 2, 3),
(1, 4, 5), (1, 6, 7)

    Now we want to see if it is possible to have more then three alignments
(We continue to assume \(C2 = 0\))

Recall that cb\_alt is our cubic.

Recall that r3\_1 (= r3\_2) is the line through \(p_6\) and \(p_7\).

Up to a permutation of the indices of the points, if there is another
alignment among the eigenpoints, \(p_6\) must be on the line
\(p_2 \vee p_4\). Hence we can find it, since is the intersection of
\(p_2 \vee p_4\) and r3.

    \begin{tcolorbox}[breakable, size=fbox, boxrule=1pt, pad at break*=1mm,colback=cellbackground, colframe=cellborder]
\prompt{In}{incolor}{106}{\boxspacing}
\begin{Verbatim}[commandchars=\\\{\}]
\PY{n}{r24} \PY{o}{=} \PY{n}{det}\PY{p}{(}\PY{n}{matrix}\PY{p}{(}\PY{p}{[}\PY{n}{p2}\PY{p}{,} \PY{n}{p4}\PY{p}{,} \PY{p}{(}\PY{n}{x}\PY{p}{,} \PY{n}{y}\PY{p}{,} \PY{n}{z}\PY{p}{)}\PY{p}{]}\PY{p}{)}\PY{p}{)}
\end{Verbatim}
\end{tcolorbox}

    \begin{tcolorbox}[breakable, size=fbox, boxrule=1pt, pad at break*=1mm,colback=cellbackground, colframe=cellborder]
\prompt{In}{incolor}{107}{\boxspacing}
\begin{Verbatim}[commandchars=\\\{\}]
\PY{n}{E1} \PY{o}{=} \PY{n}{matrix}\PY{p}{(}
    \PY{p}{[}
        \PY{p}{[}\PY{n}{r3\PYZus{}1}\PY{o}{.}\PY{n}{coefficient}\PY{p}{(}\PY{n}{xx}\PY{p}{)} \PY{k}{for} \PY{n}{xx} \PY{o+ow}{in} \PY{p}{[}\PY{n}{x}\PY{p}{,} \PY{n}{y}\PY{p}{,} \PY{n}{z}\PY{p}{]}\PY{p}{]}\PY{p}{,}
        \PY{p}{[}\PY{n}{r24}\PY{o}{.}\PY{n}{coefficient}\PY{p}{(}\PY{n}{xx}\PY{p}{)} \PY{k}{for} \PY{n}{xx} \PY{o+ow}{in} \PY{p}{[}\PY{n}{x}\PY{p}{,} \PY{n}{y}\PY{p}{,} \PY{n}{z}\PY{p}{]}\PY{p}{]}
    \PY{p}{]}
\PY{p}{)}\PY{o}{.}\PY{n}{minors}\PY{p}{(}\PY{l+m+mi}{2}\PY{p}{)}
\end{Verbatim}
\end{tcolorbox}

    \begin{tcolorbox}[breakable, size=fbox, boxrule=1pt, pad at break*=1mm,colback=cellbackground, colframe=cellborder]
\prompt{In}{incolor}{108}{\boxspacing}
\begin{Verbatim}[commandchars=\\\{\}]
\PY{n}{p6} \PY{o}{=} \PY{n}{vector}\PY{p}{(}\PY{n}{S}\PY{p}{,} \PY{p}{(}\PY{n}{E1}\PY{p}{[}\PY{l+m+mi}{2}\PY{p}{]}\PY{p}{,} \PY{o}{\PYZhy{}}\PY{n}{E1}\PY{p}{[}\PY{l+m+mi}{1}\PY{p}{]}\PY{p}{,} \PY{n}{E1}\PY{p}{[}\PY{l+m+mi}{0}\PY{p}{]}\PY{p}{)}\PY{p}{)}
\end{Verbatim}
\end{tcolorbox}

    \begin{tcolorbox}[breakable, size=fbox, boxrule=1pt, pad at break*=1mm,colback=cellbackground, colframe=cellborder]
\prompt{In}{incolor}{109}{\boxspacing}
\begin{Verbatim}[commandchars=\\\{\}]
\PY{n}{kJ} \PY{o}{=} \PY{n}{S}\PY{o}{.}\PY{n}{ideal}\PY{p}{(}
    \PY{n}{matrix}\PY{p}{(}
        \PY{p}{[}
            \PY{p}{[}\PY{n}{x}\PY{p}{,} \PY{n}{y}\PY{p}{,} \PY{n}{z}\PY{p}{]}\PY{p}{,}
            \PY{p}{[}\PY{n}{cb\PYZus{}alt}\PY{o}{.}\PY{n}{derivative}\PY{p}{(}\PY{n}{x}\PY{p}{)}\PY{p}{,} \PY{n}{cb\PYZus{}alt}\PY{o}{.}\PY{n}{derivative}\PY{p}{(}\PY{n}{y}\PY{p}{)}\PY{p}{,} \PY{n}{cb\PYZus{}alt}\PY{o}{.}\PY{n}{derivative}\PY{p}{(}\PY{n}{z}\PY{p}{)}\PY{p}{]}
        \PY{p}{]}
    \PY{p}{)}\PY{o}{.}\PY{n}{minors}\PY{p}{(}\PY{l+m+mi}{2}\PY{p}{)}
\PY{p}{)}
\end{Verbatim}
\end{tcolorbox}

    \begin{tcolorbox}[breakable, size=fbox, boxrule=1pt, pad at break*=1mm,colback=cellbackground, colframe=cellborder]
\prompt{In}{incolor}{111}{\boxspacing}
\begin{Verbatim}[commandchars=\\\{\}]
\PY{n}{kJ} \PY{o}{=} \PY{n}{kJ}\PY{o}{.}\PY{n}{saturation}\PY{p}{(}\PY{n}{S}\PY{o}{.}\PY{n}{ideal}\PY{p}{(}\PY{n}{z}\PY{p}{,} \PY{n}{y}\PY{p}{)}\PY{p}{)}\PY{p}{[}\PY{l+m+mi}{0}\PY{p}{]}  \PY{c+c1}{\PYZsh{}\PYZsh{}p1}
\PY{n}{kJ} \PY{o}{=} \PY{n}{kJ}\PY{o}{.}\PY{n}{saturation}\PY{p}{(}\PY{n}{S}\PY{o}{.}\PY{n}{ideal}\PY{p}{(}\PY{n}{p2}\PY{p}{[}\PY{l+m+mi}{0}\PY{p}{]}\PY{o}{*}\PY{n}{y}\PY{o}{\PYZhy{}}\PY{n}{p2}\PY{p}{[}\PY{l+m+mi}{1}\PY{p}{]}\PY{o}{*}\PY{n}{x}\PY{p}{,} \PY{n}{p2}\PY{p}{[}\PY{l+m+mi}{0}\PY{p}{]}\PY{o}{*}\PY{n}{z}\PY{o}{\PYZhy{}}\PY{n}{p2}\PY{p}{[}\PY{l+m+mi}{2}\PY{p}{]}\PY{o}{*}\PY{n}{x}\PY{p}{,} \PY{n}{p2}\PY{p}{[}\PY{l+m+mi}{1}\PY{p}{]}\PY{o}{*}\PY{n}{z}\PY{o}{\PYZhy{}}\PY{n}{p2}\PY{p}{[}\PY{l+m+mi}{2}\PY{p}{]}\PY{o}{*}\PY{n}{y}\PY{p}{)}\PY{p}{)}\PY{p}{[}\PY{l+m+mi}{0}\PY{p}{]} \PY{c+c1}{\PYZsh{}\PYZsh{} p2}
\PY{n}{kJ} \PY{o}{=} \PY{n}{kJ}\PY{o}{.}\PY{n}{saturation}\PY{p}{(}\PY{n}{S}\PY{o}{.}\PY{n}{ideal}\PY{p}{(}\PY{n}{p3}\PY{p}{[}\PY{l+m+mi}{0}\PY{p}{]}\PY{o}{*}\PY{n}{y}\PY{o}{\PYZhy{}}\PY{n}{p3}\PY{p}{[}\PY{l+m+mi}{1}\PY{p}{]}\PY{o}{*}\PY{n}{x}\PY{p}{,} \PY{n}{p3}\PY{p}{[}\PY{l+m+mi}{0}\PY{p}{]}\PY{o}{*}\PY{n}{z}\PY{o}{\PYZhy{}}\PY{n}{p3}\PY{p}{[}\PY{l+m+mi}{2}\PY{p}{]}\PY{o}{*}\PY{n}{x}\PY{p}{,} \PY{n}{p3}\PY{p}{[}\PY{l+m+mi}{1}\PY{p}{]}\PY{o}{*}\PY{n}{z}\PY{o}{\PYZhy{}}\PY{n}{p3}\PY{p}{[}\PY{l+m+mi}{2}\PY{p}{]}\PY{o}{*}\PY{n}{y}\PY{p}{)}\PY{p}{)}\PY{p}{[}\PY{l+m+mi}{0}\PY{p}{]} \PY{c+c1}{\PYZsh{}\PYZsh{} p3}
\PY{n}{kJ} \PY{o}{=} \PY{n}{kJ}\PY{o}{.}\PY{n}{saturation}\PY{p}{(}\PY{n}{S}\PY{o}{.}\PY{n}{ideal}\PY{p}{(}\PY{n}{p4}\PY{p}{[}\PY{l+m+mi}{0}\PY{p}{]}\PY{o}{*}\PY{n}{y}\PY{o}{\PYZhy{}}\PY{n}{p4}\PY{p}{[}\PY{l+m+mi}{1}\PY{p}{]}\PY{o}{*}\PY{n}{x}\PY{p}{,} \PY{n}{p4}\PY{p}{[}\PY{l+m+mi}{0}\PY{p}{]}\PY{o}{*}\PY{n}{z}\PY{o}{\PYZhy{}}\PY{n}{p4}\PY{p}{[}\PY{l+m+mi}{2}\PY{p}{]}\PY{o}{*}\PY{n}{x}\PY{p}{,} \PY{n}{p4}\PY{p}{[}\PY{l+m+mi}{1}\PY{p}{]}\PY{o}{*}\PY{n}{z}\PY{o}{\PYZhy{}}\PY{n}{p4}\PY{p}{[}\PY{l+m+mi}{2}\PY{p}{]}\PY{o}{*}\PY{n}{y}\PY{p}{)}\PY{p}{)}\PY{p}{[}\PY{l+m+mi}{0}\PY{p}{]} \PY{c+c1}{\PYZsh{}\PYZsh{} p4}
\PY{n}{kJ} \PY{o}{=} \PY{n}{kJ}\PY{o}{.}\PY{n}{saturation}\PY{p}{(}\PY{n}{S}\PY{o}{.}\PY{n}{ideal}\PY{p}{(}\PY{n}{p5}\PY{p}{[}\PY{l+m+mi}{0}\PY{p}{]}\PY{o}{*}\PY{n}{y}\PY{o}{\PYZhy{}}\PY{n}{p5}\PY{p}{[}\PY{l+m+mi}{1}\PY{p}{]}\PY{o}{*}\PY{n}{x}\PY{p}{,} \PY{n}{p5}\PY{p}{[}\PY{l+m+mi}{0}\PY{p}{]}\PY{o}{*}\PY{n}{z}\PY{o}{\PYZhy{}}\PY{n}{p5}\PY{p}{[}\PY{l+m+mi}{2}\PY{p}{]}\PY{o}{*}\PY{n}{x}\PY{p}{,} \PY{n}{p5}\PY{p}{[}\PY{l+m+mi}{1}\PY{p}{]}\PY{o}{*}\PY{n}{z}\PY{o}{\PYZhy{}}\PY{n}{p5}\PY{p}{[}\PY{l+m+mi}{2}\PY{p}{]}\PY{o}{*}\PY{n}{y}\PY{p}{)}\PY{p}{)}\PY{p}{[}\PY{l+m+mi}{0}\PY{p}{]} \PY{c+c1}{\PYZsh{}\PYZsh{} p5}
\PY{n}{kJ} \PY{o}{=} \PY{n}{kJ}\PY{o}{.}\PY{n}{saturation}\PY{p}{(}\PY{n}{S}\PY{o}{.}\PY{n}{ideal}\PY{p}{(}\PY{n}{matrix}\PY{p}{(}\PY{p}{[}\PY{n}{p1}\PY{p}{,} \PY{n}{p2}\PY{p}{]}\PY{p}{)}\PY{o}{.}\PY{n}{minors}\PY{p}{(}\PY{l+m+mi}{2}\PY{p}{)}\PY{p}{)}\PY{p}{)}\PY{p}{[}\PY{l+m+mi}{0}\PY{p}{]}
\PY{n}{kJ} \PY{o}{=} \PY{n}{kJ}\PY{o}{.}\PY{n}{saturation}\PY{p}{(}\PY{n}{S}\PY{o}{.}\PY{n}{ideal}\PY{p}{(}\PY{n}{matrix}\PY{p}{(}\PY{p}{[}\PY{n}{p1}\PY{p}{,} \PY{n}{p3}\PY{p}{]}\PY{p}{)}\PY{o}{.}\PY{n}{minors}\PY{p}{(}\PY{l+m+mi}{2}\PY{p}{)}\PY{p}{)}\PY{p}{)}\PY{p}{[}\PY{l+m+mi}{0}\PY{p}{]}
\PY{n}{kJ} \PY{o}{=} \PY{n}{kJ}\PY{o}{.}\PY{n}{saturation}\PY{p}{(}\PY{n}{S}\PY{o}{.}\PY{n}{ideal}\PY{p}{(}\PY{n}{matrix}\PY{p}{(}\PY{p}{[}\PY{n}{p1}\PY{p}{,} \PY{n}{p4}\PY{p}{]}\PY{p}{)}\PY{o}{.}\PY{n}{minors}\PY{p}{(}\PY{l+m+mi}{2}\PY{p}{)}\PY{p}{)}\PY{p}{)}\PY{p}{[}\PY{l+m+mi}{0}\PY{p}{]}
\PY{n}{kJ} \PY{o}{=} \PY{n}{kJ}\PY{o}{.}\PY{n}{saturation}\PY{p}{(}\PY{n}{S}\PY{o}{.}\PY{n}{ideal}\PY{p}{(}\PY{n}{matrix}\PY{p}{(}\PY{p}{[}\PY{n}{p1}\PY{p}{,} \PY{n}{p5}\PY{p}{]}\PY{p}{)}\PY{o}{.}\PY{n}{minors}\PY{p}{(}\PY{l+m+mi}{2}\PY{p}{)}\PY{p}{)}\PY{p}{)}\PY{p}{[}\PY{l+m+mi}{0}\PY{p}{]}
\end{Verbatim}
\end{tcolorbox}

    \begin{Verbatim}[commandchars=\\\{\}]
// ** `sat\_with\_exp` in use, can not be killed
// ** redefining id (parameter def id; parameter ideal j;  )
elim.lib::sat\_with\_exp:745
// ** redefining j (parameter def id; parameter ideal j;  )
elim.lib::sat\_with\_exp:745
// ** redefining ii (   int ii,kk;) elim.lib::sat\_with\_exp:746
// ** redefining kk (   int ii,kk;) elim.lib::sat\_with\_exp:746
// ** redefining i (   def i=id;) elim.lib::sat\_with\_exp:747
// ** redefining p (   int p = printlevel-voice+2;  // p=printlevel (default:
p=0)) elim.lib::sat\_with\_exp:749
    \end{Verbatim}

    \begin{Verbatim}[commandchars=\\\{\}, frame=single, framerule=2mm, rulecolor=\color{outerrorbackground}]
\textcolor{ansi-red-intense}{\textbf{---------------------------------------------------------------------------}}
\textcolor{ansi-red-intense}{\textbf{KeyboardInterrupt}}                         Traceback (most recent call last)
Cell \textcolor{ansi-green-intense}{\textbf{In[111], line 6}}
\textcolor{ansi-green}{      4} kJ \def\tcRGB{\textcolor[RGB]}\expandafter\tcRGB\expandafter{\detokenize{98,98,98}}{=} kJ\def\tcRGB{\textcolor[RGB]}\expandafter\tcRGB\expandafter{\detokenize{98,98,98}}{.}saturation(S\def\tcRGB{\textcolor[RGB]}\expandafter\tcRGB\expandafter{\detokenize{98,98,98}}{.}ideal(p4[Integer(\def\tcRGB{\textcolor[RGB]}\expandafter\tcRGB\expandafter{\detokenize{98,98,98}}{0})]\def\tcRGB{\textcolor[RGB]}\expandafter\tcRGB\expandafter{\detokenize{98,98,98}}{*}y\def\tcRGB{\textcolor[RGB]}\expandafter\tcRGB\expandafter{\detokenize{98,98,98}}{-}p4[Integer(\def\tcRGB{\textcolor[RGB]}\expandafter\tcRGB\expandafter{\detokenize{98,98,98}}{1})]\def\tcRGB{\textcolor[RGB]}\expandafter\tcRGB\expandafter{\detokenize{98,98,98}}{*}x, p4[Integer(\def\tcRGB{\textcolor[RGB]}\expandafter\tcRGB\expandafter{\detokenize{98,98,98}}{0})]\def\tcRGB{\textcolor[RGB]}\expandafter\tcRGB\expandafter{\detokenize{98,98,98}}{*}z\def\tcRGB{\textcolor[RGB]}\expandafter\tcRGB\expandafter{\detokenize{98,98,98}}{-}p4[Integer(\def\tcRGB{\textcolor[RGB]}\expandafter\tcRGB\expandafter{\detokenize{98,98,98}}{2})]\def\tcRGB{\textcolor[RGB]}\expandafter\tcRGB\expandafter{\detokenize{98,98,98}}{*}x, p4[Integer(\def\tcRGB{\textcolor[RGB]}\expandafter\tcRGB\expandafter{\detokenize{98,98,98}}{1})]\def\tcRGB{\textcolor[RGB]}\expandafter\tcRGB\expandafter{\detokenize{98,98,98}}{*}z\def\tcRGB{\textcolor[RGB]}\expandafter\tcRGB\expandafter{\detokenize{98,98,98}}{-}p4[Integer(\def\tcRGB{\textcolor[RGB]}\expandafter\tcRGB\expandafter{\detokenize{98,98,98}}{2})]\def\tcRGB{\textcolor[RGB]}\expandafter\tcRGB\expandafter{\detokenize{98,98,98}}{*}y))[Integer(\def\tcRGB{\textcolor[RGB]}\expandafter\tcRGB\expandafter{\detokenize{98,98,98}}{0})] \def\tcRGB{\textcolor[RGB]}\expandafter\tcRGB\expandafter{\detokenize{95,135,135}}{\#\# p4}
\textcolor{ansi-green}{      5} kJ \def\tcRGB{\textcolor[RGB]}\expandafter\tcRGB\expandafter{\detokenize{98,98,98}}{=} kJ\def\tcRGB{\textcolor[RGB]}\expandafter\tcRGB\expandafter{\detokenize{98,98,98}}{.}saturation(S\def\tcRGB{\textcolor[RGB]}\expandafter\tcRGB\expandafter{\detokenize{98,98,98}}{.}ideal(p5[Integer(\def\tcRGB{\textcolor[RGB]}\expandafter\tcRGB\expandafter{\detokenize{98,98,98}}{0})]\def\tcRGB{\textcolor[RGB]}\expandafter\tcRGB\expandafter{\detokenize{98,98,98}}{*}y\def\tcRGB{\textcolor[RGB]}\expandafter\tcRGB\expandafter{\detokenize{98,98,98}}{-}p5[Integer(\def\tcRGB{\textcolor[RGB]}\expandafter\tcRGB\expandafter{\detokenize{98,98,98}}{1})]\def\tcRGB{\textcolor[RGB]}\expandafter\tcRGB\expandafter{\detokenize{98,98,98}}{*}x, p5[Integer(\def\tcRGB{\textcolor[RGB]}\expandafter\tcRGB\expandafter{\detokenize{98,98,98}}{0})]\def\tcRGB{\textcolor[RGB]}\expandafter\tcRGB\expandafter{\detokenize{98,98,98}}{*}z\def\tcRGB{\textcolor[RGB]}\expandafter\tcRGB\expandafter{\detokenize{98,98,98}}{-}p5[Integer(\def\tcRGB{\textcolor[RGB]}\expandafter\tcRGB\expandafter{\detokenize{98,98,98}}{2})]\def\tcRGB{\textcolor[RGB]}\expandafter\tcRGB\expandafter{\detokenize{98,98,98}}{*}x, p5[Integer(\def\tcRGB{\textcolor[RGB]}\expandafter\tcRGB\expandafter{\detokenize{98,98,98}}{1})]\def\tcRGB{\textcolor[RGB]}\expandafter\tcRGB\expandafter{\detokenize{98,98,98}}{*}z\def\tcRGB{\textcolor[RGB]}\expandafter\tcRGB\expandafter{\detokenize{98,98,98}}{-}p5[Integer(\def\tcRGB{\textcolor[RGB]}\expandafter\tcRGB\expandafter{\detokenize{98,98,98}}{2})]\def\tcRGB{\textcolor[RGB]}\expandafter\tcRGB\expandafter{\detokenize{98,98,98}}{*}y))[Integer(\def\tcRGB{\textcolor[RGB]}\expandafter\tcRGB\expandafter{\detokenize{98,98,98}}{0})] \def\tcRGB{\textcolor[RGB]}\expandafter\tcRGB\expandafter{\detokenize{95,135,135}}{\#\# p5}
\textcolor{ansi-green-intense}{\textbf{----> 6}} kJ \def\tcRGB{\textcolor[RGB]}\expandafter\tcRGB\expandafter{\detokenize{98,98,98}}{=} \setlength{\fboxsep}{0pt}\colorbox{ansi-yellow}{kJ\strut}\def\tcRGB{\textcolor[RGB]}\expandafter\tcRGB\expandafter{\detokenize{98,98,98}}{\setlength{\fboxsep}{0pt}\colorbox{ansi-yellow}{.\strut}}\setlength{\fboxsep}{0pt}\colorbox{ansi-yellow}{saturation\strut}\setlength{\fboxsep}{0pt}\colorbox{ansi-yellow}{(\strut}\setlength{\fboxsep}{0pt}\colorbox{ansi-yellow}{S\strut}\def\tcRGB{\textcolor[RGB]}\expandafter\tcRGB\expandafter{\detokenize{98,98,98}}{\setlength{\fboxsep}{0pt}\colorbox{ansi-yellow}{.\strut}}\setlength{\fboxsep}{0pt}\colorbox{ansi-yellow}{ideal\strut}\setlength{\fboxsep}{0pt}\colorbox{ansi-yellow}{(\strut}\setlength{\fboxsep}{0pt}\colorbox{ansi-yellow}{matrix\strut}\setlength{\fboxsep}{0pt}\colorbox{ansi-yellow}{(\strut}\setlength{\fboxsep}{0pt}\colorbox{ansi-yellow}{[\strut}\setlength{\fboxsep}{0pt}\colorbox{ansi-yellow}{p1\strut}\setlength{\fboxsep}{0pt}\colorbox{ansi-yellow}{,\strut}\setlength{\fboxsep}{0pt}\colorbox{ansi-yellow}{ \strut}\setlength{\fboxsep}{0pt}\colorbox{ansi-yellow}{p2\strut}\setlength{\fboxsep}{0pt}\colorbox{ansi-yellow}{]\strut}\setlength{\fboxsep}{0pt}\colorbox{ansi-yellow}{)\strut}\def\tcRGB{\textcolor[RGB]}\expandafter\tcRGB\expandafter{\detokenize{98,98,98}}{\setlength{\fboxsep}{0pt}\colorbox{ansi-yellow}{.\strut}}\setlength{\fboxsep}{0pt}\colorbox{ansi-yellow}{minors\strut}\setlength{\fboxsep}{0pt}\colorbox{ansi-yellow}{(\strut}\setlength{\fboxsep}{0pt}\colorbox{ansi-yellow}{Integer\strut}\setlength{\fboxsep}{0pt}\colorbox{ansi-yellow}{(\strut}\def\tcRGB{\textcolor[RGB]}\expandafter\tcRGB\expandafter{\detokenize{98,98,98}}{\setlength{\fboxsep}{0pt}\colorbox{ansi-yellow}{2\strut}}\setlength{\fboxsep}{0pt}\colorbox{ansi-yellow}{)\strut}\setlength{\fboxsep}{0pt}\colorbox{ansi-yellow}{)\strut}\setlength{\fboxsep}{0pt}\colorbox{ansi-yellow}{)\strut}\setlength{\fboxsep}{0pt}\colorbox{ansi-yellow}{)\strut}[Integer(\def\tcRGB{\textcolor[RGB]}\expandafter\tcRGB\expandafter{\detokenize{98,98,98}}{0})]
\textcolor{ansi-green}{      7} kJ \def\tcRGB{\textcolor[RGB]}\expandafter\tcRGB\expandafter{\detokenize{98,98,98}}{=} kJ\def\tcRGB{\textcolor[RGB]}\expandafter\tcRGB\expandafter{\detokenize{98,98,98}}{.}saturation(S\def\tcRGB{\textcolor[RGB]}\expandafter\tcRGB\expandafter{\detokenize{98,98,98}}{.}ideal(matrix([p1, p3])\def\tcRGB{\textcolor[RGB]}\expandafter\tcRGB\expandafter{\detokenize{98,98,98}}{.}minors(Integer(\def\tcRGB{\textcolor[RGB]}\expandafter\tcRGB\expandafter{\detokenize{98,98,98}}{2}))))[Integer(\def\tcRGB{\textcolor[RGB]}\expandafter\tcRGB\expandafter{\detokenize{98,98,98}}{0})]
\textcolor{ansi-green}{      8} kJ \def\tcRGB{\textcolor[RGB]}\expandafter\tcRGB\expandafter{\detokenize{98,98,98}}{=} kJ\def\tcRGB{\textcolor[RGB]}\expandafter\tcRGB\expandafter{\detokenize{98,98,98}}{.}saturation(S\def\tcRGB{\textcolor[RGB]}\expandafter\tcRGB\expandafter{\detokenize{98,98,98}}{.}ideal(matrix([p1, p4])\def\tcRGB{\textcolor[RGB]}\expandafter\tcRGB\expandafter{\detokenize{98,98,98}}{.}minors(Integer(\def\tcRGB{\textcolor[RGB]}\expandafter\tcRGB\expandafter{\detokenize{98,98,98}}{2}))))[Integer(\def\tcRGB{\textcolor[RGB]}\expandafter\tcRGB\expandafter{\detokenize{98,98,98}}{0})]

File \textcolor{ansi-green-intense}{\textbf{\textasciitilde{}/miniforge3/lib/python3.9/site-packages/sage/rings/qqbar\_decorators.py:96}}, in \textcolor{ansi-cyan}{handle\_AA\_and\_QQbar.<locals>.wrapper}\textcolor{ansi-blue-intense}{\textbf{(*args, **kwds)}}
\textcolor{ansi-green}{     90} \def\tcRGB{\textcolor[RGB]}\expandafter\tcRGB\expandafter{\detokenize{0,135,0}}{\textbf{from}} \def\tcRGB{\textcolor[RGB]}\expandafter\tcRGB\expandafter{\detokenize{0,0,255}}{\textbf{sage}}\def\tcRGB{\textcolor[RGB]}\expandafter\tcRGB\expandafter{\detokenize{0,0,255}}{\textbf{.}}\def\tcRGB{\textcolor[RGB]}\expandafter\tcRGB\expandafter{\detokenize{0,0,255}}{\textbf{rings}}\def\tcRGB{\textcolor[RGB]}\expandafter\tcRGB\expandafter{\detokenize{0,0,255}}{\textbf{.}}\def\tcRGB{\textcolor[RGB]}\expandafter\tcRGB\expandafter{\detokenize{0,0,255}}{\textbf{abc}} \def\tcRGB{\textcolor[RGB]}\expandafter\tcRGB\expandafter{\detokenize{0,135,0}}{\textbf{import}} AlgebraicField\_common
\textcolor{ansi-green}{     92} \def\tcRGB{\textcolor[RGB]}\expandafter\tcRGB\expandafter{\detokenize{0,135,0}}{\textbf{if}} \def\tcRGB{\textcolor[RGB]}\expandafter\tcRGB\expandafter{\detokenize{175,0,255}}{\textbf{not}} \def\tcRGB{\textcolor[RGB]}\expandafter\tcRGB\expandafter{\detokenize{0,135,0}}{any}(\def\tcRGB{\textcolor[RGB]}\expandafter\tcRGB\expandafter{\detokenize{0,135,0}}{isinstance}(a, (Polynomial, MPolynomial, Ideal\_generic))
\textcolor{ansi-green}{     93}            \def\tcRGB{\textcolor[RGB]}\expandafter\tcRGB\expandafter{\detokenize{175,0,255}}{\textbf{and}} \def\tcRGB{\textcolor[RGB]}\expandafter\tcRGB\expandafter{\detokenize{0,135,0}}{isinstance}(a\def\tcRGB{\textcolor[RGB]}\expandafter\tcRGB\expandafter{\detokenize{98,98,98}}{.}base\_ring(), AlgebraicField\_common)
\textcolor{ansi-green}{     94}            \def\tcRGB{\textcolor[RGB]}\expandafter\tcRGB\expandafter{\detokenize{175,0,255}}{\textbf{or}} is\_PolynomialSequence(a)
\textcolor{ansi-green}{     95}            \def\tcRGB{\textcolor[RGB]}\expandafter\tcRGB\expandafter{\detokenize{175,0,255}}{\textbf{and}} \def\tcRGB{\textcolor[RGB]}\expandafter\tcRGB\expandafter{\detokenize{0,135,0}}{isinstance}(a\def\tcRGB{\textcolor[RGB]}\expandafter\tcRGB\expandafter{\detokenize{98,98,98}}{.}ring()\def\tcRGB{\textcolor[RGB]}\expandafter\tcRGB\expandafter{\detokenize{98,98,98}}{.}base\_ring(), AlgebraicField\_common) \def\tcRGB{\textcolor[RGB]}\expandafter\tcRGB\expandafter{\detokenize{0,135,0}}{\textbf{for}} a \def\tcRGB{\textcolor[RGB]}\expandafter\tcRGB\expandafter{\detokenize{175,0,255}}{\textbf{in}} args):
\textcolor{ansi-green-intense}{\textbf{---> 96}}     \def\tcRGB{\textcolor[RGB]}\expandafter\tcRGB\expandafter{\detokenize{0,135,0}}{\textbf{return}} \setlength{\fboxsep}{0pt}\colorbox{ansi-yellow}{func\strut}\setlength{\fboxsep}{0pt}\colorbox{ansi-yellow}{(\strut}\def\tcRGB{\textcolor[RGB]}\expandafter\tcRGB\expandafter{\detokenize{98,98,98}}{\setlength{\fboxsep}{0pt}\colorbox{ansi-yellow}{*\strut}}\setlength{\fboxsep}{0pt}\colorbox{ansi-yellow}{args\strut}\setlength{\fboxsep}{0pt}\colorbox{ansi-yellow}{,\strut}\setlength{\fboxsep}{0pt}\colorbox{ansi-yellow}{ \strut}\def\tcRGB{\textcolor[RGB]}\expandafter\tcRGB\expandafter{\detokenize{98,98,98}}{\setlength{\fboxsep}{0pt}\colorbox{ansi-yellow}{*\strut}}\def\tcRGB{\textcolor[RGB]}\expandafter\tcRGB\expandafter{\detokenize{98,98,98}}{\setlength{\fboxsep}{0pt}\colorbox{ansi-yellow}{*\strut}}\setlength{\fboxsep}{0pt}\colorbox{ansi-yellow}{kwds\strut}\setlength{\fboxsep}{0pt}\colorbox{ansi-yellow}{)\strut}
\textcolor{ansi-green}{     98} polynomials \def\tcRGB{\textcolor[RGB]}\expandafter\tcRGB\expandafter{\detokenize{98,98,98}}{=} []
\textcolor{ansi-green}{    100} \def\tcRGB{\textcolor[RGB]}\expandafter\tcRGB\expandafter{\detokenize{0,135,0}}{\textbf{for}} a \def\tcRGB{\textcolor[RGB]}\expandafter\tcRGB\expandafter{\detokenize{175,0,255}}{\textbf{in}} flatten(args, ltypes\def\tcRGB{\textcolor[RGB]}\expandafter\tcRGB\expandafter{\detokenize{98,98,98}}{=}(\def\tcRGB{\textcolor[RGB]}\expandafter\tcRGB\expandafter{\detokenize{0,135,0}}{list}, \def\tcRGB{\textcolor[RGB]}\expandafter\tcRGB\expandafter{\detokenize{0,135,0}}{tuple}, \def\tcRGB{\textcolor[RGB]}\expandafter\tcRGB\expandafter{\detokenize{0,135,0}}{set})):

File \textcolor{ansi-green-intense}{\textbf{\textasciitilde{}/miniforge3/lib/python3.9/site-packages/sage/rings/polynomial/multi\_polynomial\_ideal.py:2477}}, in \textcolor{ansi-cyan}{MPolynomialIdeal\_singular\_repr.saturation}\textcolor{ansi-blue-intense}{\textbf{(self, other)}}
\textcolor{ansi-green}{   2475}     sat \def\tcRGB{\textcolor[RGB]}\expandafter\tcRGB\expandafter{\detokenize{98,98,98}}{=} ff\def\tcRGB{\textcolor[RGB]}\expandafter\tcRGB\expandafter{\detokenize{98,98,98}}{.}elim\_\_lib\def\tcRGB{\textcolor[RGB]}\expandafter\tcRGB\expandafter{\detokenize{98,98,98}}{.}sat
\textcolor{ansi-green}{   2476} R \def\tcRGB{\textcolor[RGB]}\expandafter\tcRGB\expandafter{\detokenize{98,98,98}}{=} \def\tcRGB{\textcolor[RGB]}\expandafter\tcRGB\expandafter{\detokenize{0,135,0}}{self}\def\tcRGB{\textcolor[RGB]}\expandafter\tcRGB\expandafter{\detokenize{98,98,98}}{.}ring()
\textcolor{ansi-green-intense}{\textbf{-> 2477}} ideal, expo \def\tcRGB{\textcolor[RGB]}\expandafter\tcRGB\expandafter{\detokenize{98,98,98}}{=} \setlength{\fboxsep}{0pt}\colorbox{ansi-yellow}{sat\strut}\setlength{\fboxsep}{0pt}\colorbox{ansi-yellow}{(\strut}\def\tcRGB{\textcolor[RGB]}\expandafter\tcRGB\expandafter{\detokenize{0,135,0}}{\setlength{\fboxsep}{0pt}\colorbox{ansi-yellow}{self\strut}}\setlength{\fboxsep}{0pt}\colorbox{ansi-yellow}{,\strut}\setlength{\fboxsep}{0pt}\colorbox{ansi-yellow}{ \strut}\setlength{\fboxsep}{0pt}\colorbox{ansi-yellow}{other\strut}\setlength{\fboxsep}{0pt}\colorbox{ansi-yellow}{)\strut}
\textcolor{ansi-green}{   2478} \def\tcRGB{\textcolor[RGB]}\expandafter\tcRGB\expandafter{\detokenize{0,135,0}}{\textbf{return}} (R\def\tcRGB{\textcolor[RGB]}\expandafter\tcRGB\expandafter{\detokenize{98,98,98}}{.}ideal(ideal), ZZ(expo))

File \textcolor{ansi-green-intense}{\textbf{\textasciitilde{}/miniforge3/lib/python3.9/site-packages/sage/libs/singular/function.pyx:1298}}, in \textcolor{ansi-cyan}{sage.libs.singular.function.SingularFunction.\_\_call\_\_ (build/cythonized/sage/libs/singular/function.cpp:21339)}\textcolor{ansi-blue-intense}{\textbf{()}}
\textcolor{ansi-green}{   1296}     if not (isinstance(ring, MPolynomialRing\_libsingular) or isinstance(ring, NCPolynomialRing\_plural)):
\textcolor{ansi-green}{   1297}         raise TypeError("cannot call Singular function '\%s' with ring parameter of type '\%s'" \% (self.\_name,type(ring)))
\textcolor{ansi-green-intense}{\textbf{-> 1298}}     return call\_function(self, args, ring, interruptible, attributes)
\textcolor{ansi-green}{   1299} 
\textcolor{ansi-green}{   1300} def \_instancedoc\_(self):

File \textcolor{ansi-green-intense}{\textbf{\textasciitilde{}/miniforge3/lib/python3.9/site-packages/sage/libs/singular/function.pyx:1477}}, in \textcolor{ansi-cyan}{sage.libs.singular.function.call\_function (build/cythonized/sage/libs/singular/function.cpp:23326)}\textcolor{ansi-blue-intense}{\textbf{()}}
\textcolor{ansi-green}{   1475}     error\_messages.pop()
\textcolor{ansi-green}{   1476} 
\textcolor{ansi-green-intense}{\textbf{-> 1477}} with opt\_ctx: \# we are preserving the global options state here
\textcolor{ansi-green}{   1478}     if signal\_handler:
\textcolor{ansi-green}{   1479}         sig\_on()

File \textcolor{ansi-green-intense}{\textbf{\textasciitilde{}/miniforge3/lib/python3.9/site-packages/sage/libs/singular/function.pyx:1479}}, in \textcolor{ansi-cyan}{sage.libs.singular.function.call\_function (build/cythonized/sage/libs/singular/function.cpp:23238)}\textcolor{ansi-blue-intense}{\textbf{()}}
\textcolor{ansi-green}{   1477} with opt\_ctx: \# we are preserving the global options state here
\textcolor{ansi-green}{   1478}     if signal\_handler:
\textcolor{ansi-green-intense}{\textbf{-> 1479}}         sig\_on()
\textcolor{ansi-green}{   1480}         \_res = self.call\_handler.handle\_call(argument\_list, si\_ring)
\textcolor{ansi-green}{   1481}         sig\_off()

\textcolor{ansi-red-intense}{\textbf{KeyboardInterrupt}}: 
    \end{Verbatim}

    after these computations, kJ is the ideal of the two reminining
eigenpoints. we want that p1 defined above is an eigenpoint, so the
ideal kkJ here defined must be zero:

    \begin{tcolorbox}[breakable, size=fbox, boxrule=1pt, pad at break*=1mm,colback=cellbackground, colframe=cellborder]
\prompt{In}{incolor}{ }{\boxspacing}
\begin{Verbatim}[commandchars=\\\{\}]
\PY{n}{kkJ} \PY{o}{=} \PY{n}{kJ}\PY{o}{.}\PY{n}{subs}\PY{p}{(}\PY{p}{\PYZob{}}\PY{n}{x}\PY{p}{:}\PY{n}{p6}\PY{p}{[}\PY{l+m+mi}{0}\PY{p}{]}\PY{p}{,} \PY{n}{y}\PY{p}{:}\PY{n}{p6}\PY{p}{[}\PY{l+m+mi}{1}\PY{p}{]}\PY{p}{,} \PY{n}{z}\PY{p}{:}\PY{n}{p6}\PY{p}{[}\PY{l+m+mi}{2}\PY{p}{]}\PY{p}{\PYZcb{}}\PY{p}{)}\PY{o}{.}\PY{n}{radical}\PY{p}{(}\PY{p}{)}
\end{Verbatim}
\end{tcolorbox}

    We saturate kkJ

    \begin{tcolorbox}[breakable, size=fbox, boxrule=1pt, pad at break*=1mm,colback=cellbackground, colframe=cellborder]
\prompt{In}{incolor}{ }{\boxspacing}
\begin{Verbatim}[commandchars=\\\{\}]
\PY{n}{kkJ} \PY{o}{=} \PY{n}{kkJ}\PY{o}{.}\PY{n}{saturation}\PY{p}{(}\PY{n}{S}\PY{o}{.}\PY{n}{ideal}\PY{p}{(}\PY{n}{matrix}\PY{p}{(}\PY{p}{[}\PY{n}{p1}\PY{p}{,} \PY{n}{p3}\PY{p}{]}\PY{p}{)}\PY{o}{.}\PY{n}{minors}\PY{p}{(}\PY{l+m+mi}{2}\PY{p}{)}\PY{p}{)}\PY{p}{)}\PY{p}{[}\PY{l+m+mi}{0}\PY{p}{]}
\PY{n}{kkJ} \PY{o}{=} \PY{n}{kkJ}\PY{o}{.}\PY{n}{saturation}\PY{p}{(}\PY{n}{S}\PY{o}{.}\PY{n}{ideal}\PY{p}{(}\PY{n}{matrix}\PY{p}{(}\PY{p}{[}\PY{n}{p1}\PY{p}{,} \PY{n}{p4}\PY{p}{]}\PY{p}{)}\PY{o}{.}\PY{n}{minors}\PY{p}{(}\PY{l+m+mi}{2}\PY{p}{)}\PY{p}{)}\PY{p}{)}\PY{p}{[}\PY{l+m+mi}{0}\PY{p}{]}
\PY{n}{kkJ} \PY{o}{=} \PY{n}{kkJ}\PY{o}{.}\PY{n}{saturation}\PY{p}{(}\PY{n}{S}\PY{o}{.}\PY{n}{ideal}\PY{p}{(}\PY{n}{matrix}\PY{p}{(}\PY{p}{[}\PY{n}{p1}\PY{p}{,} \PY{n}{p5}\PY{p}{]}\PY{p}{)}\PY{o}{.}\PY{n}{minors}\PY{p}{(}\PY{l+m+mi}{2}\PY{p}{)}\PY{p}{)}\PY{p}{)}\PY{p}{[}\PY{l+m+mi}{0}\PY{p}{]}
\PY{n}{kkJ} \PY{o}{=} \PY{n}{kkJ}\PY{o}{.}\PY{n}{saturation}\PY{p}{(}\PY{n}{S}\PY{o}{.}\PY{n}{ideal}\PY{p}{(}\PY{n}{matrix}\PY{p}{(}\PY{p}{[}\PY{n}{p3}\PY{p}{,} \PY{n}{p5}\PY{p}{]}\PY{p}{)}\PY{o}{.}\PY{n}{minors}\PY{p}{(}\PY{l+m+mi}{2}\PY{p}{)}\PY{p}{)}\PY{p}{)}\PY{p}{[}\PY{l+m+mi}{0}\PY{p}{]}
\PY{n}{kkJ} \PY{o}{=} \PY{n}{kkJ}\PY{o}{.}\PY{n}{saturation}\PY{p}{(}\PY{n}{S}\PY{o}{.}\PY{n}{ideal}\PY{p}{(}\PY{n}{matrix}\PY{p}{(}\PY{p}{[}\PY{n}{p2}\PY{p}{,} \PY{n}{p6}\PY{p}{]}\PY{p}{)}\PY{o}{.}\PY{n}{minors}\PY{p}{(}\PY{l+m+mi}{2}\PY{p}{)}\PY{p}{)}\PY{p}{)}\PY{p}{[}\PY{l+m+mi}{0}\PY{p}{]}
\PY{n}{kkJ} \PY{o}{=} \PY{n}{kkJ}\PY{o}{.}\PY{n}{saturation}\PY{p}{(}\PY{n}{S}\PY{o}{.}\PY{n}{ideal}\PY{p}{(}\PY{n}{matrix}\PY{p}{(}\PY{p}{[}\PY{n}{p4}\PY{p}{,} \PY{n}{p6}\PY{p}{]}\PY{p}{)}\PY{o}{.}\PY{n}{minors}\PY{p}{(}\PY{l+m+mi}{2}\PY{p}{)}\PY{p}{)}\PY{p}{)}\PY{p}{[}\PY{l+m+mi}{0}\PY{p}{]}
\end{Verbatim}
\end{tcolorbox}

    kkJ is (1), so there are no solutions:

    \begin{tcolorbox}[breakable, size=fbox, boxrule=1pt, pad at break*=1mm,colback=cellbackground, colframe=cellborder]
\prompt{In}{incolor}{ }{\boxspacing}
\begin{Verbatim}[commandchars=\\\{\}]
\PY{k}{assert}\PY{p}{(}\PY{n}{kkJ} \PY{o}{==} \PY{n}{S}\PY{o}{.}\PY{n}{ideal}\PY{p}{(}\PY{l+m+mi}{1}\PY{p}{)}\PY{p}{)}
\end{Verbatim}
\end{tcolorbox}

    CONCLUSION (for the case \(C_2 = 0\)) when the three deltas are zero,
(hence rank of the matrix of the five points is \(8\)), we have also
\(\delta_2 = 0\) and we have the collinearities \((1, 2, 3)\),
\((1, 4, 5)\). We have a sub-case in which there is the further
collinearity \((1, 6, 7)\). No other collinearities among the 7
eigenpoints are possible.

    \begin{tcolorbox}[breakable, size=fbox, boxrule=1pt, pad at break*=1mm,colback=cellbackground, colframe=cellborder]
\prompt{In}{incolor}{ }{\boxspacing}
\begin{Verbatim}[commandchars=\\\{\}]

\end{Verbatim}
\end{tcolorbox}


    % Add a bibliography block to the postdoc
    
    
    
\end{document}
