\documentclass[11pt]{article}

    \usepackage[breakable]{tcolorbox}
    \usepackage{parskip} % Stop auto-indenting (to mimic markdown behaviour)
    

    % Basic figure setup, for now with no caption control since it's done
    % automatically by Pandoc (which extracts ![](path) syntax from Markdown).
    \usepackage{graphicx}
    % Maintain compatibility with old templates. Remove in nbconvert 6.0
    \let\Oldincludegraphics\includegraphics
    % Ensure that by default, figures have no caption (until we provide a
    % proper Figure object with a Caption API and a way to capture that
    % in the conversion process - todo).
    \usepackage{caption}
    \DeclareCaptionFormat{nocaption}{}
    \captionsetup{format=nocaption,aboveskip=0pt,belowskip=0pt}

    \usepackage{float}
    \floatplacement{figure}{H} % forces figures to be placed at the correct location
    \usepackage{xcolor} % Allow colors to be defined
    \usepackage{enumerate} % Needed for markdown enumerations to work
    \usepackage{geometry} % Used to adjust the document margins
    \usepackage{amsmath} % Equations
    \usepackage{amssymb} % Equations
    \usepackage{textcomp} % defines textquotesingle
    % Hack from http://tex.stackexchange.com/a/47451/13684:
    \AtBeginDocument{%
        \def\PYZsq{\textquotesingle}% Upright quotes in Pygmentized code
    }
    \usepackage{upquote} % Upright quotes for verbatim code
    \usepackage{eurosym} % defines \euro

    \usepackage{iftex}
    \ifPDFTeX
        \usepackage[T1]{fontenc}
        \IfFileExists{alphabeta.sty}{
              \usepackage{alphabeta}
          }{
              \usepackage[mathletters]{ucs}
              \usepackage[utf8x]{inputenc}
          }
    \else
        \usepackage{fontspec}
        \usepackage{unicode-math}
    \fi

    \usepackage{fancyvrb} % verbatim replacement that allows latex
    \usepackage{grffile} % extends the file name processing of package graphics
                         % to support a larger range
    \makeatletter % fix for old versions of grffile with XeLaTeX
    \@ifpackagelater{grffile}{2019/11/01}
    {
      % Do nothing on new versions
    }
    {
      \def\Gread@@xetex#1{%
        \IfFileExists{"\Gin@base".bb}%
        {\Gread@eps{\Gin@base.bb}}%
        {\Gread@@xetex@aux#1}%
      }
    }
    \makeatother
    \usepackage[Export]{adjustbox} % Used to constrain images to a maximum size
    \adjustboxset{max size={0.9\linewidth}{0.9\paperheight}}

    % The hyperref package gives us a pdf with properly built
    % internal navigation ('pdf bookmarks' for the table of contents,
    % internal cross-reference links, web links for URLs, etc.)
    \usepackage{hyperref}
    % The default LaTeX title has an obnoxious amount of whitespace. By default,
    % titling removes some of it. It also provides customization options.
    \usepackage{titling}
    \usepackage{longtable} % longtable support required by pandoc >1.10
    \usepackage{booktabs}  % table support for pandoc > 1.12.2
    \usepackage{array}     % table support for pandoc >= 2.11.3
    \usepackage{calc}      % table minipage width calculation for pandoc >= 2.11.1
    \usepackage[inline]{enumitem} % IRkernel/repr support (it uses the enumerate* environment)
    \usepackage[normalem]{ulem} % ulem is needed to support strikethroughs (\sout)
                                % normalem makes italics be italics, not underlines
    \usepackage{soul}      % strikethrough (\st) support for pandoc >= 3.0.0
    \usepackage{mathrsfs}
    

    
    % Colors for the hyperref package
    \definecolor{urlcolor}{rgb}{0,.145,.698}
    \definecolor{linkcolor}{rgb}{.71,0.21,0.01}
    \definecolor{citecolor}{rgb}{.12,.54,.11}

    % ANSI colors
    \definecolor{ansi-black}{HTML}{3E424D}
    \definecolor{ansi-black-intense}{HTML}{282C36}
    \definecolor{ansi-red}{HTML}{E75C58}
    \definecolor{ansi-red-intense}{HTML}{B22B31}
    \definecolor{ansi-green}{HTML}{00A250}
    \definecolor{ansi-green-intense}{HTML}{007427}
    \definecolor{ansi-yellow}{HTML}{DDB62B}
    \definecolor{ansi-yellow-intense}{HTML}{B27D12}
    \definecolor{ansi-blue}{HTML}{208FFB}
    \definecolor{ansi-blue-intense}{HTML}{0065CA}
    \definecolor{ansi-magenta}{HTML}{D160C4}
    \definecolor{ansi-magenta-intense}{HTML}{A03196}
    \definecolor{ansi-cyan}{HTML}{60C6C8}
    \definecolor{ansi-cyan-intense}{HTML}{258F8F}
    \definecolor{ansi-white}{HTML}{C5C1B4}
    \definecolor{ansi-white-intense}{HTML}{A1A6B2}
    \definecolor{ansi-default-inverse-fg}{HTML}{FFFFFF}
    \definecolor{ansi-default-inverse-bg}{HTML}{000000}

    % common color for the border for error outputs.
    \definecolor{outerrorbackground}{HTML}{FFDFDF}

    % commands and environments needed by pandoc snippets
    % extracted from the output of `pandoc -s`
    \providecommand{\tightlist}{%
      \setlength{\itemsep}{0pt}\setlength{\parskip}{0pt}}
    \DefineVerbatimEnvironment{Highlighting}{Verbatim}{commandchars=\\\{\}}
    % Add ',fontsize=\small' for more characters per line
    \newenvironment{Shaded}{}{}
    \newcommand{\KeywordTok}[1]{\textcolor[rgb]{0.00,0.44,0.13}{\textbf{{#1}}}}
    \newcommand{\DataTypeTok}[1]{\textcolor[rgb]{0.56,0.13,0.00}{{#1}}}
    \newcommand{\DecValTok}[1]{\textcolor[rgb]{0.25,0.63,0.44}{{#1}}}
    \newcommand{\BaseNTok}[1]{\textcolor[rgb]{0.25,0.63,0.44}{{#1}}}
    \newcommand{\FloatTok}[1]{\textcolor[rgb]{0.25,0.63,0.44}{{#1}}}
    \newcommand{\CharTok}[1]{\textcolor[rgb]{0.25,0.44,0.63}{{#1}}}
    \newcommand{\StringTok}[1]{\textcolor[rgb]{0.25,0.44,0.63}{{#1}}}
    \newcommand{\CommentTok}[1]{\textcolor[rgb]{0.38,0.63,0.69}{\textit{{#1}}}}
    \newcommand{\OtherTok}[1]{\textcolor[rgb]{0.00,0.44,0.13}{{#1}}}
    \newcommand{\AlertTok}[1]{\textcolor[rgb]{1.00,0.00,0.00}{\textbf{{#1}}}}
    \newcommand{\FunctionTok}[1]{\textcolor[rgb]{0.02,0.16,0.49}{{#1}}}
    \newcommand{\RegionMarkerTok}[1]{{#1}}
    \newcommand{\ErrorTok}[1]{\textcolor[rgb]{1.00,0.00,0.00}{\textbf{{#1}}}}
    \newcommand{\NormalTok}[1]{{#1}}

    % Additional commands for more recent versions of Pandoc
    \newcommand{\ConstantTok}[1]{\textcolor[rgb]{0.53,0.00,0.00}{{#1}}}
    \newcommand{\SpecialCharTok}[1]{\textcolor[rgb]{0.25,0.44,0.63}{{#1}}}
    \newcommand{\VerbatimStringTok}[1]{\textcolor[rgb]{0.25,0.44,0.63}{{#1}}}
    \newcommand{\SpecialStringTok}[1]{\textcolor[rgb]{0.73,0.40,0.53}{{#1}}}
    \newcommand{\ImportTok}[1]{{#1}}
    \newcommand{\DocumentationTok}[1]{\textcolor[rgb]{0.73,0.13,0.13}{\textit{{#1}}}}
    \newcommand{\AnnotationTok}[1]{\textcolor[rgb]{0.38,0.63,0.69}{\textbf{\textit{{#1}}}}}
    \newcommand{\CommentVarTok}[1]{\textcolor[rgb]{0.38,0.63,0.69}{\textbf{\textit{{#1}}}}}
    \newcommand{\VariableTok}[1]{\textcolor[rgb]{0.10,0.09,0.49}{{#1}}}
    \newcommand{\ControlFlowTok}[1]{\textcolor[rgb]{0.00,0.44,0.13}{\textbf{{#1}}}}
    \newcommand{\OperatorTok}[1]{\textcolor[rgb]{0.40,0.40,0.40}{{#1}}}
    \newcommand{\BuiltInTok}[1]{{#1}}
    \newcommand{\ExtensionTok}[1]{{#1}}
    \newcommand{\PreprocessorTok}[1]{\textcolor[rgb]{0.74,0.48,0.00}{{#1}}}
    \newcommand{\AttributeTok}[1]{\textcolor[rgb]{0.49,0.56,0.16}{{#1}}}
    \newcommand{\InformationTok}[1]{\textcolor[rgb]{0.38,0.63,0.69}{\textbf{\textit{{#1}}}}}
    \newcommand{\WarningTok}[1]{\textcolor[rgb]{0.38,0.63,0.69}{\textbf{\textit{{#1}}}}}


    % Define a nice break command that doesn't care if a line doesn't already
    % exist.
    \def\br{\hspace*{\fill} \\* }
    % Math Jax compatibility definitions
    \def\gt{>}
    \def\lt{<}
    \let\Oldtex\TeX
    \let\Oldlatex\LaTeX
    \renewcommand{\TeX}{\textrm{\Oldtex}}
    \renewcommand{\LaTeX}{\textrm{\Oldlatex}}
    % Document parameters
    % Document title
    \title{NB.07.F3}
    
    
    
    
    
    
    
% Pygments definitions
\makeatletter
\def\PY@reset{\let\PY@it=\relax \let\PY@bf=\relax%
    \let\PY@ul=\relax \let\PY@tc=\relax%
    \let\PY@bc=\relax \let\PY@ff=\relax}
\def\PY@tok#1{\csname PY@tok@#1\endcsname}
\def\PY@toks#1+{\ifx\relax#1\empty\else%
    \PY@tok{#1}\expandafter\PY@toks\fi}
\def\PY@do#1{\PY@bc{\PY@tc{\PY@ul{%
    \PY@it{\PY@bf{\PY@ff{#1}}}}}}}
\def\PY#1#2{\PY@reset\PY@toks#1+\relax+\PY@do{#2}}

\@namedef{PY@tok@w}{\def\PY@tc##1{\textcolor[rgb]{0.73,0.73,0.73}{##1}}}
\@namedef{PY@tok@c}{\let\PY@it=\textit\def\PY@tc##1{\textcolor[rgb]{0.24,0.48,0.48}{##1}}}
\@namedef{PY@tok@cp}{\def\PY@tc##1{\textcolor[rgb]{0.61,0.40,0.00}{##1}}}
\@namedef{PY@tok@k}{\let\PY@bf=\textbf\def\PY@tc##1{\textcolor[rgb]{0.00,0.50,0.00}{##1}}}
\@namedef{PY@tok@kp}{\def\PY@tc##1{\textcolor[rgb]{0.00,0.50,0.00}{##1}}}
\@namedef{PY@tok@kt}{\def\PY@tc##1{\textcolor[rgb]{0.69,0.00,0.25}{##1}}}
\@namedef{PY@tok@o}{\def\PY@tc##1{\textcolor[rgb]{0.40,0.40,0.40}{##1}}}
\@namedef{PY@tok@ow}{\let\PY@bf=\textbf\def\PY@tc##1{\textcolor[rgb]{0.67,0.13,1.00}{##1}}}
\@namedef{PY@tok@nb}{\def\PY@tc##1{\textcolor[rgb]{0.00,0.50,0.00}{##1}}}
\@namedef{PY@tok@nf}{\def\PY@tc##1{\textcolor[rgb]{0.00,0.00,1.00}{##1}}}
\@namedef{PY@tok@nc}{\let\PY@bf=\textbf\def\PY@tc##1{\textcolor[rgb]{0.00,0.00,1.00}{##1}}}
\@namedef{PY@tok@nn}{\let\PY@bf=\textbf\def\PY@tc##1{\textcolor[rgb]{0.00,0.00,1.00}{##1}}}
\@namedef{PY@tok@ne}{\let\PY@bf=\textbf\def\PY@tc##1{\textcolor[rgb]{0.80,0.25,0.22}{##1}}}
\@namedef{PY@tok@nv}{\def\PY@tc##1{\textcolor[rgb]{0.10,0.09,0.49}{##1}}}
\@namedef{PY@tok@no}{\def\PY@tc##1{\textcolor[rgb]{0.53,0.00,0.00}{##1}}}
\@namedef{PY@tok@nl}{\def\PY@tc##1{\textcolor[rgb]{0.46,0.46,0.00}{##1}}}
\@namedef{PY@tok@ni}{\let\PY@bf=\textbf\def\PY@tc##1{\textcolor[rgb]{0.44,0.44,0.44}{##1}}}
\@namedef{PY@tok@na}{\def\PY@tc##1{\textcolor[rgb]{0.41,0.47,0.13}{##1}}}
\@namedef{PY@tok@nt}{\let\PY@bf=\textbf\def\PY@tc##1{\textcolor[rgb]{0.00,0.50,0.00}{##1}}}
\@namedef{PY@tok@nd}{\def\PY@tc##1{\textcolor[rgb]{0.67,0.13,1.00}{##1}}}
\@namedef{PY@tok@s}{\def\PY@tc##1{\textcolor[rgb]{0.73,0.13,0.13}{##1}}}
\@namedef{PY@tok@sd}{\let\PY@it=\textit\def\PY@tc##1{\textcolor[rgb]{0.73,0.13,0.13}{##1}}}
\@namedef{PY@tok@si}{\let\PY@bf=\textbf\def\PY@tc##1{\textcolor[rgb]{0.64,0.35,0.47}{##1}}}
\@namedef{PY@tok@se}{\let\PY@bf=\textbf\def\PY@tc##1{\textcolor[rgb]{0.67,0.36,0.12}{##1}}}
\@namedef{PY@tok@sr}{\def\PY@tc##1{\textcolor[rgb]{0.64,0.35,0.47}{##1}}}
\@namedef{PY@tok@ss}{\def\PY@tc##1{\textcolor[rgb]{0.10,0.09,0.49}{##1}}}
\@namedef{PY@tok@sx}{\def\PY@tc##1{\textcolor[rgb]{0.00,0.50,0.00}{##1}}}
\@namedef{PY@tok@m}{\def\PY@tc##1{\textcolor[rgb]{0.40,0.40,0.40}{##1}}}
\@namedef{PY@tok@gh}{\let\PY@bf=\textbf\def\PY@tc##1{\textcolor[rgb]{0.00,0.00,0.50}{##1}}}
\@namedef{PY@tok@gu}{\let\PY@bf=\textbf\def\PY@tc##1{\textcolor[rgb]{0.50,0.00,0.50}{##1}}}
\@namedef{PY@tok@gd}{\def\PY@tc##1{\textcolor[rgb]{0.63,0.00,0.00}{##1}}}
\@namedef{PY@tok@gi}{\def\PY@tc##1{\textcolor[rgb]{0.00,0.52,0.00}{##1}}}
\@namedef{PY@tok@gr}{\def\PY@tc##1{\textcolor[rgb]{0.89,0.00,0.00}{##1}}}
\@namedef{PY@tok@ge}{\let\PY@it=\textit}
\@namedef{PY@tok@gs}{\let\PY@bf=\textbf}
\@namedef{PY@tok@ges}{\let\PY@bf=\textbf\let\PY@it=\textit}
\@namedef{PY@tok@gp}{\let\PY@bf=\textbf\def\PY@tc##1{\textcolor[rgb]{0.00,0.00,0.50}{##1}}}
\@namedef{PY@tok@go}{\def\PY@tc##1{\textcolor[rgb]{0.44,0.44,0.44}{##1}}}
\@namedef{PY@tok@gt}{\def\PY@tc##1{\textcolor[rgb]{0.00,0.27,0.87}{##1}}}
\@namedef{PY@tok@err}{\def\PY@bc##1{{\setlength{\fboxsep}{\string -\fboxrule}\fcolorbox[rgb]{1.00,0.00,0.00}{1,1,1}{\strut ##1}}}}
\@namedef{PY@tok@kc}{\let\PY@bf=\textbf\def\PY@tc##1{\textcolor[rgb]{0.00,0.50,0.00}{##1}}}
\@namedef{PY@tok@kd}{\let\PY@bf=\textbf\def\PY@tc##1{\textcolor[rgb]{0.00,0.50,0.00}{##1}}}
\@namedef{PY@tok@kn}{\let\PY@bf=\textbf\def\PY@tc##1{\textcolor[rgb]{0.00,0.50,0.00}{##1}}}
\@namedef{PY@tok@kr}{\let\PY@bf=\textbf\def\PY@tc##1{\textcolor[rgb]{0.00,0.50,0.00}{##1}}}
\@namedef{PY@tok@bp}{\def\PY@tc##1{\textcolor[rgb]{0.00,0.50,0.00}{##1}}}
\@namedef{PY@tok@fm}{\def\PY@tc##1{\textcolor[rgb]{0.00,0.00,1.00}{##1}}}
\@namedef{PY@tok@vc}{\def\PY@tc##1{\textcolor[rgb]{0.10,0.09,0.49}{##1}}}
\@namedef{PY@tok@vg}{\def\PY@tc##1{\textcolor[rgb]{0.10,0.09,0.49}{##1}}}
\@namedef{PY@tok@vi}{\def\PY@tc##1{\textcolor[rgb]{0.10,0.09,0.49}{##1}}}
\@namedef{PY@tok@vm}{\def\PY@tc##1{\textcolor[rgb]{0.10,0.09,0.49}{##1}}}
\@namedef{PY@tok@sa}{\def\PY@tc##1{\textcolor[rgb]{0.73,0.13,0.13}{##1}}}
\@namedef{PY@tok@sb}{\def\PY@tc##1{\textcolor[rgb]{0.73,0.13,0.13}{##1}}}
\@namedef{PY@tok@sc}{\def\PY@tc##1{\textcolor[rgb]{0.73,0.13,0.13}{##1}}}
\@namedef{PY@tok@dl}{\def\PY@tc##1{\textcolor[rgb]{0.73,0.13,0.13}{##1}}}
\@namedef{PY@tok@s2}{\def\PY@tc##1{\textcolor[rgb]{0.73,0.13,0.13}{##1}}}
\@namedef{PY@tok@sh}{\def\PY@tc##1{\textcolor[rgb]{0.73,0.13,0.13}{##1}}}
\@namedef{PY@tok@s1}{\def\PY@tc##1{\textcolor[rgb]{0.73,0.13,0.13}{##1}}}
\@namedef{PY@tok@mb}{\def\PY@tc##1{\textcolor[rgb]{0.40,0.40,0.40}{##1}}}
\@namedef{PY@tok@mf}{\def\PY@tc##1{\textcolor[rgb]{0.40,0.40,0.40}{##1}}}
\@namedef{PY@tok@mh}{\def\PY@tc##1{\textcolor[rgb]{0.40,0.40,0.40}{##1}}}
\@namedef{PY@tok@mi}{\def\PY@tc##1{\textcolor[rgb]{0.40,0.40,0.40}{##1}}}
\@namedef{PY@tok@il}{\def\PY@tc##1{\textcolor[rgb]{0.40,0.40,0.40}{##1}}}
\@namedef{PY@tok@mo}{\def\PY@tc##1{\textcolor[rgb]{0.40,0.40,0.40}{##1}}}
\@namedef{PY@tok@ch}{\let\PY@it=\textit\def\PY@tc##1{\textcolor[rgb]{0.24,0.48,0.48}{##1}}}
\@namedef{PY@tok@cm}{\let\PY@it=\textit\def\PY@tc##1{\textcolor[rgb]{0.24,0.48,0.48}{##1}}}
\@namedef{PY@tok@cpf}{\let\PY@it=\textit\def\PY@tc##1{\textcolor[rgb]{0.24,0.48,0.48}{##1}}}
\@namedef{PY@tok@c1}{\let\PY@it=\textit\def\PY@tc##1{\textcolor[rgb]{0.24,0.48,0.48}{##1}}}
\@namedef{PY@tok@cs}{\let\PY@it=\textit\def\PY@tc##1{\textcolor[rgb]{0.24,0.48,0.48}{##1}}}

\def\PYZbs{\char`\\}
\def\PYZus{\char`\_}
\def\PYZob{\char`\{}
\def\PYZcb{\char`\}}
\def\PYZca{\char`\^}
\def\PYZam{\char`\&}
\def\PYZlt{\char`\<}
\def\PYZgt{\char`\>}
\def\PYZsh{\char`\#}
\def\PYZpc{\char`\%}
\def\PYZdl{\char`\$}
\def\PYZhy{\char`\-}
\def\PYZsq{\char`\'}
\def\PYZdq{\char`\"}
\def\PYZti{\char`\~}
% for compatibility with earlier versions
\def\PYZat{@}
\def\PYZlb{[}
\def\PYZrb{]}
\makeatother


    % For linebreaks inside Verbatim environment from package fancyvrb.
    \makeatletter
        \newbox\Wrappedcontinuationbox
        \newbox\Wrappedvisiblespacebox
        \newcommand*\Wrappedvisiblespace {\textcolor{red}{\textvisiblespace}}
        \newcommand*\Wrappedcontinuationsymbol {\textcolor{red}{\llap{\tiny$\m@th\hookrightarrow$}}}
        \newcommand*\Wrappedcontinuationindent {3ex }
        \newcommand*\Wrappedafterbreak {\kern\Wrappedcontinuationindent\copy\Wrappedcontinuationbox}
        % Take advantage of the already applied Pygments mark-up to insert
        % potential linebreaks for TeX processing.
        %        {, <, #, %, $, ' and ": go to next line.
        %        _, }, ^, &, >, - and ~: stay at end of broken line.
        % Use of \textquotesingle for straight quote.
        \newcommand*\Wrappedbreaksatspecials {%
            \def\PYGZus{\discretionary{\char`\_}{\Wrappedafterbreak}{\char`\_}}%
            \def\PYGZob{\discretionary{}{\Wrappedafterbreak\char`\{}{\char`\{}}%
            \def\PYGZcb{\discretionary{\char`\}}{\Wrappedafterbreak}{\char`\}}}%
            \def\PYGZca{\discretionary{\char`\^}{\Wrappedafterbreak}{\char`\^}}%
            \def\PYGZam{\discretionary{\char`\&}{\Wrappedafterbreak}{\char`\&}}%
            \def\PYGZlt{\discretionary{}{\Wrappedafterbreak\char`\<}{\char`\<}}%
            \def\PYGZgt{\discretionary{\char`\>}{\Wrappedafterbreak}{\char`\>}}%
            \def\PYGZsh{\discretionary{}{\Wrappedafterbreak\char`\#}{\char`\#}}%
            \def\PYGZpc{\discretionary{}{\Wrappedafterbreak\char`\%}{\char`\%}}%
            \def\PYGZdl{\discretionary{}{\Wrappedafterbreak\char`\$}{\char`\$}}%
            \def\PYGZhy{\discretionary{\char`\-}{\Wrappedafterbreak}{\char`\-}}%
            \def\PYGZsq{\discretionary{}{\Wrappedafterbreak\textquotesingle}{\textquotesingle}}%
            \def\PYGZdq{\discretionary{}{\Wrappedafterbreak\char`\"}{\char`\"}}%
            \def\PYGZti{\discretionary{\char`\~}{\Wrappedafterbreak}{\char`\~}}%
        }
        % Some characters . , ; ? ! / are not pygmentized.
        % This macro makes them "active" and they will insert potential linebreaks
        \newcommand*\Wrappedbreaksatpunct {%
            \lccode`\~`\.\lowercase{\def~}{\discretionary{\hbox{\char`\.}}{\Wrappedafterbreak}{\hbox{\char`\.}}}%
            \lccode`\~`\,\lowercase{\def~}{\discretionary{\hbox{\char`\,}}{\Wrappedafterbreak}{\hbox{\char`\,}}}%
            \lccode`\~`\;\lowercase{\def~}{\discretionary{\hbox{\char`\;}}{\Wrappedafterbreak}{\hbox{\char`\;}}}%
            \lccode`\~`\:\lowercase{\def~}{\discretionary{\hbox{\char`\:}}{\Wrappedafterbreak}{\hbox{\char`\:}}}%
            \lccode`\~`\?\lowercase{\def~}{\discretionary{\hbox{\char`\?}}{\Wrappedafterbreak}{\hbox{\char`\?}}}%
            \lccode`\~`\!\lowercase{\def~}{\discretionary{\hbox{\char`\!}}{\Wrappedafterbreak}{\hbox{\char`\!}}}%
            \lccode`\~`\/\lowercase{\def~}{\discretionary{\hbox{\char`\/}}{\Wrappedafterbreak}{\hbox{\char`\/}}}%
            \catcode`\.\active
            \catcode`\,\active
            \catcode`\;\active
            \catcode`\:\active
            \catcode`\?\active
            \catcode`\!\active
            \catcode`\/\active
            \lccode`\~`\~
        }
    \makeatother

    \let\OriginalVerbatim=\Verbatim
    \makeatletter
    \renewcommand{\Verbatim}[1][1]{%
        %\parskip\z@skip
        \sbox\Wrappedcontinuationbox {\Wrappedcontinuationsymbol}%
        \sbox\Wrappedvisiblespacebox {\FV@SetupFont\Wrappedvisiblespace}%
        \def\FancyVerbFormatLine ##1{\hsize\linewidth
            \vtop{\raggedright\hyphenpenalty\z@\exhyphenpenalty\z@
                \doublehyphendemerits\z@\finalhyphendemerits\z@
                \strut ##1\strut}%
        }%
        % If the linebreak is at a space, the latter will be displayed as visible
        % space at end of first line, and a continuation symbol starts next line.
        % Stretch/shrink are however usually zero for typewriter font.
        \def\FV@Space {%
            \nobreak\hskip\z@ plus\fontdimen3\font minus\fontdimen4\font
            \discretionary{\copy\Wrappedvisiblespacebox}{\Wrappedafterbreak}
            {\kern\fontdimen2\font}%
        }%

        % Allow breaks at special characters using \PYG... macros.
        \Wrappedbreaksatspecials
        % Breaks at punctuation characters . , ; ? ! and / need catcode=\active
        \OriginalVerbatim[#1,codes*=\Wrappedbreaksatpunct]%
    }
    \makeatother

    % Exact colors from NB
    \definecolor{incolor}{HTML}{303F9F}
    \definecolor{outcolor}{HTML}{D84315}
    \definecolor{cellborder}{HTML}{CFCFCF}
    \definecolor{cellbackground}{HTML}{F7F7F7}

    % prompt
    \makeatletter
    \newcommand{\boxspacing}{\kern\kvtcb@left@rule\kern\kvtcb@boxsep}
    \makeatother
    \newcommand{\prompt}[4]{
        {\ttfamily\llap{{\color{#2}[#3]:\hspace{3pt}#4}}\vspace{-\baselineskip}}
    }
    

    
    % Prevent overflowing lines due to hard-to-break entities
    \sloppy
    % Setup hyperref package
    \hypersetup{
      breaklinks=true,  % so long urls are correctly broken across lines
      colorlinks=true,
      urlcolor=urlcolor,
      linkcolor=linkcolor,
      citecolor=citecolor,
      }
    % Slightly bigger margins than the latex defaults
    
    \geometry{verbose,tmargin=1in,bmargin=1in,lmargin=1in,rmargin=1in}
    
    

\begin{document}
    
    \maketitle
    
    

    
    \hypertarget{configurations-c_5}{%
\section{\texorpdfstring{Configurations
\((C_5)\)}{Configurations (C\_5)}}\label{configurations-c_5}}

    \begin{tcolorbox}[breakable, size=fbox, boxrule=1pt, pad at break*=1mm,colback=cellbackground, colframe=cellborder]
\prompt{In}{incolor}{5}{\boxspacing}
\begin{Verbatim}[commandchars=\\\{\}]
\PY{n}{load}\PY{p}{(}\PY{l+s+s2}{\PYZdq{}}\PY{l+s+s2}{basic\PYZus{}functions.sage}\PY{l+s+s2}{\PYZdq{}}\PY{p}{)}
\end{Verbatim}
\end{tcolorbox}

    \begin{tcolorbox}[breakable, size=fbox, boxrule=1pt, pad at break*=1mm,colback=cellbackground, colframe=cellborder]
\prompt{In}{incolor}{6}{\boxspacing}
\begin{Verbatim}[commandchars=\\\{\}]
\PY{n}{do\PYZus{}long\PYZus{}computations} \PY{o}{=} \PY{k+kc}{False}
\end{Verbatim}
\end{tcolorbox}

    In the above computations, we always assume that, if
\(P_1, \dotsc, P_5\) are eigenpoints in a \(V\)-configuration, then
\(\Phi(P_1, \dots, P_5)\) has rank 9 and not 8 (the case rank 8 is
studied elsewhere).

In order to study a \((C_5)\) configuration, alignments:

\((1, 2, 3), (1, 4, 5), (1, 6, 7), (2, 4, 6)\)

we define 6 points such that \(P_3\) is collinear with \(P_1\) and
\(P_2\), \(P_5\) is collinear with \(P_1\) and \(P_4\), and \(P_6\) is
collinear with \(P_2\) and \(P_4\). These points must be eigenpoints, so
\(\delta_1(P_2, P_1, P_4)=0\), \(\delta_1(P_4, P_1, P_2)\),
\(\delta_1(P_6, P_1, P_2)=0\), \(\delta_2(P_1, P_2, P_3, P_4, P_5)=0\).
Moreover, we can divide \(\delta_2(P_6, P_1, P_2)\) by \(w_2\).

We construct the ideal generated by these conditions and we saturate it
w.r.t. conditions that do not give restrictions.

We get that \(s_{14}=0\), \(s_{12}=0, s_{16}=0\), hence
\(P_1 = P_2 \times P_4\). We redefine therefore the points.

Below the computations:

    \begin{tcolorbox}[breakable, size=fbox, boxrule=1pt, pad at break*=1mm,colback=cellbackground, colframe=cellborder]
\prompt{In}{incolor}{7}{\boxspacing}
\begin{Verbatim}[commandchars=\\\{\}]
\PY{n}{P1} \PY{o}{=} \PY{n}{vector}\PY{p}{(}\PY{n}{S}\PY{p}{,} \PY{p}{(}\PY{n}{A1}\PY{p}{,} \PY{n}{B1}\PY{p}{,} \PY{n}{C1}\PY{p}{)}\PY{p}{)}
\PY{n}{P2} \PY{o}{=} \PY{n}{vector}\PY{p}{(}\PY{n}{S}\PY{p}{,} \PY{p}{(}\PY{n}{A2}\PY{p}{,} \PY{n}{B2}\PY{p}{,} \PY{n}{C2}\PY{p}{)}\PY{p}{)}
\PY{n}{P3} \PY{o}{=} \PY{n}{u1}\PY{o}{*}\PY{n}{P1}\PY{o}{+}\PY{n}{u2}\PY{o}{*}\PY{n}{P2}
\PY{n}{P4} \PY{o}{=} \PY{n}{vector}\PY{p}{(}\PY{n}{S}\PY{p}{,} \PY{p}{(}\PY{n}{A4}\PY{p}{,} \PY{n}{B4}\PY{p}{,} \PY{n}{C4}\PY{p}{)}\PY{p}{)}
\PY{n}{P5} \PY{o}{=} \PY{n}{v1}\PY{o}{*}\PY{n}{P1}\PY{o}{+}\PY{n}{v2}\PY{o}{*}\PY{n}{P4}
\PY{n}{P6} \PY{o}{=} \PY{n}{w1}\PY{o}{*}\PY{n}{P2}\PY{o}{+}\PY{n}{w2}\PY{o}{*}\PY{n}{P4}

\PY{c+c1}{\PYZsh{}\PYZsh{} indeed, delta1(P6, P1, P2) is divisible by w2:}
\PY{k}{assert}\PY{p}{(}\PY{n}{delta1}\PY{p}{(}\PY{n}{P6}\PY{p}{,} \PY{n}{P1}\PY{p}{,} \PY{n}{P2}\PY{p}{)}\PY{o}{.}\PY{n}{quo\PYZus{}rem}\PY{p}{(}\PY{n}{w2}\PY{p}{)}\PY{p}{[}\PY{l+m+mi}{1}\PY{p}{]} \PY{o}{==} \PY{n}{S}\PY{p}{(}\PY{l+m+mi}{0}\PY{p}{)}\PY{p}{)}

\PY{c+c1}{\PYZsh{}\PYZsh{} hence we can consider the following ideal:}
\PY{n}{J} \PY{o}{=} \PY{n}{S}\PY{o}{.}\PY{n}{ideal}\PY{p}{(}\PY{n}{delta1}\PY{p}{(}\PY{n}{P2}\PY{p}{,} \PY{n}{P1}\PY{p}{,} \PY{n}{P4}\PY{p}{)}\PY{p}{,} \PY{n}{delta1}\PY{p}{(}\PY{n}{P4}\PY{p}{,} \PY{n}{P1}\PY{p}{,} \PY{n}{P2}\PY{p}{)}\PY{p}{,} \PYZbs{}
\PY{n}{delta1}\PY{p}{(}\PY{n}{P6}\PY{p}{,} \PY{n}{P1}\PY{p}{,} \PY{n}{P2}\PY{p}{)}\PY{o}{.}\PY{n}{quo\PYZus{}rem}\PY{p}{(}\PY{n}{w2}\PY{p}{)}\PY{p}{[}\PY{l+m+mi}{0}\PY{p}{]}\PY{p}{,} \PY{n}{delta2}\PY{p}{(}\PY{n}{P1}\PY{p}{,} \PY{n}{P2}\PY{p}{,} \PY{n}{P3}\PY{p}{,} \PY{n}{P4}\PY{p}{,} \PY{n}{P5}\PY{p}{)}\PY{p}{)}

\PY{c+c1}{\PYZsh{}\PYZsh{} We saturate J and we get that J is the ideal generated }
\PY{c+c1}{\PYZsh{}\PYZsh{} by (P1|P2) and (P1|P4):}

\PY{n}{J} \PY{o}{=} \PY{n}{J}\PY{o}{.}\PY{n}{saturation}\PY{p}{(}\PY{n}{matrix}\PY{p}{(}\PY{p}{[}\PY{n}{P1}\PY{p}{,} \PY{n}{P2}\PY{p}{,} \PY{n}{P4}\PY{p}{]}\PY{p}{)}\PY{o}{.}\PY{n}{det}\PY{p}{(}\PY{p}{)}\PY{p}{)}\PY{p}{[}\PY{l+m+mi}{0}\PY{p}{]}
\PY{k}{assert}\PY{p}{(}\PY{n}{J} \PY{o}{==} \PY{n}{S}\PY{o}{.}\PY{n}{ideal}\PY{p}{(}\PY{n}{scalar\PYZus{}product}\PY{p}{(}\PY{n}{P1}\PY{p}{,} \PY{n}{P4}\PY{p}{)}\PY{p}{,} \PY{n}{scalar\PYZus{}product}\PY{p}{(}\PY{n}{P1}\PY{p}{,} \PY{n}{P2}\PY{p}{)}\PY{p}{)}\PY{p}{)}

\PY{c+c1}{\PYZsh{}\PYZsh{} Moreover, we have that s16 is in J:}

\PY{k}{assert}\PY{p}{(}\PY{n}{scalar\PYZus{}product}\PY{p}{(}\PY{n}{P1}\PY{p}{,} \PY{n}{P6}\PY{p}{)} \PY{o+ow}{in} \PY{n}{J}\PY{p}{)}

\PY{c+c1}{\PYZsh{}\PYZsh{} So we define P1 in this way and we re\PYZhy{}write the points:}

\PY{n}{P2} \PY{o}{=} \PY{n}{vector}\PY{p}{(}\PY{n}{S}\PY{p}{,} \PY{p}{(}\PY{n}{A2}\PY{p}{,} \PY{n}{B2}\PY{p}{,} \PY{n}{C2}\PY{p}{)}\PY{p}{)}
\PY{n}{P4} \PY{o}{=} \PY{n}{vector}\PY{p}{(}\PY{n}{S}\PY{p}{,} \PY{p}{(}\PY{n}{A4}\PY{p}{,} \PY{n}{B4}\PY{p}{,} \PY{n}{C4}\PY{p}{)}\PY{p}{)}
\PY{n}{P1} \PY{o}{=} \PY{n}{vector}\PY{p}{(}\PY{n}{S}\PY{p}{,} \PY{n+nb}{list}\PY{p}{(}\PY{n}{wedge\PYZus{}product}\PY{p}{(}\PY{n}{P2}\PY{p}{,} \PY{n}{P4}\PY{p}{)}\PY{p}{)}\PY{p}{)}
\PY{n}{P3} \PY{o}{=} \PY{n}{u1}\PY{o}{*}\PY{n}{P1}\PY{o}{+}\PY{n}{u2}\PY{o}{*}\PY{n}{P2}
\PY{n}{P5} \PY{o}{=} \PY{n}{v1}\PY{o}{*}\PY{n}{P1}\PY{o}{+}\PY{n}{v2}\PY{o}{*}\PY{n}{P4}
\PY{n}{P6} \PY{o}{=} \PY{n}{w1}\PY{o}{*}\PY{n}{P2}\PY{o}{+}\PY{n}{w2}\PY{o}{*}\PY{n}{P4}
\end{Verbatim}
\end{tcolorbox}

    With the above points the ideal \(J\) below is zero.

We have several orthogonalities among the lines: * \(P_1 \vee P_2\)
orthogonal \(P_4 \vee P_6\) * \(P_1 \vee P_6\) orthogonal
\(P_2 \vee P_4\) * \(P_1 \vee P_4\) orthogonal \(P_2 \vee P_6\)

Hence the line \(P_2 \vee P_4 \vee P_6\) is orthogonal to
\(P_1 \vee P_2\), to \(P_1 \vee P_4\) and to \(P_1 \vee P_6\)

    \begin{tcolorbox}[breakable, size=fbox, boxrule=1pt, pad at break*=1mm,colback=cellbackground, colframe=cellborder]
\prompt{In}{incolor}{8}{\boxspacing}
\begin{Verbatim}[commandchars=\\\{\}]
\PY{n}{J} \PY{o}{=} \PY{n}{S}\PY{o}{.}\PY{n}{ideal}\PY{p}{(}
    \PY{n}{delta1}\PY{p}{(}\PY{n}{P2}\PY{p}{,} \PY{n}{P1}\PY{p}{,} \PY{n}{P4}\PY{p}{)}\PY{p}{,}
    \PY{n}{delta1}\PY{p}{(}\PY{n}{P4}\PY{p}{,} \PY{n}{P1}\PY{p}{,} \PY{n}{P2}\PY{p}{)}\PY{p}{,}
    \PY{n}{delta1}\PY{p}{(}\PY{n}{P6}\PY{p}{,} \PY{n}{P1}\PY{p}{,} \PY{n}{P2}\PY{p}{)}\PY{p}{,}
    \PY{n}{delta2}\PY{p}{(}\PY{n}{P1}\PY{p}{,} \PY{n}{P2}\PY{p}{,} \PY{n}{P3}\PY{p}{,} \PY{n}{P4}\PY{p}{,} \PY{n}{P5}\PY{p}{)}
\PY{p}{)}

\PY{k}{assert}\PY{p}{(}\PY{n}{J} \PY{o}{==} \PY{n}{S}\PY{o}{.}\PY{n}{ideal}\PY{p}{(}\PY{n}{S}\PY{o}{.}\PY{n}{zero}\PY{p}{(}\PY{p}{)}\PY{p}{)}\PY{p}{)}

\PY{c+c1}{\PYZsh{}\PYZsh{} orthogonalities among the lines:}
\PY{k}{assert}\PY{p}{(}\PY{n}{scalar\PYZus{}product}\PY{p}{(}\PY{n}{wedge\PYZus{}product}\PY{p}{(}\PY{n}{P1}\PY{p}{,} \PY{n}{P2}\PY{p}{)}\PY{p}{,} \PY{n}{wedge\PYZus{}product}\PY{p}{(}\PY{n}{P4}\PY{p}{,} \PY{n}{P6}\PY{p}{)}\PY{p}{)} \PY{o}{==} \PY{n}{S}\PY{p}{(}\PY{l+m+mi}{0}\PY{p}{)}\PY{p}{)}
\PY{k}{assert}\PY{p}{(}\PY{n}{scalar\PYZus{}product}\PY{p}{(}\PY{n}{wedge\PYZus{}product}\PY{p}{(}\PY{n}{P1}\PY{p}{,} \PY{n}{P6}\PY{p}{)}\PY{p}{,} \PY{n}{wedge\PYZus{}product}\PY{p}{(}\PY{n}{P2}\PY{p}{,} \PY{n}{P4}\PY{p}{)}\PY{p}{)} \PY{o}{==} \PY{n}{S}\PY{p}{(}\PY{l+m+mi}{0}\PY{p}{)}\PY{p}{)}
\PY{k}{assert}\PY{p}{(}\PY{n}{scalar\PYZus{}product}\PY{p}{(}\PY{n}{wedge\PYZus{}product}\PY{p}{(}\PY{n}{P1}\PY{p}{,} \PY{n}{P4}\PY{p}{)}\PY{p}{,} \PY{n}{wedge\PYZus{}product}\PY{p}{(}\PY{n}{P2}\PY{p}{,} \PY{n}{P6}\PY{p}{)}\PY{p}{)} \PY{o}{==} \PY{n}{S}\PY{p}{(}\PY{l+m+mi}{0}\PY{p}{)}\PY{p}{)}
\end{Verbatim}
\end{tcolorbox}

    The matrix \(M\) below must have rank \(\leq 9\). We select (in a
suitable way) one particular order 10 minor of \(M\) called \(Nx\) and
we compute its determinant. If \texttt{do\_long\_computations} is
\texttt{True}, the next block requires 8 minutes.

    \begin{tcolorbox}[breakable, size=fbox, boxrule=1pt, pad at break*=1mm,colback=cellbackground, colframe=cellborder]
\prompt{In}{incolor}{9}{\boxspacing}
\begin{Verbatim}[commandchars=\\\{\}]
\PY{n}{M} \PY{o}{=} \PY{n}{condition\PYZus{}matrix}\PY{p}{(}\PY{p}{[}\PY{n}{P1}\PY{p}{,} \PY{n}{P2}\PY{p}{,} \PY{n}{P3}\PY{p}{,} \PY{n}{P4}\PY{p}{,} \PY{n}{P5}\PY{p}{,} \PY{n}{P6}\PY{p}{]}\PY{p}{,} \PY{n}{S}\PY{p}{,} \PY{n}{standard}\PY{o}{=}\PY{l+s+s2}{\PYZdq{}}\PY{l+s+s2}{all}\PY{l+s+s2}{\PYZdq{}}\PY{p}{)}

\PY{n}{Nx} \PY{o}{=} \PY{n}{M}\PY{o}{.}\PY{n}{matrix\PYZus{}from\PYZus{}rows}\PY{p}{(}\PY{p}{[}\PY{l+m+mi}{0}\PY{p}{,} \PY{l+m+mi}{1}\PY{p}{,} \PY{l+m+mi}{3}\PY{p}{,} \PY{l+m+mi}{4}\PY{p}{,} \PY{l+m+mi}{6}\PY{p}{,} \PY{l+m+mi}{7}\PY{p}{,} \PY{l+m+mi}{9}\PY{p}{,} \PY{l+m+mi}{10}\PY{p}{,} \PY{l+m+mi}{12}\PY{p}{,} \PY{l+m+mi}{15}\PY{p}{]}\PY{p}{)}

\PY{n}{dn\PYZus{}old} \PY{o}{=} \PY{n}{expand}\PY{p}{(}
    \PY{p}{(}\PY{o}{\PYZhy{}}\PY{l+m+mi}{27}\PY{p}{)} \PY{o}{*} \PY{n}{A4} \PY{o}{*} \PY{n}{A2} \PY{o}{*} \PY{n}{w2} \PY{o}{*} \PY{n}{w1} \PY{o}{*} \PY{n}{v2} \PY{o}{*} \PY{n}{v1} \PY{o}{*} \PY{n}{u2}\PY{o}{\PYZca{}}\PY{l+m+mi}{2} \PY{o}{*} \PY{n}{u1}\PY{o}{\PYZca{}}\PY{l+m+mi}{2} \PY{o}{*} \PY{p}{(}\PY{o}{\PYZhy{}}\PY{n}{C2}\PY{o}{*}\PY{n}{B4} \PY{o}{+} \PY{n}{B2}\PY{o}{*}\PY{n}{C4}\PY{p}{)} \PY{o}{*} \PY{p}{(}\PY{o}{\PYZhy{}}\PY{n}{B2}\PY{o}{*}\PY{n}{A4} \PY{o}{+} \PY{n}{A2}\PY{o}{*}\PY{n}{B4}\PY{p}{)} 
    \PY{o}{*} \PY{p}{(}\PY{n}{A2}\PY{o}{\PYZca{}}\PY{l+m+mi}{2} \PY{o}{+} \PY{n}{B2}\PY{o}{\PYZca{}}\PY{l+m+mi}{2} \PY{o}{+} \PY{n}{C2}\PY{o}{\PYZca{}}\PY{l+m+mi}{2}\PY{p}{)} \PY{o}{*} \PY{p}{(}\PY{o}{\PYZhy{}}\PY{n}{u1}\PY{o}{*}\PY{n}{C2}\PY{o}{*}\PY{n}{B4} \PY{o}{+} \PY{n}{u1}\PY{o}{*}\PY{n}{B2}\PY{o}{*}\PY{n}{C4} \PY{o}{+} \PY{n}{u2}\PY{o}{*}\PY{n}{A2}\PY{p}{)} \PY{o}{*} \PY{p}{(}\PY{o}{\PYZhy{}}\PY{n}{C2}\PY{o}{*}\PY{n}{A4}\PY{o}{\PYZca{}}\PY{l+m+mi}{2} \PY{o}{\PYZhy{}} \PY{n}{C2}\PY{o}{*}\PY{n}{B4}\PY{o}{\PYZca{}}\PY{l+m+mi}{2} \PY{o}{+} \PY{n}{A2}\PY{o}{*}\PY{n}{A4}\PY{o}{*}\PY{n}{C4} \PY{o}{+} \PY{n}{B2}\PY{o}{*}\PY{n}{B4}\PY{o}{*}\PY{n}{C4}\PY{p}{)}
    \PY{o}{*} \PY{p}{(}\PY{n}{B2}\PY{o}{\PYZca{}}\PY{l+m+mi}{2}\PY{o}{*}\PY{n}{A4}\PY{o}{\PYZca{}}\PY{l+m+mi}{2} \PY{o}{+} \PY{n}{C2}\PY{o}{\PYZca{}}\PY{l+m+mi}{2}\PY{o}{*}\PY{n}{A4}\PY{o}{\PYZca{}}\PY{l+m+mi}{2} \PY{o}{\PYZhy{}} \PY{l+m+mi}{2}\PY{o}{*}\PY{n}{A2}\PY{o}{*}\PY{n}{B2}\PY{o}{*}\PY{n}{A4}\PY{o}{*}\PY{n}{B4} \PY{o}{+} \PY{n}{A2}\PY{o}{\PYZca{}}\PY{l+m+mi}{2}\PY{o}{*}\PY{n}{B4}\PY{o}{\PYZca{}}\PY{l+m+mi}{2} \PY{o}{+} \PY{n}{C2}\PY{o}{\PYZca{}}\PY{l+m+mi}{2}\PY{o}{*}\PY{n}{B4}\PY{o}{\PYZca{}}\PY{l+m+mi}{2} \PY{o}{\PYZhy{}} \PY{l+m+mi}{2}\PY{o}{*}\PY{n}{A2}\PY{o}{*}\PY{n}{C2}\PY{o}{*}\PY{n}{A4}\PY{o}{*}\PY{n}{C4} \PY{o}{\PYZhy{}} \PY{l+m+mi}{2}\PY{o}{*}\PY{n}{B2}\PY{o}{*}\PY{n}{C2}\PY{o}{*}\PY{n}{B4}\PY{o}{*}\PY{n}{C4} \PY{o}{+} \PY{n}{A2}\PY{o}{\PYZca{}}\PY{l+m+mi}{2}\PY{o}{*}\PY{n}{C4}\PY{o}{\PYZca{}}\PY{l+m+mi}{2} \PY{o}{+} \PY{n}{B2}\PY{o}{\PYZca{}}\PY{l+m+mi}{2}\PY{o}{*}\PY{n}{C4}\PY{o}{\PYZca{}}\PY{l+m+mi}{2}\PY{p}{)}\PY{o}{\PYZca{}}\PY{l+m+mi}{6} 
    \PY{o}{*} \PY{p}{(}\PY{o}{\PYZhy{}}\PY{l+m+mi}{2}\PY{o}{*}\PY{n}{u2}\PY{o}{*}\PY{n}{v1}\PY{o}{*}\PY{n}{w1}\PY{o}{*}\PY{n}{A2}\PY{o}{\PYZca{}}\PY{l+m+mi}{3}\PY{o}{*}\PY{n}{A4} \PY{o}{\PYZhy{}} \PY{l+m+mi}{2}\PY{o}{*}\PY{n}{u2}\PY{o}{*}\PY{n}{v1}\PY{o}{*}\PY{n}{w1}\PY{o}{*}\PY{n}{A2}\PY{o}{*}\PY{n}{B2}\PY{o}{\PYZca{}}\PY{l+m+mi}{2}\PY{o}{*}\PY{n}{A4} \PY{o}{\PYZhy{}} \PY{l+m+mi}{2}\PY{o}{*}\PY{n}{u2}\PY{o}{*}\PY{n}{v1}\PY{o}{*}\PY{n}{w1}\PY{o}{*}\PY{n}{A2}\PY{o}{*}\PY{n}{C2}\PY{o}{\PYZca{}}\PY{l+m+mi}{2}\PY{o}{*}\PY{n}{A4} \PY{o}{+} \PY{l+m+mi}{2}\PY{o}{*}\PY{n}{u1}\PY{o}{*}\PY{n}{v2}\PY{o}{*}\PY{n}{w1}\PY{o}{*}\PY{n}{A2}\PY{o}{\PYZca{}}\PY{l+m+mi}{2}\PY{o}{*}\PY{n}{A4}\PY{o}{\PYZca{}}\PY{l+m+mi}{2} \PY{o}{\PYZhy{}} \PY{l+m+mi}{2}\PY{o}{*}\PY{n}{u2}\PY{o}{*}\PY{n}{v1}\PY{o}{*}\PY{n}{w2}\PY{o}{*}\PY{n}{A2}\PY{o}{\PYZca{}}\PY{l+m+mi}{2}\PY{o}{*}\PY{n}{A4}\PY{o}{\PYZca{}}\PY{l+m+mi}{2} 
    \PY{o}{+} \PY{n}{u1}\PY{o}{*}\PY{n}{v2}\PY{o}{*}\PY{n}{w1}\PY{o}{*}\PY{n}{B2}\PY{o}{\PYZca{}}\PY{l+m+mi}{2}\PY{o}{*}\PY{n}{A4}\PY{o}{\PYZca{}}\PY{l+m+mi}{2} \PY{o}{\PYZhy{}} \PY{n}{u2}\PY{o}{*}\PY{n}{v1}\PY{o}{*}\PY{n}{w2}\PY{o}{*}\PY{n}{B2}\PY{o}{\PYZca{}}\PY{l+m+mi}{2}\PY{o}{*}\PY{n}{A4}\PY{o}{\PYZca{}}\PY{l+m+mi}{2} \PY{o}{+} \PY{n}{u1}\PY{o}{*}\PY{n}{v2}\PY{o}{*}\PY{n}{w1}\PY{o}{*}\PY{n}{C2}\PY{o}{\PYZca{}}\PY{l+m+mi}{2}\PY{o}{*}\PY{n}{A4}\PY{o}{\PYZca{}}\PY{l+m+mi}{2} \PY{o}{\PYZhy{}} \PY{n}{u2}\PY{o}{*}\PY{n}{v1}\PY{o}{*}\PY{n}{w2}\PY{o}{*}\PY{n}{C2}\PY{o}{\PYZca{}}\PY{l+m+mi}{2}\PY{o}{*}\PY{n}{A4}\PY{o}{\PYZca{}}\PY{l+m+mi}{2} \PY{o}{+} \PY{l+m+mi}{2}\PY{o}{*}\PY{n}{u1}\PY{o}{*}\PY{n}{v2}\PY{o}{*}\PY{n}{w2}\PY{o}{*}\PY{n}{A2}\PY{o}{*}\PY{n}{A4}\PY{o}{\PYZca{}}\PY{l+m+mi}{3} 
    \PY{o}{\PYZhy{}} \PY{l+m+mi}{2}\PY{o}{*}\PY{n}{u2}\PY{o}{*}\PY{n}{v1}\PY{o}{*}\PY{n}{w1}\PY{o}{*}\PY{n}{A2}\PY{o}{\PYZca{}}\PY{l+m+mi}{2}\PY{o}{*}\PY{n}{B2}\PY{o}{*}\PY{n}{B4} \PY{o}{\PYZhy{}} \PY{l+m+mi}{2}\PY{o}{*}\PY{n}{u2}\PY{o}{*}\PY{n}{v1}\PY{o}{*}\PY{n}{w1}\PY{o}{*}\PY{n}{B2}\PY{o}{\PYZca{}}\PY{l+m+mi}{3}\PY{o}{*}\PY{n}{B4} \PY{o}{\PYZhy{}} \PY{l+m+mi}{2}\PY{o}{*}\PY{n}{u2}\PY{o}{*}\PY{n}{v1}\PY{o}{*}\PY{n}{w1}\PY{o}{*}\PY{n}{B2}\PY{o}{*}\PY{n}{C2}\PY{o}{\PYZca{}}\PY{l+m+mi}{2}\PY{o}{*}\PY{n}{B4} \PY{o}{+} \PY{l+m+mi}{2}\PY{o}{*}\PY{n}{u1}\PY{o}{*}\PY{n}{v2}\PY{o}{*}\PY{n}{w1}\PY{o}{*}\PY{n}{A2}\PY{o}{*}\PY{n}{B2}\PY{o}{*}\PY{n}{A4}\PY{o}{*}\PY{n}{B4} \PY{o}{\PYZhy{}} \PY{l+m+mi}{2}\PY{o}{*}\PY{n}{u2}\PY{o}{*}\PY{n}{v1}\PY{o}{*}\PY{n}{w2}\PY{o}{*}\PY{n}{A2}\PY{o}{*}\PY{n}{B2}\PY{o}{*}\PY{n}{A4}\PY{o}{*}\PY{n}{B4} 
    \PY{o}{+} \PY{l+m+mi}{2}\PY{o}{*}\PY{n}{u1}\PY{o}{*}\PY{n}{v2}\PY{o}{*}\PY{n}{w2}\PY{o}{*}\PY{n}{B2}\PY{o}{*}\PY{n}{A4}\PY{o}{\PYZca{}}\PY{l+m+mi}{2}\PY{o}{*}\PY{n}{B4} \PY{o}{+} \PY{n}{u1}\PY{o}{*}\PY{n}{v2}\PY{o}{*}\PY{n}{w1}\PY{o}{*}\PY{n}{A2}\PY{o}{\PYZca{}}\PY{l+m+mi}{2}\PY{o}{*}\PY{n}{B4}\PY{o}{\PYZca{}}\PY{l+m+mi}{2} \PY{o}{\PYZhy{}} \PY{n}{u2}\PY{o}{*}\PY{n}{v1}\PY{o}{*}\PY{n}{w2}\PY{o}{*}\PY{n}{A2}\PY{o}{\PYZca{}}\PY{l+m+mi}{2}\PY{o}{*}\PY{n}{B4}\PY{o}{\PYZca{}}\PY{l+m+mi}{2} \PY{o}{+} \PY{l+m+mi}{2}\PY{o}{*}\PY{n}{u1}\PY{o}{*}\PY{n}{v2}\PY{o}{*}\PY{n}{w1}\PY{o}{*}\PY{n}{B2}\PY{o}{\PYZca{}}\PY{l+m+mi}{2}\PY{o}{*}\PY{n}{B4}\PY{o}{\PYZca{}}\PY{l+m+mi}{2} \PY{o}{\PYZhy{}} \PY{l+m+mi}{2}\PY{o}{*}\PY{n}{u2}\PY{o}{*}\PY{n}{v1}\PY{o}{*}\PY{n}{w2}\PY{o}{*}\PY{n}{B2}\PY{o}{\PYZca{}}\PY{l+m+mi}{2}\PY{o}{*}\PY{n}{B4}\PY{o}{\PYZca{}}\PY{l+m+mi}{2} 
    \PY{o}{+} \PY{n}{u1}\PY{o}{*}\PY{n}{v2}\PY{o}{*}\PY{n}{w1}\PY{o}{*}\PY{n}{C2}\PY{o}{\PYZca{}}\PY{l+m+mi}{2}\PY{o}{*}\PY{n}{B4}\PY{o}{\PYZca{}}\PY{l+m+mi}{2} \PY{o}{\PYZhy{}} \PY{n}{u2}\PY{o}{*}\PY{n}{v1}\PY{o}{*}\PY{n}{w2}\PY{o}{*}\PY{n}{C2}\PY{o}{\PYZca{}}\PY{l+m+mi}{2}\PY{o}{*}\PY{n}{B4}\PY{o}{\PYZca{}}\PY{l+m+mi}{2} \PY{o}{+} \PY{l+m+mi}{2}\PY{o}{*}\PY{n}{u1}\PY{o}{*}\PY{n}{v2}\PY{o}{*}\PY{n}{w2}\PY{o}{*}\PY{n}{A2}\PY{o}{*}\PY{n}{A4}\PY{o}{*}\PY{n}{B4}\PY{o}{\PYZca{}}\PY{l+m+mi}{2} \PY{o}{+} \PY{l+m+mi}{2}\PY{o}{*}\PY{n}{u1}\PY{o}{*}\PY{n}{v2}\PY{o}{*}\PY{n}{w2}\PY{o}{*}\PY{n}{B2}\PY{o}{*}\PY{n}{B4}\PY{o}{\PYZca{}}\PY{l+m+mi}{3} \PY{o}{\PYZhy{}} \PY{l+m+mi}{2}\PY{o}{*}\PY{n}{u2}\PY{o}{*}\PY{n}{v1}\PY{o}{*}\PY{n}{w1}\PY{o}{*}\PY{n}{A2}\PY{o}{\PYZca{}}\PY{l+m+mi}{2}\PY{o}{*}\PY{n}{C2}\PY{o}{*}\PY{n}{C4} 
    \PY{o}{\PYZhy{}} \PY{l+m+mi}{2}\PY{o}{*}\PY{n}{u2}\PY{o}{*}\PY{n}{v1}\PY{o}{*}\PY{n}{w1}\PY{o}{*}\PY{n}{B2}\PY{o}{\PYZca{}}\PY{l+m+mi}{2}\PY{o}{*}\PY{n}{C2}\PY{o}{*}\PY{n}{C4} \PY{o}{\PYZhy{}} \PY{l+m+mi}{2}\PY{o}{*}\PY{n}{u2}\PY{o}{*}\PY{n}{v1}\PY{o}{*}\PY{n}{w1}\PY{o}{*}\PY{n}{C2}\PY{o}{\PYZca{}}\PY{l+m+mi}{3}\PY{o}{*}\PY{n}{C4} \PY{o}{+} \PY{l+m+mi}{2}\PY{o}{*}\PY{n}{u1}\PY{o}{*}\PY{n}{v2}\PY{o}{*}\PY{n}{w1}\PY{o}{*}\PY{n}{A2}\PY{o}{*}\PY{n}{C2}\PY{o}{*}\PY{n}{A4}\PY{o}{*}\PY{n}{C4} \PY{o}{\PYZhy{}} \PY{l+m+mi}{2}\PY{o}{*}\PY{n}{u2}\PY{o}{*}\PY{n}{v1}\PY{o}{*}\PY{n}{w2}\PY{o}{*}\PY{n}{A2}\PY{o}{*}\PY{n}{C2}\PY{o}{*}\PY{n}{A4}\PY{o}{*}\PY{n}{C4} \PY{o}{+} \PY{l+m+mi}{2}\PY{o}{*}\PY{n}{u1}\PY{o}{*}\PY{n}{v2}\PY{o}{*}\PY{n}{w2}\PY{o}{*}\PY{n}{C2}\PY{o}{*}\PY{n}{A4}\PY{o}{\PYZca{}}\PY{l+m+mi}{2}\PY{o}{*}\PY{n}{C4} 
    \PY{o}{+} \PY{l+m+mi}{2}\PY{o}{*}\PY{n}{u1}\PY{o}{*}\PY{n}{v2}\PY{o}{*}\PY{n}{w1}\PY{o}{*}\PY{n}{B2}\PY{o}{*}\PY{n}{C2}\PY{o}{*}\PY{n}{B4}\PY{o}{*}\PY{n}{C4} \PY{o}{\PYZhy{}} \PY{l+m+mi}{2}\PY{o}{*}\PY{n}{u2}\PY{o}{*}\PY{n}{v1}\PY{o}{*}\PY{n}{w2}\PY{o}{*}\PY{n}{B2}\PY{o}{*}\PY{n}{C2}\PY{o}{*}\PY{n}{B4}\PY{o}{*}\PY{n}{C4} \PY{o}{+} \PY{l+m+mi}{2}\PY{o}{*}\PY{n}{u1}\PY{o}{*}\PY{n}{v2}\PY{o}{*}\PY{n}{w2}\PY{o}{*}\PY{n}{C2}\PY{o}{*}\PY{n}{B4}\PY{o}{\PYZca{}}\PY{l+m+mi}{2}\PY{o}{*}\PY{n}{C4} \PY{o}{+} \PY{n}{u1}\PY{o}{*}\PY{n}{v2}\PY{o}{*}\PY{n}{w1}\PY{o}{*}\PY{n}{A2}\PY{o}{\PYZca{}}\PY{l+m+mi}{2}\PY{o}{*}\PY{n}{C4}\PY{o}{\PYZca{}}\PY{l+m+mi}{2} \PY{o}{\PYZhy{}} \PY{n}{u2}\PY{o}{*}\PY{n}{v1}\PY{o}{*}\PY{n}{w2}\PY{o}{*}\PY{n}{A2}\PY{o}{\PYZca{}}\PY{l+m+mi}{2}\PY{o}{*}\PY{n}{C4}\PY{o}{\PYZca{}}\PY{l+m+mi}{2} 
    \PY{o}{+} \PY{n}{u1}\PY{o}{*}\PY{n}{v2}\PY{o}{*}\PY{n}{w1}\PY{o}{*}\PY{n}{B2}\PY{o}{\PYZca{}}\PY{l+m+mi}{2}\PY{o}{*}\PY{n}{C4}\PY{o}{\PYZca{}}\PY{l+m+mi}{2} \PY{o}{\PYZhy{}} \PY{n}{u2}\PY{o}{*}\PY{n}{v1}\PY{o}{*}\PY{n}{w2}\PY{o}{*}\PY{n}{B2}\PY{o}{\PYZca{}}\PY{l+m+mi}{2}\PY{o}{*}\PY{n}{C4}\PY{o}{\PYZca{}}\PY{l+m+mi}{2} \PY{o}{+} \PY{l+m+mi}{2}\PY{o}{*}\PY{n}{u1}\PY{o}{*}\PY{n}{v2}\PY{o}{*}\PY{n}{w1}\PY{o}{*}\PY{n}{C2}\PY{o}{\PYZca{}}\PY{l+m+mi}{2}\PY{o}{*}\PY{n}{C4}\PY{o}{\PYZca{}}\PY{l+m+mi}{2} \PY{o}{\PYZhy{}} \PY{l+m+mi}{2}\PY{o}{*}\PY{n}{u2}\PY{o}{*}\PY{n}{v1}\PY{o}{*}\PY{n}{w2}\PY{o}{*}\PY{n}{C2}\PY{o}{\PYZca{}}\PY{l+m+mi}{2}\PY{o}{*}\PY{n}{C4}\PY{o}{\PYZca{}}\PY{l+m+mi}{2} \PY{o}{+} \PY{l+m+mi}{2}\PY{o}{*}\PY{n}{u1}\PY{o}{*}\PY{n}{v2}\PY{o}{*}\PY{n}{w2}\PY{o}{*}\PY{n}{A2}\PY{o}{*}\PY{n}{A4}\PY{o}{*}\PY{n}{C4}\PY{o}{\PYZca{}}\PY{l+m+mi}{2} 
    \PY{o}{+} \PY{l+m+mi}{2}\PY{o}{*}\PY{n}{u1}\PY{o}{*}\PY{n}{v2}\PY{o}{*}\PY{n}{w2}\PY{o}{*}\PY{n}{B2}\PY{o}{*}\PY{n}{B4}\PY{o}{*}\PY{n}{C4}\PY{o}{\PYZca{}}\PY{l+m+mi}{2} \PY{o}{+} \PY{l+m+mi}{2}\PY{o}{*}\PY{n}{u1}\PY{o}{*}\PY{n}{v2}\PY{o}{*}\PY{n}{w2}\PY{o}{*}\PY{n}{C2}\PY{o}{*}\PY{n}{C4}\PY{o}{\PYZca{}}\PY{l+m+mi}{3}\PY{p}{)}
\PY{p}{)}

\PY{k}{if} \PY{n}{do\PYZus{}long\PYZus{}computations}\PY{p}{:}
    \PY{n}{dn} \PY{o}{=} \PY{n}{Nx}\PY{o}{.}\PY{n}{det}\PY{p}{(}\PY{p}{)}
\PY{k}{else}\PY{p}{:}
    \PY{n}{dn} \PY{o}{=} \PY{n}{dn\PYZus{}old}

\PY{k}{assert}\PY{p}{(}\PY{n}{dn} \PY{o}{==} \PY{n}{dn\PYZus{}old}\PY{p}{)}
\end{Verbatim}
\end{tcolorbox}

    Some factors of dn are specific of the choice of the minor of \(M\). We
consider only the last factor. In the next block we will see that it is
enough.

    \begin{tcolorbox}[breakable, size=fbox, boxrule=1pt, pad at break*=1mm,colback=cellbackground, colframe=cellborder]
\prompt{In}{incolor}{10}{\boxspacing}
\begin{Verbatim}[commandchars=\\\{\}]
\PY{n}{fdn} \PY{o}{=} \PY{n}{dn}\PY{o}{.}\PY{n}{factor}\PY{p}{(}\PY{p}{)}
\PY{n}{ftC} \PY{o}{=} \PY{n}{fdn}\PY{p}{[}\PY{o}{\PYZhy{}}\PY{l+m+mi}{1}\PY{p}{]}\PY{p}{[}\PY{l+m+mi}{0}\PY{p}{]}
\end{Verbatim}
\end{tcolorbox}

    In the computations below we shall show that the

rank of \(M\) is \(\leq 9\) iff the polynomial \texttt{ftC} is zero.

One way to see this, is to consider the ideal of all the order 10 minors
of M, but this computation requires too much time. Hence we assume that
the point \(P_1\) is \((1: 0: 0)\) or \((1: i: 0)\). In this case the
computation of the ideal of all the order 10 minors of the corresponding
matrix \(M\) is easy to manipulate.

    \hypertarget{case-p_1-1-0-0}{%
\subsection{\texorpdfstring{Case
\(P_1 = (1: 0: 0)\)}{Case P\_1 = (1: 0: 0)}}\label{case-p_1-1-0-0}}

    Since \(s_{12} = 0\), \(s_{14} = 0\), we redefine the points and we
re-define the matrix \(M\) (i.e.~\(A_2=0\), \(A_4=0\))

    \begin{tcolorbox}[breakable, size=fbox, boxrule=1pt, pad at break*=1mm,colback=cellbackground, colframe=cellborder]
\prompt{In}{incolor}{11}{\boxspacing}
\begin{Verbatim}[commandchars=\\\{\}]
\PY{n}{P1} \PY{o}{=} \PY{n}{vector}\PY{p}{(}\PY{n}{S}\PY{p}{,} \PY{p}{(}\PY{l+m+mi}{1}\PY{p}{,} \PY{l+m+mi}{0}\PY{p}{,} \PY{l+m+mi}{0}\PY{p}{)}\PY{p}{)}
\PY{n}{P2} \PY{o}{=} \PY{n}{vector}\PY{p}{(}\PY{n}{S}\PY{p}{,} \PY{p}{(}\PY{l+m+mi}{0}\PY{p}{,} \PY{n}{B2}\PY{p}{,} \PY{n}{C2}\PY{p}{)}\PY{p}{)}
\PY{n}{P3} \PY{o}{=} \PY{n}{u1}\PY{o}{*}\PY{n}{P1}\PY{o}{+}\PY{n}{u2}\PY{o}{*}\PY{n}{P2}
\PY{n}{P4} \PY{o}{=} \PY{n}{vector}\PY{p}{(}\PY{n}{S}\PY{p}{,} \PY{p}{(}\PY{l+m+mi}{0}\PY{p}{,} \PY{n}{B4}\PY{p}{,} \PY{n}{C4}\PY{p}{)}\PY{p}{)}
\PY{n}{P5} \PY{o}{=} \PY{n}{v1}\PY{o}{*}\PY{n}{P1}\PY{o}{+}\PY{n}{v2}\PY{o}{*}\PY{n}{P4}
\PY{n}{P6} \PY{o}{=} \PY{n}{w1}\PY{o}{*}\PY{n}{P2}\PY{o}{+}\PY{n}{w2}\PY{o}{*}\PY{n}{P4}

\PY{n}{M1} \PY{o}{=} \PY{n}{condition\PYZus{}matrix}\PY{p}{(}\PY{p}{[}\PY{n}{P1}\PY{p}{,} \PY{n}{P2}\PY{p}{,} \PY{n}{P3}\PY{p}{,} \PY{n}{P4}\PY{p}{,} \PY{n}{P5}\PY{p}{,} \PY{n}{P6}\PY{p}{]}\PY{p}{,} \PY{n}{S}\PY{p}{,} \PY{n}{standard}\PY{o}{=}\PY{l+s+s2}{\PYZdq{}}\PY{l+s+s2}{all}\PY{l+s+s2}{\PYZdq{}}\PY{p}{)}
\end{Verbatim}
\end{tcolorbox}

    The matrix M1 has the 0th row equals to (0,1,0,\ldots,0) the 1st row
equals to (0,0,0,0,1,0,\ldots,0) and the 2nd row given by: (0, 0,\ldots,
0). Hence we can extract from M1 a matrix N1 which does not have the
rows 0, 1, 2 and does not have the columns 1 and 4. All the order 10
minors of M1 are 0 iff all the order 8 minors of N1 are 0.

    \begin{tcolorbox}[breakable, size=fbox, boxrule=1pt, pad at break*=1mm,colback=cellbackground, colframe=cellborder]
\prompt{In}{incolor}{12}{\boxspacing}
\begin{Verbatim}[commandchars=\\\{\}]
\PY{n}{N1} \PY{o}{=} \PY{n}{M1}\PY{o}{.}\PY{n}{matrix\PYZus{}from\PYZus{}rows\PYZus{}and\PYZus{}columns}\PY{p}{(}
    \PY{p}{[}\PY{l+m+mi}{3}\PY{p}{,}\PY{l+m+mi}{4}\PY{p}{,}\PY{l+m+mi}{5}\PY{p}{,}\PY{l+m+mi}{6}\PY{p}{,}\PY{l+m+mi}{7}\PY{p}{,}\PY{l+m+mi}{8}\PY{p}{,}\PY{l+m+mi}{9}\PY{p}{,}\PY{l+m+mi}{10}\PY{p}{,}\PY{l+m+mi}{11}\PY{p}{,}\PY{l+m+mi}{12}\PY{p}{,}\PY{l+m+mi}{13}\PY{p}{,}\PY{l+m+mi}{14}\PY{p}{,}\PY{l+m+mi}{15}\PY{p}{,}\PY{l+m+mi}{16}\PY{p}{,}\PY{l+m+mi}{17}\PY{p}{]}\PY{p}{,}
    \PY{p}{[}\PY{l+m+mi}{0}\PY{p}{,}\PY{l+m+mi}{2}\PY{p}{,}\PY{l+m+mi}{3}\PY{p}{,}\PY{l+m+mi}{5}\PY{p}{,}\PY{l+m+mi}{6}\PY{p}{,}\PY{l+m+mi}{7}\PY{p}{,}\PY{l+m+mi}{8}\PY{p}{,}\PY{l+m+mi}{9}\PY{p}{]}
\PY{p}{)}
\end{Verbatim}
\end{tcolorbox}

    We see that the following rows of N1 are linearly dependent: * row 1 and
row 2 * row 7 and row 8 * row 13 and row 14 * row 4 and row 5 and row 6
* row 10 and row 11 and row 12 * row 1 and row 13 and row 14

    Hence, in order to compute all the order 8 minors of N1, we can skip
several submatrices.

We construct therefore the list \texttt{LL1} of all the rows of 8
elements which have to be considered and we get that \texttt{LL1}
contains 1362 elements:

    \begin{tcolorbox}[breakable, size=fbox, boxrule=1pt, pad at break*=1mm,colback=cellbackground, colframe=cellborder]
\prompt{In}{incolor}{13}{\boxspacing}
\begin{Verbatim}[commandchars=\\\{\}]
\PY{k}{assert}\PY{p}{(}\PY{n}{N1}\PY{o}{.}\PY{n}{matrix\PYZus{}from\PYZus{}rows}\PY{p}{(}\PY{p}{[}\PY{l+m+mi}{0}\PY{p}{,} \PY{l+m+mi}{1}\PY{p}{]}\PY{p}{)}\PY{o}{.}\PY{n}{rank}\PY{p}{(}\PY{p}{)} \PY{o}{==} \PY{l+m+mi}{1}\PY{p}{)}

\PY{k}{assert}\PY{p}{(}\PY{n}{N1}\PY{o}{.}\PY{n}{matrix\PYZus{}from\PYZus{}rows}\PY{p}{(}\PY{p}{[}\PY{l+m+mi}{6}\PY{p}{,} \PY{l+m+mi}{7}\PY{p}{]}\PY{p}{)}\PY{o}{.}\PY{n}{rank}\PY{p}{(}\PY{p}{)} \PY{o}{==} \PY{l+m+mi}{1}\PY{p}{)}
\PY{k}{assert}\PY{p}{(}\PY{n}{N1}\PY{o}{.}\PY{n}{matrix\PYZus{}from\PYZus{}rows}\PY{p}{(}\PY{p}{[}\PY{l+m+mi}{12}\PY{p}{,} \PY{l+m+mi}{13}\PY{p}{]}\PY{p}{)}\PY{o}{.}\PY{n}{rank}\PY{p}{(}\PY{p}{)} \PY{o}{==} \PY{l+m+mi}{1}\PY{p}{)}
\PY{k}{assert}\PY{p}{(}\PY{n}{N1}\PY{o}{.}\PY{n}{matrix\PYZus{}from\PYZus{}rows}\PY{p}{(}\PY{p}{[}\PY{l+m+mi}{3}\PY{p}{,} \PY{l+m+mi}{4}\PY{p}{,} \PY{l+m+mi}{5}\PY{p}{]}\PY{p}{)}\PY{o}{.}\PY{n}{rank}\PY{p}{(}\PY{p}{)} \PY{o}{==} \PY{l+m+mi}{2}\PY{p}{)}
\PY{k}{assert}\PY{p}{(}\PY{n}{N1}\PY{o}{.}\PY{n}{matrix\PYZus{}from\PYZus{}rows}\PY{p}{(}\PY{p}{[}\PY{l+m+mi}{9}\PY{p}{,} \PY{l+m+mi}{10}\PY{p}{,} \PY{l+m+mi}{11}\PY{p}{]}\PY{p}{)}\PY{o}{.}\PY{n}{rank}\PY{p}{(}\PY{p}{)} \PY{o}{==} \PY{l+m+mi}{2}\PY{p}{)}
\PY{k}{assert}\PY{p}{(}\PY{n}{N1}\PY{o}{.}\PY{n}{matrix\PYZus{}from\PYZus{}rows}\PY{p}{(}\PY{p}{[}\PY{l+m+mi}{0}\PY{p}{,} \PY{l+m+mi}{12}\PY{p}{,} \PY{l+m+mi}{13}\PY{p}{]}\PY{p}{)}\PY{o}{.}\PY{n}{rank}\PY{p}{(}\PY{p}{)} \PY{o}{==} \PY{l+m+mi}{2}\PY{p}{)}
\end{Verbatim}
\end{tcolorbox}

    \begin{tcolorbox}[breakable, size=fbox, boxrule=1pt, pad at break*=1mm,colback=cellbackground, colframe=cellborder]
\prompt{In}{incolor}{14}{\boxspacing}
\begin{Verbatim}[commandchars=\\\{\}]
\PY{c+c1}{\PYZsh{}\PYZsh{} Here we construct the list of all the rows of 8 elements}
\PY{c+c1}{\PYZsh{}\PYZsh{} which have to be considered:}

\PY{n}{L1} \PY{o}{=} \PY{n+nb}{list}\PY{p}{(}\PY{n}{Combinations}\PY{p}{(}\PY{l+m+mi}{15}\PY{p}{,}\PY{l+m+mi}{8}\PY{p}{)}\PY{p}{)}

\PY{n}{LL1} \PY{o}{=} \PY{p}{[}\PY{p}{]}
\PY{k}{for} \PY{n}{lx} \PY{o+ow}{in} \PY{n}{L1}\PY{p}{:}
    \PY{n}{ll} \PY{o}{=} \PY{n}{Set}\PY{p}{(}\PY{n}{lx}\PY{p}{)}
    \PY{k}{if} \PY{n}{Set}\PY{p}{(}\PY{p}{[}\PY{l+m+mi}{0}\PY{p}{,}\PY{l+m+mi}{1}\PY{p}{]}\PY{p}{)}\PY{o}{.}\PY{n}{issubset}\PY{p}{(}\PY{n}{ll}\PY{p}{)} \PY{o+ow}{or} \PY{n}{Set}\PY{p}{(}\PY{p}{[}\PY{l+m+mi}{6}\PY{p}{,} \PY{l+m+mi}{7}\PY{p}{]}\PY{p}{)}\PY{o}{.}\PY{n}{issubset}\PY{p}{(}\PY{n}{ll}\PY{p}{)} \PY{o+ow}{or} \PY{n}{Set}\PY{p}{(}\PY{p}{[}\PY{l+m+mi}{12}\PY{p}{,} \PY{l+m+mi}{13}\PY{p}{]}\PY{p}{)}\PY{o}{.}\PY{n}{issubset}\PY{p}{(}\PY{n}{ll}\PY{p}{)} \PY{o+ow}{or} \PYZbs{}
    \PY{n}{Set}\PY{p}{(}\PY{p}{[}\PY{l+m+mi}{3}\PY{p}{,} \PY{l+m+mi}{4}\PY{p}{,} \PY{l+m+mi}{5}\PY{p}{]}\PY{p}{)}\PY{o}{.}\PY{n}{issubset}\PY{p}{(}\PY{n}{ll}\PY{p}{)} \PY{o+ow}{or} \PY{n}{Set}\PY{p}{(}\PY{p}{[}\PY{l+m+mi}{9}\PY{p}{,} \PY{l+m+mi}{10}\PY{p}{,} \PY{l+m+mi}{11}\PY{p}{]}\PY{p}{)}\PY{o}{.}\PY{n}{issubset}\PY{p}{(}\PY{n}{ll}\PY{p}{)} \PY{o+ow}{or} \PY{n}{Set}\PY{p}{(}\PY{p}{[}\PY{l+m+mi}{0}\PY{p}{,} \PY{l+m+mi}{12}\PY{p}{,} \PY{l+m+mi}{13}\PY{p}{]}\PY{p}{)}\PY{o}{.}\PY{n}{issubset}\PY{p}{(}\PY{n}{ll}\PY{p}{)}\PY{p}{:}
        \PY{k}{continue}
    \PY{k}{else}\PY{p}{:}
        \PY{n}{LL1}\PY{o}{.}\PY{n}{append}\PY{p}{(}\PY{n}{lx}\PY{p}{)}

\PY{c+c1}{\PYZsh{}\PYZsh{} LL1 contains 1362 elements:}
\PY{k}{assert}\PY{p}{(}\PY{n+nb}{len}\PY{p}{(}\PY{n}{LL1}\PY{p}{)} \PY{o}{==} \PY{l+m+mi}{1362}\PY{p}{)}
\end{Verbatim}
\end{tcolorbox}

    The polynomial \texttt{ftC} constructed above should appear in the
factors of the determinant of the order 8 minors of \(N_1\) (when
specialized with the condition \(A_2 = 0\), \(A_4 = 0\)) We verify this
and we collect the order 8 minors of N1 divided by the polynomial
\texttt{ftC} specialized (called \texttt{ftCs}). About 16 seconds of
computations.

    \begin{tcolorbox}[breakable, size=fbox, boxrule=1pt, pad at break*=1mm,colback=cellbackground, colframe=cellborder]
\prompt{In}{incolor}{15}{\boxspacing}
\begin{Verbatim}[commandchars=\\\{\}]
\PY{n}{ftCs} \PY{o}{=} \PY{n}{ftC}\PY{o}{.}\PY{n}{subs}\PY{p}{(}\PY{p}{\PYZob{}}\PY{n}{A2}\PY{p}{:}\PY{l+m+mi}{0}\PY{p}{,} \PY{n}{A4}\PY{p}{:}\PY{l+m+mi}{0}\PY{p}{\PYZcb{}}\PY{p}{)}

\PY{n}{JJ} \PY{o}{=} \PY{p}{[}\PY{p}{]}
\PY{k}{for} \PY{n}{nr} \PY{o+ow}{in} \PY{n}{LL1}\PY{p}{:}
    \PY{n}{NN} \PY{o}{=} \PY{n}{N1}\PY{o}{.}\PY{n}{matrix\PYZus{}from\PYZus{}rows}\PY{p}{(}\PY{n}{nr}\PY{p}{)}
    \PY{n}{dt} \PY{o}{=} \PY{n}{NN}\PY{o}{.}\PY{n}{det}\PY{p}{(}\PY{p}{)}
    \PY{n}{dvs} \PY{o}{=} \PY{n}{dt}\PY{o}{.}\PY{n}{quo\PYZus{}rem}\PY{p}{(}\PY{n}{ftCs}\PY{p}{)}
    \PY{k}{if} \PY{n}{dvs}\PY{p}{[}\PY{l+m+mi}{1}\PY{p}{]} \PY{o}{!=} \PY{l+m+mi}{0}\PY{p}{:}
        \PY{n+nb}{print}\PY{p}{(}\PY{l+s+s2}{\PYZdq{}}\PY{l+s+s2}{Unexpacted situation. The minor is not a multiple of ftCs}\PY{l+s+s2}{\PYZdq{}}\PY{p}{)}
        \PY{n+nb}{print}\PY{p}{(}\PY{l+s+s2}{\PYZdq{}}\PY{l+s+s2}{Do not trust to the next computations, something went wrong!}\PY{l+s+s2}{\PYZdq{}}\PY{p}{)}
    \PY{k}{else}\PY{p}{:}
        \PY{n}{JJ}\PY{o}{.}\PY{n}{append}\PY{p}{(}\PY{n}{dvs}\PY{p}{[}\PY{l+m+mi}{0}\PY{p}{]}\PY{p}{)}
\end{Verbatim}
\end{tcolorbox}

    We define the ideal generated by JJ and we saturate it. We get that JJ =
(1), so the order 8 minors of N1 are 0 iff ftCs is zero.

    \begin{tcolorbox}[breakable, size=fbox, boxrule=1pt, pad at break*=1mm,colback=cellbackground, colframe=cellborder]
\prompt{In}{incolor}{16}{\boxspacing}
\begin{Verbatim}[commandchars=\\\{\}]
\PY{n}{JJ} \PY{o}{=} \PY{n}{S}\PY{o}{.}\PY{n}{ideal}\PY{p}{(}\PY{n}{JJ}\PY{p}{)}
\PY{n}{JJ} \PY{o}{=} \PY{n}{JJ}\PY{o}{.}\PY{n}{saturation}\PY{p}{(}\PY{n}{u1}\PY{o}{*}\PY{n}{u2}\PY{o}{*}\PY{n}{v1}\PY{o}{*}\PY{n}{v2}\PY{o}{*}\PY{n}{w1}\PY{o}{*}\PY{n}{w2}\PY{p}{)}\PY{p}{[}\PY{l+m+mi}{0}\PY{p}{]}
\PY{n}{JJ} \PY{o}{=} \PY{n}{JJ}\PY{o}{.}\PY{n}{saturation}\PY{p}{(}\PY{n}{S}\PY{o}{.}\PY{n}{ideal}\PY{p}{(}\PY{n}{matrix}\PY{p}{(}\PY{p}{[}\PY{n}{P2}\PY{p}{,} \PY{n}{P4}\PY{p}{]}\PY{p}{)}\PY{o}{.}\PY{n}{minors}\PY{p}{(}\PY{l+m+mi}{2}\PY{p}{)}\PY{p}{)}\PY{p}{)}\PY{p}{[}\PY{l+m+mi}{0}\PY{p}{]}

\PY{k}{assert}\PY{p}{(}\PY{n}{JJ} \PY{o}{==} \PY{n}{S}\PY{o}{.}\PY{n}{ideal}\PY{p}{(}\PY{n}{S}\PY{p}{(}\PY{l+m+mi}{1}\PY{p}{)}\PY{p}{)}\PY{p}{)}
\end{Verbatim}
\end{tcolorbox}

    Hence, in case \(P_1=(1:0:0)\), the matrix \(M\) has rank \(\leq 9\) iff
tfC = 0. Now the other case

    \hypertarget{case-p_1-1-i-0}{%
\subsection{\texorpdfstring{Case
\(P_1 = (1: i: 0)\)}{Case P\_1 = (1: i: 0)}}\label{case-p_1-1-i-0}}

    Then we have to consider the case in which \(P_1=(1: i: 0)\).

But in this case we use the following result:

If Q2, Q4 are points of the plane, if Q1 = wedge\_product(Q2, Q4) and if
Q1 is on the isotropic conic, i.e.~scalar\_product(Q1,Q1) = 0, then Q1,
Q2, Q4 are aligned:

    \begin{tcolorbox}[breakable, size=fbox, boxrule=1pt, pad at break*=1mm,colback=cellbackground, colframe=cellborder]
\prompt{In}{incolor}{13}{\boxspacing}
\begin{Verbatim}[commandchars=\\\{\}]
\PY{n}{Q4} \PY{o}{=} \PY{n}{vector}\PY{p}{(}\PY{n}{S}\PY{p}{,} \PY{p}{(}\PY{n}{A4}\PY{p}{,} \PY{n}{B4}\PY{p}{,} \PY{n}{C4}\PY{p}{)}\PY{p}{)}
\PY{n}{Q2} \PY{o}{=} \PY{n}{vector}\PY{p}{(}\PY{n}{S}\PY{p}{,} \PY{p}{(}\PY{n}{A2}\PY{p}{,} \PY{n}{B2}\PY{p}{,} \PY{n}{C2}\PY{p}{)}\PY{p}{)}
\PY{n}{Q1} \PY{o}{=} \PY{n}{wedge\PYZus{}product}\PY{p}{(}\PY{n}{Q2}\PY{p}{,} \PY{n}{Q4}\PY{p}{)}
\PY{k}{assert}\PY{p}{(}\PY{n}{det}\PY{p}{(}\PY{n}{matrix}\PY{p}{(}\PY{p}{[}\PY{n}{Q1}\PY{p}{,} \PY{n}{Q2}\PY{p}{,} \PY{n}{Q4}\PY{p}{]}\PY{p}{)}\PY{p}{)} \PY{o}{==} \PY{n}{scalar\PYZus{}product}\PY{p}{(}\PY{n}{Q1}\PY{p}{,} \PY{n}{Q1}\PY{p}{)}\PY{p}{)}
\end{Verbatim}
\end{tcolorbox}

    From this we get that the case \(P_1 = (1: i: 0)\) does not need to be
considered.

    First conclusion: configuration \((C_5)\) is possible iff the polynomial
\texttt{ftC} is zero.

    \hypertarget{construction-of-the-point-p_6}{%
\subsection{\texorpdfstring{Construction of the point
\(P_6\)}{Construction of the point P\_6}}\label{construction-of-the-point-p_6}}

    We go back to the general case. We redefine the points

From the above computations we have that \(P_1\) cannot be a point on
the isotropic conic (indeed we sow that if \(P_1\) is on the isotropic
conic, then \(P_1, P_2, P_4\) are collinear). Hence \(s_{11}\) is not
zero.

We have that the condition \texttt{ftC}, which is the condition that
implies that \(P_1, P_2, P_3, P_4, P_5, P_6\) in configuration \((C_5)\)
are eigenpoints, can be expressed by:

\[
\Bigl(
\left\langle P_2, P_6 \right\rangle 
 \bigl( 
   \left\langle P_4, P_5 \right\rangle \, \left\langle P_1, P_3 \right\rangle - 
   \left\langle P_4, P_3 \right\rangle \, \left\langle P_1, P_5 \right\rangle 
 \bigr) + 
\left\langle P_4, P_6 \right\rangle 
 \bigl(
  \left\langle P_2, P_5 \right\rangle \, \left\langle P_1, P_3 \right\rangle - 
  \left\langle P_2, P_3 \right\rangle \, \left\langle P_1, P_5 \right\rangle
 \bigr)
\Bigr) / \left\langle P_1, P_1 \right\rangle
\]

    \begin{tcolorbox}[breakable, size=fbox, boxrule=1pt, pad at break*=1mm,colback=cellbackground, colframe=cellborder]
\prompt{In}{incolor}{17}{\boxspacing}
\begin{Verbatim}[commandchars=\\\{\}]
\PY{n}{P2} \PY{o}{=} \PY{n}{vector}\PY{p}{(}\PY{n}{S}\PY{p}{,} \PY{p}{(}\PY{n}{A2}\PY{p}{,} \PY{n}{B2}\PY{p}{,} \PY{n}{C2}\PY{p}{)}\PY{p}{)}
\PY{n}{P4} \PY{o}{=} \PY{n}{vector}\PY{p}{(}\PY{n}{S}\PY{p}{,} \PY{p}{(}\PY{n}{A4}\PY{p}{,} \PY{n}{B4}\PY{p}{,} \PY{n}{C4}\PY{p}{)}\PY{p}{)}
\PY{n}{P1} \PY{o}{=} \PY{n}{vector}\PY{p}{(}\PY{n}{S}\PY{p}{,} \PY{n+nb}{list}\PY{p}{(}\PY{n}{wedge\PYZus{}product}\PY{p}{(}\PY{n}{P2}\PY{p}{,} \PY{n}{P4}\PY{p}{)}\PY{p}{)}\PY{p}{)}
\PY{n}{P3} \PY{o}{=} \PY{n}{u1}\PY{o}{*}\PY{n}{P1}\PY{o}{+}\PY{n}{u2}\PY{o}{*}\PY{n}{P2}
\PY{n}{P5} \PY{o}{=} \PY{n}{v1}\PY{o}{*}\PY{n}{P1}\PY{o}{+}\PY{n}{v2}\PY{o}{*}\PY{n}{P4}
\PY{n}{P6} \PY{o}{=} \PY{n}{w1}\PY{o}{*}\PY{n}{P2}\PY{o}{+}\PY{n}{w2}\PY{o}{*}\PY{n}{P4}
\end{Verbatim}
\end{tcolorbox}

    \begin{tcolorbox}[breakable, size=fbox, boxrule=1pt, pad at break*=1mm,colback=cellbackground, colframe=cellborder]
\prompt{In}{incolor}{18}{\boxspacing}
\begin{Verbatim}[commandchars=\\\{\}]
\PY{n}{ftC1} \PY{o}{=} \PY{p}{(}
    \PY{n}{scalar\PYZus{}product}\PY{p}{(}\PY{n}{P2}\PY{p}{,}\PY{n}{P6}\PY{p}{)}\PY{o}{*}\PY{p}{(}
        \PY{n}{scalar\PYZus{}product}\PY{p}{(}\PY{n}{P4}\PY{p}{,}\PY{n}{P5}\PY{p}{)}\PY{o}{*}\PY{n}{scalar\PYZus{}product}\PY{p}{(}\PY{n}{P1}\PY{p}{,} \PY{n}{P3}\PY{p}{)}
        \PY{o}{\PYZhy{}} \PY{n}{scalar\PYZus{}product}\PY{p}{(}\PY{n}{P4}\PY{p}{,}\PY{n}{P3}\PY{p}{)}\PY{o}{*}\PY{n}{scalar\PYZus{}product}\PY{p}{(}\PY{n}{P1}\PY{p}{,} \PY{n}{P5}\PY{p}{)}
    \PY{p}{)} \PY{o}{+} \PY{n}{scalar\PYZus{}product}\PY{p}{(}\PY{n}{P4}\PY{p}{,}\PY{n}{P6}\PY{p}{)}\PY{o}{*}\PY{p}{(}
        \PY{n}{scalar\PYZus{}product}\PY{p}{(}\PY{n}{P2}\PY{p}{,}\PY{n}{P5}\PY{p}{)}\PY{o}{*}\PY{n}{scalar\PYZus{}product}\PY{p}{(}\PY{n}{P1}\PY{p}{,} \PY{n}{P3}\PY{p}{)}
        \PY{o}{\PYZhy{}} \PY{n}{scalar\PYZus{}product}\PY{p}{(}\PY{n}{P2}\PY{p}{,}\PY{n}{P3}\PY{p}{)}\PY{o}{*}\PY{n}{scalar\PYZus{}product}\PY{p}{(}\PY{n}{P1}\PY{p}{,} \PY{n}{P5}\PY{p}{)}
    \PY{p}{)}
\PY{p}{)}

\PY{k}{assert}\PY{p}{(}\PY{n}{ftC1} \PY{o}{==} \PY{o}{\PYZhy{}}\PY{l+m+mi}{2}\PY{o}{*}\PY{n}{scalar\PYZus{}product}\PY{p}{(}\PY{n}{P1}\PY{p}{,} \PY{n}{P1}\PY{p}{)}\PY{o}{*}\PY{n}{ftC}\PY{p}{)}
\end{Verbatim}
\end{tcolorbox}

    since \(s_{11}\) is never zero, we have that ftC = 0 iff ftC1 = 0.

Hence \(P_1, \dotsc, P_6\) in config (5) are eigenpoints iff

\(s_{26}(s_{45}s_{13} - s_{34}s_{15}) + s_{46}(s_{25}s_{13} - s_{23}s_{15})=0\)

This proves the formula of configuration C5 (probably (25))

    If we substitute in the expression
\(s_{26}(s_{45}s_{13} - s_{34}s_{15}) + s_{46}(s_{25}s_{13} - s_{23}s_{15})=0\)
in place of \(P_3\) the expression \(u_1P_1+u_2P_2\), we get a new
equation, linear in \(u_1\) and \(u_2\) which is equal to
\(u_1U_2+u_2U_1=0\), where \(U1\) and \(U_2\) are defined as follows:

\(U_1 = s_{12}(s_{26}s_{45}+s_{46}s_{25})-s_{26}s_{15}s_{24}-s_{46}s_{15}s_{22}\)

\(U_2 = s_{11}(s_{26}s_{45}+s_{46}s_{25})-s_{26}s_{15}s_{14}-s_{46}s_{15}s_{12}\)

Hence

ftC is zero iff \(u_1 = U_1, u_2 = -U_2\):

    \begin{tcolorbox}[breakable, size=fbox, boxrule=1pt, pad at break*=1mm,colback=cellbackground, colframe=cellborder]
\prompt{In}{incolor}{19}{\boxspacing}
\begin{Verbatim}[commandchars=\\\{\}]
\PY{n}{U1} \PY{o}{=} \PY{p}{(}
    \PY{n}{scalar\PYZus{}product}\PY{p}{(}\PY{n}{P1}\PY{p}{,} \PY{n}{P2}\PY{p}{)}\PY{o}{*}\PY{p}{(}
        \PY{n}{scalar\PYZus{}product}\PY{p}{(}\PY{n}{P2}\PY{p}{,} \PY{n}{P6}\PY{p}{)}\PY{o}{*}\PY{n}{scalar\PYZus{}product}\PY{p}{(}\PY{n}{P4}\PY{p}{,} \PY{n}{P5}\PY{p}{)}
        \PY{o}{+} \PY{n}{scalar\PYZus{}product}\PY{p}{(}\PY{n}{P4}\PY{p}{,} \PY{n}{P6}\PY{p}{)}\PY{o}{*}\PY{n}{scalar\PYZus{}product}\PY{p}{(}\PY{n}{P2}\PY{p}{,}\PY{n}{P5}\PY{p}{)}
    \PY{p}{)} \PY{o}{\PYZhy{}} \PY{n}{scalar\PYZus{}product}\PY{p}{(}\PY{n}{P2}\PY{p}{,} \PY{n}{P6}\PY{p}{)}\PY{o}{*}\PY{n}{scalar\PYZus{}product}\PY{p}{(}\PY{n}{P1}\PY{p}{,} \PY{n}{P5}\PY{p}{)}\PY{o}{*}\PY{n}{scalar\PYZus{}product}\PY{p}{(}\PY{n}{P2}\PY{p}{,} \PY{n}{P4}\PY{p}{)}
    \PY{o}{\PYZhy{}} \PY{n}{scalar\PYZus{}product}\PY{p}{(}\PY{n}{P4}\PY{p}{,} \PY{n}{P6}\PY{p}{)}\PY{o}{*}\PY{n}{scalar\PYZus{}product}\PY{p}{(}\PY{n}{P1}\PY{p}{,} \PY{n}{P5}\PY{p}{)}\PY{o}{*}\PY{n}{scalar\PYZus{}product}\PY{p}{(}\PY{n}{P2}\PY{p}{,} \PY{n}{P2}\PY{p}{)}
\PY{p}{)}

\PY{n}{U2} \PY{o}{=} \PY{p}{(}
    \PY{n}{scalar\PYZus{}product}\PY{p}{(}\PY{n}{P1}\PY{p}{,} \PY{n}{P1}\PY{p}{)}\PY{o}{*}\PY{p}{(}
        \PY{n}{scalar\PYZus{}product}\PY{p}{(}\PY{n}{P2}\PY{p}{,} \PY{n}{P6}\PY{p}{)}\PY{o}{*}\PY{n}{scalar\PYZus{}product}\PY{p}{(}\PY{n}{P4}\PY{p}{,} \PY{n}{P5}\PY{p}{)}
        \PY{o}{+} \PY{n}{scalar\PYZus{}product}\PY{p}{(}\PY{n}{P4}\PY{p}{,} \PY{n}{P6}\PY{p}{)}\PY{o}{*}\PY{n}{scalar\PYZus{}product}\PY{p}{(}\PY{n}{P2}\PY{p}{,}\PY{n}{P5}\PY{p}{)}
    \PY{p}{)} \PY{o}{\PYZhy{}} \PY{n}{scalar\PYZus{}product}\PY{p}{(}\PY{n}{P2}\PY{p}{,} \PY{n}{P6}\PY{p}{)}\PY{o}{*}\PY{n}{scalar\PYZus{}product}\PY{p}{(}\PY{n}{P1}\PY{p}{,} \PY{n}{P5}\PY{p}{)}\PY{o}{*}\PY{n}{scalar\PYZus{}product}\PY{p}{(}\PY{n}{P1}\PY{p}{,} \PY{n}{P4}\PY{p}{)}
    \PY{o}{\PYZhy{}} \PY{n}{scalar\PYZus{}product}\PY{p}{(}\PY{n}{P4}\PY{p}{,} \PY{n}{P6}\PY{p}{)}\PY{o}{*}\PY{n}{scalar\PYZus{}product}\PY{p}{(}\PY{n}{P1}\PY{p}{,} \PY{n}{P5}\PY{p}{)}\PY{o}{*}\PY{n}{scalar\PYZus{}product}\PY{p}{(}\PY{n}{P1}\PY{p}{,} \PY{n}{P2}\PY{p}{)}
\PY{p}{)}

\PY{k}{assert}\PY{p}{(}\PY{n}{ftC}\PY{o}{.}\PY{n}{subs}\PY{p}{(}\PY{p}{\PYZob{}}\PY{n}{u1}\PY{p}{:}\PY{n}{U1}\PY{p}{,} \PY{n}{u2}\PY{p}{:}\PY{o}{\PYZhy{}}\PY{n}{U2}\PY{p}{\PYZcb{}}\PY{p}{)} \PY{o}{==} \PY{n}{S}\PY{p}{(}\PY{l+m+mi}{0}\PY{p}{)}\PY{p}{)}
\end{Verbatim}
\end{tcolorbox}

    Here we see that it is not possible that U1 and U2 are zero:

    \begin{tcolorbox}[breakable, size=fbox, boxrule=1pt, pad at break*=1mm,colback=cellbackground, colframe=cellborder]
\prompt{In}{incolor}{20}{\boxspacing}
\begin{Verbatim}[commandchars=\\\{\}]
\PY{k}{assert}\PY{p}{(}
    \PY{n}{S}\PY{o}{.}\PY{n}{ideal}\PY{p}{(}\PY{n}{U1}\PY{p}{,} \PY{n}{U2}\PY{p}{)}\PY{o}{.}\PY{n}{saturation}\PY{p}{(}\PY{n}{matrix}\PY{p}{(}\PY{p}{[}\PY{n}{P1}\PY{p}{,} \PY{n}{P2}\PY{p}{,} \PY{n}{P4}\PY{p}{]}\PY{p}{)}\PY{o}{.}\PY{n}{det}\PY{p}{(}\PY{p}{)}\PY{p}{)}\PY{p}{[}\PY{l+m+mi}{0}\PY{p}{]}\PY{o}{.}\PY{n}{saturation}\PY{p}{(}\PY{n}{u1}\PY{o}{*}\PY{n}{u2}\PY{o}{*}\PY{n}{v1}\PY{o}{*}\PY{n}{v2}\PY{o}{*}\PY{n}{w1}\PY{o}{*}\PY{n}{w2}\PY{p}{)}\PY{p}{[}\PY{l+m+mi}{0}\PY{p}{]} \PY{o}{==} 
    \PY{n}{S}\PY{o}{.}\PY{n}{ideal}\PY{p}{(}\PY{n}{S}\PY{o}{.}\PY{n}{one}\PY{p}{(}\PY{p}{)}\PY{p}{)}
\PY{p}{)}
\end{Verbatim}
\end{tcolorbox}

    The condition \texttt{ftC1\ =\ 0} gives that in order to have 6 points
in configuration \((C_5)\) we can choose \(P_2\) and \(P_4\) in an
arbitrary way, \(P_1 = P_2 \times P_4\), \(P_3\) on the line
\(P_1 \vee P_2\), \(P_5\) on the line \(P_1 vee P_4\). Then \(P_6\) is a
point on the line \(P_2\vee P_4\) determined by a linear equation in
\(w_1\) and \(w_2\) given by \texttt{ftC1\ =\ 0}.

    In order to determine \(P_6\), we need to find \(w_1\) and \(w_2\). We
observe that \texttt{ftC1} is linear in \(w_1\) and \(w_2\)

    \begin{tcolorbox}[breakable, size=fbox, boxrule=1pt, pad at break*=1mm,colback=cellbackground, colframe=cellborder]
\prompt{In}{incolor}{21}{\boxspacing}
\begin{Verbatim}[commandchars=\\\{\}]
\PY{k}{assert}\PY{p}{(}\PY{n}{ftC1}\PY{o}{.}\PY{n}{degree}\PY{p}{(}\PY{n}{w1}\PY{p}{)} \PY{o}{==} \PY{l+m+mi}{1}\PY{p}{)}
\PY{k}{assert}\PY{p}{(}\PY{n}{ftC1}\PY{o}{.}\PY{n}{degree}\PY{p}{(}\PY{n}{w2}\PY{p}{)} \PY{o}{==} \PY{l+m+mi}{1}\PY{p}{)}
\end{Verbatim}
\end{tcolorbox}

    and we have that the coefficient of \texttt{ftC1} w.r.t. \(w_1\) is

\(s_{13}s_{24}s_{25} - 2s_{15}s_{22}s_{34} + s_{13}s_{22}s_{45}\)

and we have that the coefficient of \texttt{ftC1} w.r.t. \(w_2\) is

\(-(s_{15}s_{24}s_{34} + s_{15}s_{23}s_{44} - s_{13}s_{25}s_{44} - s_{13}s_{24}s_{45})\)

To verify this, we define a substitution which sends \(s_{ij}\) to the
scalar prod. of \(P_i\) and \(P_j\)

    \begin{tcolorbox}[breakable, size=fbox, boxrule=1pt, pad at break*=1mm,colback=cellbackground, colframe=cellborder]
\prompt{In}{incolor}{22}{\boxspacing}
\begin{Verbatim}[commandchars=\\\{\}]
\PY{n}{sst\PYZus{}5} \PY{o}{=} \PY{p}{\PYZob{}}
    \PY{n}{s11}\PY{p}{:}\PY{n}{scalar\PYZus{}product}\PY{p}{(}\PY{n}{P1}\PY{p}{,} \PY{n}{P1}\PY{p}{)}\PY{p}{,}
    \PY{n}{s12}\PY{p}{:}\PY{n}{scalar\PYZus{}product}\PY{p}{(}\PY{n}{P1}\PY{p}{,} \PY{n}{P2}\PY{p}{)}\PY{p}{,}
    \PY{n}{s22}\PY{p}{:}\PY{n}{scalar\PYZus{}product}\PY{p}{(}\PY{n}{P2}\PY{p}{,} \PY{n}{P2}\PY{p}{)}\PY{p}{,}
    \PY{n}{s14}\PY{p}{:}\PY{n}{scalar\PYZus{}product}\PY{p}{(}\PY{n}{P1}\PY{p}{,} \PY{n}{P4}\PY{p}{)}\PY{p}{,}
    \PY{n}{s24}\PY{p}{:}\PY{n}{scalar\PYZus{}product}\PY{p}{(}\PY{n}{P2}\PY{p}{,} \PY{n}{P4}\PY{p}{)}\PY{p}{,}
    \PY{n}{s44}\PY{p}{:}\PY{n}{scalar\PYZus{}product}\PY{p}{(}\PY{n}{P4}\PY{p}{,} \PY{n}{P4}\PY{p}{)}\PY{p}{,}
    \PY{n}{s13}\PY{p}{:}\PY{n}{scalar\PYZus{}product}\PY{p}{(}\PY{n}{P1}\PY{p}{,} \PY{n}{P3}\PY{p}{)}\PY{p}{,}
    \PY{n}{s23}\PY{p}{:}\PY{n}{scalar\PYZus{}product}\PY{p}{(}\PY{n}{P2}\PY{p}{,} \PY{n}{P3}\PY{p}{)}\PY{p}{,}
    \PY{n}{s34}\PY{p}{:}\PY{n}{scalar\PYZus{}product}\PY{p}{(}\PY{n}{P3}\PY{p}{,} \PY{n}{P4}\PY{p}{)}\PY{p}{,}
    \PY{n}{s33}\PY{p}{:}\PY{n}{scalar\PYZus{}product}\PY{p}{(}\PY{n}{P3}\PY{p}{,} \PY{n}{P3}\PY{p}{)}\PY{p}{,}
    \PY{n}{s45}\PY{p}{:}\PY{n}{scalar\PYZus{}product}\PY{p}{(}\PY{n}{P4}\PY{p}{,} \PY{n}{P5}\PY{p}{)}\PY{p}{,}
    \PY{n}{s15}\PY{p}{:}\PY{n}{scalar\PYZus{}product}\PY{p}{(}\PY{n}{P1}\PY{p}{,} \PY{n}{P5}\PY{p}{)}\PY{p}{,}
    \PY{n}{s25}\PY{p}{:}\PY{n}{scalar\PYZus{}product}\PY{p}{(}\PY{n}{P2}\PY{p}{,} \PY{n}{P5}\PY{p}{)}
\PY{p}{\PYZcb{}}

\PY{c+c1}{\PYZsh{}\PYZsh{} Then we have:}
\PY{k}{assert}\PY{p}{(}
    \PY{n}{ftC1}\PY{o}{.}\PY{n}{coefficient}\PY{p}{(}\PY{n}{w1}\PY{p}{)} \PY{o}{==} 
    \PY{p}{(}\PY{n}{s13}\PY{o}{*}\PY{n}{s24}\PY{o}{*}\PY{n}{s25} \PY{o}{\PYZhy{}} \PY{l+m+mi}{2}\PY{o}{*}\PY{n}{s15}\PY{o}{*}\PY{n}{s22}\PY{o}{*}\PY{n}{s34} \PY{o}{+} \PY{n}{s13}\PY{o}{*}\PY{n}{s22}\PY{o}{*}\PY{n}{s45}\PY{p}{)}\PY{o}{.}\PY{n}{subs}\PY{p}{(}\PY{n}{sst\PYZus{}5}\PY{p}{)}
\PY{p}{)}

\PY{k}{assert}\PY{p}{(}
    \PY{n}{ftC1}\PY{o}{.}\PY{n}{coefficient}\PY{p}{(}\PY{n}{w2}\PY{p}{)} \PY{o}{==}
    \PY{o}{\PYZhy{}}\PY{p}{(}\PY{n}{s15}\PY{o}{*}\PY{n}{s24}\PY{o}{*}\PY{n}{s34} \PY{o}{+} \PY{n}{s15}\PY{o}{*}\PY{n}{s23}\PY{o}{*}\PY{n}{s44} \PY{o}{\PYZhy{}} \PY{n}{s13}\PY{o}{*}\PY{n}{s25}\PY{o}{*}\PY{n}{s44} \PY{o}{\PYZhy{}} \PY{n}{s13}\PY{o}{*}\PY{n}{s24}\PY{o}{*}\PY{n}{s45}\PY{p}{)}\PY{o}{.}\PY{n}{subs}\PY{p}{(}\PY{n}{sst\PYZus{}5}\PY{p}{)}
\PY{p}{)}
\end{Verbatim}
\end{tcolorbox}

    Hence \(P_6\) is:

    \begin{tcolorbox}[breakable, size=fbox, boxrule=1pt, pad at break*=1mm,colback=cellbackground, colframe=cellborder]
\prompt{In}{incolor}{ }{\boxspacing}
\begin{Verbatim}[commandchars=\\\{\}]
\PY{n}{P6} \PY{o}{=} \PY{p}{(}
    \PY{p}{(}\PY{n}{s15}\PY{o}{*}\PY{n}{s24}\PY{o}{*}\PY{n}{s34} \PY{o}{+} \PY{n}{s15}\PY{o}{*}\PY{n}{s23}\PY{o}{*}\PY{n}{s44} \PY{o}{\PYZhy{}} \PY{n}{s13}\PY{o}{*}\PY{n}{s25}\PY{o}{*}\PY{n}{s44} \PY{o}{\PYZhy{}} \PY{n}{s13}\PY{o}{*}\PY{n}{s24}\PY{o}{*}\PY{n}{s45}\PY{p}{)}\PY{o}{*}\PY{n}{P2}
    \PY{o}{+} \PY{p}{(}\PY{n}{s13}\PY{o}{*}\PY{n}{s24}\PY{o}{*}\PY{n}{s25} \PY{o}{\PYZhy{}} \PY{l+m+mi}{2}\PY{o}{*}\PY{n}{s15}\PY{o}{*}\PY{n}{s22}\PY{o}{*}\PY{n}{s34} \PY{o}{+} \PY{n}{s13}\PY{o}{*}\PY{n}{s22}\PY{o}{*}\PY{n}{s45}\PY{p}{)}\PY{o}{*}\PY{n}{P4}
\PY{p}{)}

\PY{n}{P6} \PY{o}{=} \PY{n}{P6}\PY{o}{.}\PY{n}{subs}\PY{p}{(}\PY{n}{sst\PYZus{}5}\PY{p}{)}
\end{Verbatim}
\end{tcolorbox}

    Here is the formula for\(P_6\): \[
P_6 = (s_{15}s_{24}s_{34} + s_{15}s_{23}s_{44} - s_{13}s_{25}s_{44} - s_{13}s_{24}s_{45})P_2
    + (s_{13}s_{24}s_{25} - 2s_{15}s_{22}s_{34} + s_{13}s_{22}s_{45})P_4
\] and this is formula (26) (probably)

    At this point we know that the matrix
\(\Phi(P_1, P_2, P_3, P_4, P_5, P_6)\) has rank 9, so the points
\(P_1, \dotsc, P_6\) are eigenpoints.

    From the definition of \(P_6\) we have that it holds: \[
s_{16} = 0
\]

    \begin{tcolorbox}[breakable, size=fbox, boxrule=1pt, pad at break*=1mm,colback=cellbackground, colframe=cellborder]
\prompt{In}{incolor}{26}{\boxspacing}
\begin{Verbatim}[commandchars=\\\{\}]
\PY{k}{assert}\PY{p}{(}\PY{n}{scalar\PYZus{}product}\PY{p}{(}\PY{n}{P1}\PY{p}{,} \PY{n}{P6}\PY{p}{)} \PY{o}{==} \PY{l+m+mi}{0}\PY{p}{)}
\end{Verbatim}
\end{tcolorbox}

    \hypertarget{the-point-p_6-is-always-defined.}{%
\subsection{\texorpdfstring{The point \(P_6\) is always
defined.}{The point P\_6 is always defined.}}\label{the-point-p_6-is-always-defined.}}

    It is not possible that the three cordinates of \(P_6\) are all zero:

    \begin{tcolorbox}[breakable, size=fbox, boxrule=1pt, pad at break*=1mm,colback=cellbackground, colframe=cellborder]
\prompt{In}{incolor}{24}{\boxspacing}
\begin{Verbatim}[commandchars=\\\{\}]
\PY{k}{assert}\PY{p}{(}\PY{n}{S}\PY{o}{.}\PY{n}{ideal}\PY{p}{(}\PY{n+nb}{list}\PY{p}{(}\PY{n}{P6}\PY{p}{)}\PY{p}{)}\PY{o}{.}\PY{n}{saturation}\PY{p}{(}\PY{n}{u1}\PY{o}{*}\PY{n}{u2}\PY{o}{*}\PY{n}{v1}\PY{o}{*}\PY{n}{v2}\PY{p}{)}\PY{p}{[}\PY{l+m+mi}{0}\PY{p}{]}\PY{o}{.}\PY{n}{saturation}\PY{p}{(}\PY{n}{scalar\PYZus{}product}\PY{p}{(}\PY{n}{P1}\PY{p}{,} \PY{n}{P1}\PY{p}{)}\PY{p}{)}\PY{p}{[}\PY{l+m+mi}{0}\PY{p}{]} \PY{o}{==} \PY{n}{S}\PY{o}{.}\PY{n}{ideal}\PY{p}{(}\PY{l+m+mi}{1}\PY{p}{)}\PY{p}{)}
\end{Verbatim}
\end{tcolorbox}

    \hypertarget{construction-of-the-point-p_7}{%
\subsection{\texorpdfstring{Construction of the point
\(P_7\)}{Construction of the point P\_7}}\label{construction-of-the-point-p_7}}

    In order to find \(P_7\) we observe that it is symmetric to \(P_3\).
Hence, if in the above formula, where we define \(U_1\) and \(U_2\) we
exchange \(P_2\) with \(P_6\) we get \(P_7\):

    \begin{tcolorbox}[breakable, size=fbox, boxrule=1pt, pad at break*=1mm,colback=cellbackground, colframe=cellborder]
\prompt{In}{incolor}{ }{\boxspacing}
\begin{Verbatim}[commandchars=\\\{\}]
\PY{n}{L1} \PY{o}{=} \PY{p}{(}
    \PY{n}{scalar\PYZus{}product}\PY{p}{(}\PY{n}{P1}\PY{p}{,} \PY{n}{P6}\PY{p}{)}\PY{o}{*}\PY{p}{(}
        \PY{n}{scalar\PYZus{}product}\PY{p}{(}\PY{n}{P6}\PY{p}{,} \PY{n}{P2}\PY{p}{)}\PY{o}{*}\PY{n}{scalar\PYZus{}product}\PY{p}{(}\PY{n}{P4}\PY{p}{,} \PY{n}{P5}\PY{p}{)}
        \PY{o}{+} \PY{n}{scalar\PYZus{}product}\PY{p}{(}\PY{n}{P4}\PY{p}{,} \PY{n}{P2}\PY{p}{)}\PY{o}{*}\PY{n}{scalar\PYZus{}product}\PY{p}{(}\PY{n}{P6}\PY{p}{,}\PY{n}{P5}\PY{p}{)}
    \PY{p}{)} \PY{o}{\PYZhy{}} \PY{n}{scalar\PYZus{}product}\PY{p}{(}\PY{n}{P6}\PY{p}{,} \PY{n}{P2}\PY{p}{)}\PY{o}{*}\PY{n}{scalar\PYZus{}product}\PY{p}{(}\PY{n}{P1}\PY{p}{,} \PY{n}{P5}\PY{p}{)}\PY{o}{*}\PY{n}{scalar\PYZus{}product}\PY{p}{(}\PY{n}{P6}\PY{p}{,} \PY{n}{P4}\PY{p}{)}
    \PY{o}{\PYZhy{}} \PY{n}{scalar\PYZus{}product}\PY{p}{(}\PY{n}{P4}\PY{p}{,} \PY{n}{P2}\PY{p}{)}\PY{o}{*}\PY{n}{scalar\PYZus{}product}\PY{p}{(}\PY{n}{P1}\PY{p}{,} \PY{n}{P5}\PY{p}{)}\PY{o}{*}\PY{n}{scalar\PYZus{}product}\PY{p}{(}\PY{n}{P6}\PY{p}{,} \PY{n}{P6}\PY{p}{)}
\PY{p}{)}

\PY{n}{L2} \PY{o}{=} \PY{p}{(}
    \PY{n}{scalar\PYZus{}product}\PY{p}{(}\PY{n}{P1}\PY{p}{,} \PY{n}{P1}\PY{p}{)}\PY{o}{*}\PY{p}{(}
        \PY{n}{scalar\PYZus{}product}\PY{p}{(}\PY{n}{P6}\PY{p}{,} \PY{n}{P2}\PY{p}{)}\PY{o}{*}\PY{n}{scalar\PYZus{}product}\PY{p}{(}\PY{n}{P4}\PY{p}{,} \PY{n}{P5}\PY{p}{)}
        \PY{o}{+} \PY{n}{scalar\PYZus{}product}\PY{p}{(}\PY{n}{P4}\PY{p}{,} \PY{n}{P2}\PY{p}{)}\PY{o}{*}\PY{n}{scalar\PYZus{}product}\PY{p}{(}\PY{n}{P6}\PY{p}{,}\PY{n}{P5}\PY{p}{)}
    \PY{p}{)} \PY{o}{\PYZhy{}} \PY{n}{scalar\PYZus{}product}\PY{p}{(}\PY{n}{P6}\PY{p}{,} \PY{n}{P2}\PY{p}{)}\PY{o}{*}\PY{n}{scalar\PYZus{}product}\PY{p}{(}\PY{n}{P1}\PY{p}{,} \PY{n}{P5}\PY{p}{)}\PY{o}{*}\PY{n}{scalar\PYZus{}product}\PY{p}{(}\PY{n}{P1}\PY{p}{,} \PY{n}{P4}\PY{p}{)}
    \PY{o}{\PYZhy{}} \PY{n}{scalar\PYZus{}product}\PY{p}{(}\PY{n}{P4}\PY{p}{,} \PY{n}{P2}\PY{p}{)}\PY{o}{*}\PY{n}{scalar\PYZus{}product}\PY{p}{(}\PY{n}{P1}\PY{p}{,} \PY{n}{P5}\PY{p}{)}\PY{o}{*}\PY{n}{scalar\PYZus{}product}\PY{p}{(}\PY{n}{P1}\PY{p}{,} \PY{n}{P6}\PY{p}{)}
\PY{p}{)}
\end{Verbatim}
\end{tcolorbox}

    But, since we know that \(s_{1,6} = 0\) and \(s_{14} = 0\), we redefine
\(L_1\), \(L_2\) and then we define \(P_7\)

    \begin{tcolorbox}[breakable, size=fbox, boxrule=1pt, pad at break*=1mm,colback=cellbackground, colframe=cellborder]
\prompt{In}{incolor}{29}{\boxspacing}
\begin{Verbatim}[commandchars=\\\{\}]
\PY{n}{L1} \PY{o}{=} \PY{p}{(}
    \PY{o}{\PYZhy{}} \PY{n}{scalar\PYZus{}product}\PY{p}{(}\PY{n}{P6}\PY{p}{,} \PY{n}{P2}\PY{p}{)}\PY{o}{*}\PY{n}{scalar\PYZus{}product}\PY{p}{(}\PY{n}{P1}\PY{p}{,} \PY{n}{P5}\PY{p}{)}\PY{o}{*}\PY{n}{scalar\PYZus{}product}\PY{p}{(}\PY{n}{P6}\PY{p}{,} \PY{n}{P4}\PY{p}{)}
    \PY{o}{\PYZhy{}} \PY{n}{scalar\PYZus{}product}\PY{p}{(}\PY{n}{P4}\PY{p}{,} \PY{n}{P2}\PY{p}{)}\PY{o}{*}\PY{n}{scalar\PYZus{}product}\PY{p}{(}\PY{n}{P1}\PY{p}{,} \PY{n}{P5}\PY{p}{)}\PY{o}{*}\PY{n}{scalar\PYZus{}product}\PY{p}{(}\PY{n}{P6}\PY{p}{,} \PY{n}{P6}\PY{p}{)}
\PY{p}{)}

\PY{n}{L2} \PY{o}{=} \PY{p}{(}
    \PY{n}{scalar\PYZus{}product}\PY{p}{(}\PY{n}{P1}\PY{p}{,} \PY{n}{P1}\PY{p}{)}\PY{o}{*}\PY{p}{(}
        \PY{n}{scalar\PYZus{}product}\PY{p}{(}\PY{n}{P6}\PY{p}{,} \PY{n}{P2}\PY{p}{)}\PY{o}{*}\PY{n}{scalar\PYZus{}product}\PY{p}{(}\PY{n}{P4}\PY{p}{,} \PY{n}{P5}\PY{p}{)}
        \PY{o}{+} \PY{n}{scalar\PYZus{}product}\PY{p}{(}\PY{n}{P4}\PY{p}{,} \PY{n}{P2}\PY{p}{)}\PY{o}{*}\PY{n}{scalar\PYZus{}product}\PY{p}{(}\PY{n}{P6}\PY{p}{,}\PY{n}{P5}\PY{p}{)}
    \PY{p}{)}
\PY{p}{)}
\end{Verbatim}
\end{tcolorbox}

    The point \(P_7\) is therefore defined by the formula:

    \[
P_7 = (s_{26}s_{15}s_{46}+s_{24}s_{15}s_{66})P_1 + s_{11}(s_{26}s_{45}+s_{24}s_{56})P_6
\]

    This is formula (27) (probably).

    \begin{tcolorbox}[breakable, size=fbox, boxrule=1pt, pad at break*=1mm,colback=cellbackground, colframe=cellborder]
\prompt{In}{incolor}{35}{\boxspacing}
\begin{Verbatim}[commandchars=\\\{\}]
\PY{n}{P7} \PY{o}{=} \PY{o}{\PYZhy{}}\PY{n}{L1}\PY{o}{*}\PY{n}{P1} \PY{o}{+} \PY{n}{L2}\PY{o}{*}\PY{n}{P6}
\end{Verbatim}
\end{tcolorbox}

    \(P_7\) can be simplified, since the components have a common factor
which is never zero. Hence we redefine \(P_7\)

    \begin{tcolorbox}[breakable, size=fbox, boxrule=1pt, pad at break*=1mm,colback=cellbackground, colframe=cellborder]
\prompt{In}{incolor}{36}{\boxspacing}
\begin{Verbatim}[commandchars=\\\{\}]
\PY{k}{assert}\PY{p}{(}\PY{n}{gcd}\PY{p}{(}\PY{n+nb}{list}\PY{p}{(}\PY{n}{P7}\PY{p}{)}\PY{p}{)} \PY{o}{==} \PY{n}{scalar\PYZus{}product}\PY{p}{(}\PY{n}{P1}\PY{p}{,} \PY{n}{P1}\PY{p}{)}\PY{o}{\PYZca{}}\PY{l+m+mi}{4}\PY{o}{*}\PY{n}{v1}\PY{p}{)}
\PY{c+c1}{\PYZsh{}\PYZsh{} Since we know that s11 is not 0, we redefine P7:}
\PY{n}{P7} \PY{o}{=} \PY{n}{vector}\PY{p}{(}\PY{n}{S}\PY{p}{,} \PY{p}{[}\PY{n}{pp}\PY{o}{.}\PY{n}{quo\PYZus{}rem}\PY{p}{(}\PY{n}{gcd}\PY{p}{(}\PY{n+nb}{list}\PY{p}{(}\PY{n}{P7}\PY{p}{)}\PY{p}{)}\PY{p}{)}\PY{p}{[}\PY{l+m+mi}{0}\PY{p}{]} \PY{k}{for} \PY{n}{pp} \PY{o+ow}{in} \PY{n}{P7}\PY{p}{]}\PY{p}{)}
\end{Verbatim}
\end{tcolorbox}

    Partial conclusion: we have that \(P_7\) is given by the formula (30).

    \hypertarget{p_7-is-an-eigenpoint-of-the-cubic-obtained-from-phip_1-dotsc-p_5}{%
\subsection{\texorpdfstring{\(P_7\) is an eigenpoint of the cubic
obtained from
\(\Phi(P_1, \dotsc, P_5)\)}{P\_7 is an eigenpoint of the cubic obtained from \textbackslash Phi(P\_1, \textbackslash dotsc, P\_5)}}\label{p_7-is-an-eigenpoint-of-the-cubic-obtained-from-phip_1-dotsc-p_5}}

    \(P_7\) is an eigenpoints. This can be seen as follows: the rank of
\(\Phi(P_1, \dots, P_5)\) is 9, so \(\Lambda(\Phi(P_1, \dotsc, P_5))\)
is a unique cubic \(C\), this cubic also has \(P_6\) as an eigenpoint,
so the rank of \(\Phi(P_1, P_2, P_6, P_4, P_5)\) is 9, so from
\(\Lambda(P_1, P_2, P_6, P_4, P_5))\) we get a unique cubic which is
again \(C\) and this cubic has the eigenpoint \(P_7\). In this proof
there are however some points that require attention (it could happen
that the rank is not 9 but 8 \dots). In order to avoid the study of the
exceptions, we give a direct proof that \(P_7\) is an eigenpoint.

    To directly prove that \(P_7\) is an eigenpoint, we need some further
computations.

In order to simplify the computations, we redefine the point with the
condition that \(P_2 = (1:0:0)\) or \(P_2 = (1: i: 0)\).

    \hypertarget{a-lemma}{%
\subsubsection{A lemma}\label{a-lemma}}

    First of all, we show that \(P_2\) cannot be the point \((1:i:0)\). We
define \(P_2=(1:i:0)\) and, since we know that \(s_{12} = 0\), we define
\(P_1\) generic, orthogonal to \(P_2\)

    \begin{tcolorbox}[breakable, size=fbox, boxrule=1pt, pad at break*=1mm,colback=cellbackground, colframe=cellborder]
\prompt{In}{incolor}{37}{\boxspacing}
\begin{Verbatim}[commandchars=\\\{\}]
\PY{n}{P2} \PY{o}{=} \PY{n}{vector}\PY{p}{(}\PY{n}{S}\PY{p}{,} \PY{p}{(}\PY{l+m+mi}{1}\PY{p}{,} \PY{n}{ii}\PY{p}{,} \PY{l+m+mi}{0}\PY{p}{)}\PY{p}{)}
\PY{n}{P1} \PY{o}{=} \PY{n}{vector}\PY{p}{(}\PY{n}{S}\PY{p}{,} \PY{p}{(}\PY{o}{\PYZhy{}}\PY{n}{ii}\PY{o}{*}\PY{n}{B1}\PY{p}{,} \PY{n}{B1}\PY{p}{,} \PY{n}{C1}\PY{p}{)}\PY{p}{)}
\PY{k}{assert}\PY{p}{(}\PY{n}{scalar\PYZus{}product}\PY{p}{(}\PY{n}{P1}\PY{p}{,} \PY{n}{P2}\PY{p}{)} \PY{o}{==} \PY{l+m+mi}{0}\PY{p}{)}
\end{Verbatim}
\end{tcolorbox}

    In the next computation, we verify that \(P_1 \vee P_2\) is tangent to
Ciso in \(P_2\), since the intersection of \(P_1 \vee P_2\) with Ciso is
the double point given by the ideal \((x+iy,z^2)\).

    \begin{tcolorbox}[breakable, size=fbox, boxrule=1pt, pad at break*=1mm,colback=cellbackground, colframe=cellborder]
\prompt{In}{incolor}{38}{\boxspacing}
\begin{Verbatim}[commandchars=\\\{\}]
\PY{k}{assert}\PY{p}{(}
    \PY{n}{S}\PY{o}{.}\PY{n}{ideal}\PY{p}{(}\PY{n}{matrix}\PY{p}{(}\PY{p}{[}\PY{n}{P1}\PY{p}{,} \PY{n}{P2}\PY{p}{,} \PY{p}{(}\PY{n}{x}\PY{p}{,} \PY{n}{y}\PY{p}{,} \PY{n}{z}\PY{p}{)}\PY{p}{]}\PY{p}{)}\PY{o}{.}\PY{n}{det}\PY{p}{(}\PY{p}{)}\PY{p}{,} \PY{n}{Ciso}\PY{p}{)}\PY{o}{.}\PY{n}{saturation}\PY{p}{(}\PY{n}{S}\PY{o}{.}\PY{n}{ideal}\PY{p}{(}\PY{n}{matrix}\PY{p}{(}\PY{p}{[}\PY{n}{P1}\PY{p}{,} \PY{n}{P2}\PY{p}{]}\PY{p}{)}\PY{o}{.}\PY{n}{minors}\PY{p}{(}\PY{l+m+mi}{2}\PY{p}{)}\PY{p}{)}\PY{p}{)}\PY{p}{[}\PY{l+m+mi}{0}\PY{p}{]} \PY{o}{==} 
    \PY{n}{S}\PY{o}{.}\PY{n}{ideal}\PY{p}{(}\PY{n}{x}\PY{o}{+}\PY{n}{ii}\PY{o}{*}\PY{n}{y}\PY{p}{,} \PY{n}{z}\PY{o}{\PYZca{}}\PY{l+m+mi}{2}\PY{p}{)}
\PY{p}{)}
\end{Verbatim}
\end{tcolorbox}

    Hence the martix \(\Phi(P_1, \dots, P_5)\) has rank 8, but this
condition is excluded by our hypothesis. So the only possibility is that
\(P_2\) can be chosen as the point \((1:0:0)\).

    \hypertarget{we-redefine-the-points-p_1-dots-p_6}{%
\subsubsection{\texorpdfstring{We redefine the points
\(P_1, \dots, P_6\):}{We redefine the points P\_1, \textbackslash dots, P\_6:}}\label{we-redefine-the-points-p_1-dots-p_6}}

We redefine \(P_1, \dots, P_5\) with the condition that
\(P_2 = (1:i:0)\)and we redefine \(P_6\) from the above formula,
evaluated on the new points:

    \begin{tcolorbox}[breakable, size=fbox, boxrule=1pt, pad at break*=1mm,colback=cellbackground, colframe=cellborder]
\prompt{In}{incolor}{40}{\boxspacing}
\begin{Verbatim}[commandchars=\\\{\}]
\PY{n}{P2} \PY{o}{=} \PY{n}{vector}\PY{p}{(}\PY{n}{S}\PY{p}{,} \PY{p}{(}\PY{l+m+mi}{1}\PY{p}{,} \PY{l+m+mi}{0}\PY{p}{,} \PY{l+m+mi}{0}\PY{p}{)}\PY{p}{)}
\PY{n}{P4} \PY{o}{=} \PY{n}{vector}\PY{p}{(}\PY{n}{S}\PY{p}{,} \PY{p}{(}\PY{n}{A4}\PY{p}{,} \PY{n}{B4}\PY{p}{,} \PY{n}{C4}\PY{p}{)}\PY{p}{)}
\PY{n}{P1} \PY{o}{=} \PY{n}{wedge\PYZus{}product}\PY{p}{(}\PY{n}{P2}\PY{p}{,} \PY{n}{P4}\PY{p}{)}
\PY{n}{P3} \PY{o}{=} \PY{n}{u1}\PY{o}{*}\PY{n}{P1}\PY{o}{+}\PY{n}{u2}\PY{o}{*}\PY{n}{P2}
\PY{n}{P5} \PY{o}{=} \PY{n}{v1}\PY{o}{*}\PY{n}{P1}\PY{o}{+}\PY{n}{v2}\PY{o}{*}\PY{n}{P4}

\PY{n}{P6} \PY{o}{=} \PY{p}{(}
    \PY{p}{(}\PY{n}{s15}\PY{o}{*}\PY{n}{s24}\PY{o}{*}\PY{n}{s34} \PY{o}{+} \PY{n}{s15}\PY{o}{*}\PY{n}{s23}\PY{o}{*}\PY{n}{s44} \PY{o}{\PYZhy{}} \PY{n}{s13}\PY{o}{*}\PY{n}{s25}\PY{o}{*}\PY{n}{s44} \PY{o}{\PYZhy{}} \PY{n}{s13}\PY{o}{*}\PY{n}{s24}\PY{o}{*}\PY{n}{s45}\PY{p}{)}\PY{o}{*}\PY{n}{P2}
    \PY{o}{+} \PY{p}{(}\PY{n}{s13}\PY{o}{*}\PY{n}{s24}\PY{o}{*}\PY{n}{s25} \PY{o}{\PYZhy{}} \PY{l+m+mi}{2}\PY{o}{*}\PY{n}{s15}\PY{o}{*}\PY{n}{s22}\PY{o}{*}\PY{n}{s34} \PY{o}{+} \PY{n}{s13}\PY{o}{*}\PY{n}{s22}\PY{o}{*}\PY{n}{s45}\PY{p}{)}\PY{o}{*}\PY{n}{P4}
\PY{p}{)}

\PY{n}{sst\PYZus{}5b} \PY{o}{=} \PY{p}{\PYZob{}}
    \PY{n}{s11}\PY{p}{:}\PY{n}{scalar\PYZus{}product}\PY{p}{(}\PY{n}{P1}\PY{p}{,} \PY{n}{P1}\PY{p}{)}\PY{p}{,}
    \PY{n}{s12}\PY{p}{:}\PY{n}{scalar\PYZus{}product}\PY{p}{(}\PY{n}{P1}\PY{p}{,} \PY{n}{P2}\PY{p}{)}\PY{p}{,}
    \PY{n}{s22}\PY{p}{:}\PY{n}{scalar\PYZus{}product}\PY{p}{(}\PY{n}{P2}\PY{p}{,} \PY{n}{P2}\PY{p}{)}\PY{p}{,}
    \PY{n}{s14}\PY{p}{:}\PY{n}{scalar\PYZus{}product}\PY{p}{(}\PY{n}{P1}\PY{p}{,} \PY{n}{P4}\PY{p}{)}\PY{p}{,}
    \PY{n}{s24}\PY{p}{:}\PY{n}{scalar\PYZus{}product}\PY{p}{(}\PY{n}{P2}\PY{p}{,} \PY{n}{P4}\PY{p}{)}\PY{p}{,}
    \PY{n}{s44}\PY{p}{:}\PY{n}{scalar\PYZus{}product}\PY{p}{(}\PY{n}{P4}\PY{p}{,} \PY{n}{P4}\PY{p}{)}\PY{p}{,}
    \PY{n}{s13}\PY{p}{:}\PY{n}{scalar\PYZus{}product}\PY{p}{(}\PY{n}{P1}\PY{p}{,} \PY{n}{P3}\PY{p}{)}\PY{p}{,}
    \PY{n}{s23}\PY{p}{:}\PY{n}{scalar\PYZus{}product}\PY{p}{(}\PY{n}{P2}\PY{p}{,} \PY{n}{P3}\PY{p}{)}\PY{p}{,}
    \PY{n}{s34}\PY{p}{:}\PY{n}{scalar\PYZus{}product}\PY{p}{(}\PY{n}{P3}\PY{p}{,} \PY{n}{P4}\PY{p}{)}\PY{p}{,}
    \PY{n}{s33}\PY{p}{:}\PY{n}{scalar\PYZus{}product}\PY{p}{(}\PY{n}{P3}\PY{p}{,} \PY{n}{P3}\PY{p}{)}\PY{p}{,}
    \PY{n}{s45}\PY{p}{:}\PY{n}{scalar\PYZus{}product}\PY{p}{(}\PY{n}{P4}\PY{p}{,} \PY{n}{P5}\PY{p}{)}\PY{p}{,}
    \PY{n}{s15}\PY{p}{:}\PY{n}{scalar\PYZus{}product}\PY{p}{(}\PY{n}{P1}\PY{p}{,} \PY{n}{P5}\PY{p}{)}\PY{p}{,}
    \PY{n}{s25}\PY{p}{:}\PY{n}{scalar\PYZus{}product}\PY{p}{(}\PY{n}{P2}\PY{p}{,} \PY{n}{P5}\PY{p}{)}
\PY{p}{\PYZcb{}}

\PY{n}{P6} \PY{o}{=} \PY{n}{P6}\PY{o}{.}\PY{n}{subs}\PY{p}{(}\PY{n}{sst\PYZus{}5b}\PY{p}{)}
\end{Verbatim}
\end{tcolorbox}

    We define the condition matrix of \(P_1, \dotsc, P_6\) and we verify
that it has rank 9. First we extract a suitable submatrix from it which
has rank \(9\) and does not have rows obtained from \(\phi(P_6)\):

    \begin{tcolorbox}[breakable, size=fbox, boxrule=1pt, pad at break*=1mm,colback=cellbackground, colframe=cellborder]
\prompt{In}{incolor}{41}{\boxspacing}
\begin{Verbatim}[commandchars=\\\{\}]
\PY{n}{M} \PY{o}{=} \PY{n}{condition\PYZus{}matrix}\PY{p}{(}\PY{p}{[}\PY{n}{P1}\PY{p}{,} \PY{n}{P2}\PY{p}{,} \PY{n}{P3}\PY{p}{,} \PY{n}{P4}\PY{p}{,} \PY{n}{P5}\PY{p}{,} \PY{n}{P6}\PY{p}{]}\PY{p}{,} \PY{n}{S}\PY{p}{,} \PY{n}{standard}\PY{o}{=}\PY{l+s+s2}{\PYZdq{}}\PY{l+s+s2}{all}\PY{l+s+s2}{\PYZdq{}}\PY{p}{)}
\PY{c+c1}{\PYZsh{}\PYZsh{} The nine rows 0, 2, 3, 4, 6, 7, 9, 10, 12 of M are linearly independent:}
\PY{n}{M9} \PY{o}{=} \PY{n}{M}\PY{o}{.}\PY{n}{matrix\PYZus{}from\PYZus{}rows}\PY{p}{(}\PY{p}{[}\PY{l+m+mi}{0}\PY{p}{,} \PY{l+m+mi}{2}\PY{p}{,} \PY{l+m+mi}{3}\PY{p}{,} \PY{l+m+mi}{4}\PY{p}{,} \PY{l+m+mi}{6}\PY{p}{,} \PY{l+m+mi}{7}\PY{p}{,} \PY{l+m+mi}{9}\PY{p}{,} \PY{l+m+mi}{10}\PY{p}{,} \PY{l+m+mi}{12}\PY{p}{]}\PY{p}{)}
\PY{k}{assert}\PY{p}{(}\PY{n}{M9}\PY{o}{.}\PY{n}{rank}\PY{p}{(}\PY{p}{)} \PY{o}{==} \PY{l+m+mi}{9}\PY{p}{)}
\end{Verbatim}
\end{tcolorbox}

    Now we add to \(M_9\) the three rows \(\Phi(P_6)\), one by one and we
verify that they are lin. dep. from the 9 rows of \(M_9\) (3 seconds of
computations):

    \begin{tcolorbox}[breakable, size=fbox, boxrule=1pt, pad at break*=1mm,colback=cellbackground, colframe=cellborder]
\prompt{In}{incolor}{42}{\boxspacing}
\begin{Verbatim}[commandchars=\\\{\}]
\PY{n}{dtA1} \PY{o}{=} \PY{n}{M}\PY{o}{.}\PY{n}{matrix\PYZus{}from\PYZus{}rows}\PY{p}{(}\PY{p}{[}\PY{l+m+mi}{0}\PY{p}{,} \PY{l+m+mi}{2}\PY{p}{,} \PY{l+m+mi}{3}\PY{p}{,} \PY{l+m+mi}{4}\PY{p}{,} \PY{l+m+mi}{6}\PY{p}{,} \PY{l+m+mi}{7}\PY{p}{,} \PY{l+m+mi}{9}\PY{p}{,} \PY{l+m+mi}{10}\PY{p}{,} \PY{l+m+mi}{12}\PY{p}{,} \PY{l+m+mi}{15}\PY{p}{]}\PY{p}{)}\PY{o}{.}\PY{n}{det}\PY{p}{(}\PY{p}{)}
\PY{n}{dtA2} \PY{o}{=} \PY{n}{M}\PY{o}{.}\PY{n}{matrix\PYZus{}from\PYZus{}rows}\PY{p}{(}\PY{p}{[}\PY{l+m+mi}{0}\PY{p}{,} \PY{l+m+mi}{2}\PY{p}{,} \PY{l+m+mi}{3}\PY{p}{,} \PY{l+m+mi}{4}\PY{p}{,} \PY{l+m+mi}{6}\PY{p}{,} \PY{l+m+mi}{7}\PY{p}{,} \PY{l+m+mi}{9}\PY{p}{,} \PY{l+m+mi}{10}\PY{p}{,} \PY{l+m+mi}{12}\PY{p}{,} \PY{l+m+mi}{16}\PY{p}{]}\PY{p}{)}\PY{o}{.}\PY{n}{det}\PY{p}{(}\PY{p}{)}
\PY{n}{dtA3} \PY{o}{=} \PY{n}{M}\PY{o}{.}\PY{n}{matrix\PYZus{}from\PYZus{}rows}\PY{p}{(}\PY{p}{[}\PY{l+m+mi}{0}\PY{p}{,} \PY{l+m+mi}{2}\PY{p}{,} \PY{l+m+mi}{3}\PY{p}{,} \PY{l+m+mi}{4}\PY{p}{,} \PY{l+m+mi}{6}\PY{p}{,} \PY{l+m+mi}{7}\PY{p}{,} \PY{l+m+mi}{9}\PY{p}{,} \PY{l+m+mi}{10}\PY{p}{,} \PY{l+m+mi}{12}\PY{p}{,} \PY{l+m+mi}{17}\PY{p}{]}\PY{p}{)}\PY{o}{.}\PY{n}{det}\PY{p}{(}\PY{p}{)}

\PY{k}{assert}\PY{p}{(}\PY{p}{(}\PY{n}{dtA1}\PY{p}{,} \PY{n}{dtA2}\PY{p}{,} \PY{n}{dtA2}\PY{p}{)} \PY{o}{==} \PY{p}{(}\PY{l+m+mi}{0}\PY{p}{,} \PY{l+m+mi}{0}\PY{p}{,} \PY{l+m+mi}{0}\PY{p}{)}\PY{p}{)}
\end{Verbatim}
\end{tcolorbox}

    \hypertarget{we-define-p_7-in-this-case}{%
\subsubsection{\texorpdfstring{We define \(P_7\) in this
case}{We define P\_7 in this case}}\label{we-define-p_7-in-this-case}}

We take the above cofmula for \(P_7\) and we evaluate it on the points
here defined:

    \begin{tcolorbox}[breakable, size=fbox, boxrule=1pt, pad at break*=1mm,colback=cellbackground, colframe=cellborder]
\prompt{In}{incolor}{46}{\boxspacing}
\begin{Verbatim}[commandchars=\\\{\}]
\PY{c+c1}{\PYZsh{} P\PYZus{}7 = (s\PYZus{}\PYZob{}26\PYZcb{}s\PYZus{}\PYZob{}15\PYZcb{}s\PYZus{}\PYZob{}46\PYZcb{}+s\PYZus{}\PYZob{}24\PYZcb{}s\PYZus{}\PYZob{}15\PYZcb{}s\PYZus{}\PYZob{}66\PYZcb{})P\PYZus{}1 + s\PYZus{}\PYZob{}11\PYZcb{}(s\PYZus{}\PYZob{}26\PYZcb{}s\PYZus{}\PYZob{}45\PYZcb{}+s\PYZus{}\PYZob{}24\PYZcb{}s\PYZus{}\PYZob{}56\PYZcb{})P\PYZus{}6}
\end{Verbatim}
\end{tcolorbox}

    \begin{tcolorbox}[breakable, size=fbox, boxrule=1pt, pad at break*=1mm,colback=cellbackground, colframe=cellborder]
\prompt{In}{incolor}{ }{\boxspacing}
\begin{Verbatim}[commandchars=\\\{\}]
\PY{n}{P7} \PY{o}{=} \PY{p}{(}
    \PY{p}{(}
        \PY{n}{scalar\PYZus{}product}\PY{p}{(}\PY{n}{P6}\PY{p}{,} \PY{n}{P2}\PY{p}{)}\PY{o}{*}\PY{n}{scalar\PYZus{}product}\PY{p}{(}\PY{n}{P1}\PY{p}{,} \PY{n}{P5}\PY{p}{)}\PY{o}{*}\PY{n}{scalar\PYZus{}product}\PY{p}{(}\PY{n}{P6}\PY{p}{,} \PY{n}{P4}\PY{p}{)}
        \PY{o}{+}\PY{n}{scalar\PYZus{}product}\PY{p}{(}\PY{n}{P4}\PY{p}{,} \PY{n}{P2}\PY{p}{)}\PY{o}{*}\PY{n}{scalar\PYZus{}product}\PY{p}{(}\PY{n}{P1}\PY{p}{,} \PY{n}{P5}\PY{p}{)}\PY{o}{*}\PY{n}{scalar\PYZus{}product}\PY{p}{(}\PY{n}{P6}\PY{p}{,} \PY{n}{P6}\PY{p}{)}
    \PY{p}{)}\PY{o}{*}\PY{n}{P1} 
    \PY{o}{+} \PY{n}{scalar\PYZus{}product}\PY{p}{(}\PY{n}{P1}\PY{p}{,} \PY{n}{P1}\PY{p}{)}\PY{o}{*}\PY{p}{(}\PY{n}{scalar\PYZus{}product}\PY{p}{(}\PY{n}{P6}\PY{p}{,} \PY{n}{P2}\PY{p}{)}\PY{o}{*}\PY{n}{scalar\PYZus{}product}\PY{p}{(}\PY{n}{P4}\PY{p}{,} \PY{n}{P5}\PY{p}{)}
    \PY{o}{+} \PY{n}{scalar\PYZus{}product}\PY{p}{(}\PY{n}{P4}\PY{p}{,} \PY{n}{P2}\PY{p}{)}\PY{o}{*}\PY{n}{scalar\PYZus{}product}\PY{p}{(}\PY{n}{P6}\PY{p}{,}\PY{n}{P5}\PY{p}{)}\PY{p}{)}\PY{o}{*}\PY{n}{P6}
\PY{p}{)}
\end{Verbatim}
\end{tcolorbox}

    \begin{tcolorbox}[breakable, size=fbox, boxrule=1pt, pad at break*=1mm,colback=cellbackground, colframe=cellborder]
\prompt{In}{incolor}{49}{\boxspacing}
\begin{Verbatim}[commandchars=\\\{\}]
\PY{k}{assert}\PY{p}{(}\PY{n}{gcd}\PY{p}{(}\PY{n+nb}{list}\PY{p}{(}\PY{n}{P7}\PY{p}{)}\PY{p}{)} \PY{o}{==} \PY{n}{scalar\PYZus{}product}\PY{p}{(}\PY{n}{P1}\PY{p}{,} \PY{n}{P1}\PY{p}{)}\PY{o}{\PYZca{}}\PY{l+m+mi}{4}\PY{o}{*}\PY{n}{v1}\PY{p}{)}
\end{Verbatim}
\end{tcolorbox}

    \begin{tcolorbox}[breakable, size=fbox, boxrule=1pt, pad at break*=1mm,colback=cellbackground, colframe=cellborder]
\prompt{In}{incolor}{50}{\boxspacing}
\begin{Verbatim}[commandchars=\\\{\}]
\PY{c+c1}{\PYZsh{}\PYZsh{} Since we know that s11 is not 0, we redefine P7:}
\PY{n}{P7} \PY{o}{=} \PY{n}{vector}\PY{p}{(}\PY{n}{S}\PY{p}{,} \PY{p}{[}\PY{n}{p7}\PY{o}{.}\PY{n}{quo\PYZus{}rem}\PY{p}{(}\PY{n}{gcd}\PY{p}{(}\PY{n+nb}{list}\PY{p}{(}\PY{n}{P7}\PY{p}{)}\PY{p}{)}\PY{p}{)}\PY{p}{[}\PY{l+m+mi}{0}\PY{p}{]} \PY{k}{for} \PY{n}{p7} \PY{o+ow}{in} \PY{n}{P7}\PY{p}{]}\PY{p}{)}
\end{Verbatim}
\end{tcolorbox}

    Finally, we show that the three rows of the matrix \(\Phi(P_7)\) are
linearly dependent of the 9 rows of \(M_9\), hence \(P_7\) is an
eigenpoint of the unique cubic defined by
\(\Lambda(\Phi(P_1, \dots, P_5))\).

These computations require seconds about 180 seconds.

    \begin{tcolorbox}[breakable, size=fbox, boxrule=1pt, pad at break*=1mm,colback=cellbackground, colframe=cellborder]
\prompt{In}{incolor}{51}{\boxspacing}
\begin{Verbatim}[commandchars=\\\{\}]
\PY{n}{ttA} \PY{o}{=} \PY{n}{cputime}\PY{p}{(}\PY{p}{)}
\PY{n}{Phi\PYZus{}of\PYZus{}P7} \PY{o}{=} \PY{n}{condition\PYZus{}matrix}\PY{p}{(}\PY{p}{[}\PY{n}{P7}\PY{p}{]}\PY{p}{,} \PY{n}{S}\PY{p}{,} \PY{n}{standard}\PY{o}{=}\PY{l+s+s2}{\PYZdq{}}\PY{l+s+s2}{all}\PY{l+s+s2}{\PYZdq{}}\PY{p}{)}
\PY{k}{assert}\PY{p}{(}\PY{n}{det}\PY{p}{(}\PY{n}{M9}\PY{o}{.}\PY{n}{stack}\PY{p}{(}\PY{n}{Phi\PYZus{}of\PYZus{}P7}\PY{p}{[}\PY{l+m+mi}{0}\PY{p}{]}\PY{p}{)}\PY{p}{)} \PY{o}{==} \PY{l+m+mi}{0}\PY{p}{)}
\PY{k}{assert}\PY{p}{(}\PY{n}{det}\PY{p}{(}\PY{n}{M9}\PY{o}{.}\PY{n}{stack}\PY{p}{(}\PY{n}{Phi\PYZus{}of\PYZus{}P7}\PY{p}{[}\PY{l+m+mi}{1}\PY{p}{]}\PY{p}{)}\PY{p}{)} \PY{o}{==} \PY{l+m+mi}{0}\PY{p}{)}
\PY{k}{assert}\PY{p}{(}\PY{n}{det}\PY{p}{(}\PY{n}{M9}\PY{o}{.}\PY{n}{stack}\PY{p}{(}\PY{n}{Phi\PYZus{}of\PYZus{}P7}\PY{p}{[}\PY{l+m+mi}{2}\PY{p}{]}\PY{p}{)}\PY{p}{)} \PY{o}{==} \PY{l+m+mi}{0}\PY{p}{)}
\PY{n+nb}{print}\PY{p}{(}\PY{l+s+s2}{\PYZdq{}}\PY{l+s+s2}{time of computation: }\PY{l+s+s2}{\PYZdq{}}\PY{o}{+}\PY{n+nb}{str}\PY{p}{(}\PY{n}{cputime}\PY{p}{(}\PY{p}{)}\PY{o}{\PYZhy{}}\PY{n}{ttA}\PY{p}{)}\PY{p}{)}
\end{Verbatim}
\end{tcolorbox}

    \begin{Verbatim}[commandchars=\\\{\}]
time of computation: 178.981875
    \end{Verbatim}

    \hypertarget{a-final-computation}{%
\subsubsection{A final computation}\label{a-final-computation}}

    Here we see that if we impose that \((P_2, P_5, P_7\)) are aligned, then
we get a (C8) configuration:

    \begin{tcolorbox}[breakable, size=fbox, boxrule=1pt, pad at break*=1mm,colback=cellbackground, colframe=cellborder]
\prompt{In}{incolor}{52}{\boxspacing}
\begin{Verbatim}[commandchars=\\\{\}]
\PY{c+c1}{\PYZsh{}we define a list of triplets, each is given by three points that }
\PY{c+c1}{\PYZsh{} must not be aligned:}
\PY{n}{impossible\PYZus{}collin} \PY{o}{=} \PY{p}{[}
    \PY{p}{[}\PY{n}{P1}\PY{p}{,} \PY{n}{P2}\PY{p}{,} \PY{n}{P4}\PY{p}{]}\PY{p}{,} \PY{p}{[}\PY{n}{P1}\PY{p}{,} \PY{n}{P2}\PY{p}{,} \PY{n}{P6}\PY{p}{]}\PY{p}{,} \PY{p}{[}\PY{n}{P1}\PY{p}{,} \PY{n}{P4}\PY{p}{,} \PY{n}{P6}\PY{p}{]}\PY{p}{,} 
    \PY{p}{[}\PY{n}{P2}\PY{p}{,} \PY{n}{P3}\PY{p}{,} \PY{n}{P4}\PY{p}{]}\PY{p}{,} \PY{p}{[}\PY{n}{P2}\PY{p}{,} \PY{n}{P4}\PY{p}{,} \PY{n}{P5}\PY{p}{]}\PY{p}{,} \PY{p}{[}\PY{n}{P2}\PY{p}{,} \PY{n}{P6}\PY{p}{,} \PY{n}{P7}\PY{p}{]}\PY{p}{,} 
    \PY{p}{[}\PY{n}{P4}\PY{p}{,} \PY{n}{P6}\PY{p}{,} \PY{n}{P7}\PY{p}{]}\PY{p}{,} \PY{p}{[}\PY{n}{P1}\PY{p}{,} \PY{n}{P2}\PY{p}{,} \PY{n}{P7}\PY{p}{]}\PY{p}{,} \PY{p}{[}\PY{n}{P2}\PY{p}{,} \PY{n}{P3}\PY{p}{,} \PY{n}{P7}\PY{p}{]}
\PY{p}{]}
\end{Verbatim}
\end{tcolorbox}

    \begin{tcolorbox}[breakable, size=fbox, boxrule=1pt, pad at break*=1mm,colback=cellbackground, colframe=cellborder]
\prompt{In}{incolor}{ }{\boxspacing}
\begin{Verbatim}[commandchars=\\\{\}]
\PY{c+c1}{\PYZsh{}\PYZsh{} We define the condition (P2, P5, P7) aligned and we simplify it, }
\PY{c+c1}{\PYZsh{}\PYZsh{} erasing factors that are surely not zero:}

\PY{n}{J} \PY{o}{=} \PY{n}{S}\PY{o}{.}\PY{n}{ideal}\PY{p}{(}\PY{n}{matrix}\PY{p}{(}\PY{p}{[}\PY{n}{P2}\PY{p}{,} \PY{n}{P5}\PY{p}{,} \PY{n}{P7}\PY{p}{]}\PY{p}{)}\PY{o}{.}\PY{n}{det}\PY{p}{(}\PY{p}{)}\PY{p}{)}\PY{o}{.}\PY{n}{saturation}\PY{p}{(}\PY{n}{u1}\PY{o}{*}\PY{n}{u2}\PY{o}{*}\PY{n}{v1}\PY{o}{*}\PY{n}{v2}\PY{p}{)}\PY{p}{[}\PY{l+m+mi}{0}\PY{p}{]}
\PY{k}{for} \PY{n}{tr} \PY{o+ow}{in} \PY{n}{impossible\PYZus{}collin}\PY{p}{:}
    \PY{n}{J} \PY{o}{=} \PY{n}{J}\PY{o}{.}\PY{n}{saturation}\PY{p}{(}\PY{n}{matrix}\PY{p}{(}\PY{n}{tr}\PY{p}{)}\PY{o}{.}\PY{n}{det}\PY{p}{(}\PY{p}{)}\PY{p}{)}\PY{p}{[}\PY{l+m+mi}{0}\PY{p}{]}
\end{Verbatim}
\end{tcolorbox}

    J is a principal ideal. Its generator has two factors. The first implies
that \(P_3\), \(P_4\), \(P_7\) are aligned, so we have the alignments:

(1, 2, 3), (1, 4, 5), (2, 4, 6), (2, 5, 7), (3, 4, 7)

The second factor implies the alignments:

(1, 2, 3), (1, 4, 5), (2, 4, 6), (2, 5, 7), (3, 5, 6):

    \begin{tcolorbox}[breakable, size=fbox, boxrule=1pt, pad at break*=1mm,colback=cellbackground, colframe=cellborder]
\prompt{In}{incolor}{ }{\boxspacing}
\begin{Verbatim}[commandchars=\\\{\}]
\PY{n}{pdJ} \PY{o}{=} \PY{n}{J}\PY{o}{.}\PY{n}{primary\PYZus{}decomposition}\PY{p}{(}\PY{p}{)}
\PY{k}{assert}\PY{p}{(}\PY{n+nb}{len}\PY{p}{(}\PY{n}{pdJ}\PY{p}{)} \PY{o}{==} \PY{l+m+mi}{2}\PY{p}{)}
\PY{k}{assert}\PY{p}{(}\PY{n}{pdJ}\PY{p}{[}\PY{l+m+mi}{0}\PY{p}{]}\PY{o}{.}\PY{n}{reduce}\PY{p}{(}\PY{n}{matrix}\PY{p}{(}\PY{p}{[}\PY{n}{P3}\PY{p}{,} \PY{n}{P4}\PY{p}{,} \PY{n}{P7}\PY{p}{]}\PY{p}{)}\PY{o}{.}\PY{n}{det}\PY{p}{(}\PY{p}{)}\PY{p}{)} \PY{o}{==} \PY{l+m+mi}{0}\PY{p}{)}
\PY{k}{assert}\PY{p}{(}\PY{n}{pdJ}\PY{p}{[}\PY{l+m+mi}{1}\PY{p}{]}\PY{o}{.}\PY{n}{reduce}\PY{p}{(}\PY{n}{matrix}\PY{p}{(}\PY{p}{[}\PY{n}{P3}\PY{p}{,} \PY{n}{P5}\PY{p}{,} \PY{n}{P6}\PY{p}{]}\PY{p}{)}\PY{o}{.}\PY{n}{det}\PY{p}{(}\PY{p}{)}\PY{p}{)} \PY{o}{==} \PY{l+m+mi}{0}\PY{p}{)}
\end{Verbatim}
\end{tcolorbox}

    Similar computations can be done for the condition (P3, P4, P7) aligned
or (P3, P5, P6) aligned.


    % Add a bibliography block to the postdoc
    
    
    
\end{document}
