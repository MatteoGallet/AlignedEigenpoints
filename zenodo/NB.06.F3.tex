\documentclass[11pt]{article}

    \usepackage[breakable]{tcolorbox}
    \usepackage{parskip} % Stop auto-indenting (to mimic markdown behaviour)
    

    % Basic figure setup, for now with no caption control since it's done
    % automatically by Pandoc (which extracts ![](path) syntax from Markdown).
    \usepackage{graphicx}
    % Maintain compatibility with old templates. Remove in nbconvert 6.0
    \let\Oldincludegraphics\includegraphics
    % Ensure that by default, figures have no caption (until we provide a
    % proper Figure object with a Caption API and a way to capture that
    % in the conversion process - todo).
    \usepackage{caption}
    \DeclareCaptionFormat{nocaption}{}
    \captionsetup{format=nocaption,aboveskip=0pt,belowskip=0pt}

    \usepackage{float}
    \floatplacement{figure}{H} % forces figures to be placed at the correct location
    \usepackage{xcolor} % Allow colors to be defined
    \usepackage{enumerate} % Needed for markdown enumerations to work
    \usepackage{geometry} % Used to adjust the document margins
    \usepackage{amsmath} % Equations
    \usepackage{amssymb} % Equations
    \usepackage{textcomp} % defines textquotesingle
    % Hack from http://tex.stackexchange.com/a/47451/13684:
    \AtBeginDocument{%
        \def\PYZsq{\textquotesingle}% Upright quotes in Pygmentized code
    }
    \usepackage{upquote} % Upright quotes for verbatim code
    \usepackage{eurosym} % defines \euro

    \usepackage{iftex}
    \ifPDFTeX
        \usepackage[T1]{fontenc}
        \IfFileExists{alphabeta.sty}{
              \usepackage{alphabeta}
          }{
              \usepackage[mathletters]{ucs}
              \usepackage[utf8x]{inputenc}
          }
    \else
        \usepackage{fontspec}
        \usepackage{unicode-math}
    \fi

    \usepackage{fancyvrb} % verbatim replacement that allows latex
    \usepackage{grffile} % extends the file name processing of package graphics
                         % to support a larger range
    \makeatletter % fix for old versions of grffile with XeLaTeX
    \@ifpackagelater{grffile}{2019/11/01}
    {
      % Do nothing on new versions
    }
    {
      \def\Gread@@xetex#1{%
        \IfFileExists{"\Gin@base".bb}%
        {\Gread@eps{\Gin@base.bb}}%
        {\Gread@@xetex@aux#1}%
      }
    }
    \makeatother
    \usepackage[Export]{adjustbox} % Used to constrain images to a maximum size
    \adjustboxset{max size={0.9\linewidth}{0.9\paperheight}}

    % The hyperref package gives us a pdf with properly built
    % internal navigation ('pdf bookmarks' for the table of contents,
    % internal cross-reference links, web links for URLs, etc.)
    \usepackage{hyperref}
    % The default LaTeX title has an obnoxious amount of whitespace. By default,
    % titling removes some of it. It also provides customization options.
    \usepackage{titling}
    \usepackage{longtable} % longtable support required by pandoc >1.10
    \usepackage{booktabs}  % table support for pandoc > 1.12.2
    \usepackage{array}     % table support for pandoc >= 2.11.3
    \usepackage{calc}      % table minipage width calculation for pandoc >= 2.11.1
    \usepackage[inline]{enumitem} % IRkernel/repr support (it uses the enumerate* environment)
    \usepackage[normalem]{ulem} % ulem is needed to support strikethroughs (\sout)
                                % normalem makes italics be italics, not underlines
    \usepackage{soul}      % strikethrough (\st) support for pandoc >= 3.0.0
    \usepackage{mathrsfs}
    

    
    % Colors for the hyperref package
    \definecolor{urlcolor}{rgb}{0,.145,.698}
    \definecolor{linkcolor}{rgb}{.71,0.21,0.01}
    \definecolor{citecolor}{rgb}{.12,.54,.11}

    % ANSI colors
    \definecolor{ansi-black}{HTML}{3E424D}
    \definecolor{ansi-black-intense}{HTML}{282C36}
    \definecolor{ansi-red}{HTML}{E75C58}
    \definecolor{ansi-red-intense}{HTML}{B22B31}
    \definecolor{ansi-green}{HTML}{00A250}
    \definecolor{ansi-green-intense}{HTML}{007427}
    \definecolor{ansi-yellow}{HTML}{DDB62B}
    \definecolor{ansi-yellow-intense}{HTML}{B27D12}
    \definecolor{ansi-blue}{HTML}{208FFB}
    \definecolor{ansi-blue-intense}{HTML}{0065CA}
    \definecolor{ansi-magenta}{HTML}{D160C4}
    \definecolor{ansi-magenta-intense}{HTML}{A03196}
    \definecolor{ansi-cyan}{HTML}{60C6C8}
    \definecolor{ansi-cyan-intense}{HTML}{258F8F}
    \definecolor{ansi-white}{HTML}{C5C1B4}
    \definecolor{ansi-white-intense}{HTML}{A1A6B2}
    \definecolor{ansi-default-inverse-fg}{HTML}{FFFFFF}
    \definecolor{ansi-default-inverse-bg}{HTML}{000000}

    % common color for the border for error outputs.
    \definecolor{outerrorbackground}{HTML}{FFDFDF}

    % commands and environments needed by pandoc snippets
    % extracted from the output of `pandoc -s`
    \providecommand{\tightlist}{%
      \setlength{\itemsep}{0pt}\setlength{\parskip}{0pt}}
    \DefineVerbatimEnvironment{Highlighting}{Verbatim}{commandchars=\\\{\}}
    % Add ',fontsize=\small' for more characters per line
    \newenvironment{Shaded}{}{}
    \newcommand{\KeywordTok}[1]{\textcolor[rgb]{0.00,0.44,0.13}{\textbf{{#1}}}}
    \newcommand{\DataTypeTok}[1]{\textcolor[rgb]{0.56,0.13,0.00}{{#1}}}
    \newcommand{\DecValTok}[1]{\textcolor[rgb]{0.25,0.63,0.44}{{#1}}}
    \newcommand{\BaseNTok}[1]{\textcolor[rgb]{0.25,0.63,0.44}{{#1}}}
    \newcommand{\FloatTok}[1]{\textcolor[rgb]{0.25,0.63,0.44}{{#1}}}
    \newcommand{\CharTok}[1]{\textcolor[rgb]{0.25,0.44,0.63}{{#1}}}
    \newcommand{\StringTok}[1]{\textcolor[rgb]{0.25,0.44,0.63}{{#1}}}
    \newcommand{\CommentTok}[1]{\textcolor[rgb]{0.38,0.63,0.69}{\textit{{#1}}}}
    \newcommand{\OtherTok}[1]{\textcolor[rgb]{0.00,0.44,0.13}{{#1}}}
    \newcommand{\AlertTok}[1]{\textcolor[rgb]{1.00,0.00,0.00}{\textbf{{#1}}}}
    \newcommand{\FunctionTok}[1]{\textcolor[rgb]{0.02,0.16,0.49}{{#1}}}
    \newcommand{\RegionMarkerTok}[1]{{#1}}
    \newcommand{\ErrorTok}[1]{\textcolor[rgb]{1.00,0.00,0.00}{\textbf{{#1}}}}
    \newcommand{\NormalTok}[1]{{#1}}

    % Additional commands for more recent versions of Pandoc
    \newcommand{\ConstantTok}[1]{\textcolor[rgb]{0.53,0.00,0.00}{{#1}}}
    \newcommand{\SpecialCharTok}[1]{\textcolor[rgb]{0.25,0.44,0.63}{{#1}}}
    \newcommand{\VerbatimStringTok}[1]{\textcolor[rgb]{0.25,0.44,0.63}{{#1}}}
    \newcommand{\SpecialStringTok}[1]{\textcolor[rgb]{0.73,0.40,0.53}{{#1}}}
    \newcommand{\ImportTok}[1]{{#1}}
    \newcommand{\DocumentationTok}[1]{\textcolor[rgb]{0.73,0.13,0.13}{\textit{{#1}}}}
    \newcommand{\AnnotationTok}[1]{\textcolor[rgb]{0.38,0.63,0.69}{\textbf{\textit{{#1}}}}}
    \newcommand{\CommentVarTok}[1]{\textcolor[rgb]{0.38,0.63,0.69}{\textbf{\textit{{#1}}}}}
    \newcommand{\VariableTok}[1]{\textcolor[rgb]{0.10,0.09,0.49}{{#1}}}
    \newcommand{\ControlFlowTok}[1]{\textcolor[rgb]{0.00,0.44,0.13}{\textbf{{#1}}}}
    \newcommand{\OperatorTok}[1]{\textcolor[rgb]{0.40,0.40,0.40}{{#1}}}
    \newcommand{\BuiltInTok}[1]{{#1}}
    \newcommand{\ExtensionTok}[1]{{#1}}
    \newcommand{\PreprocessorTok}[1]{\textcolor[rgb]{0.74,0.48,0.00}{{#1}}}
    \newcommand{\AttributeTok}[1]{\textcolor[rgb]{0.49,0.56,0.16}{{#1}}}
    \newcommand{\InformationTok}[1]{\textcolor[rgb]{0.38,0.63,0.69}{\textbf{\textit{{#1}}}}}
    \newcommand{\WarningTok}[1]{\textcolor[rgb]{0.38,0.63,0.69}{\textbf{\textit{{#1}}}}}


    % Define a nice break command that doesn't care if a line doesn't already
    % exist.
    \def\br{\hspace*{\fill} \\* }
    % Math Jax compatibility definitions
    \def\gt{>}
    \def\lt{<}
    \let\Oldtex\TeX
    \let\Oldlatex\LaTeX
    \renewcommand{\TeX}{\textrm{\Oldtex}}
    \renewcommand{\LaTeX}{\textrm{\Oldlatex}}
    % Document parameters
    % Document title
    \title{NB.06.F3}
    
    
    
    
    
    
    
% Pygments definitions
\makeatletter
\def\PY@reset{\let\PY@it=\relax \let\PY@bf=\relax%
    \let\PY@ul=\relax \let\PY@tc=\relax%
    \let\PY@bc=\relax \let\PY@ff=\relax}
\def\PY@tok#1{\csname PY@tok@#1\endcsname}
\def\PY@toks#1+{\ifx\relax#1\empty\else%
    \PY@tok{#1}\expandafter\PY@toks\fi}
\def\PY@do#1{\PY@bc{\PY@tc{\PY@ul{%
    \PY@it{\PY@bf{\PY@ff{#1}}}}}}}
\def\PY#1#2{\PY@reset\PY@toks#1+\relax+\PY@do{#2}}

\@namedef{PY@tok@w}{\def\PY@tc##1{\textcolor[rgb]{0.73,0.73,0.73}{##1}}}
\@namedef{PY@tok@c}{\let\PY@it=\textit\def\PY@tc##1{\textcolor[rgb]{0.24,0.48,0.48}{##1}}}
\@namedef{PY@tok@cp}{\def\PY@tc##1{\textcolor[rgb]{0.61,0.40,0.00}{##1}}}
\@namedef{PY@tok@k}{\let\PY@bf=\textbf\def\PY@tc##1{\textcolor[rgb]{0.00,0.50,0.00}{##1}}}
\@namedef{PY@tok@kp}{\def\PY@tc##1{\textcolor[rgb]{0.00,0.50,0.00}{##1}}}
\@namedef{PY@tok@kt}{\def\PY@tc##1{\textcolor[rgb]{0.69,0.00,0.25}{##1}}}
\@namedef{PY@tok@o}{\def\PY@tc##1{\textcolor[rgb]{0.40,0.40,0.40}{##1}}}
\@namedef{PY@tok@ow}{\let\PY@bf=\textbf\def\PY@tc##1{\textcolor[rgb]{0.67,0.13,1.00}{##1}}}
\@namedef{PY@tok@nb}{\def\PY@tc##1{\textcolor[rgb]{0.00,0.50,0.00}{##1}}}
\@namedef{PY@tok@nf}{\def\PY@tc##1{\textcolor[rgb]{0.00,0.00,1.00}{##1}}}
\@namedef{PY@tok@nc}{\let\PY@bf=\textbf\def\PY@tc##1{\textcolor[rgb]{0.00,0.00,1.00}{##1}}}
\@namedef{PY@tok@nn}{\let\PY@bf=\textbf\def\PY@tc##1{\textcolor[rgb]{0.00,0.00,1.00}{##1}}}
\@namedef{PY@tok@ne}{\let\PY@bf=\textbf\def\PY@tc##1{\textcolor[rgb]{0.80,0.25,0.22}{##1}}}
\@namedef{PY@tok@nv}{\def\PY@tc##1{\textcolor[rgb]{0.10,0.09,0.49}{##1}}}
\@namedef{PY@tok@no}{\def\PY@tc##1{\textcolor[rgb]{0.53,0.00,0.00}{##1}}}
\@namedef{PY@tok@nl}{\def\PY@tc##1{\textcolor[rgb]{0.46,0.46,0.00}{##1}}}
\@namedef{PY@tok@ni}{\let\PY@bf=\textbf\def\PY@tc##1{\textcolor[rgb]{0.44,0.44,0.44}{##1}}}
\@namedef{PY@tok@na}{\def\PY@tc##1{\textcolor[rgb]{0.41,0.47,0.13}{##1}}}
\@namedef{PY@tok@nt}{\let\PY@bf=\textbf\def\PY@tc##1{\textcolor[rgb]{0.00,0.50,0.00}{##1}}}
\@namedef{PY@tok@nd}{\def\PY@tc##1{\textcolor[rgb]{0.67,0.13,1.00}{##1}}}
\@namedef{PY@tok@s}{\def\PY@tc##1{\textcolor[rgb]{0.73,0.13,0.13}{##1}}}
\@namedef{PY@tok@sd}{\let\PY@it=\textit\def\PY@tc##1{\textcolor[rgb]{0.73,0.13,0.13}{##1}}}
\@namedef{PY@tok@si}{\let\PY@bf=\textbf\def\PY@tc##1{\textcolor[rgb]{0.64,0.35,0.47}{##1}}}
\@namedef{PY@tok@se}{\let\PY@bf=\textbf\def\PY@tc##1{\textcolor[rgb]{0.67,0.36,0.12}{##1}}}
\@namedef{PY@tok@sr}{\def\PY@tc##1{\textcolor[rgb]{0.64,0.35,0.47}{##1}}}
\@namedef{PY@tok@ss}{\def\PY@tc##1{\textcolor[rgb]{0.10,0.09,0.49}{##1}}}
\@namedef{PY@tok@sx}{\def\PY@tc##1{\textcolor[rgb]{0.00,0.50,0.00}{##1}}}
\@namedef{PY@tok@m}{\def\PY@tc##1{\textcolor[rgb]{0.40,0.40,0.40}{##1}}}
\@namedef{PY@tok@gh}{\let\PY@bf=\textbf\def\PY@tc##1{\textcolor[rgb]{0.00,0.00,0.50}{##1}}}
\@namedef{PY@tok@gu}{\let\PY@bf=\textbf\def\PY@tc##1{\textcolor[rgb]{0.50,0.00,0.50}{##1}}}
\@namedef{PY@tok@gd}{\def\PY@tc##1{\textcolor[rgb]{0.63,0.00,0.00}{##1}}}
\@namedef{PY@tok@gi}{\def\PY@tc##1{\textcolor[rgb]{0.00,0.52,0.00}{##1}}}
\@namedef{PY@tok@gr}{\def\PY@tc##1{\textcolor[rgb]{0.89,0.00,0.00}{##1}}}
\@namedef{PY@tok@ge}{\let\PY@it=\textit}
\@namedef{PY@tok@gs}{\let\PY@bf=\textbf}
\@namedef{PY@tok@ges}{\let\PY@bf=\textbf\let\PY@it=\textit}
\@namedef{PY@tok@gp}{\let\PY@bf=\textbf\def\PY@tc##1{\textcolor[rgb]{0.00,0.00,0.50}{##1}}}
\@namedef{PY@tok@go}{\def\PY@tc##1{\textcolor[rgb]{0.44,0.44,0.44}{##1}}}
\@namedef{PY@tok@gt}{\def\PY@tc##1{\textcolor[rgb]{0.00,0.27,0.87}{##1}}}
\@namedef{PY@tok@err}{\def\PY@bc##1{{\setlength{\fboxsep}{\string -\fboxrule}\fcolorbox[rgb]{1.00,0.00,0.00}{1,1,1}{\strut ##1}}}}
\@namedef{PY@tok@kc}{\let\PY@bf=\textbf\def\PY@tc##1{\textcolor[rgb]{0.00,0.50,0.00}{##1}}}
\@namedef{PY@tok@kd}{\let\PY@bf=\textbf\def\PY@tc##1{\textcolor[rgb]{0.00,0.50,0.00}{##1}}}
\@namedef{PY@tok@kn}{\let\PY@bf=\textbf\def\PY@tc##1{\textcolor[rgb]{0.00,0.50,0.00}{##1}}}
\@namedef{PY@tok@kr}{\let\PY@bf=\textbf\def\PY@tc##1{\textcolor[rgb]{0.00,0.50,0.00}{##1}}}
\@namedef{PY@tok@bp}{\def\PY@tc##1{\textcolor[rgb]{0.00,0.50,0.00}{##1}}}
\@namedef{PY@tok@fm}{\def\PY@tc##1{\textcolor[rgb]{0.00,0.00,1.00}{##1}}}
\@namedef{PY@tok@vc}{\def\PY@tc##1{\textcolor[rgb]{0.10,0.09,0.49}{##1}}}
\@namedef{PY@tok@vg}{\def\PY@tc##1{\textcolor[rgb]{0.10,0.09,0.49}{##1}}}
\@namedef{PY@tok@vi}{\def\PY@tc##1{\textcolor[rgb]{0.10,0.09,0.49}{##1}}}
\@namedef{PY@tok@vm}{\def\PY@tc##1{\textcolor[rgb]{0.10,0.09,0.49}{##1}}}
\@namedef{PY@tok@sa}{\def\PY@tc##1{\textcolor[rgb]{0.73,0.13,0.13}{##1}}}
\@namedef{PY@tok@sb}{\def\PY@tc##1{\textcolor[rgb]{0.73,0.13,0.13}{##1}}}
\@namedef{PY@tok@sc}{\def\PY@tc##1{\textcolor[rgb]{0.73,0.13,0.13}{##1}}}
\@namedef{PY@tok@dl}{\def\PY@tc##1{\textcolor[rgb]{0.73,0.13,0.13}{##1}}}
\@namedef{PY@tok@s2}{\def\PY@tc##1{\textcolor[rgb]{0.73,0.13,0.13}{##1}}}
\@namedef{PY@tok@sh}{\def\PY@tc##1{\textcolor[rgb]{0.73,0.13,0.13}{##1}}}
\@namedef{PY@tok@s1}{\def\PY@tc##1{\textcolor[rgb]{0.73,0.13,0.13}{##1}}}
\@namedef{PY@tok@mb}{\def\PY@tc##1{\textcolor[rgb]{0.40,0.40,0.40}{##1}}}
\@namedef{PY@tok@mf}{\def\PY@tc##1{\textcolor[rgb]{0.40,0.40,0.40}{##1}}}
\@namedef{PY@tok@mh}{\def\PY@tc##1{\textcolor[rgb]{0.40,0.40,0.40}{##1}}}
\@namedef{PY@tok@mi}{\def\PY@tc##1{\textcolor[rgb]{0.40,0.40,0.40}{##1}}}
\@namedef{PY@tok@il}{\def\PY@tc##1{\textcolor[rgb]{0.40,0.40,0.40}{##1}}}
\@namedef{PY@tok@mo}{\def\PY@tc##1{\textcolor[rgb]{0.40,0.40,0.40}{##1}}}
\@namedef{PY@tok@ch}{\let\PY@it=\textit\def\PY@tc##1{\textcolor[rgb]{0.24,0.48,0.48}{##1}}}
\@namedef{PY@tok@cm}{\let\PY@it=\textit\def\PY@tc##1{\textcolor[rgb]{0.24,0.48,0.48}{##1}}}
\@namedef{PY@tok@cpf}{\let\PY@it=\textit\def\PY@tc##1{\textcolor[rgb]{0.24,0.48,0.48}{##1}}}
\@namedef{PY@tok@c1}{\let\PY@it=\textit\def\PY@tc##1{\textcolor[rgb]{0.24,0.48,0.48}{##1}}}
\@namedef{PY@tok@cs}{\let\PY@it=\textit\def\PY@tc##1{\textcolor[rgb]{0.24,0.48,0.48}{##1}}}

\def\PYZbs{\char`\\}
\def\PYZus{\char`\_}
\def\PYZob{\char`\{}
\def\PYZcb{\char`\}}
\def\PYZca{\char`\^}
\def\PYZam{\char`\&}
\def\PYZlt{\char`\<}
\def\PYZgt{\char`\>}
\def\PYZsh{\char`\#}
\def\PYZpc{\char`\%}
\def\PYZdl{\char`\$}
\def\PYZhy{\char`\-}
\def\PYZsq{\char`\'}
\def\PYZdq{\char`\"}
\def\PYZti{\char`\~}
% for compatibility with earlier versions
\def\PYZat{@}
\def\PYZlb{[}
\def\PYZrb{]}
\makeatother


    % For linebreaks inside Verbatim environment from package fancyvrb.
    \makeatletter
        \newbox\Wrappedcontinuationbox
        \newbox\Wrappedvisiblespacebox
        \newcommand*\Wrappedvisiblespace {\textcolor{red}{\textvisiblespace}}
        \newcommand*\Wrappedcontinuationsymbol {\textcolor{red}{\llap{\tiny$\m@th\hookrightarrow$}}}
        \newcommand*\Wrappedcontinuationindent {3ex }
        \newcommand*\Wrappedafterbreak {\kern\Wrappedcontinuationindent\copy\Wrappedcontinuationbox}
        % Take advantage of the already applied Pygments mark-up to insert
        % potential linebreaks for TeX processing.
        %        {, <, #, %, $, ' and ": go to next line.
        %        _, }, ^, &, >, - and ~: stay at end of broken line.
        % Use of \textquotesingle for straight quote.
        \newcommand*\Wrappedbreaksatspecials {%
            \def\PYGZus{\discretionary{\char`\_}{\Wrappedafterbreak}{\char`\_}}%
            \def\PYGZob{\discretionary{}{\Wrappedafterbreak\char`\{}{\char`\{}}%
            \def\PYGZcb{\discretionary{\char`\}}{\Wrappedafterbreak}{\char`\}}}%
            \def\PYGZca{\discretionary{\char`\^}{\Wrappedafterbreak}{\char`\^}}%
            \def\PYGZam{\discretionary{\char`\&}{\Wrappedafterbreak}{\char`\&}}%
            \def\PYGZlt{\discretionary{}{\Wrappedafterbreak\char`\<}{\char`\<}}%
            \def\PYGZgt{\discretionary{\char`\>}{\Wrappedafterbreak}{\char`\>}}%
            \def\PYGZsh{\discretionary{}{\Wrappedafterbreak\char`\#}{\char`\#}}%
            \def\PYGZpc{\discretionary{}{\Wrappedafterbreak\char`\%}{\char`\%}}%
            \def\PYGZdl{\discretionary{}{\Wrappedafterbreak\char`\$}{\char`\$}}%
            \def\PYGZhy{\discretionary{\char`\-}{\Wrappedafterbreak}{\char`\-}}%
            \def\PYGZsq{\discretionary{}{\Wrappedafterbreak\textquotesingle}{\textquotesingle}}%
            \def\PYGZdq{\discretionary{}{\Wrappedafterbreak\char`\"}{\char`\"}}%
            \def\PYGZti{\discretionary{\char`\~}{\Wrappedafterbreak}{\char`\~}}%
        }
        % Some characters . , ; ? ! / are not pygmentized.
        % This macro makes them "active" and they will insert potential linebreaks
        \newcommand*\Wrappedbreaksatpunct {%
            \lccode`\~`\.\lowercase{\def~}{\discretionary{\hbox{\char`\.}}{\Wrappedafterbreak}{\hbox{\char`\.}}}%
            \lccode`\~`\,\lowercase{\def~}{\discretionary{\hbox{\char`\,}}{\Wrappedafterbreak}{\hbox{\char`\,}}}%
            \lccode`\~`\;\lowercase{\def~}{\discretionary{\hbox{\char`\;}}{\Wrappedafterbreak}{\hbox{\char`\;}}}%
            \lccode`\~`\:\lowercase{\def~}{\discretionary{\hbox{\char`\:}}{\Wrappedafterbreak}{\hbox{\char`\:}}}%
            \lccode`\~`\?\lowercase{\def~}{\discretionary{\hbox{\char`\?}}{\Wrappedafterbreak}{\hbox{\char`\?}}}%
            \lccode`\~`\!\lowercase{\def~}{\discretionary{\hbox{\char`\!}}{\Wrappedafterbreak}{\hbox{\char`\!}}}%
            \lccode`\~`\/\lowercase{\def~}{\discretionary{\hbox{\char`\/}}{\Wrappedafterbreak}{\hbox{\char`\/}}}%
            \catcode`\.\active
            \catcode`\,\active
            \catcode`\;\active
            \catcode`\:\active
            \catcode`\?\active
            \catcode`\!\active
            \catcode`\/\active
            \lccode`\~`\~
        }
    \makeatother

    \let\OriginalVerbatim=\Verbatim
    \makeatletter
    \renewcommand{\Verbatim}[1][1]{%
        %\parskip\z@skip
        \sbox\Wrappedcontinuationbox {\Wrappedcontinuationsymbol}%
        \sbox\Wrappedvisiblespacebox {\FV@SetupFont\Wrappedvisiblespace}%
        \def\FancyVerbFormatLine ##1{\hsize\linewidth
            \vtop{\raggedright\hyphenpenalty\z@\exhyphenpenalty\z@
                \doublehyphendemerits\z@\finalhyphendemerits\z@
                \strut ##1\strut}%
        }%
        % If the linebreak is at a space, the latter will be displayed as visible
        % space at end of first line, and a continuation symbol starts next line.
        % Stretch/shrink are however usually zero for typewriter font.
        \def\FV@Space {%
            \nobreak\hskip\z@ plus\fontdimen3\font minus\fontdimen4\font
            \discretionary{\copy\Wrappedvisiblespacebox}{\Wrappedafterbreak}
            {\kern\fontdimen2\font}%
        }%

        % Allow breaks at special characters using \PYG... macros.
        \Wrappedbreaksatspecials
        % Breaks at punctuation characters . , ; ? ! and / need catcode=\active
        \OriginalVerbatim[#1,codes*=\Wrappedbreaksatpunct]%
    }
    \makeatother

    % Exact colors from NB
    \definecolor{incolor}{HTML}{303F9F}
    \definecolor{outcolor}{HTML}{D84315}
    \definecolor{cellborder}{HTML}{CFCFCF}
    \definecolor{cellbackground}{HTML}{F7F7F7}

    % prompt
    \makeatletter
    \newcommand{\boxspacing}{\kern\kvtcb@left@rule\kern\kvtcb@boxsep}
    \makeatother
    \newcommand{\prompt}[4]{
        {\ttfamily\llap{{\color{#2}[#3]:\hspace{3pt}#4}}\vspace{-\baselineskip}}
    }
    

    
    % Prevent overflowing lines due to hard-to-break entities
    \sloppy
    % Setup hyperref package
    \hypersetup{
      breaklinks=true,  % so long urls are correctly broken across lines
      colorlinks=true,
      urlcolor=urlcolor,
      linkcolor=linkcolor,
      citecolor=citecolor,
      }
    % Slightly bigger margins than the latex defaults
    
    \geometry{verbose,tmargin=1in,bmargin=1in,lmargin=1in,rmargin=1in}
    
    

\begin{document}
    
    \maketitle
    
    

    
    \hypertarget{proposition}{%
\section{Proposition}\label{proposition}}

    Let \(t\) be a line of \(\mathbb{P}^2\). * If \(t\) is not tangent to
the isotropic conic, \(t \subseteq \mathrm{Eig}(C)\) for a cubic \(C\)
if and only \(C = t^2\ell\), where \(\ell\) is any line of the plane. *
If \(t\) is tangent to the isotropic conic (in a
point\textasciitilde{}\(P\)), \(t \subseteq \mathrm{Eig}(C)\) for a
cubic \(C\) if and only if \[
    C = t^2 \ell+\lambda C(r_0),
\] where \(\ell\) is any line of the plane, \(\lambda \in \mathbb{C}\),
\(r_0\) is any fixed line passing through \(P\) different from \(t\) and
\(C(r_0)\) is defined by \[
C(r) = \bigl( r^2-3\left(a^2+b^2+c^2\right) \mathcal{Q}{\mathbb{iso}} \bigr) \, r \,
\] Moreover, if \(t\) is tangent to the isotropic conic at \(P\), any
cubic \(C\) with \(t \subseteq \mathrm{Eig}(C)\) is singular in \(P\)
and \(t\) is one of the tangents \(C\) in \(P\).

    \begin{tcolorbox}[breakable, size=fbox, boxrule=1pt, pad at break*=1mm,colback=cellbackground, colframe=cellborder]
\prompt{In}{incolor}{1}{\boxspacing}
\begin{Verbatim}[commandchars=\\\{\}]
\PY{n}{load}\PY{p}{(}\PY{l+s+s2}{\PYZdq{}}\PY{l+s+s2}{basic\PYZus{}functions.sage}\PY{l+s+s2}{\PYZdq{}}\PY{p}{)}
\end{Verbatim}
\end{tcolorbox}

    \hypertarget{if-a-cubic-c-has-an-eigenline-t-tangent-to-the-isotropic-conic-then-the-equation-of-c-is-f_1-so-we-have-a-linear-system-of-cubics-of-dimension-3.}{%
\subsection{\texorpdfstring{If a cubic \(C\) has an eigenline \(t\)
tangent to the isotropic conic, then the equation of \(C\) is \(F_1\),
so we have a linear system of cubics of dimension
3.}{If a cubic C has an eigenline t tangent to the isotropic conic, then the equation of C is F\_1, so we have a linear system of cubics of dimension 3.}}\label{if-a-cubic-c-has-an-eigenline-t-tangent-to-the-isotropic-conic-then-the-equation-of-c-is-f_1-so-we-have-a-linear-system-of-cubics-of-dimension-3.}}

\hypertarget{this-linear-system-is-of-the-form-t2elllambda-cr-for-any-fixed-line-r-different-from-t.-hence-the-family-of-cubics-f_1-and-the-family-of-cubics-t2elllambda-cr-coincide.}{%
\subsection{\texorpdfstring{This linear system is of the form
\(t^2\ell+\lambda C(r)\) for any fixed line \(r\) different from \(t\).
Hence the family of cubics \(F_1\) and the family of cubics
\(t^2\ell+\lambda C(r)\)
coincide.}{This linear system is of the form t\^{}2\textbackslash ell+\textbackslash lambda C(r) for any fixed line r different from t. Hence the family of cubics F\_1 and the family of cubics t\^{}2\textbackslash ell+\textbackslash lambda C(r) coincide.}}\label{this-linear-system-is-of-the-form-t2elllambda-cr-for-any-fixed-line-r-different-from-t.-hence-the-family-of-cubics-f_1-and-the-family-of-cubics-t2elllambda-cr-coincide.}}

    We define the point \(P_1\) and the line tg1, tangent to Ciso in
\(P_1\), the eigenpoint locus of the generic cubic F of the plane

    \begin{tcolorbox}[breakable, size=fbox, boxrule=1pt, pad at break*=1mm,colback=cellbackground, colframe=cellborder]
\prompt{In}{incolor}{2}{\boxspacing}
\begin{Verbatim}[commandchars=\\\{\}]
\PY{n}{P1} \PY{o}{=} \PY{n}{vector}\PY{p}{(}\PY{p}{(}\PY{l+m+mi}{1}\PY{p}{,} \PY{n}{ii}\PY{p}{,} \PY{l+m+mi}{0}\PY{p}{)}\PY{p}{)}
\PY{n}{tg1} \PY{o}{=} \PY{n}{x}\PY{o}{+}\PY{n}{ii}\PY{o}{*}\PY{n}{y}
\PY{n}{eigF} \PY{o}{=} \PY{n}{eig}\PY{p}{(}\PY{n}{F}\PY{p}{)}
\end{Verbatim}
\end{tcolorbox}

    In order to have that the points of tg1 are eigenpoints, we substitute
\(x=-iy\) into eigF and we get a list of three equations that have to be
satisfied for all values of \(x\)

    \begin{tcolorbox}[breakable, size=fbox, boxrule=1pt, pad at break*=1mm,colback=cellbackground, colframe=cellborder]
\prompt{In}{incolor}{3}{\boxspacing}
\begin{Verbatim}[commandchars=\\\{\}]
\PY{n}{EE} \PY{o}{=} \PY{n+nb}{list}\PY{p}{(}\PY{n}{eigF}\PY{o}{.}\PY{n}{subs}\PY{p}{(}\PY{n}{x} \PY{o}{=} \PY{o}{\PYZhy{}}\PY{n}{ii}\PY{o}{*}\PY{n}{y}\PY{p}{)}\PY{p}{)} 
\end{Verbatim}
\end{tcolorbox}

    We construct the list eqs of equations in \(a_0, \dotsc, a_9\) that must
be satisfied in order to have tg1 eigenline:

    \begin{tcolorbox}[breakable, size=fbox, boxrule=1pt, pad at break*=1mm,colback=cellbackground, colframe=cellborder]
\prompt{In}{incolor}{4}{\boxspacing}
\begin{Verbatim}[commandchars=\\\{\}]
\PY{n}{eqs} \PY{o}{=} \PY{p}{[}\PY{n}{ee}\PY{o}{.}\PY{n}{coefficient}\PY{p}{(}\PY{n}{mm}\PY{p}{)} \PY{k}{for} \PY{n}{mm} \PY{o+ow}{in} \PY{n}{mon} \PY{k}{for} \PY{n}{ee} \PY{o+ow}{in} \PY{n}{EE}\PY{p}{]}
\end{Verbatim}
\end{tcolorbox}

    In order to obtain the cubics \(F_1\) which have tg1 as eigenline, we
reduce \(F\) w.r.t.~the ideal \((eqs)\).

    \begin{tcolorbox}[breakable, size=fbox, boxrule=1pt, pad at break*=1mm,colback=cellbackground, colframe=cellborder]
\prompt{In}{incolor}{5}{\boxspacing}
\begin{Verbatim}[commandchars=\\\{\}]
\PY{n}{F1} \PY{o}{=} \PY{n}{S}\PY{o}{.}\PY{n}{ideal}\PY{p}{(}\PY{n}{eqs}\PY{p}{)}\PY{o}{.}\PY{n}{reduce}\PY{p}{(}\PY{n}{F}\PY{p}{)}
\end{Verbatim}
\end{tcolorbox}

    \(F_1\) has the variables \(x, y, z, a_2, a_3, a_6, a_9\), hence is a
linear space of dimension 3 in \(\mathbb{P}^9\) generated by four cubics
which are the polynomials \(G_1, G_2, G_3, G_4\) below:

    \begin{tcolorbox}[breakable, size=fbox, boxrule=1pt, pad at break*=1mm,colback=cellbackground, colframe=cellborder]
\prompt{In}{incolor}{6}{\boxspacing}
\begin{Verbatim}[commandchars=\\\{\}]
\PY{k}{assert}\PY{p}{(}\PY{n}{F1}\PY{o}{.}\PY{n}{variables}\PY{p}{(}\PY{p}{)} \PY{o}{==} \PY{p}{(}\PY{n}{x}\PY{p}{,} \PY{n}{y}\PY{p}{,} \PY{n}{z}\PY{p}{,} \PY{n}{a2}\PY{p}{,} \PY{n}{a3}\PY{p}{,} \PY{n}{a6}\PY{p}{,} \PY{n}{a9}\PY{p}{)}\PY{p}{)}

\PY{n}{G1}\PY{p}{,} \PY{n}{G2}\PY{p}{,} \PY{n}{G3}\PY{p}{,} \PY{n}{G4} \PY{o}{=} \PY{p}{(}
    \PY{n}{F1}\PY{o}{.}\PY{n}{coefficient}\PY{p}{(}\PY{n}{a2}\PY{p}{)}\PY{p}{,} \PY{n}{F1}\PY{o}{.}\PY{n}{coefficient}\PY{p}{(}\PY{n}{a3}\PY{p}{)}\PY{p}{,}
    \PY{n}{F1}\PY{o}{.}\PY{n}{coefficient}\PY{p}{(}\PY{n}{a6}\PY{p}{)}\PY{p}{,} \PY{n}{F1}\PY{o}{.}\PY{n}{coefficient}\PY{p}{(}\PY{n}{a9}\PY{p}{)}
\PY{p}{)}
\end{Verbatim}
\end{tcolorbox}

    We define four other cubics which are \(H_1\), \(H_2\), \(H_3\) as
follows:

    \begin{tcolorbox}[breakable, size=fbox, boxrule=1pt, pad at break*=1mm,colback=cellbackground, colframe=cellborder]
\prompt{In}{incolor}{7}{\boxspacing}
\begin{Verbatim}[commandchars=\\\{\}]
\PY{n}{H1}\PY{p}{,} \PY{n}{H2}\PY{p}{,} \PY{n}{H3} \PY{o}{=} \PY{n}{x}\PY{o}{*}\PY{n}{tg1}\PY{o}{\PYZca{}}\PY{l+m+mi}{2}\PY{p}{,} \PY{n}{y}\PY{o}{*}\PY{n}{tg1}\PY{o}{\PYZca{}}\PY{l+m+mi}{2}\PY{p}{,} \PY{n}{z}\PY{o}{*}\PY{n}{tg1}\PY{o}{\PYZca{}}\PY{l+m+mi}{2}
\end{Verbatim}
\end{tcolorbox}

    and \(H_4\), constructed in this way: we take the generic line \(r\)
passing through the point \(P_1\) , (which is \(l_1tg1+l_2z\)) and
\(H_4\) is the corresponding cubic whose eigenpoints are the two tangent
lines to Ciso in the points \(r \cap Ciso\), which is given by \(C(r)\),
i.e. the formula \((r^2-3(a^2+b^2+c^2)Ciso)r\) (where
\(a=l_1, b= il_1, c=il_2\))

    \begin{tcolorbox}[breakable, size=fbox, boxrule=1pt, pad at break*=1mm,colback=cellbackground, colframe=cellborder]
\prompt{In}{incolor}{8}{\boxspacing}
\begin{Verbatim}[commandchars=\\\{\}]
\PY{n}{r} \PY{o}{=} \PY{n}{l1}\PY{o}{*}\PY{n}{tg1}\PY{o}{+}\PY{n}{l2}\PY{o}{*}\PY{n}{z}
\PY{n}{H4} \PY{o}{=} \PY{n}{r}\PY{o}{*}\PY{p}{(}\PY{n}{r}\PY{o}{\PYZca{}}\PY{l+m+mi}{2}\PY{o}{\PYZhy{}}\PY{l+m+mi}{3}\PY{o}{*}\PY{p}{(}\PY{n}{r}\PY{o}{.}\PY{n}{coefficient}\PY{p}{(}\PY{n}{x}\PY{p}{)}\PY{o}{\PYZca{}}\PY{l+m+mi}{2}\PY{o}{+}\PY{n}{r}\PY{o}{.}\PY{n}{coefficient}\PY{p}{(}\PY{n}{y}\PY{p}{)}\PY{o}{\PYZca{}}\PY{l+m+mi}{2}\PY{o}{+}\PY{n}{r}\PY{o}{.}\PY{n}{coefficient}\PY{p}{(}\PY{n}{z}\PY{p}{)}\PY{o}{\PYZca{}}\PY{l+m+mi}{2}\PY{p}{)}\PY{o}{*}\PY{n}{Ciso}\PY{p}{)}
\end{Verbatim}
\end{tcolorbox}

    We verify that \(H_1, H_2, H_3, H_4\) are 4 linearly independent cubics,
i.e.~we see that the rank of the matrix \(N_1\) below is always 4
(unless \(l_2 = 0\), i.e.~unless tg1 = r)

    \begin{tcolorbox}[breakable, size=fbox, boxrule=1pt, pad at break*=1mm,colback=cellbackground, colframe=cellborder]
\prompt{In}{incolor}{9}{\boxspacing}
\begin{Verbatim}[commandchars=\\\{\}]
\PY{n}{N1} \PY{o}{=} \PY{n}{matrix}\PY{p}{(}\PY{p}{[}\PY{p}{[}\PY{n}{hh}\PY{o}{.}\PY{n}{coefficient}\PY{p}{(}\PY{n}{mn}\PY{p}{)} \PY{k}{for} \PY{n}{mn} \PY{o+ow}{in} \PY{n}{mon}\PY{p}{]} \PY{k}{for} \PY{n}{hh} \PY{o+ow}{in} \PY{p}{[}\PY{n}{H1}\PY{p}{,} \PY{n}{H2}\PY{p}{,} \PY{n}{H3}\PY{p}{,} \PY{n}{H4}\PY{p}{]}\PY{p}{]}\PY{p}{)}
\PY{n}{JN1} \PY{o}{=} \PY{n}{S}\PY{o}{.}\PY{n}{ideal}\PY{p}{(}\PY{n}{N1}\PY{o}{.}\PY{n}{minors}\PY{p}{(}\PY{l+m+mi}{4}\PY{p}{)}\PY{p}{)}
\PY{k}{assert}\PY{p}{(}\PY{n}{JN1}\PY{o}{.}\PY{n}{radical}\PY{p}{(}\PY{p}{)} \PY{o}{==} \PY{n}{S}\PY{o}{.}\PY{n}{ideal}\PY{p}{(}\PY{n}{l2}\PY{p}{)}\PY{p}{)} \PY{c+c1}{\PYZsh{}\PYZsh{} N1 has always rank 4}
\end{Verbatim}
\end{tcolorbox}

    We verify that the linear space generated by \(G_1, G_2, G_3, G_4\)
coincides with the linear space generated by \(H_1, H_2, H_3, H_4\)

    \begin{tcolorbox}[breakable, size=fbox, boxrule=1pt, pad at break*=1mm,colback=cellbackground, colframe=cellborder]
\prompt{In}{incolor}{10}{\boxspacing}
\begin{Verbatim}[commandchars=\\\{\}]
\PY{k}{for} \PY{n}{hh} \PY{o+ow}{in} \PY{p}{[}\PY{n}{H1}\PY{p}{,} \PY{n}{H2}\PY{p}{,} \PY{n}{H3}\PY{p}{,} \PY{n}{H4}\PY{p}{]}\PY{p}{:}
    \PY{k}{assert}\PY{p}{(}
        \PY{n}{matrix}\PY{p}{(}
            \PY{p}{[}
                \PY{p}{[}\PY{n}{gg}\PY{o}{.}\PY{n}{coefficient}\PY{p}{(}\PY{n}{mn}\PY{p}{)} \PY{k}{for} \PY{n}{mn} \PY{o+ow}{in} \PY{n}{mon}\PY{p}{]} \PY{k}{for} \PY{n}{gg} \PY{o+ow}{in} \PY{p}{[}\PY{n}{G1}\PY{p}{,} \PY{n}{G2}\PY{p}{,} \PY{n}{G3}\PY{p}{,} \PY{n}{G4}\PY{p}{]} \PY{o}{+} \PY{p}{[}\PY{n}{hh}\PY{o}{.}\PY{n}{coefficient}\PY{p}{(}\PY{n}{mn}\PY{p}{)} \PY{k}{for} \PY{n}{mn} \PY{o+ow}{in} \PY{n}{mon}\PY{p}{]}
            \PY{p}{]}
        \PY{p}{)}\PY{o}{.}\PY{n}{rank}\PY{p}{(}\PY{p}{)} \PY{o}{==} \PY{l+m+mi}{4}
    \PY{p}{)}

\PY{k}{for} \PY{n}{gg} \PY{o+ow}{in} \PY{p}{[}\PY{n}{G1}\PY{p}{,} \PY{n}{G2}\PY{p}{,} \PY{n}{G3}\PY{p}{,} \PY{n}{G4}\PY{p}{]}\PY{p}{:}
    \PY{k}{assert}\PY{p}{(}
        \PY{n}{matrix}\PY{p}{(}
            \PY{p}{[}
                \PY{p}{[}\PY{n}{hh}\PY{o}{.}\PY{n}{coefficient}\PY{p}{(}\PY{n}{mn}\PY{p}{)} \PY{k}{for} \PY{n}{mn} \PY{o+ow}{in} \PY{n}{mon}\PY{p}{]} \PY{k}{for} \PY{n}{hh} \PY{o+ow}{in} \PY{p}{[}\PY{n}{H1}\PY{p}{,} \PY{n}{H2}\PY{p}{,} \PY{n}{H3}\PY{p}{,} \PY{n}{H4}\PY{p}{]} \PY{o}{+} \PY{p}{[}\PY{n}{gg}\PY{o}{.}\PY{n}{coefficient}\PY{p}{(}\PY{n}{mn}\PY{p}{)} \PY{k}{for} \PY{n}{mn} \PY{o+ow}{in} \PY{n}{mon}\PY{p}{]}
            \PY{p}{]}
        \PY{p}{)}\PY{o}{.}\PY{n}{rank}\PY{p}{(}\PY{p}{)} \PY{o}{==} \PY{l+m+mi}{4}
    \PY{p}{)}
\end{Verbatim}
\end{tcolorbox}

    \hypertarget{all-the-cubics-f_1-are-singular-in-p_1-and-that-tg1-is-tangent-to-f_1-in-p_1.}{%
\subsection{\texorpdfstring{All the cubics \(F_1\) are singular in
\(P_1\) and that tg1 is tangent to \(F_1\) in
\(P_1\).}{All the cubics F\_1 are singular in P\_1 and that tg1 is tangent to F\_1 in P\_1.}}\label{all-the-cubics-f_1-are-singular-in-p_1-and-that-tg1-is-tangent-to-f_1-in-p_1.}}

\hypertarget{we-assume-a_9-not-zero-since-if-a_90-f_1-is-reducible-of-the-form-t2ell-where-t-is-the-line-tg1}{%
\subsection{\texorpdfstring{We assume \(a_9\) not zero, since if
\(a_9=0\), \(F_1\) is reducible of the form \(t^2\ell\), where \(t\) is
the line
tg1}{We assume a\_9 not zero, since if a\_9=0, F\_1 is reducible of the form t\^{}2\textbackslash ell, where t is the line tg1}}\label{we-assume-a_9-not-zero-since-if-a_90-f_1-is-reducible-of-the-form-t2ell-where-t-is-the-line-tg1}}

    \begin{tcolorbox}[breakable, size=fbox, boxrule=1pt, pad at break*=1mm,colback=cellbackground, colframe=cellborder]
\prompt{In}{incolor}{11}{\boxspacing}
\begin{Verbatim}[commandchars=\\\{\}]
\PY{k}{assert}\PY{p}{(}\PY{n}{F1}\PY{o}{.}\PY{n}{subs}\PY{p}{(}\PY{n}{a9}\PY{o}{=}\PY{l+m+mi}{0}\PY{p}{)}\PY{o}{.}\PY{n}{quo\PYZus{}rem}\PY{p}{(}\PY{p}{(}\PY{n}{x}\PY{o}{+}\PY{n}{ii}\PY{o}{*}\PY{n}{y}\PY{p}{)}\PY{o}{\PYZca{}}\PY{l+m+mi}{2}\PY{p}{)}\PY{p}{[}\PY{l+m+mi}{1}\PY{p}{]} \PY{o}{==} \PY{l+m+mi}{0}\PY{p}{)}  \PY{c+c1}{\PYZsh{}\PYZsh{} (x+ii*y)\PYZca{}2 is a factor of the cubic}
\PY{k}{assert}\PY{p}{(}\PY{n}{gdn}\PY{p}{(}\PY{n}{F1}\PY{p}{)}\PY{o}{.}\PY{n}{subs}\PY{p}{(}\PY{n}{substitution}\PY{p}{(}\PY{n}{P1}\PY{p}{)}\PY{p}{)} \PY{o}{==} \PY{n}{vector}\PY{p}{(}\PY{n}{S}\PY{p}{,} \PY{p}{(}\PY{l+m+mi}{0}\PY{p}{,}\PY{l+m+mi}{0}\PY{p}{,}\PY{l+m+mi}{0}\PY{p}{)}\PY{p}{)}\PY{p}{)}  \PY{c+c1}{\PYZsh{}\PYZsh{} P1 is singular}
\PY{k}{assert}\PY{p}{(}\PY{n}{S}\PY{o}{.}\PY{n}{ideal}\PY{p}{(}\PY{n}{F1}\PY{p}{,} \PY{n}{tg1}\PY{p}{)}\PY{o}{.}\PY{n}{saturation}\PY{p}{(}\PY{n}{a9}\PY{p}{)}\PY{p}{[}\PY{l+m+mi}{0}\PY{p}{]} \PY{o}{==} \PY{n}{S}\PY{o}{.}\PY{n}{ideal}\PY{p}{(}\PY{n}{tg1}\PY{p}{,} \PY{n}{z}\PY{o}{\PYZca{}}\PY{l+m+mi}{3}\PY{p}{)}\PY{p}{)}  \PY{c+c1}{\PYZsh{}\PYZsh{} P1 is a triple point}
\end{Verbatim}
\end{tcolorbox}

    In order to describe \(F_1\), we first determine when \(F_1\) splits
into a line and a conic. The line necessarily passes through \(P_1\),
hence is the line \(r\) above. A generic point of \(r\) is
\((u_1l_2, u_2l_2, -l_1(u_1+iu_2)\), hence we define the point \(p_3\):

    \begin{tcolorbox}[breakable, size=fbox, boxrule=1pt, pad at break*=1mm,colback=cellbackground, colframe=cellborder]
\prompt{In}{incolor}{12}{\boxspacing}
\begin{Verbatim}[commandchars=\\\{\}]
\PY{n}{p3} \PY{o}{=} \PY{n}{vector}\PY{p}{(}\PY{n}{S}\PY{p}{,} \PY{p}{(}\PY{n}{u1}\PY{o}{*}\PY{n}{l2}\PY{p}{,} \PY{n}{u2}\PY{o}{*}\PY{n}{l2}\PY{p}{,} \PY{o}{\PYZhy{}}\PY{n}{l1}\PY{o}{*}\PY{p}{(}\PY{n}{u1}\PY{o}{+}\PY{n}{ii}\PY{o}{*}\PY{n}{u2}\PY{p}{)}\PY{p}{)}\PY{p}{)}
\PY{k}{assert}\PY{p}{(}\PY{n}{r}\PY{o}{.}\PY{n}{subs}\PY{p}{(}\PY{n}{substitution}\PY{p}{(}\PY{n}{p3}\PY{p}{)}\PY{p}{)} \PY{o}{==} \PY{l+m+mi}{0}\PY{p}{)}
\end{Verbatim}
\end{tcolorbox}

    Now we impose that, for all \(u_1\) and \(u_2\), the point \(p_3\) is a
zero of \(F_1\). With this substitution \(F_1\) splits into
\((u_1+iu_2)^2\) and another factor, linear in \(u_1\) and \(u_2\) and
we want that for all \(u_1\) and \(u_2\) this factor is zero:

    \begin{tcolorbox}[breakable, size=fbox, boxrule=1pt, pad at break*=1mm,colback=cellbackground, colframe=cellborder]
\prompt{In}{incolor}{13}{\boxspacing}
\begin{Verbatim}[commandchars=\\\{\}]
\PY{n}{ff1} \PY{o}{=} \PY{n}{F1}\PY{o}{.}\PY{n}{subs}\PY{p}{(}\PY{n}{substitution}\PY{p}{(}\PY{n}{p3}\PY{p}{)}\PY{p}{)}
\PY{k}{assert}\PY{p}{(}\PY{n}{ff1}\PY{o}{.}\PY{n}{quo\PYZus{}rem}\PY{p}{(}\PY{p}{(}\PY{n}{u1}\PY{o}{+}\PY{n}{ii}\PY{o}{*}\PY{n}{u2}\PY{p}{)}\PY{o}{\PYZca{}}\PY{l+m+mi}{2}\PY{p}{)}\PY{p}{[}\PY{l+m+mi}{1}\PY{p}{]} \PY{o}{==} \PY{l+m+mi}{0}\PY{p}{)}
\PY{n}{ff2} \PY{o}{=} \PY{n}{ff1}\PY{o}{.}\PY{n}{quo\PYZus{}rem}\PY{p}{(}\PY{p}{(}\PY{n}{u1}\PY{o}{+}\PY{n}{ii}\PY{o}{*}\PY{n}{u2}\PY{p}{)}\PY{o}{\PYZca{}}\PY{l+m+mi}{2}\PY{p}{)}\PY{p}{[}\PY{l+m+mi}{0}\PY{p}{]}
\PY{k}{assert}\PY{p}{(}\PY{n}{ff2}\PY{o}{.}\PY{n}{degree}\PY{p}{(}\PY{n}{u1}\PY{p}{)} \PY{o}{==}\PY{l+m+mi}{1}\PY{p}{)}
\PY{k}{assert}\PY{p}{(}\PY{n}{ff2}\PY{o}{.}\PY{n}{degree}\PY{p}{(}\PY{n}{u2}\PY{p}{)} \PY{o}{==} \PY{l+m+mi}{1}\PY{p}{)}
\PY{n}{jf1} \PY{o}{=} \PY{n}{S}\PY{o}{.}\PY{n}{ideal}\PY{p}{(}\PY{n}{ff2}\PY{o}{.}\PY{n}{coefficient}\PY{p}{(}\PY{n}{u1}\PY{p}{)}\PY{p}{,} \PY{n}{ff2}\PY{o}{.}\PY{n}{coefficient}\PY{p}{(}\PY{n}{u2}\PY{p}{)}\PY{p}{)}
\end{Verbatim}
\end{tcolorbox}

    The ideal jf1 is the ideal of the conditions on the parameters of
\(F_1\) (which are \(a_2, a_3, a_6, a_9\)) in order to have that
\(V(F_1)\) splits into the line \(r\) and a conic. We can saturate it
w.r.t.~\(a_9\) and \(l_2\). We get an ideal jf2 which is generated by
the three polynomials \(h_1, h_2, h_3\) below:

    \begin{tcolorbox}[breakable, size=fbox, boxrule=1pt, pad at break*=1mm,colback=cellbackground, colframe=cellborder]
\prompt{In}{incolor}{14}{\boxspacing}
\begin{Verbatim}[commandchars=\\\{\}]
\PY{n}{jf2} \PY{o}{=} \PY{n}{jf1}\PY{o}{.}\PY{n}{saturation}\PY{p}{(}\PY{n}{a9}\PY{o}{*}\PY{n}{l2}\PY{p}{)}\PY{p}{[}\PY{l+m+mi}{0}\PY{p}{]}
\PY{n}{h1} \PY{o}{=} \PY{n}{l2}\PY{o}{*}\PY{n}{a2} \PY{o}{+} \PY{p}{(}\PY{l+m+mi}{3}\PY{o}{*}\PY{n}{ii}\PY{p}{)}\PY{o}{*}\PY{n}{l2}\PY{o}{*}\PY{n}{a3} \PY{o}{+} \PY{l+m+mi}{3}\PY{o}{*}\PY{n}{l1}\PY{o}{*}\PY{n}{a9}
\PY{n}{h2} \PY{o}{=} \PY{n}{l1}\PY{o}{*}\PY{n}{a2}\PY{o}{\PYZca{}}\PY{l+m+mi}{2} \PY{o}{+} \PY{p}{(}\PY{l+m+mi}{6}\PY{o}{*}\PY{n}{ii}\PY{p}{)}\PY{o}{*}\PY{n}{l1}\PY{o}{*}\PY{n}{a2}\PY{o}{*}\PY{n}{a3} \PY{o}{\PYZhy{}} \PY{l+m+mi}{9}\PY{o}{*}\PY{n}{l1}\PY{o}{*}\PY{n}{a3}\PY{o}{\PYZca{}}\PY{l+m+mi}{2} \PY{o}{+} \PY{p}{(}\PY{o}{\PYZhy{}}\PY{l+m+mi}{9}\PY{o}{*}\PY{n}{ii}\PY{p}{)}\PY{o}{*}\PY{n}{l2}\PY{o}{*}\PY{n}{a3}\PY{o}{*}\PY{n}{a9} \PY{o}{\PYZhy{}} \PY{l+m+mi}{9}\PY{o}{*}\PY{n}{l1}\PY{o}{*}\PY{n}{a6}\PY{o}{*}\PY{n}{a9}
\PY{n}{h3} \PY{o}{=} \PY{n}{a2}\PY{o}{\PYZca{}}\PY{l+m+mi}{3} \PY{o}{+} \PY{p}{(}\PY{l+m+mi}{9}\PY{o}{*}\PY{n}{ii}\PY{p}{)}\PY{o}{*}\PY{n}{a2}\PY{o}{\PYZca{}}\PY{l+m+mi}{2}\PY{o}{*}\PY{n}{a3} \PY{o}{\PYZhy{}} \PY{l+m+mi}{27}\PY{o}{*}\PY{n}{a2}\PY{o}{*}\PY{n}{a3}\PY{o}{\PYZca{}}\PY{l+m+mi}{2} \PY{o}{+} \PY{p}{(}\PY{o}{\PYZhy{}}\PY{l+m+mi}{27}\PY{o}{*}\PY{n}{ii}\PY{p}{)}\PY{o}{*}\PY{n}{a3}\PY{o}{\PYZca{}}\PY{l+m+mi}{3} \PY{o}{\PYZhy{}} \PY{l+m+mi}{9}\PY{o}{*}\PY{n}{a2}\PY{o}{*}\PY{n}{a6}\PY{o}{*}\PY{n}{a9} \PY{o}{+} \PY{p}{(}\PY{o}{\PYZhy{}}\PY{l+m+mi}{27}\PY{o}{*}\PY{n}{ii}\PY{p}{)}\PY{o}{*}\PY{n}{a3}\PY{o}{*}\PY{n}{a6}\PY{o}{*}\PY{n}{a9} \PY{o}{+} \PY{p}{(}\PY{l+m+mi}{27}\PY{o}{*}\PY{n}{ii}\PY{p}{)}\PY{o}{*}\PY{n}{a3}\PY{o}{*}\PY{n}{a9}\PY{o}{\PYZca{}}\PY{l+m+mi}{2}
\PY{k}{assert}\PY{p}{(}\PY{n}{S}\PY{o}{.}\PY{n}{ideal}\PY{p}{(}\PY{n}{h1}\PY{p}{,} \PY{n}{h2}\PY{p}{,} \PY{n}{h3}\PY{p}{)}\PY{o}{.}\PY{n}{saturation}\PY{p}{(}\PY{n}{a9}\PY{p}{)}\PY{p}{[}\PY{l+m+mi}{0}\PY{p}{]} \PY{o}{==} \PY{n}{jf2}\PY{p}{)}
\end{Verbatim}
\end{tcolorbox}

    The meaning of the above computation is the following: \(V(F_1)\)
contains a line \(r\) iff \(a_2, a_3, a_6, a_9\) satisfy the condition
\(h_3=0\). (The values of \(l_1\) and \(l_2\) which determine the line
\(r\) can be computed from the linear system obtained from \(h_1\) and
\(h_2\)). Hence:

\(F_1\) splits into the product of a linear factor and a quadratic
factor iff \(h_3(a_2, a_3, a_6, a_9) = 0\).

Now a similar computation: we want to see when a line passing thorugh
\(P_1\) is tangent to the cubic \(V(F_1)\) in \(P_1\). Again, the line
is the line \(r\) above, we consider the ideal generated by \(F_1\) and
\(r\) and we study it. We compute its primary decomposition and we get
two ideals: the radical of the first is the point \(P_1\), the second is
generated by two poynomials \(g_1\) and \(g_2\) linear in \(x, y, z\)
(up to an admissible saturation) (\(g_1\) is \(r\)) and \(g_1, g_2\)
give therefore a further point of intersectrion of \(r\) with the cubic
\(V(F_1)\):

    \begin{tcolorbox}[breakable, size=fbox, boxrule=1pt, pad at break*=1mm,colback=cellbackground, colframe=cellborder]
\prompt{In}{incolor}{15}{\boxspacing}
\begin{Verbatim}[commandchars=\\\{\}]
\PY{n}{J1} \PY{o}{=} \PY{n}{S}\PY{o}{.}\PY{n}{ideal}\PY{p}{(}\PY{n}{F1}\PY{p}{,} \PY{n}{r}\PY{p}{)}
\PY{n}{pd} \PY{o}{=} \PY{n}{J1}\PY{o}{.}\PY{n}{primary\PYZus{}decomposition}\PY{p}{(}\PY{p}{)}
\PY{k}{assert}\PY{p}{(}\PY{n+nb}{len}\PY{p}{(}\PY{n}{pd}\PY{p}{)} \PY{o}{==} \PY{l+m+mi}{2}\PY{p}{)}
\PY{k}{assert}\PY{p}{(}\PY{n}{pd}\PY{p}{[}\PY{l+m+mi}{0}\PY{p}{]}\PY{o}{.}\PY{n}{radical}\PY{p}{(}\PY{p}{)} \PY{o}{==} \PY{n}{S}\PY{o}{.}\PY{n}{ideal}\PY{p}{(}\PY{n}{x}\PY{o}{+}\PY{n}{ii}\PY{o}{*}\PY{n}{y}\PY{p}{,} \PY{n}{z}\PY{p}{)}\PY{p}{)}
\PY{n}{g1} \PY{o}{=} \PY{n}{pd}\PY{p}{[}\PY{l+m+mi}{1}\PY{p}{]}\PY{o}{.}\PY{n}{gens}\PY{p}{(}\PY{p}{)}\PY{p}{[}\PY{l+m+mi}{0}\PY{p}{]}
\PY{n}{g2} \PY{o}{=} \PY{n}{pd}\PY{p}{[}\PY{l+m+mi}{1}\PY{p}{]}\PY{o}{.}\PY{n}{gens}\PY{p}{(}\PY{p}{)}\PY{p}{[}\PY{l+m+mi}{1}\PY{p}{]}
\PY{k}{assert}\PY{p}{(}\PY{n}{g1} \PY{o}{==} \PY{n}{r}\PY{p}{)}
\PY{k}{assert}\PY{p}{(}\PY{n}{g1}\PY{o}{.}\PY{n}{degree}\PY{p}{(}\PY{n}{x}\PY{p}{)} \PY{o}{==} \PY{l+m+mi}{1}\PY{p}{)}
\PY{k}{assert}\PY{p}{(}\PY{n}{g1}\PY{o}{.}\PY{n}{degree}\PY{p}{(}\PY{n}{y}\PY{p}{)} \PY{o}{==} \PY{l+m+mi}{1}\PY{p}{)}
\PY{k}{assert}\PY{p}{(}\PY{n}{g1}\PY{o}{.}\PY{n}{degree}\PY{p}{(}\PY{n}{z}\PY{p}{)} \PY{o}{==} \PY{l+m+mi}{1}\PY{p}{)}
\PY{k}{assert}\PY{p}{(}\PY{n}{g2}\PY{o}{.}\PY{n}{degree}\PY{p}{(}\PY{n}{x}\PY{p}{)} \PY{o}{==} \PY{l+m+mi}{1}\PY{p}{)}
\PY{k}{assert}\PY{p}{(}\PY{n}{g2}\PY{o}{.}\PY{n}{degree}\PY{p}{(}\PY{n}{y}\PY{p}{)} \PY{o}{==} \PY{l+m+mi}{1}\PY{p}{)}
\PY{k}{assert}\PY{p}{(}\PY{n}{g2}\PY{o}{.}\PY{n}{degree}\PY{p}{(}\PY{n}{z}\PY{p}{)} \PY{o}{==} \PY{l+m+mi}{1}\PY{p}{)}
\PY{k}{assert}\PY{p}{(}\PY{n}{pd}\PY{p}{[}\PY{l+m+mi}{1}\PY{p}{]} \PY{o}{==} \PY{n}{S}\PY{o}{.}\PY{n}{ideal}\PY{p}{(}\PY{n}{g1}\PY{p}{,} \PY{n}{g2}\PY{p}{)}\PY{o}{.}\PY{n}{saturation}\PY{p}{(}\PY{n}{l2}\PY{p}{)}\PY{p}{[}\PY{l+m+mi}{0}\PY{p}{]}\PY{p}{)}
\end{Verbatim}
\end{tcolorbox}

    At this point we know that the third intersection of the line \(r\) with
\(V(F_1)\) is given by the linear system \(g_1=0, g_2=0\). We want that
this point (for suitable values of \(l_1\) and \(l_2\)) is again the
point \(P_1\). In this way we determine when \(r\) is tangent to
\(V(F_1)\). Since \(g_1\) is \(r\), it always contain \(P_1\). The only
condition that has to be verified is that \(P_1\) annihilates \(g_2\).
This condition is \(f = l_2a_2+3il_2a_3+3l_1a_9 = 0\):

    \begin{tcolorbox}[breakable, size=fbox, boxrule=1pt, pad at break*=1mm,colback=cellbackground, colframe=cellborder]
\prompt{In}{incolor}{16}{\boxspacing}
\begin{Verbatim}[commandchars=\\\{\}]
\PY{k}{assert}\PY{p}{(}\PY{n}{g1}\PY{o}{.}\PY{n}{subs}\PY{p}{(}\PY{n}{substitution}\PY{p}{(}\PY{n}{P1}\PY{p}{)}\PY{p}{)} \PY{o}{==} \PY{l+m+mi}{0}\PY{p}{)}
\PY{n}{f} \PY{o}{=} \PY{n}{l2}\PY{o}{*}\PY{n}{a2} \PY{o}{+} \PY{p}{(}\PY{l+m+mi}{3}\PY{o}{*}\PY{n}{ii}\PY{p}{)}\PY{o}{*}\PY{n}{l2}\PY{o}{*}\PY{n}{a3} \PY{o}{+} \PY{l+m+mi}{3}\PY{o}{*}\PY{n}{l1}\PY{o}{*}\PY{n}{a9}
\PY{k}{assert}\PY{p}{(}\PY{n}{g2}\PY{o}{.}\PY{n}{subs}\PY{p}{(}\PY{n}{substitution}\PY{p}{(}\PY{n}{P1}\PY{p}{)}\PY{p}{)} \PY{o}{==} \PY{n}{l2} \PY{o}{*} \PY{n}{f}\PY{p}{)}
\end{Verbatim}
\end{tcolorbox}

    Hence we obtain that \(r\) is tangent in \(P_1\) to \(V(F_1)\) when
\(l_1, l_2\) are such that \(f=0\). The tangent is \(r_2\) (we do not
get tg1, because we exclude the case \(l_2=0\)):

    \begin{tcolorbox}[breakable, size=fbox, boxrule=1pt, pad at break*=1mm,colback=cellbackground, colframe=cellborder]
\prompt{In}{incolor}{17}{\boxspacing}
\begin{Verbatim}[commandchars=\\\{\}]
\PY{n}{r2} \PY{o}{=} \PY{n}{r}\PY{o}{.}\PY{n}{subs}\PY{p}{(}\PY{p}{\PYZob{}}\PY{n}{l1}\PY{p}{:}\PY{n}{f}\PY{o}{.}\PY{n}{coefficient}\PY{p}{(}\PY{n}{l2}\PY{p}{)}\PY{p}{,} \PY{n}{l2}\PY{p}{:} \PY{o}{\PYZhy{}}\PY{n}{f}\PY{o}{.}\PY{n}{coefficient}\PY{p}{(}\PY{n}{l1}\PY{p}{)}\PY{p}{\PYZcb{}}\PY{p}{)}
\end{Verbatim}
\end{tcolorbox}

    In general, the line \(r_2\) is the second tangent in \(P_1\) to the
cubic. To verify this, we study the ideal generated by \(F_1\) and
\(r_2\). Its primary decomposition (after saturation w.r.t. \(a_9\)) has
two components, the first is the triple point \(P_1\), the second
contains the polynomial \(h_3\), which means that it appears only when
\(F_1\) is reducible.

    \begin{tcolorbox}[breakable, size=fbox, boxrule=1pt, pad at break*=1mm,colback=cellbackground, colframe=cellborder]
\prompt{In}{incolor}{18}{\boxspacing}
\begin{Verbatim}[commandchars=\\\{\}]
\PY{n}{pd2} \PY{o}{=} \PY{n}{S}\PY{o}{.}\PY{n}{ideal}\PY{p}{(}\PY{n}{F1}\PY{p}{,} \PY{n}{r2}\PY{p}{)}\PY{o}{.}\PY{n}{saturation}\PY{p}{(}\PY{n}{a9}\PY{p}{)}\PY{p}{[}\PY{l+m+mi}{0}\PY{p}{]}\PY{o}{.}\PY{n}{primary\PYZus{}decomposition}\PY{p}{(}\PY{p}{)}
\PY{k}{assert}\PY{p}{(}\PY{n+nb}{len}\PY{p}{(}\PY{n}{pd2}\PY{p}{)} \PY{o}{==} \PY{l+m+mi}{2}\PY{p}{)}
\PY{k}{assert}\PY{p}{(}\PY{n}{z}\PY{o}{\PYZca{}}\PY{l+m+mi}{3} \PY{o+ow}{in} \PY{n}{pd2}\PY{p}{[}\PY{l+m+mi}{0}\PY{p}{]}\PY{p}{)} 
\PY{k}{assert}\PY{p}{(}\PY{n}{pd2}\PY{p}{[}\PY{l+m+mi}{0}\PY{p}{]}\PY{o}{.}\PY{n}{radical}\PY{p}{(}\PY{p}{)} \PY{o}{==} \PY{n}{S}\PY{o}{.}\PY{n}{ideal}\PY{p}{(}\PY{n}{x}\PY{o}{+}\PY{n}{ii}\PY{o}{*}\PY{n}{y}\PY{p}{,} \PY{n}{z}\PY{p}{)}\PY{p}{)}  \PY{c+c1}{\PYZsh{}\PYZsh{} pd2[0] is the triple point \PYZdl{}P\PYZus{}1\PYZdl{}.}
\PY{k}{assert}\PY{p}{(}\PY{n}{h3} \PY{o+ow}{in} \PY{n}{pd2}\PY{p}{[}\PY{l+m+mi}{1}\PY{p}{]}\PY{p}{)}
\end{Verbatim}
\end{tcolorbox}

    Hence, when \(F_1\) is irreducible, \(V(F_1)\) has two tangents in
\(P_1\), which are: tg1 and r2. We want to see when they are coincident,
i.e.~when \(P_1\) is a cusp. Here we see that this happens iff
\(a_9=0\), i.e.~iff the cubic is splits into \(t^2 \ell\), where \(t\)
is the line tg1. Therefore:

If \(V(F_1)\) is irreducible, it has two distinct tangents tg1 and
\(r_2\), so the point \(P_1\) is never a cusp for the cubic.

    \begin{tcolorbox}[breakable, size=fbox, boxrule=1pt, pad at break*=1mm,colback=cellbackground, colframe=cellborder]
\prompt{In}{incolor}{19}{\boxspacing}
\begin{Verbatim}[commandchars=\\\{\}]
\PY{n}{coinc\PYZus{}l} \PY{o}{=} \PY{n}{S}\PY{o}{.}\PY{n}{ideal}\PY{p}{(}\PY{n}{matrix}\PY{p}{(}\PY{p}{[}\PY{p}{[}\PY{n}{tg1}\PY{o}{.}\PY{n}{coefficient}\PY{p}{(}\PY{n}{vr}\PY{p}{)} \PY{k}{for} \PY{n}{vr} \PY{o+ow}{in} \PY{p}{[}\PY{n}{x}\PY{p}{,} \PY{n}{y}\PY{p}{,} \PY{n}{z}\PY{p}{]}\PY{p}{]}\PY{p}{,} \PY{p}{[}\PY{n}{r2}\PY{o}{.}\PY{n}{coefficient}\PY{p}{(}\PY{n}{vr}\PY{p}{)} \PY{k}{for} \PY{n}{vr} \PY{o+ow}{in} \PY{p}{[}\PY{n}{x}\PY{p}{,} \PY{n}{y}\PY{p}{,} \PY{n}{z}\PY{p}{]}\PY{p}{]}\PY{p}{]}\PY{p}{)}\PY{o}{.}\PY{n}{minors}\PY{p}{(}\PY{l+m+mi}{2}\PY{p}{)}\PY{p}{)}
\PY{k}{assert}\PY{p}{(}\PY{n}{coinc\PYZus{}l}\PY{o}{.}\PY{n}{groebner\PYZus{}basis}\PY{p}{(}\PY{p}{)} \PY{o}{==} \PY{p}{[}\PY{n}{a9}\PY{p}{]}\PY{p}{)}
\PY{k}{assert}\PY{p}{(}\PY{n}{F1}\PY{o}{.}\PY{n}{subs}\PY{p}{(}\PY{n}{a9}\PY{o}{=}\PY{l+m+mi}{0}\PY{p}{)}\PY{o}{.}\PY{n}{quo\PYZus{}rem}\PY{p}{(}\PY{p}{(}\PY{n}{x}\PY{o}{+}\PY{n}{ii}\PY{o}{*}\PY{n}{y}\PY{p}{)}\PY{o}{\PYZca{}}\PY{l+m+mi}{2}\PY{p}{)}\PY{p}{[}\PY{l+m+mi}{1}\PY{p}{]} \PY{o}{==} \PY{l+m+mi}{0}\PY{p}{)}
\end{Verbatim}
\end{tcolorbox}

    When \(V(F_1)\) is reducible, i.e.~when \(h_3=0\), \(F_1\) splits into
the product of \(r_2\) and a conic. We can easily verify this, since
\(h_3\) is linear in \(a_6\), so the equation \(h_3=0\) can be solved
w.r.t. \(a_6\)

    \begin{tcolorbox}[breakable, size=fbox, boxrule=1pt, pad at break*=1mm,colback=cellbackground, colframe=cellborder]
\prompt{In}{incolor}{20}{\boxspacing}
\begin{Verbatim}[commandchars=\\\{\}]
\PY{k}{assert}\PY{p}{(}\PY{n}{h3}\PY{o}{.}\PY{n}{degree}\PY{p}{(}\PY{n}{a6}\PY{p}{)} \PY{o}{==} \PY{l+m+mi}{1}\PY{p}{)}
\PY{n}{st} \PY{o}{=} \PY{p}{\PYZob{}}\PY{n}{a6}\PY{p}{:} \PY{o}{\PYZhy{}}\PY{n}{h3}\PY{o}{.}\PY{n}{subs}\PY{p}{(}\PY{n}{a6}\PY{o}{=}\PY{l+m+mi}{0}\PY{p}{)}\PY{o}{/}\PY{n}{h3}\PY{o}{.}\PY{n}{coefficient}\PY{p}{(}\PY{n}{a6}\PY{p}{)}\PY{p}{\PYZcb{}}
\PY{n}{F2} \PY{o}{=} \PY{n}{S}\PY{p}{(}\PY{n}{numerator}\PY{p}{(}\PY{n}{F1}\PY{o}{.}\PY{n}{subs}\PY{p}{(}\PY{n}{st}\PY{p}{)}\PY{p}{)}\PY{p}{)}
\PY{k}{assert}\PY{p}{(}\PY{n}{F2}\PY{o}{.}\PY{n}{quo\PYZus{}rem}\PY{p}{(}\PY{n}{r2}\PY{p}{)}\PY{p}{[}\PY{l+m+mi}{1}\PY{p}{]} \PY{o}{==} \PY{l+m+mi}{0}\PY{p}{)}
\end{Verbatim}
\end{tcolorbox}


    % Add a bibliography block to the postdoc
    
    
    
\end{document}
