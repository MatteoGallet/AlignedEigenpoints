\documentclass[12pt, a4paper, reqno, captions=tableheading,bibliography=totoc]{scrartcl}

\usepackage[utf8]{inputenc}
\usepackage[T1]{fontenc}
\usepackage{mathtools}
\usepackage{amsthm}
\usepackage{amsmath}
\usepackage{amssymb}
\usepackage{lmodern}
\usepackage[english]{babel}
\usepackage{booktabs}
\usepackage{float}
\usepackage{enumitem}
\usepackage{tikz,pgf}
%\usepackage[utopia]{mathdesign}
\usepackage{palatino}
\usepackage{hyperref}
\hypersetup{
    colorlinks,
    linkcolor={red!50!black},
    citecolor={blue!50!black},
    urlcolor={blue!80!black}
}
\usepackage[nameinlink]{cleveref}

\theoremstyle{plain}
\newtheorem{lemma}{Lemma}[section]
\newtheorem{prop}[lemma]{Proposition}
\newtheorem{theorem}[lemma]{Theorem}
\newtheorem{corollary}[lemma]{Corollary}
\newtheorem{conjecture}[lemma]{Conjecture}
\newtheorem{fact}[lemma]{Fact}
\newtheorem{assumption}[lemma]{Assumption}
\newtheorem*{reduction}{Reduction}
\theoremstyle{definition}
\newtheorem{definition}[lemma]{Definition}
\newtheorem{es}[lemma]{Example}
\newtheorem*{notation}{Notation}
\newtheorem{rmk}[lemma]{Remark}


\newcommand{\N}{\mathbb{N}}
\newcommand{\Z}{\mathbb{Z}}
\newcommand{\Q}{\mathbb{Q}}
\newcommand{\R}{\mathbb{R}}
\newcommand{\C}{\mathbb{C}}
\newcommand{\p}{\mathbb{P}}
\newcommand{\sP}{\mathcal{P}}
\newcommand{\sL}{\mathcal{L}}
\newcommand{\de}{\partial}
\newcommand{\codim}{\mathrm{codim}}

\newcommand{\oo}{\mathcal{O}}
\newcommand{\Bl}{\mathrm{Bl}}

\newcommand{\iso}{\mathcal{Q}}

\newcommand{\imunit}{i}

\newcommand{\SO}{\operatorname{SO}}
\newcommand{\Eig}[1]{\operatorname{Eig}\left( {#1} \right)}
\newcommand{\polq}{{\rm Pol}_Q}
\newcommand{\comment}[1]{}

\newcommand{\scl}[2]{\left\langle {#1}, {#2} \right\rangle}

\newcommand\scalemath[2]{\scalebox{#1}{\mbox{\ensuremath{\displaystyle #2}}}}

\linespread{1.2}
\setlength{\parindent}{0pt}
\setlength{\parskip}{.25em}

\begin{document}



Given $P_1, \dots, P_5$ as usual, we want to see when
the matrix $\Phi(P_1, \dots, P_5)$ has rank $\leq 8$.
If $P_1, P_2, P_3$ are three aligned points, we set:
\[
\overline{\delta}_1(P_1, P_2, P_3)  = \scl{P_1}{P_2}\scl{P_1}{P_3}+
\scl{P_1}{P_1}\scl{P_2}{P_3}
\]

The result we get is the following:
\begin{prop}
\label{casoMatrR8}
Let $P_1, \dots, P_5$ be a ``V'' configuration. If we assume that the 
matrix $\Phi(P_1, P_2, P_3, P_4, P_5)$ has rank less then $9$, then we have
one of the following two possibilities:
\begin{enumerate}
\item The line $P_1+P_2$ is tangent to the isotropic conic in one of the
eigenpoints (say $P_2$)
and the line $P_1+P_4$ is tangent to the isotropic conic in one of the
eigenpoints (say $P_4$), 
(hence $\scl{P_1}{P_2} = 0$, $\scl{P_2}{P_2}=0$, $\scl{P_1}{P_4} = 0$
and $\scl{P_4}{P_4}=0$);
\item The five points satisfy:
\[
\delta_1(P_1, P_2, P_4) = 0, \quad \overline{\delta}_1(P_1, P_2, P_3) = 0,
\quad \overline{\delta}_1(P_1, P_4, P_5) = 0.
\]
\end{enumerate}
\end{prop}
In order to prove the proposition, we first assume that $P_1 = (1, ii, 0)$.
In this case we see that from \verb+file rank_8_2_1_ii_0.sage+
we get that there are no possibilities. Indeed, first we note that with some
elementary rows and columns operations, we can simlify the
matrix $\Phi(P_1, \dots, P_5)$ in order to reduce the study of 
the ideal of all order $9$ minors to the problem of order $7$ minors.
Then, we can successively saturate the obtained ideal w.r.t.\ the condition
of not coicidence of the five points and not alignments of $P_1, P_2, P_4$.
Eventually, we get that the ideal reduces to the ideal $(1)$.
see \verb+file rank_8_2_1_ii_0.sage+

Successively, we assume $P_1 = (1, 0, 0)$. Here we assume the other points
are generic, i.e.\ $P_2 = (A_2: B_2: C_2)$, $P_4 = (A_4: B_4: C_4)$
and $P_3=u_1P_1+u_2P_2$, $P_5 = v_1P_1+v_2P_4$.
See \verb+file rank_8_1.sage+
Again, we study the ideal of the
order $9$ minors of $M = \Phi(P_1, \dots, p_5)$ and also in this case
with elementary rows and columns operations we can simplify the problem
considering an ideal of the order $7$ minors of a matrix. Successively
saturations of the this ideal gives several possibilities: either the case
in which the line $P_1+P_2$ and $P_1+P_4$ are tangent in $P_2$ (or $P_3$)
and in $P_4$ (or $P_5$) to the isotropic conic, or we get the ideal generated
by:
\[
B_2B_4 + C_2C_4, v_2(2A_4^2 + B_4^2 + C4^2) + 2v_1A_4, u_2(2A_2^2+B_2^2+C_2^2)
+ 2u_1A_2
\]
and this ideal is therefore generated by
$\scl{P_1}{P_2}\scl{P_1}{P_4}-\scl{P_2}{P_4}$,
$\scl{P_1}{P_2}\scl{P_1}{P_3}+\scl{P_2}{P_3}$,
$\scl{P_1}{P_4}\scl{P_1}{P_5}-\scl{P_4}{P_5}$.
See \verb+file rank_8_1.sage+

Let now, provisionally, $Q_1, \dots, Q_5$ be generic points
(such that $Q_1, Q_2, Q_3$
and $Q_1, Q_4, Q_5$ are aligned. Let $\lambda$ be
such that $Q_1' = \lambda Q_1$ satisfies $\scl{Q_1'}{Q_1'} = 1$.
From what written above, if $M$ is an orthogonal matrix such that
$M \cdot Q_1'$ is the projective point $(1: 0: 0)$, we get that 
\begin{gather*}
\scl{M\cdot Q_1'}{M\cdot Q_2}\scl{M\cdot Q_1'}{M\cdot Q_4}-\scl{M\cdot Q_2}{M\cdot Q_4},\\
\scl{M\cdot Q_1'}{M\cdot Q_2}\scl{M\cdot Q_1'}{M\cdot Q_3}+\scl{M\cdot Q_1'}{M\cdot Q_1'}\scl{M\cdot Q_2}{M\cdot Q_3} = 0, \\
\scl{M\cdot Q_1'}{M\cdot Q_4}\scl{M\cdot Q_1',M\cdot Q_5}+\scl{M\cdot Q_1'}{M\cdot Q_1'}\scl{M\cdot Q_4}{M\cdot Q_5} = 0
\end{gather*}
Since $M$ is orthogonal, we have that $\scl{M\cdot P}{M\cdot Q} = \scl{P}{Q}$,
hence we get:
\begin{gather*}
\scl{Q_1}{Q_2}\scl{Q_1}{Q_4}-\scl{Q_1}{Q_1}\scl{Q_2}{Q_4} = 0, \\
\scl{Q_1}{Q_2}\scl{Q_1}{Q_3}+\scl{Q_1}{Q_1}\scl{Q_2}{Q_3} = 0, \\
\scl{Q_1}{Q_4}\scl{Q_1}{Q_5}+\scl{Q_1}{Q_1}\scl{Q_4}{Q_5} = 0
\end{gather*}
{From} this we get the thesis. 

Da qui in poi file di riferimento:
\verb+cubicaSingolare100.sage+, \verb+cubicaSingolare1ii0.sage+.

We have the following result:

\begin{prop}
  Let $P_1, \dots, P_5$ be five points in a ``V'' configuration. Then
  $P_1, \dots, P_5$ are eigenpoints of a cubic such that $P_1$ is singular
  if and only if one of the following condition is satisfied:
  \begin{itemize}
  \item $\delta_1(P_1, P_2, P_4) = 0 $ and $\overline{\delta}_1(P_1, P_2, P_3)=0$
  \item $\delta_1(P_1, P_2, P_4) = 0 $ and $\overline{\delta}_1(P_1, P_4, P_5)=0$
  \item $\overline{\delta}_1(P_1, P_2, P_3)=0$ and
    $\overline{\delta}_1(P_1, P_4, P_5)=0$
  \end{itemize}
\end{prop}

In particular, we see that all the cubic curves of
Proposition~\ref{casoMatrR8}, (2) are singular in $P_1$.

\end{document}
