\documentclass[12pt, a4paper, reqno, captions=tableheading,bibliography=totoc]{scrartcl}

\usepackage[utf8]{inputenc}
\usepackage[T1]{fontenc}
\usepackage{mathtools}
\usepackage{amsthm}
\usepackage{amsmath}
\usepackage{amssymb}
\usepackage{lmodern}
\usepackage[english]{babel}
\usepackage{booktabs}
\usepackage{float}
\usepackage{enumitem}
\usepackage{tikz,pgf}
%\usepackage[utopia]{mathdesign}
\usepackage{palatino}
\usepackage{hyperref}
\hypersetup{
    colorlinks,
    linkcolor={red!50!black},
    citecolor={blue!50!black},
    urlcolor={blue!80!black}
}
\usepackage[nameinlink]{cleveref}

\theoremstyle{plain}
\newtheorem{lemma}{Lemma}[section]
\newtheorem{prop}[lemma]{Proposition}
\newtheorem{theorem}[lemma]{Theorem}
\newtheorem{corollary}[lemma]{Corollary}
\newtheorem{conjecture}[lemma]{Conjecture}
\newtheorem{fact}[lemma]{Fact}
\newtheorem{assumption}[lemma]{Assumption}
\newtheorem*{reduction}{Reduction}
\theoremstyle{definition}
\newtheorem{definition}[lemma]{Definition}
\newtheorem{es}[lemma]{Example}
\newtheorem*{notation}{Notation}
\newtheorem{rmk}[lemma]{Remark}


\newcommand{\N}{\mathbb{N}}
\newcommand{\Z}{\mathbb{Z}}
\newcommand{\Q}{\mathbb{Q}}
\newcommand{\R}{\mathbb{R}}
\newcommand{\C}{\mathbb{C}}
\newcommand{\p}{\mathbb{P}}
\newcommand{\sP}{\mathcal{P}}
\newcommand{\sL}{\mathcal{L}}
\newcommand{\de}{\partial}
\newcommand{\codim}{\mathrm{codim}}

\newcommand{\oo}{\mathcal{O}}
\newcommand{\Bl}{\mathrm{Bl}}

\newcommand{\iso}{\mathcal{Q}}

\newcommand{\imunit}{i}

\newcommand{\SO}{\operatorname{SO}}
\newcommand{\Eig}[1]{\operatorname{Eig}\left( {#1} \right)}
\newcommand{\polq}{{\rm Pol}_Q}
\newcommand{\comment}[1]{}

\newcommand{\scl}[2]{\left\langle {#1}, {#2} \right\rangle}

\newcommand\scalemath[2]{\scalebox{#1}{\mbox{\ensuremath{\displaystyle #2}}}}

\linespread{1.2}
\setlength{\parindent}{0pt}
\setlength{\parskip}{.25em}

\begin{document}



Given $P_1, \dots, P_5$ as usual, we want to see when
the matrix $\Phi(P_1, \dots, P_5)$ has rank $\leq 8$.

If we assume $P_1 = (ii, 1, 0)$ from \verb+file rank_8_2_ii_1_0.sage+
we get that there are no possibilities.

Assume $P_1 = (1, 0, 0)$. From \verb+file rank_8_1.sage+ we have several
possibilities, the most interesting is given by the condition given by an
ideal that is generated by:
\begin{gather*}
\delta_1(P_1, P_2, P_4) = 0, \\
\scl{P_1}{P_2}\scl{P_1}{P_3}+\scl{P_2}{P_3} = 0, \\
\scl{P_1}{P_4}\scl{P_1}{P_5}+\scl{P_4}{P_5} = 0
\end{gather*}
Let now $Q_1, \dots, Q_5$ be generic points (such that $Q_1, Q_2, Q_3$
and $Q_1, Q_4, Q_5$ are aligned. Let $\lambda$ be
such that $Q_1' = \lambda Q_1$ satisfies $\scl{Q_1'}{Q_1'} = 1$.
From the above result, if $M$ is an orthogonal matrix such that
$M \cdot Q_1'$ is the projective point $(1: 0: 0)$, we get that 
\begin{gather*}
\delta_1(M\cdot Q_1', M\cdot Q_2, M\cdot Q_4) = 0, \\
\scl{M\cdot Q_1'}{M\cdot Q_2}\scl{M\cdot Q_1'}{M\cdot Q_3}+\scl{M\cdot Q_1'}{M\cdot Q_1'}\scl{M\cdot Q_2}{M\cdot Q_3} = 0, \\
\scl{M\cdot Q_1'}{M\cdot Q_4}\scl{M\cdot Q_1',M\cdot Q_5}+\scl{M\cdot Q_1'}{M\cdot Q_1'}\scl{M\cdot Q_4}{M\cdot Q_5} = 0
\end{gather*}
Since $M$ is orthogonal, we have that $\scl{M\cdot P}{M\cdot Q} = \scl{P}{Q}$,
hence we get:
\begin{gather*}
\delta_1(Q_1, Q_2, Q_4) = 0, \\
\scl{Q_1}{Q_2}\scl{Q_1}{Q_3}+\scl{Q_1}{Q_1}\scl{Q_2}{Q_3} = 0, \\
\scl{Q_1}{Q_4}\scl{Q_1}{Q_5}+\scl{Q_1}{Q_1}\scl{Q_4}{Q_5} = 0
\end{gather*}

We introduce the following notation:\\
Let $P_1, P_2,P_3$ be three aligned points. then we set:
\[
\overline{\delta}_1(P_1, P_2, P_3)  =
\scl{P_1}{P_2}\scl{P_1}{P_3}+\scl{P_1}{P_1}\scl{P_2}{P_3}
\]

Moreover, we have the following result:

\begin{prop}
  Let $P_1, \dots, P_5$ be five points in a ``V'' configuration. Then
  $P_1, \dots, P_5$ are eigenpoints of a cubic such that $P_1$ is singular
  if and only if one of the following condition is satisfied:
  \begin{itemize}
  \item $\delta_1(P_1, P_2, P_4) = 0 $ and $\overline{\delta}_1(P_1, P_2, P_3)=0$
  \item $\delta_1(P_1, P_2, P_4) = 0 $ and $\overline{\delta}_1(P_1, P_4, P_5)=0$
  \item $\overline{\delta}_1(P_1, P_2, P_3)=0$ and
    $\overline{\delta}_1(P_1, P_4, P_5)=0$
  \end{itemize}
\end{prop}
vedi file \verb+cubicaSingolare100.sage+ e \verb+cubicaSingolare1ii0.sage+

In particular, all the cubic curves of the ................
\end{document}
