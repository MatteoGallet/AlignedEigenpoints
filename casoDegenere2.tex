\documentclass[10pt, a4paper, reqno, captions=tableheading,bibliography=totoc]{scrartcl}

%\usepackage[utf8]{inputenc}
%\usepackage[T1]{fontenc}
\usepackage{amsthm}
\usepackage{amsmath}
\usepackage{amssymb}
%\usepackage{lmodern}
\usepackage[english]{babel}


%\usepackage{booktabs}
%\usepackage{float}
%\usepackage{enumitem}
%\usepackage{tikz,pgf}
%\usepackage[utopia]{mathdesign}
%\usepackage{palatino}
%\usepackage{hyperref}
%\hypersetup{
%    colorlinks,
%    linkcolor={red!50!black},
%    citecolor={blue!50!black},
%    urlcolor={blue!80!black}
%}
%\usepackage[nameinlink]{cleveref}

\theoremstyle{plain}
\newtheorem{lemma}{Lemma}[section]
\newtheorem{prop}[lemma]{Proposition}
\newtheorem{theorem}[lemma]{Theorem}
\newtheorem{cor}[lemma]{Corollary}
\newtheorem{conjecture}[lemma]{Conjecture}
\newtheorem{fact}[lemma]{Fact}
\newtheorem{assumption}[lemma]{Assumption}
\newtheorem*{reduction}{Reduction}
\theoremstyle{definition}
\newtheorem{definition}[lemma]{Definition}
\newtheorem{es}[lemma]{Example}
\newtheorem*{notation}{Notation}
\newtheorem{rmk}[lemma]{Remark}


\newcommand{\N}{\mathbb{N}}
\newcommand{\Z}{\mathbb{Z}}
\newcommand{\Q}{\mathbb{Q}}
\newcommand{\R}{\mathbb{R}}
\newcommand{\C}{\mathbb{C}}
\newcommand{\p}{\mathbb{P}}
\newcommand{\sP}{\mathcal{P}}
\newcommand{\sL}{\mathcal{L}}
\newcommand{\de}{\partial}
\newcommand{\codim}{\mathrm{codim}}

\newcommand{\oo}{\mathcal{O}}
\newcommand{\Bl}{\mathrm{Bl}}

\newcommand{\iso}{\mathcal{Q}}

\newcommand{\SO}{\operatorname{SO}}
\newcommand{\Eig}{\operatorname{Eig}}
\newcommand{\polq}{{\rm Pol}_Q}
\newcommand{\comment}[1]{}

\newcommand{\scl}[2]{\left\langle {#1}, {#2} \right\rangle}

\title{caso in cui la V \`e fatta da due rette tangenti in P2 e P4
alla conica isotropa}
\author{}
\date{}

\linespread{1.2}
\setlength{\parindent}{0pt}
\setlength{\parskip}{.25em}

\begin{document}

\maketitle

Let $P_1, \dots, P_5$ be such that $P_1, P_2, P_3$ are aligned,
$P_1, P_4, P_5$ are aligned and the line $P_1+P_2$ and the line
$P_1+P_4$ are tangent to the isotropic conic, hence
the conditions satisfied by the points (((*v. lemma che dice che
una retta tg in P2 a conica isotropa sse (P1|P2)=(P2|P2)=0*))) are:
\[
\scl{P_1}{P_2}=0, \quad \scl{P_2}{P_2}=0, \quad \scl{P_4}{P_4}=0
\]
The point $P_1$ cannot be on the isotropic conic, hence, using the
action of $\mathrm{SO}_3(\mathbb{C})$, we can assume $P_1 = (1, 0, 0)$.
Since every element of $\mathrm{SO}_3(\mathbb{C})$ fixes the
isotropic conic and send a tangent line to it into another 
tangent line to it, when we transform the point $P_1$
into $(1, 0, 0)$, we transform the points $P_2$ and $P_4$ into, respectively,
the points $(0, i, 1)$ and $(0, -i, 1)$ (which are the common points to
the isotropic conic and the tangent lines through $P_1$).
Therefore it is enough to study the
specific configuration of the points given by:
\[
P_1 = (1, 0, 0), \quad P_2=(0, i, 1), \quad P_3=u_1P_1+u_2P_2, \quad
P_4= (0, -i, 1), \quad P_5 = v_1P_1+v_2P_4.
\]
It is immediate to see that $\delta_2(P_1, P_2, P_3, P_4, P_5) = 0$.
In the $15\times 10$ matrix $\Phi_M(P_1, \dots, P_5)$ we can erase the
rows: $\phi_2(P_1)$ (which is a zero row), the row $\phi_1(P_2)$
(since $\phi_1(P_2)=i\phi_2(P_2)$) and, for similar reason, the
rows: $\phi_1(P_3)$, $\phi_1(P_4)$ and $\phi_1(P_5)$. (((*uso la notazione
$\Phi(P) = (\phi_1(P), \phi_2(P), \phi_3(P))$*)))
Moreover, since the rank of the matrix $\Phi_M(P_1, P_2, P_3)$ is $5$,
we can also erase the row $\phi_3(P_3)$ and, for the same reason, the
row $\phi_3(P_5)$. The remaining matrix $M$ is a $8\times 10$ matrix whose rank
in general is $8$. It is not possible to have that all the order $8$ minors
of $M$ are zero, since this condition implies that the five points
are not all distinct.

Hence we have proved:

\begin{prop}
Let $P_1, P_2, P_3$ three aligned points and let also $P_1, P_4, P_5$
be aligned and assume that
\[
\scl{P_1}{P_2}=0, \quad \scl{P_2}{P_2}=0, \quad \scl{P_4}{P_4}=0
\]
Then the matrix $\Phi_M(P_1, P_2, P_3, P_4, P_5)$ has rank $8$.
\label{frecciaFissata}
\end{prop}

Since it seems to hold:
\begin{prop}
  Let $P_1, P_2, P_3$ be three distinct collinear points of the
  plane and let $Q$ be a point
  in the plane. If the matrix $\Phi_M(P_1, P_2, P_3, Q)$ has rank $\leq 7$,
  then the line $P_1+P_2$ is tangent to the isotropic conic in the point
  $P_1$ (or $P_2$ or $P_3$). In particular, the matrix $\Phi_M(P_1, P_2, P_3)$
  has rank $5$.
  \label{condition3+1}
\end{prop}

the above results allow to prove the following:

\begin{prop}
  Let $P_1, P_2, P_3$ three aligned points and let also $P_1, P_4, P_5$
  be aligned. Then the matrix $\Phi_M(P_1, P_2, P_3, P_4, P_5)$ cannot
  have rank $\leq 7$. 
\end{prop}
\begin{proof}
  If the above matrix has rank $\leq 7$, in particular we have that
  the matrix $\Phi_M(P_1, P_2, P_3, P_4)$ has rank $\leq 7$ hence, from 
  proposition~\ref{condition3+1}, we get the line $P_1+P_2$ is
  tangent to the isotropic conic in one of the points $P_1, P_2, P_3$.
  Similarly, also the line $P_1+P_4$ is tangent to the isotropic conic
  in one of the points $P_1, P_4, P_5$. Hence the points are such that
  $P_1$ is not on the isotropic conic and the lines $P_1+P_2$ and
  $P_1+P_4$ are tangent to the conic in the points $P_2$ and $P_4$.
  Hence the configuration of the points is the configuration described
  in proposition~\ref{frecciaFissata} so the rank of the matrix
  considered in this proposition cannot be $\leq 7$. 
\end{proof}


We consider again the case
\[
P_1 = (1, 0, 0), P_2 = (0, i, 1), P_4 = (0, -i, 1), P_3 = u_1P_1+u_2P_2,
P_5 = v_1P_1+v_2P_2
\]
The matrix $\Phi(P_1, ..., P_5)$, 
after some manipulations, is the following rank 8 matrix:
\[
\left(\begin{array}{rrrrrrrrrr}
0 & 0 & 0 & 0 & 0 & 0 & 6 \mathit{ii} & 0 & 0 & -6 \mathit{ii} \\
0 & 0 & 0 & 0 & 0 & 2 \mathit{ii} & 0 & 0 & 0 & 0 \\
0 & 0 & 0 & 3 & 0 & 0 & 0 & 0 & -3 & 0 \\
0 & 0 & 1 & 0 & 0 & 0 & 0 & -1 & 0 & 0 \\
0 & 0 & 0 & 0 & 1 & 0 & 0 & 0 & 0 & 0 \\
0 & 1 & 0 & 0 & 0 & 0 & 0 & 0 & 0 & 0 \\
-3 u_{1}^{2} u_{2} & 0 & 0 & 0 & 0 & 0 & 0 & 2 u_{1}^{2} u_{2} & 2 \mathit{ii} u_{1} u_{2}^{2} & 2 u_{1} u_{2}^{2} \\
-3 v_{1}^{2} v_{2} & 0 & 0 & 0 & 0 & 0 & 0 & 2 v_{1}^{2} v_{2} & -2 \mathit{ii} v_{1} v_{2}^{2} & 2 v_{1} v_{2}^{2}
\end{array}\right)
\]
Solving the system with Kramer, we get the following families of
cubic curves:

\begin{eqnarray*}
  C(l_1, l_2) & = & x^{3} u_{2} v_{1} l_{1} + \frac{3}{2} x y^{2} u_{2} v_{1} l_{1} + 
  \frac{3}{2} x z^{2} u_{2} v_{1} l_{1} + x^{3} u_{1} v_{2} l_{1} + \\
  & & \frac{3}{2} x y^{2} u_{1} v_{2} l_{1} + \frac{3}{2} x z^{2} u_{1} v_{2} l_{1} + \mathit{ii} y^{3} u_{2} v_{1} l_{2} + y^{2} z u_{2} v_{1} l_{2} + 
  \mathit{ii} y z^{2} u_{2} v_{1} l_{2} + \\
  & & z^{3} u_{2} v_{1} l_{2} - \mathit{ii} y^{3} u_{1} v_{2} l_{2} + y^{2} z u_{1} v_{2} l_{2} -  \mathit{ii} y z^{2} u_{1} v_{2} l_{2} + z^{3} u_{1} v_{2} l_{2} - \\
  & & 2 x y^{2} u_{2} v_{2} l_{2} - 2 x z^{2} u_{2} v_{2} l_{2}
  \end{eqnarray*}

also described by the linear combination of two cubics:

\begin{eqnarray}
  C(l_1, l_2) & = & x \cdot (u_{2} v_{1} + u_{1} v_{2}) \cdot (x^{2} + \frac{3}{2} y^{2} + \frac{3}{2} z^{2}) l_1+\\
  & &   \left(\mathit{ii}\right) \cdot (y + \mathit{ii} z) \cdot (y - \mathit{ii} z) \cdot (y u_{2} v_{1} - \mathit{ii} z u_{2} v_{1} - y u_{1} v_{2} - \mathit{ii} z u_{1} v_{2} + 2 \mathit{ii} x u_{2} v_{2})l_2  \nonumber
\label{famigliaCubDim1}
\end{eqnarray}

If $l_2=0$ the corresponding cubic has the two lines $P_1+P_2$ and $P_1+P_4$
of eigenpoints. If we assume $l_2 \not= 0$ (and also $u_1, u_2, v_1, v_2$
different from zero) we get that the cubics $C(l_1, l_2)$ have 7
eigenpoints, five are clearly $P_1, \dots, P_5$,
the remaining two are given by an ideal $J$, one of whose gererators is the
polynomial:
\[
yu_2v_1 + (-ii)zu_2v_1 + yu_1v_2 + (ii)zu_1v_2
\]
which is linear in $x, y, z$ and is the equation of the line passing
thorugh the point $P_1$ and orthogonal to the line $P_3+P_5$. In particular,
we get that the three eigenpoints $P_1$, $P_6$ and $P_7$ are collinear.
In general, there are no other collinearities among the points (indeed, if
$P_6$ (or $P_7$) were aligned with two of the other points, its coordinates
would be rationals in the parameters and the ideal $J$ would split
into the two ideals of the points $P_6$ and $P_7$).


Now we consider the one dimensional linear system of cubics given
by~(\ref{famigliaCubDim1}) and we want to see if, for some positions
of the points $P_3$ and $P_5$ we can have further collinearities.

\textbf{Case 1.} We want to see when $P_2, P_4$ and $P_6$ are aligned. 
In this case we get the following coordinates for $P_6$ and $P_7$:

\begin{eqnarray*}
  P_6 & = & \left(0,\,3 v_{1} l_{1} - 2 v_{2} l_{2},
  \,-2 \mathit{ii} v_{2} l_{2}\right)\\
  P_7 & = & \left(9 v_{1}^{2} l_{1} - 12 v_{1} v_{2} l_{2},
  \,-6 \mathit{ii} v_{1} v_{2} l_{1} + 4 \mathit{ii} v_{2}^{2} l_{2},
  \,-4 v_{2}^{2} l_{2}\right)
\end{eqnarray*}
In this case we have the five alignments $(P_1, P_2, P_3)$, $(P_1, P_4, P_5)$,
$(P_1, P_6, P_7)$ and $(P_2, P_4, P_6)$ and the following orthogonalities
of the points:
\[
\scl{P_1}{P_2} = \scl{P_1}{P_4} = \scl{P_1}{P_6} = 
\scl{P_2}{P_2} = \scl{P_2}{P_3} = \scl{P_4}{P_4} = 
\scl{P_4}{P_5} = 0
\]
Furthermore, here we have the orthogonalities of the lines containing
the points:
\[
\begin{array}{l}
(P_1+P_2) \perp (P_1+P_2), \ (P_1+P_2) \perp (P_2+P_5), \
  (P_1+P_4) \perp (P_1+P_4), \\
  (P_1+P_4) \perp (P_3+P_4), \ (P_1+P_6) \perp (P_3+P_5)
\end{array}
  \]

  \textbf{Case 2.} If we impose the collinearities of the points
  $P_2$, $P_5$ and $P_6$, we get that also the points $P_3, P_4, P_5$
  the points $P_3, P_5, P_7$ are aligned. The orthogonalities of the
  points and of the lines of the previous case are the same.

  The same collinearities are obtained if we impose the only condition
  that $P_3, P_5, P_7$ are aligned. Hence we have that the family of
  cubic curves given by~\ref{famigliaCubDim1} can only give the
  three configurations of eigenpoints described above (which are
  (3), (5), (8) of figure.....).
\end{document}
