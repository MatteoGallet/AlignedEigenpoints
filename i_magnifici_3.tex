\documentclass[a4paper, 11pt, reqno]{amsart}

\setlength{\parindent}{.4 in}
\setlength{\textwidth}{6.5 in}
\setlength{\topmargin} {0.2 in}
\setlength{\evensidemargin}{0 in}
\setlength{\oddsidemargin}{0 in}
\usepackage[utf8]{inputenc}
\usepackage[T1]{fontenc}
\usepackage{mathtools}
\usepackage{lmodern}
\usepackage[english]{babel}
\usepackage{booktabs}
\usepackage{float}
\usepackage{enumitem}
\usepackage{tikz,pgf}
\usetikzlibrary{calc}
\usetikzlibrary{through}
\usepackage{tkz-euclide}
\usepackage{palatino}
\usepackage{graphicx}
\usepackage{xcolor}
\usepackage{hyperref}
\hypersetup{
    colorlinks,
    linkcolor={red!50!black},
    citecolor={blue!50!black},
    urlcolor={blue!80!black}
}
\usepackage[nameinlink]{cleveref}

\theoremstyle{plain}
\newtheorem{lemma}{Lemma}[section]
\newtheorem{prop}[lemma]{Proposition}
\newtheorem{theorem}[lemma]{Theorem}
\newtheorem{corollary}[lemma]{Corollary}
\newtheorem{conjecture}[lemma]{Conjecture}
\newtheorem{fact}[lemma]{Fact}
\newtheorem{assumption}[lemma]{Assumption}
\newtheorem*{reduction}{Reduction}
\theoremstyle{definition}
\newtheorem{definition}[lemma]{Definition}
\newtheorem{es}[lemma]{Example}
\newtheorem*{notation}{Notation}
\newtheorem{rmk}[lemma]{Remark}

\tikzstyle{point}=[circle, draw, fill=black, inner sep=0pt, minimum size=4pt]
\tikzstyle{line}=[line width=1.5pt, black!70!white]

\newcommand{\N}{\mathbb{N}}
\newcommand{\Z}{\mathbb{Z}}
\newcommand{\Q}{\mathbb{Q}}
\newcommand{\R}{\mathbb{R}}
\newcommand{\C}{\mathbb{C}}
\newcommand{\p}{\mathbb{P}}
\newcommand{\sL}{\mathcal{L}}
\newcommand{\sE}{\mathcal{E}}
\newcommand{\sW}{\mathcal{W}}
\newcommand{\sU}{\mathcal{U}}
\newcommand{\sV}{\mathcal{V}}
\newcommand{\sF}{\mathcal{F}}
\newcommand{\de}{\partial}
\newcommand{\oo}{\mathcal{O}}

\newcommand{\nb}[2]{\textsl{{NB}.{#1}.{#2}}}

\newcommand{\iii}{\textit{i}\,}
\newcommand{\codim}{\mathrm{codim}}
\newcommand{\rk}{\ensuremath{\mathrm{rk}}}

\newcommand{\iso}{\mathcal{Q}_{\mathrm{iso}}}

\newcommand{\SO}{\operatorname{SO}}
\newcommand{\Eig}[1]{\mathcal{E}\!\left( {#1} \right)}
\newcommand{\Sym}{\operatorname{Sym}}

\newcommand{\scl}[2]{\left\langle {#1}, {#2} \right\rangle}

\newcommand\scalemath[2]{\scalebox{#1}{\mbox{\ensuremath{\displaystyle #2}}}}

\newcommand{\blue}[1]{{\color{blue}[#1]}}


\title{Eigenpoint collinearities of plane cubics}
\author[Valentina Beorchia]{Valentina Beorchia$^{\circ}$}
\address[\textsc{Valentina Beorchia}]{University of Trieste,
Department of Mathematics, Informatics and Geosciences,
Via Valerio 12/1, 34127 Trieste, Italy}
\email{beorchia@units.it}
\thanks{$^{\circ}$The researcher is a member of ``Gruppo Nazionale per le Strutture Algebriche, Geometriche e le loro Applicazioni'', INdAM. She is partially supported by MUR funds: PRIN project GEOMETRY OF ALGEBRAIC STRUCTURES: MODULI, INVARIANTS, DEFORMATIONS, PI Ugo Bruzzo, Project code: 2022BTA242.}
\author[Matteo Gallet]{Matteo Gallet$^{\diamond}$}
\address[\textsc{Matteo Gallet}]{University of Trieste,
Department of Mathematics, Informatics and Geosciences,
Via Valerio 12/1, 34127 Trieste, Italy}
\email{matteo.gallet@units.it}
\thanks{$^{\diamond}$The researcher is a member of ``Gruppo Nazionale per le Strutture Algebriche, Geometriche e le loro Applicazioni'', INdAM.}
\author[Alessandro Logar]{Alessandro Logar}
\address[\textsc{Alessandro Logar}]{University of Trieste,
Department of Mathematics, Informatics and Geosciences,
Via Valerio 12/1, 34127 Trieste, Italy}
\email{logar@units.it}
\date{}

\linespread{1.15}
\setlength{\parindent}{0pt}
\setlength{\parskip}{.25em}

\begin{document}

\begin{abstract}
 Given a ternary homogeneous polynomial, the fixed points of the map from~$\p^2$ to itself defined by its gradient are called its \emph{eigenpoints}. We focus on cubic polynomials, and analyze configurations of eigenpoints that admit one or more alignments. We give a classification and explicit equations, in the coordinates of the points, of all configurations: this is accomplished by using both geometric techniques and by an extensive use of computer algebra. 
\end{abstract}

\maketitle


\section{Introduction}
\label{introduction}

Tensors are natural generalizations of matrices in higher dimension.
Important notions related to matrices, as for example those of \emph{rank} and of \emph{eigenvector},
can be generalized to tensors and may provide information about them.
Since roughly twenty years ago, eigenvectors of tensors have been object of study from various points of view.
Our interest will be focused on a specific class of symmetric tensors, namely cubic ternary forms, and particular geometric configurations of their eigenvectors.
Before delving into the more technical aspects of our problem,
we begin with a short resume of the existing literature on eigenvectors of tensors.

\subsection*{Eigenvectors of tensors in the literature}
In 2005, Lim \cite{Lim} and Qi \cite{Qi} independently introduced the notions of eigenvalues and eigenvectors for tensors.
Since then, these two concepts have been of interest in several applications,
as in the study of hypergraphs and dynamical systems;
see \cite[Section~4]{QZ} for an introduction or \cite{GMV} for recent developments.
%controllare Gelfand
Eigenvectors of tensors also play an important role in the best rank-one approximation problem,
which is relevant in data analysis and signal processing. 
Indeed, by Lim's Variational Principle \cite{Lim}, the critical rank-one symmetric tensors for a symmetric tensor~$f$ are of the type~$v^d$, where $v$ is an eigenvector of~$f$.
This has applications in low-rank approximation of tensors (see \cite{OttSod}) as well as maximum likelihood estimation in algebraic statistics.
For another application in optimization, consider the problem of maximizing a polynomial function~$f$ over the unit sphere in~$\R^{n+1}$: 
the eigenvectors of the symmetric tensor~$f$ are critical points of this optimization problem.


The geometry of eigenvectors is intimately related with the Waring decomposition of a polynomial, 
corresponding to the symmetric Tucker decomposition of the associated symmetric tensor, 
as clarified in~\cite{DOT} and in~\cite{Ott}. 
Indeed, any best rank~$k$ approximation of a symmetric tensor, when it exists, lies in the linear space, 
called \emph{critical space}, spanned by the rank~$1$ tensors of the type $v_i^{\otimes d}$, where $v_i$ varies among all the eigenvectors. 
Therefore, a deep comprehension of the geometry of eigenvectors will
go towards an improvement on the insight of low rank approximation problems.
A first evidence in this direction is given by the so called ODECO (``orthogonally decomposable'') tensors, 
that is, symmetric tensors that admit a decomposition
$\sum _{i=0}^n v_i ^{\otimes d}$ with $v_0, \dotsc, v_n$ an orthogonal family (see \cite{Rob, BDHE} for further details);
the $v_i$'s turn out to be eigenvectors.
Finally, in \cite{OO} Oeding and Ottaviani employ eigenvectors of tensors in an algorithm to compute Waring decompositions of homogeneous polynomials.
Ottaviani points out in \cite[Section~8]{Ottaviani24} the close connection between eigenvectors of specific symmetric tensors and the so-called L\"uroth quartics. 

\subsection*{Eigenschemes of homogeneous ternary forms}
Given a vector space $V$, we fix an isomorphism $V \cong V^\vee$ where $V^\vee$ is the dual vector space.
A non-zero vector $v \in V$ is an {\it eigenvector} of a tensor
 $T\in V ^{\otimes d} \cong ( V^\vee)^{\otimes d-1}\otimes V$ if there exists $\lambda\in \C$ such that 
 $T(v^{\otimes d-1})= \lambda v$. 
In the present paper, we will focus on the geometry of configurations of eigenvectors of order $3$ ternary symmetric tensors,
that is elements of ${\rm Sym}^3 \C^3$. 

Recall that, by fixing a nondegenerate quadratic form and an orthonormal basis of $\C^3$, the space ${\rm Sym}^d \C^3$ can be identified with the space of homogeneous ternary forms
$\C[x,y,z]_d$, and it turns out that the general definition of eigenvector $v$ for $f \in \C[x,y,z]_d$ specializes 
to $\nabla f (v)=\lambda v$, see for instance \cite[Section 1]{ASS}.
Such a condition is preserved under scalar multiplication, 
so it is natural to regard eigenvectors as points in~$\p^2$, and to talk about \emph{eigenpoints}. Moreover, 
by construction, the set of eigenpoints of $f$ is the zero locus of the ideal of $2 \times 2$ minors of the matrix%
\begin{equation}
\label{eq:def_matrix}
\begin{pmatrix}
    x & y & z \\
    \partial_x f  & \partial_y f & \partial_z f 
\end{pmatrix} \,,
\end{equation}
%
thus it can be given a natural structure of determinantal scheme called \emph{eigenscheme}, which we denote by $\Eig{f}$. Similarly, a natural scheme structure can be given to the eigenpoints of any
tensor $T \in (\C^3)^{\otimes d}$, see \cite[Section 1]{ASS}, and such a scheme will be denoted by $\Eig{T}$.

It has been proved in \cite[Corollary 5.8]{Abo} that $\Eig{f}$ is in general $0$-dimensional and reduced, and its degree has been determined in \cite[Theorem 2.1]{CartSturm}.
In the particular case of $d=3$ we generally have
$\deg \Eig{f}=7$; however,
 there are cases when it is one-dimensional, or non-reduced.

A first geometric characterization of eigenschemes of ternary tensors has been given in
\cite[Theorem 5.1]{ASS}, and it states that seven points in~$\p^2$ are eigenpoints of some tensor if and only if no six of them lie on a conic.
Moreover, by \cite[Theorem 5.7]{BGV},
the eigenscheme~$\Eig{f}$ of a general ternary cubic contains no triple of collinear points.

However, there are several examples of cubic forms, whose eigenscheme contains one or more triples of aligned points, as for instance the Fermat cubic polynomial $f=x^3+y^3+z^3$.
Understanding these situations requires a careful analysis of the alignement conditions.

\subsection*{Our results}
The goal of this paper is to classify all the situations when we have one or more triples of aligned points inside a $0$-dimensional reduced eigenscheme of a ternary cubic form.
More precisely, we
%
\begin{itemize}
    \item determine conditions imposed on the eigenpoints and on ternary cubic forms by the presence of aligned eigenpoints;
    \item compute the degree of the closure of the locus in~$\p^9$ of ternary cubic forms that admit an aligned triple of eigenpoints;
    \item characterize the situations yielding a one-dimensional eigenscheme;
    \item classify all possible combinatorial configurations of aligned eigenpoints when the eigenscheme is $0$-dimensional and reduced.
\end{itemize}
%
All the conditions on the eigenpoints will be expressed in terms of their homogeneous coordinates, and the equations that we will find have to be though as equations in $(\p^2)^7=\p^2 \times \p^2 \times \dotsb \times \p^2$.

Finally, we would like to point out that our approach gives some hints towards the problem of finding equations for the locus of $7$-uple of points, which form an eigenscheme of some symmetric tensor; this is an open question, and it has been posed in \cite{ASS} and \cite{Ottaviani24}.

\subsection*{Content of the paper}
Here is a description of the content of each section.
\Cref{invariance} describes the action of special orthogonal matrices on ternary forms, which we repeatedly use in our analyses to "pin down" one of the eigenpoints to two specific cases, thus greatly speeding up the symbolic computations we rely on.
\Cref{conditions} begins our analysis by examining what conditions on the space of ternary cubics are imposed by the existence of aligned eigenpoints. In particular, three kinds of conditions, named~$\delta_1$, $\bar{\delta}_1$, and~$\delta_2$, emerge and play a prominent role in describing configurations of two aligned triples that share a point, here called $V$- configurations.
\Cref{V-configurations} sets the spotlight on $V$- configurations, exhausting all the possible special cases that may appear.
\Cref{locus_one_alignment} introduces the locus of ternary cubics with an alignment in the eigenscheme, proves its irreducibility, and computes its degree.
\Cref{positive_dim} classifies positive-dimensional eigenschemes of ternary cubics.
\Cref{further_alignments} examines all the possible configurations of alignments in a $0$-dimensional reduced eigenscheme.

\subsection*{Symbolic computation}
Several proofs employ coPropositionmputer algebra verifications in an essential way.
We rely on SageMath \cite{SageMath} for the operations we need 
(mainly, manipulations of polynomials and polynomial ideals).
To allow the reader for an easier use of these computations, we have prepared Jupyter notebooks, available at \cite{Notebooks}.
Each notebook has a name of the form \nb{XX}{CODE} where \textit{XX} denotes the section and \textit{CODE} refers to the result it proves.

\subsection*{Notation and preliminaries}
We work in $\p^2_\C$, endowed with the canonical projective coordinate system.
We denote by~$P \vee Q$ the line through two distinct points~$P$ and~$Q$.
We denote by~$\iso$ the so-called \emph{isotropic conic}, namely the conic of equation
%
\[
 x^2 + y^2 + z^2 = 0 \,.
\]
%
Since there will be no ambiguity, sometimes we will identify $\iso$ with the 
defining polynomial.

The quadratic form associated to~$\iso$ is denoted $\left\langle \cdot, \cdot \right\rangle$ and throughout the paper, the notion of \emph{orthogonality} refers to this quadratic form.


For $P_i=(A_i:B_i:C_i)$ and $P_j=(A_j:B_j:C_j)$, we shall write
%
\[
  s_{ij} = \scl{P_i}{P_j} = A_i A_j + B_i B_j + C_i C_j \,.
\]
%
Such an expression has to be intended as a bihomogeneous polynomial in the coordinates of~$P_i$ and~$P_j$. If we think of the coordinates as indeterminates, the zero set of~$s_{ij}$ is well defined in $\p^2 \times \p^2$.

Similarly, by $P_1 \times P_2$ we shall denote the \emph{cross product} of the representing vectors
%
\begin{equation}\label{equation:cross product}
  P_1 \times P_2 = 
  (B_1 C_2 - C_1 B_2, \, C_1 A_2 - A_1 C_2, \, A_1 B_2 - B_1 A_2) \,.
\end{equation}
%
The resulting projective point is well defined. Such a notation will be used, for instance, in \Cref{definition:delta1}.

In a similar fashion, whenever we define a point in~$\p^2$ by means of the expressions we introduce in the text, as for example in \Cref{lemma_delta_case2}, where we write
%
\[
  P_3 = (s_{12}^2+s_{11}s_{22}) \, P_1 - 2s_{11}s_{12} \, P_2 \,,
\]
%
what we mean is that the right hand side is a (multi-)homogeneous polynomial, which therefore defines a rational function
%
\[
  \p^2 \times \p^2 \dashrightarrow \p^2
\]
%
and the point on the left hand side is the image of $P_1$ and $P_2$ under this map.

As said above, given a homogeneous form $f \in \C[x,y,z]_d$ of degree~$d$, the \emph{eigenscheme}~$\Eig{f} \subset \p^2$ of~$f$ is the determinantal scheme defined by the $2 \times 2$ minors of the matrix in \Cref{eq:def_matrix}.
Throughout the paper, we denote them by $g_1, g_2$ and $g_3$ and we fix the following choice:
%
\begin{equation}
\label{eq:def_minors}
 g_1 = x \partial_y f - y \partial_x f \,, \quad
 g_2 = x \partial_z f - z \partial_x f \,, \quad
 g_3 = y \partial_z f - z \partial_y f \,.
\end{equation}
%

%Since the eigenschemes of two proportional homogeneous forms are the same,
%if $C \subset \p^2$ is a reduced curve, then we can define with no ambiguity the eigenscheme of~$C$, denoted~$\Eig{C}$, as the eigenscheme of any homogeneous form defining~$C$.

The following result (see \cite[Theorem 2.1]{CartSturm}, \cite{ASS}, \cite{OO}, and \cite[Equation~5.2]{Abo}) determines the degree of zero-dimensional eigenschemes.

\begin{theorem}
\label{thm:nonempty}
Let $d \ge 2$ and let $T \in (\C^{n+1})^{\otimes d}$.
If $\dim \Eig{T}=0$, then 
%
\[
  \deg \Eig{T} =
  \frac{(d-1)^{n+1}-1}{d-2} =
  \sum_{i=0}^{n} (d-1)^i \,.
\]
%
In the matrix case, that is $d = 2$, we use the formula on the right, which evaluates to $n+1$.
\end{theorem}

We point out that \Cref{thm:nonempty} holds also in the non-reduced case. However, a result by H. Abo (\cite[Corollary 5.8]{Abo}) guarantees that general forms have $0$-dimensional and reduced eigenschemes.

\begin{definition}\label{def: eigendiscriminant}
An order~$3$ tensor $T \in (\C^3)^{\otimes 3}$ is called \emph{regular} if its eigenscheme is reduced of dimension~$0$.
We set
%
\[
 \Delta_{3,3} := \{[T]\in \p \bigl( (\C^3)^{\otimes 3} \bigr) \ | \ \Eig{T} \textrm{\ is \ not \ regular} \} \subset \p \bigl( (\C^3)^{\otimes 3} \bigr) \,.
\]
%
The locus~$\Delta_{3,3}$ is called the \emph{eigendiscriminant} (see \cite[Definition 5.5]{Abo}) and is an irreducible hypersurface (see \cite[Corollary 5.8]{Abo}).
\end{definition}

As already mentioned, a regular eigenscheme of a cubic form never contains six points on a conic by \cite[Theorem 5.1]{ASS}. As a consequence we have the following result.



\begin{lemma}
    \label{lemma: no 4 aligned}
    Given $f \in \C[x,y,z]_3$,
if $\Eig{f}$ is regular, then it contains no $4$ or more aligned points. Moreover, if $\Eig{f}$ contains two aligned triples, they must share a point.
\end{lemma}


It can be expected that the presence of one or more aligned triples of eigenpoints should impose particular geometric constraints, since by \cite[Theorem 5.7]{BGV} this is not the general case.

We conclude this section by recalling the notion of \emph{Geiser map} and by explaining the connection between its contracted locus with alignments of eigenpoints; such a point of view  will be useful in \Cref{V-configurations}.

\begin{definition}
The Geiser map associated with a zero dimensional eigenscheme $\Eig{f}$ is the rational map defined by
%
\[
  \gamma_{\Eig{f}} \colon \p ^2 \dasharrow \p^2, \quad
  \gamma_{\Eig{F}} (P) = \bigl( g_3(P):-g_2(P):g_1(P) \bigr) \,,
\]
%
where $g_1, g_2, g_3$ are as in \Cref{eq:def_minors}.
\end{definition}

Geiser maps are a classical topic and several of their properties are understood.
As an example, the map~$\gamma_{\Eig{f}}$ is generically finite of degree~$2$, see for instance \cite[Section~8.7.2]{Dolgachev}.
%Moreover, the ramification locus of~$\widetilde{\gamma _{E(F)}$ is given by the \emph{Jacobian locus}~$\Sigma$ defined by the determinant of the Jacobian $\mathrm{Jac}(I_Z)$, that is the locus of singular points of the net (see
%    \cite[Book II, Chapter~IX, Theorem~25]{Cool}). When $Z$ is general, the Jacobian locus~$\Sigma$ is a curve of degree~$6$ which is singular at $Z$ as illustrated in \cite[Chapter~IX, Theorem~27]{Cool}. We define $B(Z)$ to be the branch locus of $\widetilde{\gamma_Z}$, that is the direct image of~$\Sigma$. For modern references, see for instance \cite[Section~8.7.2]{Dolgachev} and \cite[Section~7]{OS1}, where it is proven that a general Geiser map is branched along a smooth L\"uroth quartic.

\begin{lemma}
The Geiser map~$\gamma_{\Eig{f}}$ is surjective and its fiber over a point $Q = (a:b:c)$ is given by
%
\begin{equation}
\label{eq:fibers}
  \left\{
  \begin{array}{l}
    a x + by + cz = 0 \,, \\[2pt]
    a \, \partial_x f + b \, \partial_y f + c \, \partial_z f = 0 \,,\\
  \end{array}
  \right.
\end{equation}
%
that is, the intersection between the polar line~$\ell_Q$ relative to the isotropic conic~$\iso$ and~$\mathrm{Pol}_Q f$, the
first polar curve of~$f$ with respect to~$Q$.

In particular, the only possible curves contracted by the Geiser map~$\gamma_{\Eig{f}}$ are lines.
\end{lemma}

\begin{proof}
We observe that for any point $P=(A:B:C) \in \p^2 \setminus \Eig{f}$, the homogeneous coordinates
of the image $\gamma_{\Eig{f}}(P) = \bigl( g_3(P): -g_2(P): g_1(P) \bigr)$ are the ones of the unique point in the intersection of the two lines
%
\[
  Ax + By+ Cz = 0 \,, \qquad
  \partial_x f(P) \, x + \partial_y f(P) \, y + \partial_z f(P) \, z = 0 \,.
\]
%
So for any $Q = (a:b:c) \in \p^2$, the fiber~$\gamma_{\Eig{f}}^{-1}(Q)$ consists of the points $P \in \p^2$ such that
%
\begin{equation}
\label{eq:polars}
  Aa + Bb+ Cc = 0, \quad
  \partial_x f(P) \, a + \partial_y f(P) \, b + \partial_z f(P) \, c = 0 \,,
\end{equation}
%
which proves that $\gamma_{\Eig{f}}$ is surjective and that \Cref{eq:fibers} holds.
%To prove that there are no other contracted curves than lines,
%set $Z=E(F)$ and consider the blow-up of the plane $\p^2$ along $Z$. We get a generically finite morphism
%%
%\[
%  \widetilde \gamma_Z \colon \Bl_Z \p^2 \to \p^2.
%\]
%%
%Observe that the fibers of~$\widetilde{\gamma_Z}$ are generically contained in the divisor~$W$ with bihomogenous equation $x_0 y_0 + x_1 y_1 + x_2 y_2=0$.
%Since both~$\Bl_Z \p^2$ and~$W$ are irreducible, and since $\widetilde{\gamma_Z}$ is the restriction of the second projection $p_2 \colon \p^2 \times \p^2 \rightarrow \p^2$, it follows that $S\subseteq W$;
%in particular, every fiber of~$\widetilde{\gamma_Z}$ is contained in a line and by construction the same holds for every fiber of~$\gamma_Z$.
%As a consequence, the map $\gamma_Z$ contracts only lines.
\end{proof}

\begin{prop}
\label{prop:contract_aligned}
Suppose that $\Eig{f}$ is a zero dimensional eigenscheme of a cubic ternary form.
If $\Eig{f}$ contains a triple of points on a line~$\ell \subset \p^2$, then the Geiser map~$\gamma_{\Eig{f}}$ contracts~$\ell$.
\end{prop}

\begin{proof}
%Consider the isomorphism  $\varphi \colon \p (I_{\Eig{F}}(3))^{\vee} \stackrel{\cong}{\longrightarrow} \p (I_{\Eig{F}}(3))$ associating with each point~$(a:b:c)$ the line $ax + by + cz = 0$.
%For any $P \in \p^2 \setminus \Eig{F}$, the point~$\varphi(\gamma_{\Eig{F}} (P))$ corresponds to the line~$L_P$ given by
%
We identify the codomain $\p^2$ with $\p\bigl(I_{\Eig{f}}(3)^{\vee}\bigr)$, namely the projectivization of the dual of the space of degree~$3$ forms in the homogeneous ideal of~$\Eig{f}$.
For any $P \in \p^2 \setminus \Eig{f}$, the point~$\gamma_{\Eig{f}} (P)\in \p \bigl(I_{\Eig{f}}(3)^{\vee}\bigr)$ corresponds to the line~$\ell_P$ given by
\[
  g_3 (P) \, x - g_2(P) \, y + g_1(P) \, z = 0 \,,
\]
%
which in turn determines the following
pencil of cubics $\ell_P$ in the linear system $\p \bigl(I_{\Eig{f}}(3)\bigr)$:
%
\[
  \lambda \cdot \bigl( g_2(P) \, g_3  -g_3(P) \, g_2\bigr) + \mu \cdot \bigl( g_3(P) \, g_1 - g_1(P) \, g_3 \bigr) \,,
\]
%
where $(\lambda : \mu) \in \p^1$, assuming that $g_1(P) \neq 0$ (the other cases are similar).
It is clear that the base locus of the pencil~$\ell_P$ contains $\Eig{f} \cup \{P\}$.
So $\gamma_{\Eig{f}} \colon \p^2 \dasharrow \p \bigl( I_{\Eig{f}}(3)^\vee \bigr)$
associates with each point $P \in \p^2 \setminus \Eig{f}$ the pencil of cubics through $\Eig{f} \cup P$.

If $\Eig{f}$ contains a triple of points on a line~$\ell$, for any $P \in \ell \setminus \Eig{f}$ such a pencil of cubics has $\ell$ as a fixed component and the other component varies in a pencil of conics through the remaining $4$ points, so $\gamma_{\Eig{f}}$ is constant on~$\ell$.
\end{proof}


\section{Invariance under the action of orthogonal matrices}
\label{invariance}

In what follows, it will be useful to fix particular coordinates for points and lines related to eigenschemes.
To do that, we employ a property of invariance of eigenschemes with respect to the action of the following group.

\begin{definition}
We define $\SO_3(\C)$ to be the complexification of the group of special orthogonal real matrices, namely
%
\[
  \SO_3(\C) :=
  \bigl\{
    M \in \mathrm{GL}_3(\C) \, \mid \,
    M \prescript{t} {}M = I_3 \ \text{and} \ \det(M) = 1
  \bigr\} \,.
\]
%
The group $\SO_3(\C)$ acts on $\C^3$ by matrix multiplication:
%
\[
  \begin{array}{ccc}
    \SO_3(\C) \times \C^3 & \rightarrow & \C^3 \\
    (M, v) & \mapsto & Mv
  \end{array}
\]
%
Since all the elements of~$\SO_3(\C)$ are invertible, the latter action descends to an action on $\p^2$.

Moreover, the group~$\SO_3(\C)$ acts also on ternary forms via
%
\[
  M \cdot f (x,y,z) := f(M^{-1} \cdot \prescript{t} {}( x \ y \ z ) ).
\]
%
\end{definition}

\begin{prop}
\label{two_orbits}
The action of~$\SO_3(\C)$ on $\p^2$ has two orbits:
%
\begin{align*}
  \mathcal{O}_1 &:=
  \bigl\{
    P \in \p^2 \, \mid \,
    P = (a:b:c) \ \text{with} \ a^2 + b^2 + c^2 = 0
  \bigr\} \\
  \mathcal{O}_2 &:= \p^2 \setminus \mathcal{O}_1
\end{align*}
%
A representative for~$\mathcal{O}_1$ is~$(1:\iii:0)$ and a
representative for~$\mathcal{O}_2$ is~ $(1:0:0)$.
The orbit~$\mathcal{O}_1$ is the set of points of the isotropic conic~$\iso$.
\end{prop}
\begin{proof}
Suppose that $P \in \p^2$ and $P = (a:b:c)$ with $a^2 + b^2 + c^2 = 0$.
We produce a matrix $M \in \SO_3(\C)$ such that $M \left(\begin{smallmatrix} 1 \\ \iii \\ 0 \end{smallmatrix}\right)$ and $\left(\begin{smallmatrix} a \\ b \\ c \end{smallmatrix}\right)$ are proportional.
Up to relabeling the coordinates, we can suppose that $a \neq 0$.
Hence, by rescaling the coordinates of~$P$, we have $P = (1: b: c)$ with $b^2 + c^2 = -1$.
One can check that the matrix
%
\[
  M :=
  \begin{pmatrix}
    -1 & 0 & 0 \\
    0 & \iii b & -\iii c \\
    0 & \iii c & \iii b
  \end{pmatrix}
\]
%
satisfies the requirements.

Now, suppose that $P \in \p^2$ and $P = (a:b:c)$ with $a^2 + b^2 + c^2 \neq 0$.
Up to rescaling, we can suppose that $a^2 + b^2 + c^2 = 1$.
Again, we produce a matrix $M \in \SO_3(\C)$ such that $M \left(\begin{smallmatrix} 1 \\ 0 \\ 0 \end{smallmatrix}\right)$ and~$\left(\begin{smallmatrix} a \\ b \\ c \end{smallmatrix}\right)$ are proportional.
First of all, suppose that $b^2 + c^2 \neq 0$ and let $\omega$ be a root of the polynomial $t^2 - (b^2 + c^2)$ in $\C[t]$.
Then, the matrix
%
\[
  M :=
  \begin{pmatrix}
    a & \omega & 0 \\
    b & -\frac{ab}{\omega} & \frac{c}{\omega} \\
    c & -\frac{ac}{\omega} & -\frac{b}{\omega}
  \end{pmatrix}
\]
%
satisfies the requirements.
With the same technique, if $a^2 + c^2 \neq 0$, we can produce a matrix $M \in \SO_3(\C)$ that maps $\left(\begin{smallmatrix} 0 \\ 1 \\ 0 \end{smallmatrix}\right)$ to $\left(\begin{smallmatrix} a \\ b \\ c \end{smallmatrix}\right)$; similarly, when $a^2 + b^2 \neq 0$, we can map $\left(\begin{smallmatrix} 0 \\ 0 \\ 1 \end{smallmatrix}\right)$ to $\left(\begin{smallmatrix} a \\ b \\ c \end{smallmatrix}\right)$.
Since $\left(\begin{smallmatrix} 1 \\ 0 \\ 0 \end{smallmatrix}\right)$, $\left(\begin{smallmatrix} 0 \\ 1 \\ 0 \end{smallmatrix}\right)$, and $\left(\begin{smallmatrix} 0 \\ 0 \\ 1 \end{smallmatrix}\right)$ are all $\mathrm{SO}_3(\C)$-equivalent, the only case to consider is when
%
\[
  b^2 + c^2 = a^2 + c^2 = a^2 + b^2 = 0 \,,
\]
%
which, however, can never occur.
\end{proof}

The following result is well known; we recall it for the sake of completeness.

\begin{prop}
Let $M \in \SO_3(\C)$ and let $f$ be a ternary cubic.
Let $P = (A: B: C)$ be a point in~$\p^2$.
Then we have
%
\[
  P \in \Eig{f}
  \quad \text{if and only if} \quad
  M \cdot \prescript{t} {}(A \ B \ C) \in \Eig{M \cdot f} \,.
\]
%
\end{prop}
\begin{proof}
In this proof, for convenience,
we consider the transpose of the defining matrix of an eigenscheme.
A point~$P = (A: B: C)$ is an eigenpoint for~$f$ if and only if
%
\begin{equation}
\label{eq:def_matrix_M}
  \mathrm{rk}
  \begin{pmatrix}
    A & \de_x f(P) \\
    B & \de_y f(P) \\
    C & \de_z f(P)
  \end{pmatrix}
  = 1 \,
 \quad \text{or, equivalently} \quad
  \mathrm{rk} \ M \cdot 
  \begin{pmatrix}
    A & \de_x f(P) \\
    B & \de_y f(P) \\
    C & \de_z f(P)
  \end{pmatrix}
  = 1 \,.
\end{equation}
%
By setting $\prescript{t} {}(A' \ B' \ C' ) := M \cdot \prescript{t} {}(A \ B \ C) $ and~$Q := (A':B':C')$, we have that \Cref{eq:def_matrix_M}
is equivalent to
%
\begin{equation}
\label{eq:transformed}
  \rk
  \begin{pmatrix}
    A' & \\
    B' & M \cdot \nabla f (P) \\
    C' & \\
  \end{pmatrix}
  = 1 \,.
\end{equation}
%
Now we consider the polynomial $M \cdot f$ and we observe that the chain rule gives
%
\begin{gather*}
  \partial_x (M\cdot f) = \partial_x \bigl( f(M^{-1} \ \prescript{t} {} (x \ y \ z)) \bigr) = \prescript{t} {}(M^{-1})^{(1)}(\nabla f) \bigl( M^{-1}\ \prescript{t} {} (x \ y \ z) \bigr) \,, \\
  \partial_y (M\cdot f) = \partial_y \bigl( f(M^{-1} \ \prescript{t} {} (x \ y \ z)) \bigr) = \prescript{t} {}(M^{-1})^{(2)}(\nabla f) \bigl( M^{-1}\ \prescript{t} {} (x \ y \ z) \bigr) \,, \\
  \partial_z (M\cdot f) = \partial_z \bigl( f(M^{-1} \ \prescript{t} {} (x \ y \ z)) \bigr) = \prescript{t} {}(M^{-1})^{(3)}(\nabla f) \bigl( M^{-1}\ \prescript{t} {} (x \ y \ z) \bigr) \,,
\end{gather*}
%
where $(M^{-1})^{(j)}$ denotes the $j$-th column of the matrix $M^{-1}$. Hence
%
\[
  \nabla (M \cdot f) = \prescript{t} {} M^{-1} \cdot (\nabla f) \bigl(M^{-1}\ \prescript{t} {} (x \ y \ z)\bigr) \,,
\]
%
so we have
%
\[
  \nabla (M \cdot f)(Q)=\nabla (M \cdot f)(M \cdot P)=
  \prescript{t} {} M^{-1} \cdot (\nabla f) (M^{-1}\ M \cdot P)=\prescript{t} {} M^{-1} \cdot (\nabla f)(P) \,.
\]
%
Finally, if we choose $M \in \SO_3(\C)$, we have
$\prescript{t} {} M^{-1}=M$. We deduce that
\Cref{eq:transformed} holds if and only if $Q \in \Eig{M\cdot f}$, so the statement is proved.
\end{proof}


\section{Conditions imposed by aligned eigenpoints}
\label{conditions}

Having a given point as eigenpoint determines linear conditions on the space of ternary cubic forms, 
which can hence be encoded into a matrix, called the \emph{matrix of conditions} (\Cref{definition:matrix_conditions}). 
We analyze the possible ranks of matrices of conditions relative to three to five points, with one or two aligned triples.
%, we characterize in terms of three polynomial conditions (denoted $\delta_1$, $\bar{\delta}_1$ and~$\delta_2$) the situations with two triples of aligned eigenpoints.

\subsection{The matrix of conditions}

%Imposing a cubic ternary form to have one or more aligned triples of
% eigenpoints implies conditions both on the points and on the cubics.
We begin to explore the condition that $P\in \p^2$ is an eigenpoint of a ternary cubic; such linear conditions are encoded in
a $3 \times 10$ matrix, which we now introduce.

\begin{definition}
\label{definition:matrix_conditions}
Consider $\p^9 = \p(\C[x,y,z]_3)$, the space of all ternary cubics.
Throughout this paper, we consider the standard monomial basis for $\C[x,y,z]_3$ and we set~$\mathcal{B}$ to be the following vector
%
\begin{equation}
\label{eq:vector_basis}
  \mathcal{B} := (x^3, x^2 y, x y^2, y^3, x^2 z, x y z, y^2 z, x z^2, y z^2, z^3)
  \in \bigl( \C[x,y,z]_3 \bigr)^{\oplus 10} \,.
\end{equation}
For $f \in \C[x,y,z]_3$, denote by~$[f]$ the corresponding point in~$\p^9$; we denote by~$w_f$ the (column) vector of coordinates of~$f$, and we will use the same notation for the projective coordinates of~$[f]$.
For a point $P \in \p^2$ with coordinates $(A: B: C)$, the condition that $P$ is an eigenpoint, i.e.\ that $g_1(P) = g_2(P) = g_3(P) = 0$ where the $g_i$ are as in \Cref{eq:def_minors}, can be expressed as three linear conditions in the coordinates of~$\p^9$,
hence
in the form
%
\[
  \Phi(P) \cdot w_f = 0 \,,
\]
%
where $\Phi(P)$ is a $3 \times 10$ matrix with entries depending on $A, B, C$.
The matrix $\Phi(P)$ is called the \emph{matrix of conditions} imposed by~$P$.
We denote by~$\phi_1(P)$, $\phi_2(P)$, and~$\phi_3(P)$ the rows of~$\Phi(P)$.
Written as vectors, they are
%
\begin{equation}
\label{equation:matrix_conditions_rows}
  \begin{aligned}
    \phi_1(P) &=
    \scalemath{0.9}{
    (-3A^2B, A(A^2 - 2B^2), B(2A^2 - B^2), 3AB^2,
     -2ABC, C(A^2 - B^2), 2 ABC,
     -B C^2, A C^2, 0)} \,, \\
    \phi_2(P) &= 
    \scalemath{0.9}{
    (-3A^2 C,
     -2ABC,
     -CB^2,
     0,
     A(A^2-2C^2),
     B(A^2 - C^2),
     AB^2,
     C(2A^2-C^2),
     2ABC,
     3AC^2)} \,,\\
    \phi_3(P) &=
    \scalemath{0.9}{
    (0,
     -A^2C,
     -2ABC,
     -3CB^2,
     A^2 B,
     A(B^2 - C^2),
     B(B^2-2C^2),
     2ABC,
     C(2B^2-C^2),
     3BC^2)} \,.
  \end{aligned}
\end{equation}
%
More generally, if $P_1, \dotsc, P_n$ are points in the plane, we denote by~$\Phi(P_1, \dotsc, P_n)$ the matrix whose rows are
%
\[
  \phi_1(P_1), \phi_2(P_1), \phi_3(P_1),
  \dotsc, 
  \phi_1(P_n), \phi_2(P_n), \phi_3(P_n),
\]
%
and we call it the
\emph{matrix of conditions} imposed by $P_1, \dotsc, P_n$.
\end{definition}

\begin{definition}
Given a matrix $M$ of type $m \times 10$, we denote by $\Lambda(M) \subset \p^9$ the projective linear system of cubics $[a_0 x^3 + \dotsb + a_9 z^3]$ such that $M \, \prescript{t}{}{\left( a_0 \,  \cdots \,  a_9 \right)} = 0$.
\end{definition}

The description of~$\Phi(P)$ can be shortened as follows: consider the vectors $\de_x \mathcal{B}$,
%denote the vector whose entries are the derivatives of those of~$\mathcal{B}$, and similarly for~
$\de_y \mathcal{B}$ and~$\de_z \mathcal{B}$;
if $P=(A: B: C)$, we get:
%
\begin{equation}
\label{equation:vector_conditions}
  \begin{aligned}
    \phi_1(P) &= A\cdot \de_y \mathcal{B}(P) - B\cdot \de_x \mathcal{B}(P) \,, \\
    \phi_2(P) &= A\cdot \de_z \mathcal{B}(P) - C\cdot \de_x \mathcal{B}(P) \,, \\
    \phi_3(P) &= B\cdot \de_z \mathcal{B}(P) - C\cdot \de_y \mathcal{B}(P) \,.
  \end{aligned}
\end{equation}
%
\begin{rmk}
\label{remark:rank_2}
By analyzing the entries of \Cref{equation:matrix_conditions_rows}, it is not difficult to check that the rank of~$\Phi(P)$ is never $\leq 1$. \Cref{equation:vector_conditions} gives the relation
%
\begin{equation}
\label{eq:syzygy}
  C \, \phi_1(P) - B \, \phi_2(P) + A \, \phi_3(P) = 0 \,.
\end{equation}
%
Therefore, the vectors~$\phi_1(P)$, $\phi_2(P)$, and~$\phi_3(P)$ are linearly dependent and the matrix~$\Phi(P)$ has rank~$2$.
As a consequence, if $P_1, \dots, P_n$ are $n$ points of the plane, we have:
%
\[
  \rk \,\Phi(P_1, \dots, P_n) \leq \min \left\{2n, 10 \right\} \,.
\]
%
\end{rmk}

\subsection{Possible ranks of the matrix of conditions}

In what follows, we want to study the possible values of the rank of the matrix
$\Phi(P_1, \dots, P_n)$ for several configurations of points $P_1, \dots, P_n$
(and several values of~$n$).
In particular, we study the ideal~$J_k$ of order $k$ minors of the
involved matrices of conditions and we deduce some bounds about the rank from the primary
decomposition of~$J_k$. 
Most of these computations are done with the aid of a computer algebra system.
Nevertheless, in many cases, the result cannot be reached just by brute force, 
but it is necessary to preprocess the ideal~$J_k$. 
In particular, it is often convenient to first saturate the ideal~$J_k$ with respect to
the condition that some points are distinct or that three of them are not aligned 
(when this is the case). 
Another important simplification that we adopt sometimes, makes use
of the action of~$\SO_3(\C)$: thanks to it, we can assume that one of
the point is either $(1: 0: 0)$ or $(1: \iii: 0)$; see \Cref{two_orbits}.

We start with the following lemma, which is extremely useful
to speed up the computations.

\begin{lemma}
\label{lemma:minors}
Let $l_1 < \cdots <l_n$ be $n$ indices (where $3 \leq n \leq 10$) and let $P = (A: B: C)$ be a point of the plane.
Construct three $1 \times n$ matrices $w_1$, $w_2$, $w_3$ by extracting the entries of position $l_1, \dotsc, l_n$ from~$\phi_1(P)$, $\phi_2(P)$, and~$\phi_3(P)$, respectively. If $L$ is a $(n-2) \times n$ matrix, set:
%
\[
  L_1 := \left( \begin{array}{c} w_1 \\ w_2 \\ L \end{array} \right), \quad
  L_2 := \left( \begin{array}{c} w_1 \\ w_3 \\ L \end{array} \right), \quad
  L_3 := \left( \begin{array}{c} w_2 \\ w_3 \\ L \end{array} \right)
\]
%
Then
%
\[
  B \det(L_1) = A \det(L_2), \quad
  C \det(L_1) = A \det(L_3), \quad
  C \det(L_2) = B \det(L_3)
\]
%
hence $(A: B: C) = \bigl( \det(L_1): \det(L_2): \det(L_3) \bigr)$.
\end{lemma}
\begin{proof}
The statement easily follows from the equality $C w_1 - B w_2 + A w_3 = 0$, which is a direct consequence of \Cref{eq:syzygy}.
\end{proof}

% \Cref{lemma:minors} justifies the following choice.

% \begin{definition}
% \label{definition:reduced_matrix_conditions}
% For $n$ points $P_1, \dotsc, P_n$ in the plane, the \emph{cut matrix of conditions} of $P_1, \dotsc, P_n$ is the submatrix of~$\Phi(P_1, \dotsc, P_n)$ whose rows are $\phi_1(P_1), \phi_2(P_1), \dotsc, \phi_1(P_n), \phi_2(P_n)$.
% \end{definition}

Next, we point out a property of the lines that are tangent to the isotropic conic~$\iso$.
First of all, we fix some notation that is used throughout the paper.

\begin{definition}
\label{definition:sigma}
For $P_1 = (A_1: B_1: C_1)$ and $P_2 = (A_2: B_2: C_2)$, we set
%
\begin{equation}
\label{formula:sigma}
\begin{aligned}
  \sigma(P_1, P_2) &= \scl{P_1}{P_1} \scl{P_2}{P_2} - \scl{P_1}{P_2}^2 \\
   &= s_{11}s_{22}-s_{12}^2
\end{aligned}
\end{equation}
%
This is a bihomogeneous polynomial of bidegree~$(2,2)$ on $\p^2 \times \p^2$.
\end{definition}

\begin{rmk}
\label{rmk:sigma_discr}
The form~$\sigma$ is the discriminant of the intersection between the line~$P_1 \vee P_2$ and~$\iso$.
Indeed, if $u_1 P_1 + u_2P_2$ is a generic point on the line $P_1 \vee P_2$, the relation 
%
\[
  \scl{u_1 P_1 + u_2 P_2}{u_1 P_1 + u_2 P_2} =0.
\]
%
gives the intersection $(P_1 \vee P_2) \cap \iso$.
The discriminant of the latter (as a polynomial in $u_1$ and $u_2$) is $4 \bigl( \scl{P_1}{P_2}^2 - \scl{P_1}{P_1} \scl{P_2}{P_2} \bigr)$, and this proves the claim.
\end{rmk}

\begin{prop}
\label{proposition:sigma_tangency}
Let $P_1$, $P_2$ be two distinct points in the plane and let $r=P_1 \vee P_2$.
Then the following are equivalent:
%
\begin{enumerate}
  \item $\sigma(P_1, P_2) = 0$;
  \item $\sigma(Q_1, Q_2) = 0$ for all pairs of distinct points $Q_1, Q_2 \in r$;
  \item the line~$r$ is tangent to~$\iso$ at some point;
  \item there exists a point $T \in r$ such that $T \in \iso$ and $\scl{T}{Q} = 0$ for all $Q \in r$, $Q \neq T$.
\end{enumerate}
%
\end{prop}
\begin{proof}
By \Cref{rmk:sigma_discr}, we have that $\sigma(P_1, P_2) = 0$ if and only if $r$ is tangent to~$\iso$ in a point~$T$; this shows that the first three items are equivalent.

Now, if $\sigma(P_1, P_2) = 0$, then $r$ is tangent to~$\iso$ at a point~$T$, hence $\scl{T}{T} = 0$.
Moreover, by the second item $\sigma(T, Q) = 0$ for all $Q \in r$ with $Q \neq T$, and by the definition of $\sigma$ the claim follows.
The converse is immediate.
\end{proof}

\begin{prop}
\label{proposition:three_distinct_ranks}
%%%  conto si trova su file prop47_5ago.sage
Let $P_1, P_2, P_4$ be three distinct points of the plane. Then:
%
\begin{itemize}
  \item $5 \leq \rk \,\Phi(P_1, P_2, P_4) \leq 6$;
  \item
  $\rk \, \Phi(P_1, P_2, P_4) = 5$ if and only if $P_1, P_2, P_4$
  are aligned and the line joining them is tangent to~$\iso$
  in one of the three points.
\end{itemize}
%
\end{prop}
\begin{proof}
The rank of $\Phi(P_1, P_2, P_4)$ cannot be larger than~$6$ by \Cref{remark:rank_2}.
\nb{03}{F1} gives that $\Phi(P_1, P_2, P_4)$ cannot have rank $4$ (since its minors of order~$5$ never vanish all at the same time) and it also proves the second item.
\end{proof}

\begin{prop}
\label{proposition:three_aligned_plus_one}
Let $P_1, P_2, P_3, P_4$ be four distinct points of the plane such that
$P_1, P_2, P_3$ are aligned, let $r = P_1 \vee P_2 \vee P_3$ and assume $P_4 \not \in r$.

If $\rk \,\Phi(P_1, P_2, P_3, P_4) \leq 7$ then $r$ is tangent to~$\iso$ in one of the three points~$P_1$, $P_2$, and~$P_3$.
\end{prop}
\begin{proof}
\nb{03}{F2} shows that the condition imposed by the vanishing of all order $8$ minors of the matrix~$\Phi(P_1, P_2, P_3, P_4)$ is equivalent to the statement.
\end{proof}

% \begin{rmk}
% As we clarified in \Cref{lemma: no 4 aligned}, four aligned eigenpoints are never contained in a $0$-dimensional eigenscheme.
% In \Cref{lemma:four_points_on_line}, we will show that if an eigenscheme contains four distinct aligned eigenpoints, then it contains the whole line joining them.
% \end{rmk}

% \begin{prop}
% \label{proposition:four_aligned}
% Let $Q_1, \dotsc, Q_4$ be four distinct aligned points of the plane and
% let $r$ be the line through them. Then:
% %
% \begin{itemize}
%   \item $6 \leq \rk \,\Phi(Q_1, \dotsc, Q_4) \leq 7$;
%   \item $\rk \,\Phi(Q_1, \dotsc, Q_4) = 6$ if and only if $r$ is tangent to~$\iso$.
% \end{itemize}
% %
% The same statements hold when we consider $k \geq 4$ distinct aligned points $Q_1, \dotsc, Q_k$.
% \end{prop}
% \begin{proof}
% A symbolic computation shows that all the maximal minors of~$\Phi(Q_1, \dotsc, Q_4)$ are zero, so we get 
% $\rk \, \Phi(Q_1, \dotsc, Q_4) \leq 7$.
% Indeed, in view of \Cref{lemma:minors}, we can just check the order~$8$ minors of the cut matrix of conditions of $Q_1, \dotsc, Q_4$;
% it turns out that they are all zero.

% On the other hand, the rank of $\Phi(Q_1, \dotsc, Q_4)$ cannot be $5$:
% if this were the case, for any triple~$T$ extracted from the four points, the matrix of conditions would have rank~$5$, hence by \Cref{proposition:three_distinct_ranks} the line~$r$ would be tangent to~$\iso$ in an element of~$T$; this is a contradiction.
% So, the first item of the statement is proven.

% We proceed to prove the second item.
% We take the cut matrix of conditions of~$Q_1, Q_2, Q_3$, to which we add the row~$\phi_1(Q_4)$.
% We call this matrix~$M_1$.
% Similarly, we construct the matrices $M_2$ and~$M_3$.
% We consider the coordinates of the points as variables and we compute the ideal~$J_1$,
% the radical of the ideal of the maximal minors of~$M_1$, namely, the order~$7$ minors;
% similarly, we obtain $J_2$ and~$J_3$.
% The saturation of the ideal sum $J_1 + J_2 + J_3$ by the distinct point condition is a principal ideal generated by~$\sigma(r)$.
% Since the four points $Q_1, \dotsc, Q_4$ play symmetric roles, the statement is proven.
% \end{proof}

\subsection{The conditions \texorpdfstring{$\delta_1$}{delta1}, \texorpdfstring{$\bar{\delta}_1$}{deltabar1}, and \texorpdfstring{$\delta_2$}{delta2}}

We now define three expressions depending on the homogenous coordinates of a triple or on a $5$-tuple of points in the plane.
Such expressions will be crucial in describing what happens when we have aligned eigenpoints.

\begin{definition}
\label{definition:delta1}
We define a multihomogeneous polynomial of multidegree~$(2,1,1)$ on $\p^2 \times \p^2 \times \p^2$:
if $P_i = (A_i: B_i: C_i)$ for $i \in \{1, 2, 4\}$, we set
%
\begin{align*}
  \delta_1(P_1, P_2, P_4) &:=
  \scl{P_1}{P_1} \scl{P_2}{P_4} - \scl{P_1}{P_2}\scl{P_1}{P_4} =
  \scl{P_1\times P_2}{P_1 \times P_4} \\
  &\phantom{:}= s_{11} s_{24}-s_{12}s_{14} \,,
\end{align*}
%
where $\times$ denotes the cross product, see \Cref{equation:cross product}.
\end{definition}

\begin{rmk}
\label{rmk:delta1_meaning}
Geometrically, the condition $\delta_1(P_1, P_2, P_4) = 0$ corresponds to the orthogonality of the lines~$P_1 \vee P_2$ and~$P_1 \vee P_4$, if the three points are not collinear, while
$\delta_1(P_1, P_2, P_4) = 0$ implies that $\sigma (P_1,P_2)=0$ if they are collinear.
\end{rmk}

\begin{definition}
\label{definition:delta1b}
Let $P_1$, $P_2$ and~$P_3$ be distinct aligned points in the plane.
We define the polynomial
%
\begin{align*}
  \overline{\delta}_1(P_1, P_2, P_3) &:=
  \scl{P_1}{P_1} \scl{P_2}{P_3} + \scl{P_1}{P_2}\scl{P_1}{P_3} \\
  &\phantom{:}= s_{11} s_{24}-s_{12}s_{14}\,.
\end{align*}
%
\end{definition}

\begin{definition}
\label{definition:Vconf}
Let $P_1, P_2, P_3, P_4, P_5$ be five distinct points of the plane
such that $P_1, P_2, P_3$ and $P_1, P_4, P_5$ are aligned, and 
$P_1,P_2,P_4$ not aligned.
We call such a configuration a \emph{$V$- configuration}.
\end{definition}

\begin{definition}
Let $P_1, \dots, P_5$ be a $V$- configuration.
We define the polynomial
%
\begin{align*}
  \delta_2(P_1, P_2, P_3, P_4, P_5) &:=
  \scl{P_1}{P_2} \scl{P_1}{P_3} \scl{P_4}{P_5} -
  \scl{P_1}{P_4} \scl{P_1}{P_5} \scl{P_2}{P_3} \\
  &\phantom{:}= s_{12}s_{13}s_{45}-s_{14}s_{15} s_{23} \,.
\end{align*}
%
\end{definition}

The next results give a geometric description of the zero loci of the expressions just defined.
\begin{lemma}
\label{lemma:characteristics_d1_d2}
It holds
%
\begin{align}
\label{lemma_delta_case1}
  \delta_1(P_1, P_2, P_4) = 0 \mbox{ iff } &\scl{P_4}{s_{11}P_2-s_{12}P_1} = 0\\
  \mbox{iff } &\scl{P_2}{s_{11}P_4-s_{14}P_1} = 0 \,. \nonumber
\end{align}
%
\begin{align}
\label{lemma_delta_case2}
  \overline{\delta}_1(P_1, P_2, P_3) = 0 \mbox{ iff } &
  P_1 \mbox{ is on~$\iso$ and } P_1 \vee P_2 \vee P_3 \mbox{ is tangent to~$\iso$ in $P_1$, or} \\
  & P_3 = (s_{12}^2+s_{11}s_{22}) \, P_1 - 2s_{11}s_{12} \, P_2 \mbox{ or} \nonumber \\
  & P_2 = (s_{13}^2+s_{11}s_{33}) \, P_1 - 2s_{11}s_{13} \, P_3 \nonumber
\end{align}
%
\end{lemma}
\begin{proof}
 The proof is a symbolic verification; one can find it in \nb{03}{F3}.
\end{proof}

In general, if we fix $P_1,P_2,P_4,P_5$, the condition
$\delta_2=0$ uniquely determines $P_3$. However, there are some exceptions, which are listed in the following proposition.

% Suppose the point~$P_1$ is fixed. In general, in order to have
% $\delta_2(P_1, P_2, P_3, P_4, P_5) = 0$, we can fix three of the other points and determine the fourth one, 
% by observing that $\delta_2=0$ gives a linear equation in its coordinates.
% There are, however, some exceptions. 
% From the definition of~$\delta_2$, we see for instance that
% if $P_2$ and~$P_4$ satisfy $\{s_{12}=0, s_{14}=0\}$ or $P_2$ and~$P_5$ satisfy $\{s_{12}=0, s_{15}=0\}$ or $P_2$ and~$P_3$ satisfy
% $\{s_{12}=0, s_{23}=0\}$, etc., then $\delta_2$ is zero,
% despite of the value of the other points.
% To be more precise, we have:
%
\begin{prop}
\label{prop:definitionP3}
For five points $P_1, \dots, P_5$ in a $V$- configuration, it holds that
$\delta_2(P_1, P_2, P_3, P_4, P_5) = 0$ if and only if (up to a permutation of $P_2, \dots, P_5$) at least one of the following conditions
is satisfied:
%
\begin{enumerate}
  \item $s_{12} = 0$ and $s_{14} = 0$;
  \label{defP3_1}
  \item $s_{12} = 0$ and $s_{22} = 0$;
  \label{defP3_2}
  \item $\sigma(P_1, P_2) = 0$ and $\sigma(P_1, P_4) = 0$;
  \label{defP3_3}
  \item $P_3 = (s_{14}s_{15}s_{22}-s_{12}^2s_{45})P_1  +s_{12}(s_{11}s_{45}-s_{14}s_{15})P_2$.
  \label{defP3_4}
\end{enumerate}
%
\end{prop}

\begin{proof}
%% file: whenP3isNotDefined.sage
Condition~(\ref{defP3_4}), when defined (i.e., when the coefficients of~$P_1$ and~$P_2$ are not zero), easily comes from the definition of~$\delta_2$.
The point~$P_3$ is not defined by that formula if and only if
$s_{14}s_{15}s_{22}-s_{12}^2s_{45}=0$ and $s_{11}s_{12}s_{45}-s_{12}s_{14}s_{15}=0$.
Hence, we study the ideal generated by these two polynomials, together with the
polynomial $s_{12}s_{13}s_{45}-s_{14}s_{15} s_{23}$ which defines
$\delta_2$ in terms of the $s_{ij}$. The corresponding
ideal decomposes into several ideals which are, up to a permutation of
the indices: $(s_{12}, s_{14})$, $(s_{12}, s_{22})$ and
%
\[
  J = (s_{13}s_{22} - s_{12}s_{23}, s_{14}s_{15} - s_{11}s_{45}, s_{12}s_{13} -
  s_{11}s_{23}, s_{12}^2 - s_{11}s_{22}) \,.
\]
%
Then we give generic coordinates to the five
points and substitute them into the ideal $J$. It is possible to see that
$J$ and the ideal $\bigl(\sigma(P_1, P_2), \sigma(P_1, P_4)\bigr)$ are equal (up to
saturations w.r.t. the condition that all points are distinct and that $P_1, P_2, P_4$ are not aligned).
These computations are carried in \nb{03}{F4}.
\end{proof}
%
\begin{rmk}
In the first case of \Cref{prop:definitionP3}, the corresponding cubic
is studied in \Cref{further_alignments}, configuration $(C_5)$.
In the second case, the conditions $s_{12}=0$ and $s_{22}=0$ imply $\sigma(P_1, P_2) = 0$ and $P_2\in \iso$, so the line~$P_1 \vee P_2 \vee P_3$ is tangent to~$\iso$ in~$P_2$
and this case was considered in~\Cref{proposition:three_distinct_ranks}; in particular,
the matrix $\Phi(P_1, P_2, P_3)$ has rank~$5$, so
$\Phi(P_1, \dots, P_5)$ has rank $\le 9$, regardless of
$P_4$ and~$P_5$.
The third case gives either that all the
points of the two lines of the $V$- configuration are eigenpoints (in case
$P_2 \not\in \iso$ or $P_4 \not\in \iso$) or the matrix $\Phi(P_1, \dots, P_5)$
has rank~$8$ (if $P_2, P_4 \in \iso$); see \Cref{rank_8} and \Cref{positive_dim}, respectively.
The last case is the generic one and expresses $P_3$ in terms of the remaining four points.
\end{rmk}
%
\begin{prop}
\label{prop:d1d2}
Let $P_1, \dots, P_5$ be a $V$- configuration. Then
%
\[
  \rk \,\Phi(P_1, \dots, P_5) \leq 9
  \quad \mbox{if and only if} \quad
  \delta_1(P_1, P_2, P_4) \cdot \delta_2(P_1, \dots, P_5) = 0 \,.
\]
%
\end{prop}
\begin{proof}
Let
%
\[
  P_1 = (A_1: B_1: C_1) \,, \quad
  P_2 = (A_2: B_2: C_2) \,, \quad
  P_4 = (A_4: B_4: C_4) \,,
\]
%
and $P_3 = u_1 \, P_1 + u_2 \, P_2$ and $P_5 = v_1 \, P_1 + v_2 \, P_4$ for some $u_1, u_2, v_1, v_2$.
We find, via symbolic computation, that the determinant of
the submatrix of the matrix of conditions of $P_1, \dotsc, P_5$ obtained by selecting the first two rows from each $\Phi(P_i)$ for $i \in \{1, \dotsc, 5\}$ is
%
\begin{gather}
\label{delta1delta2}
  A_1A_2A_4(u_1A_1+u_2A_2)(v_1A_1+v_2A_4) \cdot u_1^2u_2^2v_1^2v_2^2 \cdot D^5 \cdot
  \delta_1(P_1,P_2,P_4) \cdot \delta_2(P_1,\dots,P_5) \,,
\end{gather}
%
where $D$ is the determinant of the matrix whose rows are $P_1, P_2, P_4$.
As a consequence of \Cref{lemma:minors}, the order~$10$ minors of the matrix of conditions are polynomials of the form
%
\begin{gather*}
  X_1X_2X_4 X_3 X_5 \cdot u_1^2u_2^2v_1^2v_2^2 \cdot D^5 \cdot
  \delta_1(P_1,P_2,P_4) \cdot \delta_2(P_1,\dots,P_5) \,,
\end{gather*}
%
where $X_i$ varies among all coordinates of~$P_i$ for $i \in \{1, \dotsc, 5\}$.
Since each of $P_1, \dotsc, P_5$ has at least one non-zero coordinate, 
and $D$ is non-zero, as well as are non-zero $u_1, u_2, v_1, v_2$ (since we assume that $P_1, P_2, P_4$ are not aligned and the points are distinct), we have that
$\rk \, \Phi(P_1, \dots, P_5) \leq 9$ if and only if
%
\[
  \delta_1(P_1, P_2, P_4) \cdot \delta_2(P_1, \dots, P_5) = 0 \,. \qedhere
\]
%
\end{proof}

%% for the proof of the avobe lemma, see the first part of the file
%% rank_8_twoTangIso.sage"
%%
\begin{lemma}
\label{lemma:special_case_rank_8}
Let $P_1, \dots, P_5$ be a $V$- configuration and assume that
%
$
  s_{12} = 
  s_{22} = 
  s_{14} = 
  s_{44} = 0 \,.
$
%
Then the matrix $\Phi(P_1, \dots, P_5)$ has rank~$8$.
\end{lemma}
\begin{proof}
%% first part of the file rank_8_twoTangIso.sage
By \Cref{proposition:sigma_tangency},
the lines~$P_1 \vee P_2$ and~$P_1 \vee P_4$ are tangent to~$\iso$ in~$P_2$ and~$P_4$, respectively. 
The point~$P_1$ cannot be on~$\iso$, hence, using the
action of~$\SO_3(\C)$, we can assume $P_1 = (1: 0: 0)$.
Since every element of~$\SO_3(\C)$ leaves $\iso$ invariant,
when we transform the point~$P_1$ into $(1: 0: 0)$, 
we transform the points~$P_2$ and~$P_4$ into, respectively,
the points $(0: \iii: 1)$ and $(0: -\iii: 1)$ (which are the common points to~$\iso$ and the tangent lines through~$P_1$).
Therefore, it is enough to study the
specific configuration of the points:
%
\begin{gather*}
  P_1 = (1: 0: 0) \,, \quad P_2=(0: \iii: 1) \,, \quad P_3=(u_1, \iii u_2, u_2) \,, \\
  P_4 = (0: -\iii: 1) \,, \quad P_5 = (v_1, -\iii v_2, v_2) \,,
\end{gather*}
%
where $(u_1: u_2), (v_1: v_2) \in \p^1$.
\nb{03}{F5} proves that the matrix of conditions of this configuration has rank~$8$.
\end{proof}

The following result gives a complete description of the possible ranks of matrix of conditions for $V$- configurations.

\begin{theorem}
\label{theorem:rank_V}
Let $P_1, \dots, P_5$ be a $V$- configuration. Then we have:
%
\begin{enumerate}
  \item $8 \leq \rk \,\Phi(P_1, \dots, P_5) \leq 10$\,;
  \item $\rk \,\Phi(P_1, \dots, P_5) \leq 9$ if and only if
  $\delta_1(P_1, P_2, P_4) \cdot \delta_2(P_1, \dots, P_5) =0$\,;
  \item $\rk \,\Phi(P_1, \dots, P_5) = 8$ if and only if
  %
  \begin{itemize}
    \item $\delta_1(P_1, P_2, P_4) = 0$, \
    $\overline{\delta}_1(P_1, P_2, P_3) = 0$, \ 
    $\overline{\delta}_1(P_1, P_4, P_5) = 0$\,; or
    \item the line~$P_1 \vee P_2$ is tangent to~$\iso$ in~$P_2$ or~$P_3$
    and the line~$P_1 \vee P_4$ is tangent to~$\iso$ in~$P_4$ or~$P_5$; 
    moreover, in this case we have $\delta_1(P_1, P_2, P_4) \neq 0$.
  \end{itemize}
  %
  In particular, in both cases $\delta_2(P_1, P_2, P_3, P_4, P_5) = 0$ holds.
\end{enumerate}
%
\end{theorem}
\begin{proof}
%%% si basa sui file:
%%% rank_8_2_1_ii_0.sage e
%%% rank_8_1.sage
If the rank is $\leq 7$, from
\Cref{proposition:three_aligned_plus_one} applied to $P_1, P_2, P_3, P_4$ and $P_1, P_4, P_5, P_2$,
the lines~$P_1 \vee P_2$ and~$P_1 \vee P_4$ are tangent to~$\iso$ (the first in $P_2$ or $P_3$ and the second in $P_4$ or $P_5$).
Then, by \Cref{proposition:sigma_tangency} and \Cref{lemma:special_case_rank_8} we get a contradiction.
This shows the first item.

The second item is \Cref{prop:d1d2}.

We are left to proving the third item. By \Cref{lemma:special_case_rank_8} and \Cref{prop:d1d2}, each of the two conditions implies that the rank is~$8$.
\nb{03}{F6} proves the converse; in particular, it shows that if $P_1 \in \iso$, then $\rk \,\Phi(P_1, \dots, P_5) \geq 9$. In the second case, it is necessarily $\delta_1(P_1,P_2,P_4)
\neq 0$; indeed, in the case $P_2, P_4 \in \iso$, then 
%
\[
  \scl{P_2 +P_4}{P_2+P_4} = 
  \scl{P_2}{P_2} + \scl{P_4}{P_4}+2\scl{P_2}{P_4} =
  2\scl{P_2}{P_4} \neq 0 \,,
\]
%
as the point~$P_2 + P_4$ is different from both~$P_2$ and~$P_4$, and the line~$P_2 \vee P_4$ has no other intersection points with~$\iso$. It follows that
$\delta_1 (P_1,P_2,P_4)=s_{11}s_{24} \neq 0$, as $P_1 \not\in \iso$. The cases $P_3 \in \iso$, respectively $P_5\in \iso$, are similar.
Therefore, if $\rk \, \Phi(P_1, \dots, P_5) = 8$, then one of the two conditions above must hold.
\end{proof}


\section{\texorpdfstring{$V$}{V}- configurations}
\label{V-configurations}

In order to get a $V$- configuration, there are two possible constructions:
%
\begin{itemize}
    \item[(a)] Recalling \Cref{lemma:characteristics_d1_d2}, it is quite easy to construct five points $P_1, \dots, P_5$ that are in a $V$- configuration
and such that $\delta_1(P_1, P_2, P_4)= 0$: the points~$P_1$
and~$P_2$ can be taken in an arbitrary way, $P_4$ has to be chosen in such
a way that it satisfies \Cref{lemma_delta_case1}
and~$P_3$ and~$P_5$ have to be chosen on the lines~$P_1 \vee P_2$ and~$P_1 \vee P_4$,
respectively. In particular, the corresponding locus of cubic curves
has dimension~$7$.
The construction of a random cubic
of five points as above, gives a smooth cubic curve whose $7$ eigenpoints
do not have other collinearities (in addition to those of a
$V$- configuration).
    \item[(b)] If we want a $V$- configuration that satisfies the condition
$\delta_2(P_1, \dots, P_5) = 0$, we choose $P_1$, $P_2$, $P_4$ arbitrarily;
we choose $P_3$ on the line~$P_1 \vee P_2$ and~$P_5$ on the line~$P_1 \vee P_4$ so that $P_3$ satisfies \Cref{prop:definitionP3}, \Cref{defP3_4}.
Also in this case, the locus of cubics is of dimension~$7$.
We will show that in this situation there are further alignments.
\end{itemize}

\subsection{\texorpdfstring{$V$}{V}- configurations of rank~\texorpdfstring{$9$}{9}}
\label{rank_9}

In this section, we consider five points $P_1, \dots, P_5$ in a $V$- configuration and we assume that the rank of $\Phi(P_1, \dots, P_5)$ is~$9$.
We prove that when $\delta_2(P_1, \dotsc, P_5) = 0$, if the unique corresponding cubic has a regular eigenscheme, then the two other eigenpoints $P_6$ and $P_7$ are aligned with $P_1$.

The following two results are instrumental to the symbolic computations in the proof of the main result.

\begin{lemma}
\label{lemma:construct_cubic}
Let $\mathcal{H}$ be a $9 \times 10$-submatrix of rank~$9$ of $\Phi(P_1, \dots, P_5)$.
Let $\mathcal{H}_i$ be the minor of~$\mathcal{H}$ given by deleting the $i$-th column ($i=1, \dots, 10$).
Then a polynomial defining the unique cubic~$C=V(f)$ with $P_1, \dotsc, P_5 \in \Eig{f}$ is 
%
\[
  f(X) = \sum_{i=1}^{10}(-1)^i\det(\mathcal{H}_i)\cdot \mathcal{B}_i 
  = \det \left( 
  \begin{array}{c} \mathcal{H} \\ \mathcal{B} \end{array}
  \right) \,.
\]
%
where $\mathcal{B}$ is the monomial vector as in \Cref{eq:vector_basis}. 
\end{lemma}
\begin{proof}
 The statement follows from Cramer's rule for the resolution of linear systems of maximal rank.
\end{proof}

The matrix~$\mathcal{H}$ allows one also to compute the generators of the ideal of the eigenscheme.

\begin{prop}
\label{proposition:geiser1}
The three minors given in \Cref{eq:def_minors} are proportional to
%
\[
  \det \left( 
  \begin{array}{c} \mathcal{H} \\ \phi_1(X) \end{array}
  \right),\quad
  \det \left( 
  \begin{array}{c} \mathcal{H} \\ \phi_2(X) \end{array}
  \right), \quad
  \det \left( 
  \begin{array}{c} \mathcal{H} \\ \phi_3(X) \end{array}
  \right)
\]
%
\end{prop}
\begin{proof} 
We have
%
\begin{align*}
  g_1 & = x \cdot \de_y f(X)- y \cdot \de_x f(X)  =
  x \cdot \de_y \det \left(
  \begin{array}{c} \mathcal{H} \\ \mathcal{B} \end{array} 
  \right) - y \cdot
  \de_x \det \left(
  \begin{array}{c} \mathcal{H} \\ \mathcal{B} \end{array}
  \right) \\
  & = \det \left(
  \begin{array}{c} \mathcal{H} \\ x \cdot \de_y \mathcal{B} - y \cdot \de_x \mathcal{B} \end{array}
  \right)  = \det \left(
  \begin{array}{c} \mathcal{H} \\ \phi_1(X) \end{array}
  \right)
\end{align*}
%
and similarly for $g_2 = x \cdot \de_z f(X)- z \cdot \de_x f(X)$
and $g_3 = y \cdot \de_z f(X)- z \cdot \de_y f(X)$.
\end{proof}

% \begin{es}
% Consider the following five points in a $V$- configuration:
% %
% \begin{gather*}
%   p_1 = (2: -1: 1) \,, \quad p_2 = (-1: 1: 3) \,, \quad p_4 = (3: 6: -1) \,, \\
%   p_3 = p_1+p_2 \, \quad p_5 = p_1+p_4 \,.
% \end{gather*}
% %
% They satisfy the condition $\delta_2(p_1, \dots, p_5) = 0$.
% We consider the $9\times 10$ matrix $\mathcal{H}$ whose rows
% (suitable rescaled) are $\phi_i(p_j)$ for $i = 1, 2$ and $j = 1, 2, 3, 4$, and $\phi_1(p_5)$
% %
% \[
%   \mathcal{H} =
%   \left(
%   \begin{array}{rrrrrrrrrr}
%     12 & 4 & -7 & 6 & 4 & 3 & -4 & 1 & 2 & 0 \\
%     -12 & 4 & -1 & 0 & 4 & -3 & 2 & 7 & -4 & 6 \\
%     -3 & 1 & 1 & -3 & 6 & 0 & -6 & -9 & -9 & 0 \\
%     -9 & 6 & -3 & 0 & 17 & -8 & -1 & -21 & -6 & -27 \\
%     0 & 1 & 0 & 0 & 0 & 4 & 0 & 0 & 16 & 0 \\
%     -12 & 0 & 0 & 0 & -31 & 0 & 0 & -56 & 0 & 48 \\
%     54 & 63 & 36 & -108 & -12 & -9 & 12 & 2 & -1 & 0\\
%     27 & 36 & 36 & 0 & 21 & 48 & 108 & -17 & -36 & 9 \\
%     336 & 328 & 119 & -588 & -56 & -33 & 56 & 7 & -4 & 0
%   \end{array}
%   \right)
% \]
% %
% The cubic with $p_1, \dots, p_5$ as eigenpoints is given by
% %
% \[
%   \det \left(
%   \begin{array}{c}
%     \mathcal{H}\\
%     \mathcal{B}
%   \end{array}
%   \right)
% \]
% %
% whose value is (after a suitable rescaling):
% %
% \[
%   766x^3 - 1176x^2y + 417xy^2 - 128y^3 + 1488x^2z - 1722xyz
%   + 597y^2z - 207xz^2 + 504yz^2 + 911z^3
% \]
% %
% The polynomials~$g_1$, $g_2$, and~$g_3$ are:
% %
% \begin{align*}
%   g_1 &= 392x^3 + 488x^2y - 656xy^2 + 139y^3 + 574x^2z + 594xyz - 574y^2z - 168xz^2 - 69yz^2 \\
%   g_2 &= 496x^3 - 574x^2y + 199xy^2 - 904x^2z + 1120xyz - 139y^2z - 81xz^2 + 574yz^2 + 69z^3 \\
%   g_3 &= 496x^2y - 574xy^2 + 199y^3 + 392x^2z - 416xyz + 464y^2z + 574xz^2 + 513yz^2 - 168z^3
% \end{align*}
% %
% and can be computed either using their definition,
% or~\Cref{proposition:geiser1}.
% The common zeros of~$g_1$, $g_2$, $g_3$ are the points $p_1, \dots, p_5$
% and the points $p_6 = (21: 34: -8)$ and $p_7 = (19: 35: -9)$.
% \end{es}

% \begin{rmk}
% When the matrix $\Phi(P_1, \dots, P_5)$ has rank~$8$, as in \Cref{rank_8}, the family of
% cubics with $P_1, \dots, P_5$ as eigenpoints is one dimensional, so the
% above construction can be modified, by substituting one of the rows of~$\mathcal{H}$ with a random row of elements of~$\C$.
% \end{rmk}



% Cubics whose eigenpoints satisfy the condition $\delta_2(P_1, \dotsc, P_5) = 0$ and $\rk \, \Phi(P_1, \dotsc, P_5) = 9$ exhibit further peculiarities. It turns out that the $7$ eigenpoints of such cubics satisfy the further
% condition that also $P_6$ and~$P_7$ are aligned with~$P_1$. To explain
% why, we proceed as follows.

% Consider again a matrix $\mathcal{H}$ of rank $9$ extracted from
% the matrix $\Phi(P_1, P_2, P_3, P_4, P_5)$, where now $P_3$ is determined by Equation~\eqref{} \textbf{(SERVE QUESTA CONDIZIONE O E' SUFFICIENTE AVERE RANGO 9 E $\delta_2 = 0$?)}
% and the three polynomials $g_1, g_2, g_3$ are constructed
% in \Cref{proposition:geiser1}. The condition
% $\rk \,(\mathcal{H}) = 9$ ensures that
% $g_1, g_2$ and~$g_3$ cannot be zero and are polynomials of degree
% $3$ in $x, y, z$.
% The computation shows that three polynomials have a common factor,
% hence are of the form $\Omega \cdot g_1$, $\Omega \cdot g_2$, $\Omega \cdot g_3$, where
% $\Omega$ is a polynomial in the variables
% $A_1, B_1, C_1, A_2, B_2, C_2, A_4, B_4, C_4, v_1, v_2$ but
% does not contain the variables $x, y, z$.
% Hence $g_1, g_2, g_3$ are again
% degree three polynomials in $x, y, z$, with coefficients in the variables
% $A_1, \dots, C_4, v_1, v_2$. Their common zeros are the eigenpoints of the
% cubic~$C$ since by construction they are proportional to the three minors of the matrix
% %
% \[
%   \left(
%   \begin{array}{ccc}
%     x & y & z \\
%     \partial_x f & \partial_x f & \partial_x f
%   \end{array}
%   \right)
% \]
% %

\begin{prop}
\label{proposition:G_split}
Assume that $P_1, \dots, P_5 $ satisfy $\delta_2(P_1, \dotsc, P_5) = 0$ and $\rk \, \Phi(P_1, \dotsc, P_5) = 9$. Then the cubic curve
of equation $A_1 g_3 - B_1 g_2 + C_1 g_1=0$ contains the lines~$P_1 \vee P_2$ and~$P_1 \vee P_4$, where $g_1, g_2, g_3$ are given in \Cref{eq:def_minors}.
% The polynomial $C_1G_1-B_1G_2+A_1G_3$ splits into three factors $r_1$, $r_2$, $r_3$, each linear in $x, y, z$.
\end{prop}
%We know that $G_1, G_2, G_3$ admit the syzygy $z \, G_1 - y \, G_2 + x \, G_3 = 0$.
%Then, due to the exactness of the Koszul complex of the regular sequence $(x,y,z)$ (see, for instance, \cite[Theorem 7.3.13]{Dolgachev}),
%we know that $G_1, G_2, G_3$ are the $2\times2$ minors of a matrix of the form
%%
%\[
% \left(
% \begin{array}{ccc}
%  x & y & z \\
%  H_1 & H_2 & H_3
% \end{array}
% \right)
%\]
%%
%where $H_1, H_2, H_3$ are quadratic ternary forms.

%Consider now the map
%%
%\[
%  \bar{\gamma} \colon \p^2 \setminus \Eig{f} \rightarrow \p^2, \quad P \mapsto (H_1(P): H_2(P): H_3(P)).
%\]
%%
%This map is the composition of~$\gamma_{\Eig{f}}$ with a linear isomorphism.
%In particular, the map $\bar{\gamma}$ contracts both the lines $P_1 +P_2$ and $P_1 +P_4$.
\begin{proof}
The equation of the line~$P_1 \vee P_2$ can be expressed in the following way
%
\begin{equation}
\label{eq:lineP1P2}
  \left\langle P_1 \times P_2, (x,y,z) \right\rangle = 0 \,;
\end{equation}
%
since $P_1 \vee P_2$ is contracted by the Geiser map~$\gamma_{\Eig{f}}$,
and by the description of the fibers given in \Cref{eq:fibers}, such a line is contained in the conic of equation
%
\[
  \left\langle P_1 \times P_2, (\de_x f, \de_y f, \de_z f) \right\rangle = 0 \,.
\]
%
As a consequence, for any point $\overline{P} = (\bar x: \bar y: \bar z)$ of the line from \Cref{eq:lineP1P2}, the following equations are satisfied:
%
\[
  \left\{
  \begin{array}{l}
    \left\langle P_1 \times P_2, P_1 \right\rangle = 0 \,,\\[2pt]
    \bigl\langle P_1 \times P_2, \overline{P} \bigr\rangle = 0 \,,\\[2pt]
    \bigl\langle P_1 \times P_2, \nabla f (\overline{P}) \bigr\rangle = 0 \,.
  \end{array}
  \right.
\]
%
It follows that the linear system in the variables $X,Y,Z$
%
\[
  \left\{
  \begin{array}{l}
    \bigl\langle P_1, (X,Y,Z) \bigr\rangle = 0 \,,\\[2pt]
    \bigl\langle \overline{P}, (X,Y,Z) \bigr\rangle = 0 \,,\\[2pt]
    \bigl\langle \nabla f (\overline{P}),
    (X,Y,Z) \bigr\rangle = 0 \,.
  \end{array}
  \right.
\]
%
admits $P_1 \times P_2$ as a non-zero solution,
hence the determinant of the $3 \times 3$ coefficient matrix with rows
 $P_1$, $\overline{P}$, $\nabla f (\overline{P})$ is zero. This is equivalent to saying that $A_1 g_3 - B_1 g_2 + C_1 g_1$ vanishes at all the points~$\overline{P}$ of the line~$P_1 \vee P_2$.

We argue similarly for line~$P_1 \vee P_4$.
\end{proof}
%
\begin{prop}
\label{proposition:third_alignment}
Let $P_1, \dots, P_5$ be a $V$- configuration of five points such that  
$\rk\, \Phi(P_1, \dots, P_5) = 9$ and
$\delta_2(P_1, \dots, P_5) = 0$. Let $V(f)$ be
the cubic with $P_1, \dots, P_5$
eigenpoints and suppose $\Eig{f}$ is 
regular. Then the two other eigenpoints are aligned with~$P_1$.
\end{prop}
\begin{proof}
According to \Cref{proposition:G_split}, the polynomial
$C_1g_1-B_1g_2+A_1g_3$ splits into three linear factors $r_1$, $r_2$, and~$r_3$, which
correspond to the line~$P_1 \vee P_2$, the line~$P_1 \vee P_4$, and the line~$P_6 \vee P_7$. Using \Cref{proposition:geiser1}, Notebook \nb{04}{F1} shows that~$P_1\in P_6 \vee P_7$.
\end{proof}


\subsection{\texorpdfstring{$V$}{V}-configurations of rank~\texorpdfstring{$8$}{8}}
\label{rank_8}
%% file contiCasoDegenere2.sage and file conf_sigma12_sigma14.sage
%% molti altri file via via indicati nei commenti di latex
%
In this section, we study the possible configurations of
eigenpoints for cubics with a $V$- configuration
and $\rk \, \Phi(P_1, \dots, P_5) = 8$. According
to~\Cref{theorem:rank_V}, we have to distinguish two cases
%
\begin{gather}
  \delta_1(P_1, P_2, P_4)=\overline{\delta}_1(P_1, P_2, P_3) =
  \overline{\delta}_1(P_1, P_4, P_5) = 0 \,, \text{or} 
  \label{rk8_1} \\
  \sigma(P_1, P_2) = \sigma(P_1, P_4) = 0 \ \ \mbox{and} \ \ s_{22} = s_{44} = 0 \,;
  \label{rk8_2}
\end{gather}
%
the latter condition means that \emph{the two lines of
the $V$- configuration are tangent to $\iso$ in $P_2$ and~$P_4$}.
%
To start, we characterize when, in a $V$- configuration
$P_1, \dotsc, P_5$, the point~$P_1$ is singular.

\begin{prop}
\label{proposition:P1_sing}
Let $P_1, \dots, P_5$ be a $V$- configuration and let
$C = V(f)$ be a cubic with
$P_1, \dots, P_5\in \Eig{f}$. Then $P_1$ is
singular for~$C$ if
and only if $P_1$ is not on $\iso$ and one of the following conditions
is satisfied:
%
\begin{enumerate}
  \item $\delta_1(P_1, P_2, P_4) = 0$ and $\overline{\delta}_1(P_1, P_2, P_3) = 0$;
  \item $\delta_1(P_1, P_2, P_4) = 0$ and $\overline{\delta}_1(P_1, P_4, P_5) = 0$;
  \item $\overline{\delta}_1(P_1, P_2, P_3) = 0$ and
  $\overline{\delta}_1(P_1, P_4, P_5) = 0$.
\end{enumerate}
%
\end{prop}
\begin{proof}
%% file: singularCubic_1ii0.sage e singularCubic_100.sage
See Notebook \nb{04}{F2}.
\end{proof}

\begin{prop}
\label{prop:char_rank_8}
  Let $P_1, \dotsc, P_5$ be a $V$- configuration such that \Cref{rk8_1} holds.
  Then $P_4$ is orthogonal to $s_{11} \, P_2 - s_{12} \, P_1$ and, up to swapping $2 \leftrightarrow 3$ and $4 \leftrightarrow 5$, it holds
  %
  \[
   P_3 = (s_{12}^2+s_{11}s_{22}) \, P_1 - 2s_{11}s_{12} \, P_2 \,, \quad
   P_5 = (s_{14}^2+s_{11}s_{44}) \, P_1 - 2s_{11}s_{14} \, P_4 \,.
  \]
  %
\end{prop}
\begin{proof}
  The statement about~$P_4$ follows from \Cref{lemma_delta_case1}. 
  From \Cref{lemma_delta_case2}, we obtain
  %
  \[
  \left\{
  \begin{array}{ll}
  (L_1) & \left\langle P_1, P_1 \right\rangle = \sigma(P_1, P_2) = 0 \,, \text{ or} \\
  (L_2) & P_3 = (s_{12}^2+s_{11}s_{22}) \, P_1 - 2s_{11}s_{12} \, P_2 \,, \text{ or} \\
  (L_3) & P_2 = (s_{13}^2+s_{11}s_{33}) \, P_1 - 2s_{11}s_{13} \, P_3
  \end{array}
  \right.
  \quad \text{and} \quad
  \left\{
  \begin{array}{ll}
  (R_1) & \left\langle P_1, P_1 \right\rangle = \sigma(P_1, P_4) = 0 \,, \text{ or} \\
  (R_2) & P_5 = (s_{14}^2+s_{11}s_{44}) \, P_1 - 2s_{11}s_{14} \, P_4 \,, \text{ or} \\
  (R_3) & P_4 = (s_{15}^2+s_{11}s_{55}) \, P_1 - 2s_{11}s_{15} \, P_5
  \end{array}
  \right.
  \]
  %
  Conditions $(L_1)$ and $(R_1)$ are not compatible: indeed, together they imply $P_1 \vee P_2 = P_1 \vee P_4$. 
  Notebook \nb{04}{F3} shows that $(L_1)$ is not compatible with $(R_2)$ or $(R_3)$ and $(R_1)$ is not compatible with $(L_2)$ or $(L_3)$.
  The possibilities that are left show the statement.
\end{proof}

\begin{rmk}
From \Cref{prop:char_rank_8}, it follows that if we choose
% The three equalities of~\Cref{rk8_1} give that,
% using \Cref{lemma_delta_case2}, either $P_1$ is on~$\iso$ and
% $P_1 \vee P_2$ is the tangent line to~$\iso$ in~$P_1$ (but this is not compatible
% with the equation $\overline{\delta}_1(P_1, P_4, P_5)=0$), or,
% from \Cref{lemma_delta_case1}, we can choose the points as follows:
%
\begin{itemize}
  \item $P_1$ and $P_2$ in an arbitrary way;
  \item $P_4$ in the $\p^1$
  space of points orthogonal to $s_{11} \, P_2 - s_{12} \, P_1$;
  \item $P_3 = (s_{12}^2+s_{11}s_{22}) \, P_1 - 2s_{11}s_{12} \, P_2$;
  \item $P_5 = (s_{14}^2+s_{11}s_{44}) \, P_1 - 2s_{11}s_{14} \, P_4$;
\end{itemize}
%
the matrix $M = \Phi(P_1, \dots, P_5)$ has rank $8$
%% verifica nel file matrix_of_order_8.sage
and therefore $\dim \Lambda(M) = 1$ and the dimension of the variety
of the corresponding cubics is $6$.
Moreover, from~\Cref{proposition:P1_sing} all these cubics
are singular in $P_1$.
If we take a random point of this variety, the corresponding
cubic has $7$ distinct eigenpoints (and is as expected, singular in $P_1$) and the
$7$ points do not satisfy other collinearities in addition to those of the
$V$- configuration.
In particular, there is no alignment $(P_1, P_6, P_7)$.
Since \Cref{rk8_1} implies $\delta_2(P_1, \dots, P_5) = 0$, the hypothesis ``rank $9$'' in~\Cref{proposition:third_alignment} is necessary. An example can be found in \nb{04}{F4} \textbf{(AGGIUNGERE)}.
\end{rmk}

\begin{rmk}
\label{rmk:particular_cases}
In general, a cubic having eigenpoints $P_1, \dots, P_5$ in a $V$- configuration satisfying \Cref{rk8_1} has only the two alignments $(P_1, P_2, P_3)$ and $(P_1, P_4, P_5)$ among its eigenpoints;
in particular, it does not have the collinearity $(P_1, P_6, P_7)$.
Since \Cref{rk8_1} implies $\delta_2(P_1, \dots, P_5) = 0$, the hypothesis ``rank $9$'' in~\Cref{proposition:third_alignment} is necessary.
\end{rmk}

Concerning possible sub-cases with further collinearities of the points,
the following results hold:
%%%%
%% tutti i conti si trovano nei file:
%% three_deltas_I.sage
%% three_deltas_II.sage
%% three_deltas_III.sage
%% three_deltas_IV.sage
%% c'e' anche il file three_d_alt.sage che puo' essere di aiuto
%% ma non e' essenziale
%%
%% c'e' anche il file three_d_alt.sage che calcola tutti gli autopunti
%% in configurazione (8) nel caso dei 3 delta = 0.
%%
\begin{prop}
\label{three_d_three_alignments}
If $P_1, \dots, P_5$ satisfy \Cref{rk8_1},
then, in the pencil $\Lambda \bigl( \Phi(P_1, \dotsc, P_5)\bigr)$ there is
a cubic curve with $7$ eigenpoints with the following three alignments:
%
\[
 (P_1, P_2, P_3) \,, \quad (P_1, P_4, P_5) \,, \quad \text{and} \quad (P_1, P_6, P_7) \,.
\]
%
No choices of $P_1, \dots, P_5$ allow one to obtain further alignments of the $7$ eigenpoints.
\end{prop}
%
\begin{proof}
%% file: three_deltas_I.sage and three_deltas_II.sage
As observed in the proof of~\Cref{theorem:rank_V}, if $P_1 = (1: \iii: 0)$,
%% Vedi file rank_8_2_1_ii_0.sage+)
the matrix
$\Phi(P_1, \dots, P_5)$ cannot have rank smaller than $9$, so the only
case to consider is $P_1 = (1: 0: 0)$.
Notebook \nb{04}{F5} provides the proof of the statement.
\end{proof}
%
Moreover, we have:
%
\begin{prop}
\label{prop:d2_6allin}
If the five points $P_1, \dots, P_5$ satisfy \Cref{rk8_1}
and if we impose the condition that there is an eigenpoint, say $P_6$, aligned with $P_2$ and~$P_4$, then the eigenpoints satisfy all these
alignments:
%
\[
  (P_1, P_2, P_3), (P_1, P_4, P_5),
  (P_2, P_4, P_6), (P_2, P_5, P_7),
  (P_3, P_4, P_7), (P_3, P_5, P_6) \,.
\]
%
Hence the points~$P_6$ and~$P_7$ are determined by $P_1, \dots, P_5$
since
$P_6 = (P_2 \vee P_4) \cap (P_3 \vee P_5)$
and $P_7 = (P_3 \vee P_4) \cap (P_2 \vee P_5)$.
A similar result holds if we take $P_3$ in place of~$P_2$ or~$P_5$
in place of~$P_4$.
\end{prop}
%
\begin{proof}
%% file three_deltas_III.sage, three_deltas_IV.sage
We define the points $P_1, \dots, P_5$ as in the previous proposition, then 
we consider a point of the line $P_2 \vee P_4$ and we impose that it is 
an eigenpoint (i.e., we impose that $\rk \, \Phi(P_1, \dotsc, P_6) < 10$).
This request defines a point~$P_6$ and it turns out that $P_6$ is also aligned 
with~$P_3$ and~$P_5$. We then define $P_7$ as the point 
$(P_2 \vee P_5) \cap (P_3 \vee P_4)$. It turns out that 
$\rk \, \Phi(P_1, \dots, P_7) < 10$, so $P_7$ is an eigenpoint. The seven 
points $P_1, \dotsc, P_7$ satisfy the above collinearities.
This is shown in Notebook \nb{04}{F6}.
\end{proof}

\Cref{three_d_three_alignments} and \Cref{prop:d2_6allin} exhaust all the possible configurations
of collinearities in case of \Cref{rk8_1}.
Now we consider the \Cref{rk8_2}.
%% La dimostrazione della proposizione prop:rk8_2B si trova nel file
%% rank_8_twoTangIso.sage
%% Il file confV_tg_iso.sage fa il caso generale (cioe' quando P1
%% e' il punto (A1, B1, C1) e non (1, 0, 0)) e serve per provare
%% le formule realtive a P6 e P7 nella configurazione (5).
%% Il file confV2_tg_iso.sage contiene i conti per la configurazione (8)
%% e prova la formula P6 = s15*P3+s13*P5

%
A generic cubic which satisfies \Cref{rk8_2} is
a cubic of the one-dimensional linear system
$\Lambda\bigl(\Phi(P_1, \dotsc, P_5)\bigr)$, where:
%
\begin{itemize}
  \item $P_1$ is any point of the plane (not on $\iso$);
  \item $P_2$ and $P_4$ are the two tangency points to $\iso$ given by the tangents from~$P_1$;
  \item $P_3$ is any point on the line~$P_1 \vee P_2$ (different from~$P_1$ and~$P_2$)
  and $P_5$ is any point on the line~$P_1 \vee P_4$ (different from~$P_1$ and~$P_4$).
\end{itemize}
%
The variety of all the cubics with a $V$- configuration
that satisfies condition~(\ref{rk8_2}) is therefore five-dimensional.
The five points satisfy the condition $\delta_2(P_1, P_2, P_3, P_4, P_5) = 0$.
%% 2+1+1+1, l'ultimo 1 perche' rango matrice = 8

The reciprocal position of the eigenpoints of the cubics of this family
is described by the following:
%
\begin{prop}
\label{prop:rk8_2B}
The generic cubic of the family of cubics satisfying \Cref{rk8_2}
has seven eigenpoints with the alignments:
%
\[
  (P_1, P_2, P_3), \ (P_1, P_4, P_5), \ (P_1, P_6, P_7)
\]
%
Among these points we have the relation
$\scl{P_1 \times P_6}{P_3\times P_5}=0$
(i.e., the lines~$P_1 \vee P_6$ and~$P_3 \vee P_5$ are orthogonal).
%% ortogonalita': provata nel file rank_8
In the family there is a sub-family of cubics whose eigenpoints have the following alignments:
%
\[
  (P_1, P_2, P_3),\ (P_1, P_4, P_5),\ (P_1, P_6, P_7),\ (P_2, P_4, P_6).
\]
%
In this case the points~$P_6$ and~$P_7$ are given by the formulas:
%
\begin{equation}
\label{formuleP6_P7}
P_6 = s_{15}s_{34}\, P_2 + s_{13}s_{25}\, P_4, \quad 
P_7 = s_{15}(s_{26}s_{46}+s_{24}s_{66})\, P_1+ s_{11}s_{24}s_{56}\, P_6
\end{equation}
%
and a sub-family whose eigenpoints have the following
alignments:
%
\[
  (P_1, P_2, P_3),\ (P_1, P_4, P_5), \ 
  (P_1, P_6, P_7),\ (P_2, P_5, P_6), \ 
  (P_3, P_4, P_6),\ (P_3, P_5, P_7)
\]
%
In this case, the point~$P_6$ (given, for instance, by the formula
$P_6 = s_{15} \, P_3 + s_{13} \, P_5$) is obtained as
the intersection~$(P_2 \vee P_5) \cap (P_3 \vee P_4)$ and consequently
$P_7 = (P_1 \vee P_6) \cap (P_3 \vee P_5)$.\\
No other collinearities among the eigenpoints are possible.
\end{prop}
%
\begin{proof}
%% file rank_8_twoTangIso.sage
By considering again the action of~$\SO_3(\C)$, we can assume:
%
\[
  P_1 = (1: 0: 0), \quad
  P_2 = (0: \iii: 1), \quad
  P_4 = (0: -\iii: 1)
\]
%
and $P_3 = u_1 \, P_1 + u_2 \, P_2$, $P_5 = v_1 \, P_1 + v_2 \, P_4$. 
In this situation, it
is easy to see from \Cref{lemma:construct_cubic} that $\Lambda\bigl(\Phi(P_1, \dots, P_5)\bigr)$
is the following pencil of cubic forms:
%
\[
  f(l_1, l_2) = l_1 \, f_1 + l_2 \, f_2
\]
%
where $l_1, l_2$ are parameters and
%
\begin{align*}
 f_1 & = x \cdot \left(2x^{2} + 3 y^{2} + 3 z^{2}\right)\\
  f_2 & = (y + \iii z) \cdot (y - \iii z)
  \cdot \bigl(2 \iii x u_{2} v_{2} + y (u_{2} v_{1}- u_{1} v_{2})
  - \iii z (u_{2} v_{1} + u_{1} v_{2})\bigr)
\end{align*}
%
%% C_1 and C_2 are in the file rank_8_twoTangIso.sage, see ***C1_C2***
%%
The explicit expression $f(l_1, l_2)$ allows us to
construct the ideal of the seven eigenpoints and (after
saturations w.r.t.\ the condition that the five points $P_1, \dotsc, P_5$ are distinct),
we get an ideal generated by a line~$r$ and a conic~$\Gamma$, whose zeros are
the points~$P_6$ and~$P_7$. Since the line~$r$ contains the point~$P_1$,
the points $P_1, P_6, P_7$ are collinear and $r$ results orthogonal to~$P_3 \vee P_5$.
In order to see if there are further collinearities among the
eigenpoints, we have to distinguish three cases (up to permutation
of the indices): $P_2, P_4, P_6$ are collinear or $P_2, P_5, P_6$ are
collinear, or $P_3, P_5, P_6$ are collinear. In the first case, $P_6$
is the point $r \cap (P_2\vee P_4)$. If it is an eigenpoint, it must be
on $\Gamma$, hence $l_1$ and~$l_2$ have a specific value which gives a sub-family
of~$f(l_1, l_2)$ and we can compute the explicit coordinates of
all the seven eigenpoints of the cubics of this family. The other two
cases can be studied in a similar way. See \nb{04}{F7}.
%% file rank_8_twoTangIso.sage, file confV_tg_iso.sage, file
%% confV2_tg_iso.sage.
\end{proof}
\begin{rmk}
\label{rmk:three_orthog} For further references, it is perhaps worth noting that the explicit 
construction of the 
points in~\Cref{prop:d2_6allin} and~\Cref{prop:rk8_2B} allows one to verify
several relations among them. In particular, the seven eigenpoints
of~\Cref{prop:d2_6allin} are such that
\[
\scl{P_1\times P_2}{P_1 \times P_4}=0, \quad 
\scl{P_2\times P_4}{P_3 \times P_5}=0, \quad 
\scl{P_3\times P_4}{P_2 \times P_5}=0;
\]
and it holds:
\[
P_4 = (P_2\times P_3)s_{25}s_{35}-s_{23}(P_2\times P_5)s_{35}+ s_{23}s_{25}(P_3\times P_5).
\]
In the first sub-family described in~\Cref{prop:rk8_2B} we have 
$P_1 = P_2 \times P_4$, hence 
\[
\scl{P_1\times P_2}{P_2 \times P_4}=0, \quad 
\scl{P_1\times P_4}{P_2 \times P_4}=0, \quad 
\scl{P_1\times P_6}{P_2 \times P_4}=0
\]
and in the second family:
\[
\scl{P_1\times P_4}{P_3 \times P_4}=0, \quad 
\scl{P_1\times P_2}{P_2 \times P_5}=0, \quad 
\scl{P_3\times P_5}{P_1 \times P_6}=0.
\]
hence it holds:
\[
P_3 = (P_1 \times P_5)s_{16}s_{56}-s_{15}(P_1\times P_6)s_{56}+s_{15}s_{16}(P_5 \times P_6) \,.
\]
\end{rmk}
%

% \begin{rmk}
% \label{rmk:delta1_and_delta2}
% It is easy to verify that in
% a $V$- configuration which satisfies \Cref{C8:cnd2}, we have
% $\delta_1(P_1, P_2, P_4)\neq 0$ and
% $\delta_2(P_1, P_2, P_3, P_4, P_5) = 0$,
% while in \Cref{C8:cnd1} both $\delta_1(P_1, P_2, P_4)$ and
% $\delta_2(P_1, P_2, P_3, P_4, P_5)$ are zero.
% \end{rmk}


\section{The locus of cubic ternary forms with an aligned triple of eigenpoints}
\label{locus_one_alignment}

The results of the previous sections allow one to determine the dimension and the degree of the locus of cubics having
at least one aligned triple of eigenpoints. 

\begin{definition}
\label{def:locus_L}
Let $\sU \subset \p^9$ be the following locus:
%
\[
  \sU:= \{[f]\in \p^9 \setminus \Delta_{3,3} \ | \ \Eig{f} \ \textrm{contains \ an \ aligned \ triple}\}, \,
\]
%
where $\Delta_{3,3}$ denotes the eigendiscriminant of \Cref{def: eigendiscriminant}, 
and define $\sL \subseteq \p^9$ as the closure of~$\sU$:
%
\[
  \sL := \overline \sU \subset \p^9 \,.
\]
%
\end{definition}

\begin{theorem}
\label{theorem:irreducible}
The variety~$\sL$ is an irreducible hypersurface.
\end{theorem}

\begin{proof}
We observe that as there exist
polynomials $f$ such that $\Eig{f}$ has no aligned triple, we have $\dim \sL \le 8$.

Consider the symmetric product $(\p^2)^{(3)}$ and set $\mathcal{AL} \subset (\p^2)^{(3)}$ to be the locus of unordered triples of distinct aligned points. Observe that $\mathcal{AL}$ is irreducible of dimension $5$ as its closure is the hypersurface given by the vanishing of the determinant of the order~$3$ matrix of the coordinates of three general points. Alternatively, the closure of the locus $\mathcal {AL}$ can be seen as the symmetric quotient of the projective line bundle $\sF \subset \p^2 \times \p^2 \times \p^2$ parametrized by the triples $(P_1, P_2, u_1 P_1 +u_2P_2)$, where $(u_1:u_2) \in \p^1$.

If we set
%
\[
  \sU':= \{[f]\in \p^9 \setminus \Delta_{3,3} \ | \ \Eig{f} \ \text{contains exactly $1$ aligned triple}\} \,,
\]
%
we have a surjective morphism
%
\[
  \alpha \colon \mathcal{U}' \to \mathcal{AL} \,,
\]
%
assigning to each $[f] \in \mathcal{U}'$ the unique triple of eigenpoints.
By \Cref{proposition:three_distinct_ranks}, the fibers of~$\alpha$ are projective linear systems of dimension~$3$ over the irreducible open subset
%
\[
  \mathcal{W} := \mathcal{AL}
  \setminus \{(P_1,P_2,P_3) \in \mathcal{AL}
  \ | \ \sigma(P_1,P_2)=0, s_{11} s_{22} s_{33}=0\}
\]
%
of aligned triples not lying
on a tangent line to the isotropic conic~$\iso$ and with tangency point one of the~$P_i$'s. 
By the Fiber Dimension Theorem, this implies that the open subset $\alpha^{-1} (\mathcal{W}) \subset \sU$ is irreducible of dimension~$8$, so $\dim \sL =8$.

To prove the irreducibility, we observe that since the eigenpoint condition corresponds to two linear conditions on the coefficients of cubic polynomials, by linear algebra methods we can determine a local parametrization of~$\sL$.
Specifically, given a general aligned triple $(P_1, P_2, u_1 P_1 +u_2P_2)$, the matrix of conditions $M=\Phi(P_1, P_2, u_1 P_1 +u_2P_2)$ has rank $6$, so the associated linear system
$\Lambda \bigl( \Phi(P_1, P_2, u_1 \, P_1 + u_2 \, P_2) \bigr)$ has dimension~$3$. We can then express $6$ coefficients of the general polynomial
$[f]\in \Lambda \bigl( \Phi(P_1, P_2, u_1 \, P_1 + u_2 \, P_2) \bigr)$ as rational functions of $P_1,P_2,u_1,u_2$ and of the remaining $4$ coefficients. The free coefficients depend on the position of a non zero minor of order six.
We claim that the last $3$ columns of the matrix
$M$ are always in the linear span of the first $7$ columns.

To prove the claim, set $c_0, \dots, c_9$ to be the $10$ columns of~$M$.
Moreover,
if the three components of~$P_1 \times P_2$ are called $\alpha, \beta, \gamma$, we set:
%
\[
  N_1 = \left(
  \begin{array}{ccc}
    \alpha & 0 & 0 \\
    0 & \beta & 0\\
    0 & 0 & \gamma
  \end{array}
  \right), \quad
  N_2 = \left(
  \begin{array}{ccc}
    0 & \alpha & 0 \\
    \gamma & 0 & 0\\
    0 & 0 & \beta
  \end{array}
  \right).
\]
%
The six
columns $c_0, c_1, c_2, c_4, c_5, c_7$ of~$M$ are linearly dependent. Indeed, if $L_1$ is the $9\times 3$ matrix whose columns are
$c_0, c_2, c_7$ and $L_2$ is given by the columns $c_1, c_4, c_5$,
then we have:
%
\begin{equation}
  (L_1 N_1 + 2 L_2N_2) (P_1 \times P_2) = 0 \,;
  \label{combLin}
\end{equation}
%
the linear combination of $c_0, c_1, c_2, c_4, c_5, c_7$ which is zero
is obtained by expanding this expression. This computation is contained in \nb{05}{F1}.

Similarly, the columns $c_1, c_2, c_3, c_5, c_6, c_8$ are linearly dependent
and~(\ref{combLin}) holds if $L_1$ in this case is $[c_1, c_3, c_8]$ and
$L_2$ is $[c_2, c_5, c_6]$. Finally, the columns
$c_4, c_5, c_6, c_7, c_8, c_9$ are linearly dependent and~(\ref{combLin})
holds if we take $L_1 = [c_4, c_6, c_9]$ and $L_2 = [c_5, c_7, c_8]$.

As a consequence, local parametrizations of~$\sL$ can be given by considering the following open subsets:
for any multiindex
%
\[
  I = \{i_1, \dots, i_6\} \subset \{0, 1, \dots, 6\} \,,
\]
%
we set
%
\begin{multline*}
  \sV_I :=
  \bigl\{
    [(P_1, P_2, u_1 P_1 +u_2P_2)] \in \mathcal{AL} \ | \ \text{the\ columns\ of\ } M \ 
    \text{relative \ to} \ I\ \text{are\ independent}
  \bigr\} \,.
\end{multline*}
%
Then for any $[(P_1, P_2, u_1 \, P_1 + u_2 \, P_2)] \in \sV_I$, by expressing any element
$[f] =[\mathcal{B} \cdot w_f]\in \Lambda \bigl( M \bigr)$ with
$ w_f = (b_0,
  b_1,
  b_2,
  b_3,
  b_4,
  b_5, b_6,b_7,b_8,b_9)$,
%
if $i\in I$ the coefficients $b_i$ depend on the free parameters $b_i=b_i(P_1,P_2,u_1,u_2,b_j,b_7,b_8,b_9)$, where $j \not \in I$, $0\le j \le 6$. Hence we can interpret $w_f=w_f(P_1,P_2,u_1,u_2,b_j,b_7,b_8,b_9)$ as a suitable rational function.
This leads to the definition of the rational map
%
\[
  \alpha_I \colon \sV_I \times \p^3 \to \sL, \quad
  \alpha_I (P_1,P_2,u_1,u_2,b_j,b_7,b_8,b_9)=
  [\mathcal{B} \cdot w_f(P_1,P_2,u_1,u_2,b_j,b_7,b_8,b_9)] \,.
\]
%
Next we observe that for any $I \subset \{0,\dots, 6\}$, the image
%
\[
  W_I := \alpha_I (\sV_I \times \p^3)
\]
%
is irreducible, being image of an irreducible variety, and it
contains an open subset. Indeed, assume for simplicity (the other cases are similar) that
$I=\{0,1,2,3,4,5\}$; by suitably adapting the argument of \Cref{lemma:construct_cubic}, we see that an element $[f]\in W_I$ is represented by the determinant of the
matrix given by the rows
$0,1,3,4,6,7$ of $M$ and the additional $4$ rows
%
\[
  \left(
  \begin{array}{cccccc}
    0 & \cdots & 1&0&0&-b_6 \\
    0 & \cdots & 0&1&0&-b_7 \\
    0 & \cdots & 0&0&1&-b_8 \\
    & & & \mathcal{B} & & \\
  \end{array}
  \right) \,.
\]
%
In particular, the coefficient $b_9$ of~$z^3$ is equal to the determinant of the order~$6$ minor of $M$ relative to the rows $0,1,3,4,6,7$ and the first $6$ columns. It follows that $W_I \supset \sU \cap \{ a_9 \neq 0\}$. 

Finally, we claim that the irreducible subvarieties $W_I$ for $I\subset \{0,\dots, 6\}$ have a common non-empty intersection
including internal points. Indeed, as one can verify with an explicit symbolic computation, an example of a polynomial
class~$[g]$ with an aligned triple of eigenpoints and satisfying
%
\[
  [g] \in \bigcap_I W_I \setminus \Delta_{3,3}
\]
%
is given by a random choice of~$P_1$, $P_2$ and $(u_1:u_2)$ and $b_7,b_8,b_9$.

Hence $\sU$ is covered by irreducible open subsets, each intersecting every other, so it is irreducible, and the same holds for its closure $\sL$.
\end{proof}

Next we want to determine the degree of the hypersurface $\sL$. We shall need the following.

\begin{prop}
The locus $\sV \subset \sL \subset \p^9$ of cubic forms with two or more aligned triples of eigenpoints has dimension
$\dim \sV = 7$.
\end{prop}
\begin{proof}
In $(\p^2)^5$, we consider the locus $\sW$ of $V$- configurations $(P_1,P_2,P_3,P_4,P_5)$. By a dimension count it is $\dim \sW=8$.
By \Cref{theorem:rank_V}, the five points are eigenpoints if and only if $\delta_1 (P_1,P_2,P_4) \cdot
\delta_2 (P_1,P_2,P_3,P_4,P_5)=0$. The two equations define two divisors in $\sW$, so the locus $\sE\sW$ of $V$- configurations constituted by eigenpoints has dimension $7$.

Finally, there is an open subset of $\sE\sW$ where the rank of the condition matrix is~$9$, hence $\sE\sW$ is birational to~$\sV$.
\end{proof}

As a consequence, we have the following result.

\begin{corollary}
\label{lemma:pencil_one_aligned}
If $[f],[g] \in \p^9 \setminus \Delta_{3,3}$ are general cubics,
then the general cubic in the pencil
$\lambda f + \mu g$ for $(\lambda: \mu) \in \p^1$ has no aligned triples of eigenpoints, and there is a finite number of cubics having exactly one aligned triple.
\end{corollary}

%\begin{definition}
% We define $\Delta \subset \sL$ to be the closure of the locus of cubics with at least two aligned triples of eigenpoints.
%\end{definition}
%
%\begin{prop}
%  The variety~$\Delta$ has dimension~$7$ and it is the union of two irreducible components~$\Delta_1$ and~$\Delta_2$.
%\end{prop}


If $g_1, g_2, g_3$ are the three minors of \Cref{eq:def_matrix} relative
to a cubic form~$f$, we denote by
%
\[
  \Sigma_f := \p \bigl( \left\langle g_1, g_2, g_3 \right\rangle \bigr),
\]
%
the net of cubics, whose base locus is the eigenscheme~$\Eig{f}$.


\begin{lemma}
\label{lemma:scroll}
If $f$ and $g$ are general cubics, then
%
\[
  \mathcal{N} := \bigcup_{(\lambda : \mu) \in \p^1} \Sigma_{\lambda f + \mu g} \subset \p^9
\]
%
is an embedding of a rational projective bundle and has degree~$3$.
\end{lemma}
\begin{proof}
Consider the projective bundle given by the family of planes
%
\[
  \mathcal{P} := \{ \Sigma_{\lambda f + \mu g} \, : \, (\lambda: \mu) \in \p^1 \} \subset \p^1 \times \p^9
\]
%
Observe that we can assume that $\Sigma_{\lambda f + \mu g} \cong \p^2$
for every $(\lambda:\mu) \in \p^1$. Indeed, it fails to be a net if and only if $g_1,g_2,g_3$ are linearly dependent, and this happens if and only if the three partials are linearly dependent. The latter condition is satisfied if and only if $V(\lambda f + \mu g)$ is a set of concurrent lines. In this case $[\lambda f + \mu g]$ belongs to a $5$ - dimensional locus.

Then $\mathcal{N}$ is the projection of~$\mathcal{P}$ on the second factor.
However, the map $\mathcal{P} \to \mathcal{N}$ contracts no subvariety of any plane of~$\mathcal{P}$, so either it is an embedding or it contracts some horizontal curve. In the latter case, all the planes of the family should intersect in at least one point. In particular, the two nets $\Sigma_f$ and~$\Sigma_g$ should have non-empty intersection.
If we denote by~$g_1$, $g_2$ and~$g_3$ the $2 \times 2$ minors relative to~$f$, and by~$h_1$, $h_2$ and~$h_3$ the ones relative to~$g$, the vectorial dimension of the linear span $\left\langle g_1, g_2, g_3, h_1, h_2, h_3 \right\rangle$ should be strictly less than $6$. This can be avoided, since such a condition corresponds to a proper closed subscheme of~$\p^9 \times \p^9$.

It follows that if $f$ and~$g$ are general enough, then $\mathcal{N}$ is a $3$-dimensional rational normal scroll in
%
\[
  \mathcal{N} \subset \p(\left\langle g_1, g_2, g_3, h_1, h_2, h_3 \right\rangle) \cong \p^5.
\]
%
Being a variety of minimal degree, its degree is $5+1-3 = 3$ by the classical result of \cite{EH}.
\end{proof}

\begin{theorem}
The degree of~$\sL$ is equal to
%
\[
  \deg \ \sL = 15 \,.
\]
%
\end{theorem}

\begin{proof}
We start by observing that a reduced $0$-dimensional eigenscheme contains an aligned triple if and only if the net of cubics
$\Sigma_f$ contains a cubic which splits in three lines, a so called \emph{triangle}. Moreover, if $f$ is general enough, we have exactly one aligned triple and the other $4$ points are in general position; in this case, the net~$\Sigma_f$ contains exactly three triangles, namely the unions of the line passing through the aligned triple and the reducible conics through the $4$ points in general position.

To determine the degree of~$\sL$, we consider a general pencil of cubic forms $\lambda f + \mu g$, and we compute the number of elements with associated net $\Sigma_{\lambda f + \mu g}$ containing a triangle.

To this aim, denote by $\mathcal{T} \subset \p^9$ the variety of triangles; it is a classical result that its dimension is~$6$ and its degree is~$15$,
see for instance \cite[Section~2.2.2]{3264}. We now consider the variety~$\mathcal{N}$ from \Cref{lemma:scroll}.
Note that, since each net containing a triangle, actually contains exactly $3$ of them, the number of nets of~$\mathcal{N}$ containing some triangle is given by
%
\[
  \frac{\mathcal{T} \cdot \mathcal{N}}{3} = \frac{{15} \cdot {3}}{3} = 15 \,.
\]
%
This implies that $\deg \sL = 15$.
\end{proof}

\section{Eigenschemes of positive dimension}
\label{positive_dim}

In this section, we consider positive dimensional eigenschemes. By \cite{BGV}, an eigenscheme cannot be of pure dimension~$1$, and the possible $1$-dimensional components are of degree~$1$ or~$2$.
In what follows, we shall determine the degree of the 
$0$-dimensional residual scheme in both cases.

\begin{prop}
\label{p2}
Let $C = V(f) \subset \p^2$ be a cubic curve.
Assume that $\dim \Eig{f} = 1$.
%
\begin{enumerate}
  \item If the $1$-dimensional component of~$\Eig{f}$ is a line~$\ell$,
  then the residual subscheme $Z := \mathrm{Res}_{\ell} \bigl( \Eig{f} \bigr)$ in~$\Eig{f}$ with respect to~$\ell$ has degree~$3$. Moreover, the scheme $Z$ is not contained in a line.
  \item If the $1$-dimensional component of~$\Eig{f}$ is a conic~$\Gamma$,
  then the residual subscheme $Z := \mathrm{Res}_{\Gamma} \bigl( \Eig{f} \bigr)$ in~$\Eig{f}$ with respect to~$\Gamma$ has degree~$1$.
\end{enumerate}
%
\end{prop}

The proof relies on the exactness of the Koszul complex associated with the regular sequence~$x$, $y$, and~$z$, specifically we use the following result (see \cite[Theorem~7.3.13]{Dolgachev}).

\begin{lemma}
\label{lem:Koszul}
Let $h_1,h_2,h_3\in\C[x,y,z]_d$ with $d \ge 1$. Then
%
\begin{equation}
\label{eq:linear_relazion}
  zh_1-yh_2+xh_3 = 0
\end{equation}
%
if and only if there exist $m_1,m_2,m_3\in\C[x,y,z]_{d-1}$ such that
%
\begin{equation}
\label{eq:minors_lemma}
  h_1 = xm_2-ym_3 \,, \qquad
  h_2 = zm_3-xm_1 \,, \qquad
  h_3 = ym_1-zm_2 \,.
\end{equation}
%
\end{lemma}

\begin{proof}
If $h_1,h_2,h_3$ satisfy \Cref{eq:minors_lemma}, then it is immediate to check that they satisfy \Cref{eq:linear_relazion} as well. 
Conversely, assume that \Cref{eq:linear_relazion} holds and let $R = \C[x,y,z]$.
The Koszul complex in the ring~$R$ is an exact sequence of $R$-modules
%
\[
  0 \to R\xrightarrow{\alpha} R^{\oplus 3} \xrightarrow{\beta} R^{\oplus 3} \xrightarrow{\gamma} R \to R/(x,y,z) \to 0 \,,
\]
%
where the maps are $\alpha(p) = (p x, p y, p z)$, $\gamma (w_1,w_2,w_3) = w_1 x + w_2 y + w_3 z$ and $\beta$ is defined by the matrix
%
\[
  \left(
  \begin{array}{ccc}
    0 & -z & y\\
    z & 0 & -x\\
    -y & x & 0 \\
  \end{array}
  \right) \,.
\]
%
The syzygy $zh_1-yh_2+xh_3=0$ implies that $(h_3, -h_2, h_1)$ is in the kernel of~$\gamma$,
and since the Koszul complex is exact, the triple $(h_3,-h_2, h_1)$ lies in the image of~$\beta$.
It follows that there exist $m_1,m_2,m_3 \in \C[x,y,z]_{d-1}$ such that $\beta (m_3,m_2,m_1)=(h_3,-h_2, h_1)$,
so \Cref{eq:minors_lemma} holds.
\end{proof}

Now we are in the position to prove \Cref{p2}.

\begin{proof}[Proof of \Cref{p2}]
Let $g_1$, $g_2$ and~$g_3$ be the order~$2$ minors determining the eigenscheme of~$f$, and let $g$ be the greatest common factor.
By writing
%
\[
  g_i = g \, h_i, \quad i=1,2,3
\]
%
we have that the residual scheme is defined by the ideal
$(h_1,h_2,h_3)$. Moreover, the linear
syzygy between the generators~$g_i$ gives rise to the syzygy:
%
\[
  z\, h_1 - y\, h_2 + x\, h_3 = 0 \,.
\]
%
By \Cref{lem:Koszul}, the triple $(h_1,h_2,h_3)$ is the triple of order two minors of a matrix
%
\[
  \begin{pmatrix}
    x & y & z \\
    m_3 & m_2 & m_1
  \end{pmatrix} \,.
\]
%
for suitable forms $m_i \in \C[x,y,z]_r$, where $r =2 - \deg g \ge 0$.
By the assumption that $g$ is the greatest common factor of the three minors $g_1$, $g_2$ and~$g_3$, the zero scheme $Z$ of $(h_1,h_2,h_3)$ is $0$-dimensional, and
by \Cref{thm:nonempty} its degree is~$3$ if $r=1$. Observe that $Z$, being intersection of three conics with no common component, is not contained in a line. 

Finally, if $r=0$, the triple $(h_1,h_2,h_3)$ corresponds to three linear forms belonging to a pencil, hence the zero locus is a point.
\end{proof}


\begin{es}
Consider the form
%
\[
  f(x, y, z) = x^2 (y - z) \,.
\]
%
In the language of \Cref{p2} and its proof, we have $\ell=x$,
%
\[
 h_1 = x^2-2y^2+2yz \,, \quad 
 h_2 = -x^2-2yz+2z^2\,, \quad 
 h_3 =-x(y+z) \,.
\]
%
The two syzygies in degree~$3$ are:
%
\[
  z \, h_1 - y \, h_2 + x \, h_3 = 0, \quad 
  xh_1 \, +x\,h_2 - 2(y+z) \, h_3 = 0.
\]
%
Finally, $Z = \{ (0:1:1),(2:1:-1),(-2:1:-1) \}$.
Observe that one point is on the singular line $x=0$.
\end{es}


%%%%%%%%%%%%%%%%
\subsection{Eigenschemes containing a line}
Here we study the cases in which $\Eig{f}$ contains a line. We shall see that
this condition is equivalent to the condition of having four collinear
eigenpoints; this will allow us to charaterize the cubics $C = V(f)$ which have a line
in $\Eig{f}$ and, finally, to prove that the polynomials $f$
belong all to the locus $\sL \subset \p^9$ (see \Cref{def:locus_L}).
We first need the following result, which will be generalized in Section 6.2.

\begin{lemma}
\label{lemma:twoTangentsCiso} Let $r = ax+by+cz$ be a line of the plane 
and suppose it intersects $\iso$ in two
distinct points~$P_1$ and~$P_2$. Consider the cubic with equation:
\begin{equation}
\label{2_lines_of_eigenpoints}
  f(r) = \left( r^2-3\left(a^2+b^2+c^2\right)\iso \right) \, r \,;
\end{equation}
then, the two tangent lines to $\iso$ in~$P_1$ and~$P_2$
are contained in $\Eig {f(r)}$.
\end{lemma}
\begin{proof}
%% file: lemma_twoTangentsCiso.sage 
A generic point on $\iso$ is of the form $(\lambda^2 + \mu^2, 
\iii\lambda^2 -\iii\mu^2, 2\iii\lambda \mu)$ for $(\lambda: \mu) \in \p^1$.
We can therefore easily obtain two points~$P_1$ and~$P_2$ on~$\iso$,
determine the line $r = P_1 \vee P_2$ and define~$f(r)$ as above.
The two tangents to~$\iso$ in~$P_1$ and~$P_2$ turn out to be lines of eigenpoints
for~$f(r)$. For details, consult \nb{06}{F1}.
\end{proof}


\begin{lemma}
\label{lemma:four_points_on_line}
Suppose that $P_1, P_2, P_3, P_4$ are four distinct points belonging to a line~$t$.
Given a cubic $C=V(f)$, then $P_1, \dotsc, P_4 \in \Eig{f}$ if and only if
$t \subseteq \Eig{f}$.
Moreover,
%
\begin{equation*}
  6 \leq \rk \,\Phi(P_1, P_2, P_3, P_4) \leq 7
\end{equation*}
%
and the rank is~$6$ if and only if $\sigma(P_1, P_2) = 0$, i.e.\ if
and only if $t$ is tangent to the isotropic conic.
\end{lemma}
\begin{proof}
%%% +file: fourCollinearPoints.sage
The computations are in \nb{06}{F2}.
\end{proof}

We are now in the position to give a characterization of cubics with an eigenscheme containing a line.
\begin{prop}
\label{prop:eigenline_tangent}
Let $t$ be a line of $\mathbb{P}^2$.
%
\begin{itemize}
   \item 
   If $t$ is not tangent to the isotropic conic, then $t \subseteq \Eig{f}$ for a cubic $C=V(f)$ if and 
only $f = t^2\ell$, where $\ell$ is any line of the plane.
  \item 
  If $t$ is tangent to the isotropic conic~$\iso$ (in a point~$P$),
then $t \subseteq \Eig{f}$ for a cubic $C=V(f)$ if and only if
  %
  \begin{equation}
  \label{eq:cubics_with_tangent_eigenline}
    f = t^2 \ell+\lambda f(r_0),
  \end{equation}
  %
  where $\ell$ is any line of the plane, $\lambda \in \C$, 
  $r_0\neq t$ is any fixed line passing through $P$
  and~$f(r_0)$ is defined by \Cref{2_lines_of_eigenpoints}.
  \end{itemize}
  Moreover, if $t$ is tangent to $\iso$, any cubic of
  \Cref{eq:cubics_with_tangent_eigenline} is singular in $P$ and, if $V(f)$ is irreducible, then $V(f)$ is nodal and the line $t$ belongs to the tangent cone of $V(f)$ in $P$.
%
\end{prop}
\begin{proof}
%% file: una_retta_tg_autop.sage
Suppose $t$ is not tangent to $\iso$ and let $P_1, \dots, P_4$ be four 
distinct points on it. By~\Cref{lemma:four_points_on_line}, the projective linear system
$\Lambda(\Phi(P_1, \dotsc, P_4))$ is two-dimensional, but also all the cubics
defined by $t^2\ell$ (for $\ell$ any line of $\mathbb{P}^2$) have $t$ among
their eigenpoints and form a
two-dimensional linear system of cubics, hence the two linear systems
coincide.\\
If $t$ is tangent to $\iso$ in a point $P$, w.l.o.g.\ we can assume that
$P$ is $(1: \iii: 0)$ and $t$ is $x+\iii y$. If we impose to the generic
cubic of $\mathbb{P}^2$ to contain $t$ in the eigenscheme, we get a linear
system of cubics of dimension $3$. It can be shown that a basis of such a linear system is given by $H_1 = t^2x$, $H_2 = t^2y$, $H_3 = t^2z$ and
$H_4 = f(r_0)$, where $r_0$ is any line passing through $P$ and different from $t$. This proves the
second claim. It is immediate to see that $P$ is singular for the cubics of the linear system. \nb{06}{F3} contains the details of the computations which allow to conclude.
\end{proof}
%
\begin{prop}
\label{prop:limitCubics}
Any cubic that has a line~$t$ in the eigenscheme is the limit of a family of cubics whose general member has a $0$-dimensional eigenscheme with an aligned triple.
\end{prop}
\begin{proof}
%% file cubic_limit.sage
If the line~$t$ is not tangent to $\iso$, we can assume it is the line $z=0$. We fix on it the three points
%
\[
  P_1= (1: 0: 0), \ P_2 = (0: 1: 0), \ P_3 = (1: 1: 0)
\]
%
and we consider the three dimensional linear system of cubic forms $\Lambda \bigl( \Phi(P_1, P_2, P_3) \bigr)$, whose elements $f$ are of the form
%
\[
  f = (ax + by + cz)z^2 + d(x^3+y^3), \quad a, b, c, d \in \C \,.
\]
%
If $d=0$, we have $f=t^2 \ell$, hence, by \Cref{prop:eigenline_tangent} is the 
generic cubic not tangent to $\iso$ that has the line $t$ in the eigenscheme.
If $d \neq 0$, the eigenpoints of~$f$ different from $P_1, P_2, P_3$ are
the common zeros of the following polynomials:
%
\begin{align*}
  h_1 & = 3dy^2 - 2axy - 2by^2 + bz^2 - 3cyz \,,\\
  h_2 & = 3dx^2 - 2ax^2 + az^2 - 2bxy - 3cxz \,.
\end{align*}
%
In general, the ideal $(h_1, h_2)$ gives $4$ points in 
general position. \\
In case the line $t$ is tangent to the isotropic conic, we can 
assume it is the line $x+\iii y =0$. Here we fix the three
points
\[
P_1 = (1: \iii: 0), \ P_2 = (0: 0: 1), \ P_3 = (1: \iii: 1) \,.
\]
In this case the linear system 
$\Lambda \bigl(\Phi(P_1, P_2, P_3)\bigr)$ is four dimensional and 
is given by:
\[
f = (x+\iii y)^2(ax + by+cz)+
 d(x^2 + y^2 + 2/3z^2)z+e (x^3 -\iii y^3 + z^3), 
 \quad a, b, c, d, e \in \C
\]
If $e=0$, the form $f$ is described by \Cref{eq:cubics_with_tangent_eigenline}, hence is the generic cubic which contains the line $t$ in the eigenscheme. If $e \not= 0$, as above, 
the remaining eigenpoints of~$f$ are given by an ideal generated 
by four polynomials $h_1, \dotsc, h_4$.
% %
% \begin{align*}
% h_1 & = xza + \iii yza + \iii xzb - yzb - x^2c -2\iii xyc + y^2c - x^2d - y^2d + 3xze -3\iii yze - 3z^2e \,, \\
% h_2 & = 2x^2a + 3\iii xya - y^2a -\iii x^2b + 3xyb + 2\iii y^2b + 2xzc + 2\iii yzc +3\iii xye \,, \\
% h_3 & = 3x^2zb + 6\iii xyzb - 3y^2zb + 2\iii x^3c - 5x^2yc -4\iii xy^2c + y^3c + 2\iii xz^2c - 2yz^2c \\
% &\phantom{=} + 2\iii x^3d - x^2yd + 2\iii xy^2d - y^3d -6\iii x^2ze - 6xyze -3\iii y^2ze + 6\iii xz^2e - 3yz^2e \,, \\
% h_4 & = 3y^2za -\iii yz^2a + 6xyzb + 9\iii y^2zb -3\iii xz^2b + 4yz^2b + 4\iii x^2yc - 9xy^2c -5\iii y^3c \\
% &\phantom{=} + 2x^2zc + 4\iii xyzc - 2y^2zc + 4\iii yz^2c + 2z^3c + 4\iii x^2yd - xy^2d + 5\iii y^3d + 2x^2zd \\
% &\phantom{=} + 2y^2zd -12\iii xyze - 15y^2ze - 6xz^2e + 21\iii yz^2e + 6z^3e 
% \end{align*}
% %
Also here, the zeros of general $h_1, \dotsc, h_4$ are in general position.
Notebook \nb{06}{F4} collects the computations for this proof.
\end{proof}

\subsection{Eigenschemes containing a conic}

%%%%%%%%%%%%%
\begin{theorem} Suppose that $\Gamma$ is a conic and let $C=V(f)$ be a cubic such that $\Gamma \subseteq \Eig{f}$. Then we have three possible cases:
    \begin{itemize}
        \item $\Gamma = \iso$. This is true if and only if $f = \ell\iso$ where $\ell$ is any line of the plane;
        \item $\Gamma$ is bitangent to $\iso$ in two distinct points $P_1$ and $P_2$. This is true iff $f = r(\lambda \iso+ \mu r^2)$, where $\lambda, \mu \in \mathbb{C}$, $r = P_1 \vee P_2$. In this case $\Gamma = V(\lambda \iso+3 \mu r^2)$;
        \item $\Gamma$ is iperosculating $\iso$ in a point $P$. This is true iff $f = r(\iso-r^2)$, where $r$ is the tangent to $\iso$ in $P$. In this case 
        $\Gamma = V(\iso-3r^2)$.
    \end{itemize}
\end{theorem}
\begin{proof} Suppose $P_1, \dots, P_4$ are four distinct points on $\iso$ and let $C=V(f)$ be a cubic with $P_1, \dots, P_4\in \Eig{f}$. Then we claim that
$\iso\subseteq \Eig{f}$. Indeed, the linear system $\Lambda(\Phi(P_1, \dotsc, P_4))$ is two-dimensional and is contained in the linear system given by all the cubics of the form $\ell \iso$ (whose cubics have $\iso$ in the eigenscheme) and is also two dimensional, so they coincide. From this we have the first point. If $C$ is a cubic that contains in the eigenscheme a conic $\Gamma$ different from $\iso$, then $\Gamma$ and $\iso$ intersect in $4$ points. From the above result, the four points cannot be distinct, hence we have to consider four cases: $\Gamma$ is tangent to $\iso$ in one point, $\Gamma$ is bitangent to
$\iso$, $\Gamma$ is osculating $\iso$ and $\Gamma$ is iperosculating $\iso$. In all these cases we can assume that 
$P_1 = (1: i: 0) \in \Gamma \cap \iso$ and that
the line $x+iy$ is the common tangent to $\Gamma$ and $\iso$ in $P_1$. 
In the first case we consider the pencil of conics $\Gamma_s$ passing through $P_1, P_2, P_3\in \Gamma \cap \iso$ and tangent to $\iso$ in $P_1$ and 
we take two distinct points $P_4, P_5$ on $\Gamma_s$. The matrix of conditions 
$\Phi(P_1, \dotsc, P_5)$ must have rank $9$ or smaller, but the computations show that 
this is not possible. In the second case, we take a generic point $P_2$ on $\iso$, we construct 
the pencil of conics~$\Gamma_s$ which are bitangent to~$\iso$ in $P_1$ and $P_2$ and we take three other points $P_3, P_4, P_5$ 
on $\Gamma_s$. Again we check when the matrix of conditions $\Phi(P_1, \dotsc, P_5)$ has rank $9$ (or less). In this case we get that there is only one solution, which is given by the cubic 
$V(r(\lambda \iso+\mu r^2))$. The remaining two cases are similar. The computational details can be find in \nb{06}{F5}.
\end{proof}
\begin{rmk}
    \Cref{lemma:twoTangentsCiso} can be seen as a particular case of the second item of the Theorem above.
\end{rmk}

\section{Possible configurations of the seven eigenpoints}
\label{further_alignments}

\begin{table}[ht]
\caption{All possible combinatorial configurations of seven points with at least one alignment and no six on a conic.}
\centering
\begin{tabular}{|clc|}\hline
  n. lines & collinear vertices & config.\\ \hline
 1& (1, 2, 3) &  $(C_1)$\\
 2& (1, 2, 3), (1, 4, 5) &  $(C_2)$\\
 3& (1, 2, 3), (1, 4, 5), (1, 6, 7) & $(C_3)$\\
 3& (1, 2, 3), (1, 4, 5), (2, 4, 6) & $(C_4)$\\
 4& (1, 2, 3), (1, 4, 5), (1, 6, 7), (2, 4, 6) & $(C_5)$\\
 4& (1, 2, 3), (1, 4, 5), (2, 4, 6), (3, 5, 6) & $(C_6)$\\
 5& (1, 2, 3), (1, 4, 5), (1, 6, 7), (2, 4, 6), (2, 5, 7)& $(C_7)$\\
 6& (1, 2, 3), (1, 4, 5), (1, 6, 7), (2, 4, 6), (2, 5, 7), (3, 4, 7)& $(C_8)$\\
 7& (1, 2, 3), (1, 4, 5), (1, 6, 7), (2, 4, 6), (2, 5, 7), (3, 4, 7), (3, 5, 6) &  $(C_9)$\\ \hline
\end{tabular}
\label{table:all_alignments}
\end{table}
%
\begin{figure}[ht]
  \centering
  \begin{tikzpicture}
    \begin{scope}[scale=0.8]
    %% C1
    \begin{scope}
      \node[point, label={[label distance = -4pt, below]$P_1$}] (P1) at (0,0) {};
      \node[point, label={[label distance = -4pt, below]$P_3$}] (P3) at (3.6,0) {};
      \node[point, label={[label distance = -4pt, below]$P_2$}] (P2) at ($(P1)!0.5!(P3)$) {};
      \node[point, label={[label distance = -4pt, left]$P_4$}] (P4) at (-0.5, 0.9) {};
      \node[point, label={[label distance = -4pt, left]$P_5$}] (P5) at (0.7, 2.3) {};
      \node[point, label={[label distance = -4pt, right]$P_6$}] (P6) at (1.8, 2.1) {};
      \node[point, label={[label distance = -4pt, right]$P_7$}] (P7) at (2.9, 1.2) {};
      \draw[line, shorten <= -0.5cm, shorten >= -0.5cm] (P1) -- (P2); 
      \draw[line, shorten <= -0.5cm, shorten >= -0.5cm] (P2) -- (P3);
    \end{scope}
    %% C2
    \begin{scope}[xshift=6cm]
      \node[point, label={[label distance = 0pt]210:$P_1$}] (P1) at (0,0) {};
      \node[point, label={[label distance = -4pt, below]$P_3$}] (P3) at (3.6,0) {};
      \node[point, label={[label distance = -4pt, below]$P_2$}] (P2) at ($(P1)!0.5!(P3)$) {};
      \node[point, label={[label distance = 0pt, left]$P_5$}] (P5) at (1.5, 2.4) {};
      \node[point, label={[label distance = 0pt, left]$P_4$}] (P4) at ($(P1)!0.3!(P5)$) {};
      \node[point, label={[label distance = 0pt, right]$P_6$}] (P6) at (2.2, 2.1) {};
      \node[point, label={[label distance = 0pt, right]$P_7$}] (P7) at (3.2, 1.2) {};
      \draw[line, shorten <= -0.5cm, shorten >= -0.5cm] (P1) -- (P2);
      \draw[line, shorten <= -0.5cm, shorten >= -0.5cm] (P2) -- (P3);
      \draw[line, shorten <= -0.5cm, shorten >= -0.5cm] (P1) -- (P4);
      \draw[line, shorten <= -0.5cm, shorten >= -0.5cm] (P4) -- (P5);
    \end{scope}
    %% C3
    \begin{scope}[xshift=12cm]
      \node[point, label={[label distance = 0pt]210:$P_1$}] (P1) at (0,0) {};
      \node[point, label={[label distance = -4pt, below]$P_3$}] (P3) at (3.6,0) {};
      \node[point, label={[label distance = -4pt, below]$P_2$}] (P2) at ($(P1)!0.5!(P3)$) {};
      \node[point, label={[label distance = 0pt, left]$P_5$}] (P5) at (1.5, 2.4) {};
      \node[point, label={[label distance = 0pt, left]$P_4$}] (P4) at ($(P1)!0.3!(P5)$) {};
      \node[point, label={[label distance = 0pt, above]$P_6$}] (P6) at (1.6, 1) {};
      \node[point, label={[label distance = 0pt, above]$P_7$}] (P7) at (3.2, 2) {};
      \draw[line, shorten <= -0.5cm, shorten >= -0.5cm] (P1) -- (P2);
      \draw[line, shorten <= -0.5cm, shorten >= -0.5cm] (P2) -- (P3);
      \draw[line, shorten <= -0.5cm, shorten >= -0.5cm] (P1) -- (P4);
      \draw[line, shorten <= -0.5cm, shorten >= -0.5cm] (P4) -- (P5);
      \draw[line, shorten <= -0.5cm, shorten >= -0.5cm] (P1) -- (P6);
      \draw[line, shorten <= -0.5cm, shorten >= -0.5cm] (P6) -- (P7);
    \end{scope}
    %% C4
    \begin{scope}[yshift=-6cm]
      \node[point, label={[label distance = 0pt]210:$P_1$}] (P1) at (0,0) {};
      \node[point, label={[label distance = -4pt, below]$P_3$}] (P3) at (3.6,0) {};
      \node[point, label={[label distance = -4pt, below]$P_2$}] (P2) at ($(P1)!0.5!(P3)$) {};
      \node[point, label={[label distance = 4pt]90:$P_5$}] (P5) at (1.5, 2.4) {};
      \node[point, label={[label distance = 0pt, left]$P_4$}] (P4) at ($(P1)!0.3!(P5)$) {};
      \node[point, label={[label distance = 0pt, right]$P_6$}] (P6) at ($(P5)!0.25!(P3)$) {};
      \node[point, label={[label distance = 0pt, left]$P_7$}] (P7) at (2, 0.9) {};
      \draw[line, shorten <= -0.5cm, shorten >= -0.5cm] (P1) -- (P2);
      \draw[line, shorten <= -0.5cm, shorten >= -0.5cm] (P2) -- (P3);
      \draw[line, shorten <= -0.5cm, shorten >= -0.5cm] (P1) -- (P4);
      \draw[line, shorten <= -0.5cm, shorten >= -0.5cm] (P4) -- (P5);
      \draw[line, shorten <= -0.5cm, shorten >= -0.5cm] (P3) -- (P6);
      \draw[line, shorten <= -0.5cm, shorten >= -0.5cm] (P6) -- (P5);
    \end{scope}
    %% C5
    \begin{scope}[xshift=6cm, yshift=-6cm]
      \node[point, label={[label distance = 0pt]210:$P_1$}] (P1) at (0,0) {};
      \node[point, label={[label distance = -4pt, below]$P_3$}] (P3) at (3.6,0) {};
      \node[point, label={[label distance = -4pt, below]$P_2$}] (P2) at ($(P1)!0.5!(P3)$) {};
      \node[point, label={[label distance = 0pt]120:$P_5$}] (P5) at (1.5, 2.4) {};
      \node[point, label={[label distance = 2pt]180:$P_4$}] (P4) at ($(P1)!0.55!(P5)$) {};
      \node[point, label={[label distance = 0pt]90:$P_6$}] (P6) at ($(P2)!0.5!(P4)$) {};
      \node[point, label={[label distance = 0pt, above]$P_7$}] (P7) at ($(P1)!2.4!(P6)$) {};
      \draw[line, shorten <= -0.5cm, shorten >= -0.5cm] (P1) -- (P2);
      \draw[line, shorten <= -0.5cm, shorten >= -0.5cm] (P2) -- (P3);
      \draw[line, shorten <= -0.5cm, shorten >= -0.5cm] (P1) -- (P4);
      \draw[line, shorten <= -0.5cm, shorten >= -0.5cm] (P4) -- (P5);
      \draw[line, shorten <= -0.5cm, shorten >= -0.5cm] (P1) -- (P6);
      \draw[line, shorten <= -0.5cm, shorten >= -0.5cm] (P6) -- (P7);
      \draw[line, shorten <= -0.5cm, shorten >= -0.5cm] (P2) -- (P6);
      \draw[line, shorten <= -0.5cm, shorten >= -0.5cm] (P6) -- (P4);
    \end{scope}
    %% C6
    \begin{scope}[xshift=12cm, yshift=-6cm]
      \node[point, label={[label distance = 0pt]210:$P_1$}] (P1) at (0,0) {};
      \node[point, label={[label distance = 0pt, above]$P_3$}] (P3) at (3.6,0) {};
      \node[point, label={[label distance = -4pt, below]$P_2$}] (P2) at (2.2,0) {};
      \node[point, label={[label distance = 4pt]90:$P_5$}] (P5) at (1.5, 2.4) {};
      \node[point, label={[label distance = 2pt]100:$P_4$}] (P4) at ($(P1)!0.3!(P5)$) {};
      \tkzInterLL(P3,P5)(P2,P4) \tkzGetPoint{P6}
      \node[point, label={[label distance = 0pt, right]$P_6$}] at (P6) {};
      \node[point, label={[label distance = 0pt, above]$P_7$}] (P7) at (3.2, 1.8) {};
      \draw[line, shorten <= -0.5cm, shorten >= -0.5cm] (P1) -- (P2);
      \draw[line, shorten <= -0.5cm, shorten >= -0.5cm] (P2) -- (P3);
      \draw[line, shorten <= -0.5cm, shorten >= -0.5cm] (P1) -- (P4);
      \draw[line, shorten <= -0.5cm, shorten >= -0.5cm] (P4) -- (P5);
      \draw[line, shorten <= -0.5cm, shorten >= -0.5cm] (P3) -- (P6);
      \draw[line, shorten <= -0.5cm, shorten >= -0.5cm] (P6) -- (P5);
      \draw[line, shorten <= -0.5cm, shorten >= -0.5cm] (P4) -- (P2);
      \draw[line, shorten <= -0.5cm, shorten >= -0.5cm] (P2) -- (P6);
    \end{scope}
    %% C7
    \begin{scope}[yshift=-12cm]
      \node[point, label={[label distance = 0pt]210:$P_1$}] (P1) at (0,0) {};
      \node[point, label={[label distance = -4pt, below]$P_2$}] (P2) at (3,0) {};
      \node[point, label={[label distance = -4pt, below]$P_3$}] (P3) at ($(P1)!0.4!(P2)$) {};
      \node[point, label={[label distance = 4pt]90:$P_4$}] (P4) at (1.5, 2.6) {};
      \node[point, label={[label distance = 2pt]100:$P_5$}] (P5) at ($(P1)!0.5!(P4)$) {};
      \node[point, label={[label distance = 2pt, above]:$P_7$}] (P7) at ($(P2)!0.65!(P5)$) {};
      \tkzInterLL(P1,P7)(P2,P4) \tkzGetPoint{P6}
      \node[point, label={[label distance = 2pt, above]$P_6$}] at (P6) {};
      \draw[line, shorten <= -0.5cm, shorten >= -0.5cm] (P1) -- (P3);
      \draw[line, shorten <= -0.5cm, shorten >= -0.5cm] (P3) -- (P2);
      \draw[line, shorten <= -0.5cm, shorten >= -0.5cm] (P1) -- (P5);
      \draw[line, shorten <= -0.5cm, shorten >= -0.5cm] (P5) -- (P4);
      \draw[line, shorten <= -0.5cm, shorten >= -0.5cm] (P2) -- (P6);
      \draw[line, shorten <= -0.5cm, shorten >= -0.5cm] (P6) -- (P4);
      \draw[line, shorten <= -0.5cm, shorten >= -0.5cm] (P5) -- (P7);
      \draw[line, shorten <= -0.5cm, shorten >= -0.5cm] (P7) -- (P2);
      \draw[line, shorten <= -0.5cm, shorten >= -0.5cm] (P1) -- (P7);
      \draw[line, shorten <= -0.5cm, shorten >= -0.5cm] (P7) -- (P6);
    \end{scope}
    %% C8
    \begin{scope}[xshift=6cm, yshift=-12cm]
      \node[point, label={[label distance = 0pt]210:$P_1$}] (P1) at (0,0) {};
      \node[point, label={[label distance = -4pt, below]$P_{2}$}] (P2) at (3,0) {};
      \node[point, label={[label distance = 0pt]195:$P_{3}$}] (P3) at ($(P1)!0.5!(P2)$) {};
      \node[point, label={[label distance = 4pt]90:$P_{4}$}] (P4) at (1.5, 2.6) {};
      \node[point, label={[label distance = 2pt]100:$P_{5}$}] (P5) at ($(P1)!0.5!(P4)$) {};
      \node[point, label={[label distance = 2pt]80:$P_6$}] (P6) at ($(P2)!0.5!(P4)$) {};
      \tkzInterLL(P2,P5)(P3,P4) \tkzGetPoint{P7}
      \node[point, label={[label distance = 0pt]0:$P_7$}] at (P7) {};
      \draw[line, shorten <= -0.5cm, shorten >= -0.5cm] (P1) -- (P3);
      \draw[line, shorten <= -0.5cm, shorten >= -0.5cm] (P3) -- (P2);
      \draw[line, shorten <= -0.5cm, shorten >= -0.5cm] (P1) -- (P5);
      \draw[line, shorten <= -0.5cm, shorten >= -0.5cm] (P5) -- (P4);
      \draw[line, shorten <= -0.5cm, shorten >= -0.5cm] (P2) -- (P6);
      \draw[line, shorten <= -0.5cm, shorten >= -0.5cm] (P6) -- (P4);
      \draw[line, shorten <= -0.5cm, shorten >= -0.5cm] (P5) -- (P7);
      \draw[line, shorten <= -0.5cm, shorten >= -0.5cm] (P7) -- (P2);
      \draw[line, shorten <= -0.5cm, shorten >= -0.5cm] (P4) -- (P7);
      \draw[line, shorten <= -0.5cm, shorten >= -0.5cm] (P7) -- (P3);
      \draw[line, shorten <= -0.5cm, shorten >= -0.5cm] (P1) -- (P7);
      \draw[line, shorten <= -0.5cm, shorten >= -0.5cm] (P7) -- (P6);
    \end{scope}
    %% C9
    \begin{scope}[xshift=12cm, yshift=-12cm]
      \node[point, label={[label distance = 0pt]210:$P_1$}] (P1) at (0,0) {};
      \node[point, label={[label distance = -4pt, below]$P_{2}$}] (P2) at (3,0) {};
      \node[point, label={[label distance = 0pt]195:$P_{3}$}] (P3) at ($(P1)!0.5!(P2)$) {};
      \node[point, label={[label distance = 4pt]90:$P_{4}$}] (P4) at (1.5, 2.6) {};
      \node[point, label={[label distance = 2pt]100:$P_{5}$}] (P5) at ($(P1)!0.5!(P4)$) {};
      \node[point, label={[label distance = 2pt]80:$P_6$}] (P6) at ($(P2)!0.5!(P4)$) {};
      \tkzInterLL(P2,P5)(P3,P4) \tkzGetPoint{P7}
      \node[point, label={[label distance = 0pt]0:$P_7$}] at (P7) {};
      \node[draw, line] at (P7) [circle through={(P3)}] {};
      \draw[line, shorten <= -0.5cm, shorten >= -0.5cm] (P1) -- (P3);
      \draw[line, shorten <= -0.5cm, shorten >= -0.5cm] (P3) -- (P2);
      \draw[line, shorten <= -0.5cm, shorten >= -0.5cm] (P1) -- (P5);
      \draw[line, shorten <= -0.5cm, shorten >= -0.5cm] (P5) -- (P4);
      \draw[line, shorten <= -0.5cm, shorten >= -0.5cm] (P2) -- (P6);
      \draw[line, shorten <= -0.5cm, shorten >= -0.5cm] (P6) -- (P4);
      \draw[line, shorten <= -0.5cm, shorten >= -0.5cm] (P5) -- (P7);
      \draw[line, shorten <= -0.5cm, shorten >= -0.5cm] (P7) -- (P2);
      \draw[line, shorten <= -0.5cm, shorten >= -0.5cm] (P4) -- (P7);
      \draw[line, shorten <= -0.5cm, shorten >= -0.5cm] (P7) -- (P3);
      \draw[line, shorten <= -0.5cm, shorten >= -0.5cm] (P1) -- (P7);
      \draw[line, shorten <= -0.5cm, shorten >= -0.5cm] (P7) -- (P6);
    \end{scope}
    \end{scope}
  \end{tikzpicture}
  \caption{Graphical representations of the $9$ cases of possible alignments of seven points as described in \Cref{table:all_alignments}.}
  \label{figure:all_alignments}
\end{figure}
%    
In this section, we want to identify which of the nine configurations in \Cref{table:all_alignments} can be realized by
the seven eigenpoints of a regular ternary cubic polynomial and, if so, which are the cubics with that configuration of eigenpoints.

We say that the eigenpoints of a cubic curve are in a \emph{$(C_i)$ 
configuration} if they are aligned according to~$(C_i)$; we say 
that the eigenpoints are in a \emph{strict $(C_i)$ configuration} if, 
in addition, there are no further alignments among them.

First of all, it is well known that configuration~$(C_9)$ cannot be realized
by seven points of the plane over a field of zero
characteristic (see \cite{Whitney1935}), therefore we do not consider
it in our analysis.

\subsection*{Configuration~\texorpdfstring{$(C_1)$}{C1}}
This configuration can be realized. \Cref{proposition:three_distinct_ranks} and \Cref{locus_one_alignment}
give a description of the cubics with such a configuration of points:
we fix two points~$P_1$ and~$P_2$ in~$\p^2$, we take a generic~$P_3$
on the line~$P_1 \vee P_2$, all the cubics with
$P_1, P_2, P_3$ eigenpoints are given by $\Lambda \bigl( \Phi(P_1, P_2, P_3) \bigr)$, the three
dimensional linear subspace of~$\p^9$ (four dimensional, if the
line~$P_1 \vee P_2$ is tangent in~$P_1$, $P_2$, or~$P_3$ to~$\iso$). For generic choices of the points, the cubics have the eigenpoints in a strict 
$(C_1)$ configuration.

\subsection*{Configuration~\texorpdfstring{$(C_2)$}{C2}}
In this case, the points
$P_1, \dots, P_5$ are in a
$V$- configuration, so $\rk \, \Phi(P_1, \dotsc, P_5)$
must be~$9$ or~$8$. If the rank is~$9$, then $\delta_1(P_1, P_2, P_4) = 0$
or $\delta_2(P_1, \dotsc, P_5) = 0$, see \Cref{theorem:rank_V}.
From \Cref{proposition:third_alignment}, the only case
which admits a strict $(C_2)$ configuration 
is $\delta_1(P_1, P_2, P_4) = 0$. Hence, if we fix two points~$P_1$ and~$P_2$
in the plane in an arbitrary way, from \Cref{lemma_delta_case1} we can choose $P_4$ so that
$\scl{P_4}{s_{11}\, P_2 - s_{12} \, P_1}=0$, then we fix any~$P_3$
on the line~$P_1 \vee P_2$ and any~$P_5$ on the line~$P_1 \vee P_4$; in this way, we get a cubic with a configuration of type~$(C_2)$, which is generally strict. If the rank is~$8$, configuration~$(C_2)$ can only be obtained from \Cref{rk8_1}, hence we get sub-cases of the case $\delta_1(P_1, P_2, P_4)=0$, see~\Cref{rmk:particular_cases}.
@@ qui c'e' un'osservazione tolta che forse va rimessa


\subsection*{Configuration~\texorpdfstring{$(C_3)$}{C3}}
If we have the alignments $(P_1, P_2, P_3)$, $(P_1, P_4, P_5)$ and $(P_1, P_6, P_7)$, then
%
\[
 (P_1, P_2, P_3, P_4, P_5) \,, \quad (P_1, P_2, P_3, P_6, P_7) \,, \quad (P_1, P_4, P_5, P_6, P_7)
\]
%
are three $V$- configurations. It holds:

\begin{lemma}
\label{no_delta1_delta1} Suppose we have seven eigenpoints $P_1, \dots, P_7$
of a cubic in configuration $(C_3)$. Then among the $7$ points there is a
$V$- configuration that satisfies a $\delta_2 = 0$ condition.
\end{lemma}
\begin{proof} (\nb{07}{F1})
%% la dim si trova in config3.sage
The points $P_1, P_2, P_3, P_4, P_5$ are in a $V$- configuration.
If $\rk \, \Phi(P_1, \dots, P_5) = 8$, the result follows from~\Cref{rank_8}.
Therefore, assume that the matrix $\Phi(P_1, \dots, P_5)$ has rank~$9$.
If $\delta_2(P_1, \dots, P_5) = 0$, the statement follows;
otherwise, $\delta_1(P_1, P_2, P_4) = 0$.
Then consider the $V$- configuration $P_1$, $P_2$, $P_3$, $P_6$, $P_7$.
As above, we suppose $\delta_1(P_1, P_2, P_6) = 0$.
These two equations are
linear in the coordinates of~$P_2$.
If the matrix of the associated linear system has
maximal rank, the unique solution gives a point~$P_2$ which coincides, as a projective point, to $P_1$, which is impossible.
Hence we are led to consider the
case in which the matrix does not have maximal rank.
This condition implies that $P_1$ is on the
isotropic conic. As usual, we can assume 
$P_1 = (1: \iii: 0)$ and again
we can determine~$P_2$ such $\delta_1(P_1, P_2, P_4)$ and
$\delta_1(P_1, P_2, P_6)$ are zero. The matrix $\Phi(P_1, P_4, P_5, P_6, P_7)$
must have rank~$9$ or smaller, then either
$\delta_2(P_1, P_4, P_5, P_6, P_7)=0$ or $\delta_1(P_1, P_4, P_6) = 0$. In
the first case, we have a $\delta_2$ condition among the points, hence
we assume $\delta_1(P_1, P_4, P_6) = 0$. Solving this equation and
considering the corresponding points, we get that either
$\delta_2(P_1, P_2, P_3, P_6, P_7) = 0$ or
$\delta_2(P_1, P_2, P_3, P_4, P_5) = 0$.
\end{proof}

As a consequence of \Cref{no_delta1_delta1}, configuration $(C_3)$ can
be obtained only if we have $5$ points such that
$\delta_2(P_1, \dotsc, P_5) = 0$. \Cref{prop:definitionP3} describes how to
obtain five points which give a configuration $(C_3)$ of eigenpoints (which, in the case $P_3$  is defined by (4), is strict).

\subsection*{Configuration~\texorpdfstring{$(C_4)$}{C4}}
It holds:
\begin{prop}
\label{conf4no} If the eigenpoints of a cubic are in a $(C_4)$ configuration, then they actually are in a $(C_8)$ configuration.
\end{prop}
\begin{proof} (\nb{07}{F2})
%% il file config4.sage contiene la dimostrazione
From the results of \Cref{rank_8}, we know that $(C_4)$
cannot be obtained from a rank 8 $V$- configuration,
hence it remains to consider the case $\delta_1(P_1, P_2, P_4) = 0$,
$\delta_1(P_2, P_1, P_4) = 0$ and $\delta_1(P_4, P_1, P_2) = 0$,
which implies
%
\[
  \delta_2(P_1, P_2, P_3, P_4, P_5) = 0 \,, \quad
  \delta_2(P_2, P_1, P_3, P_4, P_6) = 0 \,, \quad
  \delta_2(P_4, P_1, P_5, P_2, P_6) = 0 \,. \qedhere
\]
%
\end{proof}

\subsection*{Configuration~\texorpdfstring{$(C_5)$}{C5}} 
%% file: config5.sage
\begin{prop}
\label{proposition:condition_5}
%\textbf{(AGGIUNGERE IPOTESI SU RANGO NON 8)}
Suppose we have $7$ points which are eigenpoints of a cubic and are in a strict~$(C_5)$ configuration.
Then it holds
%
\[
  P_1 = P_2 \times P_4 
\]
%
and among the points $P_1, \dots, P_6$, where $P_6$ is collinear with~$P_2$ and~$P_4$, we have the relation
%
\begin{equation}
\label{cndC5}
  s_{26}(s_{45}s_{13}-s_{34}s_{15})+s_{46}(s_{25}s_{13}-s_{23}s_{15}) = 0 \,.
\end{equation}
%
Moreover, the points~$P_6$ and~$P_7$ can be determined as follows:
%
\begin{equation}
\label{p6formula}
\begin{multlined}
  P_6 = (s_{15}s_{24}s_{34}+s_{15}s_{23}s_{44} -s_{13}s_{25}s_{44} -s_{13}s_{24}s_{45}) \, P_2 \\ + (s_{13}s_{24}s_{25}-2s_{15}s_{22}s_{34}+s_{13}s_{22}s_{45}) \, P_4 \,, 
\end{multlined}
\end{equation}
%
\begin{equation}
\label{p7formula}
P_7 = (s_{26}s_{15}s_{46}+s_{24}s_{15}s_{66})P_1 + s_{11}(s_{26}s_{45}+s_{24}s_{56})P_6\,.
\end{equation}
%
\end{prop}
\begin{proof}
We can define the points as follows: $P_1$, $P_2$ and~$P_4$ with generic
coordinates, meanwhile $P_3 = u_1 \, P_1 + u_2 \, P_2$, $P_5 = v_1 \, P_1 + v_2 \, P_4$ and
$P_6 = w_1 \, P_2 + w_2 \, P_4$. The points $(P_2, P_1, P_3, P_4, P_6)$ are in a
$V$- configuration. If it is of rank $8$, 
from \Cref{rank_8} we see that it must be 
a $(C_3)$ configuration but this is not possible, so
we can assume that $\rk \, \Phi(P_1, P_2, P_3, P_4, P_6) = 9$, hence necessarily $\delta_1(P_2, P_1, P_4)=0$.
Analogously, $\delta_1(P_4, P_1, P_2) = 0$ and $\delta_1(P_6, P_1, P_2) = 0$. Moreover, by \Cref{no_delta1_delta1} and by possibly relabelling the points, we may assume that $\delta_2(P_1, P_2, P_3, P_4, P_5)=0$. The saturation of the ideal generated by the four polynomials
above by the condition that $P_1$, $P_2$, and~$P_4$ are not aligned equals the ideal generated by
$s_{12}, s_{14}$, which also contains~$s_{16}$. This shows that
$s_{12} = s_{14}=s_{16}=0$. In particular, we see that $P_1 = P_2 \times P_4$.

To prove \Cref{cndC5}, we take $P_2$ and~$P_4$ generic,
$P_1 = P_2 \times P_4$ and~$P_3$, $P_5$, $P_6$ generic points on the lines
$P_1 \vee P_2$, $P_1 \vee P_4$, $P_2 \vee P_4$, respectively. 
The matrix $M = \Phi(P_1, \dots, P_6)$ must have rank~$9$ (or smaller), 
hence, in order to see when $P_6$ is an eigenpoint,
we have to see when all the order $10$-minors are zero. This computation gives that we can obtain a polynomial $f$ (in the 
variables $u_i, v_i, w_i$ and the coordinates of $P_2$ and $P_4$) such that $\rk M \leq 9$ if and only if $f=0$. Furthermore, 
a manipulation of $f$ shows that $f=0$ if and only if \Cref{cndC5} holds.

If in \Cref{cndC5} in place of~$P_6$ we substitute $w_1 \, P_2 + w_2 \, P_4$,
we find~$w_1$ and~$w_2$, which give \Cref{p6formula}.

Finally, to prove \Cref{p7formula}, we change~$P_3$ with~$P_7$ in \Cref{cndC5} and we take into account that $P_7$ is collinear with~$P_1$ and~$P_6$. All the detailed computations can be found in \nb{07}{F3}.
If $\rk\, \Phi(P_2, P_1, P_3, P_4, P_6) = 8$
\end{proof}
%
The converse of the above proposition is also true, more precisely:
\begin{prop}
   Suppose $P_1, P_2, P_3, P_4, P_5$ are in a $V$- configuration such that
%
\[
  P_1 = P_2 \times P_4 \,,
\]
%
and suppose there is a cubic that has $7$ eigenpoints, five of which are $P_1, \dotsc, P_5$. Then the points~$P_6$ and~$P_7$, given by \Cref{p6formula} and \Cref{p7formula}, are the remaining eigenpoints of the cubic, hence $(P_2, P_4, P_6)$ are aligned, so the points are in a $(C_5)$ configuration. Furthermore, in the general case, the configuration is 
strict but, in the case of $\rk \, \Phi(P_1, \dotsc, P_5) = 9$, there are sub-cases in which $(P_2, P_5, P_7)$ or $(P_3, P_4, P_7)$ or $(P_3, P_5, P_7)$ are aligned. In all these cases the points are in a $(C_8)$ configuration.
\end{prop}
%
\begin{proof}
First we suppose that $\rk \, \Phi(P_1, \dotsc, P_5) = 9$.
We fix generic coordinates for the points.
Since $P_1 = P_2 \times P_4$ implies $s_{12}=s_{14}=0$,
we have that $\delta_2 (P_1,P_2,P_3,P_4,P_5)=0$. 
It follows from \Cref{proposition:third_alignment} that the corresponding unique cubic having such $5$ points as eigenpoints has $P_6$ and~$P_7$ aligned with~$P_1$.
Suppose now that $P_6$ is given by \Cref{p6formula}.
In order to see if it is an eigenpoint, it is enough to select nine linearly independent rows from $\Phi(P_1, \dotsc, P_5)$
and to compute the determinant of the three order $10$ matrices whose rows are the nine rows above plus $\phi_1(P_6)$, $\phi_2(P_6)$, and~$\phi_3(P_6)$ respectively.
Since we obtain that these three determinants are zero, we get that $\rk \, \Phi(P_1, \dotsc, P_6) = 9$.
A similar argument shows that $P_7$ is given by \Cref{p7formula}.
A random example shows that, in general, there are no further collinearities among the points.
If we impose that $(P_2, P_5, P_7)$ are aligned, we get that this condition splits into two cases:
one imposes the collinearity of $(P_3, P_4, P_7)$, the other imposes that the collinearity of $(P_3, P_5, P_6)$,
hence we get two $(C_8)$ configurations. Similarly for the other two cases. 

If $\rk \, \Phi(P_1, \dotsc, P_5) = 8$ and $P_1 = P_2 \times P_4$, then $s_{12} = s_{14} = 0$, so the $V$- configuration cannot satisfy \Cref{rk8_1}, since we would have \Cref{lemma_delta_case2} and by $s_{12} = 0$, it would be $P_1 = P_3$. Hence,
from the results of \Cref{rank_8}, the lines $P_1 \vee P_2$ and $P_1 \vee P_4$ are tangent to the isotropic conic in~$P_2$ and~$P_4$. Thus, $s_{22}, s_{44}, s_{23}, s_{45}$ are all zero and \Cref{p6formula} and \Cref{p7formula} specialize to \Cref{formuleP6_P7}; finally, \Cref{prop:rk8_2B} shows that the points are in a $(C_5)$ configuration. Notebook \nb{07}{F3} contains some details of the computations.
\end{proof}
%

\begin{rmk}
\label{rmk:C5rk8}
Since in a $(C_5)$ configuration we have 
$P_1 = P_2 \times P_4$ and $(P_2, P_4, P_6)$ aligned, it follows that $s_{12} = 0$, $s_{14} = 0$ and $s_{16} = 0$.
Moreover, it is
immediate to verify that the line~$P_2 \vee P_4$ is orthogonal to the
lines~$P_1 \vee P_2$, $P_1 \vee P_4$, and~$P_1 \vee P_6$, that is
%
\[
  \scl{P_1 \times P_2}{P_2 \times P_4} = 0, \quad
  \scl{P_1 \times P_4}{P_2 \times P_4} = 0, \quad
  \scl{P_1 \times P_6}{P_2 \times P_4} = 0,
\]
hence \Cref{rmk:three_orthog} gives that the $(C_5)$ configuration obtained in \Cref{prop:rk8_2B} is a sub-case of the $(C_5)$ configuration described here (\Cref{p6formula} and \Cref{p7formula} particularize to \Cref{formuleP6_P7}).@@ brutto
%
\end{rmk}
%
%\begin{rmk}
%The family of all the cubics with the eigenpoints in configuration $(C_5)$
%are therefore obtained taking $P_2, P_4$ generic, $P_1 = P_2 \times P_4$,
%$P_5$ generic on the line~$P_1 \vee P_4$ and $P_3$ given by the \Cref{p3formula}.
%It is easy to see that the cubics with eigenpoints in configuration $(C_5)$
%described in~\Cref{prop:rk8_2B} are a sub-family of the cubics here described,
%obtained by the conditions $s_{22} = s_{44} = 0$.
%\end{rmk}

%% conti nel file config5.sage.
\subsection*{Configurations~\texorpdfstring{$(C_6)$}{C6} and \texorpdfstring{$(C_7)$}{C7}}

The strict configurations $(C_6)$ and~$(C_7)$ cannot be realized by eigenpoints.
Indeed, if we study the ideals given by the conditions
$\delta_1=0$ and~$\delta_2=0$ that must be satisfied by the points in order to get the configurations,
we see that there are no compatible solutions.
See notebooks \nb{07}{F4} and \nb{07}{F5}.

\subsection*{Configuration~\texorpdfstring{$(C_8)$}{C8}}
This configuration is realizable.  
%% file config8N2.sage
The $(C_8)$ configuration we consider in this section is the following: 
%
\begin{equation}
\label{config:c8}
(P_1, P_2, P_3), \ (P_1, P_4, P_5), \ (P_1, P_6, P_7), \ (P_2, P_4, P_6), \ 
(P_2, P_5, P_7), \ (P_3, P_4, P_7)    
\end{equation}
%
Hence the points $P_1, P_2, P_4, P_7$ lie on three lines, while the points
$P_3, P_5, P_6$ lie on two lines.
The following lemma is easy to verify:

\begin{lemma}
%% v. file configN82.sage
\label{lemma:6ortog} If we have four distinct points of the projective
plane, such that no triplets of them are collinear, then the scalar 
product of at least two of them is not zero.
\end{lemma}

Furthermore:

\begin{lemma}
\label{lemma:three_s_zero}
Suppose that $P_1, \dotsc, P_7$ are eigenpoints in a $(C_8)$ configuration.
We also assume that the matrix of conditions of any $V$- configuration contained in $P_1, \dotsc, P_7$ has rank~$9$. 
Let $U, V, W$ be three points in the set $\{P_1, P_2, P_4, P_7\}$
such that $\scl{U}{V} = 0$ and $\scl{U}{W} = 0$.
Then $\scl{V}{W} = 0$.
\end{lemma}
\begin{proof}
%% vedi file config8N2.sage
From the symmetries of the points, we can assume that $U=P_1$, $V=P_2$, $W=P_7$.
We give to the points $P_2, P_4, P_7$ arbitrary coordinates and 
we define $P_1 = P_2 \times P_7$. Then $P_3$, $P_5$ and $P_6$
are determined by the $(C_8)$ configuration. The matrix of conditions of the $V$- configuration 
$(P_3, P_1, P_2, P_4, P_7)$ 
is of rank~$9$, hence $\delta_1(P_3, P_1, P_4) = 0$ 
(see~\Cref{theorem:rank_V} and~\Cref{proposition:third_alignment}). 
Similarly, $\delta_1(P_5, P_1, P_2) = 0$ and $\delta_1(P_6, P_1, P_2) = 0$. In this way we can construct an ideal generated by 
$\delta_1$ conditions as above and it can be verified that, 
after some saturations, this ideal is generated by $s_{27}$.
More details on the file \nb{07}{F6}.
\end{proof}

\begin{prop}
\label{prop:conf8_partA}
If $P_1, \dotsc, P_7$ are eigenpoints in a $(C_8)$ configuration, then, up to a permutation of 
the labels of the points, the quantity~$s_{12}$ is not zero and it holds:
\begin{equation}
\label{formula:ortocentro}
P_7 = (P_1 \times P_2)s_{14}s_{24} -
  s_{12}(P_1 \times P_4)s_{24} + s_{12}s_{14}(P_2 \times P_4)
\end{equation}
and the lines $P_1 \vee P_2$, $P_1 \vee P_4$ and~$P_1 \vee P_6$
are orthogonal to, respectively, $P_3 \vee P_4$, $P_2 \vee P_5$ and 
$P_2 \vee P_4$, hence:
%
\[
  \scl{P_1 \times P_2}{P_3 \times P_4} = 0, \quad 
  \scl{P_1 \times P_4}{P_2 \times P_5} = 0, \quad 
  \scl{P_1 \times P_6}{P_2 \times P_4} = 0 \,.
\]
%
\end{prop}
\begin{proof}
%% file config8N2.sage
If some of the $V$- configurations contained in $(C_8)$ are such that their matrices of conditions have rank $8$, 
the result follows from \Cref{rank_8} (see \Cref{rmk:three_orthog}).
We fix now generic coordinates for the points $P_1, P_2, P_4, P_7$.
As a consequence of the previous two lemmas, it is not possible
to have $s_{12}=0, s_{14}=0, s_{17}=0$, hence we can assume $s_{12} \not=0$.
Hence, again from \Cref{lemma:three_s_zero}, it is not possible to 
have $s_{14}=s_{24}=0$, hence at least one of the three coefficients~$s_{14}s_{24}$, $s_{12}s_{24}$, or~$s_{12}s_{14}$  in \Cref{formula:ortocentro} is not zero. Since the 
three vectors $P_1\times P_2$, $P_1\times P_4$, $P_2\times P_4$ are linearly independent, we see that $P_7$ is a non zero vector. Also in this general situation, if the points are eigenpoints, it must hold: 
$\delta_1(P_3, P_1, P_4) = 0$, $\delta_1(P_5, P_1, P_2) = 0$ and 
$\delta_1(P_6, P_1, P_2)=0$. These three relations, up to non zero factors, can be expressed as 
%
\begin{align*}
 e_1 &= s_{12}s_{47}-s_{17}s_{24} =0\,,\\
 e_2 &= s_{12}s_{47}-s_{14}s_{27} =0\,,\\ 
 e_3 &= s_{17}s_{24}-s_{14}s_{27} =0\,,
\end{align*}
%
hence $e_1-e_2+e_3 = 0$ and the system
$e_1=0, e_2 = 0$ is a linear system in the coordinates of the points. 
Its solution (w.r.t.\ the coordinates of~$P_7$) gives new coordinates for the point~$P_7$ and it is easy to see that \Cref{formula:ortocentro} holds. The complete computations are in \nb{07}{F7}.
\end{proof}

We have also the converse, i.e.\
\begin{prop}
Suppose $P_1, P_2, P_4$ are three points of the plane. Suppose that \Cref{formula:ortocentro} defines a point $P_7$ 
in~$\p^2$ and that
%
\[
P_3 = (P_1 \vee P_2) \cap(P_4 \vee P_7), \quad 
P_5 = (P_1 \vee P_4) \cap (P_2 \vee P_7), \quad
P_6 = (P_1 \vee P_7) \cap (P_2 \vee P_4).
\]
%
also define points in~$\p^2$.
Then the points $P_1, \dotsc, P_7$ are in a $(C_8)$ configuration
and are eigenpoints of a unique cubic 
of the plane. In particular:
\[
  \scl{P_1 \times P_2}{P_3 \times P_4} = 0 \,, \quad 
  \scl{P_1 \times P_4}{P_2 \times P_5} = 0 \,, \quad 
  \scl{P_1 \times P_6}{P_2 \times P_4} = 0 \,.
\]
\end{prop}
\begin{proof}
%% file connfig8N2.sage
It is enough to show that the rank of the matrix $\Phi(P_1, \dots, P_7)$
is $9$ and this computation can be done in only two
cases, assuming that 
$P_1$ is the point~$(1: 0: 0)$ or the point~$(1: \iii: 0)$.
(See \nb{07}{F7}).
\end{proof}

\begin{corollary}
The locus of ternary cubic forms~$F$ with a $(C_8)$ configuration has dimension~$6$.
\end{corollary}


\subsection*{Comparison with ODECO tensors}
%% file odeco.sage
As mentioned in the introduction, a class of
symmetric tensors fitting in the framework of collinearities in the eigenscheme is represented by ODECO tensors, which were introduced by~\cite{Rob} and studied by \cite{BDHE, Koiran2021, Biaggi2022}.

Possibly after an $\SO_3(\C)$ transformation, such forms are of the type
%
\[
  f = \lambda_1 x^3 +\lambda_2 y^3 + \lambda_3 z^3
  \quad \text{for some } \lambda_1, \lambda_2, \lambda_3 \in \C \,,
\]
%
and, using the language of \cite{Rob}, they admit the following three pairwise orthogonal eigenvectors $Q_1, Q_2, Q_3 \in \C^3$
%
\[
  Q_1 = \left( \frac{1}{\lambda_1},0,0 \right) \,, \quad
  Q_2 = \left( 0,\frac{1}{\lambda_2},0 \right) \,, \quad
  Q_3 = \left( 0,0,\frac{1}{\lambda_3} \right) \,.
\]
%
The remaining $4$ eigenvectors are uniquely determined, precisely:
%
\[
  Q_4 = Q_1+Q_2+Q_3\,, \ Q_5 = Q_1+Q_2,\, \ Q_6 = Q_1+Q_3 \,, \ Q_7 = Q_2+Q_3\,.
\]
%
In particular, such eigenschemes are in a $(C_8)$ configuration, with the additional condition
%
\[
  \left\langle Q_i,Q_j \right\rangle = 0
  \quad \text{ for } (i, j) \in \{(1, 2), (1, 3), (2, 3), (1, 7), (2, 6), (3, 5)\} \,.
\]
%
The seven vectors fit into the description of \Cref{prop:conf8_partA}, since $Q_3 $ is proportional to $Q_1 \times Q_2$ and this is coherent with \Cref{formula:ortocentro}
(because $\scl{Q_1}{Q_2} = 0$)
and
%
\[
  (Q_1 \vee Q_2) \times (Q_3 \vee Q_5) = 0 \,, \quad
  (Q_1 \vee Q_3) \times (Q_2 \vee Q_6) = 0 \,, \quad
  (Q_2 \vee Q_3) \times (Q_1 \vee Q_7) = 0 \,.
\]
%
In particular, the locus of ODECO forms is a $5$-dimensional sublocus of the $(C_8)$ locus.

\medskip
In Notebook \nb{07}{F8} it is possible to find a construction of a 
generic cubic which has a $(C_i)$ configuration for all possible $i$.

It is a challenging question to identify the irreducible components and the degrees of the loci of $(C_i)$ configurations.


\bibliographystyle{alphaurl}
\bibliography{biblio}

\end{document}