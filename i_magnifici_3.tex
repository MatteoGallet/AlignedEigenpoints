%\documentclass[11pt, a4paper, reqno, captions=tableheading,bibliography=totoc]{scrartcl}
%
%\usepackage[utf8]{inputenc}
%\usepackage[T1]{fontenc}
%\usepackage{mathtools}
%\usepackage{amsthm}
%\usepackage{amsmath}

%%%%%%%%%%%%%setting di Valentina
\documentclass{amsart}

\usepackage{amsthm}
\usepackage{amsmath}
\usepackage{graphicx}
\usepackage[utf8]{inputenc}
\usepackage[T1]{fontenc}
\usepackage{mathtools}
\usepackage{amssymb}
\setlength{\parindent}{.4 in}
\setlength{\textwidth}{7 in}
\setlength{\topmargin} {-.1 in}
\setlength{\evensidemargin}{0 in}
\setlength{\oddsidemargin}{0 in}
%%%%%%%%%%%%%%%%%%%fine setting di Valentina
%\usepackage{amssymb}
\usepackage{lmodern}
\usepackage[english]{babel}
\hyphenation{ei-gen-scheme}
\usepackage{booktabs}
\usepackage{float}
\usepackage{enumitem}
\usepackage{tikz,pgf}
%\usepackage[utopia]{mathdesign}
\usepackage{palatino}
\usepackage{hyperref}
\hypersetup{
    colorlinks,
    linkcolor={red!50!black},
    citecolor={blue!50!black},
    urlcolor={blue!80!black}
}
\usepackage[nameinlink]{cleveref}
\usepackage{xcolor}

\theoremstyle{plain}
\newtheorem{lemma}{Lemma}[section]
\newtheorem{prop}[lemma]{Proposition}
\newtheorem{theorem}[lemma]{Theorem}
\newtheorem{corollary}[lemma]{Corollary}
\newtheorem{conjecture}[lemma]{Conjecture}
\newtheorem{fact}[lemma]{Fact}
\newtheorem{assumption}[lemma]{Assumption}
\newtheorem*{reduction}{Reduction}
\theoremstyle{definition}
\newtheorem{definition}[lemma]{Definition}
\newtheorem{es}[lemma]{Example}
\newtheorem*{notation}{Notation}
\newtheorem{rmk}[lemma]{Remark}


\newcommand{\N}{\mathbb{N}}
\newcommand{\Z}{\mathbb{Z}}
\newcommand{\Q}{\mathbb{Q}}
\newcommand{\R}{\mathbb{R}}
\newcommand{\C}{\mathbb{C}}
\newcommand{\p}{\mathbb{P}}
\newcommand{\sP}{\mathcal{P}}
\newcommand{\sL}{\mathcal{L}}
\newcommand{\sU}{\mathcal{U}}
\newcommand{\sV}{\mathcal{V}}
\newcommand{\sF}{\mathcal{F}}
\newcommand{\sY}{\mathcal{Y}}
\newcommand{\de}{\partial}
\newcommand{\codim}{\mathrm{codim}}

\newcommand{\oo}{\mathcal{O}}
\newcommand{\Bl}{\mathrm{Bl}}

\newcommand{\iso}{\mathcal{Q}_{\mathrm{iso}}}

\newcommand{\imunit}{i}

\newcommand{\SO}{\operatorname{SO}}
\newcommand{\Eig}[1]{\mathcal{E}\!\left( {#1} \right)}
\newcommand{\Sym}{\operatorname{Sym}}
\newcommand{\polq}{{\rm Pol}_Q}
\newcommand{\comment}[1]{}

\newcommand{\scl}[2]{\left\langle {#1}, {#2} \right\rangle}

\newcommand\scalemath[2]{\scalebox{#1}{\mbox{\ensuremath{\displaystyle #2}}}}

%%\newcommand{\iii}{\textit{i}\,}  <--- @@ piu' bello?
\newcommand{\iii}{\textbf{i}}

\definecolor{MyDarkGreen}{cmyk}{0.7,0,1,0}
\newcommand{\blue}[1]{{\color{blue}  [#1]}}
\newcommand{\cbc}{\ensuremath{\mathcal{C}}}
\newcommand{\rk}{\ensuremath{\mathrm{rk}}}


\title{Eigenpoint collinearities of plane cubics}
\author[Valentina Beorchia]{Valentina Beorchia$^{\circ}$}
\address[\textsc{Valentina Beorchia}]{University of Trieste,
Department of Mathematics, Informatics and Geosciences,
Via Valerio 12/1, 34127 Trieste, Italy}
\email{beorchia@units.it}
\thanks{$^{\circ}$The researcher is a member of ``Gruppo Nazionale per le Strutture Algebriche, Geometriche e le loro Applicazioni'', INdAM. She is partially supported by MUR funds: PRIN project GEOMETRY OF ALGEBRAIC STRUCTURES: MODULI, INVARIANTS, DEFORMATIONS, PI Ugo Bruzzo, Project code: 2022BTA242.}
\author[Matteo Gallet]{Matteo Gallet$^{\diamond}$}
\address[\textsc{Matteo Gallet}]{University of Trieste,
Department of Mathematics, Informatics and Geosciences,
Via Valerio 12/1, 34127 Trieste, Italy}
\email{matteo.gallet@units.it}
\thanks{$^{\diamond}$The researcher is a member of ``Gruppo Nazionale per le Strutture Algebriche, Geometriche e le loro Applicazioni'', INdAM.}
\author[Alessandro Logar]{Alessandro Logar}
\address[\textsc{Alessandro Logar}]{University of Trieste,
Department of Mathematics, Informatics and Geosciences,
Via Valerio 12/1, 34127 Trieste, Italy}
\email{logar@units.it}
\date{}

\linespread{1.15}
\setlength{\parindent}{0pt}
\setlength{\parskip}{.25em}

\begin{document}

\begin{abstract}
 Given a homogenous polynomial in $n+1$ variables, the fixed points of the map from $\p^n$ to itself given by its gradient are called the \emph{eigenpoints} of the polynomial. We focus on cubic polynomials in three variables, which have already been object of study regarding their eigenpoints, and study configurations of eigenpoints that admit one or more alignments. We give a classification of all possible configurations with alignments: this is accomplished using both geometric characterizations of these situations and an extensive use of computer algebra.
\end{abstract}

\maketitle

\section{Introduction}
\label{introduction}

Tensors are natural generalizations of matrices in higher dimension and can be studied via their eigenvectors. There are several notions of eigenvalues and eigenvectors of tensors, as introduced independently in 2005 by Lim \cite{Lim} and Qi \cite{Qi}. All of them are of interest in several applications, like in the study of hypergraphs and the dynamical systems governed by those, see \cite[Section 4]{QZ} for an introduction or \cite{GMV} for recent developements. Eigenvalues of tensors also play an important role in the best rank-one approximation problem, which is relevant in data analysis and signal processing.
As an example, consider the problem of maximizing a polynomial function~$f$ over the unit sphere in~$\R^{n+1}$: eigenvectors of the symmetric tensor~$f$ are critical points of this optimization problem. Another interesting framework in which eigenvectors of symmetric tensors arise is the variational context: by Lim's Variational Principle \cite{Lim}, given a symmetric tensor~$f$, the critical rank-one symmetric tensors for~$f$ are exactly of the form~$v^d$, where $v$ is an eigenvector of~$f$.
This has applications in low-rank approximation of tensors (see \cite{OttSod}) as well as maximum likelihood estimation in algebraic statistics. Finally, in \cite{OO} Oeding and Ottaviani employ eigenvectors of tensors to produce an algorithm to compute Waring decompositions of homogeneous polynomials.

When dealing with eigenvectors, it is not restrictive to focus on partially symmetric tensors, that is elements of $\Sym^{d-1}\C^{n+1}\otimes\C^{n+1}$.
 If we choose a basis for $\C^{n+1}$, then we can identify $T\in \Sym^{d-1}\C^{n+1}\otimes\C^{n+1}$ with a tuple of homogeneous polynomials $(g_0,\dots,g_n)$. An \emph{eigenvector} of $T$ is a nonzero vector $v\in\C^{n+1}$ such that
%
\begin{equation}\label{eq:eigenvector_definition}
(g_0(v), \ldots, g_n(v))=\lambda v
\end{equation}
for some constant $\lambda$.
%
Since %the property of being an eigenvector
property \eqref{eq:eigenvector_definition} is preserved under scalar multiplication, it is natural to consider the rational map $T \colon \p^n\dashrightarrow\p^n$ defined by $T(P)=\bigl(g_0(P):\ldots:g_n(P)\bigr)$ and regard eigenvectors as points in $\p^n$; hence the name \emph{eigenpoints} instead of eigenvectors. In the symmetric case, we define the eigenpoints of a homogeneous polynomial~$f$ as the fixed points of the polar map, or equivalently the eigenpoints of the partially symmetric tensor $\nabla f = (\de_0f,\dots,\de_n f)$.

Moreover, condition \eqref{eq:eigenvector_definition} can be rephrased as the condition
\begin{equation}
\label{eq:def_matrix_general}
    \rk \begin{pmatrix}
    x_0 & x_1 & \dots & x_n \\
    g_0 & g_1 & \dots & g_n
    \end{pmatrix} \le 1,
\end{equation}
so that the set of eigenpoints can be given a natural structure of determinantal scheme, which is commonly called \emph{eigenscheme}, denoted $\Eig{T}$.

The dimension and degree of a general eigenscheme are settled by the following result (see \cite[Theorem 2.1]{CartSturm}, \cite{ASS}, \cite{OO}, and \cite[Equation~5.2]{Abo}):

\begin{lemma}\label{lem:nonempty}
Let $d\ge 3$ and let $T \in (\C^{n+1})^{\otimes d}$.
If $T$ is general, then $\dim \Eig{T}=0$ and $\deg \Eig{T}=\frac{(d-1)^{n+1}-1}{d-2}$.

Moreover, among $0$-dimensional eigenschemes, the general one is reduced.
\end{lemma}

The geometry of eigenpoints is intimately related with the Waring decomposition of a polynomial,
corresponding to the symmetric Tucker decomposition of the associated symmetric tensor,
as clarified
in \cite{DOT} and in \cite{Ott}. Indeed, any best rank $k$ approximation of a symmetric tensor lies in the linear
space, called {critical space}, spanned by the rank $1$ tensors of the type $v_i ^{\otimes d}$, where $v_i$ varies among all the eigenvectors. Therefore, a deep comprehension of the geometry of the subschemes of an eigenscheme will
go towards an improvement on the insight of low rank approximation problems. A first evidence in this direction is given by the so called ODECO ("orthogonally decomposable") tensors, that is symmetric tensors~$T$ that admit a decomposition
$T= \sum _{i=0}^n v_i ^{\otimes d}$ with $v_0 \ dots , v_n$ an orthogonal family; see \cite{Rob} and \cite{BDHE}
for further details. The $v_i$'s turn out to be eigenvectors of $T$; so in this case there is a subscheme of the eigenscheme with a very special geometric configuration.
\textbf{(CITARE \cite{Ottaviani24})}

 In the present paper we focus on the geometry of eigenschemes of $3\times 3 \times 3$ symmetric tensors, that is the case $n=2$ and $d=3$. By Lemma \ref{lem:nonempty}, the general such tensor
 satisfies $\deg \Eig{T}=7$. A geometric characterization of eigenschemes of ternary forms is given by the following results.

\begin{theorem} (\cite[Theorem 5.1]{ASS})
A configuration of seven points in $\p^2$ is the eigenconfiguration
of a $3\times 3 \times 3$-tensor if and only if no six of the seven points lie on a conic.
\end{theorem}

Moreover, we have also the following result (see \cite[Theorem 5.7]{BGV}):
\begin{theorem}
\label{thm:general_no_d_collinear}
Let $d\ge 3$. The eigenscheme $\Eig{f}$ of a general homogeneous polynomial $f \in \C[x_0,x_1,x_2]_d$ contains no $d$ collinear points.
\end{theorem}

However, there are several examples of cubic polynomials, whose eigenscheme contains one or more triples of aligned points, as for instance the Fermat cubic polynomial.

The goal of this paper is to classify all the situations when we have one or more triples of aligned points inside a zero-dimensional reduced eigenscheme of a cubic plane curve.
\Cref{aligned} clarifies the reasons of interest in the situation of triples of aligned eigenpoints.





\section{Aligned eigenpoints of ternary cubic form}
\label{aligned}

We recall that, given a homogeneous form $f \in \C[x,y,z]_d$ of degree~$d$, the eigenscheme~$\Eig{f}$ of~$f$ is the determinantal scheme defined by the $2 \times 2$ minors of the matrix
%
\begin{equation}
\label{eq:def_matrix}
    \begin{pmatrix}
    x & y & z \\
    \de_x f & \de_y f & \de_z f
    \end{pmatrix}.
\end{equation}
%
Since the eigenschemes of two proportional homogeneous forms are the same,
if $C$ is the curve defined by these forms,
we can write $\Eig{C}$ for such eigenscheme and hence talk about the eigenscheme of a plane curve.

The eigenscheme of a general ternary cubic form has no aligned triples of points. This is a consequence of the geometric properties of the classical Geiser map associated with seven points in the plane, and has been proved in \cite[Proposition~4.5]{BGV}.

Moreover, whenever the eigenscheme of a ternary form is zero-dimensional, it never contains $4$ or more aligned points.
To see this, we recall a property of eigenschemes of ternary forms, namely,
that when zero-dimensional they are somehow ``general'' with respect to conics.

\begin{lemma}
\label{lemma:no_six_conic}
Let $f \in \C[x,y,z]_d$ be a homogeneous form of degree~$d$.
If $\Eig{f}$ is zero-dimensional,
then no degree six subscheme of~$\Eig{f}$ lies on a conic.
\end{lemma}
\begin{proof}
See \cite[Lemma~9.1]{OS1} for the reduced case.
The proof works also in the non-reduced case.
\end{proof}

\begin{corollary}
\label{corollary:general_no_triple}
As a consequence of \Cref{lemma:no_six_conic}, a zero-dimensional eigenscheme~$\Eig{f}$ for $f \in \C[x,y,z]_d$ never contains $4$ or more aligned points.
Moreover, if $\Eig{f}$ contains two triples of aligned points, those must share a point.
\end{corollary}

\section{Invariance under the action of orthogonal matrices}
\label{invariance}

In what follows, it will be useful to fix particular coordinates for points and lines related to eigenschemes. To do that, we employ a property of invariance of eigenschemes with respect to the action of the following group.

\begin{definition}
 We define $\mathrm{SO}_3(\mathbb{C})$ to be the complexification of the group of special orthogonal real matrices, namely
 %
 \[
  \mathrm{SO}_3(\mathbb{C}) :=
  \bigl\{
   M \in \mathrm{GL}_3(\C) \, \mid \,
   M M^t = I_3 \ \text{and} \ \det(M) = 1
  \bigr\} \,.
 \]
 %
 The group $\mathrm{SO}_3(\mathbb{C})$ acts on $\C^3$ by matrix multiplication:
 %
 \[
  \begin{array}{ccc}
   \mathrm{SO}_3(\mathbb{C}) \times \C^3 & \rightarrow & \C^3 \\
   (M, v) & \mapsto & Mv
  \end{array}
 \]
 %
 Since all the elements of $\mathrm{SO}_3(\mathbb{C})$ are invertible, the latter action descends to an action on $\p^2(\C)$.

 Moreover, the group~$\mathrm{SO}_3(\mathbb{C})$ acts also on ternary forms via
 \[
  M \cdot f (x,y,z) = f(M^{-1} \cdot \prescript{t} {}( x \ y \ z ) ).
 \]
\end{definition}

\begin{prop}
\label{two_orbits}
 The action of $\mathrm{SO}_3(\mathbb{C})$ on $\p^2(\C)$ has two orbits:
 %
 \begin{align*}
  \mathcal{O}_1 &:=
  \bigl\{
   P \in \p^2(\C) \, | \,
   P = (a:b:c) \ \text{with} \ a^2 + b^2 + c^2 = 0
  \bigr\} \\
  \mathcal{O}_2 &:= \p^2(\C) \setminus \mathcal{O}_1
 \end{align*}
 %
 A representative for $\mathcal{O}_1$ is $(1:\iii:0)$ and a
representative for $\mathcal{O}_2$ is $(1:0:0)$.
The orbit $\mathcal{O}_1$ is the set of points of the so-called \emph{isotropic conic}, denoted $\iso$.
\end{prop}
\begin{proof}
 Suppose that $P \in \p^2(\C)$ and $P = (a:b:c)$ with $a^2 + b^2 + c^2 = 0$.
 We produce a matrix $M \in \mathrm{SO}_3(\C)$ such that $M \left(\begin{smallmatrix} 1 \\ \iii \\ 0 \end{smallmatrix}\right)$ and $\left(\begin{smallmatrix} a \\ b \\ c \end{smallmatrix}\right)$ are proportional.
 Up to relabeling the coordinates, we can suppose that $a \neq 0$.
 Hence, by rescaling the coordinates of $P$, we have $P = (1: b: c)$ with $b^2 + c^2 = -1$.
 One can check that the matrix
 %
 \[
  M :=
  \begin{pmatrix}
   -1 & 0 & 0 \\
   0 & \iii b & -\iii c \\
   0 & \iii c & \iii b
  \end{pmatrix}
 \]
 %
 satisfies the requirements.

 Now suppose that $P \in \p^2(\C)$ and $P = (a:b:c)$ with $a^2 + b^2 + c^2 \neq 0$.
 Up to rescaling, we can suppose that $a^2 + b^2 + c^2 = 1$.
 Again, we produce a matrix $M \in \mathrm{SO}_3(\C)$ such that $M \left(\begin{smallmatrix} 1 \\ 0 \\ 0 \end{smallmatrix}\right)$ and $\left(\begin{smallmatrix} a \\ b \\ c \end{smallmatrix}\right)$ are proportional.
 First of all, suppose that $b^2 + c^2 \neq 0$ and let $\omega$ be a root of the polynomial $t^2 - (b^2 + c^2)$ in $\C[t]$.
 Then, the matrix
 %
 \[
   M :=
   \begin{pmatrix}
     a & \omega & 0 \\
     b & -\frac{ab}{\omega} & \frac{c}{\omega} \\
     c & -\frac{ac}{\omega} & -\frac{b}{\omega}
   \end{pmatrix}
 \]
 %
 satisfies the requirements.
 With the same technique, if $a^2 + c^2 \neq 0$, we can produce a matrix $M \in \mathrm{SO}_3(\C)$ that maps $\left(\begin{smallmatrix} 0 \\ 1 \\ 0 \end{smallmatrix}\right)$ to $\left(\begin{smallmatrix} a \\ b \\ c \end{smallmatrix}\right)$; similarly, when $a^2 + b^2 \neq 0$, we can map $\left(\begin{smallmatrix} 0 \\ 0 \\ 1 \end{smallmatrix}\right)$ to $\left(\begin{smallmatrix} a \\ b \\ c \end{smallmatrix}\right)$.
 Since $\left(\begin{smallmatrix} 1 \\ 0 \\ 0 \end{smallmatrix}\right)$, $\left(\begin{smallmatrix} 0 \\ 1 \\ 0 \end{smallmatrix}\right)$, and $\left(\begin{smallmatrix} 0 \\ 0 \\ 1 \end{smallmatrix}\right)$ are all $\mathrm{SO}_3(\C)$-equivalent, the only case to consider is when
 %
 \[
  b^2 + c^2 = a^2 + c^2 = a^2 + b^2 = 0 \,,
 \]
 %
 which, however, can never occur.
\end{proof}

The following result is well known; we recall it for the sake of completeness.

\begin{prop}
 Let $M \in \mathrm{GL}_3(\C)$ and let $f$ be a ternary cubic.
 Let $P = (A: B: C)$ be a point in~$\p^2$.
 Then we have
 %
 \[
  P \in \Eig{f} \iff M \cdot \prescript{t} {}(A \ B \ C) \in \Eig{M \cdot f} \,.
 \]
 %
\end{prop}
\begin{proof}
In the present proof,
for convenience, we shall consider the transpose of the defining matrix of an eigenscheme.

A point $P = (A: B: C)$ is an eigenpoint for~$f$ if and only if
\begin{equation*}
  \mathrm{rk}  \begin{pmatrix}
    A & \de_x f(P) \\
    B & \de_y f(P)  \\
    C & \de_z f(P)
    \end{pmatrix}=1,
\end{equation*}
which is equivalent to
\begin{equation}
\label{eq:def_matrix_M}
    \mathrm{rk} \quad M \cdot \begin{pmatrix}
    A & \de_x f(P) \\
    B & \de_y f(P)  \\
    C & \de_z f(P)
    \end{pmatrix}
    =1.
\end{equation}
By setting $\prescript{t} {}(A' \ B' \ C' )= M \cdot \prescript{t} {}(A \ B \ C) $ and $Q=(A':B':C')$, we have that \Cref{eq:def_matrix_M}
is equivalent to
%
\begin{equation}
\label{eq:transformed}
  \rk
  \begin{pmatrix}
    A' & \\
    B' & M \cdot \nabla f (P) \\
    C' & \\
  \end{pmatrix}=1.
\end{equation}
%
Now we consider the polynomial $M \cdot f$ and we observe that the chain rule gives
%
\begin{gather*}
\partial_x (M\cdot f) = \partial_x \bigl( f(M^{-1} \ \prescript{t} {} (x \ y \ z)) \bigr) = \prescript{t} {}(M^{-1})^{(1)}(\nabla f) \bigl( M^{-1}\  \prescript{t} {} (x \ y \ z) \bigr) \,, \\
\partial_y (M\cdot f) = \partial_y \bigl( f(M^{-1} \ \prescript{t} {} (x \ y \ z)) \bigr) = \prescript{t} {}(M^{-1})^{(2)}(\nabla f) \bigl( M^{-1}\  \prescript{t} {} (x \ y \ z) \bigr) \,, \\
\partial_z (M\cdot f) = \partial_z \bigl( f(M^{-1} \ \prescript{t} {} (x \ y \ z)) \bigr) = \prescript{t} {}(M^{-1})^{(3)}(\nabla f) \bigl( M^{-1}\  \prescript{t} {} (x \ y \ z) \bigr) \,,
\end{gather*}
%
where $(M^{-1})^{(j)}$ denotes the $j$-th column of the matrix $M^{-1}$. Hence
%
\[
\nabla (M \cdot f) = \prescript{t} {} M^{-1} \cdot (\nabla f) (M^{-1}\  \prescript{t} {} (x \ y \ z)),
\]
%
so we have
%
\[
\nabla (M \cdot f)(Q)=\nabla (M \cdot f)(M \cdot P)=
\prescript{t} {} M^{-1} \cdot (\nabla f) (M^{-1}\  M \cdot P)=\prescript{t} {} M^{-1} \cdot (\nabla f)(P).
\]
%
Finally, if we choose $M \in \mathrm{SO}_3(\mathbb{C})$, we have
$\prescript{t} {} M^{-1}=M$. We deduce that
\Cref{eq:transformed} holds if and only if $Q \in \Eig{M\cdot f}$, so the statement is proved.
\end{proof}

\section{Conditions imposed by aligned eigenpoints}
\label{conditions}

\subsection{The matrix of conditions}

Imposing a cubic ternary form to have one or more aligned triples of eigenschemes implies conditions both on the points and on the cubics.
We begin to explore these conditions by introducing, for each point
$P \in \p^2$,
a $3 \times 10$ matrix encoding the condition that $P$ is an eigenpoint of a ternary cubic.

\begin{definition}
\label{definition:matrix_conditions}
 Consider $\p^9 = \p(\C[x,y,z]_3)$, the space of all ternary cubics.
 Through this paper, we will consider the standard monomial basis for $\C[x,y,z]_3$ and we set~$\mathcal{B}$ to be the following vector
 \begin{eqnarray}
  \mathcal{B} = (x^3, x^2 y, x y^2, y^3, x^2 z, x y z, y^2 z, x z^2, y z^2, z^3)
  \in \bigl( \C[x,y,z]_3 \bigr)^{\oplus 10} \,.
  \label{vector_basis}
 \end{eqnarray}
 For $f \in \C[x,y,z]_3$, denote by $[f]$ the corresponding point in~$\p^9$; we denote by $w_f$ the (column) vector of coordinates of~$F$ with respect to the basis above; then $w_f$ is also a vector of projective coordinates of~$[f]$.
 For a point $P \in \p^2$ with coordinates $(A: B: C)$, the condition on elements~$[f]$ of~$\p^9$ that $P$ is an eigenpoint of the ternary cubic form~$F$ can be expressed in the form
 %
 \[
  \Phi(P) \cdot w_f
  = 0 \,,
 \]
 %
 where $\Phi(P)$ is a $3 \times 10$ matrix with entries depending on $A, B, C$.
 The matrix $\Phi(P)$ is called the \emph{matrix of conditions} imposed by~$P$.
We denote by $\phi_1(P)$, $\phi_2(P)$, and~$\phi_3(P)$ the rows of~$\Phi(P)$.
Written as vectors, they are
%
\begin{equation}
\label{equation:matrix_conditions_rows}
\begin{gathered}
\scalemath{0.9}{(-3A^2B, A(A^2 - 2B^2), B(2A^2 - B^2), 3AB^2,
 -2ABC, C(A^2 - B^2), 2 ABC,
 -B C^2, A C^2, 0)} \,, \\
\scalemath{0.9}{(-3A^2 C,
-2ABC,
-CB^2,
0,
A(A^2-2C^2),
B(A^2 - C^2),
AB^2,
C(2A^2-C^2),
2ABC,
3AC^2)} \,,\\
\scalemath{0.9}{(0,
-A^2C,
-2ABC,
-3CB^2,
A^2 B,
A(B^2 - C^2),
B(B^2-2C^2),
2ABC,
C(2B^2-C^2),
3BC^2)} \,.
\end{gathered}
\end{equation}
%
More generally, if $P_1, \dotsc, P_n$ are points in the plane, we denote by $\Phi(P_1, \dotsc, P_n)$ the matrix
%
\[
 \left(
 \begin{array}{c}
  \Phi(P_1) \\
  \vdots \\
  \Phi(P_n)
 \end{array}
 \right)
\]
%
namely, the $3n \times 10$ matrix obtained by vertically stacking the
matrices of conditions of~$P_1, \dotsc, P_n$ and we call it again the
\emph{matrix of
conditions} imposed by $P_1, \dotsc, P_n$. \\
Moreover, given a matrix $M$ of type $m \times 10$, we denote by
$\Lambda(M)$ the linear system of cubics whose coefficients
satisfy the equations associated to the matrix $M$.
\end{definition}

We introduce a new piece of notation that makes the description of $\Phi(P)$ more compact: $\de_x \mathcal{B}$ denotes the vector whose entries are the derivatives of those of $\mathcal{B}$, and similarly for $\de_y \mathcal{B}$ and $\de_z \mathcal{B}$.
In this way, if $P=(A: B: C)$, we get:
%
\begin{equation}
\label{equation:vector_conditions}
\begin{gathered}
\phi_1(P) = A\cdot \de_x \mathcal{B}(P) - B\cdot \de_y \mathcal{B}(P) \,, \\
\phi_2(P) = A\cdot \de_z \mathcal{B}(P) - C\cdot \de_y \mathcal{B}(P) \,, \\
\phi_3(P) = B\cdot \de_z \mathcal{B}(P) - C\cdot \de_y \mathcal{B}(P) \,.
\end{gathered}
\end{equation}
%
\begin{rmk}
By analyzing the entries of \Cref{equation:matrix_conditions_rows}, it is not difficult to check that the rank of $\Phi(P)$ is never $\leq 1$. \Cref{equation:vector_conditions} gives
\begin{equation}
  C \, \phi_1(P) - B \, \phi_2(P) + A \, \phi_3(P) = 0 \,,
  \label{eq:base}
\end{equation}
therefore,
among the three vectors, at most two of them are linearly independent and the matrix~$\Phi(P)$ has rank~$2$.
\end{rmk}

As a consequence, if $P_1, \dots, P_n$ are $n$ points of the plane, we have:
\begin{equation}
\label{bound_rank}
\rk \ \Phi(P_1, \dots, P_n) \leq \min \left\{2n, 10 \right\} \,.
\end{equation}

\subsection{Possible ranks of the matrix of conditions}

In what follows, we want to study the possible values of the rank of the matrix
$\Phi(P_1, \dots, P_n)$ for several configurations of points $P_1, \dots, P_n$
(and several values of $n$).
In particular, we will study the ideal~$J_k$ of order $k$ minors of the
involved matrix and we will deduce some bounds about the rank from the possible
decompositions of the ideal~$J_k$. Most of these computations will be done
with the aid of a computer algebra system. Nevertheless, in many cases,
the result cannot be reached just by brute force, but it is necessary to
make some preprocessing on the ideal~$J_k$. In particular, it turns out that
it is often convenient to first saturate the ideal~$J_k$ with respect to
the distinct point condition or that three of them are not aligned (when this is the
case). Another important simplification that we adopt sometimes, makes use
of the action of $\SO_3(\C)$: thanks to it we can assume that one of
the point is either $(1: 0: 0)$ or $(1: \iii: 0)$; see \Cref{two_orbits}.

We start with the following lemma, which will be extremely useful
to speed up the computations.

\begin{lemma}
\label{lemma:minors}
Let $l_1 < \cdots <l_n$ be $n$ indices (where $3 \leq n \leq 10$) and let $P = (A: B: C)$ be a point of the plane.
Construct three $1 \times n$ matrices $w_1$, $w_2$, $w_3$ by extracting the entries of position $l_1, \dotsc, l_n$ from $\phi_1(P)$, $\phi_2(P)$, and~$\phi_3(P)$, respectively. If $L$ is a $(n-2) \times n$ matrix, set:
  \[
  L_1 := \left(\begin{array}{c}w_1 \\ w_2 \\ L\end{array} \right), \quad
  L_2 := \left(\begin{array}{c}w_1 \\ w_3 \\ L\end{array} \right), \quad
  L_3 := \left(\begin{array}{c}w_2 \\ w_3 \\ L\end{array} \right)
  \]
  Then
  \[
  B \det(L_1) = A \det(L_2), \quad
  C \det(L_1) = A \det(L_3), \quad
  C \det(L_2) = B \det(L_3)
  \]
  hence $(A: B: C) = \bigl( \det(L_1): \det(L_2): \det(L_3) \bigr)$.
\end{lemma}
\begin{proof}
  The thesis easily follows from the equality $C w_1 - B w_2 + A w_3 = 0$, which is a direct consequence of \Cref{eq:base}.
\end{proof}

\Cref{lemma:minors} justifies the following choice.

\begin{definition}
 \label{definition:reduced_matrix_conditions}
 For $n$ points $P_1, \dotsc, P_n$ in the plane, the \emph{cut matrix of conditions} of $P_1, \dotsc, P_n$ is the submatrix of~$\Phi(P_1, \dotsc, P_n)$ whose rows are $\phi_1(P_1), \phi_2(P_1), \dotsc, \phi_1(P_n), \phi_2(P_n)$.
\end{definition}

Next, we point out a property of the lines, which are tangent to the absolute conic~$\iso$.
First of all we fix a notation that will be used through the paper. If
$P_i = (A_i: B_i: C_i)$ and $P_j = (A_j: B_j: C_j)$, we set
\[
s_{ij} = \scl{P_i}{P_j} = A_i A_j + B_i B_j + C_i C_j
\]

\begin{definition}
\label{definition:sigma}
We define a bihomogeneous polynomial of bidegree~$(2,2)$ on $\p^2 \times \p^2$: for $P_1 = (A_1: B_1: C_1)$ and $P_2 = (A_2: B_2: C_2)$, we set
%
\begin{eqnarray}
\label{formula:sigma}
  \sigma(P_1, P_2) &=& \scl{P_1}{P_1} \scl{P_2}{P_2} - \scl{P_1}{P_2}^2 \\
   & = & s_{11}s_{22}-s_{12}^2
\end{eqnarray}
\end{definition}

\begin{rmk}\label{rmk:sigma_discr}
The form $\sigma$ is the discriminant of the intersection between the line $P_1 \vee P_2$ and $\iso$.
Since the general point of the line $P_1 \vee P_2$ can be expressed in the form
$t P_1 + sP_2$ where $(t:s) \in \p^1$, the intersection between $P_1+P_2$
and $\iso$ is determined by the pairs $( \overline t: \overline s)\in \p^1$ which are solutions of
$$
\scl{t P_1 + sP_2}{t P_1 + sP_2} =0.
$$
The bilinearity and symmetry of the form $\scl{}{}$ gives the quadratic polynomial
\begin{equation}\label{eq:intersection_isotropic}
\scl{t P_1 + sP_2}{t P_1 + sP_2}= \scl{P_1}{P_1} t^2
+2\scl{P_1}{P_2} ts + \scl{P_2}{P_2}s^2,
\end{equation}
whose discriminant is $4 ( \scl{P_1}{P_2}^2 - \scl{P_1}{P_1} \scl{P_2}{P_2})$.
Hence the condition $\sigma(P_1,P_2)=0$ implies that the polynomial in \Cref{eq:intersection_isotropic}
has a double root, and this proves the statement.
\end{rmk}

\begin{prop}
\label{proposition:sigma_tangency}
  Let $P_1$, $P_2$ be two distinct points in the plane and let $r$ be the line joining them.
  Then the following are equivalent:
  \begin{enumerate}
  \item $\sigma(P_1, P_2) = 0$;
  \item $\sigma(Q_1, Q_2) = 0$ for all pairs of distinct points $Q_1, Q_2 \in r$;
  \item the line $r$ is tangent to~$\iso$ at some point;
  \item there exists a point $T \in r$ such that $T \in \iso$ and $\scl{T}{Q} = 0$ for all $Q \in r$, $Q \neq T$; in this case we say that $T$ is \emph{orthogonal} to $r$.
  \end{enumerate}
\end{prop}
\begin{proof}
  By \Cref{rmk:sigma_discr}, we have that $\sigma(P_1, P_2) = 0$ if and only if $r$ is tangent to~$\iso$ in a point~$T$; this shows that the first three items are equivalent.

  Now, if $\sigma(P_1, P_2) = 0$, then $r$ is tangent to~$\iso$ at a point~$T$, hence $\scl{T}{T} = 0$. Moreover, we also have that $\sigma(T, Q) = 0$ for all $Q \in r$ with $Q \neq T$. Hence, $T$ is orthogonal to~$r$.
  The converse is immediate.
\end{proof}

\begin{prop}
\label{proposition:three_aligned_ranks}
Let $P_1, P_2, P_3$ be three distinct aligned points of the plane and let
$r$ be the line passing through them. Then:
\begin{itemize}
\item $5 \leq \rk \ \Phi(P_1, P_2, P_3) \leq 6$;
\item
$\rk \ \Phi(P_1, P_2, P_3) = 5$ if and only if $r$ is tangent
to~$\iso$ in one of the three points $P_1, P_2$, or $P_3$.
\end{itemize}
\end{prop}
\begin{proof} We denote the coordinates of the points as follows:
%
\[
P_1 = (A_1: B_1: C_1) \,, \quad P_2 = (A_2: B_2: C_2) \,, \quad P_3 = u_1P_1+u_2P_2 \,.
\]
%
The characterization of the triples $P_1, P_2, P_3$ of distinct points such that $\rk \ \Phi(P_1, P_2, P_3) < 6$ follows from the computation of a suitable saturation of the ideal generated by the $17640$ order six minors of $\Phi(P_1, P_2, P_3)$.
We claim that we can greatly simplify the computations.

Indeed, consider the matrix~$M$ constructed with the following six rows:
\[
\phi_{i_1}(P_1), \ \phi_{i_2}(P_1), \ \phi_{j_1}(P_2),\ \phi_{j_2}(P_2),
\phi_{k_1}(P_3), \ \phi_{k_2}(P_3)
\]
where $i_1 \not= i_2 \in \{1, 2, 3\}$; $j_1 \not= j_2 \in \{1, 2, 3\}$;
$k_1 \not= k_2 \in \{1, 2, 3\}$.
First, assume that $i_1=j_1=k_1=1$ and
$i_2=j_2=k_2=2$. Fix six columns
$1\leq l_1 < \cdots < l_6 \leq 10$ of $M$ and let $N$ be the order $6$ matrix
obtained from $M$ with these columns. Since two rows of $N$ are obtained from
entries of $\phi_1(P_1)$ and of $\phi_2(P_1)$,
\Cref{lemma:minors} gives that $A_1$ divides $\det(N)$. For a similar
reason, also $A_2$ and $u_1A_1+u_2A_2$ divide $\det(N)$, so
$\det(N) = A_1A_2(u_1A_1+u_2A_2)\cdot D$ for a suitable $D$.
Again by \Cref{lemma:minors}, we also see that if we take different values of
$i_1, i_2, j_1, j_2, k_1, k_2$ in the definition of $M$ above, then
the corresponding $\det(N)$ would be of the form $X_1X_2X_3\cdot D$ where
$X_1$ is a coordinate of $P_1$, $X_2$ is a coordinate of $P_2$ and $X_3$ is
a coordinate of $P_3$ (and $D$ is the same as above). Since each of the three
points have at least one non-zero coordinate, if the order six minor
constructed with the columns $l_1, \dots, l_6$ is zero, then necessarily $D$
must be zero. In this way we see that, in order to have that all the order
six minors of $\Phi(P_1, P_2, P_3)$ are zero, it is enough to compute the
ideal of the order six minors of the matrix $M$ above and divide each
minor by $A_1A_2(u_1A_1+u_2A_2)$. By this procedure, after
a saturation w.r.t.\ the condition that $P_1$, $P_2$, and $P_1, P_3$ and
$P_2$, $P_3$ are distinct, we get an ideal that has
a very simple primary decomposition, which is given by the following three
ideals:
%
\[
\bigl( \scl{P_i}{P_1}, \scl{P_i}{P_2}, \scl{P_i}{P_3} \bigr) \quad
\mbox{for } i = 1, 2, 3 \,.
\]
%
As a consequence, the line $r$ is tangent to~$\iso$
(in $P_1$ or $P_2$ or $P_3$).
It is not difficult to see via a symbolic computation that it is not possible to have that all the order
$4$-minors of $\Phi(P_1, P_2, P_3)$ are zero.
\end{proof}

\begin{prop}
\label{manca il riferimento su ancillary  non e': condition_rank_aligned}
%%%  conto si trova su file prop47_5ago.sage
Let $P_1, P_2, P_4$ be three distinct points of the plane. Then:
%
\begin{itemize}
 \item $5 \leq \rk \ \Phi(P_1, P_2, P_4) \leq 6$;
 \item if
 $\rk \ \Phi(P_1, P_2, P_4) = 5$, then $P_1, P_2, P_4$
 are aligned and the line joining them is tangent to~$\iso$
 in one of the three points.
\end{itemize}
%
\end{prop}
\begin{proof}
By \Cref{two_orbits}, we can split the proof into two parts, considering the case in
which $P_4 = (1: 0: 0)$ and the case in which $P_4 = (1: \iii: 0)$.
In both cases, the computation of the ideal of order five minors of the matrix
$\Phi(P_1, P_2, P_4)$ and the saturation of it w.r.t.\ the distinct point condition gives the whole ring, so the matrix cannot have rank smaller than~$5$.
Meanwhile, the computation
of the ideal of the order six minors of $\Phi(P_1, P_2, P_4)$ and its
saturation w.r.t.\ the distinct point condition gives that
$P_1, P_2, P_4$ must be aligned and, according to \Cref{proposition:three_aligned_ranks},
that line is tangent to~$\iso$ in one of the three given points.
\end{proof}


\begin{prop}
\label{prop:condition3+1}
Let $P_1, P_2, P_3, P_4$ be four distinct points of the plane such that
$P_1, P_2, P_3$ are aligned and let $r$
be the line joining them. If
$\rk \ \Phi(P_1, P_2, P_3, P_4) \leq 7$ then $r$ is tangent to~$\iso$ in one of the three points $P_1, P_2, P_3$.
\end{prop}
\begin{proof}
Again, we distinguish two cases: $P_1 = (1: 0: 0)$ and
$P_1 = (1: \iii: 0)$. In both cases, the ideal of the order $8$
minors of $\Phi(P_1, P_2, P_3, P_4)$ can be computed and saturated
w.r.t.\ the distinct point condition and that
$P_1, P_2, P_4$ are not aligned.
The direct inspection of the ideal obtained from these procedures gives the thesis.
\end{proof}

\begin{rmk}
 As we clarified in \Cref{thm:general_no_d_collinear}, four aligned eigenpoints are never contained in a zero-dimensional, eigenscheme. In \Cref{4ptiSuRetta2}, we will show that if an eigenscheme contains four distinct aligned eigenpoints, then it contains the whole line joining them.
\end{rmk}

\begin{prop}
\label{proposition:four_aligned}
Let $Q_1, \dotsc, Q_4$ be four distinct aligned points of the plane and
let $r$ be the line through them. Then:
\begin{itemize}
\item $6 \leq \rk \ \Phi(Q_1, \dotsc, Q_4) \leq 7$;
\item $\rk \ \Phi(Q_1, \dotsc, Q_4) = 6$ if and only if $r$ is tangent
to~$\iso$.
\end{itemize}
\end{prop}
\begin{proof}
  A symbolic computation shows that all the maximal minors of~$\Phi(Q_1, \dotsc, Q_4)$ are
  zero, so
  \mbox{$\rk \, \Phi(Q_1, \dotsc, Q_4) \leq 7$}.
  Indeed, in view of \Cref{lemma:minors}, we can just check the order~$8$ minors of the cut matrix of conditions of $Q_1, \dotsc, Q_4$; it turns out that they are all zero.

  On the other hand, the rank of $\Phi(Q_1, \dotsc, Q_4)$ cannot be $5$:
  if this were the case, for any triple~$T$ extracted from the four points, the matrix of conditions would have rank~$5$, hence by \Cref{proposition:three_aligned_ranks} the line~$r$ would be tangent to~$\iso$ in an element of $T$; this is a contradiction.
  So, the first item of the statement is proven.

  We proceed to prove the second item.
  We take the cut matrix of conditions of~$Q_1, Q_2, Q_3$, to which we add the row~$\phi_1(Q_4)$.
  We call this matrix~$M_1$.
  Similarly, we construct the matrices $M_2$ and~$M_3$.
  We consider the coordinates of the points as variables and we compute
  the ideal $J_1$, the radical of the ideal of the maximal minors of $M_1$; similarly, we obtain $J_2$ and~$J_3$.
  The saturation of the ideal sum $J_1 + J_2 + J_3$ by the distinct point condition is a principal ideal generated by~$\sigma(r)$.
  Since the four points $Q_1, \dotsc, Q_4$ play symmetric roles, the statement is proven.
\end{proof}

In a similar way, it is possible to generalize the above proposition:
\begin{prop}
\label{proposition:n_aligned}
Let $Q_1, \dotsc, Q_r$ be $r\geq 4$ distinct aligned points of the plane and
let $r$ be the line through them. Then:
\begin{itemize}
\item $6 \leq \rk \ \Phi(Q_1, \dotsc, Q_r) \leq 7$;
\item $\rk \ \Phi(Q_1, \dotsc, Q_r) = 6$ if and only if $r$ is tangent
to~$\iso$.
\end{itemize}
\end{prop}

\subsection{The conditions \texorpdfstring{$\delta_1$}{delta1}, \texorpdfstring{$\bar{\delta}_1$}{deltabar1}, and \texorpdfstring{$\delta_2$}{delta2}}

We now define three quantities depending on a triple or on a $5$-tuple of points in the plane.
These quantities are crucial to describe what happens when we have aligned eigenpoints.
In the following, we denote by $P + Q$ the line through two distinct points $P$ and $Q$.

\begin{definition}
\label{definition:delta1}
 Let $P_1$, $P_2$ and~$P_4$ be distinct points in the plane.
 We define the homogeneous polynomial of degree~$(2,1,1)$ on $\p^2 \times \p^2 \times \p^2$:
 %
 \[
  \delta_1(P_1, P_2, P_4) :=
  \scl{P_1}{P_1} \scl{P_2}{P_4} - \scl{P_1}{P_2}\scl{P_1}{P_4}
  =
  \scl{P_1\times P_2}{P_1 \times P_4} \,,
 \]
 %
 where $\times$ denotes the cross product, i.e.,
 %
 \[
  P_1 \times P_2 = (B_1 C_2 - C_1 B_2, \, C_1 A_2 - A_1 C_2, \, A_1 B_2 - B_1 A_2) \,.
 \]
 %
 and $P_1 = (A_1: B_1: C_1)$ and similarly for $P_2$ and $P_4$.
\end{definition}

\begin{rmk}\label{rmk: significato di delta1}
 Geometrically, the condition $\delta_1(P_1, P_2, P_4) = 0$ corresponds to the orthogonality of the vector planes corresponding to the lines $P_1 \vee P_2$ and $P_1 \vee P_4$, if the three points are not collinear, while
 $\delta_1(P_1, P_2, P_4) = 0$ implies that $\sigma (P_1,P_2)=0$ if they are collinear.
\end{rmk}


\begin{definition}
\label{definition:delta1b}
 Let $P_1$, $P_2$ and~$P_3$ be distinct aligned points in the plane.
 We define the polynomial
 %
 \[
  \overline{\delta}_1(P_1, P_2, P_3) :=
  \scl{P_1}{P_1} \scl{P_2}{P_3} + \scl{P_1}{P_2}\scl{P_1}{P_3} \,.
  \]
 %
\end{definition}

\begin{definition}
\label{Vconf}
Let $P_1, P_2, P_3, P_4, P_5$ be five distinct points of the plane~$\p^2$
such that $P_1, P_2, P_3$ and $P_1, P_4, P_5$ are aligned.
We call such a configuration a \emph{$V$-configuration}.
\end{definition}


\begin{definition}
 Let $P_1, \dots, P_5$ be a $V$-configuration.
We define the polynomial
 %
 \[
  \delta_2(P_1, P_2, P_3, P_4, P_5) :=
  \scl{P_1}{P_2} \scl{P_1}{P_3} \scl{P_4}{P_5} -
  \scl{P_1}{P_4} \scl{P_1}{P_5} \scl{P_2}{P_3} \,.
 \]
 %
\end{definition}
%%
Recall that $\scl{P_i}{P_j}$ is also denoted by $s_{ij}$.
Then the expressions $\delta_1$ and $\delta_2$ become
\[
\delta_1(P_1, P_2, P_4) = s_{11} s_{24}-s_{12}s_{14},
\quad \delta_2(P_1, P_2, P_3, P_4, P_5) =s_{12}s_{13}s_{45}-s_{14}s_{15} s_{23}.
\]
%
\begin{rmk}
\label{rmk:characteristics_d1_d2}
\textbf{Verifiche contenute nel file} \verb+formulae.sage+ \\
It is possible to see that
%
\begin{align}
\label{rmk_delta_case1}
\delta_1(P_1, P_2, P_4) = 0 \mbox{ iff } &\scl{P_4}{s_{11}P_2-s_{12}P_1} = 0\\
 \mbox{iff } &\scl{P_2}{s_{11}P_4-s_{14}P_1} = 0 \nonumber
\end{align}
\begin{align}
\label{rmk_delta_case2}
\overline{\delta}_1(P_1, P_2, P_3) = s_{11} s_{23}+s_{12}s_{13} =0 \mbox{ iff } &
P_1 \mbox{ is on~$\iso$ and } P_1 + P_2 + P_3 \mbox{ is tangent to~$\iso$ in $P_1$, or} \\
& P_3 = (s_{12}^2+s_{11}s_{22})P_1-2s_{11}s_{12}P_2 \mbox{ or} \nonumber \\
& P_2 = (s_{13}^2+s_{11}s_{33})P_1-2s_{11}s_{13}P_3 \nonumber
\end{align}

\end{rmk}
Suppose the point $P_1$ is fixed. In general, in order to have
$\delta_2(P_1, P_2, P_3, P_4, P_5) = 0$, we can fix three of the other points
and determine the fourth one, by observing
that $\delta_2=0$ gives a linear equation in its coordinates. There are
however
some exceptions. From the definition of $\delta_2$ we see for instance that
if $P_2$ and $P_4$ satisfy $\{s_{12}=0, s_{14}=0\}$ or $P_2$ and
$P_5$ satisfy $\{s_{12}=0, s_{15}=0\}$ or $P_2$ and $P_3$ satisfy
$\{s_{12}=0, s_{23}=0\}$, etc., then $\delta_2$ is zero,
despite of the value of the other points. To be more precise, we have:
%
\begin{prop}
\label{prop:definitionP3}
Let $P_1, \dots, P_5$ be five points in a $V$-configuration. Then it holds:
$\delta_2(P_1, P_2, P_3, P_4, P_5) = 0$ if and only if (up to a permutation
of some of the points $P_2, \dots, P_4$) one of the following conditions
is satisfied:
\begin{enumerate}
\item $s_{12} = 0$ and $s_{14} = 0$;
\label{defP3_1}
\item $s_{12} = 0$ and $s_{22} = 0$;
\label{defP3_2}
\item $\sigma(P_1, P_2) = 0$ and $\sigma(P_1, P_4) = 0$;
\label{defP3_3}
\item $P_3 = (s_{14}s_{15}s_{22}-s_{12}^2s_{45})P_1  +s_{12}(s_{11}s_{45}-s_{14}s_{15})P_2$.
\label{defP3_4}
\end{enumerate}
\end{prop}
%
\begin{proof}
\verb+whenP3isNotDefined.sage+\\
Condition~(\ref{defP3_4}), when defined (i.e., when the coefficients of $P_1$ and~$P_2$ are not zero), easily comes from the definition of $\delta_2$.
From it we see that $P_3$ is not defined by that formula if and only if
$s_{14}s_{15}s_{22}-s_{12}^2s_{45}=0$ and $s_{11}s_{12}s_{45}-s_{12}s_{14}s_{15}=0$.
Hence we study the ideal generated by these two polynomials, plus the
polynomial $s_{12}s_{13}s_{45}-s_{14}s_{15} s_{23}$ which defines
$\delta_2$ in terms of the scalar product of the points. The corresponding
ideal decomposes into several ideals which are, up to a permutation of
the indices: $(s_{12}, s_{14})$, $(s_{12}, s_{22})$ and:
%
\[
J = (s_{13}s_{22} - s_{12}s_{23}, s_{14}s_{15} - s_{11}s_{45}, s_{12}s_{13} -
s_{11}s_{23}, s_{12}^2 - s_{11}s_{22}) \,.
\]
%
Then we give generic coordinates to the five
points and substitute them into the ideal $J$. It is possible to see that
$J$ and the ideal $\bigl(\sigma(P_1, P_2), \sigma(P_1, P_4)\bigr)$ are equal (up to
saturations w.r.t.\ $v_2$ and the condition that $P_1, P_2, P_4$ are not
aligned).
\end{proof}
%
\begin{rmk}
In the first case of the above proposition, the corresponding cubic
is studied in @@@configuration (5)@@, in the
second case, $\sigma(P_1, P_2) = 0$ and $P_2\in \iso$, so
we have that the line $P_1+P_2+P_3$ is tangent to $\iso$ in $P_2$
and this case was considered in~\Cref{proposition:three_aligned_ranks}
(the matrix $\Phi(P_1, P_2, P_3)$ has rank $5$, so
$\Phi(P_1, \dots, P_5)$ has rank $9$ (or smaller) regardless of
$P_4$ and $P_5$)
the third case gives either that all the
points of the two lines of the $V$-configurations are eigenpoints (in case
$P_2 \not\in \iso$ or $P_4 \not\in \iso$) or the matrix $\Phi(P_1, \dots, P_5)$
has rank $8$ (if $P_2, P_4 \in \iso$), (see, respectively,~\Cref{rank_8}
and~\Cref{positive_dim}),
the last case is the generic one and expresses $P_3$ in terms of the remaining
four points.
\end{rmk}
%
\begin{prop}
\label{prop:d1d2}
Let $P_1, \dots, P_5$ be a $V$-configuration. Then
\[
\rk \ \Phi(P_1, \dots, P_5) \leq 9
\quad \mbox{
if and only if} \quad
\delta_1(P_1, P_2, P_4) \cdot \delta_2(P_1, \dots, P_5) = 0.
\]
\end{prop}
\begin{proof}
We denote the coordinate of the points as follows:
%
\[
 P_1 = (A_1: B_1: C_1) \,, \quad
 P_2 = (A_2: B_2: C_2) \,, \quad
 P_4 = (A_4: B_4: C_4) \,,
\]
%
then $P_3 = u_1P_1+u_2P_2$ and $P_5 = v_1P_1+v_2P_4$ for some $u_1, u_2, v_1, v_2$.
We proceed as in
the proof of \Cref{proposition:three_aligned_ranks}: we find, via symbolic computation, that the determinant of
the cut matrix of conditions of $P_1, \dotsc, P_5$ is
%
\begin{gather}
\label{delta1delta2}
A_1A_2A_4(u_1A_1+u_2A_2)(v_1A_1+v_2A_4) \cdot u_1^2u_2^2v_1^2v_2^2 \cdot D^5 \cdot
\delta_1(P_1,P_2,P_4) \cdot \delta_2(P_1,\dots,P_5)
\end{gather}
%
where $D$ is the determinant of the matrix whose rows are $P_1, P_2, P_4$
and is non-zero, as well as are non-zero $u_1, u_2, v_1, v_2$ (since we
assume that $P_1, P_2, P_4$ are not aligned and the points are distinct).
As a consequence of \Cref{lemma:minors}, we have that when we compute all the order~$10$ minors of the matrix of conditions, we obtain all polynomials of the form
%
\begin{gather*}
X_1X_2X_4 X_3 X_5 \cdot u_1^2u_2^2v_1^2v_2^2 \cdot D^5 \cdot
\delta_1(P_1,P_2,P_4) \cdot \delta_2(P_1,\dots,P_5)
\end{gather*}
%
where $X_i$ varies among all components of~$P_i$ for $i \in \{1, \dotsc, 5\}$.
Since each of $P_1, \dotsc, P_5$ has at least one non-zero coordinate, we have that
$\rk \ \Phi(P_1, \dots, P_5) \leq 9$ if and only if
\[
\delta_1(P_1, P_2, P_4) \cdot \delta_2(P_1, \dots, P_5) = 0 \,. \qedhere
\]
\end{proof}

%% dimostrazione lemma seguente: per esempio si trova nella prima parte
%% del file contiCasoDegenere2.sage
%% for the proof of the avobe lemma, see the first part of the file
%% rank_8_twoTangIso.sage"
%%
\begin{lemma}
\label{lemma:special_case_rank_8}
Let $P_1, \dots, P_5$ be a $V$-configuration of points and assume that
\[
\scl{P_1}{P_2}=0, \quad \scl{P_2}{P_2}=0, \quad \scl{P_1}{P_4}=0,
\quad \scl{P_4}{P_4}=0.
\]
Then the matrix $\Phi(P_1, \dots, P_5)$ has rank $8$.
\end{lemma}
\begin{proof}
By \Cref{proposition:sigma_tangency},
the lines $P_1 \vee P_2$ and $P_1 \vee P_4$ are tangent to~$\iso$ in~$P_2$ and~$P_4$, respectively. The point~$P_1$ cannot be on~$\iso$, hence, using the
action of $\mathrm{SO}_3(\mathbb{C})$, we can assume $P_1 = (1: 0: 0)$.
Since every element of $\mathrm{SO}_3(\mathbb{C})$ fixes the whole~$\iso$ (and sends a tangent line to it into another tangent line to it), when we transform the point~$P_1$
into $(1: 0: 0)$, we transform the points~$P_2$ and~$P_4$ into, respectively,
the points $(0: \iii: 1)$ and $(0: -\iii: 1)$ (which are the common points to
$\iso$ and the tangent lines through~$P_1$).
Therefore, it is enough to study the
specific configuration of the points:
%
\begin{gather*}
P_1 = (1: 0: 0), \quad P_2=(0: \iii: 1), \quad P_3=(u_1, \iii u_2, u_2), \\
P_4 = (0: -\iii: 1), \quad P_5 = (v_1, -\iii v_2, v_2),
\end{gather*}
%
where $(u_1: u_2), (v_1: v_2) \in \p^1$.
In the $15\times 10$ matrix $\Phi(P_1, \dots, P_5)$ we can erase the
rows: $\phi_2(P_1)$ (which is a zero row), the row $\phi_1(P_2)$
(since $\phi_1(P_2)=\iii\phi_2(P_2)$) and, for similar reasons, the
rows: $\phi_1(P_3)$, $\phi_1(P_4)$ and $\phi_1(P_5)$.
Moreover, since the rank of the matrix $\Phi(P_1, P_2, P_3)$ is $5$,
we can also erase the row $\phi_3(P_3)$ and, for the same reason, the
row $\phi_3(P_5)$. The remaining matrix $M$ is a $8\times 10$ matrix.
It is not possible that all the order $8$ minors
of $M$ are zero: the ideal they generate, after saturations with respect to the distinct point condition, is the whole ring.
\end{proof}

\begin{theorem}
\label{theorem:rank_V}
Let $P_1, \dots, P_5$ be a $V$-configuration of
points. Then we have:
\begin{enumerate}
\item $8 \leq \rk \ \Phi(P_1, \dots, P_5) \leq 10$\,;
\item $\rk \ \Phi(P_1, \dots, P_5) \leq 9$ if and only if
$\delta_1(P_1, P_2, P_4) \cdot \delta_2(P_1, \dots, P_5) =0$\,;
\item $\rk \ \Phi(P_1, \dots, P_5) = 8$ if and only if, one of
the following two conditions is satisfied:
%
\begin{itemize}
\item $\delta_1(P_1, P_2, P_4) = 0$, \
$\overline{\delta}_1(P_1, P_2, P_3) = 0$,
\ $\overline{\delta}_1(P_1, P_4, P_5) = 0$\,;
  \item the line $P_1 \vee P_2$ is tangent to~$\iso$ in $P_2$ or $P_3$
and the line $P_1 \vee P_4$ is tangent to~$\iso$ in $P_4$ or $P_5$; moreover, in this case we have $\delta_1(P_1, P_2, P_4) \neq 0$.
\end{itemize}
%
In particular, in both cases $\delta_2(P_1, P_2, P_3, P_4, P_5) = 0$ holds.
\end{enumerate}
\end{theorem}
\begin{proof}
%%% si basa sui file:
%%% rank_8_2_1_ii_0.sage e
%%% rank_8_1.sage
If the rank is $\leq 7$, from
\Cref{prop:condition3+1} applied to $P_1, P_2, P_3, P_4$ and $P_1, P_4, P_5, P_2$,
the lines $P_1+P_2$ and $P_1 + P_4$ are tangent to~$\iso$ (the first in $P_2$ or $P_3$ and the second in $P_4$ or $P_5$).
Then we get a contradiction from \Cref{lemma:special_case_rank_8}.
This shows the first item.

The second item follows from \Cref{prop:d1d2}.

We are left to proving the third item. By \Cref{lemma:special_case_rank_8} and \Cref{prop:d1d2}, the two conditions imply that the rank is~$8$.

For the converse, by \Cref{invariance}, we distinguish two possibilities:
$P_1 = (1:\iii :0)$ and
$P_1 = (1: 0: 0)$.
Let $P_2 = (A_2: B_2: C_2)$, $P_4 = (A_4: B_4: C_4)$ and
$P_3 = u_1P_1+u_2P_2$, $P_5 = v_1P_1+v_2P_4$ be the other points.
If $P_1 = (1: \iii: 0)$, we transform the matrix $M = \Phi(P_1, \dots, P_5)$ by elementary row and column operations into $\mathbb{I}_2 \oplus M'$, where $\mathbb{I}_2$ is the order~$2$ identity matrix; thus, $\rk M \leq 9$ if and only if $\rk M' \leq 7$.
The computations require several other strategies, but at the end we are
able to see that in this case (i.e.\ the case in which $P_1$ is on~$\iso$), the matrix $M'$ cannot have rank less then $7$.
If $P_1 = (1: 0: 0)$, then the ideal of the order $9$ minors of
$\Phi(P_1, \dots, P_5)$ (after several manipulations and saturations)
gives that there a few possibilities which can be summarized by the
following conditions: either $\delta_1(P_1, P_2, P_4) = 0$,
\ $\overline{\delta}_1(P_1, P_2, P_3) = 0$,
\ $\overline{\delta}_1(P_1, P_4, P_5) = 0$, or the line $P_1 \vee P_2$ is tangent to~$\iso$ (in $P_2$ or $P_3$) and the line $P_1 \vee P_4$ is tangent to~$\iso$ (in $P_4$ or $P_5$). We see that in the second case it is necessarily $\delta_1(P_1,P_2,P_4)
\neq 0$; indeed, in the case $P_2, P_4 \in \iso$, then $\scl{P_2 +P_4}{P_2+P_4}= \scl{P_2}{P_2} + \scl{P_4}{P_4}+2\scl{P_2}{P_4} =2\scl{P_2}{P_4} \neq 0$, as 
the point $P_2 + P_4$ is different from both $P_2$ and $P_4$and the line $P_2 \vee P_4$ has no other intersection points with~$\iso$. It follows that 
$\delta_1 (P_1,P_2,P_4)=s_{11}s_{24} \neq 0$, as $P_1 \not\in \iso$. The cases $P_3 \in \iso$, respectively $P_5\in \iso$, are similar.

Therefore, if $\rk \ \Phi(P_1, \dots, P_5) = 8$, then one of the two conditions above must hold. 
\end{proof}

\section{\texorpdfstring{$V$}{V}-configurations of rank~$9$}
\label{rank_9}

In this section, we consider the following problem:
given five generic points in a $V$-configuration and imposing one
of the two conditions
$\delta_1(P_1, P_2, P_4) = 0$ or $\delta_2(P_1, \dots, P_5) = 0$, assuming the rank of $\Phi(P_1, \dots, P_5)$ is~$9$, what can we say about all the eigenpoints of the corresponding cubics?

Since the rank of $\Phi(P_1, \dots, P_5)$ is~$9$, there exists precisely one cubic curve $\cbc$ having $P_1, \dots, P_5$ among its eigenpoints. Another immediate consequence of the rank $9$ condition is that the
matrix $\Phi(P_1, \dots, P_5)$ has $9$ linearly independent rows, so
we can extract from it a $9 \times 10$-submatrix
$\mathcal{H}$ whose order $9$ minors are not all zero. If
$\mathcal{H}_i$ is the order $9$ minor of
$\mathcal{H}$ given by erasing the $i$-th column ($i=1, \dots, 10)$, then,
from Cramer's rule, a polynomial defining $\cbc$ is given by
\[
 f(X) = \sum_{i=1}^{10}(-1)^i\det(\mathcal{H}_i)\cdot \mathcal{B}_i
\]
(where $\mathcal{B}$ is as in \Cref{vector_basis}). In other words,
\[
f(X) = \det \left( \begin{array}{c} \mathcal{H}\\ \mathcal{B}
 \end{array} \right)
\]
From the matrix $\mathcal{H}$ it is moreover possible to construct
the three minors of \Cref{eq:def_matrix}, indeed we have:
\begin{prop}
\label{proposition:geiser1}
If $g_1, g_2, g_3$ are the three minors of the matrix
\[
\left(
\begin{array}{ccc}
x & y & z \\
\de_x f(X) & \de_y f(X) & \de_z f(X)
\end{array}
\right)
\]
they can be computed by:
\[
\det \left( \begin{array}{c} \mathcal{H}\\
\phi_1(X)
\end{array} \right),\quad
\det \left( \begin{array}{c} \mathcal{H}\\
\phi_2(X)
\end{array} \right), \quad
\det \left( \begin{array}{c} \mathcal{H}\\
\phi_3(X)
\end{array} \right)
\]
\end{prop}
\begin{proof} We have:
\begin{eqnarray*}
g_1 & = & x \cdot \de_y f(X)- y \cdot \de_x f(X) \\
& = &
x\cdot \de_y \det \left( \begin{array}{c} \mathcal{H}\\ \mathcal{B}
\end{array} \right) -y \cdot
\de_x \det \left( \begin{array}{c} \mathcal{H}\\ \mathcal{B}
\end{array} \right) \\
& = & \det \left( \begin{array}{c} \mathcal{H}\\ x \cdot \de_y \mathcal{B}-
y \cdot \de_x \mathcal{B}
\end{array} \right)\\
& = & \left( \begin{array}{c} \mathcal{H}\\
\phi_1(X)
\end{array} \right)
\end{eqnarray*}
and similarly for $g_2 = x \cdot \de_z f(X)- z \cdot \de_x f(X)$
and $g_3 = y \cdot \de_z f(X)- z \cdot \de_y f(X)$.
\end{proof}


\begin{es} Consider the following five points in a $V$-configuration:
\begin{gather*}
p_1 = (2: -1: 1) \,, \quad p_2 = (-1: 1: 3) \,, \quad p_4 = (3: 6: -1) \,, \\
p_3 = p_1+p_2 \, \quad p_5 = p_1+p_4 \,.
\end{gather*}
They satisfy the condition $\delta_2(p_1, \dots, p_5) = 0$.
We consider the $9\times 10$ matrix $\mathcal{H}$ whose rows
(suitable rescaled) are:
$\phi_i(p_j)$ for $i = 1, 2$ and $j = 1, 2, 3, 4$ and $\phi_1(p_5)$:
%
\[
\mathcal{H} =
\left(\begin{array}{rrrrrrrrrr}
12 & 4 & -7 & 6 & 4 & 3 & -4 & 1 & 2 & 0 \\
-12 & 4 & -1 & 0 & 4 & -3 & 2 & 7 & -4 & 6 \\
-3 & 1 & 1 & -3 & 6 & 0 & -6 & -9 & -9 & 0 \\
-9 & 6 & -3 & 0 & 17 & -8 & -1 & -21 & -6 & -27 \\
0 & 1 & 0 & 0 & 0 & 4 & 0 & 0 & 16 & 0 \\
-12 & 0 & 0 & 0 & -31 & 0 & 0 & -56 & 0 & 48 \\
54& 63& 36& -108& -12& -9& 12& 2& -1& 0\\
27 & 36 & 36 & 0 & 21 & 48 & 108 & -17 & -36 & 9 \\
336& 328& 119& -588& -56& -33& 56& 7& -4& 0
\end{array}
\right)
\]
%
The cubic with $p_1, \dots, p_5$ as eigenpoints is given by
%
\[
\det \left(
\begin{array}{c}
\mathcal{H}\\
\mathcal{B}
\end{array}
\right)
\]
%
whose value is (after a suitable rescaling):
%
\[
766x^3 - 1176x^2y + 417xy^2 - 128y^3 + 1488x^2z - 1722xyz
+ 597y^2z - 207xz^2 + 504yz^2 + 911z^3
\]
%
and the polynomials~$g_1$, $g_2$, and $g_3$ are:
\begin{eqnarray*}
g_1 & = & 392x^3 + 488x^2y - 656xy^2 + 139y^3 + 574x^2z + 594xyz - 574y^2z - 168xz^2 - 69yz^2\\
g_2 & = & 496x^3 - 574x^2y + 199xy^2 - 904x^2z + 1120xyz - 139y^2z - 81xz^2 + 574yz^2 + 69z^3\\
g_3 & = & 496x^2y - 574xy^2 + 199y^3 + 392x^2z - 416xyz + 464y^2z + 574xz^2 + 513yz^2 - 168z^3
\end{eqnarray*}
and can be computed either using their definition,
or~\Cref{proposition:geiser1}.
The common zeros of $g_1, g_2, g_3$ are the points $p_1, \dots, p_5$
and the points $p_6 = (21: 34: -8)$ and $p_7 = (19: 35: -9)$.
\end{es}


\begin{rmk}
When the matrix $\Phi(P_1, \dots, P_5)$ has rank $8$, as in \Cref{rank_8}, the family of
cubics with $P_1, \dots, P_5$ as eigenpoints is one dimensional, so the
above construction can be modified, by substituting one of the rows of
$\mathcal{H}$ with a random row of elements of the field~$K$.
\end{rmk}

\begin{rmk}
\label{rmk:construction_five_d1}
Recalling \Cref{rmk:characteristics_d1_d2}, it is quite easy to construct five points $P_1, \dots, P_5$ that are in a $V$-configuration
and such that $\delta_1(P_1, P_2, P_4)= 0$: the points~$P_1$
and $P_2$ can be taken in an arbitrary way, $P_4$ has to be chosen in such
a way that it satisfies \Cref{rmk_delta_case1}
and $P_3$ and $P_5$ have to be chosen on the lines $P_1 \vee P_2$ and $P_1 \vee P_4$,
respectively. In particular, the corresponding locus of cubic curves
has dimension $7$.
The construction of a random example
of five points as above, gives a smooth cubic curve whose $7$ eigenpoints
do not have other collinearities (in addition to those of a
$V$-configuration).
\end{rmk}

\begin{rmk}
\label{rmk:construction_five_d2}
If we want a $V$-configuration that satisfies the condition
$\delta_2(P_1, \dots, P_5) = 0$, we choose $P_1$, $P_2$, $P_4$ arbitrarily; 
we choose $P_3$ on the line $P_1 \vee P_2$ and $P_5$ on the line $P_1 \vee P_4$ so that $P_3$ satisfies \Cref{prop:definitionP3}, \Cref{defP3_4}.

Also in this case the locus of cubics is of dimension $7$.
\end{rmk}

Cubics whose eigenpoints satisfy the condition $\delta_2(P_1, \dotsc, P_5) = 0$ and $\rk \Phi(P_1, \dotsc, P_5) = 9$ exhibit further peculiarities. It turns out that the $7$ eigenpoints of such cubics satisfy the further
condition that also $P_6$ and $P_7$ are aligned with $P_1$. To explain
why, we proceed as follows.

Consider again a matrix $\mathcal{H}$ of rank $9$ extracted from
the matrix $\Phi(P_1, P_2, P_3, P_4, P_5)$, where now $P_3$ is determined by Equation~\eqref{} \textbf{(SERVE QUESTA CONDIZIONE O E' SUFFICIENTE AVERE RANGO 9 E $\delta_2 = 0$?)}
and the three polynomials $g_1, g_2, g_3$ are constructed
in \Cref{proposition:geiser1}. The condition
$\rk \,(\mathcal{H}) = 9$ ensures that
$g_1, g_2$ and $g_3$ cannot be zero and are polynomials of degree
$3$ in $x, y, z$.
The computation shows that three polynomials have a common factor,
hence are of the form $\Omega \cdot g_1$, $\Omega \cdot g_2$, $\Omega \cdot g_3$, where
$\Omega$ is a polynomial in the variables
$A_1, B_1, C_1, A_2, B_2, C_2, A_4, B_4, C_4, v_1, v_2$ but
does not contain the variables $x, y, z$.
Hence $g_1, g_2, g_3$ are again
degree three polynomials in $x, y, z$, with coefficients in the variables
$A_1, \dots, C_4, v_1, v_2$. Their common zeros are the eigenpoints of the
cubic $\cbc$ since by construction they are proportional to the three minors of the matrix
%
\[
 \left(
 \begin{array}{ccc}
  x & y & z \\
  \partial_x f & \partial_x f & \partial_x f
 \end{array}
 \right)
\]
%

\begin{prop}
\label{proposition:G_split}
 Assume that $P_1, \dots, P_5 $ satisfy $\delta_2(P_1, \dotsc, P_5) = 0$ and $\rk \, \Phi(P_1, \dotsc, P_5) = 9$. Then the cubic curve $A_1 g_3 - B_1 g_2 + C_1 g_1=0$ contains the lines $P_1 \vee P_2$ and $P_1 \vee P_4$.

% The polynomial $C_1G_1-B_1G_2+A_1G_3$ splits into three factors $r_1$, $r_2$, $r_3$, each linear in $x, y, z$.
\end{prop}
\begin{definition}
%   Let $\Eig{f}$ be zero-dimensional and reduced, and set
%%
%\[
%  M_1 :=y \partial _z f - z \partial _y f, \quad
%  M_2 := z \partial _x f - x \partial _z f, \quad
%  M_3 :=x \partial _y f - y \partial _x f.
%\]
%
The Geiser map associated with a zero dimensional eigenscheme $\Eig{f}$ is the rational map defined by
%
\[
\gamma_{\Eig{f}} \colon \p ^2 \dasharrow \p^2, \quad
\gamma_{\Eig{F}} (P) = \bigl( g_1(P):g_2(P):g_3(P) \bigr) \,,
\]
%
where $\p^2 = \p(I_{\Eig{f}}(3)^{\vee})$, namely the projectivization of the dual of the space of degree~$3$ forms in the homogeneous ideal of~$\Eig{f}$.
\end{definition}

Geiser maps are a classical topic and several of their properties are understood. As an example, $\gamma_{\Eig{f}}$ is generically finite of degree~$2$, see for instance \cite[Section~8.7.2]{Dolgachev}.
%Moreover, the ramification locus of~$\widetilde{\gamma _{E(F)}$ is given by the \emph{Jacobian locus}~$\Sigma$ defined by the determinant of the Jacobian $\mathrm{Jac}(I_Z)$, that is the locus of singular points of the net (see
%    \cite[Book II, Chapter~IX, Theorem~25]{Cool}). When $Z$ is general, the Jacobian locus~$\Sigma$ is a curve of degree~$6$ which is singular at $Z$ as illustrated in \cite[Chapter~IX, Theorem~27]{Cool}. We define $B(Z)$ to be the branch locus of $\widetilde{\gamma_Z}$, that is the direct image of~$\Sigma$. For modern references, see for instance \cite[Section~8.7.2]{Dolgachev} and \cite[Section~7]{OS1}, where it is proven that a general Geiser map is branched along a smooth L\"uroth quartic.

\begin{lemma}
The Geiser map~$\gamma_{\Eig{f}}$ is surjective and its fiber over a point $Q = (a:b:c)$ is given by
%
\begin{equation}
\label{eq:fibers}
    \left\{
    \begin{array}{l}
    a x + by + cz = 0, \\[2pt]
    a \, \partial_x f + b \, \partial_y f + c \, \partial_z f = 0,\\
 \end{array}\right.
\end{equation}
%
that is, the intersection between the polar line $L_Q$ relative to the isotropic conic and $\mathrm{Pol}_Q f$, the
first polar curve of~$F$ with respect to~$Q$.

In particular, the only possible curves contracted by the Geiser map~$\gamma_{\Eig{f}}$ are lines.
\end{lemma}

\begin{proof}
We observe that for any point $P=(A:B:C) \in \p^2 \setminus \Eig{f}$, the homogeneous coordinates
of the image $\gamma_{\Eig{f}}(P) = \bigl( g_1(P): g_2(P): g_3(P) \bigr)$ are the ones of the unique point in the intersection of the two lines
%
\[
  Ax + By+ Cz = 0, \qquad
  \partial_x f(P) \, x + \partial_y f(P) \, y+ \partial_z f(P) \, z = 0.
\]
%
So for any $Q = (a:b:c) \in \p^2$, the fiber $\gamma_{\Eig{f}}^{-1}(Q)$ consists of the points $P \in \p^2$ such that
%
\begin{equation}
\label{eq:polars}
Aa + Bb+ Cc = 0, \quad
\partial_x f(P) \, a + \partial_y f(P) \, b + \partial_z f(P) \, c = 0,
\end{equation}
%
which proves that $\gamma_{\Eig{f}}$ is surjective and that \eqref{eq:fibers} holds.
%To prove that there are no other contracted curves than lines,
%set $Z=E(F)$ and consider the blow-up of the plane $\p^2$ along $Z$. We get a generically finite morphism
%%
%\[
%  \widetilde \gamma_Z \colon \Bl_Z \p^2 \to \p^2.
%\]
%%
%Observe that the fibers of~$\widetilde{\gamma_Z}$ are generically contained in the divisor~$W$ with bihomogenous equation $x_0 y_0 + x_1 y_1 + x_2 y_2=0$.
%Since both~$\Bl_Z \p^2$ and~$W$ are irreducible, and since $\widetilde{\gamma_Z}$ is the restriction of the second projection $p_2 \colon \p^2 \times \p^2 \rightarrow \p^2$, it follows that $S\subseteq W$;
%in particular, every fiber of~$\widetilde{\gamma_Z}$ is contained in a line and by construction the same holds for every fiber of~$\gamma_Z$.
%As a consequence, the map $\gamma_Z$ contracts only lines.
\end{proof}

\begin{prop}
\label{prop:allineati_contrae}
Suppose that $\Eig{f}$ is a zero dimensional eigenscheme of a cubic ternary form.
If $\Eig{f}$ contains a triple of points on a line~$L \subset \p^2$, then the Geiser map~$\gamma_{\Eig{f}}$ contracts~$L$.
\end{prop}

\begin{proof}
%Consider the isomorphism  $\varphi \colon \p (I_{\Eig{F}}(3))^{\vee} \stackrel{\cong}{\longrightarrow} \p (I_{\Eig{F}}(3))$ associating with each point~$(a:b:c)$ the line $ax + by + cz = 0$.
%For any $P \in \p^2 \setminus \Eig{F}$, the point~$\varphi(\gamma_{\Eig{F}} (P))$ corresponds to the line~$L_P$ given by
%
For any $P \in \p^2 \setminus \Eig{f}$, the point~$\gamma_{\Eig{f}} (P)\in \p \bigl(I_{\Eig{f}}(3)^{\vee}\bigr)$ corresponds to the line~$L_P$ given by
\[
  g_1 (P) \, x + g_2(P) \, y + g_3(P) \, z = 0,
\]
%
which in turn determines the following
pencil of cubics $L_P$ in the linear system $\p \bigl(I_{\Eig{f}}(3)\bigr)$:
%
\[
  s \cdot \bigl( g_1(P) \, g_2 - g_2(P) \, g_1 \bigr) + t \cdot \bigl( g_1(P) \, g_3 - g_3(P) \, g_1 \bigr),
\]
%
where $(s: t) \in \p^1$, assuming that $g_1(P) \neq 0$ (the other cases are similar).
It is clear that the base locus of the pencil~$L_P$ contains the eight points of~$\Eig{f} \cup \{P\}$.
So $\gamma_{\Eig{f}} \colon \p^2 \dasharrow \p \bigl( I_{\Eig{f}}(3)^\vee \bigr)$
associates with each point $P \in \p^2 \setminus \Eig{f}$ the pencil of cubics through $\Eig{f} \cup P$.

If $\Eig{f}$ contains a triple of points on a line~$L$, for any $P \in L \setminus \Eig{f}$ such a pencil of cubics has $L$ as a fixed component and the other component varies in a pencil of conics through the remaining $4$ points, so $\gamma_{\Eig{f}}$ is constant on~$L$.
\end{proof}

%We know that $G_1, G_2, G_3$ admit the syzygy $z \, G_1 - y \, G_2 + x \, G_3 = 0$.
%Then, due to the exactness of the Koszul complex of the regular sequence $(x,y,z)$ (see, for instance, \cite[Theorem 7.3.13]{Dolgachev}),
%we know that $G_1, G_2, G_3$ are the $2\times2$ minors of a matrix of the form
%%
%\[
% \left(
% \begin{array}{ccc}
%  x & y & z \\
%  H_1 & H_2 & H_3
% \end{array}
% \right)
%\]
%%
%where $H_1, H_2, H_3$ are quadratic ternary forms.

%Consider now the map
%%
%\[
%  \bar{\gamma} \colon \p^2 \setminus \Eig{f} \rightarrow \p^2, \quad P \mapsto (H_1(P): H_2(P): H_3(P)).
%\]
%%
%This map is the composition of~$\gamma_{\Eig{f}}$ with a linear isomorphism.
%In particular, the map $\bar{\gamma}$ contracts both the lines $P_1 +P_2$ and $P_1 +P_4$.

We now turn back to our situation, when we have five points $P_1, \dots, P_5$ forming a $V$-configuration and satisfying $\delta_2(P_1, \dots, P_5)=0$ and $\rk \, \Phi(P_1, \dotsc, P_5) = 9$. This means that there exists a unique (up to moltiplication by scalars) cubic ternary form~$f$ such that $P_1, \dotsc, P_5 \in \Eig{f}$.

\begin{proof}[Proof of \Cref{proposition:G_split}]
The equation of the line $P_1 \vee P_2$ can be expressed in the following way
%
\begin{equation}
\label{eq:lineP1P2}
  \left\langle P_1 \times P_2, (x,y,z) \right\rangle = 0;
\end{equation}
%
since it is contracted by $\gamma_{\Eig{f}}$, and by the description of the fibers given in \Cref{eq:fibers}, we have that such a line is contained in the conic of equation
%
\[
  \left\langle P_1 \times P_2, (\de_x f, \de_y f, \de_z f) \right\rangle = 0.
\]
%
As a consequence, for any point $\overline{P} = (\bar x: \bar y: \bar z)$ of the line from \Cref{eq:lineP1P2}, the following equations are satisfied:
%
\[
\left\{
\begin{array}{l}
  \left\langle P_1 \times P_2, P_1 \right\rangle = 0 \,,\\[2pt]
  \bigl\langle P_1 \times P_2, \overline{P} \bigr\rangle = 0 \,,\\[2pt]
  \bigl\langle P_1 \times P_2, \nabla f (\overline{P}) \bigr\rangle = 0 \,.
\end{array}
\right.
\]
%
It follows that the linear system in the variables $X,Y,Z$
%
\[
\left\{
\begin{array}{l}
  \bigl\langle P_1, (X,Y,Z) \bigr\rangle = 0 \,,\\[2pt]
  \bigl\langle \overline{P}, (X,Y,Z) \bigr\rangle = 0 \,,\\[2pt]
  \bigl\langle \nabla f (\overline{P}),
  (X,Y,Z) \bigr\rangle = 0 \,.
\end{array}
\right.
\]
%
admits as a non-zero solution $P_1 \times P_2$,
hence the determinant of the order~$3$ coefficient matrix
%
\[
 \left(
 \begin{array}{c}
  P_1 \\
  \overline{P} \\
  \nabla f (\overline{P})
 \end{array}
 \right)
\]
%
is zero. This is equivalent to saying that $A_1 g_3 - B_1 g_2 +C_1 g_1$ vanishes at all the points~$\overline{P}$ of the line $P_1 \vee P_2$.

The argument concerning the line $P_1 \vee P_4$ can be repeated by replacing~$P_2$ with~$P_4$.
\end{proof}
%
\begin{prop}
\label{proposition:third_alignment}
If we have a $V$-configuration of five points $P_1, \dots, P_5$
such that the rank of the matrix $\Phi(P_1, \dots, P_5)$ is $9$ and
such that $\delta_2(P_1, \dots, P_5) = 0$,
then the unique ternary cubic determined by the condition that $P_1, \dots, P_5$
are eigenpoints has also the two other eigenpoints $P_6$ and $P_7$
aligned with $P_1$.
\end{prop}
\begin{proof}
Let $f$ be the unique form such that $P_1, \dotsc, P_5 \in \Eig{f}$.
According to \Cref{proposition:G_split}, the polynomial
$C_1g_1-B_1g_2+A_1g_3$ splits into three factors $r_1$, $r_2$, $r_3$,
linear in $x, y, z$, which
correspond to the line $P_1 \vee P_2$, the line $P_1 \vee P_4$, and the line $P_6 \vee P_7$.
The factorization of $F$ gives that the factor $r_3$ is:
\[
r_3 = 2yv_1A_1B_1^4A_2^2B_2A_4^3-2xv_1B_1^5A_2^2B_2A_4^3+\cdots
-2 yv_1A_1B_1^4A_2^2B_2A_4^3
\]
and is composed by almost $2000$ monomials. When we
evaluate $r_3$ on the coordinates of $P_1$, we get zero.
\end{proof}

\section{\texorpdfstring{$V$}{V}-configurations of rank~$8$}
\label{rank_8}
file \verb+contiCasoDegenere2.sage+ and file
\verb+conf_sigma12_sigma14.sage+\\
\verb+molti altri file via +\\
\verb+via indicati nei commenti di latex+

In this section, we study the possible configurations of
eigenpoints for cubics with a $V$-configuration
satisfying the condition $\rk \Phi(P_1, \dots, P_5) = 8$. According
to~\Cref{theorem:rank_V}, we have to distinguish the two cases
%
\begin{gather}
\delta_1(P_1, P_2, P_4)=\overline{\delta}_1(P_1, P_2, P_3) =
\overline{\delta}_1(P_1, P_4, P_5) = 0,
\label{rk8_1}\\
\sigma(P_1, P_2) = 0, \ \sigma(P_1, P_4) = 0 \ \ \mbox{and} \ \ s_{22} = 0,
\ s_{44} = 0.
\label{rk8_2}
\end{gather}
%
(the latter condition means that \emph{the two lines of
the $V$-configuration are tangent to $\iso$ in $P_2$ and $P_4$}).
%
To start, we characterize when, in a $V$-configuration
$P_1, \dotsc, P_5$, the point $P_1$ is singular.

\begin{prop}
\label{proposition:P1_sing}
If $P_1, \dots, P_5$ is a $V$-configuration and if
$\mathcal{C}$ is a cubic which
has $P_1, \dots, P_5$ among its eigenpoints, then $P_1$ is
singular for~$\mathcal{C}$ if
and only if $P_1$ is not on $\iso$ and one of the following conditions
is satisfied:
\begin{enumerate}
\item $\delta_1(P_1, P_2, P_4) = 0$ and $\overline{\delta}_1(P_1, P_2, P_3) = 0$;
\item $\delta_1(P_1, P_2, P_4) = 0$ and $\overline{\delta}_1(P_1, P_4, P_5) = 0$;
\item $\overline{\delta}_1(P_1, P_2, P_3) = 0$ and
$\overline{\delta}_1(P_1, P_4, P_5) = 0$.
\end{enumerate}
\end{prop}
\begin{proof}
\verb+file: singularCubic_1ii0.sage+ e \verb+singularCubic_100.sage+\\
As usual, we can split the problem into two cases: $P_1 = (1: \iii: 0)$
and $P_1 = (1: 0: 0)$. In both cases we construct the matrix
$\Phi(P_1, \dots, P_5)$ (for $P_2$, $P_4$ generic, $P_3$ aligned with
$P_1$ and $P_2$ and $P_5$ aligned with $P_1$ and $P_4$). To this matrix we add
a row which correspond to the condition that $P_1$ is singular, obtaining
in this way a matrix $M$. After
suitable manipulations of $M$, we see that $\Lambda(M)$ is empty
in the case $P_1=(1: \iii: 0)$, meanwhile if $P_1$ is the point $(1:0:0)$
and one of the three above conditions is satisfied, then
$\dim(\Lambda(M)) \geq 0$.
\end{proof}

Let us start to analyze the two conditions from \Cref{rk8_1} and \Cref{rk8_2}.
The three equalities of~\Cref{rk8_1} give that,
using \Cref{rmk_delta_case2}, either $P_1$ is on~$\iso$ and
$P_1+ \vee _2$ is the tangent line to $\iso$ in $P_1$ (but this is not compatible
with the equation $\overline{\delta}_1(P_1, P_4, P_5)=0$), or,
from \Cref{rmk_delta_case1}, we can choose the points as follows:
%
\begin{itemize}
\item $P_1$ and $P_2$ in an arbitrary way;
\item $P_4$ in the $\mathbb{P}^1$
space of points orthogonal to $s_{11}P_2-s_{12}P_1$;
\item $P_3 = (s_{12}^2+s_{11}s_{22})P_1-2s_{11}s_{12}P_2$;
\item $P_5 = (s_{14}^2+s_{11}s_{44})P_1-2s_{11}s_{14}P_4$.
\end{itemize}
%

For such a choice
of points, the matrix $M = \Phi(P_1, \dots, P_5)$ has rank $8$
%% verifica nel file matrix_of_order_8.sage
and therefore $\dim \Lambda(M) = 1$ and the dimension of the variety
of the corresponding cubics is $6$.
Moreover, from~\Cref{proposition:P1_sing}, we have that all these cubics
are singular in $P_1$.
If we take a random point of this variety, we see that the corresponding
cubic has $7$ distinct eigenpoints (is as expected, singular in $P_1$) and the
$7$ points do not satisfy other collinearities in addition to those of the
$V$-configuration.

\begin{rmk}
\label{rmk:particular_cases}
If $P_1, \dots, P_5$ satisfy the conditions~(\ref{rk8_1}), as seen,
in general we have only $(P_1, P_2, P_3)$ and $(P_1, P_4, P_5)$ aligned, but
we do not have $(P_1, P_6, P_7)$ aligned. Since conditions~(\ref{rk8_1})
implies that $\delta_2(P_1, \dots, P_5) = 0$, we see
that the hypothesis ``rank $9$''
in~\Cref{proposition:third_alignment} is necessary.
\end{rmk}

Concerning possible sub-cases with further collinearities of the points,
the following results hold:
%%%%
%% tutti i conti si trovano nei file:
%% three_deltas_I.sage
%% three_deltas_II.sage
%% three_deltas_III.sage
%% three_deltas_IV.sage
%% c'e' anche il file three_d_alt.sage che puo' essere di aiuto
%% ma non e' essenziale
%%
%% c'e' anche il file three_d_alt.sage che calcola tutti gli autopunti
%% in configurazione (8) nel caso dei 3 delta = 0.
%%
\begin{prop}
\label{three_d_three_alignments}
If the five points $P_1, \dots, P_5$ satisfy \Cref{rk8_1},
then, in $\Lambda \bigl( \Phi(P_1, \dotsc, P_5)\bigr)$ there is
a cubic curve with $7$ eigenpoints with the following three alignments:
$(P_1, P_2, P_3)$, $(P_1, P_4, P_5)$ and $(P_1, P_6, P_7)$. \\
No choices of $P_1, \dots, P_5$ allow one to obtain further alignments of the
$7$ eigenpoints.
\end{prop}
Moreover, we have:
\begin{prop}
\label{prop:d2_6allin}
If the five points $P_1, \dots, P_5$ satisfy~(\ref{rk8_1})
and if we impose the condition that there is an eigenpoint, say $P_6$,
which is aligned with $P_2$ and $P_4$ then the eigenpoints satisfy all these
alignments:
\[
(P_1, P_2, P_3), (P_1, P_4, P_5), (P_2, P_4, P_6), (P_2, P_5, P_7),
(P_3, P_4, P_7), (P_3, P_5, P_6)
\]
Hence the points $P_6$ and $P_7$ are determined by $P_1, \dots, P_5$
since
$P_6 = (P_2 \vee P_4) \cap (P_3 \vee P_5)$
and $P_7 = (P_3 \vee P_4) \cap (P_2 \vee P_5)$.
(Of course, a similar result holds if we take $P_3$ in place of $P_2$ or $P_5$
in place of $P_4$).
\end{prop}
The two propositions above exhaust all the possible configurations
of collinearities of the eigenpoints in case of condition~(\ref{rk8_1}).

The proof we give to the two propositions above is computational.
Using the action of $\mathrm{SO}_3(\mathbb{C})$ (see~\Cref{two_orbits}),
we can assume $P_1= (1: \iii: 0)$ or $P_1= (1: 0: 0)$. As observed
in the proof of~\Cref{theorem:rank_V}, if $P_1 = (1: \iii: 0)$,\\
(@@ \verb+Vedi file rank_8_2_1_ii_0.sage+), \\
the matrix
$\Phi(P_1, \dots, P_5)$ cannot have rank smaller then $9$, so the only
case to consider is given by $P_1 = (1: 0: 0)$. Hence
$P_2 = (A_2: B_2: C_2)$, $P_4 = (A_4: B_4: C_4)$ and
$P_3 = u_1P_1+u_2P_2$, $P_5=v_1P_1+v_2P_4$.
The condition $\delta_1(P_1, P_2, P_4)=0$ reduces to $B_2B_4+C_2C_4$
and the explicit coordinates of the involved points are easily constructed.

The problem of considering all the possible further
collinearities of the points can be converted into the study of suitable
ideals in the coordinates of the points and can be attacked with
computational tools. More precisely,
consider the matrix $\Phi(P_1, \dots, P_5)$ and select $8$ linearly independent
rows (since $P_1$ is numeric, it can be seen that for instance the rows
of position $0, 1, 3, 5, 6, 8, 10, 12$ are a good choice and avoid to consider
many exceptions) and let $M_1$ be a matrix obtained in this way.
We fix two numeric $10$-dimensional vectors $V_a$ and $V_b$ and we construct
$M_a$ (resp.\ $M_b$) obtained by horizontally stacking vector
$V_a$ (resp.\ $V_b$)
to $M_1$. The rank of $M_a$ and $M_b$ is $9$ (unless an unlucky choice
of the vectors, in which case they can be changed). The matrices $M_a$ and
$M_b$ play the same role of the matrix $\mathcal{H}$
of~\Cref{rank_9}, hence we can do the same construction explained
there, in order to obtain the line $r_a$ passing through the two
eigenpoints $P_6$ and $P_7$ of the cubic obtained from $M_a$ and the
line $r_b$ passing through the analogous eigenpoints of the cubic
obtained from $M_b$.
Hence the line $r = w_ar_a+w_br_b$
(where $w_a, w_b$ are parameters) is the line passing through the eigenpoints
$P_6$ and $P_7$
of the generic cubic of the one dimensional family given by the line
of $\mathbb{P}^9$ determined by the cubic obtained from $M_a$ and
the cubic obtained from $M_b$. The condition $P_1 \in r$
determines the values of $w_a$ and $w_b$, hence we get the cubic
with the three collinearities $(P_1, P_2, P_3)$,
$(P_1, P_4, P_5)$ and $(P_1, P_6, P_7)$.
%
%% La dimostrazione della proposizione prop:rk8_2B si trova nel file
%% rank_8_twoTangIso.sage
%% Il file confV_tg_iso.sage fa il caso generale (cioe' quando P1
%% e' il punto (A1, B1, C1) e non (1, 0, 0)) e serve per provare
%% le formule realtive a P6 e P7 nella configurazione (5).
%% Il vile confV2_tg_iso.sage contiene i conti per la configurazione (8)
%% e prova la formula P6 = s15*P3+s13*P5

Now we consider the condition~(\ref{rk8_2}).

%
A generic cubic which satisfies the condition~(\ref{rk8_2}) is
a cubic of the one-dimensional linear system
$\Lambda(\Phi(P_1, \dotsc, P_5))$, where:
\begin{itemize}
\item $P_1$ is any point of the plane (not on $\iso$);
\item $P_2$ and $P_4$ are the two tangency points to $\iso$ given by the lines
from $P_1$;
\item $P_3$ is any point on $P_1+P_2$ (different from $P_1$ and $P_2$)
and $P_5$ is any point on $P_1+P_4$ (different from $P_1$ and~$P_4$).
\end{itemize}

The variety of all the cubics with a $V$-configuration
that
satisfies condition~(\ref{rk8_2}) is therefore five dimensional.
The five points satisfy the condition $\delta_2(P_1, P_2, P_3, P_4, P_5) = 0$.
%% 2+1+1+1, l'ultimo 1 perche' rango matrice = 8


The reciprocal position of the eigenpoints of the cubics of this family
is described by the following:
%
\begin{prop}
\label{prop:rk8_2B}
The generic cubic of the family of cubics which satisy condition~(\ref{rk8_2})
has seven eigenpoints with the alignments:
\[
(P_1, P_2, P_3), \ (P_1, P_4, P_5), \ (P_1, P_6, P_7)
\]
Among these points we have the relation
$\scl{P_1 \times P_6}{P_3\times P_5}=0$
(i.e.\ $P_1+P_6$ and $P_3+P_5$ are orthogonal).
%% ortogonalita': provata nel file rank_8
In the family there is a sub-family of cubics
whose eigenpoints have the following
alignments:
\[
(P_1, P_2, P_3),\ (P_1, P_4, P_5),\ (P_1, P_6, P_7),\ (P_2, P_4, P_6).
\]
In this case the points $P_6$ and $P_7$ are given by the formulas:
\[P_6 = s_{11}s_{15}P_3-2s_{13}s_{15}P_1+s_{11}s_{13}P_5, \ \mbox{and} \
P_7 = s_{11}s_{15}P_3+s_{13}s_{15}P_1+s_{11}s_{13}P_5;
\]
and a sub-family whose eigenpoints have the following
alignments:
\[
(P_1, P_2, P_3),\ (P_1, P_4, P_5), \ (P_1, P_6, P_7),\ (P_2, P_5, P_6),\
(P_3, P_4, P_6),\ (P_3, P_5, P_7)
\]
In this case the point $P_6$ (given by the formula
$P_6 = s_{15}P_3+s_{13}P_5$) is obtained as
$(P_2+P_5) \cap (P_3+P_4)$ and consequently
$P_7 = (P_1+P_6) \cap (P_3+P_5)$.\\
No other collinearities among the eigenpoints are possible.
\end{prop}

In order to get the proof of~\Cref{prop:rk8_2B}
we consider again the action of $\SO_3(\C)$ and here we can assume:
\[
P_1 = (1: 0: 0), \quad P_2 = (0: \iii: 1), \quad P_4 = (0: -\iii: 1)
\]
(and $P_3 = u_1P_1+u_2P_2$, $P_5 = v_1P_1+v_2P_4$). In this situation it
is easy to see that $\Lambda(\Phi(P_1, \dots, P_5))$
is the following family of cubic curves:
\[
\mathcal{C}(l_1, l_2) = l_1\mathcal{C}_1+l_2\mathcal{C}_2
\]
where
\begin{align*}
  \mathcal{C}_1 & = x \cdot \left(2x^{2} + 3 y^{2} + 3 z^{2}\right)\\
  \mathcal{C}_2 & = (y + \iii z) \cdot (y - \iii z)
\cdot \left(2 \iii x u_{2} v_{2} + y (u_{2} v_{1}- u_{1} v_{2})
- \iii z (u_{2} v_{1} + u_{1} v_{2})\right)
\end{align*}
%% C_1 and C_2 are in the file rank_8_twoTangIso.sage, see ***C1_C2***
%%
The explicit equation of the cubics $\mathcal{C}(l_1, l_2)$ allows to
construct the ideal of the seven eigenpoints and (after
saturations w.r.t.\ the ideals of the points $P_1, \dotsc, P_5$),
an ideal generated by a line $r$ and a conic $c$, whose zeros are
the points $P_6$ and $P_7$. Since the line $r$ contains the point $P_1$,
we have that $P_1, P_6, P_7$ are collinear and $r$ is orthogonal to
$P_3\vee P_5$. In order to see if there are further collinearities among the
eigenpoints, we have to distinguish three cases (up to permutation
of the indices): $P_2, P_4, P_6$ are collinear or $P_2, P_5, P_6$ are
collinear, or $P_3, P_5, P_6$ are collinear. In the first case, $P_6$
is the point $r \cap (P_2\vee P_4)$. If it is an eigenpoint, it must be
on $c$, hence $l_1$ and $l_2$ have a specific value which give a sub-family
of $\mathcal{C}(l_1, l_2)$ and we can compute the explicit coordinates of
all the seven eigenpoints of the cubics of this family. The other two
cases can be studied in a similar way.
%% file rank_8_twoTangIso.sage, file confV_tg_iso.sage, file
%% confV2_tg_iso.sage.

Then, from these equations, all the computations which allow to prove
the two above propositions are easy.
%
\begin{rmk}
\label{rmk:delta1_and_delta2}
It is easy to verify that in
a $V$-configuration which satisfies~(\ref{C8:cnd2}), we have that
$\delta_1(P_1, P_2, P_4)\neq 0$ and
$\delta_2(P_1, P_2, P_3, P_4, P_5) = 0$,
meanwhile in~(\ref{C8:cnd1}) both $\delta_1(P_1, P_2, P_4)$ and
$\delta_2(P_1, P_2, P_3, P_4, P_5)$ are zero.
\end{rmk}



\section{The locus of cubic ternary forms with an aligned triple of eigenpoints}
\label{locus_one_alignment}


The results of the previous sections allow one to determine the dimension and the degree of the locus of cubics having
at least one aligned triple of eigenpoints. We first recall a result by Abo concerning the locus of cubics with a non reduced or positive-dimensional eigenscheme.

\begin{definition}
    An order $d$ tensor $T \in (\C^n)^{\otimes d}$ is called \emph{regular} if $\Eig{T}\subset \p^{n-1}$ is reduced of dimension $0$.

We set
$$
\Delta_{n,d} := \{[T]\in \p \bigl( (\C^n)^{\otimes d} \bigr) \ | \ \Eig{T} \textrm{\ is \ non \ regular} \} \subset \p \bigl( (\C^n)^{\otimes d} \bigr) \,.
$$
The locus~$\Delta_{n,d}$ is called the \emph{eigendiscriminant} (see \cite[Definition 5.5]{Abo}).
\end{definition}

\begin{theorem}
\label{theorem:eigendiscriminant}
   The eigendiscriminant $\Delta_{n,d}\subseteq \p \bigl( (\C^n)^{\otimes d} \bigr)$ is an irreducible hypersurface of
   degree $n(n-1)(d-1)^{n-1}$.
\end{theorem}

\begin{proof}
    See \cite[Corollary 5.8]{Abo}.
\end{proof}

When restricting to symmetric tensors, the statement of \Cref{theorem:eigendiscriminant} is still valid, as they form a linear subspace of~$\p \bigl( (\C^n)^{\otimes d} \bigr)$ and the generic symmetric tensor has a regular eigenscheme.

In the particular case $n=d=3$, we have that
the eigendiscriminant hypersurface has degree
$$
\deg \Delta _{3,3}=24.
$$
\begin{definition}
Let $\sU \subset \p^9$ be the following locus:
$$
\sU:= \{[f]\in \p^9 \setminus \Delta_{3,3} \ | \ \Eig{f} \ \textrm {contains \ an \ aligned \ triple}\},
$$
and define $\mathcal{L} \subseteq \p^9$ as the closure of $\sU$:
 $$
 \sL := \overline \sU \subset \p^9.
 $$
\end{definition}

\begin{theorem}
\label{theorem:irreducible}
The variety~$\mathcal{L}$ is an irreducible hypersurface.
\end{theorem}

\begin{proof}
We first prove that $\dim \mathcal{L}=8$. As there exist
polynomials $f$ such that $\Eig{f}$ has no aligned triple, we have $\dim \mathcal{L} \le 8$.


Consider the symmetric product $(\p^2) ^{(3)}$ and set $\mathcal {AL} \subset (\p^2) ^{(3)}$ to be the locus of unordered triples of distinct aligned points. Observe that $\mathcal {AL}$ is irreducible of dimension $5$ as its closure is the hypersurface given by the vanishing of the determinant of the order $3$ matrix of the coordinates of three general points. Alternatively, the closure of the locus $\mathcal {AL}$ can be seen as the symmetric quotient of the projective line bundle $\sF \subset \p^2 \times \p^2 \times \p^2$ parametrized by the triples $(P_1, P_2, u_1 P_1 +u_2P_2)$, where $(u_1:u_2) \in \p^1$.

If we set
$$
\sU':= \{[f]\in \p^9 \setminus \Delta_{3,3} \ | \ \Eig{f} \ \textrm {contains \ exactly \ $1$ \ aligned \ triple}\},
$$
we have a surjective morphism
$$
\alpha : \mathcal{U}' \to \mathcal {AL},
$$
assigning to each $[f] \in \mathcal{U}'$ the unique triple of eigenpoints. Over the irreducible open subset
$$
\mathcal W := \mathcal {AL}
\setminus \{(P_1,P_2,P_3)\in\mathcal {AL}
\ | \ \sigma(P_1,P_2)=0, s_{11} s_{22} s_{33}=0\}
$$
of aligned triples not lying
on a tangent line to the isotropic conic $\iso$ and with tangency point one of the $P_i$'s, the fibers of $\alpha$ are projective linear systems of dimension $3$ by
Proposition \ref{manca il riferimento su ancillary  non e': condition_rank_aligned}. By the Fiber Dimension Theorem this implies that the open subset $\alpha ^{-1} (\mathcal W)$ is irreducible of dimension
$$
\dim \alpha ^{-1} (\mathcal W)
\subset \sU=8 ,
$$
so $\sL$ is a hypersurface.


To prove the irreducibility, we observe that since the eigenpoint condition corresponds to two linear conditions on the coefficients of cubic polynomials, by linear algebra methods we can determine a local parametrization of $\sL$.
Specifically, given a general aligned triple $(P_1, P_2, u_1 P_1 +u_2P_2)$, the matrix of conditions $\Phi(P_1, P_2, u_1 P_1 +u_2P_2)$ has rank $6$, so the associated linear system
$\Lambda (\Phi(P_1, P_2, u_1 P_1 +u_2P_2))$ has dimension $3$. We can then express $6$ coefficients of the general polynomial
$[f]\in \Lambda (\Phi(P_1, P_2, u_1 P_1 +u_2P_2))$ as rational function of $P_1,P_2,u_1,u_2$ and of the remaining $4$ coefficients. The free coefficients depend on the position of a non zero minor of order six, which, in principle, could involve any choice of columns. In fact this is not the case, as it turns out, that the last $3$ columns of the matrix
$M = \Phi(P_1, P_2, u_1 P_1 +u_2P_2)$ are always in the linear span of the first $7$ columns.

If the three components of $P_1 \times P_2$ are called $\alpha, \beta, \gamma$,
let:
\[
N_1 = \left(
\begin{array}{ccc}
\alpha & 0 & 0 \\
0 & \beta & 0\\
0 & 0 & \gamma
\end{array}
\right), \quad
N_2 = \left(
\begin{array}{ccc}
0 & \alpha & 0 \\
\gamma & 0 & 0\\
0 & 0 & \beta
\end{array}
\right).
\]
Moreover, let $c_0, \dots, c_9$ be the $10$ columns of $M$. Then the six
columns $c_0, c_1, c_2, c_4, c_5, c_7$ of $M$ are linearly dependent. More
precisely, if $L_1$ is the $9\times 3$ matrix whose columns are
$c_0, c_2, c_7$ and $L_2$ is given by the columns $c_1, c_4, c_5$,
then we have:
\begin{equation}
(L_1 N_1 + L_2N_2) (P_1\times P_2) = 0
\label{combLin}
\end{equation}
(the linear combination of $c_0, c_1, c_2, c_4, c_5, c_7$ which is zero
is obtained expanding this expression).
Similarly, the columns $c_1, c_2, c_3, c_5, c_6, c_8$ are linearly dependent
and~(\ref{combLin}) holds if $L_1$ in this case is $[c_1, c_3, c_8]$ and
$L_2$ is $[c_2, c_5, c_6]$. Finally, the columns
$c_4, c_5, c_6, c_7, c_8, c_9$ are linearly dependent and~(\ref{combLin})
holds if we take $L_1 = [c_4, c_6 c_9]$ and $L_2 = [c_5, c_7, c_8]$.

As a consequence, local parametrizations of $\sL$ can be given by considering the following open subsets:
for any multiindex
$$
I =\{i_1,\dots,i_6\}\subset \{ 0, 1,\dots 6\},
$$
we set
$$
\sV_I := \{ [(P_1, P_2, u_1 P_1 +u_2P_2)] \in \mathcal {AL} \ | \ \text{the\ columns\ of\ } \Phi
(P_1, P_2, u_1 P_1 +u_2P_2) \text{\ relative \ to} \ I\ \text{are\ independent}\}.
$$
Then for $[(P_1, P_2, u_1 P_1 +u_2P_2)]\in \sV_I$, the coefficients of any element
$[f] =[{\mathcal B} \cdot w_f]$ of the linear system
$\Lambda \bigl( \Phi (P_1, P_2, u_1 P_1 +u_2P_2) \bigr)$ can be expressed in the form
\begin{equation}\label{eq: parametrizzazione}
w_f =(b_0,
b_1,
b_2,
b_3,
b_4,
b_5, b_6,b_7,b_8,b_9)=w_f(P_1,P_2,u_1,u_2,b_j,b_7,b_8,b_9),
\end{equation}
where $b_i=b_i (P_1,P_2,u_1,u_2,b_j,b_7,b_8,b_9)$ for $i\in I$ and $j\not \in I$, $1\le j \le 7$, are suitable rational functions. On such an open subset we have then the rational
map
$$
\alpha_I : \sV_I \times \p^3 \to \sL, \quad
\alpha_I (P_1,P_2,u_1,u_2,b_j,b_7,b_8,b_9)=
[{\mathcal B} \cdot w_f(P_1,P_2,u_1,u_2,b_j,b_7,b_8,b_9)].
$$
Next we observe that for any $I \subset \{1,\dots, 7\}$, the image
$$
W_I:=\alpha_I (\sV_I \times \p^3)
$$
is irreducible, being image of an irreducible variety, and it
contains an open subset. Indeed, assume for simplicity that
$I=\{1,2,3,4,5,6\}$; an element $[f]\in W_I$ is represented by the determinant of the
matrix given by the rows $0,1,3,4,6,7$ of $\Phi (P_1, P_2, u_1 P_1 +u_2P_2)$ and the additional $4$ rows
\[
 \left(
 \begin{array}{cccccc}
  0 & \dots & 1&0&0&-b_6 \\
 0 & \dots & 0&1&0&-b_7 \\
 0 & \dots & 0&0&1&-b_8 \\
  & & &\mathcal {B} & & \\
 \end{array}
 \right).
\]
In particular, the coefficient $b_9$ of $z^3$ is equal to the determinant of the order six minor of $\Phi (P_1, P_2, u_1 P_1 +u_2P_2)$ relative to the first $6$ columns. It follows that $W_I \supset \sL \cap \{ a_9 \neq 0\}$.

\blue{assicurarsi che funziona anche negli altri casi}

Finally, we claim that the irreducible subvarieties $W_I$, $I\subset \{1,\dots, 7\}$ have a common non-empty intersection
including internal points. Indeed, an example of a polynomial
class $[g]$ with an aligned triple of eigenpoints and satisfying
$$
[g] \in \bigcap _I W_I \cap \Delta_{3,3}
$$
is given by a random choice of $P_1$, $P_2$ and $(u_1:u_2)$, and of $b_7,b_8,b_9$.


Hence $\sL$ is covered by irreducible open subsets, each intersecting every other, so it is irreducible.
\end{proof}


\begin{prop}
    The locus $\sV \subset \sL \subset \p^9$ of cubic forms with two or more aligned triples of eigenpoints has dimension
    $$
    \dim \sV=7.
    $$
\end{prop}
\begin{proof}
   By \Cref{theorem:rank_V}, five points forming a $V$-configuration are eigenpoints of some cubic form of and only if $\delta_1 (P_1,P_2,P_4) \cdot
   \delta_2 (P_1,P_2,P_3,P_4,P_5)=0$. Moreover, there is an open subset of the space of $V$-configurations, where the rank of the condition matrix is~$9$. As such a space has dimension $8$ and the two equations define two divisors, such a locus is of dimension~$7$ and it is birational to~$\sV$.
\end{proof}
 As a consequence, we have the following result.
\begin{corollary}
\label{lemma:pencil_one_aligned}
 If $[f],[g] \in \p^9 \setminus \Delta_{3,3}$ are general cubics,
 then the general cubic in the pencil
 $\lambda f + \mu g$ for $(\lambda: \mu) \in \p^1$ has no aligned triples of eigenpoints, and there is a finite number of cubics having exactly one aligned triple.
\end{corollary}

%\begin{definition}
% We define $\Delta \subset \mathcal{L}$ to be the closure of the locus of cubics with at least two aligned triples of eigenpoints.
%\end{definition}
%
%\begin{prop}
%  The variety~$\Delta$ has dimension~$7$ and it is the union of two irreducible components~$\Delta_1$ and~$\Delta_2$.
%\end{prop}

In what follows, we will use the following notation: if $g_1, g_2, g_3$ are the three minors of \Cref{eq:def_matrix} relative
to a cubic form~$f$, we denote by
$$
\Sigma_f = \p( \langle g_1, g_2, g_3 \rangle),
$$
the net of cubics, whose base locus is the eigenscheme~$\Eig{f}$.
%
\begin{lemma}
\label{lemma:scroll}
 If $f$ and $g$ are general cubics, then
 %
 \[
   \mathcal{N} := \bigcup_{(\lambda : \mu) \in \p^1} \Lambda_{\lambda f + \mu g} \subset \p^9
 \]
 %
 is an embedding of a rational projective bundle and has degree~$3$.
\end{lemma}
\begin{proof}
Consider the projective bundle given by the family of planes
%
\[
 \mathcal{P} := \{ \Lambda_{\lambda f + \mu g} \, : \, (\lambda: \mu) \in \p^1 \} \subset \p^1 \times \p^9
\]
%
Then $\mathcal{N}$ is the projection of~$\mathcal{P}$ on the second factor.
However, the map $\mathcal{P} \to \mathcal{N}$ contracts no subvariety of any plane of ${\mathcal P}$, so either it is an embedding or it contracts some horizontal curve. In the latter case, all the planes of the family should intersect in at least one point. In particular, the two nets $\Lambda_f$ and $\Lambda_g$ should have non-empty intersection.
If we denote by $g_1$, $g_2$ and $g_3$ the $2 \times 2$ minors relative to~$f$, and by $h_1$, $h_2$ and $h_3$ the ones relative to~$g$, the vectorial dimension of the linear span $\left\langle g_1, g_2, g_3, h_1, h_2, h_3 \right\rangle$ should be strictly less than $6$. This can be avoided, since such a condition corresponds to a proper closed subscheme of $\p^9 \times \p^9$.

It follows that if $f$ and $g$ are general enough, then $\mathcal{N}$ is a $3$-dimensional rational normal scroll in
%
\[
\mathcal{N} \subset \p(\left\langle g_1, g_2, g_3, h_1, h_2, h_3 \right\rangle) \cong \p^5.
\]
%
Being a variety of minimal degree, its degree is $5+1-3 = 3$.
\end{proof}

\begin{theorem}
The degree of $\mathcal{L}$ is equal to
%
\[
  \deg \ \mathcal{L} = 15 \,.
\]
%
\end{theorem}

\begin{proof}
We start by observing that a reduced $0$-dimensional eigenscheme contains an aligned triple if and only if the net of cubics
$\Lambda_f = \left\langle g_1, g_2, g_3 \right\rangle$ contains a cubic which splits in three lines, a so called \emph{triangle}. Moreover, if $f$ is general enough, we have exactly one aligned triple and the other $4$ points are in general position; in this case, the net~$\Lambda_f$ contains exactly three triangles, namely the unions of the line passing through the aligned triple and the reducible conics through the $4$ points in general position.

To determine the degree of $\mathcal{L}$ we consider a general pencil of cubic forms $\lambda f + \mu g$, and we will compute the number of elements with associated net $\Lambda_{\lambda f + \mu g}$ containing a triangle.

To this aim, denote by ${\mathcal T} \subset \p^9$ the variety of triangles; it is a classical result that its dimension is~$6$ and its degree is~$15$,
see for instance \cite[Section~2.2.2]{3264}. We now consider the variety~$\mathcal{N}$ from \Cref{lemma:scroll}.
Note that, since each net containing a triangle, actually contains exactly $3$ of them, the number of nets of $\mathcal{N}$ containing some triangle is given by
%
\[
 \frac{\mathcal{T} \cdot \mathcal{N}}{3} = \frac{{15} \cdot {3}}{3} = 15 \,.
\]
%
This implies that $\deg \mathcal{L} = 15$.
\end{proof}

\begin{es}
The following pencil of cubic forms admits exactly $15$ cubics with an aligned triple of eigenpoints ..
\end{es}


\section{Eigenschemes of positive dimension}
\label{positive_dim}

In this section, we consider positive dimensional eigenschemes.
By \cite{BGV}, an eigenscheme cannot be of pure dimension~$1$, and the possible $1$-dimensional components are of degree~$1$ or~$2$.
In what follows, we shall determine the degree of the $0$-dimensional residual scheme in both cases.

\begin{prop}
\label{p2}
Let $C = V(f) \subset \p^2$ be a cubic curve.
Assume that $\dim \Eig{f} = 1$.
%
\begin{enumerate}
\item If the $1$-dimensional component of $\Eig{f}$ is a line~$L$,
then the residual subscheme $Z := \mathrm{Res}_l \bigl( \Eig{f} \bigr)$ in~$\Eig{f}$ with respect to~$l$ has degree~$3$.
\item If the $1$-dimensional component of $\Eig{f}$ is a conic~$q$,
then the residual subscheme $Z := \mathrm{Res}_q \bigl( \Eig{f} \bigr)$ in~$\Eig{f}$ with respect to~$q$ has degree~$1$.
\end{enumerate}
%
\end{prop}
\begin{proof}
(1)
Let $g_1$, $g_2$, and $g_3$ be the order~$2$ minors determining the eigenscheme of~$f$.
Since $g_1$, $g_2$, and $g_3$ have a common linear component~$l$, by writing
\[
 g_i = l \, h_i, \quad i=1,2,3
\]
we have that $\mathcal{I}_{Z, \p^2} = \mathcal{I}_{\Eig{F}), \p^2} (-1)$
fits in the exact sequence
\[
 0 \to \mathcal{G} \to \oo_{\p^2} (-2) \oplus \oo_{\p^2} (-2) \oplus \oo_{\p^2} (-2) \to \mathcal{I}_{Z,\p^2} \to 0 \,,
\]
where $\mathcal{G}$ is a rank two reflexive sheaf by \cite[Proposition 1]{Hartshorne1980}.
But on a smooth surface reflexive implies locally free (see \cite[Example~1.1.6]{Huybrechts2010}),
so $\mathcal{G}$ is a rank two vector bundle.

Next we observe that the two independent
syzygies between the generators $g_i$ give rise to the syzygies:
\[
z\, h_1 - y\, h_2 + x\, h_3 = 0, \qquad \partial_z f\, h_1 - \partial_y f \, h_2 + \partial_x f \ h_3=0 \,,
\]
which occur in degrees $3$ and $4$. We claim that the two relations are independent, again,
otherwise the $g_i$'s would be identically zero. So we apply \cite[Proposition~12]{Ellia2020}, and we have that $\mathcal {G}$ splits. Observe that as $c_1(\mathcal {I}_{Z,\p^2})=0$, we have $c_1(\mathcal {G})=-6$. Moreover, there is no syzygy in degree $2$, since
otherwise the $h_i$ would be linearly dependent, so the $h_i$ would belong to a pencil, which is a contradiction \textbf{(giustificare)}.

It follows that the splitting of $\mathcal{G}$ is of the form
%
\[
\mathcal{G} \cong \oo_{\p^2} (-3) \oplus \oo_{\p^2} (-3) \,.
\]
%
We get the free resolution of the ideal sheaf:
%
\[
0\to \oo_{\p^2} (-3)\oplus \oo_{\p^2} (-3) \to 3\oo_{\p^2} (-2)\to \mathcal {I}_{Z,\p^2} \to 0 \,.
\]
%
Moreover
%
\[
c_2 (\mathcal{I}_{Z,\p^2} ) = c_2 \bigl( 3\oo_{\p^2}(-2) \bigr) - c_2 (\mathcal{G}) = 12 - 9 = 3 \,,
\]
%
and $h^0 \bigl( \mathcal{I}_{Z,\p^2}(1) \bigr) = 0$, which proves the first statement.

\medskip
(2) is similar, to be completed.
\end{proof}

\blue{
The proof relies on the exactness of the Kozsul complex associated with the regular sequence $x,y,z$, specifically we will use the following result (see \cite[Theorem 7.3.13]{Dolgachev} or \cite[Lemma 3.9]{BGV}).



\begin{lemma}\label{lem: Kozsul}
Let $h_1,h_2,h_3\in\C[x,y,z]_d$ with $d \ge 1$. Then 
\begin{equation}\label{relazione lineare}
xh_1-yh_2+zh_3=0
\end{equation}
 if and only if there exist $m_1,m_2,m_3\in\C[x,y,z]_{d-1}$ such that
\begin{equation}\label{eq: minors lemma}
    h_1 = ym_3-zm_2,\qquad h_2 = zm_1-xm_3,\qquad 
    h_3 = xm_2-ym_1.
\end{equation}

\end{lemma}

\begin{proof}
If $h_1,h_2,h_3$ satisfy \eqref{eq: minors lemma}, then it is immediate to check that they satisfy \eqref{relazione lineare} as well. Conversely, assume that \eqref{relazione lineare} holds and let $R = \C[x,y,z]$.
The Koszul complex in the ring
$R$ is an exact sequence of $R$-modules
$$
0\to R\xrightarrow{\alpha} R^{\oplus 3} \xrightarrow{\beta} R^{\oplus 3} \xrightarrow{\gamma} R \to R/(x,y,z)
\to 0,
$$
where the maps are $\alpha(p)=(p x,p y,p z)$, $\gamma (w_1,w_2,w_3)=w_1 x + w_2 y + w_3 z$ and $\beta$ is defined by the matrix
$$
\left(
\begin{array}{ccc}
0 & -z & y\\
z & 0 & -x\\
-y & x & 0 \\
\end{array}
\right).
$$
The syzygy $xh_1-yh_2+zh_3=0$ implies that $(h_1, -h_2, h_3)$ is in the kernel of $\gamma$, and since the Koszul complex is exact, the triple $(h_1,-h_2, h_3)$ lies in the image of $\beta$. It follows that there exist $m_1,m_2,m_3\in\C[x,y,z]_{d-1}$ such that \eqref{eq: minors lemma} holds.
\end{proof}

Now we are in the position for proving \Cref{p2}.


\begin{proof}
Let $g_1$, $g_2$ and $g_3$ be the order~$2$ minors determining the eigenscheme of~$f$, and let $g$ be the greatest common factor.
By writing
\[
 g_i = g \, h_i, \quad i=1,2,3
\]
we have that the residual scheme is defined by the ideal
$(h_1,h_2,h_3)$. Moreover, the
linear
syzygy between the generators $g_i$ gives rise to the syzygy:
\[
z\, h_1 - y\, h_2 + x\, h_3 = 0 \,.
\]
By \Cref{lem: Kozsul}, the triple $(h_1,h_2,h_3)$ is the triple of order two minors of a matrix
\[
\begin{pmatrix}
    x & y & z \\
    m_1 & m_2 & m_3
    \end{pmatrix}.
\]
for suitable forms $m_i \in \C[x,y,z]_r$, where $r =2 - \deg g \ge 0$. By the assumption that $g$ is the greatest common factor of the three minors $g_1$, $g_2$ and $g_3$, we have that the zero scheme of $(h_1,h_2,h_3)$ is $0$-dimensional, and 
by \Cref{lem:nonempty} its degree is $3$ if $r=1$, and the degree is $1$ if $r=0$.



\end{proof}
}

\begin{es}
Consider
\[
 f(x, y, z) = x^2 (y - z)
\]
We have $l=x$,
\[
 h_1=x^2-2y^2+2y z, \quad h_2= 2z^2-x^2-2y z, \quad h_3= 2yz-2z^2-xy \,.
\]
The two syzygies in degree~$3$ are:
\[
z \, h_1 - y \, h_2 + x \, h_3 = 0, \quad x \, h_2 + 2(y+z) \, h_3 + x \, h_1 = 0.
\]
Finally, $Z= \{ (0:1:1),(2:1:-1),(-2:1:-1) \}$. Observe that one point is on the singular line $x=0$.
\end{es}

\subsection{Eigenschemes containing a line}

In this section, we consider polynomials having a whole line in the eigenscheme. We shall prove that residually we then have, in general, other three eigenpoints, and that the condition of having a line of eigenpoints is equivalent to having four collinear eigenpoints.

Finally, we shall prove that such polynomials belong all to the locus $\sL \subset \p^9$.

\begin{lemma}
\label{4ptiSuRetta2}
Suppose $P_1, P_2, P_3, P_4$ are four distinct points belonging to a line~$t$
and let $C$ be a cubic such that has $P_1, \dots, P_4$ among its eigenpoints. 
Then all the points of $t$ are eigenpoints of~$C$.
Moreover,
%
\begin{equation*}
6 \leq \rk \ \Phi(P_1, P_2, P_3, P_4) \leq 7
\end{equation*}
%
and the rank is~$6$ if and only if $\sigma(P_1, P_2) = 0$, i.e.\ if
and only if $t$ is tangent to the isotropic conic. If the rank of
the above matrix is~$7$, then all the cubics which have the line $t$
of eigenpoints are given by~$t^2 \ell$ where $\ell$ is any line of the plane.
\end{lemma}
\begin{proof}
\verb+file: fourCollinearPoints.sage+\\
It is not restrictive to assume that one of the points (say~$P_1$) is
the point~$(1: 0: 0)$.
With this simplification, the computation of the order~$6$ and order~$7$
minors of $\phi(P_1, \dots, P_4)$ is not hard and
the first part of the thesis follows. If the rank of $\Phi(P_1, \dots, P_4)$
is $7$, then the cubics which have $P_1, \dots, P_4$ as eigenpoints
give a two-dimensional linear subspace~$W$ of~$\p^9$, but also all
the cubics defined by $t^2 \ell$ have $P_1, \dots, P_4$ as eigenpoints and are
a two-dimensional linear space of~$\p^9$ contained in~$W$. Hence the two linear
spaces coincide.
\end{proof}

\begin{prop}
\label{prop:eigenline_non_tangent}
Given a line~$t$ of the plane, if $t$ is not tangent to the isotropic
conic, then all the cubics which have, among their
eigenpoints, the points of $t$ are defined by the polynomial $t^2l$,
where $l$ is any line of the plane. If $t$ is tangent to $\iso$, then
all the cubics which have among their eigenpoints the points of $t$ are
defined by the polynomials
\begin{equation}
\label{cubiche_con_retta_autop_tg}
t^2l+\lambda C(r),
\end{equation}
where $\lambda \in K$ and $r$ is a line
passing through the tangent point $\iso\cap t$ and $C(r)$ is defined
in \Cref{2_lines_of_eigenpoints}. Conversely, if $r$ is a line and
$t$ is another line, tangent $\iso$ in one of the two points of intersections
of $r$ with $\iso$, then all the cubics given
by \Cref{cubiche_con_retta_autop_tg} have, among their eigenpoints,
the line~$t$.
\end{prop}
\begin{proof}
\verb+una_retta_tg_autop.sage+\\
The first part of the proposition is contained in~\Cref{4ptiSuRetta2}.
Suppose $t$ is tangent to $\iso$. We assume that $t$ is the line
$x+\iii y$, hence tangent to $\iso$ in $P = (1: \iii: 0)$.
We impose to the points
of $t$ to be eigenpoints of a generic cubic $a_0x^3+a_1x^2y+\cdots+a_9z^3$
and we get a linear system in $a_0, \dots, a_9$ whose solution gives that
the cubics can be expressed as a linear combinations of the following
four cubics:
%
\[
H_1 = t^2x, \quad H_2 = t^2y, \quad H_3 = t^2z, \quad H_4 = C(r)
\]
%
where $r$ is a line passing through~$P$ (if, for instance, $r$
is $z$, then $H_4 = z(x^2 + y^2 + 2/3z^2)$. Hence the cubics are of the
form of \Cref{cubiche_con_retta_autop_tg}. Conversely, it is easy to verify
that all the cubics $C(r)$ of~(\ref{cubiche_con_retta_autop_tg}) are
linear combinations of $H_1, \dots, H_4$, hence it is enough
to see that every linear combination of $H_1, \dots, H_4$ gives a
cubic which has the line~$t$ among its eigenpoints, and the this result
follows from an explicit computation.
\end{proof}

\begin{prop}
Let $t$ be a line tangent to the isotropic conic in a point~$P$. 
Then all the cubics of the family from \Cref{cubiche_con_retta_autop_tg} are singular in~$P$ and $t$ is one of the two tangents to the cubic in~$P$.
\end{prop}
\begin{proof}
As seen in \Cref{prop:eigenline_non_tangent}, up to the action of~$\SO_3(\C)$, all the
cubics are a linear combination of $H_1, \dots, H_4$ and a direct
computation gives the thesis.
\end{proof}

NOTA: Le cubiche con P1 singolare e tangente in P1 data da tg1 sono
un sistema lineare di dimensione 6, le cubiche
di~(\ref{cubiche_con_retta_autop_tg}) sono di dimensione
4, Che altre proprieta' deve avere una cubica per rientrare
in~(\ref{cubiche_con_retta_autop_tg})?

\medskip
Now we show that any line of the plane can be realized as a line of eigenpoints for some limit polynomial in a family, where the general member has a $0$ dimensional eigenscheme with an aligned triple. The idea is to fix three points on a line and a fourth point varying on another line. We consider a family having such four points as eigenpoints; it turns out that when the fourth point is collinear with the first three points, the line containing them is entirely contained in the eigenscheme.

\begin{prop}
\label{prop:line_as_limit}
    Any line $t \subset \p^2$ can be realized as the positive dimensional part of a limit eigenscheme in a family of zero dimensional eigenschemes having an aligned triple.
\end{prop}

\begin{proof}
    Without loss of generality we can assume that $t$ is the line $x=0$ if it is not tangent to $\iso$, respectively $x+\iii y=0$ if it is tangent to $\iso$.

    In the first case, consider the points
    \[
    P_1=(0:1:0), \ P_2=(0:0:1), \ P_3=(0:1:1), \ P_4=(u+1:1:-1).
    \]
    The cubic polynomials having these points as eigenpoints form the $2$-dimensional family depending on the parameters $m$, $l$, and $u$:
    %
    \begin{multline*}
    (3mu^2 + 6mu - 9m) x^3 + (3 l u^2+(6l + 18m)u - 9l + 18m) x^2 y + (3lu^2 + 6lu - 9l) z x^2 + \\
    +(-l u^4+(-4 l-3 m) u^3+(-2 l-9 m) u^2+(4 l-9 m) u + 3 l-3 m) y^3+ \\
    +(-l u^4 + (-4 l-3 m)u^3 + (-2l - 9m) u^2 + (4l - 9m)u + 3l - 3m) z^3 \,.
    \end{multline*}
    %
    By setting, for instance $m=l=1$, we get the $1$-dimensional family depending on the affine parameter $u$:
    %
    \begin{multline*}
    (3 u^2+6 u-9) x^3+(3 u^2+24 u+9) x^2 y+(3 u^2+6 u-9) z x^2+\\
    (-u^4-7 u^3-11 u^2-5 u) y^3+(-u^4-7 u^3-11 u^2-5 u) z^3 \,.
    \end{multline*}
    %
    A direct computation shows that if $u \neq -1$, the associated eigenscheme is $0$-dimensional, while for $u=-1$ we get the polynomial
    %
    \[
     -12 x^2 (x+ y+ z).
    \]
    %
    Similarly, if we consider the line $x+\iii y=0$, we set
    %
    \[
    P_1=(\iii:-1:1), \ P_2=(0:0:1), \ P_3=(-\iii:1:1), \ P_4=(u+1: \iii:0).
    \]
    %
    The cubic polynomials having these points as eigenpoints form the $2$-dimensional family depending on the parameters $m$, $l$, and $u$:
$$
3m(u-1)x^3+3l(u-1)zx^2-9\iii mx^2y+9m(u+1) xy^2+3 \iii l(u+2)(u-1)zyx+3\iii (2u+1)my^3+(-3u^2+3)ly^2 z-l u z^3 (u-1).
$$
By setting, for instance $m=l=1$, we get the $1$-dimensional family depending on the affine parameter $u$:
$$
3(u-1)x^3+3(u-1)zx^2-9\iii x^2y+9(u+1) xy^2+3 \iii (u+2)(u-1)zyx+3\iii (2u+1)y^3+(-3u^2+3)y^2 z- u z^3 (u-1).
$$
A direct computation shows that if $u \neq 0$, the associated eigenscheme is $0$-dimensional, while for $u=0$ we get the polynomial
\[
-3(x+\iii y+z)(x+\iii y)^2. \qedhere
\]
\end{proof}

\subsection{Eigenschemes containing two lines}
\begin{prop}
\label{cubiche_con_2_rette}
Let $C$ be a cubic which has, among its eigenpoints, two lines $t_1$
and $t_2$. Then $t_1$ and $t_2$ are tangent to the isotropic conic
in two points $Q_1$ and $Q_2$ and, if $r$ is the line $Q_1 \vee Q_2$
of equation $ax+by+cz=0$, the polynomial defining~$C$ is:
%
\begin{equation}
C(r) = \left(r^2-3\left(a^2+b^2+c^2\right)\iso\right)r.
\label{2_lines_of_eigenpoints}
\end{equation}
%
The cubic~$C$ has, in addition to $t_1$ and $t_2$, also the eigenpoint $(a: b: c)$
(the pole of~$r$ w.r.t.\ $\iso$). \\
Conversely, if $C$ is defined by~(\ref{2_lines_of_eigenpoints}), where
$r$ is any line $ax+by+cz$ (with $a, b, c \in K$), then
the eigenpoints of $C$ are the lines $t_1$ and $t_2$ (tangent to~$C$
in the points $Q_1$ and $Q_2$ of intersection of~$r$ with~$\iso$) and
the point $(a: b: c)$.
\end{prop}
\begin{proof}
\verb+file: due_rette_autopunti.sage+\\
Let $t_1$ be a line of eigenpoints of~$C$. If $t_1$ is not tangent
to $\iso$, from~\Cref{4ptiSuRetta2} $C$ is given by $t_1^2l$ (for
a suitable line $l$), but this cubic cannot have the line $t_2$
of eigenpoints.
In this way, we see that $t_1$ and $t_2$ must be tangent to~$\iso$ (in~$Q_1$ and~$Q_2$, respectively). Let $Q_3 = t_1 \cap t_2$. We can assume
that $Q_3 = (1: 0: 0)$, hence $t_1$, $t_2$ and $r$ have equations,
respectively, $x+\iii y=0$, $x-\iii y = 0$ and $x=0$.
If we impose to the generic cubic of the plane to have
$t_1$ and $t_2$ of eigenpoints, we get only one cubic defined by
$x(2x^2 + 3y^2 + 3z^2)$, which is of the
form of \Cref{2_lines_of_eigenpoints}. If, conversely, we fix a line
of the plane of equation $ax+by+cz=0$ and we consider the cubic~$C$ from \Cref{2_lines_of_eigenpoints}, the ideal of
the eigenpoints of~$C$ decomposes into an ideal whose radical is
the ideal of the point $(a:, b: c)$ and another ideal generated by
a degree two polynomial which give a singular conic of the pencil
of conics bitangent to~$\iso$ in the two points~$\iso \cap r$, hence
this conic splits into the two tangent lines to~$\iso$ in~$\iso \cap r$.
\end{proof}


\subsection{Eigenschemes containing an irreducible conic}
\begin{lemma}
\label{lemma:fourOnIso}
Suppose $P_1, P_2, P_3, P_4$ are four distinct points on $\iso$ and let
$C$ be a cubic such that has $P_1, \dots, P_4$ among its eigenpoints.
Then all the points of $\iso$ are eigenpoints and $C$ splits into $\iso$
and a line $r$. In particular, if four
distinct points of $\iso$ are eigenpoints of a cubic, then all the
points of $\iso$ are eigenpoints of the cubic.
\end{lemma}
\begin{proof}
\verb+file eigenpointsConic.sage+\\
As usual, we can assume that $P_1 = (1: \iii: 0)$. Hence all the other points
of $\iso$ are given by
%
\[
  P(w) = \bigl(\iii(w^2 + 1): 1 - w^2: -2w\bigr) \text{ with } w \in K \,,
\]
%
so we can assume
$P_2 = P(w_1)$, $P_3 = P(w_2)$, and $P_4 = P(w_3)$, where the $w_i$'s are
all distinct. The cubics which have $P_1, \dots, P_4$ among the eigenpoints
are obtained from the linear system whose associated matrix is
$M = \Phi(P_1, P_2, P_3, P_4)$. The rank of~$M$ is $7$. In order to see this,
we can, first of all, select a submatrix of~$M$ obtained erasing a
row for each of the points~$P_i$ (since we know that the three vectors~$\phi(P)$ are linearly dependent for every~$P$). In the obtained matrix
all the order $8$ minors are easy to compute and are zero.
If all the order~$7$
minors are set to zero, we get an ideal that, after suitable saturations,
is the whole ring. Since the rank of~$M$ is~$7$, all the cubics for which the
four points are eigenpoints are a linear variety~$\mathcal{K}$
in~$\p^9$
of dimension~$2$, but all the cubics which split into~$\iso$ and a line
of the plane do have all the points of~$\iso$ among the eigenpoints and
these cubics give a linear variety of dimension~$2$ in $\p^9$
which therefore coincides with~$\mathcal{K}$.
\end{proof}

\begin{lemma}
\label{lemma3ptiSuCiso}
It is not possible to have a cubic $C$ which has, among its eigenpoints,
an irreducible conic $\Gamma$ tangent in a point $P_1$ to the isotropic
conic and intersecting $\iso$ in two other distinct points.
\end{lemma}
\begin{proof}
\verb+caso3ptiIso_2.sage+\\
As in the proof~\Cref{lemma:fourOnIso}, we can assume that
$P_1 = (1: \iii: 0)$
and $P_2$ and $P_3$ are $P(w_1)$ and $P(w_2)$. Then we construct
the pencil of conics passing through $P_1$, $P_2$ and $P_3$ and
tangent to~$\iso$ in~$P_1$ and we construct two distinct points~$Q_1$ and~$Q_2$ on a conic of this family. If there is a conic~$\Gamma$
tangent to~$\iso$ in~$P_1$ and passing through~$P_2$ and~$P_3$,
then, for suitable values of the parameters defining the points
$P_1, P_2, P_3$ and $Q_1$ and $Q_2$, the matrix
$M = \Phi(P_1, P_2, P_3, Q_1, Q_2)$
must have rank~$9$ or smaller. Again, erasing some rows of~$M$
which are linear dependent from the others and observing that,
with row elementary operations we can erase some columns of~$M$,
we can see that $M$ cannot have rank~$8$ or smaller and if $M$
has rank~$9$, then the corresponding cubic does not have a conic
among its eigenpoints.
\end{proof}

\begin{lemma}
\label{lemma:bitangentToCiso}
If a cubic $C$ has, among its eigenpoints, an irreducible conic $\Gamma$
which is tangent to $\iso$ in two distinct points $P_1$ and $P_2$, then
the equation of $C$ is of the form $(\lambda \iso + \mu r^2)r$ where
$r$ is the equation of the line $P_1\vee P_2$ and $\lambda, \mu$ are parameters
in $K$.
\end{lemma}
\begin{proof}
\verb+caso2ptiIso_partI.sage+\\
\verb+caso2ptiIso_partII.sage+\\
The proof is quite similar to the one of~\Cref{lemma3ptiSuCiso}. Also
here we can assume that $P_1 = (1: \iii: 0)$ and we construct the pencil of
bitangent conics in $P_1$ and $P_2$ to $\iso$. Then we construct three
points $Q_1, Q_2$ and $Q_3$ on a conic of the pencil
and we study the matrix $M = \Phi(P_1, P_2, Q_1, Q_2, Q_3)$. We can see
that the rank of this matrix is $9$ (for each admissible value of the
involved parameters), hence $M$ identifies a cubic~$C$. The computation
of $C$ shows that its equation is of the above form.
\end{proof}


\begin{prop} All the cubics $C$ defined by the polynomial
\begin{equation}
\label{cub_conica_eig}
(\lambda \iso + \mu r^2)r \quad \mbox{$\lambda, \mu \in K$}
\end{equation}
where $r=ax+by+cz$ is a line of the plane,
have the eigenpoints given by the conic
$\lambda \iso+3\mu r^2$ (bitangent to $\iso$ in the points $\iso \cap r$)
and the point $(a:b:c)$ (the pole of $r$
w.r.t.\ $\iso$). Conversely, if
$C$ is a cubic of the plane which has a conic among its eigenpoints,
then there exists a line $r$ in the plane and $\lambda, \mu \in K$
such that the equation of
$C$ is given by~(\ref{cub_conica_eig}).
\end{prop}
\begin{proof}
\verb+caso2ptiIso_partI.sage+ (parte finale del file).\\
If the equation of $C$ is of the form described above, a direct computation
shows that the eigenpoints of $C$ are the pole of $r$ with respect to $\iso$
and the conic $\lambda \iso+3\mu r^2$. Suppose, conversely, that $C$ has
a conic among its eigenpoints. First, we assume that the conic $\Gamma$
is irreducible. If $\Gamma$ is the isotropic conic,
then~\Cref{lemma:fourOnIso} gives that $C$ splits into~$\iso$ and a line.
If $\Gamma$ does not coincide with the
isotropic conic, as a consequence
of~\Cref{lemma:fourOnIso} and~\Cref{lemma3ptiSuCiso} must be bitangent,
in two points $P_1$ and $P_2$, to $\iso$. Then we get the thesis
from~\Cref{lemma:bitangentToCiso}. If $\Gamma$ is reducible,
from~\Cref{cubiche_con_2_rette} we have that $\Gamma$ splits
into two tangent lines to $\iso$.
\end{proof}
\begin{rmk}
if $r$ is tangent to $\iso$ in a point $P$, @@tutto pare funzionare,
ma sistemare
\end{rmk}



\section{Possible configurations of the seven eigenpoints}
\label{further_alignments}

\begin{table}
\caption{All possible configurations of $7$ points with alignments that can appear as eigenschemes of a ternary cubic form, provided that the eigenscheme is zero-dimensional and reduced.}
\centering
\begin{tabular}{|llll|}\hline
  num. lines  & collinear vertices & name & conf.\\ \hline
 1& [(1, 2, 3)] & "line" & $(C_1)$\\
 2& [(1, 2, 3), (1, 4, 5)] & "X shape"& $(C_2)$\\
 3& [(1, 2, 3), (1, 4, 5), (1, 6, 7)] & "star" & $(C_3)$\\
  & [(1, 2, 3), (1, 4, 5), (2, 4, 6)] & "triangle" & $(C_4)$\\
 4& [(1, 2, 3), (1, 4, 5), (1, 6, 7), (2, 4, 6)] & "triangle + altitude"
& $(C_5)$\\
  & [(1, 2, 3), (1, 4, 5), (2, 4, 6), (3, 5, 6)] & "two X shapes" & $(C_6)$\\
 5& [(1, 2, 3), (1, 4, 5), (1, 6, 7), & "two stars" & $(C_7)$\\
  & \phantom{[}(2, 4, 6), (2, 5, 7)] & &\\
 6& [(1, 2, 3), (1, 4, 5), (1, 6, 7), & "triangle + three altitudes" & $(C_8)$\\
  & \phantom{[} (2, 4, 6), (2, 5, 7), (3, 4, 7)] & & \\
 7& [(1, 2, 3),
   (1, 4, 5),
   (1, 6, 7),
   (2, 4, 6), & "Fano matroid" & $(C_9)$\\
  & \phantom{[} (2, 5, 7),
   (3, 4, 7),
   (3, 5, 6)] & & \\ \hline
\end{tabular}
\label{table:all_alignments}
\end{table}

\Cref{table:all_alignments} lists all the possible configuration
of 7 distinct points of the plane, according to their collinearities. Here
we want to see which of those nine configurations can be realized by
the seven eigenpoints of a cubic and, in case the configuration is
realizable, which are the cubics with that configuration of eigenpoints.

First of all, it is well-known that configuration (9) cannot be realized
by seven points of the plane over a field of zero
characteristic (see \cite{Whitney1935}), therefore we will not consider
it in our analysis.

\underline{Configuration $(C_1)$} can be realized,
\Cref{proposition:three_aligned_ranks} and \Cref{locus_one_alignment}
give a description of the cubics with such a configuration of points:
we fix two points $P_1$ and $P_2$ in $\mathbb{P}^2$, we take a generic $P_3$
on the line $P_1\vee P_2$, all the cubics with
$P_1, P_2, P_3$ eigenpoints is $\Lambda(\Phi(P_1, P_2, P_3))$, the three
dimensional linear subspace of $\mathbb{P}^9$ (four dimensional, if the
line $P_1\vee P_2$ is tangent in $P_1, P_2$ or $P_3$ to $\iso$).

\underline{Configuration $(C_2)$} means that the points
$P_1, \dots, P_5$ are in a
$V$-configuration, hence the rank of the matrix $\Phi(P_1, \dots, P_5)$
must be $9$ or $8$. If the rank is $9$, then $\delta_1(P_1, P_2, P_4) = 0$
or $\delta_2(P_1, \dots, P_5) = 0$, see~\Cref{theorem:rank_V}.
From \Cref{proposition:third_alignment} we have that the only possibility
is $\delta_1(P_1, P_2, P_4) = 0$. Hence if we fix two points $P_1$ and $P_2$
in the plane in an arbitrary way, we choose $P_4$ such that
$\scl{P_4}{s_{11}P_2-s_{12}P_1}=0$
(see~Equation~\eqref{rmk_delta_case1}) and then we chose any $P_3$
on the line $P_1\vee P_2$ and any $P_5$ on the line $P_1\vee P_4$, we have a cubic
with a configuration of type $(C_2)$. In case of rank $8$, as seen
(see~\Cref{rmk:particular_cases}),
configuration $(C_2)$ can only be
obtained from the conditions~(\ref{rk8_1}) (and hence are sub-cases
of the case $\delta_1(P_1, P_2, P_4)=0$).

\underline{Configuration $(C_3)$} Suppose we have $7$ eigenpoints such that
$(P_1, P_2, P_3)$, $(P_1, P_4, P_5)$ and $(P_1, P_6, P_7)$ are
aligned. Hence
$V_1 = (P_1, P_2, P_3, P_4, P_5)$, $V_2 = (P_1, P_2, P_3, P_6, P_7)$ and
$V_3 = (P_1, P_4, P_5, P_6, P_7)$, are three $V$-configurations. It holds:

\begin{lemma}
\label{no_delta1_delta1} Suppose we have seven eigenpoints $P_1, \dots, P_7$
of a cubic in configuration $(C_3)$. Then among the $7$ points there is a
$V$-configuration that satisfies a $\delta_2$ condition.
\end{lemma}
\begin{proof}
%% la dim si trova in config3.sage
The points $P_1, P_2, P_3, P_4, P_5$ are in a $V$-configuration. If
$\rk \Phi(P_1, \dots, P_5) = 8$, the result comes from~\Cref{rank_8}.
Therefore, assume that the matrix $\Phi(P_1, \dots, P_5)$ has
rank~$9$. If
$\delta_2(P_1, \dots, P_5) = 0$, we have the thesis, otherwise,
$\delta_1(P_1, P_2, P_4) = 0$. Then consider the $V$-configuration
$P_1, P_2, P_3, P_6, P_7$. As above, we suppose $\delta_1(P_1, P_2, P_6) = 0$.
The two equations obtained from the two $\delta_1$ give two
linear equations in the coordinates
of $P_2$. Assuming the matrix of the system of linear equations have
maximal rank, the solution gives a point $P_2$ that coincide, as
a projective point, to $P_1$, which is impossible. Hence we are forced to
consider the
case in which the matrix of the linear system do not have maximal rank.
This condition implies that $\scl{P_1}{P_1} = 0$, hence $P_1$ is on the
isotropic conic. W.l.o.g.\ we can assume $P_1 = (1: \iii: 0)$ and again
we can determine $P_2$ such $\delta_1(P_1, P_2, P_4)$ and
$\delta_1(P_1, P_2, P_6)$ are zero. The matrix $\Phi(P_1, P_4, P_5, P_6, P_7)$
must have rank $9$ or smaller, then either
$\delta_2(P_1, P_4, P_5, P_6, P_7)=0$ or $\delta_1(P_1, P_4, P_6) = 0$. In
the first case, we have a $\delta_2$ condition among the points, hence
we assume $\delta_1(P_1, P_4, P_6) = 0$. Solving this equation and
considering the corresponding points, we get that either
$\delta_2(P_1, P_2, P_3, P_6, P_7) = 0$ or
$\delta_2(P_1, P_2, P_3, P_4, P_5) = 0$.
\end{proof}

As a consequence of \Cref{no_delta1_delta1}, we have that configuration $(C_3)$ can
be obtained only if we have 5 points such that
$\delta_2(P_1, \dotsc, P_5) = 0$.~\Cref{prop:definitionP3} describes how to
obtain five points which give a configuration $(C_3)$ of eigenpoints.

\underline{Configuration $(C_4)$}. It holds:
\begin{prop}
\label{conf4no} If we have five points that are in
configuration $(C_4)$ and such that are eigenpoints of a cubic, then the
cubic has $7$ eigenpoints which are in configuration $(C_8)$. In particular,
configuration $(C_4)$ is not realizable.
\end{prop}
\begin{proof}
%% il file config4.sage contiene la dimostrazione
From the results of~\Cref{rank_8}, we know that $(C_4)$
cannot be obtained from a rank 8 $V$-configuration,
hence it remains to consider the case $\delta_1(P_1, P_2, P_4) = 0$,
$\delta_1(P_2, P_1, P_4) = 0$ and $\delta_1(P_4, P_1, P_2) = 0$,
which implies
%
\[
  \delta_2(P_1, P_2, P_3, P_4, P_5)=0 \,, \quad
  \delta_2(P_2, P_1, P_3, P_4, P_6)=0 \,, \quad
  \delta_2(P_4, P_1, P_5, P_2, P_6)=0 \,. \qedhere
\]
%
\end{proof}


\underline{Configuration $(C_5)$}
\verb+file config5.sage+
\begin{prop}
%\textbf{(AGGIUNGERE IPOTESI SU RANGO NON 8)} 
Suppose we have $7$ points which are eigenpoints of a cubic and are
aligned according to configuration $(C_5)$. Then 
it holds
\[
P_1 = P_2 \times P_4,
\]
and among the points $P_1, \dots, P_6$, where $P_6$ is collinear with $P_2$ and $P_4$, we have the relation
\begin{equation}
s_{26}(s_{45}s_{13}-s_{34}s_{15})+s_{46}(s_{25}s_{13}-s_{23}s_{15}) = 0.
\label{cndC5}
\end{equation}
Moreover, the points $P_6$ and $P_7$ are determined as follows:
 \begin{eqnarray}
    P_6 & = & (s_{15}s_{24}s_{34}+s_{15}s_{23}s_{44} -s_{13}s_{25}s_{44} -s_{13}s_{24}s_{45})P_2 + (s_{13}s_{24}s_{25}-2s_{15}s_{22}s_{34}+s_{13}s_{22}s_{45})P_4 ,\label{p6formula}\\
P_7 & = & (s_{16}(s_{26}s_{45}+s_{24}s_{56})-s_{26}s_{15}s_{46}-s_{24}s_{15}s_{66})P_1-
(s_{11}(s_{26}s_{45}+s_{24}s_{56})-s_{26}s_{15}s_{14}-s_{24}s_{15}s_{16})P_6 ,
\label{p7formula}
\end{eqnarray}



Conversely, if $P_1, P_2, P_3, P_4, P_5$ are in a $V$-configuration such that 
\[
P_1=P_2 \times P_4,
\]
and the condition matrix satisfies
$\rk \Phi (P_1, \dots, P_5) =9$, then if the corresponding unique cubic has exactly $7$ distinct eigenpoints, they are in a $(C_5)$ configuration; in particular, the relations \eqref{cndC5}, \eqref{p6formula} and \eqref{p7formula} are satisfied.

\end{prop}

\bigskip

\begin{proof}
We can define the points as follows: $P_1$, $P_2$ and $P_4$ with generic
coordinates, meanwhile $P_3 = u_1P_1+u_2P_2$, $P_5 = v_1P_1+v_2P_4$ and
$P_6 = w_1P_2+w_2P_4$. The points $(P_2, P_1, P_3, P_4, P_6)$ are in a
$V$-configuration and we may assume that $\rk \, \Phi(P_2, P_1, P_3, P_4, P_6) = 9$ (since the case of rank~$8$ has already been discussed), hence necessarily $\delta_1(P_2, P_1, P_4)=0$ because we suppose that $P_2$, $P_5$, and~$P_7$ are not aligned.
Analogously, $\delta_1(P_4, P_1, P_2) = 0$ and $\delta_1(P_6, P_1, P_2) = 0$. Moreover, by \Cref{no_delta1_delta1} and by possibly relabelling the points, we may assume that $\delta_2(P_1, P_2, P_3, P_4, P_5)=0$. The saturation of the ideal generated by the four polynomials
above by the condition that $P_1$, $P_2$, and $P_4$ are not aligned equals the ideal generated by
$s_{12}, s_{14}$, which also contains~$s_{16}$. This shows that
$s_{12} = s_{14}=s_{16}=0$. In particular, we see that $P_1 = P_2 \times P_4$. 

To prove \eqref{cndC5}, we take $P_2$ and $P_4$ generic,
$P_1 = P_2 \times P_4$ and $P_3, P_5, P_6$ generic points on the lines
$P_1 \vee P_2, P_1 \vee P_4$ and $P_2 \vee P_4$, respectively. The matrix
$M = \Phi(P_1, \dots, P_6)$ must have rank $9$ (or smaller) but the computations
of all the order $10$-minors of $M$ takes too much time, so we need another
strategy to find when $\rk (M) < 10$: we select
only one order~$10$ minor of~$M$; it is a polynomial which
factorizes into
several factors (many of them consequence of the specific minor chosen).
We select the factor which involves all the variables used in the definition
of the points (call it $f$) and we claim it is zero if and only if the
rank of $M$
is smaller than $10$. In order to see this, we specialize the point $P_1$
into the point $(1:0:0)$. In this situation the order $10$ minors of $M$
specialized on this point $P_1$ are easy to compute and it is easy to
verify that they are zero if and only if the polynomial $f$, specialized
on the point $(1:0:0)$ is zero. To complete the proof of the claim, we should
also take for $P_1$ the point $(1:i:0)$, but this is not necessary, since
the configuration of the points implies that $\scl{P_1}{P_1}=0$ is not
possible. A manipulation of the polynomial $f$ gives that it given by
$((s_{26}(s_{45}s_{13}-s_{34}s_{15})+s_{46}(s_{25}s_{13}-s_{23}s_{15}))/s_{11}$
hence the points $P_1, \dots, P_6$ are eigenpoints iff~(\ref{cndC5}) is
satisfied. 

If in~(\ref{cndC5}) in place of $P_6$ we substitute $w_1P_2+w_2P_4$,
we find $w_1$ and $w_2$ which give the formula~(\ref{p6formula}) for $P_6$.

Finally, to prove \eqref{p7formula}, we observe that if we change $P_3$ with $P_7$ in formula \eqref{cndC5}, and taking into account that $P_7$ is collinear with $P_1$ and $P_6$, we get the claim.



%If we exchange $P_2$ with $P_6$ and $P_3$ with $P_7$ in the configuration
%$(C_5)$, we see that $P_7$ is given from~(\ref{p7formula}), exchanging
%$P_2$ and $P_6$. 
%In this way we get the formula~(\ref{p7formula}) for
%$P_7$.

To prove the converse, we observe that $P_1= P_2 \times P_4$ implies
$s_{12}=s_{14}=0$, so $\delta_2 (P_1,P_2,P_3,P_4,P_5)=0$. It follows that the corresponding unique cubic having such $5$ points as eigenpoints has $P_6$ and $P_7$ aligned with $P_1$.
\end{proof}

\begin{rmk}
As shown in the previous proof,
    in a $(C_5)$ configuration we
have the also following relations among the points:
\[
s_{12} = 0, \quad s_{14} = 0, \quad s_{16} = 0.
\]
Moreover, it is
immediate to verify that the line $P_2\vee P_4$ is orthogonal to the
lines $P_1\vee P_2$, $P_1\vee P_4$ and $P_1\vee P_6$, that is
\[
\scl{P_1\times P_2}{P_2\times P_4} = 0, \quad
\scl{P_1\times P_4}{P_2\times P_4} = 0, \quad
\scl{P_1\times P_6}{P_2\times P_4} = 0.
\]
\end{rmk}


The above proposition gives a necessary condition in order to have
$P_1, \dots, P_7$ eigenpoints of a cubic curve. The next proposition
completes the results.

\begin{rmk}
The family of all the cubics with the eigenpoints in configuration $(C_5)$
are therefore obtained taking $P_2, P_4$ generic, $P_1 = P_2 \times P_4$,
$P_5$ generic on the line $P_1\vee P_4$ and $P_3$ given by the
formula~(\ref{p3formula}).
It is easy to see that the cubics with eigenpoints in configuration $(C_5)$
described in~\Cref{prop:rk8_2B} are a sub-family of the cubics here described,
obtained by the conditions $s_{22} = s_{44} = 0$.
\end{rmk}

%% conti nel file config5.sage.

\underline{Configuration (6)} Is not realizable.
\verb+file config6.sage+

%% conti nel file config6.sage.


\underline{Configuration (7)} Is not realizable.
\verb+file config7.sage+

%% conti nel file config7.sage.


\underline{Configuration (8)} Is realizable.
\verb+file config8N.sage+

We will use the following notation:
if $P_1, P_2, P_3$ are three not aligned, distinct points in the plane,
we set
\[
\Omega(P_1, P_2, P_3) = (P_1 \times P_2)s_{13}s_{23} -
  s_{12}(P_1 \times P_3)s_{23} + s_{12}s_{13}(P_2 \times P_3).
\]

\begin{lemma}
  Suppose $P_1, \dots, P_7$ are seven distinct points of the plane with the
  following alignments:
\begin{equation}
  \label{C8_alignments}
(P_1, P_2, P_5), (P_1, P_3, P_7), (P_1, P_4, P_6), (P_2, P_4, P_7),
(P_2, P_3, P_6),(P_3, P_4, P_5)
\end{equation}
and suppose $s_{13}s_{23} \not = 0$ or $s_{12}s_{23} \not = 0$ or
$s_{12}s_{23} \not = 0$.
Then
\begin{gather}
P_4 = \Omega(P_1, P_2, P_3)
  \label{C8:cnd1}
\end{gather}
if and only if
\begin{gather}
  \label{C8:cnd2}
  \scl{P_1\times P_4}{P_2 \times P_3} = 0,\
  \scl{P_1\times P_3}{P_2 \times P_4} = 0,\
  \scl{P_1\times P_2}{P_3 \times P_4} = 0.
\end{gather}
If $s_{12} = s_{13} = s_{23} = 0$ then, for any $P_4$, we have
$s_{24} s_{34} \not = 0$ and~(\ref{C8:cnd2}) is equivalent to
$P_1 = \Omega(P_2, P_3, P_4)$ (or $P_2 = \Omega(P_1, P_3, P_4)$
or $P_3 = \Omega(P_1, P_2, P_4)$)
\end{lemma}
\begin{proof}
%% file proofLemma8alignments.sage
Assume that at least one of the coefficients $s_{13}s_{23}, s_{12}s_{23}$
or $s_{12}s_{13}$ is not zero. Hence $\Omega(P_1, P_2, P_3)$ defines a
projective point and assume it is the point $P_4$. The points $P_5, P_6, P_7$
are determined by the alignments and it is easy to verify
that~(\ref{C8_alignments}) holds.
Conversely, fix three generic points $P_1, P_2, P_3$, take
$P_6 = u_1P_1+u_2P_3$, $P_4 = v_1P_1+v_2P_6$ and therefore take
$P_5 = (P_1+P_2) \cap (P_3+P_4)$ and $P_7 = (P_1+P_3) \cap (P_2+P_4)$.
In our hypotheses, $\Omega(P_1, P_2, P_3)$ is not $0$,
so defines a projective point $Q_4$. Consider the ideal $J$ generated
by $\scl{P_1\times P_4}{P_2\times P_3}$,
$\scl{P_1\times P_3}{P_2\times P_4}$,
$\scl{P_1\times P_2}{P_3\times P_4}$. A direct computation shows that
the order two minors of the matrix whose rows are $P_4$ and $Q_4$ are all
zero modulo $J$, therefore $P_4 = Q_4$ as projective points.
If $s_{12} = s_{13} = s_{23} = 0$, then $\Omega(P_1, P_2, P_3) = 0$,
but, since $P_2 \times P_3$, $P_2 \times P_4$, $P_3 \times P_4$ are linearly
independent, if $\Omega(P_2, P_3, P_4) = 0$, then $s_{24}s_{34}$ would be
zero, but this is not possible, since $P_4$ is not on the line $P_1\vee P_3$
neither on $P_1\vee P_2$. Hence, from the first part of the lemma, we have the
thesis. A similar argument holds for the other cases.
\end{proof}
%
The above lemma allows to understand when a cubic has the eigenpoints
aligned according to~(\ref{C8_alignments}).
%
\begin{prop}
  Let $\mathcal{C}$ be a cubic of the plane with the $7$ eigenpoints
  aligned according to~(\ref{C8_alignments}).
  Then the relations~(\ref{C8:cnd2}) are satisfied.
  Conversely, if $P_1, P_2, P_3$ are three points of the plane
  and if $s_{12}s_{13} \not = 0$ or $s_{12}s_{23} \not = 0$ or
  $s_{12}s_{23} \not = 0$, then let $P_4 = \Omega(P_1, P_2, P_3)$
  and let $P_5 = (P_1\vee P_2) \cap (P_3\vee P_4)$,
  $P_6 = (P_1\vee P_4) \cap (P_2\vee P_3)$, $P_7 = (P_1\vee P_3) \cap (P_2\vee P_4)$.
  Then the points $P_1, \dots, P_7$ satisfy the alignments given
  in~(\ref{C8_alignments}) and there exists a cubic $\mathcal{C}$
  of the plane whose eigenpoints are $P_1, \dots, P_7$. Moreover, if
  we have that among $s_{12}, s_{13}, s_{23}$ two of them are zero,
  then we can choose for $P_4$ a generic point of the plane and if we construct
  $P_5, P_6, P_7$ as above, there exists a cubic $\mathcal{C}$ of the
  plane with the eigenpoints $P_1, \dotsc, P_7$ if and only if
  all the three scalar product $s_{12}, s_{13}, s_{23}$ are zero.
\end{prop}
%
\begin{proof}
%% conti nel file config8N.sage.
We take four generic points
$P_1, \dots, P_4$ and $P_5, P_6, P_7$ such that~(\ref{C8_alignments})
is satisfied. Consider the following three $V$-configurations:
$V_1 = (P_5, P_1, P_2, P_3, P_4)$, $V_2 = (P_6, P_1, P_4, P_2, P_3)$
and $(P_7, P_1, P_3, P_2, P_4)$. If the seven points are eigenpoints,
then necessarily $\Phi(V_1)$, $\Phi(V_2)$, $\Phi(V_3)$ have rank
smaller than $10$, so, from~\Cref{delta1_and_delta2} and
from~\Cref{proposition:third_alignment}
we have that
\[
\delta_1(P_5, P_1, P_4) = 0, \quad \delta_1(P_6, P_1, P_2) = 0, \quad
\delta_1(P_7, P_3, P_4) = 0.
\]
These three equations are equivalent to
\[
s_{14}s_{23}-s_{13}s_{24} = 0, \quad s_{12}s_{34}-s_{13}s_{24} = 0,
\quad s_{12}s_{34}-s_{14}s_{23} = 0
\]
This is a linear system in the coordinates of $P_4$. The solution is
not unique iff $(s_{12} = s_{13} = 0)$ or $(s_{23} = s_{13} = 0)$
or $(s_{23} = s_{12} = 0)$. This case will be considered afterwards.
When the solution is unique, we get new coordinates for $P_4$ so we
re-define $P_4$ and the points constructed from it (i.e.\ $P_5, P_6, P_7$).
A direct computation shows that the two lines $P_1\vee P_4$ and $P_2\vee P_3$,
the two lines $P_1\vee P_3$ and $P_3\vee P_4$ and the two lines $P_1\vee P_2$, $P_3\vee P_4$
are orthogonal, hence condition~(\ref{C8:cnd2}) is satisfied,
therefore, from the
previous lemma, $P_4 = \Omega(P_1, P_2, P_3)$. The rank
of $\Phi(P_1, \dotsc, P_7)$ is less than $10$ (the computation is
not so immediate, but can be simplified considering only the
cases in which $P_1$ is one of the points $(1: 0: 0)$ or
$(1: \iii: 0)$), we get that the seven points
are the eigenpoints of a cubic. Conversely, if the seven points are
eigenpoints, the same computations as above, show that
$P_4 = \Omega(P_1, P_2, P_3)$.\\
It remains to consider the case in which $s_{12} = s_{13} = 0$ (or one of
the other two, which are similar). As above, we again compute
$\delta_1(P_5, P_1, P_4)$, $\delta_1(P_6, P_1, P_2) = 0$ and
$\delta_1(P_7, P_3, P_4)$ and we see that these polynomials are zero
iff $s_{12} = 0$.
\end{proof}

\subsection{Comparison with ODECO tensors}
As mentioned in the introduction, a class of
symmetric tensors fitting in the framework of collinearities in the eigenscheme is represented by ODECO tensors, which were introduced by~\cite{Rob} and studied by (Boralevi, Draisma, Robeva, Ottaviani ?...)

Possibly after an $\SO_3(\C)$ transformation, such forms are of the type
$$
f=\lambda_1 x^3 +\lambda_2 y^3 + \lambda_3 z^3,
$$
and they admit the three pairwise orthogonal eigenvectors
$$
Q_1=(\frac{1}{\lambda_1},0,0), \quad 
Q_2=(0,\frac{1}{\lambda_2},0), \quad
Q_3=(0,0,\frac{1}{\lambda_3}).
$$
Moreover, the remaining $4$ eigenvectors are uniquely determined, precisely:
$$
Q_1+Q_2, \ Q_1+Q_3, \ Q_2+Q_3, \ Q_1+Q_2+Q_3.
$$
In particular, we see, that such eigenschemes are in a $(C_8)$ configuration, with the additional condition
$$
\langle Q_i,Q_j \rangle = 0, \quad i \neq j.
$$

\bibliographystyle{amsalpha}
\bibliography{biblio}

\end{document}
%%%%%%%%%%%%%%%%%%%%%%%%%%%%%%%%%%%%%%%%%%
%%%%%%%%%%%%%%%%%%%%
Then we have a surjective morphism
$$
\alpha : \mathcal{U} \to \mathcal {AL},
$$
assigning to each $[F] \in \mathcal{U}$ the unique triple of eigenpoints. Over the irreducible open subset
$$
\mathcal W := \mathcal {AL}
\setminus \{(P_1,P_2,P_3)\in\mathcal {AL}
\ | \ \sigma(P_1,P_2)=0, s_{11} s_{22} s_{33}=0\}
$$
of aligned triples not lying
on a tangent line to the isotropic conic $\iso$, with tangency point one of the $P_i$'s, the fibers of $\alpha$ are projective linear systems of dimension $3$ by
Proposition \ref{manca il riferimento su ancillary  non e': condition_rank_aligned}. By the Fiber Dimension Theorem this implies that the open subset $\alpha ^{-1} (\mathcal W)$ is irreducible of dimension
$$
\dim \alpha ^{-1} (\mathcal W)
\subset \sU=8 ,
$$
so $\sL$ is a hypersurface. Next we set
$$
\Sigma:=\{(P_1,P_2,P_3)\in\mathcal {AL}
\ | \ \sigma(P_1,P_2)=0, s_{11} s_{22} s_{33}=0\}
$$
and we observe that $\Sigma \subset \overline {\mathcal {AL}}$. Being $\alpha$ a morphism we get
that
$$
\sU \subseteq \overline {\alpha^{-1} (\mathcal W)}.
$$

%that is, any $[F] \in \sU$ such that $\alpha (F)$
%satisfies $\sigma(P_1,P_2)=0$
%is a limit of $[F_t]\in \alpha^{-1} (\mathcal W)$.
%
%Without loss of generality we can assume that the tangent line $l$ to $\iso$ is given by $\iii x-y=0$ and $P_1=(1:\iii:0)$. We consider the pencil of lines $\lambda x + \mu y$ with base point $(0:0:1)$, and
%for any pair $P_2= (A_2: \iii A_2: C_2), P_3
%=(A_3: \iii A_3 :C_3)\in l$, $P_i \neq P_1$, we set
%$$
%P_1^{\lambda, \mu}=(-\mu : \lambda :0), \quad
%P_2^{\lambda, \mu}=(-\mu A_2: \lambda A_2:C_2), \quad
%P_1^{\lambda, \mu}=(-\mu A_3: \lambda A_3:C_3).
%$$
%By adding other three linear conditions, we can construct a one-dimensional family of cubics with general
%member in $\sV$ and the special one satisfying $\sigma (P_1,P_2)=0$



It follows that $\sU$ is irreducible, and therefore $\sL$ is irreducible too.
\end{proof}

\begin{prop}
The class of any ternary form $[F]$ with two or more aligned triples of eigenpoints satisfies
$
[F] \in \sL.
$

\end{prop}

\begin{proof}
Let $\sV \subset \p^9$ denote the locus of classes $[F]$ of cubic forms adimitting a $V$-configuration among its eigenpoints.

 For simplicity we shall only consider the general cubic form
 $[F] \in \sV \setminus \Delta_{3,3}$, that is the ones corresponding to a rank $9$ matrix $\Phi(P_1, \dots, P_5)$; the cases with more alignments can be treated similarly.

 Observe that all the irreducible components of $\sV \setminus \Delta_{3,3}$ are open and nonempty in $\sV$ by
 Remark \ref{rmk_delta_case3}.

If $F$ is general enough, we have
$\rk \ \Phi(P_1, \dots, P_5)=9$ and $\rk \ \Phi(P_1, \dots, P_4)=8$.
We can also assume that $\Eig{F}$ has no other collinearities.
The pencil of cubics with the assigned eigenpoints $P_1, \dots, P_4$
has a general member in $\sU$ since $[F] \not\in \Delta_{3,3}$ and $P_1,P_2,P_3$ are aligned. This proves the statement.
\end{proof}

\begin{prop}
    The locus $\sV$ of cubics that have at least two aligned triples of eigenpoints has codimension~$2$ in~$\p^9$.
\end{prop}
\begin{proof}
In the product $(\p^2)^{5}$, the locus $\sF$ of
unordered $5$-tuples of distinct points containing $4$ collinear points is closed; indeed, a $5$-uple $(P_1,\dots,P_5)$ is in $\sF$ if and only if the points do not impose independent conditions to conics, that is if and only if the $5 \times 6$ matrix
associated with the
conditions of passing through $(P_1,\dots,P_5)$ has rank less or equal to $4$.

We then consider the open subset $\mathcal {Y}:=(\p^2)^{5} \setminus \sF$.
The locus $\mathfrak{V}$ of $V$-configurations in $\sY$ has dimension $8$ and can be characterized as follows:
$$
\mathfrak{V}= \{(P_1,\dots,P_5) \in \sY \ | \ \langle P_1 \times P_2, P_3\rangle=0, \ \langle P_1 \times P_4, P_5\rangle=0\}.
$$
Finally, a $V$-configuration is contained in an eigenscheme of some polynomial if and only if $\delta_1 (P_i,P_j,P_k) \cdot \delta_2 (P_1,\dots, P_5)=0$.

By setting $\Delta_{ijk} :=\{\delta_1 (P_i, P_j,P_k)=0\}$
and $\Delta_2:=\{\delta_2=0\}$, and by considering in $\sV$ the locus
$\mathcal{V}'$ of classes $[F]$ with $\Eig{F}$ containing exactly one $V$-configuration,
we get a morphism
$$
\beta : \mathcal{V}' \to (\bigcup_{i,j,k}\Delta_{i,j,k} \cup \Delta_2) \cap \mathfrak{V} \cap \sY.
$$
By the results of Section 6, the locus with $\rk \Phi(P_1, \dots,P_5)\le 8$ is closed, so $\beta$ is generically injective and $\dim \mathcal{V}'=\dim \overline {\mathcal{V}'}=7$.
%
%%%%%%%%%%%%%%
%VECCHIA PROPOSIZIONE 9.5
\begin{prop}
\textbf{(AGGIUNGERE IPOTESI SU RANGO NON 8)} 
Let $P_1, \dotsc, P_7$ be seven points of $\mathbb{P}^2$ in configuration
$(C_5)$. Then they are eigenpoints if and only if
\begin{equation}
s_{26}(s_{45}s_{13}-s_{34}s_{15})+s_{46}(s_{25}s_{13}-s_{23}s_{15}) = 0.
\label{cndC5}
\end{equation}
In this case, the points~$P_3$ and~$P_7$ are then given by the following formulas:
%
\begin{eqnarray}
P_3 & = & (s_{12}(s_{26}s_{45}+s_{46}s_{25})-s_{26}s_{15}s_{24}-s_{46}s_{15}s_{22})P_1-
(s_{11}(s_{26}s_{45}+s_{46}s_{25})-s_{26}s_{15}s_{14}-s_{46}s_{15}s_{12})P_2
\label{p3formula}\\
P_5 & = & (s_{12}s_{13}s_{44}-s_{14}^2 s_{23})P_1
+s_{14}(s_{11} s_{23}-s_{12}s_{13}) P_4 \\
P_6 & = & (-s_{15}s_{24}s_{34}-s_{15}s_{23}s_{44} + s_{13}s_{25}s_{44} +s_{13}s_{24}s_{45})P_2 + (-s_{13}s_{24}s_{25}+2s_{15}s_{22}s_{34}-s_{13}s_{22}s_{45})P_4 ,\\
P_7 & = & (s_{16}(s_{26}s_{45}+s_{24}s_{56})-s_{26}s_{15}s_{46}-s_{24}s_{15}s_{66})P_1-
(s_{11}(s_{26}s_{45}+s_{24}s_{56})-s_{26}s_{15}s_{14}-s_{24}s_{15}s_{16})P_6 ,
\label{p7formula}
\end{eqnarray}
%
\end{prop}
\begin{proof}
\end{proof}

\begin{prop}
