\documentclass[11pt, a4paper, reqno, captions=tableheading,bibliography=totoc]{scrartcl}

\usepackage[utf8]{inputenc}
\usepackage[T1]{fontenc}
\usepackage{mathtools}
\usepackage{amsthm}
\usepackage{amsmath}
\usepackage{amssymb}
\usepackage{lmodern}
\usepackage[english]{babel}
\hyphenation{ei-gen-scheme}
\usepackage{booktabs}
\usepackage{float}
\usepackage{enumitem}
\usepackage{tikz,pgf}
%\usepackage[utopia]{mathdesign}
\usepackage{palatino}
\usepackage{hyperref}
\hypersetup{
    colorlinks,
    linkcolor={red!50!black},
    citecolor={blue!50!black},
    urlcolor={blue!80!black}
}
\usepackage[nameinlink]{cleveref}
\usepackage{xcolor}

\theoremstyle{plain}
\newtheorem{lemma}{Lemma}[section]
\newtheorem{prop}[lemma]{Proposition}
\newtheorem{theorem}[lemma]{Theorem}
\newtheorem{corollary}[lemma]{Corollary}
\newtheorem{conjecture}[lemma]{Conjecture}
\newtheorem{fact}[lemma]{Fact}
\newtheorem{assumption}[lemma]{Assumption}
\newtheorem*{reduction}{Reduction}
\theoremstyle{definition}
\newtheorem{definition}[lemma]{Definition}
\newtheorem{es}[lemma]{Example}
\newtheorem*{notation}{Notation}
\newtheorem{rmk}[lemma]{Remark}


\newcommand{\N}{\mathbb{N}}
\newcommand{\Z}{\mathbb{Z}}
\newcommand{\Q}{\mathbb{Q}}
\newcommand{\R}{\mathbb{R}}
\newcommand{\C}{\mathbb{C}}
\newcommand{\p}{\mathbb{P}}
\newcommand{\sP}{\mathcal{P}}
\newcommand{\sL}{\mathcal{L}}
\newcommand{\de}{\partial}
\newcommand{\codim}{\mathrm{codim}}

\newcommand{\oo}{\mathcal{O}}
\newcommand{\Bl}{\mathrm{Bl}}

\newcommand{\iso}{\mathcal{C}_{\mathrm{iso}}}

\newcommand{\imunit}{i}

\newcommand{\SO}{\operatorname{SO}}
\newcommand{\Eig}[1]{\operatorname{Eig}\left( {#1} \right)}
\newcommand{\polq}{{\rm Pol}_Q}
\newcommand{\comment}[1]{}

\newcommand{\scl}[2]{\left\langle {#1}, {#2} \right\rangle}

\newcommand\scalemath[2]{\scalebox{#1}{\mbox{\ensuremath{\displaystyle #2}}}}

\newcommand{\iii}{\textbf{i}}
\definecolor{MyDarkGreen}{cmyk}{0.7,0,1,0}
\newcommand{\blue}[1]{{\color{blue}  [#1]}}
\newcommand{\cbc}{\ensuremath{Cbc}}
\newcommand{\rk}{\ensuremath{\mathrm{rk}}}


\title{Eigenpoint collinearities of plane cubics}
\author{}
\date{}

\linespread{1.15}
\setlength{\parindent}{0pt}
\setlength{\parskip}{.25em}

\begin{document}

\maketitle

\section{Introduction}
\label{introduction}

However, there are several examples of cubic polynomials, whose eigenscheme contains one or more triples of aligned points, as for instance the Fermat cubic polynomial.

The goal of this paper is to classify all the situations when we have one or more triples of aligned points inside a zero-dimensional reduced eigenscheme of a cubic plane curve.

\Cref{aligned} clarifies the reasons of interest in the situation of triples of aligned eigenpoints.


\section{Aligned eigenpoints of ternary cubic form}
\label{aligned}

We recall that, given a homogeneous form $f \in \C[x,y,z]_d$ of degree~$d$, the eigenscheme~$\Eig{f}$ of~$f$ is the determinantal scheme defined by the $2 \times 2$ minors of the matrix
%
\begin{equation}
\label{eq:def_matrix}
    \begin{pmatrix}
    x & y & z \\
    \de_x f & \de_y f & \de_z f
    \end{pmatrix}.
\end{equation}
%
Since the eigenschemes of two proportional homogeneous forms are the same,
if $C$ is the curve defined by these forms,
we can write $\Eig{C}$ for such eigenscheme and hence talk about the eigenscheme of a plane curve.

The eigenscheme of a general ternary cubic form has no aligned triples of points. This is a consequence of the geometric properties of the classical Geiser map associated with seven points in the plane, and has been proved in \cite[Proposition~4.5]{BGV}.

Moreover, whenever the eigenscheme of a ternary form is zero-dimensional, it never cointains $4$ or more aligned points.
To see this, we recall a property of eigenschemes of ternary forms, namely,
that when zero-dimensional they are somehow ``general'' with respect to conics.

\begin{lemma}
\label{lemma:no_six_conic}
Let $f \in \C[x,y,z]_d$ be a homogeneous form of degree~$d$.
If $E(f)$ is zero-dimensional,
then no degree six subscheme of~$E(f)$ lies on a conic.
\end{lemma}
\begin{proof}
See \cite[Lemma~9.1]{OS1} for the reduced case.
The proof works also in the non-reduced case.
\end{proof}

\begin{corollary}
\label{corollary:general_no_triple}
As a consequence of \Cref{lemma:no_six_conic}, a zero-dimensional eigenscheme~$\Eig{f}$ for $f \in \C[x,y,z]_d$ never contains $4$ or more aligned points.
Moreover, if $\Eig{f}$ contains two triples of aligned points, those must share a point.
\end{corollary}

\section{Invariance under the action of orthogonal matrices}
\label{invariance}

In what follows, it will be useful to fix particular coordinates for points and lines related to eigenschemes. To do that, we employ a property of invariance of eigenschemes with respect to the action of the following group.

\begin{definition}
 We define $\mathrm{SO}_3(\mathbb{C})$ to be the complexification of the group of special orthogonal real matrices, namely
 %
 \[
  \mathrm{SO}_3(\mathbb{C}) :=
  \bigl\{
   M \in \mathrm{GL}_3(\C) \, \mid \,
   M M^t = I_3 \  \text{and} \  \det(M) = 1
  \bigr\} \,.
 \]
 %
 The group $\mathrm{SO}_3(\mathbb{C})$ acts on $\C^3$ by matrix multiplication:
 %
 \[
  \begin{array}{ccc}
   \mathrm{SO}_3(\mathbb{C}) \times \C^3 & \rightarrow & \C^3 \\
   (M, v) & \mapsto & Mv
  \end{array}
 \]
 %
 Since all the elements of $\mathrm{SO}_3(\mathbb{C})$ are invertible, the latter action descends to an action on $\p^2(\C)$.

 Moreover, the group~$\mathrm{SO}_3(\mathbb{C})$ acts also on ternary forms via
 \[
  M \cdot f (x,y,z) = f(M^{-1} \cdot \prescript{t} {}( x \ y \ z )  ).
 \]
\end{definition}

\begin{prop}
\label{two_orbits}
 The action of $\mathrm{SO}_3(\mathbb{C})$ on $\p^2(\C)$ has two orbits:
 %
 \begin{align*}
  \mathcal{O}_1 &:=
  \bigl\{
   P \in \p^2(\C) \, | \,
   P = (a:b:c) \  \text{with} \  a^2 + b^2 + c^2 = 0
  \bigr\} \\
  \mathcal{O}_2 &:= \p^2(\C) \setminus \mathcal{O}_1
 \end{align*}
 %
 A representative for $\mathcal{O}_1$ is $(1:\iii:0)$ and a
representative for $\mathcal{O}_2$ is $(1:0:0)$.
The orbit $\mathcal{O}_1$ is the set of points of the so-called \emph{isotropic conic}, denoted $\iso$.
\end{prop}
\begin{proof}
 Suppose that $P \in \p^2(\C)$ and $P = (a:b:c)$ with $a^2 + b^2 + c^2 = 0$.
 We produce a matrix $M \in \mathrm{SO}_3(\C)$ such that $M \left(\begin{smallmatrix} 1 \\ \iii \\ 0 \end{smallmatrix}\right)$ and $\left(\begin{smallmatrix} a \\ b \\ c \end{smallmatrix}\right)$ are proportional.
 Up to relabeling the coordinates, we can suppose that $a \neq 0$.
 Hence, by rescaling the coordinates of $P$, we have $P = (1: b: c)$ with $b^2 + c^2 = -1$.
 One can check that the matrix
 %
 \[
  M :=
  \begin{pmatrix}
   -1 & 0 & 0 \\
   0 & \iii b & -\iii c \\
   0 & \iii c & \iii b
  \end{pmatrix}
 \]
 %
 satisfies the requirements.

 Now suppose that $P \in \p^2(\C)$ and $P = (a:b:c)$ with $a^2 + b^2 + c^2 \neq 0$.
 Up to rescaling, we can suppose that $a^2 + b^2 + c^2 = 1$.
 Again, we produce a matrix $M \in \mathrm{SO}_3(\C)$ such that $M \left(\begin{smallmatrix} 1 \\ 0 \\ 0 \end{smallmatrix}\right)$ and $\left(\begin{smallmatrix} a \\ b \\ c \end{smallmatrix}\right)$ are proportional.
 First of all, suppose that $b^2 + c^2 \neq 0$ and let $\omega$ be a root of the polynomial $t^2 - (b^2 + c^2)$ in $\C[t]$.
 Then, the matrix
 %
 \[
   M :=
   \begin{pmatrix}
     a & \omega & 0 \\
     b & -\frac{ab}{\omega} & \frac{c}{\omega} \\
     c & -\frac{ac}{\omega} & -\frac{b}{\omega}
   \end{pmatrix}
 \]
 %
 satisfies the requirements.
 With the same technique, if $a^2 + c^2 \neq 0$, we can produce a matrix $M \in \mathrm{SO}_3(\C)$ that maps $\left(\begin{smallmatrix} 0 \\ 1 \\ 0 \end{smallmatrix}\right)$ to $\left(\begin{smallmatrix} a \\ b \\ c \end{smallmatrix}\right)$; similarly, when $a^2 + b^2 \neq 0$, we can map $\left(\begin{smallmatrix} 0 \\ 0 \\ 1 \end{smallmatrix}\right)$ to $\left(\begin{smallmatrix} a \\ b \\ c \end{smallmatrix}\right)$.
 Since $\left(\begin{smallmatrix} 1 \\ 0 \\ 0 \end{smallmatrix}\right)$, $\left(\begin{smallmatrix} 0 \\ 1 \\ 0 \end{smallmatrix}\right)$, and $\left(\begin{smallmatrix} 0 \\ 0 \\ 1 \end{smallmatrix}\right)$ are all $\mathrm{SO}_3(\C)$-equivalent, the only case to consider is when
 %
 \[
  b^2 + c^2 = a^2 + c^2 = a^2 + b^2 = 0 \,,
 \]
 %
 which, however, can never occur.
\end{proof}

The following result is well known; we recall it for the sake of completeness.

\begin{prop}
 Let $M \in \mathrm{GL}_3(\C)$ and let $f$ be a ternary cubic.
 Let $P = (A: B: C)$ be a point in~$\p^2$.
 Then we have
 %
 \[
  P \in \Eig{f} \iff M \cdot \prescript{t} {}(A \ B \ C) \in \Eig{M \cdot f}.
 \]
 %
\end{prop}
\begin{proof}
In the present proof,
for convenience, we shall consider the transpose of the defining matrix of an eigenscheme.

A point $P = (A: B: C)$ is an eigenpoint for~$f$ if and only if
\begin{equation*}
  \mathrm{rk}  \begin{pmatrix}
    A & \de_x f(P) \\
    B & \de_y f(P)  \\
    C & \de_z f(P)
    \end{pmatrix}=1,
\end{equation*}
which is equivalent to
\begin{equation}
\label{eq:def_matrix_M}
    \mathrm{rk} \quad  M  \cdot \begin{pmatrix}
    A & \de_x f(P) \\
    B & \de_y f(P)  \\
    C & \de_z f(P)
    \end{pmatrix}
    =1.
\end{equation}
By setting $\prescript{t} {}(A' \ B' \ C' )= M \cdot \prescript{t} {}(A \ B \ C) $ and $Q=(A':B':C')$, we have that \Cref{eq:def_matrix_M}
is equivalent to
%
\begin{equation}
\label{eq:transformed}
  \rk
  \begin{pmatrix}
    A' &  \\
    B' & M \cdot \nabla f (P) \\
    C' & \\
  \end{pmatrix}=1.
\end{equation}
%
Now we consider the polynomial $M \cdot f$ and we observe that the chain rule gives
%
\begin{gather*}
\partial_x (M\cdot f) = \partial_x \bigl( f(M^{-1}  \ \prescript{t} {} (x \ y \ z)) \bigr) = \prescript{t} {}(M^{-1})^{(1)}(\nabla f) \bigl( M^{-1}\   \prescript{t} {} (x \ y \ z) \bigr) \,, \\
\partial_y (M\cdot f) = \partial_y \bigl( f(M^{-1}  \ \prescript{t} {} (x \ y \ z)) \bigr) = \prescript{t} {}(M^{-1})^{(2)}(\nabla f) \bigl( M^{-1}\   \prescript{t} {} (x \ y \ z) \bigr) \,, \\
\partial_z (M\cdot f) = \partial_z  \bigl( f(M^{-1}  \ \prescript{t} {} (x \ y \ z)) \bigr) = \prescript{t} {}(M^{-1})^{(3)}(\nabla f) \bigl( M^{-1}\   \prescript{t} {} (x \ y \ z) \bigr) \,,
\end{gather*}
%
where $(M^{-1})^{(j)}$ denotes the $j$-th column of the matrix $M^{-1}$. Hence
%
\[
\nabla (M \cdot f) = \prescript{t} {} M^{-1} \cdot (\nabla f) (M^{-1}\   \prescript{t} {} (x \ y \ z)),
\]
%
so we have
%
\[
\nabla (M \cdot f)(Q)=\nabla (M \cdot f)(M \cdot P)=
\prescript{t} {} M^{-1} \cdot (\nabla f) (M^{-1}\   M \cdot P)=\prescript{t} {} M^{-1} \cdot (\nabla f)(P).
\]
%
Finally, if we choose $M \in \mathrm{SO}_3(\mathbb{C})$, we have
$\prescript{t} {} M^{-1}=M$. We deduce that
\Cref{eq:transformed} holds if and only if $Q \in E(M\cdot f)$, so the statement is proved.
\end{proof}

\section{Conditions imposed by aligned eigenpoints}
\label{conditions}

\subsection{The matrix of conditions}

Imposing a cubic ternary form to have one or more aligned triples of eigenschemes implies conditions both on the points and on the cubics.
We begin to explore these conditions by introducing, for each point
$P \in \p^2$,
a $3 \times 10$ matrix encoding the condition that $P$ is an eigenpoint of a ternary cubic.

\begin{definition}
\label{definition:matrix_conditions}
 Consider $\p^9 = \p(\C[x,y,z]_3)$, the space of all ternary cubics.
 Through this paper, we fix the following vector basis for $\C[x,y,z]_3$:
 \begin{eqnarray}
  \mathcal{B} = (x^3, x^2 y, x y^2, y^3, x^2 z, x y z, y^2 z, x z^2, y z^2, z^3)
  \label{vector_basis}
 \end{eqnarray}
 For $f \in \C[x,y,z]_3$, denote by $[f]$ the corresponding point in~$\p^9$; we denote by $w_f$ the (column) vector of coordinates of~$f$ with respect to the basis above; then $w_f$ is also a vector of projective coordinates of~$[f]$.
 For a point $P \in \p^2$ with coordinates $(A: B: C)$, the condition on elements~$[f]$ of~$\p^9$ that $P$ is an eigenpoint of the ternary cubic form~$f$ can be expressed in the form
 %
 \[
  \Phi(P) \cdot w_f
  = 0 \,,
 \]
 %
 where $\Phi(P)$ is a $3 \times 10$ matrix with entries depending on $A, B, C$.
 The matrix $\Phi(P)$ is called the \emph{matrix of conditions} imposed by~$P$.
We denote by $\phi_1(P)$, $\phi_2(P)$, and~$\phi_3(P)$ the rows of~$\Phi(P)$.
Written as vectors, they are
%
\begin{equation}
\label{equation:matrix_conditions_rows}
\begin{gathered}
\scalemath{0.9}{(-3A^2B, A(A^2 - 2B^2), B(2A^2 - B^2), 3AB^2,
 -2ABC, C(A^2 - B^2), 2 ABC,
 -B  C^2, A  C^2, 0)} \,, \\
\scalemath{0.9}{(-3A^2 C,
-2ABC,
-CB^2,
0,
A(A^2-2C^2),
B(A^2 - C^2),
AB^2,
C(2A^2-C^2),
2ABC,
3AC^2)} \,,\\
\scalemath{0.9}{(0,
-A^2C,
-2ABC,
-3CB^2,
A^2 B,
A(B^2 - C^2),
B(B^2-2C^2),
2ABC,
C(2B^2-C^2),
3BC^2)} \,.
\end{gathered}
\end{equation}
%
Moreover, if $P_1, \dotsc, P_n$ are points in the plane, we denote by $\Phi(P_1, \dotsc, P_n)$ the matrix
%
\[
 \left(
 \begin{array}{c}
  \Phi(P_1) \\
  \vdots \\
  \Phi(P_n)
 \end{array}
 \right)
\]
%
namely, the $3n \times 10$ matrix obtained by vertically stacking the matrices of conditions of~$P_1, \dotsc, P_n$.
\end{definition}

We introduce a new piece of notation that makes the description of $\Phi(P)$ more compact.
Let us consider $a_0, \dotsc, a_9$ as formal variables; given a formal linear combination
%
\[
 \Lambda = a_0\lambda_0+\cdots+a_9\lambda_9 \,,
\]
%
denote by $v(\Lambda)$ the $10$-component vector given by
$(\lambda_0, \dots, \lambda_9)$. In particular, if $X=(x: y: z)$ and
$f(X) = a_0x^3+a_1x^2y+\cdots+ a_9z^3$ is a degree $3$ homogeneous polynomial, then
$v \bigl(f(X)\bigr) = (x^3, x^2y, \dots, z^3)$, while $v \bigl(\de_x(f)(X) \bigr) =
(3x^2, 2xy, \dots, 0)$.
We get that $f(X) = v\bigl(f(X)\bigr) \cdot w_f$.
With this notation, if $P=(A: B: C)$, we get:
%
\begin{equation}
\label{equation:vector_conditions}
\begin{gathered}
\phi_1(P) = A\cdot v(f_x)(P)-B\cdot v(f_y)(P) \,, \\
\phi_2(P) = A\cdot v(f_z)(P)-C\cdot v(f_y)(P) \,, \\
\phi_3(P) = B\cdot v(f_z)(P)-C\cdot v(f_y)(P) \,.
\end{gathered}
\end{equation}
%
\begin{rmk}
By analyzing the entries of \Cref{equation:matrix_conditions_rows}, it is not difficult to check that the rank of $\Phi(P)$ is never $\leq 1$. \Cref{equation:vector_conditions} gives
\begin{equation}
  C \, \phi_1(P) - B \, \phi_2(P) + A \, \phi_3(P) = 0 \,,
  \label{eq:base}
\end{equation}
therefore,
among the three vectors, at most two of them are linearly independent and the matrix~$\Phi(P)$ has rank~$2$.
\end{rmk}

As a consequence, if $P_1, \dots, P_n$ are $n$ points of the plane, we have:
\begin{equation}
\label{bound_rank}
\rk  \  \Phi(P_1, \dots, P_n) \leq \min \left\{2n, 10 \right\}
\end{equation}

\subsection{Possible ranks of the matrix of conditions}

In what follows we want to study the possible values of the rank of the matrix
$\Phi(P_1, \dots, P_n)$ for several configurations of points $P_1, \dots, P_n$
(and several values of $n$).
In particular, we will study the ideal~$J_k$ of order $k$ minors of the
involved matrix and we will deduce some bounds about the rank from the possible
decompositions of the ideal~$J_k$. Most of these computations will be done
with the aid of a computer algebra system. Nevertheless, in many cases,
the result cannot be reached just by brute force, but it is necessary to
make some preprocessing on the ideal~$J_k$. In particular, it turns out that
it is often convenient to first saturate the ideal~$J_k$ with respect to
the conditions that the
points are distinct or that three of them are not aligned (when this is the
case). Another important simplification that we adopt sometimes, makes use
of the action of $\SO_3(\C)$: thanks to it we can assume that one of
the point is either $(1: 0: 0)$ or $(1: \iii: 0)$; see \Cref{two_orbits}.

We start with the following lemma, which will be extremely useful
to speed up the computations.

\begin{lemma}
\label{lemma:minors}
Let $l_1 < \cdots <l_n$ be $n$ indices (where $3 \leq n \leq 10$) and let $P = (A: B: C)$ be a point of the plane.
Construct three $1 \times n$ matrices $w_1$, $w_2$, $w_3$ by extracting the entries of position $l_1, \dotsc, l_n$ from $\phi_1(P)$, $\phi_2(P)$, and~$\phi_3(P)$, respectively. If $L$ is a $(n-2) \times n$ matrix, set:
  \[
  L_1 := \left(\begin{array}{c}w_1 \\ w_2 \\ L\end{array}  \right), \quad
  L_2 := \left(\begin{array}{c}w_1 \\ w_3 \\ L\end{array}  \right), \quad
  L_3 := \left(\begin{array}{c}w_2 \\ w_3 \\ L\end{array}  \right)
  \]
  Then
  \[
  B \det(L_1) = A \det(L_2), \quad
  C \det(L_1) = A \det(L_3), \quad
  C \det(L_2) = B \det(L_3)
  \]
  hence $(A: B: C) = \bigl( \det(L_1): \det(L_2): \det(L_3) \bigr)$.
\end{lemma}
\begin{proof}
  The thesis easily follows from the equality $C w_1 - B w_2 + A w_3 = 0$, which is a direct consequence of \Cref{eq:base}.
\end{proof}

\Cref{lemma:minors} justifies the following choice.

\begin{definition}
 \label{definition:reduced_matrix_conditions}
 For $n$ points $P_1, \dotsc, P_n$ in the plane, the \emph{cut matrix of conditions} of $P_1, \dotsc, P_n$ is the submatrix of~$\Phi(P_1, \dotsc, P_n)$ whose rows are $\phi_1(P_1), \phi_2(P_1), \dotsc, \phi_1(P_n), \phi_2(P_n)$.
\end{definition}

Next, we point out a property of the lines, which are tangent to~$\iso$.

\textbf{RIVEDERE BUONA DEFINIZIONE!!!}
\begin{definition}
\label{definition:sigma}
We define a bihomogenous polynomial of bidegree~$(2,2)$ on $\p^2 \times \p^2$: for $P_1 = (A_1: B_1: C_1)$ and $P_2 = (A_2: B_2: C_2)$, we set
%
\[
  \sigma(P_1, P_2) := \scl{P_1}{P_1} \scl{P_2}{P_2} - \scl{P_1}{P_2}^2 \,,
\]
%
where
%
\[
 \scl{P_1}{P_2} = A_1 A_2 + B_1 B_2 + C_1 C_2
\]
%
and similarly for the other quantities. The form~$\sigma$ is the discriminant of the intersection between the line~$r$ through~$P_1$ and~$P_2$ and~$\iso$.
\end{definition}

\begin{prop}
\label{proposition:sigma_tangency}
  Let $P_1$, $P_2$ be two distinct points in the plane and let $r$ be the line joining them.
  Then the following are equivalent:
  \begin{enumerate}
  \item $\sigma(P_1, P_2) = 0$;
  \item $\sigma(Q_1, Q_2) = 0$ for all pairs of distinct points $Q_1, Q_2 \in r$;
  \item the line $r$ is tangent to~$\iso$ at some point;
  \item there exists a point $T \in r$ such that  $T \in \iso$ and $\scl{T}{Q} = 0$ for all $Q \in r$, $Q \neq T$; in this case we say that $T$ is \emph{orthogonal} to $r$.
  \end{enumerate}
\end{prop}
\begin{proof}
  By recalling the characterization of~$\sigma$ as a discriminant, we have that $\sigma(P_1, P_2) = 0$ if and only if $r$ is tangent to~$\iso$ in a point~$T$; this shows that the first three items are equivalent.

  Now, if $\sigma(P_1, P_2) = 0$, then $r$ is tangent to~$\iso$ at a point~$T$, hence $\scl{T}{T} = 0$. Moreover, we also have that $\sigma(T, Q) = 0$ for all $Q \in r$ with $Q \neq T$. Hence, $T$ is orthogonal to~$r$.
  The converse is immediate.
\end{proof}

\begin{prop}
\label{proposition:three_aligned_ranks}
Let $P_1, P_2, P_3$ be three distinct aligned points of the plane and let
$r$ be the line passing through them. Then:
\begin{itemize}
\item $5 \leq \rk \ \Phi(P_1, P_2, P_3) \leq 6$;
\item
$\rk \ \Phi(P_1, P_2, P_3) = 5$ if and only if $r$ is tangent
to~$\iso$ in one of the three points $P_1, P_2$, or $P_3$.
\end{itemize}
\end{prop}
\begin{proof} We denote the coordinates of the points as follows:
%
\[
P_1 = (A_1: B_1: C_1) \,, \quad P_2 = (A_2: B_2: C_2) \,, \quad P_3 = u_1P_1+u_2P_2 \,.
\]
%
The characterization of the triples $P_1, P_2, P_3$ of distinct points such that $\rk \ \Phi(P_1, P_2, P_3) < 6$ follows from the computation of a suitable saturation of the ideal generated by the $17640$ order six minors of $\Phi(P_1, P_2, P_3)$.
We claim that we can greatly simplify the computations.

Indeed, consider the matrix~$M$ constructed with the following six rows:
\[
\phi_{i_1}(P_1), \ \phi_{i_2}(P_1), \ \phi_{j_1}(P_2),\  \phi_{j_2}(P_2),
\phi_{k_1}(P_3), \ \phi_{k_2}(P_3)
\]
where $i_1 \not= i_2 \in \{1, 2, 3\}$; $j_1 \not= j_2 \in \{1, 2, 3\}$;
$k_1 \not= k_2 \in \{1, 2, 3\}$.
First, assume that $i_1=j_1=k_1=1$ and
$i_2=j_2=k_2=2$. Fix six columns
$1\leq l_1 < \cdots < l_6 \leq 10$ of $M$ and let $N$ be the order $6$ matrix
obtained from $M$ with these columns. Since two rows of $N$ are obtained from
entries of $\phi_1(P_1)$ and of $\phi_2(P_1)$,
\Cref{lemma:minors} gives that $A_1$ divides $\det(N)$. For a similar
reason, also $A_2$ and $u_1A_1+u_2A_2$ divide $\det(N)$, so
$\det(N) = A_1A_2(u_1A_1+u_2A_2)\cdot D$ for a suitable $D$.
Again by \Cref{lemma:minors}, we also see that if we take different values of
$i_1, i_2, j_1, j_2, k_1, k_2$ in the definition of $M$ above, then
the corresponding $\det(N)$ would be of the form $X_1X_2X_3\cdot D$ where
$X_1$ is a coordinate of $P_1$, $X_2$ is a coordinate of $P_2$ and $X_3$ is
a coordinate of $P_3$ (and $D$ is the same as above). Since each of the three
points have at least one non-zero coordinate, if the order six minor
constructed with the columns $l_1, \dots, l_6$ is zero, then necessarily $D$
must be zero. In this way we see that, in order to have that all the order
six minors of $\Phi(P_1, P_2, P_3)$ are zero, it is enough to compute the
ideal of the order six minors of the matrix $M$ above and divide each
minor by $A_1A_2(u_1A_1+u_2A_2)$. By this procedure, after
a saturation w.r.t.\ the condition that $P_1$, $P_2$, and $P_1, P_3$ and
$P_2$, $P_3$ are distinct, we get an ideal that has
a very simple primary decomposition, which is given by the following three
ideals:
%
\[
\left(\scl{P_i}{P_1}, \scl{P_i}{P_2},\scl{P_i}{P_3}\right) \quad
\mbox{for } i = 1, 2, 3 \,.
\]
%
As a consequence, the line $r$ is tangent to~$\iso$
(in $P_1$ or $P_2$ or $P_3$).
It is not difficult to see via a symbolic computation that it is not possible to have that all the order
$4$-minors of $\Phi(P_1, P_2, P_3)$ are zero.
\end{proof}

\begin{prop}
\label{manca il riferimento su ancillary    non e': condition_rank_aligned}
%%%  conto si trova su file prop47_5ago.sage
Let $P_1, P_2, P_4$ be three distinct points of the plane. Then:
\begin{itemize}
\item $5 \leq \rk \ \Phi(P_1, P_2, P_4) \leq 6$;
\item if
$\rk \ \Phi(P_1, P_2, P_4) = 5$, then $P_1, P_2, P_4$
 are aligned.
\end{itemize}
In the last case, the line joining $P_1$, $P_2$, and $P_4$ is tangent to~$\iso$
in one of the three points.
\end{prop}
\begin{proof}
By \Cref{two_orbits}, we can split the proof into two parts, considering the case in
which $P_4 = (1: 0: 0)$ and the case in which $P_4 = (1: \iii: 0)$.
In both cases, the computation of the ideal of order five minors of the matrix
$\Phi(P_1, P_2, P_4)$ and the saturation of it w.r.t.\ the distinct point condition gives the whole ring, so the matrix cannot have rank smaller than~$5$.
Meanwhile, the computation
of the ideal of the order six minors of $\Phi(P_1, P_2, P_4)$ and its
saturation w.r.t.\ the distinct point condition gives that
$P_1, P_2, P_4$ must be aligned and, according to \Cref{proposition:three_aligned_ranks},
that line is tangent to~$\iso$ in one of the three given points.
\end{proof}


\begin{prop}
\label{prop:condition3+1}
Let $P_1, P_2, P_3, P_4$ be four distinct points of the plane such that
$P_1, P_2, P_3$ are aligned and let $r$
be the line joining them. If
$\rk \ \Phi(P_1, P_2, P_3, P_4) \leq 7$ then $r$ is tangent to~$\iso$ in one of the three points $P_1, P_2, P_3$.
\end{prop}
\begin{proof}
Again, we distinguish two cases: $P_1 = (1: 0: 0)$ and
$P_1 = (1: \iii: 0)$. In both cases, the ideal of the order $8$
minors of $\Phi(P_1, P_2, P_3, P_4)$ can be computed and saturated
w.r.t.\ the conditions that the points are different and that
$P_1, P_2, P_4$ are not aligned.
The direct inspection of the ideal obtained from these procedures gives the thesis.
\end{proof}

\begin{rmk}
 As we clarified in \Cref{...}, four aligned eigenpoints are never contained in a zero-dimensional, eigenscheme. In \Cref{...}, we will show that if an eigenscheme contains four distinct aligned eigenpoints, then it contains the whole line joining them.
\end{rmk}

\begin{prop}
\label{proposition:four_aligned}
Let $Q_1, \dotsc, Q_4$ be four distinct aligned points of the plane and
let $r$ be the line through them. Then:
\begin{itemize}
\item $6 \leq \rk  \ \Phi(Q_1, \dotsc, Q_4) \leq 7$;
\item $\rk  \ \Phi(Q_1, \dotsc, Q_4) = 6$ if and only if $r$ is tangent
to~$\iso$.
\end{itemize}
\end{prop}
\begin{proof}
  A direct symbolic computation shows that all the maximal minors of~$\Phi(Q_1, \dotsc, Q_4)$ are
  zero, so $\rk \,(\Phi(Q_1, \dotsc, Q_4)) \leq 7$.
  Indeed, in view of \Cref{lemma:minors}, we can just check the order~$8$ minors of the cut matrix of conditions of $Q_1, \dotsc, Q_4$; it turns out that they are all zero.

  On the other hand, the rank of $\Phi(Q_1, \dotsc, Q_4)$ cannot be $5$:
  if this were the case, for any triple~$T$ extracted from the four points, the matrix of conditions would have rank~$5$, hence by \Cref{proposition:three_aligned_ranks} the line~$r$ would be tangent to~$\iso$ in an element of $T$; this is a contradiction.
  So, the first item of the statement is proven.

  We proceed to prove the second item.
  We take the cut matrix of conditions of~$Q_1, Q_2, Q_3$, to which we add the row~$\phi_1(Q_4)$.
  We call this matrix~$M_1$.
  Similarly, we construct the matrices $M_2$ and~$M_3$.
  We consider the coordinates of the points as variables and we compute
  the ideal $J_1$, the radical of the ideal of the maximal minors of $M_1$ ; similarly, we obtain $J_2$ and~$J_3$.
  The saturation of the sum $J_1 + J_2 + J_3$ by the conditions that the points are distinct is a principal ideal generated by~$\sigma(r)$.
  Since the four points $Q_1, \dotsc, Q_4$ play symmetric roles, the statement is proven.
\end{proof}

In a similar way, it is possible to generalize the above proposition:
\begin{prop}
\label{proposition:n_aligned}
Let $Q_1, \dotsc, Q_r$ be $r\geq 4$ distinct aligned points of the plane and
let $r$ be the line through them. Then:
\begin{itemize}
\item $6 \leq \rk  \ \Phi(Q_1, \dotsc, Q_r) \leq 7$;
\item $\rk  \ \Phi(Q_1, \dotsc, Q_r) = 6$ if and only if $r$ is tangent
to~$\iso$.
\end{itemize}
\end{prop}

\subsection{The conditions $\delta_1$ and $\delta_2$}

We now define three quantities depending on a triple or on a $5$-tuple of points in the plane.
These quantities are crucial to describe what happens when we have aligned eigenpoints.
In the following, we denote by $P + Q$ the line through two distinct points $P$ and $Q$.

\begin{definition}
\label{definition:delta1}
 Let $P$, $Q$ and~$R$ be distinct, not aligned points in the plane.
 We define the quantity
 %
 \[
  \delta_1(P, Q, R) :=
  \scl{P}{P} \scl{Q}{R} - \scl{P}{Q}\scl{P}{R}
  =
  \scl{P\times Q}{P \times R} \,,
 \]
 %
 where $\times$ denotes the cross product, i.e.,
 %
 \[
  P \times Q = (P_2 Q_3 - P_3 Q_2, \, P_3 Q_1 - P_1 Q_3, \, P_1 Q_2 - P_2 Q_1) \,.
 \]
 %
\end{definition}

\begin{rmk}
 Geometrically, the condition $\delta_1(P, Q, R) = 0$ corresponds to the orthogonality of the vector planes corresponding to the lines $P + Q$ and $P + R$.
\end{rmk}


\begin{definition}
\label{definition:delta1b}
 Let $P$, $Q$ and~$R$ be distinct aligned points in the plane.
 We define the quantity
 %
 \[
  \overline{\delta}_1(P, Q, R) :=
  \scl{P}{P} \scl{Q}{R} + \scl{P}{Q}\scl{P}{R} \,.
  \]
 %
\end{definition}

\begin{definition}
\label{Vconf}
Let $P_1, P_2, P_3, P_4, P_5$ be five distinct points of the plane
such that $P_1, P_2, P_3$ and $P_1, P_4, P_5$ are aligned.
We call such a configuration a \emph{$V$-configuration}.
\end{definition}


\begin{definition}
 Let $P_1, \dots, P_5$ be a $V$-configuration.
We define the quantity
 %
 \[
  \delta_2(P_1, P_2, P_3, P_4, P_5) :=
  \scl{P_1}{P_2} \scl{P_1}{P_3} \scl{P_4}{P_5} -
  \scl{P_1}{P_4} \scl{P_1}{P_5} \scl{P_2}{P_3} \,.
 \]
 %
\end{definition}
%%
\begin{rmk}
Here and in the sequel, $s_{ij}$ denotes the quantity $\scl{P_i}{P_j}$.
It is possible to see that
\begin{eqnarray}
\label{rmk_delta_1}
\delta_1(P_1, P_2, P_4) = 0  &\mbox{iff} &\scl{P_4}{s_{11}P_2-s_{12}P_1} = 0\\
&\mbox{iff} &\scl{P_2}{s_{11}P_4-s_{14}P_1} = 0\\
\label{rmk_delta_2}
\overline{\delta}_1(P_1, P_2, P_3) = 0 &\mbox{iff}&
P_3 = (s_{12}^2+s_{11}s_{12})P_1-2s_{11}s_{12}P_2\\
\label{rmk_delta_3}
\delta_2(P_1, P_2, P_3, P_4, P_5) = 0 & \mbox{iff}
&P_3 = (s_{14}s_{15}s_{22}-s_{12}^2s_{45})P_1 \nonumber \\
&& +(s_{11}s_{12}s_{45}-s_{12}s_{14}s_{15})P_2\\
& \mbox{iff} &
P_5 = (s_{12}s_{13}s_{44}-s_{14}^2s_{23})P_1 \nonumber \\
&& +(s_{11}s_{14}s_{23}-s_{14}s_{12}s_{13})P_4 \nonumber
\end{eqnarray}
\end{rmk}

\begin{prop}
\label{prop:d1d2}
Let $P_1, \dots, P_5$ be a $V$-configuration. Then
\[
\rk \ \Phi(P_1, \dots, P_5) \leq 9
\quad \mbox{
if and only if} \quad
\delta_1(P_1, P_2, P_4) \cdot \delta_2(P_1, \dots, P_5) = 0.
\]
\end{prop}
\begin{proof}
We denote the coordinate of the points as follows:
%
\[
 P_1 = (A_1: B_1: C_1) \,, \quad
 P_2 = (A_2: B_2: C_2) \,, \quad
 P_4 = (A_4: B_4: C_4) \,,
\]
%
then $P_3 = u_1P_1+u_2P_2$ and $P_5 = v_1P_1+v_2P_4$ for some $u_1, u_2, v_1, v_2$.
We proceed as in
the proof of \Cref{proposition:three_aligned_ranks}: we find, via symbolic computation, that the determinant of
the cut matrix of conditions of $P_1, \dotsc, P_5$ is
%
\begin{gather}
\label{delta1delta2}
A_1A_2A_4(u_1A_1+u_2A_2)(v_1A_1+v_2A_4) \cdot u_1^2u_2^2v_1^2v_2^2 \cdot D^5 \cdot
\delta_1(P_1,P_2,P_4) \cdot \delta_2(P_1,\dots,P_5)
\end{gather}
%
where $D$ is the determinant of the matrix whose rows are $P_1, P_2, P_4$
and is non-zero, as well as are non-zero $u_1, u_2, v_1, v_2$ (since we
assume that $P_1, P_2, P_4$ are not aligned and the points are distinct).
As a consequence of \Cref{lemma:minors}, we have that when we compute all the order~$10$ minors of the matrix of conditions, we obtain all polynomials of the form
%
\begin{gather*}
X_1X_2X_4 X_3 X_5 \cdot u_1^2u_2^2v_1^2v_2^2 \cdot D^5 \cdot
\delta_1(P_1,P_2,P_4) \cdot \delta_2(P_1,\dots,P_5)
\end{gather*}
%
where $X_i$ varies among all components of~$P_i$ for $i \in \{1, \dotsc, 5\}$.
Since each of $P_1, \dotsc, P_5$ has at least one non-zero coordinate, we have that
$\rk \ \Phi(P_1, \dots, P_5) \leq 9$ if and only if
\[
\delta_1(P_1, P_2, P_4) \cdot \delta_2(P_1, \dots, P_5) = 0 \,. \qedhere
\]
\end{proof}

%% dimostrazione lemma seguente: per esempio si trova nella prima parte
%% del file contiCasoDegenere2.sage
\begin{lemma}
\label{lemma:special_case_rank_8}
Let $P_1, \dots, P_5$ be a $V$-configuration of points and assume that
\[
\scl{P_1}{P_2}=0, \quad \scl{P_2}{P_2}=0, \quad \scl{P_1}{P_4}=0,
\quad \scl{P_4}{P_4}=0.
\]
Then the matrix $\Phi(P_1, \dots, P_5)$ has rank $8$.
\end{lemma}
\begin{proof}
By \Cref{proposition:sigma_tangency},
the lines $P_1+P_2$ and $P_1+P_4$ are tangent to~$\iso$ in~$P_2$ and~$P_4$, respectively. The point $P_1$ cannot be on~$\iso$, hence, using the
action of $\mathrm{SO}_3(\mathbb{C})$, we can assume $P_1 = (1: 0: 0)$.
Since every element of $\mathrm{SO}_3(\mathbb{C})$ fixes the whole~$\iso$ (and sends a tangent line to it into another tangent line to it), when we transform the point~$P_1$
into $(1: 0: 0)$, we transform the points~$P_2$ and~$P_4$ into, respectively,
the points $(0: \iii: 1)$ and $(0: -\iii: 1)$ (which are the common points to
$\iso$ and the tangent lines through~$P_1$).
Therefore, it is enough to study the
specific configuration of the points:
%
\begin{gather*}
P_1 = (1: 0: 0), \quad P_2=(0: \iii: 1), \quad P_3=(u_1, \iii u_2, u_2), \\
P_4 = (0: -\iii: 1), \quad P_5 = (v_1, -\iii v_2, v_2),
\end{gather*}
%
where $(u_1: u_2), (v_1: v_2) \in \p^1$.
In the $15\times 10$ matrix $\Phi(P_1, \dots, P_5)$ we can erase the
rows: $\phi_2(P_1)$ (which is a zero row), the row $\phi_1(P_2)$
(since $\phi_1(P_2)=\iii\phi_2(P_2)$) and, for similar reasons, the
rows: $\phi_1(P_3)$, $\phi_1(P_4)$ and $\phi_1(P_5)$.
Moreover, since the rank of the matrix $\Phi(P_1, P_2, P_3)$ is $5$,
we can also erase the row $\phi_3(P_3)$ and, for the same reason, the
row $\phi_3(P_5)$. The remaining matrix $M$ is a $8\times 10$ matrix.
It is not possible that all the order $8$ minors
of $M$ are zero: the ideal they generate, after saturations ensuring
that the points are distinct, is the whole ring.
\end{proof}

\begin{theorem}
\label{theorem:rank_V}
Let $P_1, \dots, P_5$ be a $V$-configuration of
points. Then we have:
\begin{enumerate}
\item $8 \leq \rk \ \Phi(P_1, \dots, P_5) \leq 10$\,;
\item $\rk \ \Phi(P_1, \dots, P_5) \leq 9$ if and only if
$\delta_1(P_1, P_2, P_4) \cdot \delta_2(P_1, \dots, P_5) =0$\,;
\item $\rk \ \Phi(P_1, \dots, P_5) = 8$ if and only if, one of
the following two conditions is satisfied:
%
\begin{itemize}
\item $\delta_1(P_1, P_2, P_4) = 0$, \
$\overline{\delta}_1(P_1, P_2, P_3) = 0$,
\ $\overline{\delta}_1(P_1, P_4, P_5) = 0$\,;
  \item the line $P_1+P_2$ is tangent to~$\iso$ in $P_2$ or $P_3$
and the line $P_1+P_4$ is tangent to~$\iso$ in $P_4$ or $P_5$.
\end{itemize}
%
\end{enumerate}
\end{theorem}
\begin{proof}
%%% si basa sui file:
%%% rank_8_2_1_ii_0.sage e
%%% rank_8_1.sage
If the rank is $\leq 7$, from
\Cref{prop:condition3+1} applied to $P_1, P_2, P_3, P_4$ and $P_1, P_4, P_5, P_2$,
the lines $P_1+P_2$ and $P_1 + P_4$ are tangent to~$\iso$ (the first in $P_2$ or $P_3$ and the second in $P_4$ or $P_5$).
Then we get a contradiction from \Cref{lemma:special_case_rank_8}.
This shows the first item.

The second item follows from \Cref{prop:d1d2}.

In the rest of the proof, by \Cref{invariance}, we distinguish two possibilities:
$P_1 = (1:\iii :0)$ and
$P_1 = (1: 0: 0)$.
Let $P_2, P_4 = (A_2: B_2: C_2), (A_4: B_4: C_4)$ and
$P_3 = u_1P_1+P_2$, $P_5 = v_1P_1+v_2P_4$ be the other points.
In the first case, we transform the matrix $M = \Phi(P_1, \dots, P_5)$ by elementary row and column operations into $\mathbb{I}_2 \oplus M'$, where $\mathbb{I}_2$ is the order~$2$ identity matrix; thus, $\rk  M \leq 9$ if and only if $\rk  M' \leq 7$.
The computations require several other strategies, but at the end we are
able to see that in this case (i.e.\ the case in which $P_1$ is on~$\iso$), the matrix $M$ cannot have rank less then $9$.
If $P_1 = (1: 0: 0)$, then the ideal of the order $9$ minors of
$\Phi(P_1, \dots, P_5)$ (after several manipulations and saturations)
gives that there a few possibilities which can be summarized by the
following conditions: either the line $P_1+P_2$ is tangent to~$\iso$ (in $P_2$ or $P_3$) and the line $P_1+P_4$ is tangent to~$\iso$ (in $P_4$ or $P_5$) or $\delta_1(P_1, P_2, P_4) = 0$,
\ $\overline{\delta}_1(P_1, P_2, P_3) = 0$,
\ $\overline{\delta}_1(P_1, P_4, P_5) = 0$.
To conclude the proof, we use \Cref{lemma:special_case_rank_8} and \Cref{prop:d1d2}.
\end{proof}

\section{$V$-configurations of rank~$9$}
\label{sezione_delta_2}

Here we want to consider a $V$-configuration of points $P_1, \dots, P_5$
such that $\delta_1(P_1, P_2, P_4) = 0$ or $\delta_2(P_1, \dots, P_5) = 0$.
Moreover, we assume that the rank of the matrix $M$
of the conditions imposed by
the five points (which, by~\Cref{theorem:rank_V}, is at most~$9$)
is indeed $9$, hence there exists precisely one cubic curve $\cbc$ having
$P_1, \dots, P_5$ among its eigenpoints. Another immediate consequence of
the rank $9$ condition is that the
matrix $\Phi(P_1, \dots, P_5)$ has $9$ linearly independent rows, so
we can extract from it a  $9 \times 10$-submatrix
$\mathcal{H}$ whose order $9$ minors are not all zero. If
$\mathcal{H}_i$ is the order $9$ minor of
$\mathcal{H}$ given by erasing the $i$-th column ($i=1, \dots, 10)$, then,
from Cramer's rule, the polynomial defining $\cbc$ is given by
\[
 F(X) = \sum_{i=1}^{10}(-1)^i\det(\mathcal{H}_i)\cdot v(f)(X)_i
\]
(where $v(f)(X)_i$ is the $i$-th monomial of $v(f)(X)$
and $X=(x: y: z)$). In other words,
\[
F(X) = \det \left( \begin{array}{c} \mathcal{H}\\ v(f)(X)
 \end{array} \right)
\]
From the matrix $\mathcal{H}$ it is moreover possible to construct
the three minors of (\ref{eq:def_matrix}) (i.e.\ the Geiser polynomials
\textbf{(@@si chiamano cosi'?@@)}), indeed we have:
\begin{prop}
\label{proposition:geiser1}
If $g_1, g_2, g_3$ are the three minors of the matrix
\[
\left(
\begin{array}{ccc}
x & y & z \\
\de_xF(X) & \de_yF(X) & \de_z F(X)
\end{array}
\right)
\]
they can be computed by:
\[
\det \left( \begin{array}{c} \mathcal{H}\\
\phi_1(X)
\end{array} \right),\quad
\det \left( \begin{array}{c} \mathcal{H}\\
\phi_2(X)
\end{array} \right), \quad
\det \left( \begin{array}{c} \mathcal{H}\\
\phi_3(X)
\end{array} \right)
\]
\end{prop}
\begin{proof} We have:
\begin{eqnarray*}
g_1 & =  & x \cdot \de_y F(X)- y \cdot \de_xF(X) \\
& = &
x\cdot \de_y \det \left( \begin{array}{c} \mathcal{H}\\ v(f)(X)
\end{array} \right) -y \cdot
\de_x \det \left( \begin{array}{c} \mathcal{H}\\ v(f)(X)
\end{array} \right) \\
& = & \det \left( \begin{array}{c} \mathcal{H}\\ x \cdot v(f_y)(X)-
y \cdot v(f_x)(X)
\end{array} \right)\\
& = & \left( \begin{array}{c} \mathcal{H}\\
\phi_1(X)
\end{array} \right)
\end{eqnarray*}
and similarly for $g_2 = x \cdot \de_z F(X)- z \cdot \de_xF(X)$
and $g_3 = y \cdot \de_z F(X)- z \cdot \de_yF(X)$.
\end{proof}


\begin{es} Consider the following five points in a $V$-configuration:
\begin{gather*}
p_1, \ p_2, \ p_4 = (2: -1: 1), \ (-1: 1: 3), \ (3: 6: -1), \\
p_3, \ p_5 = p_1+p_2,  \ p_1+p_4
\end{gather*}
They satisfy the condition $\delta_2(p_1, \dots, p_5) = 0$.
We consider the $9\times 10$ matrix $\mathcal{H}$ whose rows
(suitable rescaled) are:
$\phi_i(p_j)$ for $i = 1, 2$ and $j = 1, 2, 3, 4$ and $\phi_1(p_5)$:
\[
\mathcal{H} =
\left(\begin{array}{rrrrrrrrrr}
12 & 4 & -7 & 6 & 4 & 3 & -4 & 1 & 2 & 0 \\
-12 & 4 & -1 & 0 & 4 & -3 & 2 & 7 & -4 & 6 \\
-3 & 1 & 1 & -3 & 6 & 0 & -6 & -9 & -9 & 0 \\
-9 & 6 & -3 & 0 & 17 & -8 & -1 & -21 & -6 & -27 \\
0 & 1 & 0 & 0 & 0 & 4 & 0 & 0 & 16 & 0 \\
-12 & 0 & 0 & 0 & -31 & 0 & 0 & -56 & 0 & 48 \\
54& 63& 36& -108& -12& -9& 12& 2& -1& 0\\
27 & 36 & 36 & 0 & 21 & 48 & 108 & -17 & -36 & 9 \\
336& 328& 119& -588& -56& -33& 56& 7& -4& 0
\end{array}
\right)
\]
The cubic with $p_1, \dots, p_5$ as eigenpoints is given by
\[
\det \left(
\begin{array}{c}
\mathcal{H}\\
x^3 \ \ x^2y \ \ xy^2 \ \dots \  z^3
\end{array}
\right)
\]
whose value is (after a suitable rescaling):
\[
766x^3 - 1176x^2y + 417xy^2 - 128y^3 + 1488x^2z - 1722xyz
+ 597y^2z - 207xz^2 + 504yz^2 + 911z^3
\]
and the Geiser (@@) polynomials are:
\begin{eqnarray*}
g_1 & = & 392x^3 + 488x^2y - 656xy^2 + 139y^3 + 574x^2z + 594xyz - 574y^2z - 168xz^2 - 69yz^2\\
g_2 & = & 496x^3 - 574x^2y + 199xy^2 - 904x^2z + 1120xyz - 139y^2z - 81xz^2 + 574yz^2 + 69z^3\\
g_3 & = & 496x^2y - 574xy^2 + 199y^3 + 392x^2z - 416xyz + 464y^2z + 574xz^2 + 513yz^2 - 168z^3
\end{eqnarray*}
and can be computed either using their definition,
or~\Cref{proposition:geiser1}.
The common zeros of $g_1, g_2, g_3$ are the points $p_1, \dots, p_5$
and the points $p_6 = (21: 34: -8)$ and $p_7 = (19: 35: -9)$.
\end{es}

\begin{rmk}
In case the matrix $\Phi(P_1, \dots, P_5)$ has rank $8$, the family of
cubics with $P_1, \dots, P_5$ as eigenpoints is one dimensional, so the
above construction can be modified, by substituting one of the rows of
$\mathcal{H}$ with a random row of elements of the field~$K$.
\end{rmk}

Here we want to consider the following problem:
given five generic points in a $V$-configuration and imposing one
of the two conditions
$\delta_1(P_1, P_2, P_4) = 0$ or $\delta_2(P_1, \dots, P_5) = 0$, assuming
again that the rank of $\Phi(P_1, \dots, P_5)$ is~$9$, what
can we say about all the eigenpoints of the corresponding cubics?


To construct five points $P_1, \dots, P_5$ which are in a $V$-configuration
and such that $\delta_1(P_1, P_2, P_4)= 0$ is quite easy: $P_1$
and $P_2$ can be taken in an arbitrary way, $P_4$ has to be chosen in such
a way that it satisfies the linear
equation $\scl{P_4}{P_2}\scl{P_1}{P_1}-\scl{P_4}{P_1}\scl{P_1}{P_2} = 0$
and $P_3$ and $P_5$ have to be chosen on the lines $P_1+P_2$ and $P_1+P_4$
respectively. In particular, the corresponding variety of cubic curves
has dimension $7$ (parametrized by an opes subset of
$\mathbb{P}^2\times \mathbb{P}^2 \times
\mathbb{P}^1\times  \mathbb{P}^1$).
The construction of a random example
of five points as above, gives a smooth cubic curve whose $7$ eigenpoints
do not have other collinearities (in addition to those of a
$V$-configuration).

The case in which the $V$-configuration satisfies the condition
$\delta_2(P_1, \dots, P_5) = 0$ is somehow different.

We define:
\begin{equation}
  \begin{split}
    U_1 & =  \langle P_1, P_2\rangle \left(\langle P_1, P_1\rangle
  \langle P_4,P_5\rangle - \langle P_1, P_4\rangle \langle P_1, P_5\rangle
  \right)\\
  U_2 & =  \langle P_1, P_2\rangle^2\langle P_4, P_5\rangle
  -\langle P_1, P_4\rangle \langle P_1, P_5\rangle \langle P_2, P_2\rangle
  \label{sst2}
  \end{split}
\end{equation}
in such a way that
$\delta_2(P_1, \dots, P_5) = U_1u_1+U_2u_2$. In order to
have a $V$-configuration satisfying $\delta_2(P_1, \dots, P_5) = 0$ we can
therefore fix $P_1, P_2, P_4$ arbitrarily, $P_5$ any point
on the line $P_1+P_4$ and in this
way $P_3$ is determined by $u_1 = U_2$ and $u_2 = -U_1$. Note that,
since we assume the points distinct, $U_1$ and $U_2$ cannot be zero.
Also in this case
the subvariety of $\mathbb{P}^9$ given by the cubics with $5$
eigenpoints satisfying $\delta_2 = 0$ is of dimension $7$
(parametrized by an open subset of
$\mathbb{P}^2\times \mathbb{P}^2 \times \mathbb{P}^2\times \mathbb{P}^1$).
Also in this case, it is easy to construct examples, but now it turns out that
the $7$ eigenpoints of the obtained cubics satisfy the further
condition that also $P_6$ and $P_7$ are aligned with $P_1$. To explain
why, we proceed as follows.

Consider again a matrix $\mathcal{H}$ of rank $9$ extracted from
the matrix $\Phi(P_1, P_2, P_3, P_4, P_5)$, where now $P_3 = U_2P_1-U_1P_2$
and the three polynomials $g_1, g_2, g_3$ constructed
in~(\ref{proposition:geiser1}). The condition
$\rk \,(\mathcal{H}) = 9$ ensures that
$g_1, g_2$ and $g_3$ cannot be zero and are polynomials of degree
$3$ in $x, y, z$.
The computation shows that three polynomials have a common factor,
hence are of the form $\Omega G_1$, $\Omega G_2$, $\Omega G_3$, where
($\Omega$ is a polynomial in the variables
$A_1, B_1, C_1, A_2, B_2, C_2, A_4, B_4, C_4, v_1, v_2$ but
does not contain the variables $x, y, z$), hence $G_1, G_2, G_3$ are again
degree three polynomials in $x, y, z$, with coefficients in the variables
$A_1, \dots, C_4, v_1, v_2$. The common zeros are the eigenpoints of the
cubic $\cbc$.
%
\begin{rmk} For computational reasons, it is much more convenient
to postpone the substitution $u_1 = U_2$ and $u_2 = -U_1$ after the
computation of the three polynomials constructed
in~\Cref{proposition:geiser1}.
\end{rmk}
%
\begin{prop}
\label{proposition:G_split}
 The polynomial
$F := C_1G_1-B_1G_2+A_1G_3$ splits into three factors $r_1$, $r_2$, $r_3$,
linear in $x, y, z$.
\end{prop}
\begin{proof}
%%%%%%%%%%%%%%%%%%%%
@@@@@@@@@@@@@@@@@@@@@@@ DA QUI......

 We know that $G_1, G_2, G_3$ admit the syzygy $z G_1 - y G_2 + x G_3 = 0$.
Then, due to the exactness of the Koszul complex of $(x,y,z)$
(see \cite[Chapter~17]{Eisenbud1995}),
we know that $G_1, G_2, G_3$ are the $2 \times 2$ minors of a matrix of the form
%
\[
 \left(
 \begin{array}{ccc}
  x & y & z \\
  H_1 & H_2 & H_3
 \end{array}
 \right)
\]
%
Let $E$ be the zero-set of $G_1, G_2, G_3$.
If $E$ is zero-dimensional and reduced,
then \cite{BGV} proved that in the ideal of~$E$ there exists a unique triple of generators of the form
%
\[
 \left(
 \begin{array}{ccc}
  x & y & z \\
  \partial_x f & \partial_x f & \partial_x f
 \end{array}
 \right)
\]
%
where $f$ is a ternary cubic.
We actually know that $E$ is zero-dimensional, because this can be checked by taking particular values in the $G_i$.
Hence we obtain that the $G_i$ are the $2\times 2$ minors of the previous matrix.

@@@@@@@@@@@@@@@@@ ..... FINO A QUI probabilmente ora e' superfluo
%%%%%%%%%%%%%%%

Consider the Laguerre map
%
\[
 \gamma \colon \p^2 \setminus E \longrightarrow \p^2 =
 \p \bigl( \left\langle G_1, G_2, G_3 \right\rangle^\vee \bigr) \,.
\]
%
By construction, the map $\gamma$ maps a point~$P$ to the pencil of cubics passing through~$E$ and~$P$.
Whenever we have an aligned triple of eigenpoints and we pick $P \in \p^2$ that is aligned with them, then the pencil of cubics is given by the line together with the pencil of conics passing through the remaining four eigenpoints.
This means that if we have an aligned triple of eigenpoints, then any point on such a line is mapped by $\gamma$ to the same element.
Hence $\gamma$ contracts the line passing through the three aligned eigenpoints.
Another description of~$\gamma$ is given by $P \mapsto (G_1(P): G_2(P): G_3(P))$.
This can hence be written as $P \mapsto P \wedge \nabla(f)(P)$.
The image of~$P$ is then the intersection point of the two lines:
%
\[
 \left\{
 \begin{array}{l}
  A x + B y + Cz  = 0\\
  \partial_x f(P) + \partial_y f (P) + \partial_z f(P) = 0
 \end{array}
 \right.
\]
%
Since we know that there is a whole line that is contracted to the same point~$Q$,
we know that the first equation is a factor of the second.
Therefore we have $\left\langle Q, (x,y,z) \right\rangle = 0$, so $\left\langle Q, \nabla f (x,y,z) \right\rangle = 0$.
It then follows that $\left\langle P, (x, y, z) \wedge \nabla f(x,y,z) \right\rangle = 0$, which implies that $A G_1 + B G_2 + C G_3$
\\
@@@@
QUI C'E' un meno da qualche parte? \\
@@@\\
evaluates to zero for each $(x,y,z)$ such that $\left\langle Q, (x,y,z) \right\rangle= 0$, namely for each point in the line $P_1 P_2$.
This means that the line $P_1 P_2$ is contained in $E$.
By repeating this argument in the case $P_1 P_4$ we see that $E$ splits into three distinct lines.
\end{proof}

According to \Cref{proposition:G_split}, the polynomial
$C_1G_1-B_1G_2+A_1G_3$ splits into three factors $r_1$, $r_2$, $r_3$,
linear in $x, y, z$, which
correspond to the line $P_1+P_2$, the line $P_1+P_4$ and the line $P_6+P_7$.
The factorization of $F$ gives that the factor $r_3$ is:
\[
r_3 = 2yv_1A_1B_1^4A_2^2B_2A_4^3-2xv_1B_1^5A_2^2B_2A_4^3+\cdots
-2 yv_1A_1B_1^4A_2^2B_2A_4^3
\]
and is composed by almost $2000$ monomials. The relevant fact is that when we
evaluate $r_3$ on the coordinates of $P_1$, we get zero. Hence we have:
\begin{prop}
\label{proposition:terzo_allineamento}
If we have a $V$-configuration of five points $P_1, \dots, P_5$
such that the rank of the matrix $\Phi(P_1, \dots, P_5)$ is $9$ and
such that $\delta_2(P_1, \dots, P_5) = 0$,
then the unique cubic determined by the condition that $P_1, \dots, P_5$
are eigenpoints, has also the two other eigenpoints $P_6$ and $P_7$
aligned with $P_1$.
\end{prop}

\textbf{(@@@ Forse qui va la parte sulle cubiche singolari? @@@)}

Finally, we want to see when, in a $V$-configuration, we have that the point
$P_1$ is singular.

\begin{prop}
\label{proposition:P1_sing}
If $P_1, \dots, P_5$ is a $V$ configuration and if $C$ is a cubic which
has among the eigenpoints $P_1, \dots, P_5$, then $P_1$ is singular if
and only if one of the following conditions is satisfied:
\begin{enumerate}
\item $\delta_1(P_1, P_2, P_4) = 0$ and $\overline{\delta}_1(P_1, P_2, P_3) = 0$;
\item $\delta_1(P_1, P_2, P_4) = 0$ and $\overline{\delta}_1(P_1, P_4, P_5) = 0$;
\item $\overline{\delta}_1(P_1, P_2, P_3) = 0$ and
$\overline{\delta}_1(P_1, P_4, P_5) = 0$.
\end{enumerate}
\end{prop}
\begin{proof}
\verb+file: cubicaSingolare1ii0.sage+ e \verb+cubicaSingolare100.sage+\\
As usual, we can split the problem into two cases: $P_1 = (1: \iii: 0)$
and $P_1 = (1: 0: 0)$. In both cases we construct the matrix
$\Phi(P_1, \dots, P_5)$ (for $P_2$, $P_4$ generic, $P_3$ aligned with
$P_1$ and $P_2$ and $P_5$ aligned with $P_1$ and $P_4$). To this matrix we add
a row which correspond to the condition that $P_1$ is singular, obtaining
in this way a matrix $M$. After
suitable manipulations of the matrix $M$, we see that the condition that
the linear system associated to $M$ (which gives the coefficient of the cubic
having $P_1, \dots, P_5$ eigenpoints and $P_1$ singular) gives no solution
for the case $P_1=(1: \iii: 0)$ and the three above conditions for the
case in which $P = (1: 0: 0)$.
\end{proof}

\section{$V$-configurations of rank~$8$}
file \verb+contiCasoDegenere2.sage+ and file
\verb+conf_sigma12_sigma14.sage+\\
\verb+molti altri file via +\\
\verb+via indicati nei commenti di latex+

In this Section we study the possible configurations of
eigenpoints for cubics with a $V$-configuration
satisfying the condition $\rk \Phi(P_1, \dots, P_5) = 8$. According
to~\Cref{theorem:rank_V}, we have to distinguish the two cases
\begin{eqnarray}
\delta_1(P_1, P_2, P_4)=\overline{\delta}_1(P_1, P_2, P_3) =
\overline{\delta}_1(P_1, P_4, P_5) = 0,
\label{rk8_1}\\
\sigma(P_1, P_2) = 0 \mbox{ and } \sigma(P_1, P_4) = 0
\label{rk8_2}
\end{eqnarray}
(hence condition~(\ref{rk8_2}) means that \emph{the two lines of
the $V$-configuration are tangent to $\iso$ in $P_2$ and $P_4$}).


Let us start with the first case.

The three conditions~(\ref{rk8_1}) give (using (\ref{rmk_delta_1})
and (\ref{rmk_delta_2})) give that we can chose the points as follows:
\begin{itemize}
\item $P_1$ and $P_2$ in an arbitrary way;
\item $P_4$ in the $\mathbb{P}^1$
space of points orthogonal to $s_{11}P_2-s_{12}P_1$;
\item $P_3 = (s_{12}^2+s_{11}s_{12})P_1-2s_{11}s_{12}P_2$;
\item $P_5 = (s_{12}^2+s_{11}s_{12})P_1-2s_{11}s_{14}P_4$.
\end{itemize}

Such a choice
of points gives the matrix $M = \Phi(P_1, \dots, P_5)$ of rank $8$ and
therefore the linear system associated to $M$, which gives the
coefficients of the cubics with eigenpoints $P_1, \dots, P_5$, has $\infty^1$
solutions. This shows that the variety $\mathcal{V}$ of these cubics is
of dimension $6$ and is parametrized by an open subset of
$\mathbb{P}^2 \times \mathbb{P}^2\times \mathbb{P}^1\times \mathbb{P}^1$.
Moreover, from~\Cref{proposition:P1_sing}, we have that all these cubics
are singular in $P_1$.
If we take a random point of this variety, we see that the corresponding
cubic has $7$ distinct eigenpoints, is singular in $P_1$ and the $7$ points
do not satisfy other collinearities in addition to those of $P_1, P_2, P_3$
and $P_1, P_4, P_5$.

\begin{rmk}
If $P_1, \dots, P_5$ satisfy the conditions~(\ref{rk8_1}),
it follows that also $\delta_2(P_1, \dots, P_5) = 0$. Since,
in general, we do not have the collinearity $P_1, P_6, P_7$,
we see that the hypothesis ``rank $9$''
in~\Cref{proposition:terzo_allineamento} is necessary.
\end{rmk}

Concerning possible sub-cases with further collinearities of the points,
the following results hold:
%%%%
%% tutti i conti si trovano nei file:
%% three_d_partA.sage
%% three_d_partB.sage
%% three_d_partC.sage
%% three_d_partD.sage
%% three_d_partE.sage
%%
%% c'e' anche il file three_d_alt.sage che calcola tutti gli autopunti
%% in configurazione (8) nel caso dei 3 delta = 0.
%%
\begin{prop}
\label{three_d_three_alignments}
If the five points $P_1, \dots, P_5$ satisfy~(\ref{rk8_1}),
then, among the $\infty^1$ cubic curves that we get from the
solution of the linear system whose coefficients matrix is $M$, there is
a cubic curve with $7$ eigenpoints with the following three alignments:
$(P_1, P_2, P_3)$, $(P_1, P_4, P_5)$, $(P_1, P_6, P_7)$. \\
No choices of $P_1, \dots, P_5$ allow one to obtain further alignments of the
$7$ eigenpoints.
\end{prop}
And, moreover,
\begin{prop}
\label{prop:d2_6allin}
If the five points $P_1, \dots, P_5$ satisfy~(\ref{rk8_1})
and if we impose the condition that there is an eigenpoint, say $P_6$,
which is aligned with $P_2$ and $P_4$ then the eigenpoints satisfy all these
alignments:
\[
(P_1, P_2, P_3), (P_1, P_4, P_5), (P_2, P_4, P_6), (P_2, P_5, P_7),
(P_3, P_4, P_7), (P_3, P_5, P_6)
\]
Hence the points $P_6$ and $P_7$ are such that
$P_6 = (P_2+P_4) \cap (P_3+P_5)$
and $P_7 = (P_3+P_4) \cap (P_2+P_5)$.
(Of course, a similar result holds if we take $P_3$ in place of $P_2$ or $P_5$
in place of $P_4$).
\end{prop}
The two above propositions exhaust all the possible configurations
of collinearities of the eigenpoints in case of condition~(\ref{rk8_1}).

The proof we give to the two propositions above is computational.
Using the action of $\mathrm{SO}_3(\mathbb{C})$ (see~\Cref{two_orbits}),
we can assume $P_1= (1: \iii: 0)$ or $P_1= (1: 0: 0)$. As observed
in the proof of~\Cref{theorem:rank_V}, if $P_1 = (1: \iii: 0)$,\\
(@@ \verb+Vedi file rank_8_2_1_ii_0.sage+), \\
the martix
$\Phi(P_1, \dots, P_5)$ cannot have rank smaller then $9$, so the only
case to consider is given by $P_1 = (1: 0: 0)$. Hence
$P_2 = (A_2: B_2: C_2)$, $P_4 = (A_4: B_4: C_4)$ and
$P_3 = u_1P_1+u_2P_2$, $P_5=v_1P_1+v_2P_4$.
The condition $\delta_1(P_1, P_2, P_4)=0$ reduces to $B_2B_4+C_2C_4$
and it is easy to construct the explicit coordinates of the involved points.

The problem of considering all the possible further
collinearities of the points can be converted into the study of suitable
ideals in the coordinates of the points and can be attacked with
computational tools. More precisely,
consider the matrix $\Phi(P_1, \dots, P_5)$ and select $8$ linearly independent
rows (since $P_1$ is numeric, it can be seen that for instance the rows
of position $0, 1, 3, 5, 6, 8, 10, 12$ are a good choice and avoid to consider
many exceptions) and let $M_1$ be a matrix obtained in this way.
We fix two numeric $10$-dimensional vectors $V_a$ and $V_b$ and we construct
$M_a$ (resp.\ $M_b$) obtained by vertically stacking $V_a$ (resp.\ $V_b$)
vector to $M_1$. The rank of $M_a$ and $M_b$ is $9$ (unless an unlucky choice
of the vectors, in which case they can be changed). The matrices $M_a$ and
$M_b$ play the same role of the matrix $\mathcal{H}$
of~\Cref{sezione_delta_2}, hence we can do the same construction explained
there, in order to obtain the line $r_a$ passing through the two
eigenpoints $P_6$ and $P_7$ of the cubic obtained from $M_a$ and the
line $r_b$ passing through the analogous eigenpoints of the cubic
obtained from $M_b$.
Hence the line $r = w_ar_a+w_br_b$
(where $w_a, w_b$ are parameters) is the line passing through the eigenpoints
$P_6$ and $P_7$
of the generic cubic of the $\infty^1$ family, i.e.\ the cubic obtained
the three collinearities $(P_1, P_2, P_3)$,
$(P_1, P_4, P_5)$ and $(P_1, P_6, P_7)$.

%% La dimostrazione della prop:rk8_2B di sotto si trova essenzialmente
%% nel file contiCasoDegenere2.sage.
%%
%% Anche i file seguenti sono utili:
%% confV_tg_iso2.sage, dove si fa il caso P1 = (1, 0, 0)
%% e si prova che l'allineamento P6 con P2, P4 da' 7 punti
%% con allineamenti:
%% (1, 2, 3), (1, 4, 5), (1, 6, 7), (2, 4, 6)
%% C'e' poi il file confV_tg_iso.sage che cerca di fare i conti
%% in generale, ma e' forse da completare.

%% il file casoDegenere2.tex contiene anche i risultati scritti
%% qui sotto.

Among all the possible cubics of the $5$ dimensional variety we get from
condition~(\ref{rk8_2}) there are some subfamilies in which the eigenpoints
have other alignments. More precisely,
\begin{prop}
\label{prop:rk8_2B}
In the family of all cubics which satisfy condition~(\ref{rk8_2}) there
are cubics which have $7$ eigenpoints with the alignments:
\[
(P_1, P_2, P_3),\  (P_1, P_4, P_5),\  (P_1, P_6, P_7),\  (P_2, P_4, P_6)
\]
and cubics which have $7$ eigenpoints with the alignments:
\[
(P_1, P_2, P_3),\  (P_1, P_4, P_5), \ (P_1, P_6, P_7),\  (P_2, P_5, P_6),\
(P_3, P_4, P_6),\  (P_3, P_5, P_7)
\]
No other collinearities among the eigenpoints are possible.
\end{prop}


In order to get the proof of~\Cref{prop:rk8_2B}
we consider again the action of $\SO_3(\C)$ and here we can assume:
\[
P_1 = (1: 0: 0), \quad P_2 = (0: \iii: 1), \quad P_4 = (0: -\iii: 1)
\]
(and $P_3 = u_1P_1+u_2P_2$, $P_5 = v_1P_1+v_2P_4$). In this situation it
is easy to see that the solution of the system $\Phi(P_1, \dots, P_5)$
gives the following family of cubic curves:
\[
\mathcal{C}(l_1, l_2) = l_1\mathcal{C}_1+l_2\mathcal{C}_2
\]
where
\begin{align*}
  \mathcal{C}_1 & = x \cdot \left(2x^{2} + 3 y^{2} + 3 z^{2}\right)\\
  \mathcal{C}_2 & = (y + \iii z) \cdot (y - \iii z)
\cdot (y u_{2} v_{1} - \iii z u_{2} v_{1} - y u_{1} v_{2} -
\iii z u_{1} v_{2} + 2 \iii x u_{2} v_{2})
\end{align*}
Then, from these equations, all the computations which allow to prove
the two above propositions are easy.

We can describe in an explicit way the eigenpoints points
of~\Cref{prop:rk8_2B} as follows:
\begin{itemize}
\item $P_1$ can be chosen in an arbitrary way (out of $\iso$);
\item $P_2$ and $P_4$ are the two points in which the
tangents from $P_1$ to $\iso$ meet $\iso$;
\item $P_3$ and $P_5$ can be chosen in an arbitrary way on the lines
$P_1+P_2$ and $P_1+P_4$ respectively;
\item $P_6 = s_{11}s_{15}P_3-2s_{13}s_{15}P_1 + s_{11}s_{13}P_5$;
\item $P_7 = s_{11}s_{15}P_3+s_{13}s_{15}P_1 + s_{11}s_{13}P_5$.
\end{itemize}
@@ trovare simili formule per i punti con i 6 allineamenti della
II parte della prop.

@@ ci sono varie ortogonalita' tra i punti che si trovano nel file
casoDegenere2.pdf e che qui non sono riportate.

\section{Eigenpoints of positive dimension}


\begin{lemma}
\label{4ptiSuRetta2}
Suppose $P_1, P_2, P_3, P_4$ are four distinct points belonging to a line $t$
and let
$C$ be a cubic such that has $P_1, \dots, P_4$ among its eigenpoints. Then
all the points of $t$ are eigenpoints of~$C$. Moreover,
\begin{equation*}
6 \leq \rk  \  \Phi(P_1, P_2, P_3, P_4) \leq 7
\end{equation*}
and the rank is $6$ if and only if $\sigma(P_1, P_2) = 0$, i.e.\ if
and only if $t$ is tangent to the isotropic conic. If the rank of
the above matrix is $7$, then all the cubics which have the line $t$
of eigenpoints are given by $t^2l$ where $l$ is any line of the plane.
\end{lemma}
\begin{proof}
\verb+file: fourCollinearPoints.sage+\\
It is not restrictive to assume that one of the points (say $P_1$) is
the point $(1: 0: 0)$.
With this simplification, the computation of the order $6$ and order $7$
minors of $\phi(P_1, \dots, P_4)$ is not hard and
the first part of the thesis follows. If the rank of $\Phi(P_1, \dots, P_4)$
is $7$, then the cubics which have $P_1, \dots, P_4$ as eigenpoints
give a two dimensional linear subspace $W$ of $\mathbb{P}^9$, but also all
the cubics defined by $t^2l$ have $P_1, \dots, P_4$ as eigenpoints and are
a linear space of $\mathbb{P}^9$ contained in $W$. Hence the two linear
spaces coincide.
\end{proof}

\begin{prop}
\label{cubiche_con_2_rette}
Let $C$ be a cubic which has, among its eigenpoints, two lines $t_1$
and $t_2$. Then $t_1$ and $t_2$ are tangent to the isotropic conic
in two points $Q_1$ and $Q_2$ and, if $r$ is the line $Q_1+Q_2$
of equation $ax+by+cz=0$, the polynomial defining $C$ is:
\begin{equation}
C(r) = \left(r^2-3\left(a^2+b^2+c^2\right)\iso\right)r.
\label{2_lines_of_eigenpoints}
\end{equation}
$C$ has, in addition to $t_1$ and $t_2$, also the eigenpoint $(a: b: c)$
(the pole of $r$ w.r.t.\ $\iso$). \\
Conversely, if $C$ is defined by~(\ref{2_lines_of_eigenpoints}), where
$r$ is any line $ax+by+cz$ ($a, b, c \in K$), then
the eigenpoints of $C$ are the lines $t_1$ and $t_2$ (tangent to $C$
in the points $Q_1$ and $Q_2$ of intersection of $r$ with $\iso$) and
the point $(a: b: c)$.
\end{prop}
\begin{proof}
\verb+file: due_rette_autopunti.sage+\\
Let $t_1$ be a line of eigenpoints of $C$. If $t_1$ is not tangent
to $\iso$, from~\Cref{4ptiSuRetta2} $C$ is given by $t_1^2l$ (for
a suitable line $l$), but this cubic cannot have the line $t_2$
of eigenpoints.
In this way we see that $t_1$ and $t_2$ must be tangent to $\iso$ (in
$Q_1$ and $Q_2$ respectively). Let $Q_3 = t_1 \cap t_2$. We can assume
that $Q_3 = (1: 0: 0)$, hence $t_1$, $t_2$ and $r$ have equations,
respectively, $x+\iii y=0$, $x-\iii y = 0$ and  $x=0$.
If we impose to the generic cubic of the plane to have
$t_1$ and $t_2$ of eigenpoints, we get only one cubic defined by
$x(2x^2 + 3y^2 + 3z^2)$, which is of the
form~(\ref{2_lines_of_eigenpoints}). If, conversely, we fix a line
of the plane of equation $ax+by+cz=0$ and we consider the cubic $C$ of
equation given by~(\ref{2_lines_of_eigenpoints}), the ideal of
the eigenpoints of $C$ decomposes into an ideal whose radical is
the ideal of the point $(a:, b: c)$ and another ideal generated by
a degree two polynomial which give a singular conic of the pencil
of conics bitangent to $\iso$ in the two points $\iso \cap r$, hence
this conic splits into the two tangent lines to $\iso$ in $\iso \cap r$.
\end{proof}

\begin{prop}
Given a line $t$ of the plane, if $t$ is not tangent to the isotropic
conic, then all the cubics which have, among their
eigenpoints, the points of $t$ are defined by the polynomial $r^2l$,
where $l$ is any line of the plane. If $t$ is tangent to $\iso$, then
all the cubics which have among their eigenpoints the points of $t$ are
defined by the polynomials
\begin{equation}
\label{cubiche_con_retta_autop_tg}
t^2l+\lambda C(r),
\end{equation}
where $\lambda \in K$ and $r$ is a line
passing through the tangent point $\iso\cap t$ (and $C(r)$ is defined
in~(\ref{2_lines_of_eigenpoints})). Conversely, if $r$ is a line and
$t$ is another line, tangent $\iso$ in one of the two points of intersections
of $r$ with $\iso$, then all the cubics given
by~(\ref{cubiche_con_retta_autop_tg}) have, among their eigenpoints,
the line $t$.
\end{prop}
\begin{proof}
\verb+una_retta_tg_autop.sage+\\
The first part of the proposition is contained in~\Cref{4ptiSuRetta2}.
Suppose $t$ is tangent to $\iso$. We assume that $t$ is the line
$x+\iii y$, hence tangent to $\iso$ in $P = (1: \iii: 0)$.
We impose to the points
of $t$ to be eigenpoints of a generic cubic $a_0x^3+a_1x^2y+\cdots+a_9z^3$
and we get a linear system in $a_0, \dots, a_9$ whose solution gives that
the cubics can be expressed as a linear combinations of the following
four cubics:
%
\[
H_1 = t^2x, \quad H_2 = t^2y, \quad H_3 = t^2z, \quad H_4 = C(r)
\]
%
where $r$ is a line passing through $P$ (if, for instance, $r$
is $z$, then $H_4 = z(x^2 + y^2 + 2/3z^2)$. Hence the cubics are of the
form~(\ref{cubiche_con_retta_autop_tg}). Conversely, it is easy to verify
that all the cubics $C(r)$ of~(\ref{cubiche_con_retta_autop_tg}) are
linear combinations of $H_1, \dots, H_4$, hence it is enough
to see that every linear combination of $H_1, \dots, H_4$ gives a
cubic which has the line $t$ among its eigenpints, and the this result
follows from an explicit computation.
\end{proof}

\begin{prop}
Let $t$ be a line tangent to the isotropic conic in a point $P$. Then all the
cubics of the family~(\ref{cubiche_con_retta_autop_tg}) are singular in
$P$ and $t$ is one of the two tangents to the cubic in $P$.
\end{prop}
\begin{proof}
As seen in the previous proposition, up to the action of $\SO$, all the
cubics are a linear combination of $H_1, \dots, H_4$ and a direct
computation gives the thesis.
\end{proof}
NOTA: Le cubiche con P1 singolare e tangente in P1 data da tg1 sono
un sistema lineare di dimensione 6, le cubiche
di~(\ref{cubiche_con_retta_autop_tg}) sono di dimensione
4, Che altre proprieta' deve avere una cubica per rientrare
in~(\ref{cubiche_con_retta_autop_tg})?


\begin{lemma}
\label{lemma:fourOnIso}
Suppose $P_1, P_2, P_3, P_4$ are four distinct points on $\iso$ and let
$C$ be a cubic such that has $P_1, \dots, P_4$ among its eigenpoints.
Then all the points of $\iso$ are eigenpoints and $C$ splits into $\iso$
and a line $r$. In particular, if four
distinct points of $\iso$ are eigenpoints of a cubic, then all the
points of $\iso$ are eigenpoints of the cubic.
\end{lemma}
\begin{proof}
\verb+file eigenpointsConic.sage+\\
As usual, we can assume that $P_1 = (1: \iii: 0)$. Hence all the other points
of $\iso$ are given by
$P(w) = (\iii(w^2 + 1): 1 - w^2: -2w)$ ($w\in K$), so we can assume
$P_2 = P(w_1)$, $P_3 = P(w_2)$ and $P_4 = P(w_3)$, where $w_i$'s are
all distinct. The cubics which have $P_1, \dots, P_4$ among the eigenpoints
are obtained from the linear system whose associated matrix is
$M = \Phi(P_1, P_2, P_3, P_4)$. The rank of $M$ is $7$. In order to see this,
we can, first of all, select a submatrix of $M$ obtained erasing a
row for each of the points $P_i$ (since we know that the three vectors
$\phi(P)$ are linearly dependent for every $P$). In the obtained matrix
all the order $8$ minors are easy to compute and are zero.
If all the order~$7$
minors are set to zero, we get an ideal that, after suitable saturations,
is $(1)$. Since the rank of $M$ is~$7$, all the cubics for which the
four points are eigenpoints are a linear variety $\mathcal{L}$
in $\mathbb{P}^9$
of dimension $2$, but all the cubics which split into $\iso$ and a line
of the plane do have all the points of $\iso$ among the eigenpoints and
these cubics give a linear variety of dimension $2$ in $\mathbb{P}^9$
which therefore coincides with $\mathcal{L}$.
\end{proof}

\begin{lemma}
\label{lemma3ptiSuCiso}
It is not possible to have a cubic $C$ which has, among its eigenpoints,
an irreducible conic $\Gamma$ tangent in a point $P_1$ to the isotropic
conic and intersecting $\iso$ in two other distinct points.
\end{lemma}
\begin{proof}
\verb+caso3ptiIso_2.sage+\\
As in the proof~\Cref{lemma:fourOnIso}, we can assume that
$P_1 = (1: \iii: 0)$
and $P_2$ and $P_3$ are $P(w_1)$ and $P(w_2)$. Then we construct
the pencil of conics passing through $P_1$, $P_2$ and $P_3$ and
tangent to Ciso in $P_1$ and we construct two distinct points
$Q_1$ and $Q_2$ on a conic of this family. If there is a conic $\Gamma$
tangent to $\iso$ in $P_1$ and passing through $P_2$ and $P_3$,
then, for suitable values of the parameters defining the points
$P_1, P_2, P_3$ and $Q_1$ and $Q_2$, the matrix
$M = \Phi(P_1, P_2, P_3, Q_1, Q_2)$
must have rank $9$ or smaller. Again, erasing some rows of $M$
which are linear dependent from the others and observing that,
with row elementary operations we can erase some columns of $M$,
we can see that $M$ cannot have rank $8$ or smaller and if $M$
has rank $9$, then the corresponding cubic does not have a conic
among its eigenpoints.
\end{proof}

\begin{lemma}
\label{lemma:bitangentToCiso}
If a cubic $C$ has, among its eigenpoints, an irreducible conic $\Gamma$
which is tangent to $\iso$ in two distinct points $P_1$ and $P_2$, then
the equation of $C$ is of the form $(\lambda \iso + \mu r^2)r$ where
$r$ is the equation of the line $P_1+P_2$ and $\lambda, \mu$ are parameters
in $K$.
\end{lemma}
\begin{proof}
\verb+caso2ptiIso_partI.sage+\\
\verb+caso2ptiIso_partII.sage+\\
The proof is quite similar to the proof of~\Cref{lemma3ptiSuCiso}. Also
here we can assume that $P_1 = (1: \iii: 0)$ and we construct the pencil of
bitangent conics in $P_1$ and $P_2$ to $\iso$. Then we construct three
points $Q_1, Q_2$ and $Q_3$ on a concic of the pencil
and we study the matrix $M = \Phi(P_1, P_2, Q_1, Q_2, Q_3)$. We can see
that the rank of this matrix is $9$ (for each admissible value of the
involved parameters), hence $M$ identifies a cubic~$C$. The computation
of $C$ shows that its equation is of the above form.
\end{proof}


\begin{prop} All the cubics $C$ defined by the polynomial
\begin{equation}
\label{cub_conica_eig}
(\lambda \iso + \mu r^2)r \quad \mbox{$\lambda, \mu \in K$}
\end{equation}
where $r=ax+by+cz$ is a line of the plane,
have the eigenpoints given by the conic
$\lambda \iso+3\mu r^2$ (bitangent to $\iso$ in the points $\iso \cap r$)
and the point $(a:b:c)$ (the pole of $r$
w.r.t.\ $\iso$). Conversely, if
$C$ is a cubic of the plane which has a conic among its eigenpoints,
then there exists a line $r$ in the plane and $\lambda, \mu \in K$
such that the equation of
$C$ is given by~(\ref{cub_conica_eig}).
\end{prop}
\begin{proof}
\verb+caso2ptiIso_partI.sage+ (parte finale del file).\\
If the equation of $C$ is of the form described above, a direct computation
shows that the eigenpoints of $C$ are the pole of $r$ with respect to $\iso$
and the conic $\lambda \iso+3\mu r^2$. Suppose, conversely, that $C$ has
a conic among its eigenpoints. First, we assume that the conic $\Gamma$
is irreducible. If $\Gamma$ is the isotropic conic,
then~\Cref{lemma:fourOnIso} gives that $C$ splits into $\iso$ and a line.
If $\Gamma$ does not coincide with the
isotropic conic, as a consequence
of~\Cref{lemma:fourOnIso} and~\Cref{lemma3ptiSuCiso} must be bitangent,
in two points $P_1$ and $P_2$, to $\iso$. Then we get the thesis
from~\Cref{lemma:bitangentToCiso}. If $\Gamma$ is reducible,
from~\Cref{cubiche_con_2_rette}  we have that $\Gamma$ splits
into two tangent lines to $\iso$.
\end{proof}
\begin{rmk}
if $r$ is tangent to $\iso$ in a point $P$, @@tutto pare funzionare,
ma sistemare
\end{rmk}


\section{The locus of cubic ternary forms with an aligned triple of eigenpoints}
\label{locus_one_alignment}

\begin{definition}
 We set $\mathcal{L} \subseteq \p^9$ to be the closure of the locus of classes of cubic forms $f$ having a reduced zero-dimensional eigenscheme $E(f)$ with an aligned eigentriple.
\end{definition}

\begin{theorem}
The variety~$\mathcal{L}$ is an irreducible hypersurface.
\end{theorem}

\begin{proof}
By the Gallet - Logar construction, for a a general fixed aligned triple of points, the set of cubic forms having such a triple as eigenpoints is a linear system of dimension 3. Since the variety of aligned triples is $5$-dimensional, this proves that $\mathcal{L}$ is a hypersurface.

For the irreducibility, we maybe need the argument on the Geiser map. Is there a simpler one?
\end{proof}

\begin{lemma}
\label{lemma:pencil_one_aligned}
 If $f$ and $g$ are general cubics, then any cubic in the pencil $\lambda f + \mu g$ for $(\lambda: \mu) \in \p^1$ has at most one aligned triple of eigenpoints.
\end{lemma}

This is a corollary of the following result.

\begin{prop}
    The locus of cubics that have at least two aligned triples of eigenpoints has codimension~$2$ in~$\p^9$.
\end{prop}
\begin{proof}
    Conti di Sandro.
\end{proof}

\begin{definition}
 We define $\Delta \subset \mathcal{L}$ to be the closure of the locus of cubics with at least two aligned triples of eigenpoints.
\end{definition}

\begin{prop}
  The variety~$\Delta$ has dimension~$7$ and it is the union of two irreducible components~$\Delta_1$ and~$\Delta_2$.
\end{prop}

In what follows, we shall use the following notation.
Denote by $M_1$, $M_2$ and $M_3$ the three minors of \Cref{eq:def_matrix} relative to a cubic form $f$.
The net of cubics, which base locus is the eigenscheme~$\Eig{f}$, will be denoted by $\Lambda_f = \langle M_1, M_2, M_3 \rangle$.
\begin{lemma}
\label{lemma:scroll}
 If $f$ and $g$ are general cubics, then
 %
 \[
   \mathcal{N} := \bigcup_{(\lambda : \mu) \in \p^1} \Lambda_{\lambda f + \mu g} \subset \p^9
 \]
 %
 is an embedding of a rational projective bundle and has degree~$3$.
\end{lemma}
\begin{proof}
Consider the projective bundle given by the family of planes
%
\[
{\mathcal P} := \{ \Lambda_{\lambda f + \mu g} \, : \, (\lambda: \mu)\in \p^1 \} \subset \p^1 \times \p^9
\]
%
Then $\mathcal{N}$ is the projection of~$\mathcal{P}$ on the second factor.
However, the map ${\mathcal P} \to {\mathcal N}$ contracts no subvariety of any plane of ${\mathcal P}$, so either it is an embedding or it contracts some horizontal curve. In the latter case, all the planes of the family should intersect in at least one point. In particular, the two nets $\Lambda_f$ and $\Lambda_g$ should have non-empty intersection.
If we denote by $M_1$, $M_2$ and $M_3$ the $2 \times 2$ minors relative to~$f$, and by $N_1$, $N_2$ and $N_3$ the ones relative to~$g$, the vectorial dimension of the linear span $\left\langle M_1, M_2, M_3, N_1, N_2, N_3 \right\rangle$ should be strictly less than $6$. This can be avoided, since such a condition corresponds to a proper closed subscheme of $\p^9 \times \p^9$.

It follows that if $f$ and $g$ are general enough, then $\mathcal{N}$ is a $3$-dimensional rational normal scroll in $\p(\left\langle M_1, M_2, M_3, N_1, N_2, N_3 \right\rangle) \cong \p^5$.
Being a variety of minimal degree, its degree is $5+1-3 = 3$.
\end{proof}

\begin{theorem}
The degree of $\mathcal L$ is equal to
\[
  \deg \ \mathcal L =  15.
\]
\end{theorem}

\begin{proof}
We start by observing that a reduced $0$-dimensional eigenscheme contains an aligned triple if and only if the net of cubics $\Lambda_f = \langle M_1, M_2, M_3 \rangle$ contains a cubic which splits in three lines, a so called \emph{triangle}. Moreover, if $f$ is general enough, we have exactly one aligned triple and the other $4$ points are in general position; in this case, the net $\Lambda_f$ contains exactly three triangles, namely the unions of the line passing through the aligned triple and the reducible conics through the $4$ points in general position.

To determine the degree of $\mathcal L$ we consider a general pencil of cubic forms $\lambda f + \mu g$, and we will compute the number of elements with associated net $\Lambda_{\lambda f + \mu g}$ containing a triangle.

To this aim, denote by ${\mathcal T} \subset \p^9$ the variety of triangles; it is a classical result that its dimension is $6$ and its degree is $15$,
see for instance \cite[Section 2.2.2]{3264}. We now consider the variety~${\mathcal N}$ from \Cref{lemma:scroll}.
\comment{
given by the union of the nets of cubics of the pencil
$$
{\mathcal N} = \bigcup_{(\lambda : \mu) \in \p^1} \Lambda_{\lambda f + \mu g} \subset \p^9.
$$
Observe that we can assume that ${\mathcal N}$ is an embedding of a rational projective bundle; indeed, it can be seen as an immersion of the $\p^2$-bundle over $\p^1$ given by the family the planes ${\mathcal P}=
\{\Lambda_{\lambda f + \mu g}\ : \ (\lambda:\mu)\in \p^1\} \subset \p^1 \times \p^9$. The map ${\mathcal P} \to {\mathcal N}$ contracts no subvariety of any plane of ${\mathcal P}$, so it is either an embedding or it contracts some horizontal curve. In the latter case, all the planes of the family should intersect in at least one point. In particular, the two nets $\Lambda_f$ and $\Lambda_g$ should have non-empty intersection.
If we denote by $M_1$, $M_2$ and $M_3$ the $2 \times 2$ minors relative to $f$, and by $N_1$, $N_2$ and $N_3$ the ones relative to $g$, the vectorial dimension of the linear span
$\langle M_1,M_2,M_3,N_1,N_2,N_3
\rangle$ should be strictly less than $6$. This can be avoided, since such a condition corresponds to a proper closed subscheme of $\p^9 \times \p^9$.

It follows that if ${\mathcal N}$ is general enough, it is a $3$-dimensional rational normal scroll in $\p(\langle M_1,M_2,M_3,N_1,N_2,N_3
\rangle) \cong \p^5$, and being a variety of minimal degree, its degree is $\deg {\mathcal N}=5+1-3=3$.
}
Note that, since each net containing a triangle, actually contains exactly $3$ of them, the number of nets of ${\mathcal N}$ containing some triangle is given by
%
\[
\frac {{\mathcal T} \cdot {\mathcal N}}{3} =\frac{{15} \cdot {3}}{3}=15 \,.
\]
%
This hence implies that $\deg {\mathcal L} = 15$.
\end{proof}

\begin{es}
The following pencil of cubic forms admits exactly $15$ cubics with an aligned triple of eigenpoints ..
\end{es}

\section{Possible configurations of the seven eigenpoints}
\label{further_alignments}

\begin{table}
\caption{All possible configurations of $7$ points with alignments that can appear as eigenschemes of a ternary cubic form, provided that the eigenscheme is zero-dimensional and reduced.}
\centering
\begin{tabular}{|llll|}\hline
  num. lines  & collinear vertices & name & conf.\\ \hline
 1& [(1, 2, 3)] & "line" & (1)\\
 2& [(1, 2, 3), (1, 4, 5)] & "X shape"& (2)\\
 3& [(1, 2, 3), (1, 4, 5), (1, 6, 7)] & "star" & (3)\\
  & [(1, 2, 3), (1, 4, 5), (2, 4, 6)] & "triangle" & (4)\\
 4& [(1, 2, 3), (1, 4, 5), (1, 6, 7), (2, 4, 6)] & "triangle + altitude" & (5)\\
  & [(1, 2, 3), (1, 4, 5), (2, 4, 6), (3, 5, 6)] & "two X shapes" & (6)\\
 5& [(1, 2, 3), (1, 4, 5), (1, 6, 7),  & "two stars" & (7)\\
  & \phantom{[}(2, 4, 6), (2, 5, 7)] & &\\
 6& [(1, 2, 3), (1, 4, 5), (1, 6, 7), & "triangle + three altitudes" & (8)\\
  & \phantom{[} (2, 4, 6), (2, 5, 7), (3, 4, 7)] & & \\
 7& [(1, 2, 3),
   (1, 4, 5),
   (1, 6, 7),
   (2, 4, 6), & "Fano matroid" & (9)\\
  & \phantom{[} (2, 5, 7),
   (3, 4, 7),
   (3, 5, 6)] & & \\ \hline
\end{tabular}
\label{table:all_alignments}
\end{table}

\Cref{table:all_alignments} lists all the possible configuration
of 7 distinct points of the plane, according to their collinearities.Here
we want to see which of those nine configurations can be realized by
the seven eigenpoints of a cubic and, in case the configuration is
realizable, which are the cubics with that configuration of eigenpoints.

First of all, it is well-known that configuration (9) cannot be realized
by seven points of the plane over a field of zero
characteristic (see \cite{Whitney1935}), therefore we will not consider
it in our analysis.

\underline{Configuration (1)} can be realized,
\Cref{proposition:three_aligned_ranks} and Section~\ref{locus_one_alignment}
give a description of the cubics with such a configuration of points.

\underline{Configuration (2)} means that the points $P_1, \dots, P_5$ are in a
$V$-configuration, hence the rank of the matrix $\Phi(P_1, \dots, P_5)$
must be $9$ or $8$. If the rank is $9$, then $\delta_1(P_1, P_2, P_4) = 0$
or $\delta_2(P_1, \dots, P_5) = 0$, see~\Cref{theorem:rank_V}.
From \Cref{proposition:terzo_allineamento} we have that the only possibility
is $\delta_1(P_1, P_2, P_4) = 0$. Hence if we fix two points $P_1$ and $P_2$
in the plane and we choose $P_4$ such that $\delta_1(P_1, P_2, P_4) = 0$
(@@ si puo' mettere migliore condizione) and then if we chose $P_3$
on the line $P_1+P_2$ and $P_5$ on the line $P_1+P_4$, we have a cubic
with a configuration of type (2). In case of rank $8$, as seen,
configuration (2) can be
obtained from the conditions~(\ref{rk8_1}) (and hence are sub-cases
of the case $\delta_1(P_1, P_2, P_4)=0$). Configuration (2) can also be
obtained from~(\ref{rk8_2}) in the general case. @@ e' un caso delta1 = 0?
da verificare.

\underline{Configuration (3)} If inside the configuration (3) we
have a $V$-configuration
such that the rank of the matrix $\Phi(P_1, \dots, P_5)$ is $8$
we know that ..... @@ (completare)

Here we assume therefore that we have among the $7$ points of (3),
five points $P_1, \dots, P_5$ in a $V$-configuration such that the
rank of the matrix $\Phi(P_1, \dots, P_5)$ is $9$.

Vedi file: \verb+config3.sage+
\begin{lemma}
\label{no_delta1_delta1} Suppose we have  seven eigenpoints $P_1, \dots, P_7$
of a cubic in configuration (3). Then among the $7$ points there is a
$V$-configuration that satisfies a $\delta_2$ condition.
\end{lemma}
\begin{proof}
%% la dim si trova in config3.sage
The points $P_1, P_2, P_3, P_4, P_5$ are in a $V$-configuration. As said,
we assume the matrix $\Phi(P_1, \dots, P_5)$ has
rank $9$. If
$\delta_2(P_1, \dots, P_5) = 0$, we have the thesis, otherwise,
$\delta_1(P_1, P_2, P_4) = 0$. Then consider the $V$-configuration
$P_1, P_2, P_3, P_6, P_7$. As above, we suppose $\delta_1(P_1, P_2, P_6) = 0$.
The two $\delta_1$-equations give two linear equations in the coordinates
of $P_2$. Assuming the matrix of the system of linear equations have
maximal rank, the solution gives a point $P_2$ that coincide, as
a projective point, to $P_1$, which is impossible. Hence we consider the
case in which the matrix of the linear system do not have maximal rank.
This condition implies that $\scl{P_1}{P_1} = 0$, hence $P_1$ is on the
isotropic conic. W.l.o.g.\ we can assume $P_1 = (1: \iii: 0)$ and again
we can determine $P_2$ such $\delta_1(P_1, P_2, P_4) = 0$ and
$\delta_1(P_1, P_2, P_6)$ are zero. The matrix $\Phi(P_1, P_4, P_5, P_6, P_7)$
must have rank $9$ or smaller, then either
$\delta_2(P_1, P_4, P_5, P_6, P_7)=0$ or $\delta_1(P_1, P4, P_6) = 0$. In
the first case, we have a $\delta_2$ condition among the points, hence
we assume $\delta_1(P_1, P_4, P_6) = 0$. Solving this equation and
considering the corresponding points, we get that either
$\delta_2(P_1, P_2, P_3, P_6, P_7) = 0$ or
$\delta_2(P_1, P_2, P_3, P_4, P_5) = 0$.
\end{proof}

From the above lemma, we have that configuration (3) can be
realized only by a $\delta_2$ condition (a meno dei casi particolari
da inserire bene @@@).

Here we list the ways in which we can obtain configuration (3)
@@@ completare

\underline{Configuration (4)}. It holds:
\begin{prop}
\label{conf4no} If we have five points in configuration that are in
configuration (4) and such that are eigenpoints of a cubic, then the
cubic has $7$ eigenpoints which are in configuration (8). In particular,
configuration (4) is not realizable.
\end{prop}
\begin{proof}
%% il file config4.sage contiene la dimostrazione
We know that configuration cannot be obtained from a rank 8 $V$-configuration,
hence we have to study the case $\delta_1(P_1, P_2, P_4) = 0$,
$\delta_1(P_2, P_1, P_4) = 0$ and $\delta_1(P_4, P_1, P_2) = 0$,
which is not difficult.
\end{proof}

\underline{Configuration (5)} Is realizable.
\verb+file config5.sage+
Descrizione del luogo

%% conti nel file config5.sage.

\underline{Configuration (6)} Is not realizable.
\verb+file config6.sage+

%% conti nel file config6.sage.


\underline{Configuration (7)} Is not realizable.
\verb+file config7.sage+

%% conti nel file config7.sage.


\underline{Configuration (8)} Is realizable.
\verb+file config8.sage+
descrizione del luogo

%% conti nel file config8.sage.



\section{The locus of forms with positive-dimensional eigenscheme}

\begin{prop}\label{p2}
Let $C = V(f) \subset \p^2$ be a cubic curve.
Assume that $\dim \Eig{f} = 1$ and that the $1$-dimensional component is a line~$L$.
Then the residual subscheme $Z := \mathrm{Res}_L \bigl( \Eig{f} \bigr)$ in~$\Eig{f}$ with respect to~$L$ is zero-dimensional of degree~$3$ if $\dim \Lambda = 2$, or of degree~$4$, if $\dim \Lambda = 1$.
\end{prop}
\begin{proof}
Since $G_1$, $G_2$ and $G_3$ have a common linear component~$L$, by writing
\[
G_i=L \ H_i, \quad i=1,2,3
\]
we have that the residual subscheme $Z = \mathrm{Res}_L \bigl( \Eig{f} \bigr)$ is a quasi complete intersection
determined by the polynomials $H_1$, $H_2$ and $H_3$; indeed, we recall that $\mathcal{I}_{Z, \p^2} = \mathcal{I}_{\Eig{f}), \p^2} (-1)$.

Let us assume $\dim \Lambda = 2 = \dim \p \bigl( \left\langle H_1,H_2,H_3 \right\rangle \bigr)$.
So we have an exact sequence
\[
 0 \to \mathcal{G} \to \oo_{\p^2} (-2) \oplus \oo_{\p^2} (-2) \oplus \oo_{\p^2} (-2) \to \mathcal{I}_{Z,\p^2} \to 0 \,,
\]
where $\mathcal{G}$ is a rank two reflexive sheaf by \cite[Proposition 1]{Hartshorne1980}.
But on a smooth surface reflexive implies locally free (see \cite[Example~1.1.6]{Huybrechts2010}),
so $\mathcal{G}$ is a rank two vector bundle.

Next we observe that the two independent
syzygies between the generators $G_i$ give rise to the syzygies:
\[
x_2\ H_1 - x_1\ H_2 + x_0\ H_3=0, \qquad \partial_2 f\ H_1 - \partial_1 f \ H_2 +\partial_0 f \ H_3=0,
\]
which occur in degrees $3$ and $4$. We claim that the two relations are independent, again,
otherwise the $G_i$'s would be identically zero. So we apply \cite[Proposition~12]{Ellia2020}, and we have that $\mathcal {G}$ splits. Observe that as $c_1(\mathcal {I}_{Z,\p^2})=0$, we have $c_1(\mathcal {G})=-6$. Moreover, there is no syzygy in degree $2$, since
otherwise the $H_i$ would be linearly dependent, so the $G_i$ would belong to a pencil, which we excluded.

It follows that the splitting of $\mathcal{G}$ is of the form
%
\[
\mathcal{G} \cong \oo_{\p^2} (-3) \oplus \oo_{\p^2} (-3) \,.
\]
%
We get the free resolution of the ideal sheaf:
%
\[
0\to \oo_{\p^2} (-3)\oplus \oo_{\p^2} (-3) \to 3\oo_{\p^2} (-2)\to \mathcal {I}_{Z,\p^2} \to 0 \,.
\]
%
Moreover
%
\[
c_2 (\mathcal{I}_{Z,\p^2} ) = c_2 \bigl( 3\oo_{\p^2}(-2) \bigr) - c_2 (\mathcal{G}) = 12 - 9 = 3 \,,
\]
%
and $h^0 \bigl( \mathcal{I}_{Z,\p^2}(1) \bigr) = 0$.

Finally, assume $\dim \Lambda =1 =\dim \p \bigl( \left\langle H_1, H_2, H_3 \right\rangle \bigr)$.
Then $Z = \mathrm{Res}_L \bigl( \Eig{f} \bigr)$ is the complete intersection of two conics.
\end{proof}


\begin{es}
Consider
\[
f(x_0,x_1,x_2)=x_0^2 (x_1 - x_2)
\]
(Example~\eqref{item: ternary cubic, no regular eigenpoints} with $t=i$).
We have $L=x_0$,
\[
H_1=x_0^2-2x_1^2+2x_1 x_2, \quad H_2= 2x_2^2-x_0^2-2x_1 x_2, \quad H_3= 2x_1x_2-2x_2^2-x_0x_1 \,.
\]
The two syzygies in degree~$3$ are:
\[
x_2 \, H_1 - x_1 \, H_2 + x_0 \, H_3 = 0, \quad x_0 \, H_2 + 2(x_1+x_2) \, H_3 + x_0 \, H_1 = 0.
\]
Finally, $Z= \{ (0:1:1),(2:1:-1),(-2:1:-1)\}$. Observe that one point is on the singular line $x_0=0$
and that the Jacobian scheme is non reduced, it consists of a line with an embedded point.
\end{es}


\bibliographystyle{amsalpha}
\bibliography{ooms}

\end{document}
