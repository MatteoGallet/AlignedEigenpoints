\documentclass[10pt, a4paper, reqno, captions=tableheading,bibliography=totoc]{scrartcl}

%\usepackage[utf8]{inputenc}
%\usepackage[T1]{fontenc}
\usepackage{amsthm}
\usepackage{amsmath}
\usepackage{amssymb}
%\usepackage{lmodern}
\usepackage[english]{babel}


%\usepackage{booktabs}
%\usepackage{float}
%\usepackage{enumitem}
%\usepackage{tikz,pgf}
%\usepackage[utopia]{mathdesign}
%\usepackage{palatino}
%\usepackage{hyperref}
%\hypersetup{
%    colorlinks,
%    linkcolor={red!50!black},
%    citecolor={blue!50!black},
%    urlcolor={blue!80!black}
%}
%\usepackage[nameinlink]{cleveref}

\theoremstyle{plain}
\newtheorem{lemma}{Lemma}[section]
\newtheorem{prop}[lemma]{Proposition}
\newtheorem{theorem}[lemma]{Theorem}
\newtheorem{cor}[lemma]{Corollary}
\newtheorem{conjecture}[lemma]{Conjecture}
\newtheorem{fact}[lemma]{Fact}
\newtheorem{assumption}[lemma]{Assumption}
\newtheorem*{reduction}{Reduction}
\theoremstyle{definition}
\newtheorem{definition}[lemma]{Definition}
\newtheorem{es}[lemma]{Example}
\newtheorem*{notation}{Notation}
\newtheorem{rmk}[lemma]{Remark}


\newcommand{\N}{\mathbb{N}}
\newcommand{\Z}{\mathbb{Z}}
\newcommand{\Q}{\mathbb{Q}}
\newcommand{\R}{\mathbb{R}}
\newcommand{\C}{\mathbb{C}}
\newcommand{\p}{\mathbb{P}}
\newcommand{\sP}{\mathcal{P}}
\newcommand{\sL}{\mathcal{L}}
\newcommand{\de}{\partial}
\newcommand{\codim}{\mathrm{codim}}

\newcommand{\oo}{\mathcal{O}}
\newcommand{\Bl}{\mathrm{Bl}}

\newcommand{\iso}{\mathcal{Q}}

\newcommand{\SO}{\operatorname{SO}}
\newcommand{\Eig}{\operatorname{Eig}}
\newcommand{\polq}{{\rm Pol}_Q}
\newcommand{\comment}[1]{}

\newcommand{\scl}[2]{\left\langle {#1}, {#2} \right\rangle}

\title{i polinomi $g_1, g_2,g_3$}
\author{}
\date{}

\linespread{1.2}
\setlength{\parindent}{0pt}
\setlength{\parskip}{.25em}

\begin{document}

\maketitle

We assume that we have five points $P_1, \dots, P_5$ in a ``V'' configuration
as usual and we also assume that the $15\times 10$ matrix
$\Phi(P_1, \dots, P_5)$ has rank $9$. Hence there exists a unique cubic
curve $C$ which has $P_1, \dots, P_5$ among its eigenpoints. 
We call $\mathcal{H}$ the $8\times 10$ matrix whose rows are the vectors
$\phi_1(P_i), \phi_2(P_i)$ ($i = 1, \dots, 4$) and we can assume
that the matrix whose rows are the rows of $\mathcal{H}$ and the
row $\phi_1(P_5)$
has rank $9$. If $P=(x, y, z)$ is a generic point of the plane,
consider the following three square matrices of order $10$:
\[ \mathcal{G}_1 = 
\left(
\begin{array}{c}
  \mathcal{H} \\
  \phi_1(P_5)\\
  \phi_1(P)
\end{array}
\right),
\quad
\mathcal{G}_2 = 
\left(
\begin{array}{c}
  \mathcal{H} \\
  \phi_1(P_5)\\
  \phi_2(P)
\end{array}
\right),
\quad
\mathcal{G}_3 = 
\left(
\begin{array}{c}
  \mathcal{H} \\
  \phi_1(P_5)\\
  \phi_3(P)
\end{array}
\right)
\]
and let $g_1$, $g_2$ and $g_3$ be their determinants.
It holds:
\begin{prop}
  A point $P=(x, y, z)$ of the plane is an eigenpoint of $C$ if and only if
  $P$ is a common zero of $g_1, g_2, g_3$. Moreover, among these three
  polynomials we have the syzygy:
  \[
   z g_1-y g_2 +x g_3 = 0
   \]
   \label{threeG}
\end{prop}
\begin{proof}
  Suppose that $P$ is an eigenpoint of $C$. Then the $18\times 10$ matrix
  $\mathcal{M} = \Phi(P_1, \dots, P_5, P)$, which is the matrix
  associated to the linear
  system obtained by imposing that the six points are eigenpoints, 
  has rank $9$, since we know that the linear system has a solution.
  In particular, when $P$ is an eigenpoint of $C$, the matrices
  $\mathcal{G}_1, \mathcal{G}_2$ and $\mathcal{G}_3$, which are
  submatrices of $\mathcal{M}$, have determinant zero, so
  $g_1(P) = g_2(P) = g_3(P) = 0$. \\
  Conversely, suppose that $P$ is a point such that
  $g_1(P) = g_2(P) = g_3(P) = 0$. Then $\phi_1(P)$ , $\phi_2(P)$ and
  $\phi_3(P)$ are linear combinations of $\phi_1(P_5)$ and the rows
  of $\mathcal{H}$, so the matrix
  \[
\left(
\begin{array}{c}
  \mathcal{H} \\
  \phi_1(P_5)\\
  \phi_1(P)\\
  \phi_2(P)\\
  \phi_3(P)
\end{array}
\right)
\]
has rank $9$ and the linear system associated to it has a solution,
which is che cubic $C$, therefore $P$ is an eigenpoint of $C$. \\
We know that $z \phi_1(P)-y\phi_2(P) +x\phi_3(P) = 0$, hence
\[
\det
\left(
\begin{array}{c}
  \mathcal{H} \\
  \phi_1(P_5)\\
  z \phi_1(P)-y\phi_2(P) +x\phi_3(P)
\end{array}
\right) = 0
\]
using the multilinearity of the determinant, we get the syzygy among
$g_1, g_2, g_3$.
\end{proof}


\end{document}
