\documentclass[a4paper, 11pt, reqno]{amsart}

\setlength{\parindent}{.4 in}
\setlength{\textwidth}{6.5 in}
\setlength{\topmargin} {0.2 in}
\setlength{\evensidemargin}{0 in}
\setlength{\oddsidemargin}{0 in}
\usepackage[utf8]{inputenc}
\usepackage[T1]{fontenc}
\usepackage{mathtools}
\usepackage{lmodern}
\usepackage[english]{babel}
\usepackage{booktabs}
\usepackage{float}
\usepackage{enumitem}
\usepackage{tikz,pgf}
\usetikzlibrary{calc}
\usetikzlibrary{through}
\usepackage{tkz-euclide}
\usepackage{palatino}
\usepackage{graphicx}
\usepackage{xcolor}
\usepackage{hyperref}
\hypersetup{
    colorlinks,
    linkcolor={red!50!black},
    citecolor={blue!50!black},
    urlcolor={blue!80!black}
}
\usepackage[nameinlink]{cleveref}

\theoremstyle{plain}
\newtheorem{lemma}{Lemma}[section]
\newtheorem{prop}[lemma]{Proposition}
\newtheorem{theorem}[lemma]{Theorem}
\newtheorem{theoremintro}{Theorem}
\newtheorem{corollary}[lemma]{Corollary}
\newtheorem{conjecture}[lemma]{Conjecture}
\newtheorem{fact}[lemma]{Fact}
\newtheorem{assumption}[lemma]{Assumption}
\newtheorem*{reduction}{Reduction}
\theoremstyle{definition}
\newtheorem{definition}[lemma]{Definition}
\newtheorem{es}[lemma]{Example}
\newtheorem*{notation}{Notation}
\newtheorem{rmk}[lemma]{Remark}

\tikzstyle{point}=[circle, draw, fill=black, inner sep=0pt, minimum size=4pt]
\tikzstyle{line}=[line width=1.5pt, black!70!white]

\newcommand{\N}{\mathbb{N}}
\newcommand{\Z}{\mathbb{Z}}
\newcommand{\Q}{\mathbb{Q}}
\newcommand{\R}{\mathbb{R}}
\newcommand{\C}{\mathbb{C}}
\newcommand{\p}{\mathbb{P}}
\newcommand{\sL}{\mathcal{L}}
\newcommand{\sU}{\mathcal{U}}
\newcommand{\sV}{\mathcal{V}}
\newcommand{\sF}{\mathcal{F}}
\newcommand{\de}{\partial}
\newcommand{\oo}{\mathcal{O}}

\newcommand{\iii}{\textit{i}\,}
\newcommand{\codim}{\mathrm{codim}}
\newcommand{\rk}{\ensuremath{\mathrm{rk}}}

\newcommand{\iso}{\mathcal{Q}_{\mathrm{iso}}}

\newcommand{\SO}{\operatorname{SO}}
\newcommand{\Eig}[1]{\mathcal{E}\!\left( {#1} \right)}
\newcommand{\Sym}{\operatorname{Sym}}

\newcommand{\scl}[2]{\left\langle {#1}, {#2} \right\rangle}

\newcommand\scalemath[2]{\scalebox{#1}{\mbox{\ensuremath{\displaystyle #2}}}}

\newcommand{\blue}[1]{{\color{blue}[#1]}}


\title{Equations for eigenconfigurations of ternary cubics}
\author[Valentina Beorchia]{Valentina Beorchia$^{\circ}$}
\address[\textsc{Valentina Beorchia}]{University of Trieste,
Department of Mathematics, Informatics and Geosciences,
Via Valerio 12/1, 34127 Trieste, Italy}
\email{beorchia@units.it}
\thanks{$^{\circ}$The researcher is a member of ``Gruppo Nazionale per le Strutture Algebriche, Geometriche e le loro Applicazioni'', INdAM. She is partially supported by MUR funds: PRIN project GEOMETRY OF ALGEBRAIC STRUCTURES: MODULI, INVARIANTS, DEFORMATIONS, PI Ugo Bruzzo, Project code: 2022BTA242.}
\author[Matteo Gallet]{Matteo Gallet$^{\diamond}$}
\address[\textsc{Matteo Gallet}]{University of Trieste,
Department of Mathematics, Informatics and Geosciences,
Via Valerio 12/1, 34127 Trieste, Italy}
\email{matteo.gallet@units.it}
\thanks{$^{\diamond}$The researcher is a member of ``Gruppo Nazionale per le Strutture Algebriche, Geometriche e le loro Applicazioni'', INdAM.}
\author[Alessandro Logar]{Alessandro Logar}
\address[\textsc{Alessandro Logar}]{University of Trieste,
Department of Mathematics, Informatics and Geosciences,
Via Valerio 12/1, 34127 Trieste, Italy}
\email{logar@units.it}
\date{}

\linespread{1.15}
\setlength{\parindent}{0pt}
\setlength{\parskip}{.25em}

\begin{document}

\maketitle


\section{The locus of $7$-uples of eigenpoints of cubics}

\section{Cubics with an eigenconic}




We recall the following result, which relies on the exactness of the Koszul complex associated with the regular sequence~$x$, $y$, and~$z$, specifically we use the following result (see \cite[Theorem~7.3.13]{Dolgachev} or \cite[Lemma~3.9]{BGV}), which will be used in what follows.

\begin{lemma}
\label{lem:Koszul}
Let $h_1,h_2,h_3\in\C[x,y,z]_d$ with $d \ge 1$. Then
%
\begin{equation}
\label{eq:linear_relazion}
  xh_1-yh_2+zh_3 = 0
\end{equation}
%
if and only if there exist $m_1,m_2,m_3\in\C[x,y,z]_{d-1}$ such that
%
\begin{equation}
\label{eq:minors_lemma}
  h_1 = ym_3-zm_2 \,, \qquad
  h_2 = xm_3-zm_1 \,, \qquad
  h_3 = xm_2-ym_1 \,.
\end{equation}
%
\end{lemma}

\begin{proof}
If $h_1,h_2,h_3$ satisfy \Cref{eq:minors_lemma}, then it is immediate to check that they satisfy \Cref{eq:linear_relazion} as well. 
Conversely, assume that \Cref{eq:linear_relazion} holds and let $R = \C[x,y,z]$.
The Koszul complex in the ring~$R$ is an exact sequence of $R$-modules
%
\[
  0 \to R\xrightarrow{\alpha} R^{\oplus 3} \xrightarrow{\beta} R^{\oplus 3} \xrightarrow{\gamma} R \to R/(x,y,z) \to 0 \,,
\]
%
where the maps are $\alpha(p) = (p x, p y, p z)$, $\gamma (w_1,w_2,w_3) = w_1 x + w_2 y + w_3 z$ and $\beta$ is defined by the matrix
%
\[
  \left(
  \begin{array}{ccc}
    0 & -z & y\\
    z & 0 & -x\\
    -y & x & 0 \\
  \end{array}
  \right) \,.
\]
%
The syzygy $xh_1-yh_2+zh_3=0$ implies that $(h_1, -h_2, h_3)$ is in the kernel of~$\gamma$,
and since the Koszul complex is exact, the triple $(h_1,-h_2, h_3)$ lies in the image of~$\beta$.
It follows that there exist $m_1,m_2,m_3 \in \C[x,y,z]_{d-1}$ such that \Cref{eq:minors_lemma} holds.
\end{proof}

We shall need also the following result, see \cite[Theorem 3.11]{BGV}:
\begin{theorem}\label{thm: defining equations}Let $d\ge 2$. Three homogeneous forms $f_1,f_2,f_3\in \C[x,y,z]_d$ are the determinantal equations defining the eigenscheme of a form $f\in\C[x,y,z]_d$ if and only if 
	\begin{align}
	    &xf_1 - y f_2 + z f_3 = 0\label{eq: second}\mbox{ and }\\
	&\partial_x f_1 - \partial_y f_2 + \partial_z f_3 = 0 \label{eq: first}.
	\end{align}
\end{theorem}

\begin{prop}
    Let $C = V(f) \subset \p^2$ be a cubic curve.
Assume that $\dim \Eig{f} = 1$ and that the $1$-dimensional component of~$\Eig{f}$ is a conic~$g$. Then $g$ has the form ...
\end{prop}

\begin{proof}
    

Let $C = V(f) \subset \p^2$ be a cubic curve.
Assume that $\dim \Eig{f} = 1$ and that the $1$-dimensional component of~$\Eig{f}$ is a conic~$g$.

Let 
\begin{equation}\label{eq: 3 minors}
g_1=y \,\partial_z f - z \,\partial_y f, \qquad g_2=x\,\partial_z f - z\, \partial_x f, \qquad g_3= x \,\partial_y f - y \,\partial_x f
\end{equation}
be the order~$2$ minors determining the eigenscheme of~$f$, and let $g$ be the greatest common factor.

By writing
%
\begin{equation}\label{eq: common component}
    g_i = g \, h_i, \quad i=1,2,3
\end{equation}
%
we have that the residual scheme is defined by the ideal
$(h_1,h_2,h_3)$. Moreover, the
linear
syzygy between the generators~$g_i$ gives rise to the syzygy:
%
\[
  z\, h_1 - y\, h_2 + x\, h_3 = 0 \,.
\]
%
By \Cref{lem:Koszul}, the triple $(h_1,h_2,h_3)$ is the triple of order two minors of a matrix
%
\[
  \begin{pmatrix}
    x & y & z \\
    m_1 & m_2 & m_3
  \end{pmatrix} \,.
\]
%
for suitable constants $m_i \in \C$.
By the assumption that $g$ is the greatest common factor of the three minors $g_1$, $g_2$ and~$g_3$, the triple $(h_1,h_2,h_3)$ corresponds to three linear forms belonging to a pencil, and 
they have the form
\[
 h_1 = ym_3-zm_2 \,, \qquad
  h_2 = xm_3-zm_1 \,, \qquad
  h_3 = xm_2-ym_1 \,.
\]
In particular, we see that 
\begin{equation}\label{eq: partials hi}
m_1 h_1 -m_2 h_2 +m_3h_3=0, \qquad 
\de_x h_1 - \de_y h_2 +\de_z h_3=0.
\end{equation}
Now we observe that the minors $g_i$ satisfy also the identity
\begin{equation}\label{eq: partials}
    \partial_x g_1 - \partial_y g_2 +\partial_z g_3=0.
\end{equation}
By using \eqref{eq: common component}, we get
\begin{equation}\label{eq: differential relation}
  h_1 \, \de_x g - h_2\, \de_y g +h_1 \, \de_z g+ ( \de_xh_1 -\de_y h_2 + \de_z h_3) \, g =0, 
\end{equation}
and by \eqref{eq: partials hi}, this gives a syzygy between the three partials
with linear coefficients:
\[
 h_1 \, \de_x g - h_2\, \de_y g +h_3 \, \de_z g=0.
\]
If $V(g)$ is an irreducible, and hence smooth conic, the partials 
$(\de_x g, \de_y g,\de_z g)$ form a regular sequence, so their syzygy module is generated by the three Koszul relations; in particular
\[
(h_1,-h_2,h_3)=a(-\de_y g, \de_x g,0) + b (-\de_z g, 0, \de_x g) + c(0,-\de_z g, \de_y g)
=
\]
\[
=(-a\de_y g -b\de_z g, a\de_x g -c\de_z g, b\de_x g+c\de_y g).
\]
Now, by \eqref{eq: partials hi}, we have also
\[
m_1 (-a\de_y g -b\de_z g)-m_2(a\de_x g -c\de_z g)+m_3(b\de_x g+c\de_y g)=0,
\]
that is
\[
(-am_2+bm_3)\de_x g +(-am_1+cm_3)\de_y g +(-bm_1+cm_2)\de_zg=0.
\]
Since we are assuming $V(g)$ smooth, its partials are linearly independent; indeed, it is classically known that the partials of a terniary form are dependent if and only if 
the zero locus is a set of concurrent lines.
It follows that the three coefficients are all zero, and being the order two minors of the matrix
\[
\begin{pmatrix}
    a&b&c\\
    m_3 & m_2 &m_1\\
\end{pmatrix},
\]
the two rows are proportional, so we have
\[
(h_1,-h_2,h_3)=(-m_3\de_y g -m_2\de_z g, m_3\de_x g -m_1\de_z g, m_2\de_x g+m_1\de_y g).
\]
\end{proof}

\bibliographystyle{alphaurl}
\bibliography{biblio}

\end{document}

